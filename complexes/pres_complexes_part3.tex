
%%%%%%%%%%%%%%%%%% PREAMBULE %%%%%%%%%%%%%%%%%%

\documentclass[aspectratio=169,utf8]{beamer}
%\documentclass[aspectratio=169,handout]{beamer}

\usetheme{Boadilla}
%\usecolortheme{seahorse}
\usecolortheme[RGB={245,66,24}]{structure}
\useoutertheme{infolines}

% packages
\usepackage{amsfonts,amsmath,amssymb,amsthm}
\usepackage[utf8]{inputenc}
\usepackage[T1]{fontenc}
\usepackage{lmodern}

\usepackage[francais]{babel}
\usepackage{fancybox}
\usepackage{graphicx}

\usepackage{float}
\usepackage{xfrac}

%\usepackage[usenames, x11names]{xcolor}
\usepackage{tikz}
\usepackage{pgfplots}
\usepackage{datetime}



%-----  Package unités -----
\usepackage{siunitx}
\sisetup{locale = FR,detect-all,per-mode = symbol}

%\usepackage{mathptmx}
%\usepackage{fouriernc}
%\usepackage{newcent}
%\usepackage[mathcal,mathbf]{euler}

%\usepackage{palatino}
%\usepackage{newcent}
% \usepackage[mathcal,mathbf]{euler}



% \usepackage{hyperref}
% \hypersetup{colorlinks=true, linkcolor=blue, urlcolor=blue,
% pdftitle={Exo7 - Exercices de mathématiques}, pdfauthor={Exo7}}


%section
% \usepackage{sectsty}
% \allsectionsfont{\bf}
%\sectionfont{\color{Tomato3}\upshape\selectfont}
%\subsectionfont{\color{Tomato4}\upshape\selectfont}

%----- Ensembles : entiers, reels, complexes -----
\newcommand{\Nn}{\mathbb{N}} \newcommand{\N}{\mathbb{N}}
\newcommand{\Zz}{\mathbb{Z}} \newcommand{\Z}{\mathbb{Z}}
\newcommand{\Qq}{\mathbb{Q}} \newcommand{\Q}{\mathbb{Q}}
\newcommand{\Rr}{\mathbb{R}} \newcommand{\R}{\mathbb{R}}
\newcommand{\Cc}{\mathbb{C}} 
\newcommand{\Kk}{\mathbb{K}} \newcommand{\K}{\mathbb{K}}

%----- Modifications de symboles -----
\renewcommand{\epsilon}{\varepsilon}
\renewcommand{\Re}{\mathop{\text{Re}}\nolimits}
\renewcommand{\Im}{\mathop{\text{Im}}\nolimits}
%\newcommand{\llbracket}{\left[\kern-0.15em\left[}
%\newcommand{\rrbracket}{\right]\kern-0.15em\right]}

\renewcommand{\ge}{\geqslant}
\renewcommand{\geq}{\geqslant}
\renewcommand{\le}{\leqslant}
\renewcommand{\leq}{\leqslant}
\renewcommand{\epsilon}{\varepsilon}

%----- Fonctions usuelles -----
\newcommand{\ch}{\mathop{\text{ch}}\nolimits}
\newcommand{\sh}{\mathop{\text{sh}}\nolimits}
\renewcommand{\tanh}{\mathop{\text{th}}\nolimits}
\newcommand{\cotan}{\mathop{\text{cotan}}\nolimits}
\newcommand{\Arcsin}{\mathop{\text{arcsin}}\nolimits}
\newcommand{\Arccos}{\mathop{\text{arccos}}\nolimits}
\newcommand{\Arctan}{\mathop{\text{arctan}}\nolimits}
\newcommand{\Argsh}{\mathop{\text{argsh}}\nolimits}
\newcommand{\Argch}{\mathop{\text{argch}}\nolimits}
\newcommand{\Argth}{\mathop{\text{argth}}\nolimits}
\newcommand{\pgcd}{\mathop{\text{pgcd}}\nolimits} 


%----- Commandes divers ------
\newcommand{\ii}{\mathrm{i}}
\newcommand{\dd}{\text{d}}
\newcommand{\id}{\mathop{\text{id}}\nolimits}
\newcommand{\Ker}{\mathop{\text{Ker}}\nolimits}
\newcommand{\Card}{\mathop{\text{Card}}\nolimits}
\newcommand{\Vect}{\mathop{\text{Vect}}\nolimits}
\newcommand{\Mat}{\mathop{\text{Mat}}\nolimits}
\newcommand{\rg}{\mathop{\text{rg}}\nolimits}
\newcommand{\tr}{\mathop{\text{tr}}\nolimits}


%----- Structure des exercices ------

\newtheoremstyle{styleexo}% name
{2ex}% Space above
{3ex}% Space below
{}% Body font
{}% Indent amount 1
{\bfseries} % Theorem head font
{}% Punctuation after theorem head
{\newline}% Space after theorem head 2
{}% Theorem head spec (can be left empty, meaning ‘normal’)

%\theoremstyle{styleexo}
\newtheorem{exo}{Exercice}
\newtheorem{ind}{Indications}
\newtheorem{cor}{Correction}


\newcommand{\exercice}[1]{} \newcommand{\finexercice}{}
%\newcommand{\exercice}[1]{{\tiny\texttt{#1}}\vspace{-2ex}} % pour afficher le numero absolu, l'auteur...
\newcommand{\enonce}{\begin{exo}} \newcommand{\finenonce}{\end{exo}}
\newcommand{\indication}{\begin{ind}} \newcommand{\finindication}{\end{ind}}
\newcommand{\correction}{\begin{cor}} \newcommand{\fincorrection}{\end{cor}}

\newcommand{\noindication}{\stepcounter{ind}}
\newcommand{\nocorrection}{\stepcounter{cor}}

\newcommand{\fiche}[1]{} \newcommand{\finfiche}{}
\newcommand{\titre}[1]{\centerline{\large \bf #1}}
\newcommand{\addcommand}[1]{}
\newcommand{\video}[1]{}

% Marge
\newcommand{\mymargin}[1]{\marginpar{{\small #1}}}

\def\noqed{\renewcommand{\qedsymbol}{}}


%----- Presentation ------
\setlength{\parindent}{0cm}

%\newcommand{\ExoSept}{\href{http://exo7.emath.fr}{\textbf{\textsf{Exo7}}}}

\definecolor{myred}{rgb}{0.93,0.26,0}
\definecolor{myorange}{rgb}{0.97,0.58,0}
\definecolor{myyellow}{rgb}{1,0.86,0}

\newcommand{\LogoExoSept}[1]{  % input : echelle
{\usefont{U}{cmss}{bx}{n}
\begin{tikzpicture}[scale=0.1*#1,transform shape]
  \fill[color=myorange] (0,0)--(4,0)--(4,-4)--(0,-4)--cycle;
  \fill[color=myred] (0,0)--(0,3)--(-3,3)--(-3,0)--cycle;
  \fill[color=myyellow] (4,0)--(7,4)--(3,7)--(0,3)--cycle;
  \node[scale=5] at (3.5,3.5) {Exo7};
\end{tikzpicture}}
}


\newcommand{\debutmontitre}{
  \author{} \date{} 
  \thispagestyle{empty}
  \hspace*{-10ex}
  \begin{minipage}{\textwidth}
    \titlepage  
  \vspace*{-2.5cm}
  \begin{center}
    \LogoExoSept{2.5}
  \end{center}
  \end{minipage}

  \vspace*{-0cm}
  
  % Astuce pour que le background ne soit pas discrétisé lors de la conversion pdf -> png
\begin{tikzpicture}
        \fill[opacity=0,green!60!black] (0,0)--++(0,0)--++(0,0)--++(0,0)--cycle; 
\end{tikzpicture}

% toc S'affiche trop tot :
% \tableofcontents[hideallsubsections, pausesections]
}

\newcommand{\finmontitre}{
  \end{frame}
  \setcounter{framenumber}{0}
} % ne marche pas pour une raison obscure

%----- Commandes supplementaires ------

% \usepackage[landscape]{geometry}
% \geometry{top=1cm, bottom=3cm, left=2cm, right=10cm, marginparsep=1cm
% }
% \usepackage[a4paper]{geometry}
% \geometry{top=2cm, bottom=2cm, left=2cm, right=2cm, marginparsep=1cm
% }

%\usepackage{standalone}


% New command Arnaud -- november 2011
\setbeamersize{text margin left=24ex}
% si vous modifier cette valeur il faut aussi
% modifier le decalage du titre pour compenser
% (ex : ici =+10ex, titre =-5ex

\theoremstyle{definition}
%\newtheorem{proposition}{Proposition}
%\newtheorem{exemple}{Exemple}
%\newtheorem{theoreme}{Théorème}
%\newtheorem{lemme}{Lemme}
%\newtheorem{corollaire}{Corollaire}
%\newtheorem*{remarque*}{Remarque}
%\newtheorem*{miniexercice}{Mini-exercices}
%\newtheorem{definition}{Définition}

% Commande tikz
\usetikzlibrary{calc}
\usetikzlibrary{patterns,arrows}
\usetikzlibrary{matrix}
\usetikzlibrary{fadings} 

%definition d'un terme
\newcommand{\defi}[1]{{\color{myorange}\textbf{\emph{#1}}}}
\newcommand{\evidence}[1]{{\color{blue}\textbf{\emph{#1}}}}
\newcommand{\assertion}[1]{\emph{\og#1\fg}}  % pour chapitre logique
%\renewcommand{\contentsname}{Sommaire}
\renewcommand{\contentsname}{}
\setcounter{tocdepth}{2}



%------ Figures ------

\def\myscale{1} % par défaut 
\newcommand{\myfigure}[2]{  % entrée : echelle, fichier figure
\def\myscale{#1}
\begin{center}
\footnotesize
{#2}
\end{center}}


%------ Encadrement ------

\usepackage{fancybox}


\newcommand{\mybox}[1]{
\setlength{\fboxsep}{7pt}
\begin{center}
\shadowbox{#1}
\end{center}}

\newcommand{\myboxinline}[1]{
\setlength{\fboxsep}{5pt}
\raisebox{-10pt}{
\shadowbox{#1}
}
}

%--------------- Commande beamer---------------
\newcommand{\beameronly}[1]{#1} % permet de mettre des pause dans beamer pas dans poly


\setbeamertemplate{navigation symbols}{}
\setbeamertemplate{footline}  % tiré du fichier beamerouterinfolines.sty
{
  \leavevmode%
  \hbox{%
  \begin{beamercolorbox}[wd=.333333\paperwidth,ht=2.25ex,dp=1ex,center]{author in head/foot}%
    % \usebeamerfont{author in head/foot}\insertshortauthor%~~(\insertshortinstitute)
    \usebeamerfont{section in head/foot}{\bf\insertshorttitle}
  \end{beamercolorbox}%
  \begin{beamercolorbox}[wd=.333333\paperwidth,ht=2.25ex,dp=1ex,center]{title in head/foot}%
    \usebeamerfont{section in head/foot}{\bf\insertsectionhead}
  \end{beamercolorbox}%
  \begin{beamercolorbox}[wd=.333333\paperwidth,ht=2.25ex,dp=1ex,right]{date in head/foot}%
    % \usebeamerfont{date in head/foot}\insertshortdate{}\hspace*{2em}
    \insertframenumber{} / \inserttotalframenumber\hspace*{2ex} 
  \end{beamercolorbox}}%
  \vskip0pt%
}


\definecolor{mygrey}{rgb}{0.5,0.5,0.5}
\setlength{\parindent}{0cm}
%\DeclareTextFontCommand{\helvetica}{\fontfamily{phv}\selectfont}

% background beamer
\definecolor{couleurhaut}{rgb}{0.85,0.9,1}  % creme
\definecolor{couleurmilieu}{rgb}{1,1,1}  % vert pale
\definecolor{couleurbas}{rgb}{0.85,0.9,1}  % blanc
\setbeamertemplate{background canvas}[vertical shading]%
[top=couleurhaut,middle=couleurmilieu,midpoint=0.4,bottom=couleurbas] 
%[top=fondtitre!05,bottom=fondtitre!60]



\makeatletter
\setbeamertemplate{theorem begin}
{%
  \begin{\inserttheoremblockenv}
  {%
    \inserttheoremheadfont
    \inserttheoremname
    \inserttheoremnumber
    \ifx\inserttheoremaddition\@empty\else\ (\inserttheoremaddition)\fi%
    \inserttheorempunctuation
  }%
}
\setbeamertemplate{theorem end}{\end{\inserttheoremblockenv}}

\newenvironment{theoreme}[1][]{%
   \setbeamercolor{block title}{fg=structure,bg=structure!40}
   \setbeamercolor{block body}{fg=black,bg=structure!10}
   \begin{block}{{\bf Th\'eor\`eme }#1}
}{%
   \end{block}%
}


\newenvironment{proposition}[1][]{%
   \setbeamercolor{block title}{fg=structure,bg=structure!40}
   \setbeamercolor{block body}{fg=black,bg=structure!10}
   \begin{block}{{\bf Proposition }#1}
}{%
   \end{block}%
}

\newenvironment{corollaire}[1][]{%
   \setbeamercolor{block title}{fg=structure,bg=structure!40}
   \setbeamercolor{block body}{fg=black,bg=structure!10}
   \begin{block}{{\bf Corollaire }#1}
}{%
   \end{block}%
}

\newenvironment{mydefinition}[1][]{%
   \setbeamercolor{block title}{fg=structure,bg=structure!40}
   \setbeamercolor{block body}{fg=black,bg=structure!10}
   \begin{block}{{\bf Définition} #1}
}{%
   \end{block}%
}

\newenvironment{lemme}[0]{%
   \setbeamercolor{block title}{fg=structure,bg=structure!40}
   \setbeamercolor{block body}{fg=black,bg=structure!10}
   \begin{block}{\bf Lemme}
}{%
   \end{block}%
}

\newenvironment{remarque}[1][]{%
   \setbeamercolor{block title}{fg=black,bg=structure!20}
   \setbeamercolor{block body}{fg=black,bg=structure!5}
   \begin{block}{Remarque #1}
}{%
   \end{block}%
}


\newenvironment{exemple}[1][]{%
   \setbeamercolor{block title}{fg=black,bg=structure!20}
   \setbeamercolor{block body}{fg=black,bg=structure!5}
   \begin{block}{{\bf Exemple }#1}
}{%
   \end{block}%
}


\newenvironment{miniexercice}[0]{%
   \setbeamercolor{block title}{fg=structure,bg=structure!20}
   \setbeamercolor{block body}{fg=black,bg=structure!5}
   \begin{block}{Mini-exercices}
}{%
   \end{block}%
}


\newenvironment{tp}[0]{%
   \setbeamercolor{block title}{fg=structure,bg=structure!40}
   \setbeamercolor{block body}{fg=black,bg=structure!10}
   \begin{block}{\bf Travaux pratiques}
}{%
   \end{block}%
}
\newenvironment{exercicecours}[1][]{%
   \setbeamercolor{block title}{fg=structure,bg=structure!40}
   \setbeamercolor{block body}{fg=black,bg=structure!10}
   \begin{block}{{\bf Exercice }#1}
}{%
   \end{block}%
}
\newenvironment{algo}[1][]{%
   \setbeamercolor{block title}{fg=structure,bg=structure!40}
   \setbeamercolor{block body}{fg=black,bg=structure!10}
   \begin{block}{{\bf Algorithme}\hfill{\color{gray}\texttt{#1}}}
}{%
   \end{block}%
}


\setbeamertemplate{proof begin}{
   \setbeamercolor{block title}{fg=black,bg=structure!20}
   \setbeamercolor{block body}{fg=black,bg=structure!5}
   \begin{block}{{\footnotesize Démonstration}}
   \footnotesize
   \smallskip}
\setbeamertemplate{proof end}{%
   \end{block}}
\setbeamertemplate{qed symbol}{\openbox}


\makeatother
\usecolortheme[RGB={102,102,0}]{structure}

%%%%%%%%%%%%%%%%%%%%%%%%%%%%%%%%%%%%%%%%%%%%%%%%%%%%%%%%%%%%%
%%%%%%%%%%%%%%%%%%%%%%%%%%%%%%%%%%%%%%%%%%%%%%%%%%%%%%%%%%%%%

\begin{document}




\title{{\bf Nombres complexes}}
\subtitle{Argument et trigonométrie}


\begin{frame}
  
  \debutmontitre

  \pause

{\footnotesize
\hfill
\setbeamercovered{transparent=50}
\begin{minipage}{0.6\textwidth}
  \begin{itemize}
    \item<3-> Argument
    \item<4-> Notation exponentielle
    \item<5-> Racine $n$-ième
    \item<6-> Trigonométrie
  \end{itemize}
\end{minipage}
}
\vspace*{1cm}
\end{frame}

\setcounter{framenumber}{0}






%%%%%%%%%%%%%%%%%%%%%%%%%%%%%%%%%%%%%%%%%%%%%%%%%%%%%%%%%%%%%%%%
\section{Argument d'un nombre complexe}

\begin{frame}

Soit $z \in \Cc^*$. Un \defi{argument} de $z$ est un nombre $\theta\in\mathbb{R}$ tel que 
\mybox{$z=|z|(\cos \theta + \ii \sin \theta)$}

\pause

\vspace*{-2ex}
{
\hfill
\begin{minipage}{0.4\linewidth}
On le note $\theta = \arg(z)$
\end{minipage}
\hfill
\pause
\begin{minipage}{0.5\linewidth}
\myfigure{0.9}{
\tikzinput{fig_complexes11}
}
\end{minipage}
\hfill\hfill
}
\pause
\begin{remarque}

\centerline{$\theta' = \theta \pmod {2\pi} \quad  \Longleftrightarrow \quad  \exists k \in
     \Zz \ \  \theta' = \theta + 2 k \pi$ }

\pause
\medskip

L'argument est déterminé modulo $2\pi$





\pause
\medskip

On peut obtenir l'unicité en imposant par exemple $\theta\in]-\pi,+\pi]$
\end{remarque}

\end{frame}



\begin{frame}

  \begin{itemize}
    \item $\arg \left( z \times z' \right) = \arg (z) + \arg \left(
    z' \right) \pmod {2\pi}$
   \end{itemize}

\pause

\begin{proof}
  \begin{eqnarray*}
    z \times z' & = & \left| z \right|  \left( \cos \theta + \ii  \sin \theta \right) 
    \left| z' \right|  \left( \cos \theta' + \ii  \sin \theta' \right)\\
\pause
    & = & \left| zz' \right|  \left( \cos \theta \cos \theta' - \sin \theta
    \sin \theta' + \ii  \left( \cos \theta \sin \theta' + \sin \theta \cos
    \theta' \right) \right)\\
\pause
    & = & \left| zz' \right|  \left( \cos \left( \theta + \theta' \right) + \ii 
    \sin \left( \theta + \theta' \right) \right)
  \end{eqnarray*}
\end{proof}

     \begin{itemize}
\pause
    \item $\arg \left( z^n \right) = n \arg (z) \pmod {2\pi}$

\pause    
    \item $\arg \left( 1 / z \right) = - \arg (z) \pmod {2\pi}$

\pause    
    \item $\arg (\overline{z}) = - \arg z \pmod{2 \pi}$
  \end{itemize}



\end{frame}


%%%%%%%%%%%%%%%%%%%%%%%%%%%%%%%%%%%%%%%%%%%%%%%%%%%%%%%%%%%%%%%%
\section{Notation exponentielle}

\begin{frame}

\defi{Formule de Moivre} 
\mybox{$
  \left( \cos \theta + \ii \sin \theta \right)^n = \cos \left( n \theta \right)
  + \ii  \sin \left( n \theta \right)$}

\pause
\uncover<2->{
\begin{proof}
Par récurrence sur $n$
\begin{itemize}
\item La formule est vraie pour $n=0$
\pause
\item Hérédité
$$
  \begin{array}{rcl}
    \left( \cos \theta + \ii  \sin \theta \right)^n & = & ( \cos \theta + \ii \sin \theta)^{n-1}
 \times \left( \cos \theta + \ii  \sin \theta \right)
    \\
\pause
  & = & \left( \cos \left(
    \left( n - 1 \right) \theta \right) + \ii  \sin \left( \left( n - 1 \right)
    \theta \right) \right) \times \left( \cos \theta + \ii  \sin \theta \right)
    \\
\pause
    & = & \left( \cos \left( \left( n - 1 \right) \theta \right) \cos \theta
    - \sin \left( \left( n - 1 \right) \theta \right) \sin \theta \right) 
    \\
    && + \ii 
    \left( \cos \left( \left( n - 1 \right) \theta \right) \sin \theta + \sin
    \left( \left( n - 1 \right) \theta \right) \cos \theta \right)
    \\
\pause
    & = & \cos n \theta + \ii  \sin n \theta
  \end{array}$$
\pause
  \vspace*{-3ex}
  \item Conclusion : la formule est vraie pour tout $n$
\end{itemize}
\end{proof}
}


\end{frame}


\begin{frame}
\medskip

\defi{Notation exponentielle :} 
\myboxinline{$e^{\ii  \theta} = \cos \theta + \ii  \sin \theta$}\\
\bigskip
\pause
Tout nombre complexe $z$ s'écrit 
\myboxinline{$z = \rho e^{\ii  \theta}$} \\
avec $\rho = \left| z \right|$ le module et $\theta = \arg (z)$ un argument

\pause

\bigskip
{\small
Si\quad  $z = \rho e^{\ii  \theta}$ \quad et \quad $z' = \rho' e^{\ii  \theta'}$ \quad  alors
\begin{itemize}
\item $zz' = \rho \rho' e^{\ii  \theta} e^{\ii  \theta'} = \rho \rho' e^{\ii  (\theta + \theta')}$
\pause
\item Formule de Moivre : $\left(e^{\ii\theta}\right)^n = e^{\ii n \theta}$
\pause
\item $z^n = \left( \rho e^{\ii  \theta} \right)^n = \rho^n  \left( e^{\ii  \theta}
     \right)^n = \rho^n e^{\ii n \theta}$

\pause
\item $\frac 1  z = \frac{1}{\rho e^{\ii  \theta}} 
= \frac{1}{\rho} e^{- \ii \theta}$
\pause
\item $\overline{z} = \rho e^{-\ii \theta}$
\end{itemize}
}

\pause
\mybox{
$\rho e^{\ii \theta} = \rho' e^{\ii \theta'}$ (avec $\rho, \rho' > 0$) \ 
$\Longleftrightarrow\ \left\{ \begin{array}{l}\rho = \rho'\\\theta = \theta' \pmod{2\pi}\end{array}\right.$
}

\end{frame}

%%%%%%%%%%%%%%%%%%%%%%%%%%%%%%%%%%%%%%%%%%%%%%%%%%%%%%%%%%%%%%%%
\section{Racines $n$-i\`eme}

\begin{frame}

Pour $z \in \Cc$ et $n \in \Nn$, une \defi{racine $n$-ième} est
un $\omega \in \Cc$ tel que
\mybox{$\omega^n = z$}

\pause

Les racines $n$-ièmes $\omega_0, \omega_1, \ldots, \omega_{n - 1}$ 
de $z=\rho e^{\ii  \theta}$ sont les $n$ complexes :
\mybox{$\omega_k = \rho^{\frac 1 n} \, e^{\frac{\ii\theta \, + \, 2\ii k \pi}{n}}  \qquad k = 0,1,
     \ldots, n - 1$}
     \vspace{-0.5em}

\pause


\begin{proof}
Cherchons $\omega=re^{\ii t}$ tel que $z=\omega^n$ \hspace{3ex}
\pause
$\Longleftrightarrow\quad \rho e^{\ii \theta}  = \left(re^{\ii t}\right)^n 
= r^n e^{\ii nt}$\\
\pause
$\quad\Longleftrightarrow\ 
\left\{\begin{array}{l}
\only<5>{\left|\rho e^{\ii \theta} \right| =  \left|r^n e^{\ii nt}\right|}
\only<6->{\rho = r^n}\\
\only<5,6>{\arg(\rho e^{\ii \theta}) = \arg(r^n e^{\ii nt}) \pmod {2\pi} }
\only<7->{nt = \theta \pmod {2\pi}} 
\uncover<8->{ \ \textrm{ c-à-d } \  nt = \theta + 2k\pi \textrm{ (pour $k\in\Zz$)} }
\end{array}\right.$
\pause
\pause
\pause
 \pause

$\quad\Longleftrightarrow\ 
\left\{\begin{array}{l}
r = \rho^{1/n}\\
\pause
t = \frac{\theta}{n} + \frac{2k\pi}{n}
\end{array}\right.$

\pause

Les solutions sont les $\omega_k = \rho^{1/n} e^{\frac{\ii\theta + 2\ii k\pi}{n}}$, $k\in\Zz$ 

\pause

Mais $\omega_n=\omega_0$, $\omega_{n+1}=\omega_1$, \ldots\\
Ainsi les $n$ solutions sont $\omega_0,\omega_1,\ldots,\omega_{n-1}$

\end{proof}

\end{frame}


\begin{frame}

$z=1=e^{\ii 0}$ et $n=3$

Les racines $3$-ième de l'unité sont $\{1,e^{2\ii \pi/3},e^{4\ii \pi/3}\}$

\medskip


\begin{minipage}{.4\textwidth}
  \myfigure{0.6}{
\tikzinput{fig_complexes14}
} 
\end{minipage} \pause\hspace*{1cm}
\begin{minipage}{.4\textwidth}
\myfigure{0.6}{
\tikzinput{fig_complexes15}
} 
\end{minipage}

\medskip

$z=-1=e^{\ii \pi}$ et $n=3$

Racines $3$-ième de $-1$ : $\{ e^{\frac{\ii\pi}{3}}, e^{\frac{\ii\pi+2\ii \pi}{3}}, e^{\frac{\ii\pi+4\ii \pi}{3}}\}
=\{ e^{\frac{\ii\pi}{3}}, -1, e^{-\frac{\ii\pi}{3}} \}$

 
\end{frame}


\begin{frame}
$z=1 \text{ et } n=5$

Les racines $5$-ième de l'unité sont 
$\displaystyle \left\{ 1,e^{\frac{2\ii\pi}{5}}, e^{\frac{4\ii\pi}{5}}, e^{\frac{6\ii\pi}{5}},e^{\frac{8\ii\pi}{5}}\right\}$

\myfigure{1}{
\tikzinput{fig_complexes12}
} 

\end{frame}



%%%%%%%%%%%%%%%%%%%%%%%%%%%%%%%%%%%%%%%%%%%%%%%%%%%%%%%%%%%%%%%%
\section{Applications \`a la trigonom\'etrie}

\begin{frame}

Partant de la définition de la notation exponentielle, 
$$e^{\ii  \theta} = \cos \theta + \ii  \sin \theta$$
\pause
on a donc aussi
$$e^{-\ii  \theta} = \cos \theta - \ii  \sin \theta$$

\bigskip

\pause

On obtient les \defi{formules d'Euler}, pour $\theta \in \Rr$ :

\mybox{$ \cos \theta = \dfrac{e^{\ii  \theta} + e^{- \ii  \theta}}{2} \quad , \quad 
   \sin \theta = \dfrac{e^{\ii  \theta} - e^{- \ii  \theta}}{2 \ii }$}



\end{frame}


\begin{frame}

\defi{Développement}\\
Exprimer $\sin n \theta$ ou $\cos n \theta$ en
fonction des puissances de $\cos \theta$ et $\sin \theta$

\pause

\begin{exemple}
\vspace{-1em}
\begin{eqnarray*}
  \cos 3 \theta + \ii  \sin 3 \theta & = & \left( \cos \theta + \ii  \sin \theta
  \right)^3\\
\pause
  & = & \cos^3 \theta + 3 \ii  \cos^2 \theta \sin \theta - 3 \cos \theta \sin^2
  \theta - \ii  \sin^3 \theta\\
\pause
  & = & \left( \cos^3 \theta - 3 \cos \theta \sin^2 \theta \right) + \ii  \left(
  3 \cos^2 \theta \sin \theta - \sin^3 \theta \right)
\end{eqnarray*}
\pause
D'où :
\[ \cos 3 \theta = \cos^3 \theta - 3 \cos \theta \sin^2 \theta \quad \text{ et } \quad 
   \sin 3 \theta = 3 \cos^2 \theta \sin \theta - \sin^3 \theta  \]
\end{exemple}

\end{frame}


\begin{frame}

\defi{Linéarisation}\\
Exprimer $\cos^n \theta$ ou $\sin^n \theta$ en fonction des $\cos k \theta$ 
et $\sin k\theta$

\pause

\begin{exemple}
\vspace{-1em}
\begin{eqnarray*}
  \sin^3 \theta 
\pause & = & \left( \frac{e^{\ii  \theta} - e^{- \ii  \theta}}{2 \ii }
  \right)^3\\
\pause
 & = & \frac{1}{- 8 \ii }  \left( (e^{\ii  \theta})^3 - 3 (e^{\ii  \theta})^2e^{- \ii  \theta} 
+ 3 e^{\ii  \theta}(e^{-\ii  \theta})^2 - (e^{- \ii  \theta})^3 \right)\\
\pause
 & = & \frac{1}{- 8 \ii }  \left( e^{3 \ii  \theta} - 3 e^{\ii  \theta} + 3 e^{-
  \ii  \theta} - e^{- 3 \ii  \theta} \right)\\
\pause
  & = & - \frac{1}{4} \left( \frac{e^{3 \ii  \theta} -  e^{-3 \ii \theta}}{2 \ii } - 3
  \frac{e^{\ii  \theta} - e^{-\ii  \theta}}{2 \ii } \right)\\
\pause
  & = & - \frac{\sin 3 \theta}{4}
  + \frac{3 \sin \theta}{4}
\end{eqnarray*}
\end{exemple}

\end{frame}


%%%%%%%%%%%%%%%%%%%%%%%%%%%%%%%%%%%%%%%%%%%%%%%%%%%%%%%%%%%%%%%%
\section{Mini-exercices}

\begin{frame}
\begin{miniexercice}
\begin{enumerate}
  \item Mettre les nombres suivants sont la forme module-argument (avec la notation exponentielle) :
$1$, $\ii$, $-1$, $-\ii$, $3\ii$, $1+\ii$, $\sqrt{3}-\ii$, $\overline{\sqrt{3}-\ii}$, $\frac{1}{\sqrt{3}-\ii}$, $(\sqrt{3}-\ii)^{20xx}$
où $20xx$ est l'année en cours.
  \item Calculer les racines $5$-ième de $\ii$. 
  \item Calculer les racines carrées de $\frac{\sqrt{3}}{2}+\frac{\ii}{2}$ de deux façons différentes.
En déduire les valeurs de $\cos \frac{\pi}{12}$ et $\sin \frac{\pi}{12}$.
  \item Donner sans calcul la valeur de $\omega_0 + \omega_1 + \cdots + \omega_{n-1}$, 
où les $\omega_i$ sont les racines $n$-ième de $1$.
  \item Développer $\cos (4\theta)$ ; linéariser $\cos^4 \theta$ ; calculer une primitive de $\theta\mapsto\cos^4 \theta$.
\end{enumerate}
\end{miniexercice}
\end{frame}



\end{document}