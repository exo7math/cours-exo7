
%%%%%%%%%%%%%%%%%% PREAMBULE %%%%%%%%%%%%%%%%%%

\documentclass[aspectratio=169,utf8]{beamer}
%\documentclass[aspectratio=169,handout]{beamer}

\usetheme{Boadilla}
%\usecolortheme{seahorse}
\usecolortheme[RGB={245,66,24}]{structure}
\useoutertheme{infolines}

% packages
\usepackage{amsfonts,amsmath,amssymb,amsthm}
\usepackage[utf8]{inputenc}
\usepackage[T1]{fontenc}
\usepackage{lmodern}

\usepackage[francais]{babel}
\usepackage{fancybox}
\usepackage{graphicx}

\usepackage{float}
\usepackage{xfrac}

%\usepackage[usenames, x11names]{xcolor}
\usepackage{tikz}
\usepackage{pgfplots}
\usepackage{datetime}



%-----  Package unités -----
\usepackage{siunitx}
\sisetup{locale = FR,detect-all,per-mode = symbol}

%\usepackage{mathptmx}
%\usepackage{fouriernc}
%\usepackage{newcent}
%\usepackage[mathcal,mathbf]{euler}

%\usepackage{palatino}
%\usepackage{newcent}
% \usepackage[mathcal,mathbf]{euler}



% \usepackage{hyperref}
% \hypersetup{colorlinks=true, linkcolor=blue, urlcolor=blue,
% pdftitle={Exo7 - Exercices de mathématiques}, pdfauthor={Exo7}}


%section
% \usepackage{sectsty}
% \allsectionsfont{\bf}
%\sectionfont{\color{Tomato3}\upshape\selectfont}
%\subsectionfont{\color{Tomato4}\upshape\selectfont}

%----- Ensembles : entiers, reels, complexes -----
\newcommand{\Nn}{\mathbb{N}} \newcommand{\N}{\mathbb{N}}
\newcommand{\Zz}{\mathbb{Z}} \newcommand{\Z}{\mathbb{Z}}
\newcommand{\Qq}{\mathbb{Q}} \newcommand{\Q}{\mathbb{Q}}
\newcommand{\Rr}{\mathbb{R}} \newcommand{\R}{\mathbb{R}}
\newcommand{\Cc}{\mathbb{C}} 
\newcommand{\Kk}{\mathbb{K}} \newcommand{\K}{\mathbb{K}}

%----- Modifications de symboles -----
\renewcommand{\epsilon}{\varepsilon}
\renewcommand{\Re}{\mathop{\text{Re}}\nolimits}
\renewcommand{\Im}{\mathop{\text{Im}}\nolimits}
%\newcommand{\llbracket}{\left[\kern-0.15em\left[}
%\newcommand{\rrbracket}{\right]\kern-0.15em\right]}

\renewcommand{\ge}{\geqslant}
\renewcommand{\geq}{\geqslant}
\renewcommand{\le}{\leqslant}
\renewcommand{\leq}{\leqslant}
\renewcommand{\epsilon}{\varepsilon}

%----- Fonctions usuelles -----
\newcommand{\ch}{\mathop{\text{ch}}\nolimits}
\newcommand{\sh}{\mathop{\text{sh}}\nolimits}
\renewcommand{\tanh}{\mathop{\text{th}}\nolimits}
\newcommand{\cotan}{\mathop{\text{cotan}}\nolimits}
\newcommand{\Arcsin}{\mathop{\text{arcsin}}\nolimits}
\newcommand{\Arccos}{\mathop{\text{arccos}}\nolimits}
\newcommand{\Arctan}{\mathop{\text{arctan}}\nolimits}
\newcommand{\Argsh}{\mathop{\text{argsh}}\nolimits}
\newcommand{\Argch}{\mathop{\text{argch}}\nolimits}
\newcommand{\Argth}{\mathop{\text{argth}}\nolimits}
\newcommand{\pgcd}{\mathop{\text{pgcd}}\nolimits} 


%----- Commandes divers ------
\newcommand{\ii}{\mathrm{i}}
\newcommand{\dd}{\text{d}}
\newcommand{\id}{\mathop{\text{id}}\nolimits}
\newcommand{\Ker}{\mathop{\text{Ker}}\nolimits}
\newcommand{\Card}{\mathop{\text{Card}}\nolimits}
\newcommand{\Vect}{\mathop{\text{Vect}}\nolimits}
\newcommand{\Mat}{\mathop{\text{Mat}}\nolimits}
\newcommand{\rg}{\mathop{\text{rg}}\nolimits}
\newcommand{\tr}{\mathop{\text{tr}}\nolimits}


%----- Structure des exercices ------

\newtheoremstyle{styleexo}% name
{2ex}% Space above
{3ex}% Space below
{}% Body font
{}% Indent amount 1
{\bfseries} % Theorem head font
{}% Punctuation after theorem head
{\newline}% Space after theorem head 2
{}% Theorem head spec (can be left empty, meaning ‘normal’)

%\theoremstyle{styleexo}
\newtheorem{exo}{Exercice}
\newtheorem{ind}{Indications}
\newtheorem{cor}{Correction}


\newcommand{\exercice}[1]{} \newcommand{\finexercice}{}
%\newcommand{\exercice}[1]{{\tiny\texttt{#1}}\vspace{-2ex}} % pour afficher le numero absolu, l'auteur...
\newcommand{\enonce}{\begin{exo}} \newcommand{\finenonce}{\end{exo}}
\newcommand{\indication}{\begin{ind}} \newcommand{\finindication}{\end{ind}}
\newcommand{\correction}{\begin{cor}} \newcommand{\fincorrection}{\end{cor}}

\newcommand{\noindication}{\stepcounter{ind}}
\newcommand{\nocorrection}{\stepcounter{cor}}

\newcommand{\fiche}[1]{} \newcommand{\finfiche}{}
\newcommand{\titre}[1]{\centerline{\large \bf #1}}
\newcommand{\addcommand}[1]{}
\newcommand{\video}[1]{}

% Marge
\newcommand{\mymargin}[1]{\marginpar{{\small #1}}}

\def\noqed{\renewcommand{\qedsymbol}{}}


%----- Presentation ------
\setlength{\parindent}{0cm}

%\newcommand{\ExoSept}{\href{http://exo7.emath.fr}{\textbf{\textsf{Exo7}}}}

\definecolor{myred}{rgb}{0.93,0.26,0}
\definecolor{myorange}{rgb}{0.97,0.58,0}
\definecolor{myyellow}{rgb}{1,0.86,0}

\newcommand{\LogoExoSept}[1]{  % input : echelle
{\usefont{U}{cmss}{bx}{n}
\begin{tikzpicture}[scale=0.1*#1,transform shape]
  \fill[color=myorange] (0,0)--(4,0)--(4,-4)--(0,-4)--cycle;
  \fill[color=myred] (0,0)--(0,3)--(-3,3)--(-3,0)--cycle;
  \fill[color=myyellow] (4,0)--(7,4)--(3,7)--(0,3)--cycle;
  \node[scale=5] at (3.5,3.5) {Exo7};
\end{tikzpicture}}
}


\newcommand{\debutmontitre}{
  \author{} \date{} 
  \thispagestyle{empty}
  \hspace*{-10ex}
  \begin{minipage}{\textwidth}
    \titlepage  
  \vspace*{-2.5cm}
  \begin{center}
    \LogoExoSept{2.5}
  \end{center}
  \end{minipage}

  \vspace*{-0cm}
  
  % Astuce pour que le background ne soit pas discrétisé lors de la conversion pdf -> png
\begin{tikzpicture}
        \fill[opacity=0,green!60!black] (0,0)--++(0,0)--++(0,0)--++(0,0)--cycle; 
\end{tikzpicture}

% toc S'affiche trop tot :
% \tableofcontents[hideallsubsections, pausesections]
}

\newcommand{\finmontitre}{
  \end{frame}
  \setcounter{framenumber}{0}
} % ne marche pas pour une raison obscure

%----- Commandes supplementaires ------

% \usepackage[landscape]{geometry}
% \geometry{top=1cm, bottom=3cm, left=2cm, right=10cm, marginparsep=1cm
% }
% \usepackage[a4paper]{geometry}
% \geometry{top=2cm, bottom=2cm, left=2cm, right=2cm, marginparsep=1cm
% }

%\usepackage{standalone}


% New command Arnaud -- november 2011
\setbeamersize{text margin left=24ex}
% si vous modifier cette valeur il faut aussi
% modifier le decalage du titre pour compenser
% (ex : ici =+10ex, titre =-5ex

\theoremstyle{definition}
%\newtheorem{proposition}{Proposition}
%\newtheorem{exemple}{Exemple}
%\newtheorem{theoreme}{Théorème}
%\newtheorem{lemme}{Lemme}
%\newtheorem{corollaire}{Corollaire}
%\newtheorem*{remarque*}{Remarque}
%\newtheorem*{miniexercice}{Mini-exercices}
%\newtheorem{definition}{Définition}

% Commande tikz
\usetikzlibrary{calc}
\usetikzlibrary{patterns,arrows}
\usetikzlibrary{matrix}
\usetikzlibrary{fadings} 

%definition d'un terme
\newcommand{\defi}[1]{{\color{myorange}\textbf{\emph{#1}}}}
\newcommand{\evidence}[1]{{\color{blue}\textbf{\emph{#1}}}}
\newcommand{\assertion}[1]{\emph{\og#1\fg}}  % pour chapitre logique
%\renewcommand{\contentsname}{Sommaire}
\renewcommand{\contentsname}{}
\setcounter{tocdepth}{2}



%------ Figures ------

\def\myscale{1} % par défaut 
\newcommand{\myfigure}[2]{  % entrée : echelle, fichier figure
\def\myscale{#1}
\begin{center}
\footnotesize
{#2}
\end{center}}


%------ Encadrement ------

\usepackage{fancybox}


\newcommand{\mybox}[1]{
\setlength{\fboxsep}{7pt}
\begin{center}
\shadowbox{#1}
\end{center}}

\newcommand{\myboxinline}[1]{
\setlength{\fboxsep}{5pt}
\raisebox{-10pt}{
\shadowbox{#1}
}
}

%--------------- Commande beamer---------------
\newcommand{\beameronly}[1]{#1} % permet de mettre des pause dans beamer pas dans poly


\setbeamertemplate{navigation symbols}{}
\setbeamertemplate{footline}  % tiré du fichier beamerouterinfolines.sty
{
  \leavevmode%
  \hbox{%
  \begin{beamercolorbox}[wd=.333333\paperwidth,ht=2.25ex,dp=1ex,center]{author in head/foot}%
    % \usebeamerfont{author in head/foot}\insertshortauthor%~~(\insertshortinstitute)
    \usebeamerfont{section in head/foot}{\bf\insertshorttitle}
  \end{beamercolorbox}%
  \begin{beamercolorbox}[wd=.333333\paperwidth,ht=2.25ex,dp=1ex,center]{title in head/foot}%
    \usebeamerfont{section in head/foot}{\bf\insertsectionhead}
  \end{beamercolorbox}%
  \begin{beamercolorbox}[wd=.333333\paperwidth,ht=2.25ex,dp=1ex,right]{date in head/foot}%
    % \usebeamerfont{date in head/foot}\insertshortdate{}\hspace*{2em}
    \insertframenumber{} / \inserttotalframenumber\hspace*{2ex} 
  \end{beamercolorbox}}%
  \vskip0pt%
}


\definecolor{mygrey}{rgb}{0.5,0.5,0.5}
\setlength{\parindent}{0cm}
%\DeclareTextFontCommand{\helvetica}{\fontfamily{phv}\selectfont}

% background beamer
\definecolor{couleurhaut}{rgb}{0.85,0.9,1}  % creme
\definecolor{couleurmilieu}{rgb}{1,1,1}  % vert pale
\definecolor{couleurbas}{rgb}{0.85,0.9,1}  % blanc
\setbeamertemplate{background canvas}[vertical shading]%
[top=couleurhaut,middle=couleurmilieu,midpoint=0.4,bottom=couleurbas] 
%[top=fondtitre!05,bottom=fondtitre!60]



\makeatletter
\setbeamertemplate{theorem begin}
{%
  \begin{\inserttheoremblockenv}
  {%
    \inserttheoremheadfont
    \inserttheoremname
    \inserttheoremnumber
    \ifx\inserttheoremaddition\@empty\else\ (\inserttheoremaddition)\fi%
    \inserttheorempunctuation
  }%
}
\setbeamertemplate{theorem end}{\end{\inserttheoremblockenv}}

\newenvironment{theoreme}[1][]{%
   \setbeamercolor{block title}{fg=structure,bg=structure!40}
   \setbeamercolor{block body}{fg=black,bg=structure!10}
   \begin{block}{{\bf Th\'eor\`eme }#1}
}{%
   \end{block}%
}


\newenvironment{proposition}[1][]{%
   \setbeamercolor{block title}{fg=structure,bg=structure!40}
   \setbeamercolor{block body}{fg=black,bg=structure!10}
   \begin{block}{{\bf Proposition }#1}
}{%
   \end{block}%
}

\newenvironment{corollaire}[1][]{%
   \setbeamercolor{block title}{fg=structure,bg=structure!40}
   \setbeamercolor{block body}{fg=black,bg=structure!10}
   \begin{block}{{\bf Corollaire }#1}
}{%
   \end{block}%
}

\newenvironment{mydefinition}[1][]{%
   \setbeamercolor{block title}{fg=structure,bg=structure!40}
   \setbeamercolor{block body}{fg=black,bg=structure!10}
   \begin{block}{{\bf Définition} #1}
}{%
   \end{block}%
}

\newenvironment{lemme}[0]{%
   \setbeamercolor{block title}{fg=structure,bg=structure!40}
   \setbeamercolor{block body}{fg=black,bg=structure!10}
   \begin{block}{\bf Lemme}
}{%
   \end{block}%
}

\newenvironment{remarque}[1][]{%
   \setbeamercolor{block title}{fg=black,bg=structure!20}
   \setbeamercolor{block body}{fg=black,bg=structure!5}
   \begin{block}{Remarque #1}
}{%
   \end{block}%
}


\newenvironment{exemple}[1][]{%
   \setbeamercolor{block title}{fg=black,bg=structure!20}
   \setbeamercolor{block body}{fg=black,bg=structure!5}
   \begin{block}{{\bf Exemple }#1}
}{%
   \end{block}%
}


\newenvironment{miniexercice}[0]{%
   \setbeamercolor{block title}{fg=structure,bg=structure!20}
   \setbeamercolor{block body}{fg=black,bg=structure!5}
   \begin{block}{Mini-exercices}
}{%
   \end{block}%
}


\newenvironment{tp}[0]{%
   \setbeamercolor{block title}{fg=structure,bg=structure!40}
   \setbeamercolor{block body}{fg=black,bg=structure!10}
   \begin{block}{\bf Travaux pratiques}
}{%
   \end{block}%
}
\newenvironment{exercicecours}[1][]{%
   \setbeamercolor{block title}{fg=structure,bg=structure!40}
   \setbeamercolor{block body}{fg=black,bg=structure!10}
   \begin{block}{{\bf Exercice }#1}
}{%
   \end{block}%
}
\newenvironment{algo}[1][]{%
   \setbeamercolor{block title}{fg=structure,bg=structure!40}
   \setbeamercolor{block body}{fg=black,bg=structure!10}
   \begin{block}{{\bf Algorithme}\hfill{\color{gray}\texttt{#1}}}
}{%
   \end{block}%
}


\setbeamertemplate{proof begin}{
   \setbeamercolor{block title}{fg=black,bg=structure!20}
   \setbeamercolor{block body}{fg=black,bg=structure!5}
   \begin{block}{{\footnotesize Démonstration}}
   \footnotesize
   \smallskip}
\setbeamertemplate{proof end}{%
   \end{block}}
\setbeamertemplate{qed symbol}{\openbox}


\makeatother
\usecolortheme[RGB={204,0,0}]{structure}
   
%%%%%%%%%%%%%%%%%%%%%%%%%%%%%%%%%%%%%%%%%%%%%%%%%%%%%%%%%%%%%
%%%%%%%%%%%%%%%%%%%%%%%%%%%%%%%%%%%%%%%%%%%%%%%%%%%%%%%%%%%%%


\begin{document}


\title{{\bf Déterminants}}
\subtitle{Propriétés du déterminant}

\begin{frame}
  
  \debutmontitre

  \pause

{\footnotesize
\hfill
\setbeamercovered{transparent=50}
\begin{minipage}{0.6\textwidth}
  \begin{itemize}
    \item<3-> Déterminant et matrices élémentaires
    \item<4-> Déterminant d'un produit
    \item<5-> Déterminant de l'inverse d'une matrice inversible
    \item<6-> Déterminant de la transposée
  \end{itemize}
\end{minipage}
}

\end{frame}

\setcounter{framenumber}{0}


%%%%%%%%%%%%%%%%%%%%%%%%%%%%%%%%%%%%%%%%%%%%%%%%%%%%%%%%%%%%%%%%
\section{Déterminant et matrices élémentaires}

\begin{frame}
 A chaque opération élémentaire on associe $E$ telle que $A'=A\times E$

\begin{enumerate}
  \item\pause  $C_i \leftarrow \lambda C_i$ avec $\lambda \neq 0$
  
  $E_{C_i \leftarrow \lambda C_i} = \text{diag}(1,\ldots,1,\lambda,1,\ldots,1)$
  \item\pause  $C_i \leftarrow C_i+\lambda C_j$ avec $\lambda \in \Kk$ (et $j\neq i$)  
  \[
E_{C_i \leftarrow C_i+\lambda C_j} = \left( \begin{smallmatrix}
1 \\ &&& \ddots \\ &&&&1 &&& \lambda  \\ 
 &&&&&&& \ddots \\ &&&&&&&& 1
\end{smallmatrix} \right)
\]
  \item\pause  $C_i \leftrightarrow C_j$
\[
E_{C_i \leftrightarrow C_j} = \left( \begin{smallmatrix}
1 \\ && \ddots  \\ &&& 1 \\ &&&& 0 &&& 1 \\ &&&&&& \ddots  \\ &&&& 1 &&& 0 \\
&&&&&&&&& \ddots \\ &&&&&&&&&& 1
\end{smallmatrix} \right)
\]
\end{enumerate}  

\end{frame}

%---------------------------------

\begin{frame}

\begin{proposition}
\begin{enumerate}
  \item\pause $\det E_{C_i \leftarrow \lambda C_i} = \lambda$
  \item\pause $\det E_{C_i \leftarrow C_i+\lambda C_j} = +1$
  \item\pause $\det E_{C_i \leftrightarrow C_j} = -1$
  \item\pause Si $E$ est une matrice élémentaire, $\det \left( A \times E \right) = \det A \times \det E$
\end{enumerate}  
\end{proposition}

\begin{itemize}
  \item\pause Par l'algorithme de Gauss pour toute matrice $A$ \pause
\[
T = A  \cdot E_1 \cdots E_r
\]
où $T$ est triangulaire et $E_i$ des matrices élémentaires
  \item\pause Alors
  $$
\begin{array}{rcl}
\det T 
  &=& \det (A \cdot E_1 \cdots E_r) \\
\pause  &=& \det (A \cdot E_1 \cdots E_{r-1})\cdot \det E_r \\
\pause  &=& \cdots \\
  &=& \det A \cdot \det E_1\cdot \det E_2 \cdots \det E_r
\end{array}
$$
\end{itemize}

\end{frame}

%---------------------------------

\begin{frame}

\begin{exemple}
Calculer \  $\det A$ \  avec 
$A  = \begin{pmatrix}
        0 & 3 & 2\\
        1 & -6 & 6\\
        5 & 9 & 1  
       \end{pmatrix}$

\pause     \pause
 $\det A $  \vspace{-.3cm}    
$$\begin{array}{rl}
  & =   
   (-1)\times \det \begin{pmatrix}
        3 & 0 & 2\\
        -6& 1 & 6\\
        9 & 5 & 1  
       \end{pmatrix}  \pause   \pause
 =   (-1)\times 3 \times \det \begin{pmatrix}
        1 & 0 & 2\\
        -2& 1 & 6\\
        3 & 5 & 1  
       \end{pmatrix}       \\ \pause  
 &  \qquad \onslide<2,3>\color{gray}{C_1 \leftrightarrow C_2}   \qquad \qquad \qquad \qquad  \onslide<4,5>\color{gray}{C_1 \leftarrow \frac{1}{3} C_1} \\ \pause
 & =   (-1)\times 3 \times \det \begin{pmatrix}
        1 & 0 & 0\\
        -2& 1 & 10\\
        3 & 5 & -5  
       \end{pmatrix}       \pause  \pause  
   = (-1)\times 3 \times \det \begin{pmatrix}
        1 & 0 & 0\\
        -2& 1 & 0\\
        3 & 5 & -55  
       \end{pmatrix}        \\ 
&     \qquad \onslide<6,7> \color{gray}{C_3 \leftarrow C_3-2C_1 }  \qquad \qquad  \qquad\qquad \onslide<8,9>\color{gray}{ C_3 \leftarrow C_3-10C_2 }\\ \pause  
& =   (-1)\times 3 \times (-55)  \\\pause  
& =  165 
\end{array}$$

\end{exemple}  

\end{frame}


%%%%%%%%%%%%%%%%%%%%%%%%%%%%%%%%%%%%%%%%%%%%%%%%%%%%%%%%%%%%%%%%
\section{Déterminant d'un produit}

\begin{frame}

\begin{theoreme}
\mybox{$\det (A \times B)=\det A  \times \det B$}
\end{theoreme}

\end{frame}





%%%%%%%%%%%%%%%%%%%%%%%%%%%%%%%%%%%%%%%%%%%%%%%%%%%%%%%%%%%%%%%%
\section{Déterminant des matrices inversibles}

\begin{frame}

\begin{theoreme} 
Une matrice carrée $A$ est inversible si et seulement si son
déterminant est non nul\pause.  De plus si $A$ est inversible,  
alors: 
\mybox{$\displaystyle \det \big(A^{-1}\big)=\frac1{\det A}$}
\end{theoreme}

\pause
\begin{exemple}
Deux  matrices semblables ont même déterminant

\begin{itemize}
  \item \pause Soit $B=P^{-1}AP$ avec $P\in GL_n(\Kk)$ 
  \item\pause Par multiplicativité du déterminant
$$\det B=\det (P^{-1}AP)=\det P^{-1}\det A\det P \pause =\det A$$
puisque $\det P^{-1}=\frac{1}{\det P}$
\end{itemize}
\end{exemple}

\end{frame}


%%%%%%%%%%%%%%%%%%%%%%%%%%%%%%%%%%%%%%%%%%%%%%%%%%%%%%%%%%%%%%%%
\section{Déterminant de la transposée}

\begin{frame}

\begin{theoreme}
\mybox{$\det \big(A^T\big)=\det A$}
\end{theoreme}

\pause
\bigskip

\begin{remarque}
Opérations élémentaires sur les lignes :
\begin{enumerate}
  \item\pause $L_i \leftarrow \lambda L_i$ avec $\lambda \neq 0$: le déterminant est 
  multiplié par $\lambda$
  
  \item\pause $L_i \leftarrow L_i+\lambda L_j$ avec $\lambda \in \Kk$ (et $j\neq i$) :
  le déterminant ne change pas
  
  \item\pause $L_i \leftrightarrow L_j$ : le déterminant change de signe
\end{enumerate} 
\end{remarque}

\end{frame}

%%%%%%%%%%%%%%%%%%%%%%%%%%%%%%%%%%%%%%%%%%%%%%%%%%%%%%%%%%%%%%%%
\section{Déterminant d'un produit}

%---------------------------------

\begin{frame}
\centerline{Théorème: $\det (A \times B)=\det A  \times \det B$}
\vspace*{-1ex}
\begin{proof}
\pause
\begin{itemize}
 \item Si  $E$ est une matrice élémentaire $\det (A \times E) = \det A \times \det E$
 
 \pause 

 \item Si $B$ est inversible 
 \begin{itemize}
  \item\pause $B E_1\cdots E_r = I_{n}$ \  où les $E_i$ sont des matrices élémentaires  \vspace{.2cm}

    \item\pause $\det (B  \cdot E_1 E_2 \cdots E_r)
\pause=\det B  \cdot \det E_1  \cdot \det E_2 \cdots \det E_r \pause= \det I_n = 1$\vspace{.2cm}
 \item\pause On en déduit 
$
\det B=\frac1{\det E_1 \det E_2 \cdots \det E_r}
$\vspace{.2cm}
 \item\pause Or 
$
(AB) \cdot (E_1\cdots E_r) = A \cdot I_n =A 
$\vspace{.2cm}
 \item\pause Ainsi 
$
\det(A B E_1 \cdots E_r) = \det(A B) \cdot \det E_1  \cdots \det E_r \pause= \det A
$\vspace{.2cm}
 \item\pause Donc :
$\det(AB) = \det A \times \frac1{\det E_1 \cdots \det E_r} \pause= \det A \times \det B$\vspace{.2cm}
\end{itemize}
\item \pause Si $B$ n'est pas inversible, c'est-à-dire si $\rg B<n$
\begin{itemize}
  \item\pause Alors il existe un vecteur colonne $X$ tel que $BX=0$ 
\pause $\implies \det B=0$ 
  \item\pause  Or $BX=0 \implies (AB)X=0$ 
  \item\pause Donc $AB$ n'est pas inversible\pause $\implies \det (AB)=0\pause =\det A\det B$\qedhere
\end{itemize}
\end{itemize} 
\end{proof}
\end{frame}

%%%%%%%%%%%%%%%%%%%%%%%%%%%%%%%%%%%%%%%%%%%%%%%%%%%%%%%%%%%%%%%%
\section{Mini-exercices}

\begin{frame}
\begin{miniexercice}
\begin{enumerate}
  \item Soient $A = \begin{pmatrix}a&1&3\\0&b&2\\0&0&c\end{pmatrix}$
  et $B = \begin{pmatrix}1&0&2\\0&d&0\\-2&0&1\end{pmatrix}$.
  Calculer, lorsque c'est possible, les déterminants des matrices $A$, $B$,
  $A^{-1}$, $B^{-1}$, $A^T$, $B^T$, $AB$, $BA$.
  
  \item Calculer le déterminant des matrices suivantes en se 
  ramenant par des opérations élémentaires à une matrice triangulaire.
\begin{gather*}
\begin{pmatrix}a&b\\c&d\end{pmatrix}
  \qquad
  \begin{pmatrix}1&0&6\\3&4&15\\5&6&21\end{pmatrix}\\
  \begin{pmatrix}1&1&1\\1&j&j^2\\1&j^2&j\end{pmatrix} \text{ avec } j=e^{\frac{2\ii\pi}{3}}\qquad
  \begin{pmatrix}1&2&4&8\\1&3&9&27\\1&4&16&64\\1&5&25&125\end{pmatrix}
\end{gather*}
\end{enumerate}
\end{miniexercice}
\end{frame}

\end{document}