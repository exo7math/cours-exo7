
%%%%%%%%%%%%%%%%%% PREAMBULE %%%%%%%%%%%%%%%%%%


\documentclass[12pt]{article}

\usepackage{amsfonts,amsmath,amssymb,amsthm}
\usepackage[utf8]{inputenc}
\usepackage[T1]{fontenc}
\usepackage[francais]{babel}


% packages
\usepackage{amsfonts,amsmath,amssymb,amsthm}
\usepackage[utf8]{inputenc}
\usepackage[T1]{fontenc}
%\usepackage{lmodern}

\usepackage[francais]{babel}
\usepackage{fancybox}
\usepackage{graphicx}

\usepackage{float}

%\usepackage[usenames, x11names]{xcolor}
\usepackage{tikz}
\usepackage{datetime}

\usepackage{mathptmx}
%\usepackage{fouriernc}
%\usepackage{newcent}
\usepackage[mathcal,mathbf]{euler}

%\usepackage{palatino}
%\usepackage{newcent}


% Commande spéciale prompteur

%\usepackage{mathptmx}
%\usepackage[mathcal,mathbf]{euler}
%\usepackage{mathpple,multido}

\usepackage[a4paper]{geometry}
\geometry{top=2cm, bottom=2cm, left=1cm, right=1cm, marginparsep=1cm}

\newcommand{\change}{{\color{red}\rule{\textwidth}{1mm}\\}}

\newcounter{mydiapo}

\newcommand{\diapo}{\newpage
\hfill {\normalsize  Diapo \themydiapo \quad \texttt{[\jobname]}} \\
\stepcounter{mydiapo}}


%%%%%%% COULEURS %%%%%%%%%%

% Pour blanc sur noir :
%\pagecolor[rgb]{0.5,0.5,0.5}
% \pagecolor[rgb]{0,0,0}
% \color[rgb]{1,1,1}



%\DeclareFixedFont{\myfont}{U}{cmss}{bx}{n}{18pt}
\newcommand{\debuttexte}{
%%%%%%%%%%%%% FONTES %%%%%%%%%%%%%
\renewcommand{\baselinestretch}{1.5}
\usefont{U}{cmss}{bx}{n}
\bfseries

% Taille normale : commenter le reste !
%Taille Arnaud
%\fontsize{19}{19}\selectfont

% Taille Barbara
%\fontsize{21}{22}\selectfont

%Taille François
\fontsize{25}{30}\selectfont

%Taille Pascal
%\fontsize{25}{30}\selectfont

%Taille Laura
%\fontsize{30}{35}\selectfont


%\myfont
%\usefont{U}{cmss}{bx}{n}

%\Huge
%\addtolength{\parskip}{\baselineskip}
}


% \usepackage{hyperref}
% \hypersetup{colorlinks=true, linkcolor=blue, urlcolor=blue,
% pdftitle={Exo7 - Exercices de mathématiques}, pdfauthor={Exo7}}


%section
% \usepackage{sectsty}
% \allsectionsfont{\bf}
%\sectionfont{\color{Tomato3}\upshape\selectfont}
%\subsectionfont{\color{Tomato4}\upshape\selectfont}

%----- Ensembles : entiers, reels, complexes -----
\newcommand{\Nn}{\mathbb{N}} \newcommand{\N}{\mathbb{N}}
\newcommand{\Zz}{\mathbb{Z}} \newcommand{\Z}{\mathbb{Z}}
\newcommand{\Qq}{\mathbb{Q}} \newcommand{\Q}{\mathbb{Q}}
\newcommand{\Rr}{\mathbb{R}} \newcommand{\R}{\mathbb{R}}
\newcommand{\Cc}{\mathbb{C}} 
\newcommand{\Kk}{\mathbb{K}} \newcommand{\K}{\mathbb{K}}

%----- Modifications de symboles -----
\renewcommand{\epsilon}{\varepsilon}
\renewcommand{\Re}{\mathop{\text{Re}}\nolimits}
\renewcommand{\Im}{\mathop{\text{Im}}\nolimits}
%\newcommand{\llbracket}{\left[\kern-0.15em\left[}
%\newcommand{\rrbracket}{\right]\kern-0.15em\right]}

\renewcommand{\ge}{\geqslant}
\renewcommand{\geq}{\geqslant}
\renewcommand{\le}{\leqslant}
\renewcommand{\leq}{\leqslant}

%----- Fonctions usuelles -----
\newcommand{\ch}{\mathop{\mathrm{ch}}\nolimits}
\newcommand{\sh}{\mathop{\mathrm{sh}}\nolimits}
\renewcommand{\tanh}{\mathop{\mathrm{th}}\nolimits}
\newcommand{\cotan}{\mathop{\mathrm{cotan}}\nolimits}
\newcommand{\Arcsin}{\mathop{\mathrm{Arcsin}}\nolimits}
\newcommand{\Arccos}{\mathop{\mathrm{Arccos}}\nolimits}
\newcommand{\Arctan}{\mathop{\mathrm{Arctan}}\nolimits}
\newcommand{\Argsh}{\mathop{\mathrm{Argsh}}\nolimits}
\newcommand{\Argch}{\mathop{\mathrm{Argch}}\nolimits}
\newcommand{\Argth}{\mathop{\mathrm{Argth}}\nolimits}
\newcommand{\pgcd}{\mathop{\mathrm{pgcd}}\nolimits} 

\newcommand{\Card}{\mathop{\text{Card}}\nolimits}
\newcommand{\Ker}{\mathop{\text{Ker}}\nolimits}
\newcommand{\id}{\mathop{\text{id}}\nolimits}
\newcommand{\ii}{\mathrm{i}}
\newcommand{\dd}{\mathrm{d}}
\newcommand{\Vect}{\mathop{\text{Vect}}\nolimits}
\newcommand{\Mat}{\mathop{\mathrm{Mat}}\nolimits}
\newcommand{\rg}{\mathop{\text{rg}}\nolimits}
\newcommand{\tr}{\mathop{\text{tr}}\nolimits}
\newcommand{\ppcm}{\mathop{\text{ppcm}}\nolimits}

%----- Structure des exercices ------

\newtheoremstyle{styleexo}% name
{2ex}% Space above
{3ex}% Space below
{}% Body font
{}% Indent amount 1
{\bfseries} % Theorem head font
{}% Punctuation after theorem head
{\newline}% Space after theorem head 2
{}% Theorem head spec (can be left empty, meaning ‘normal’)

%\theoremstyle{styleexo}
\newtheorem{exo}{Exercice}
\newtheorem{ind}{Indications}
\newtheorem{cor}{Correction}


\newcommand{\exercice}[1]{} \newcommand{\finexercice}{}
%\newcommand{\exercice}[1]{{\tiny\texttt{#1}}\vspace{-2ex}} % pour afficher le numero absolu, l'auteur...
\newcommand{\enonce}{\begin{exo}} \newcommand{\finenonce}{\end{exo}}
\newcommand{\indication}{\begin{ind}} \newcommand{\finindication}{\end{ind}}
\newcommand{\correction}{\begin{cor}} \newcommand{\fincorrection}{\end{cor}}

\newcommand{\noindication}{\stepcounter{ind}}
\newcommand{\nocorrection}{\stepcounter{cor}}

\newcommand{\fiche}[1]{} \newcommand{\finfiche}{}
\newcommand{\titre}[1]{\centerline{\large \bf #1}}
\newcommand{\addcommand}[1]{}
\newcommand{\video}[1]{}

% Marge
\newcommand{\mymargin}[1]{\marginpar{{\small #1}}}



%----- Presentation ------
\setlength{\parindent}{0cm}

%\newcommand{\ExoSept}{\href{http://exo7.emath.fr}{\textbf{\textsf{Exo7}}}}

\definecolor{myred}{rgb}{0.93,0.26,0}
\definecolor{myorange}{rgb}{0.97,0.58,0}
\definecolor{myyellow}{rgb}{1,0.86,0}

\newcommand{\LogoExoSept}[1]{  % input : echelle
{\usefont{U}{cmss}{bx}{n}
\begin{tikzpicture}[scale=0.1*#1,transform shape]
  \fill[color=myorange] (0,0)--(4,0)--(4,-4)--(0,-4)--cycle;
  \fill[color=myred] (0,0)--(0,3)--(-3,3)--(-3,0)--cycle;
  \fill[color=myyellow] (4,0)--(7,4)--(3,7)--(0,3)--cycle;
  \node[scale=5] at (3.5,3.5) {Exo7};
\end{tikzpicture}}
}



\theoremstyle{definition}
%\newtheorem{proposition}{Proposition}
%\newtheorem{exemple}{Exemple}
%\newtheorem{theoreme}{Théorème}
\newtheorem{lemme}{Lemme}
\newtheorem{corollaire}{Corollaire}
%\newtheorem*{remarque*}{Remarque}
%\newtheorem*{miniexercice}{Mini-exercices}
%\newtheorem{definition}{Définition}




%definition d'un terme
\newcommand{\defi}[1]{{\color{myorange}\textbf{\emph{#1}}}}
\newcommand{\evidence}[1]{{\color{blue}\textbf{\emph{#1}}}}



 %----- Commandes divers ------

\newcommand{\codeinline}[1]{\texttt{#1}}

%%%%%%%%%%%%%%%%%%%%%%%%%%%%%%%%%%%%%%%%%%%%%%%%%%%%%%%%%%%%%
%%%%%%%%%%%%%%%%%%%%%%%%%%%%%%%%%%%%%%%%%%%%%%%%%%%%%%%%%%%%%



\begin{document}

\debuttexte


%%%%%%%%%%%%%%%%%%%%%%%%%%%%%%%%%%%%%%%%%%%%%%%%%%%%%%%%%%%
\diapo

\change


Dans cette troisième  partie du chapitre sur les systèmes linéaires, nous allons parler de la méthode du pivot de Gauss.


\change

Nous commencerons par définir ce qu'est un système échelonné,

\change

puis nous parlerons des opérations sur un système qui permettent de conserver un système équivalent,

\change

avant de passer à la méthode du pivot en elle-même.

\change

Nous conclurons en examinant le cas particulier des systèmes homogènes.

%%%%%%%%%%%%%%%%%%%%%%%%%%%%%%%%%%%%%%%%%%%%%%%%%%%%%%%%%%
\diapo

Nous commençons donc par la définition de *système échelonné*. 
C'est un système dont le nombre de coefficients nuls en début de ligne croit 
d'au moins $1$ lorsque l'on passe d'une ligne à une autre.

\change

Si de plus le premier coefficient non nul de chaque ligne vaut $1$, 
et que c'est le seul élément non nul sur sa colonne, on dit que le système est *échelonné réduit*.

\change

Voici deux exemples.
Le premier système est échelonné : le nombre de coefficients nuls en début de ligne, ici
aucun, là $1$ coefficient nul, ici $3$, augmente de ligne en ligne. 
Mais il n'est pas réduit, car par exemple la première ligne commence par le coefficient $2$.

\change


Le deuxième système n'est pas échelonné : la dernière ligne commence 
avec la même variable que la ligne au-dessus.


%%%%%%%%%%%%%%%%%%%%%%%%%%%%%%%%%%%%%%%%%%%%%%%%%%%%%%%%%%%
\diapo
Il se trouve que les systèmes linéaires sous une forme échelonnée réduite 
sont particulièrement simples à résoudre. L'exemple suivant le montre bien :

\change

Ce système se résout facilement, on trouve :

$$\left\{\begin{array}{ccc}
x_1&=&25-2x_3\\
x_2&=&16+2x_3\\
x_4&=&1.
\end{array}\right.
$$

\change



En d'autres termes, pour toute valeur de $x_3$ réelle, 
les valeurs de $x_1$, $x_2$ et $x_4$ calculées ci-dessus fournissent 
une solution du système, et on les a ainsi toutes obtenues. 

%%%%%%%%%%%%%%%%%%%%%%%%%%%%%%%%%%%%%%%%%%%%%%%%%%%%%%%%%%
\diapo

Les trois opérations suivantes permettent de passer d'un système linéaire
à un système linéaire %sans changer l'ensemble des solutions.
*équivalent*, en d'autres termes on ne change pas l'ensemble des solutions.

\change

\begin{enumerate}
	\item Première opération : $L_i \leftarrow \lambda L_i$ 
	
	\change
	
	on peut multiplier une ligne par un réel *non nul*.
	
\change

	\item Deuxième opération : $L_i \leftarrow L_i+\lambda L_j$
	
	
	\change
	
	on peut ajouter à une équation un multiple d'une *autre* équation (attention ici $j\neq i$ signifie que les équations sont distinctes).
	
\change

	\item Troisième opération : $L_i \leftrightarrow L_j$
	
	\change
	
	l'échange de deux équations.
	
\change

Chacune de ces opérations ne change pas l'ensemble des solutions.

Les systèmes avant et celui obtenu après l'opération sont équivalents.

\end{enumerate}


%%%%%%%%%%%%%%%%%%%%%%%%%%%%%%%%%%%%%%%%%%%%%%%%%%%%%%%%%%
\diapo

Utilisons ces opérations élémentaires pour résoudre le système suivant.

\change

On garde les lignes 1 et 3.

\change

Commençons par l'opération $L_2 \leftarrow L_2 - 2L_1$ : 
on soustrait à la deuxième équation deux fois la première équation.

\change

On obtient un système équivalent avec une nouvelle deuxième ligne (plus simple) :
$-3y -9z  =  -3$

\change

Ensuite on garde les 2 premières lignes.

\change



On effectue $L_3 \leftarrow L_3 + L_1$  pour simplifier la troisième ligne, 

\change

ce qui donne : $-2y -2z  =  -6$.


\change

On continue pour faire apparaître un coefficient $1$ en tête de la deuxième ligne ;
pour cela on divise la ligne $L_2$ par $-3$ :

\change

$y   +3z  =  1$.


%%%%%%%%%%%%%%%%%%%%%%%%%%%%%%%%%%%%%%%%%%%%%%%%%%%%%%%%%%%
\diapo

Repartant du système précédent,

\change

\change


on applique l'opération $L_3 \leftarrow L_3 + 2L_2$  pour simplifier à nouveau la troisième ligne. 

\change

$4z  = -4$.

\change

\change

Puis on divise la dernière ligne par $4$ 

\change

pour aboutir à un système échelonné :


\change

Avec un peut plus d'efforts, on pourrait même obtenir un système échelonné et réduit,

c-à-d avec des $0$ ici, là et là.

Cependant, ce n'est pas nécessaire, ce dernier système se résout aisément 
de proche en proche en partant de la dernière ligne. On obtient ainsi $z=-1$,

puis à partir de la deuxième ligne on obtient  $y=4$,

et enfin en reportant les valeurs déjà obtenu dans la première ligne on trouve $x=2$. 
L'unique solution du système est donc $(2,4,-1)$.


%%%%%%%%%%%%%%%%%%%%%%%%%%%%%%%%%%%%%%%%%%%%%%%%%%%%%%%%%%
\diapo

La méthode du pivot de Gauss permet de trouver les solutions de n'importe 
quel système linéaire.
Nous allons décrire cet algorithme sur un exemple.
Il s'agit d'une description précise d'une suite d'opérations à effectuer, 
qui dépendent de la situation et d'un ordre précis. Ce processus 
aboutit toujours (et en plus assez rapidement)
à un système échelonné puis réduit, qui conduit immédiatement aux solutions du système.

\change

La première étape est de passer à une forme échelonnée.

\change



Pour appliquer la méthode du pivot de Gauss, il faut 
d'abord que le premier coefficient de la première ligne soit non nul.

\change

Comme ce n'est pas le cas ici,
on conserve la troisième ligne.
et on échange les deux premières lignes 

\change 

par l'opération élémentaire
$L_1 \leftrightarrow L_2$ :


\change

Nous avons déjà un coefficient $1$ devant le $x_1$ de la première ligne.
On dit que nous avons un \defi{pivot} en position $(1,1)$ (première ligne, première colonne).
Ce pivot sert de base pour éliminer tous les autres termes sur la même colonne.

Il n'y a pas de terme $x_1$ sur le deuxième ligne.
Faisons disparaître le terme $x_1$ de la troisième ligne ;

\change

\change

pour cela on fait l'opération élémentaire $L_3 \leftarrow L_3+L_1$ :

$x_2   -3x_4  = 3$.


%%%%%%%%%%%%%%%%%%%%%%%%%%%%%%%%%%%%%%%%%%%%%%%%%%%%%%%%%%%
\diapo
On change le signe de la seconde ligne ($L_2 \leftarrow -L_2$) pour faire apparaître 
$1$ au coefficient du pivot $(2,2)$ (deuxième ligne, deuxième colonne) :

\change

 On fait disparaître le terme $x_2$ de la troisième ligne,
 
\change
 
par l'opération $L_3 \leftarrow L_3 -L_2$ :

\change

$2x_3 +10x_4  = 8$.


\change
 puis on fait apparaître un coefficient $1$ pour le pivot de la position $(3,3)$ :
 
 
 \change
 
$L_3 \leftarrow \frac12 L_3$

\change

Le système est maintenant sous forme échelonnée.  



%%%%%%%%%%%%%%%%%%%%%%%%%%%%%%%%%%%%%%%%%%%%%%%%%%%%%%%%%%
\diapo
Il reste à mettre le système sous forme échelonnée réduite :

il s'agit de faire apparaître des zéros au-dessus des pivots.

Pour cela, on ajoute à une ligne des multiples adéquats des lignes situées 
au-dessous d'elle, en partant du bas à droite vers le haut à gauche.

Repartant du dernier système obtenu, on fait apparaître des $0$ sur la troisième colonne en utilisant le pivot de la troisième ligne. 

\change

\change

\change

L'opération $L_2 \leftarrow L_2 + 2L_3$ permet d'obtenir un premier zéro en position $(2,3)$,

\change

\change

\change

puis l'opération $L_1 \leftarrow L_1 -3 L_3$ donne un zéro en position $(1,3)$.

\change

\change

\change

On fait apparaître des $0$ sur la deuxième colonne (en utilisant le pivot de la deuxième ligne) : 
l'opération $L_1 \leftarrow L_1 + 2L_2$ permet d'obtenir un zéro en position $(1,2)$.

Le système est sous forme échelonnée réduite.



%%%%%%%%%%%%%%%%%%%%%%%%%%%%%%%%%%%%%%%%%%%%%%%%%%%%%%%%%%%
\diapo

Le système est maintenant très simple à résoudre. En choisissant $x_4$ comme variable libre, on peut exprimer
$x_1,x_2,x_3$ en fonction de $x_4$ :

\change

$$x_1=4x_4-2, \quad x_2 = 3x_4+3,\quad x_3=-5x_4+4.$$


\change

Ce qui permet d'obtenir toutes les solutions du système :
$$\mathcal{S}= \big\lbrace (4x_4-2,3x_4+3,-5x_4+4,x_4) \mid x_4 \in \Rr \big\rbrace.$$

%%%%%%%%%%%%%%%%%%%%%%%%%%%%%%%%%%%%%%%%%%%%%%%%%%%%%%%%%%%
\diapo

Le fait que l'on puisse toujours se ramener à un système échelonné 
réduit implique le résultat suivant :

Tout système linéaire homogène dont le nombre d'inconnues est strictement 
supérieur au nombre d'équations admet une infinité de solutions. 


\change


Par exemple, considérons le système homogène à cinq inconnues et quatre équations suivant :

\change


Après calculs, on trouve que sa forme échelonnée réduite est la suivante, ici il y a toujours 
5 inconnues mais seulement 3 équations :

\change

On obtient que l'ensemble des solutions dépend des deux paramètres \(x_2\) et  \(x_5\), et est donc bien infini. 




%%%%%%%%%%%%%%%%%%%%%%%%%%%%%%%%%%%%%%%%%%%%%%%%%%%%%%%%%%%
\diapo


Voici deux pages d'exercices pour vous aider à maîtriser la méthode du pivot de Gauss.

%%%%%%%%%%%%%%%%%%%%%%%%%%%%%%%%%%%%%%%%%%%%%%%%%%%%%%%%%%%
\diapo

Voici la fin des exercices.

\end{document}
