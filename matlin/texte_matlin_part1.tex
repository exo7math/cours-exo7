
%%%%%%%%%%%%%%%%%% PREAMBULE %%%%%%%%%%%%%%%%%%


\documentclass[12pt]{article}

\usepackage{amsfonts,amsmath,amssymb,amsthm}
\usepackage[utf8]{inputenc}
\usepackage[T1]{fontenc}
\usepackage[francais]{babel}


% packages
\usepackage{amsfonts,amsmath,amssymb,amsthm}
\usepackage[utf8]{inputenc}
\usepackage[T1]{fontenc}
%\usepackage{lmodern}

\usepackage[francais]{babel}
\usepackage{fancybox}
\usepackage{graphicx}

\usepackage{float}

%\usepackage[usenames, x11names]{xcolor}
\usepackage{tikz}
\usepackage{datetime}

\usepackage{mathptmx}
%\usepackage{fouriernc}
%\usepackage{newcent}
\usepackage[mathcal,mathbf]{euler}

%\usepackage{palatino}
%\usepackage{newcent}


% Commande spéciale prompteur

%\usepackage{mathptmx}
%\usepackage[mathcal,mathbf]{euler}
%\usepackage{mathpple,multido}

\usepackage[a4paper]{geometry}
\geometry{top=2cm, bottom=2cm, left=1cm, right=1cm, marginparsep=1cm}

\newcommand{\change}{{\color{red}\rule{\textwidth}{1mm}\\}}

\newcounter{mydiapo}

\newcommand{\diapo}{\newpage
\hfill {\normalsize  Diapo \themydiapo \quad \texttt{[\jobname]}} \\
\stepcounter{mydiapo}}


%%%%%%% COULEURS %%%%%%%%%%

% Pour blanc sur noir :
%\pagecolor[rgb]{0.5,0.5,0.5}
% \pagecolor[rgb]{0,0,0}
% \color[rgb]{1,1,1}



%\DeclareFixedFont{\myfont}{U}{cmss}{bx}{n}{18pt}
\newcommand{\debuttexte}{
%%%%%%%%%%%%% FONTES %%%%%%%%%%%%%
\renewcommand{\baselinestretch}{1.5}
\usefont{U}{cmss}{bx}{n}
\bfseries

% Taille normale : commenter le reste !
%Taille Arnaud
%\fontsize{19}{19}\selectfont

% Taille Barbara
%\fontsize{21}{22}\selectfont

%Taille François
\fontsize{25}{30}\selectfont

%Taille Pascal
%\fontsize{25}{30}\selectfont

%Taille Laura
%\fontsize{30}{35}\selectfont


%\myfont
%\usefont{U}{cmss}{bx}{n}

%\Huge
%\addtolength{\parskip}{\baselineskip}
}


% \usepackage{hyperref}
% \hypersetup{colorlinks=true, linkcolor=blue, urlcolor=blue,
% pdftitle={Exo7 - Exercices de mathématiques}, pdfauthor={Exo7}}


%section
% \usepackage{sectsty}
% \allsectionsfont{\bf}
%\sectionfont{\color{Tomato3}\upshape\selectfont}
%\subsectionfont{\color{Tomato4}\upshape\selectfont}

%----- Ensembles : entiers, reels, complexes -----
\newcommand{\Nn}{\mathbb{N}} \newcommand{\N}{\mathbb{N}}
\newcommand{\Zz}{\mathbb{Z}} \newcommand{\Z}{\mathbb{Z}}
\newcommand{\Qq}{\mathbb{Q}} \newcommand{\Q}{\mathbb{Q}}
\newcommand{\Rr}{\mathbb{R}} \newcommand{\R}{\mathbb{R}}
\newcommand{\Cc}{\mathbb{C}} 
\newcommand{\Kk}{\mathbb{K}} \newcommand{\K}{\mathbb{K}}

%----- Modifications de symboles -----
\renewcommand{\epsilon}{\varepsilon}
\renewcommand{\Re}{\mathop{\text{Re}}\nolimits}
\renewcommand{\Im}{\mathop{\text{Im}}\nolimits}
%\newcommand{\llbracket}{\left[\kern-0.15em\left[}
%\newcommand{\rrbracket}{\right]\kern-0.15em\right]}

\renewcommand{\ge}{\geqslant}
\renewcommand{\geq}{\geqslant}
\renewcommand{\le}{\leqslant}
\renewcommand{\leq}{\leqslant}

%----- Fonctions usuelles -----
\newcommand{\ch}{\mathop{\mathrm{ch}}\nolimits}
\newcommand{\sh}{\mathop{\mathrm{sh}}\nolimits}
\renewcommand{\tanh}{\mathop{\mathrm{th}}\nolimits}
\newcommand{\cotan}{\mathop{\mathrm{cotan}}\nolimits}
\newcommand{\Arcsin}{\mathop{\mathrm{Arcsin}}\nolimits}
\newcommand{\Arccos}{\mathop{\mathrm{Arccos}}\nolimits}
\newcommand{\Arctan}{\mathop{\mathrm{Arctan}}\nolimits}
\newcommand{\Argsh}{\mathop{\mathrm{Argsh}}\nolimits}
\newcommand{\Argch}{\mathop{\mathrm{Argch}}\nolimits}
\newcommand{\Argth}{\mathop{\mathrm{Argth}}\nolimits}
\newcommand{\pgcd}{\mathop{\mathrm{pgcd}}\nolimits} 

\newcommand{\Card}{\mathop{\text{Card}}\nolimits}
\newcommand{\Ker}{\mathop{\text{Ker}}\nolimits}
\newcommand{\id}{\mathop{\text{id}}\nolimits}
\newcommand{\ii}{\mathrm{i}}
\newcommand{\dd}{\mathrm{d}}
\newcommand{\Vect}{\mathop{\text{Vect}}\nolimits}
\newcommand{\Mat}{\mathop{\mathrm{Mat}}\nolimits}
\newcommand{\rg}{\mathop{\text{rg}}\nolimits}
\newcommand{\tr}{\mathop{\text{tr}}\nolimits}
\newcommand{\ppcm}{\mathop{\text{ppcm}}\nolimits}

%----- Structure des exercices ------

\newtheoremstyle{styleexo}% name
{2ex}% Space above
{3ex}% Space below
{}% Body font
{}% Indent amount 1
{\bfseries} % Theorem head font
{}% Punctuation after theorem head
{\newline}% Space after theorem head 2
{}% Theorem head spec (can be left empty, meaning ‘normal’)

%\theoremstyle{styleexo}
\newtheorem{exo}{Exercice}
\newtheorem{ind}{Indications}
\newtheorem{cor}{Correction}


\newcommand{\exercice}[1]{} \newcommand{\finexercice}{}
%\newcommand{\exercice}[1]{{\tiny\texttt{#1}}\vspace{-2ex}} % pour afficher le numero absolu, l'auteur...
\newcommand{\enonce}{\begin{exo}} \newcommand{\finenonce}{\end{exo}}
\newcommand{\indication}{\begin{ind}} \newcommand{\finindication}{\end{ind}}
\newcommand{\correction}{\begin{cor}} \newcommand{\fincorrection}{\end{cor}}

\newcommand{\noindication}{\stepcounter{ind}}
\newcommand{\nocorrection}{\stepcounter{cor}}

\newcommand{\fiche}[1]{} \newcommand{\finfiche}{}
\newcommand{\titre}[1]{\centerline{\large \bf #1}}
\newcommand{\addcommand}[1]{}
\newcommand{\video}[1]{}

% Marge
\newcommand{\mymargin}[1]{\marginpar{{\small #1}}}



%----- Presentation ------
\setlength{\parindent}{0cm}

%\newcommand{\ExoSept}{\href{http://exo7.emath.fr}{\textbf{\textsf{Exo7}}}}

\definecolor{myred}{rgb}{0.93,0.26,0}
\definecolor{myorange}{rgb}{0.97,0.58,0}
\definecolor{myyellow}{rgb}{1,0.86,0}

\newcommand{\LogoExoSept}[1]{  % input : echelle
{\usefont{U}{cmss}{bx}{n}
\begin{tikzpicture}[scale=0.1*#1,transform shape]
  \fill[color=myorange] (0,0)--(4,0)--(4,-4)--(0,-4)--cycle;
  \fill[color=myred] (0,0)--(0,3)--(-3,3)--(-3,0)--cycle;
  \fill[color=myyellow] (4,0)--(7,4)--(3,7)--(0,3)--cycle;
  \node[scale=5] at (3.5,3.5) {Exo7};
\end{tikzpicture}}
}



\theoremstyle{definition}
%\newtheorem{proposition}{Proposition}
%\newtheorem{exemple}{Exemple}
%\newtheorem{theoreme}{Théorème}
\newtheorem{lemme}{Lemme}
\newtheorem{corollaire}{Corollaire}
%\newtheorem*{remarque*}{Remarque}
%\newtheorem*{miniexercice}{Mini-exercices}
%\newtheorem{definition}{Définition}




%definition d'un terme
\newcommand{\defi}[1]{{\color{myorange}\textbf{\emph{#1}}}}
\newcommand{\evidence}[1]{{\color{blue}\textbf{\emph{#1}}}}



 %----- Commandes divers ------

\newcommand{\codeinline}[1]{\texttt{#1}}

%%%%%%%%%%%%%%%%%%%%%%%%%%%%%%%%%%%%%%%%%%%%%%%%%%%%%%%%%%%%%
%%%%%%%%%%%%%%%%%%%%%%%%%%%%%%%%%%%%%%%%%%%%%%%%%%%%%%%%%%%%%



\begin{document}

\debuttexte


%%%%%%%%%%%%%%%%%%%%%%%%%%%%%%%%%%%%%%%%%%%%%%%%%%%%%%%%%%%
\diapo


Ce chapitre est l'aboutissement de toutes les notions d'algèbre linéaire 
vues jusqu'ici : espaces vectoriels, dimension, applications linéaires, matrices.
Nous allons voir que dans le cas des espaces vectoriels de dimension finie, 
l'étude des applications linéaires se ramène à l'étude des matrices, ce qui
facilite les calculs.

\change
Le plan est le suivant :

\change
La définition du rang d'une famille de vecteurs

\change
Puis du rang d'une matrice.

\change
Nous verrons quelles opérations conservent le rang

\change
Et on terminera par deux résultats théoriques :

le lien entre rang et matrice inversible

\change
Et on verra que le rang est aussi engendré par les vecteurs lignes


%%%%%%%%%%%%%%%%%%%%%%%%%%%%%%%%%%%%%%%%%%%%%%%%%%%%%%%%%%%
\diapo

\change
En termes simples le rang d'une famille de vecteurs est la dimension du plus petit sous-espace
vectoriel contenant tous ces vecteurs.



On part d'un espace vectoriel $E$  et d'une
famille finie de vecteurs $\{v_1, \ldots ,v_p\}$. 

\change
Le sous-espace vectoriel  engendré par $\{v_1, \ldots ,v_p\}$ étant de dimension
finie, on peut donc donner la définition suivante :

\change
Le \defi{rang} de la famille $\{v_1, \ldots ,v_p\}$ 
est la dimension du sous-espace vectoriel 
engendré par les vecteurs $v_1, \dots ,v_p$.

\change
Ce que je peux écrire aussi 
$\rg(v_1, \dots ,v_p) = \dim \Vect(v_1, \ldots ,v_p)$

\change
Calculer le rang d'une famille de vecteurs n'est pas toujours évident, cependant
il y a des inégalités qui découlent directement de la définition.

Tout d'abord le rang est inférieur ou égal au nombre d'éléments dans la famille.

Le rang vaut $0$ si et seulement si tous les vecteurs sont nuls.

Plus intéressant, le rang vaut $p$ si et seulement si la famille $\{v_1, \ldots ,v_p\}$ est libre.

\change
Enfin si $E$ est de dimension finie alors le rang est plus petit que la dimension de l'espace ambiant.


%%%%%%%%%%%%%%%%%%%%%%%%%%%%%%%%%%%%%%%%%%%%%%%%%%%%%%%%%%%
\diapo

On veut savoir quel est le rang de ces $3$ vecteurs qui sont des vecteurs de l'espace vectoriel $\Rr^4$.

\change
On commence par des remarques :

Ce sont des vecteurs de $\Rr^4$ donc le rang est $\le 4$.

\change
Mais comme il n'y a que $3$ vecteurs, alors on a mieux, le rang est $\le 3$.

\change
Ensuite comme par exemple le vecteur $v_1$ est non nul le rang n'est pas $0$.

\change
Le rang des $3$ vecteurs est toujours plus grand que le rang de $2$ de ces vecteurs.

Et comme il est clair que $v_1$ et $v_2$ ne sont pas des vecteurs colinéaires
alors le rang est plus grand que $2$.


Il reste donc à déterminer si le rang vaut $2$ ou $3$.


\change
Cela revient à savoir si  la famille $\{v_1,v_2,v_3\}$ est libre ou liée.

\change
Ce que l'on détermine en résolvant le système linéaire
$\lambda_1 v_1 + \lambda_2 v_2 + \lambda_3 v_3 = 0$. 

\change
Après un petit calcul on trouve une solution non triviale
$v_1-v_2+v_3=0$. La famille est donc liée.

\change
Ainsi $v_3$ est combinaison linéaire de $v_1$ et $v_2$ donc 
$\Vect(v_1,v_2,v_3)= \Vect(v_1,v_2)$, 

\change
et finalement $\rg (v_1,v_2,v_3) = \dim \Vect(v_1,v_2,v_3) = 2$.



%%%%%%%%%%%%%%%%%%%%%%%%%%%%%%%%%%%%%%%%%%%%%%%%%%%%%%%%%%%
\diapo

Une matrice peut être vue comme une juxtaposition de vecteurs colonnes.


On définit le rang d'une matrice comme étant le rang de ses vecteurs colonnes. 

\change
Voyons un exemple simple, calculons le rang de cette matrice.

\change

On considère que cette matrice représente $4$ vecteurs colonnes
et par définition le rang de la matrice est le rang
des $4$ vecteurs colonnes que voici.

\change
Mais sur cet exemple tous les vecteurs sont colinéaires à $v_1$, 

\change
ce qui signifie que le rang de la famille est exactement $1$ 

\change
et ainsi $\rg A = 1$.



%%%%%%%%%%%%%%%%%%%%%%%%%%%%%%%%%%%%%%%%%%%%%%%%%%%%%%%%%%%
\diapo

On dit qu'une matrice est échelonnée par rapport aux colonnes si
le nombre de zéros commençant une colonne croît strictement colonne après colonne,
jusqu'à ce qu'il ne reste plus que des zéros.\\

Ici les $*$ désignent des coefficients quelconques, les $+$ des coefficients non nuls :\\

Autrement dit, la matrice transposée est échelonnée par rapport aux lignes.

Le rang d'une matrice échelonnée est très simple à calculer.

Proposition : Le rang d'une matrice échelonnée par colonnes est égal au nombre
de colonnes non nulles.

Par exemple, dans cette matrice échelonnée, $4$ colonnes sur $6$ sont non nulles, donc le rang 
de cette matrice est $4$.\\

La démonstration de cette proposition consiste à remarquer que
les vecteurs colonnes non nuls sont linéairement indépendants, 
ce qui est clair au vu de la forme échelonnée de la matrice.



%%%%%%%%%%%%%%%%%%%%%%%%%%%%%%%%%%%%%%%%%%%%%%%%%%%%%%%%%%%
\diapo

Le rang d'une matrice ayant les colonnes $C_1, C_2, \ldots, C_p$
n'est pas modifié par les opérations élémentaires suivantes sur les 
vecteurs colonnes :

\change
$C_i \leftarrow \lambda C_i$ : 
  on peut multiplier une colonne par un scalaire non nul.

\change
$C_i \leftarrow C_i+\lambda C_j$ :
  on peut ajouter à la colonne $C_i$ un multiple d'une autre colonne $C_j$.

\change
  $C_i \leftrightarrow C_j$ : on peut échanger deux colonnes.

\change
Pour gagner du temps on peut même effectuer
l'opération $C_i \leftarrow C_i + \lambda_1 C_1 + \lambda_2 C_2 + \cdots + \lambda_p C_p$
qui est la répétition de plusieurs opérations de ce type [montrer 2.]


On a même un résultat plus fort, comme vous pouvez le voir dans la démonstration :
non seulement le rang est conservé par ces opérations élémentaires,
mais en plus l'espace vectoriel engendré par les vecteurs colonnes est conservé
par ces opérations.


%%%%%%%%%%%%%%%%%%%%%%%%%%%%%%%%%%%%%%%%%%%%%%%%%%%%%%%%%%%
\diapo

Voyons dans la pratique comment calculer le rang d'une matrice ou d'un système de vecteurs.

\change
Il s'agit d'appliquer la méthode de Gauss sur les colonnes de la matrice $A$
(considérée comme une juxtaposition de vecteurs colonnes).

\change
Le principe de la méthode de Gauss affirme que les $3$ opérations élémentaires
$C_i \leftarrow \lambda C_i$, 
$C_i \leftarrow C_i+\lambda C_j$,
$C_i \leftrightarrow C_j$

\change
permettent de transformer la matrice $A$ en une matrice échelonnée
par rapport aux colonnes.

\change
Le rang de la matrice est alors le nombre de colonnes non nulles.



%%%%%%%%%%%%%%%%%%%%%%%%%%%%%%%%%%%%%%%%%%%%%%%%%%%%%%%%%%%
\diapo

Voici une famille de 5 vecteurs de l'espace vectoriel $\Rr^4$.

Quel est son rang ?

\change
Calculer le rang de la famille de vecteurs c'est calculer le rang de cette matrice,
où l'on a juxtaposé les vecteurs colonnes.

Nous allons appliquer la méthode du pivot de Gauss sur les colonnes pour rendre cette matrice échelonnée.

On veut obtenir des zéros sur la première ligne à droite du premier pivot.

\change
Pour cela on effectue ces opérations élémentaires.

\change
Chacune de ces opérations fait apparaître un $0$.

Ce n'est bien sûr pas encore fini !


%%%%%%%%%%%%%%%%%%%%%%%%%%%%%%%%%%%%%%%%%%%%%%%%%%%%%%%%%%%
\diapo

Repartant de cette matrice, 

\change
\change
on échange les colonnes $C_2$ et $C_3$ 

pour avoir le coefficient $-1$ en position de pivot et ainsi 
éviter d'introduire des fractions. 


On veut maintenant des zéros à droite de ce deuxième pivot

\change
Ce que l'on obtient en faisant ces opérations 

\change
Enfin, en faisant les opérations $C_4\leftarrow C_4-C_3$
et $C_5\leftarrow C_5-2C_3$, on obtient une matrice échelonnée par colonnes.

\change
Il y a $3$ colonnes non nulles : on en déduit que le rang de 
la famille de vecteurs est $3$.

\change
On a même trouvé une base pour l'espace vectoriel engendré par 
$(v_1,v_2,v_3,v_4,v_5)$,
une telle base est formée par les $3$ vecteurs colonnes 
non nulles de la matrice échelonnée.


On s'arrête là pour les exemples mais vous devez en faire plein d'autres.


%%%%%%%%%%%%%%%%%%%%%%%%%%%%%%%%%%%%%%%%%%%%%%%%%%%%%%%%%%%
\diapo

Nous anticipons sur la suite, pour énoncer un résultat important :

Théorème : Une matrice carrée de taille *$n$* est inversible si et seulement si elle
est de rang exactement *$n$*.


La preuve repose sur plusieurs résultats qui seront vus au fil de ce chapitre.


%%%%%%%%%%%%%%%%%%%%%%%%%%%%%%%%%%%%%%%%%%%%%%%%%%%%%%%%%%%
\diapo

On termine par un autre résultat important à la fois pratique et théorique.

On a considéré jusqu'ici une matrice 
comme une juxtaposition de vecteurs colonnes 
et défini le rang comme la dimension de l'espace engendré par 
ses colonnes.

Considérons maintenant que $A$ est aussi une superposition 
de vecteurs lignes $(w_1,\ldots,w_n)$.

\change
Proposition : $\rg A = \dim \Vect (w_1,\ldots,w_n)$


\change
Autrement dit : \emph{l'espace vectoriel engendré par les vecteurs colonnes
et l'espace vectoriel engendré par les vecteurs lignes sont de même dimension.}

\change
Une formulation plus théorique est que 
$\rg A = \rg A^T$

\change
le rang d'une matrice est égal au rang de sa transposée.


Attention ! Les espaces vectoriels engendrés par les colonnes
et les lignes d'une matrice $A$ sont de même dimension mais ne sont pas égaux.


%%%%%%%%%%%%%%%%%%%%%%%%%%%%%%%%%%%%%%%%%%%%%%%%%%%%%%%%%%%
\diapo

On termine comme d'habitude par des exercices.

\end{document}
