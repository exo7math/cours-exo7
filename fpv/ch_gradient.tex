
\documentclass[12pt, class=report,crop=false]{standalone}
\usepackage[screen]{../exo7book}


\begin{document}

%====================================================================
\chapitre{Gradient -- Théorème des accroissements finis}
%====================================================================

% A garder
%\DeclareMathOperator{\grad}{grad}  % dans le préambule
\newcommand{\grad}{\operatorname{grad}} % dans le document


%A vérifier :
%
%[[pour les surface de niveau, etc. vérifier si différentiable ou C1]]
%
%[[notation G ou S pour le graphe d'une fonction]]
%
%[[Derivee directionnelle avec vecteur qcq (ou unitaire) ?]]

Le calcul différentiel s'applique au calcul des équations des tangentes aux courbes et des plans tangents aux surfaces.
Il permet aussi d'approcher les fonctions de plusieurs variables par des formules linéaires.


%%%%%%%%%%%%%%%%%%%%%%%%%%%%%%%%%%%%%%%%%%%%%%%%%%%%%
\section{Gradient}

Le gradient est un vecteur dont les coordonnées sont les dérivées partielles. 
Il est très important en physique et a des nombreuses applications géométriques, car il donne l'équation des tangences aux courbes et surfaces.

%----------------------------------------------------
\subsection{Définition}

\begin{definition}
Soit $f : \Rr^n \to \Rr$ une fonction admettant des dérivées partielles.
Le \defi{gradient} en $x = (x_1,\ldots,x_n) \in \Rr^n$, noté 
$\grad f (x)$, est le vecteur 
$$\grad f (x) =
\begin{pmatrix} \dfrac{\partial f}{\partial x_{1}} (x)\\ \vdots \\ \dfrac{\partial f}{\partial x_n}(x)\end{pmatrix}.$$
\end{definition}

Les physiciens notent souvent $\nabla f (x)$ pour $\grad f (x)$.
Le symbole $\nabla$ se lit \og{}nabla\fg{}.

\bigskip

Pour une fonctions $f(x,y)$ de deux variables, au point $(x_0,y_0)$, on a donc
$$\grad f(x_0,y_0) = \begin{pmatrix}  \dfrac{\partial f}{\partial x} (x_0,y_0) \\ \dfrac{\partial f}{\partial y}(x_0,y_0)\end{pmatrix}.$$


\begin{exemple}
\sauteligne
\begin{itemize}
\item $f(x,y) = x^2y^3$, $\grad f (x,y) =  \begin{pmatrix}2xy^3\\3x^2y^2\end{pmatrix}$,
au point $(x_0,y_0)=(2,1)$, $\grad f (2,1) =  \begin{pmatrix}4\\12\end{pmatrix}$.

\item $f(x,y,z) = x^2\sin(yz)$, $\grad f (x,y,z) = \begin{pmatrix} \big(2x\sin(yz) \\ x^2z \cos(yz) \\ x^2y\cos(yz) \end{pmatrix}$.

\item $f(x_1,\ldots,x_n)= x_1^2+x_2^2+\cdots + x_n^2$, $\grad f (x_1,\ldots,x_n) =  \begin{pmatrix}2x_1\\ \vdots \\2x_n\end{pmatrix}$.
\end{itemize}
\end{exemple}

\begin{remarque*}
Le gradient est un élément de $\Rr^n$ écrit comme un vecteur colonne. Parfois pour alléger l'écriture, on peut aussi l'écrire sous la forme d'un vecteur ligne.
\end{remarque*}


%----------------------------------------------------
\subsection{Lien avec la différentielle}


\evidence{Lien avec la différentielle.}

Le gradient est une autre écriture possible de la différentielle.
Si $f$ est différentiable en $x \in \Rr^n$, et $h \in \Rr^n$ alors :
\mybox{$\dd_f (x) (h) = \langle \grad f (x) \mid h \rangle.$}
Ce qui pour $f: \Rr^2 \to \Rr$ et $(x_0,y_0),(h,k) \in \Rr^2$ s'écrit :
$$\dd_f (x_0,y_0) (h,k) = \langle \grad f (x_0,y_0) \mid \left(\begin{smallmatrix}h\\k\end{smallmatrix}\right) \rangle
= h\frac{\partial f}{\partial x}(x_0,y_0)
+k\frac{\partial f}{\partial y}(x_0,y_0).$$

La différentielle $\dd_f(x)$ est une application linéaire de $\Rr^n \to \Rr$, $\grad f (x)$ est la transposée de sa matrice dans la base canonique.
On pourrait donc aussi écrire le produit de matrices $\dd_f(x)(h) = \grad f (x)^T \cdot h$.

\bigskip

\evidence{Lien avec la dérivée directionnelle.}

Si $f$ est différentiable :
$$D_v f(x) = \dd_f (x) (v) = \langle \grad f (x) \mid v \rangle.$$



%----------------------------------------------------
\subsection{Tangentes aux lignes de niveau}


Soit $f : \Rr^2 \to \Rr$ une fonction différentiable. On considère les lignes de niveau $f(x,y)=k$.


\begin{proposition}
Le vecteur gradient $\grad f(x_0,y_0)$ est orthogonal à la ligne de niveau de $f$ passant au point $(x_0,y_0)$. 
\end{proposition}


Sur ce premier dessin, vous avez \couleurnb{(en rouge)}{} la ligne de niveau passant par le point $(x_0,y_0)$. En ce point est dessiné \couleurnb{(en vert)}{} un vecteur tangent $v$ et la tangente à la ligne de niveau. 
Le vecteur gradient \couleurnb{(en bleu)}{} est orthogonal à la ligne de niveau en ce point.


\bigskip

\myfigure{1}{
  \tikzinput{fig-gradient-02}
}

\bigskip

En chaque point du plan part un vecteur gradient. Ce vecteur gradient est orthogonal à la ligne de niveau passant par ce point.

\myfigure{1}{
  \tikzinput{fig-gradient-01}
}




Précisons la notion de tangente :
\begin{itemize}
  \item On se place en un point $(x_0,y_0)$ où les deux dérivées partielles ne s'annulent pas en même temps, c'est-à-dire $\grad f(x_0,y_0)$ n'est pas le vecteur nul. Considérons $\mathcal{C}$, la ligne de niveau de $f$ qui passe par ce point $(x_0,y_0)$. Le théorème des fonctions implicites (que sera vu plus tard) montre qu'il est possible de trouver une paramétrisation de $\mathcal{C}$ au voisinage de $(x_0,y_0)$. Notons $\gamma : [-1,1] \to \Rr^2$, $t \mapsto \gamma(t)= (\gamma_1(t),\gamma_2(t))$ cette paramétrisation, en supposant que $\gamma(0) = (x_0,y_0)$. 


  \item La \defi{tangente} à la courbe $\mathcal{C}$ en $(x_0,y_0)$ est la droite passant par le point $(x_0,y_0)$ et de vecteur directeur, le vecteur dérivé $\gamma'(0) = (\gamma_1'(0),\gamma_2'(0))$.
  
  \item Un vecteur $v$ est \defi{orthogonal} (ou \defi{normal} si $v$ n'est pas nul) à la courbe $\mathcal{C}$ en $(x_0,y_0)$, s'il est orthogonal à la tangente en ce point, c'est-à-dire $\langle v \mid \gamma'(0)\rangle = 0$.
\end{itemize}

On peut maintenant prouver la proposition.
\begin{proof}
Notons $k = f(x_0,y_0)$, alors $\mathcal{C}$ est la ligne de niveau $f(x,y)=k$.
Dire que $\gamma(t)$ est une paramétrisation de $\mathcal{C}$ (autour de $(x_0,y_0)$), 
c'est exactement dire 
que 
$$\forall t \in [-1,1] \qquad f \big( \gamma(t) \big) = k$$
Comme $f \circ \gamma (t)$ est une fonction constante, alors sa dérivée est nulle.
La formule de la différentielle d'une composition s'écrit :
$$J_f (\gamma(t)) \times J_\gamma (t) = 0$$
et donc ici 
$$
\begin{pmatrix}
\frac{\partial f}{\partial x} (\gamma(t)) & 
\frac{\partial f}{\partial y} (\gamma(t)) \\
\end{pmatrix}
\times
\begin{pmatrix}
\gamma_1'(t)\\
\gamma_2'(t)
\end{pmatrix}
= 0$$
En $t=0$ on trouve exactement :
$$\langle \grad f(x_0,y_0) \mid \gamma'(0) \rangle = 0$$
ce qui signifie que le gradient est orthogonal au vecteur tangent.
\end{proof}


Dans la pratique, c'est l'équation de la tangente qui nous intéresse :

\begin{proposition}
L'équation de la tangente à la ligne de niveau de $f$ en $(x_0,y_0)$ est 
\mybox{$\displaystyle
\frac{\partial f}{\partial x}(x_0,y_0)(x-x_0)+\frac{\partial f}{\partial y}(x_0,y_0)(y-y_0)=0
$}
pourvu que le gradient de $f$ en ce point ne soit pas le vecteur nul.
\end{proposition}

\begin{proof}
C'est l'équation de la droite dont un vecteur normal est $\grad f(x_0,y_0)$ et qui passe par $(x_0,y_0)$.
\end{proof}


\begin{exemple}[Tangente à une ellipse]
Trouver les tangentes à l'ellipse $\mathcal{E}$ d'équation $\frac{x^2}{a^2}+\frac{y^2}{b^2} = 1$.

\myfigure{0.8}{
  \tikzinput{fig-gradient-03}
}


\begin{itemize}
  \item \textbf{Par les lignes de niveau.} 
  
  Cette ellipse $\mathcal{E}$ est la ligne de niveau $f(x,y)=1$ de la fonction
  $f(x,y) = \frac{x^2}{a^2}+\frac{y^2}{b^2}$. 
  Les dérivées partielles sont
  $$\frac{\partial f}{\partial x}(x_0,y_0) = \frac{2x_0}{a^2} \qquad
\frac{\partial f}{\partial y}(x_0,y_0) = \frac{2y_0}{b^2}$$
  Donc l'équation de la tangente à l'ellipse $\mathcal{E}$ en un de ces points $(x_0,y_0)$ est
  $$\frac{2x_0}{a^2}(x-x_0)+\frac{2y_0}{b^2}(y-y_0)=0 $$
  Mais comme $\frac{x_0^2}{a^2}+\frac{y_0^2}{b^2} = 1$ alors l'équation de la tangente se simplifie en $\displaystyle \frac{x_0}{a^2}x + \frac{y_0}{b^2} y = 1$.
  
  \item \textbf{Par une paramétrisation.}
  
  Une autre approche est de paramétrer l'ellipse $\mathcal{E}$ par 
  $\gamma(t) = (a \cos t, b \sin t)$, $t \in [0,2\pi[$.
  Le vecteur dérivé étant $\gamma'(t) = (-a \sin t, b \cos t)$, la tangente
  en $\gamma(t)$ est dirigée par le vecteur $(-b \cos t,-a \sin t)$.
  L'équation de la tangente en $\gamma(t)$ est donc :
  $$-b \cos t (x - a \cos t) -a\sin t (y - b\sin t) = 0$$
  (en posant $x_0 = a \cos t$ et $y_0 = b \sin t$, on retrouve l'équation ci-dessus).

\end{itemize}
\end{exemple}

\begin{exemple}
Soit $f(x,y) = x^3-y^2-x$.
Nous allons calculer l'équation des tangentes aux courbes de niveau de $f$.

\begin{itemize}
  \item \textbf{Calcul du gradient.}  
  $\grad f(x,y) = \begin{pmatrix} 3x^2-1 \\ -2y \end{pmatrix}$.
  
  \item \textbf{Points où le gradient s'annule.} 
  $P_1 = \left( -\frac{\sqrt 3}{3},0 \right)$, $P_2 =\left( +\frac{\sqrt 3}{3},0 \right)$
  
  On calcule $f(P_1) = +\frac{2\sqrt3}{9}$, $f(P_2) = -\frac{2\sqrt3}{9}$. Ainsi $P_1$ est sur la ligne de niveau $f=-\frac{2\sqrt3}{9}$ et $P_2$ sur celle $f=+\frac{2\sqrt3}{9}$.
   
  \item \textbf{\'Equation de la tangente.}  
  En dehors de ces deux points, les courbes de niveau ont une tangente.
  Au point $(x_0,y_0)$ l'équation est 
  $$(3x_0^2-1)(x-x_0) -2y_0(y-x_0) = 0.$$
\end{itemize}  


Voici quelques lignes de niveau de $f$.
Le point $P_1$ est le point isolé \couleurnb{du niveau rouge}{}, il n'y a pas de tangente en ce point. Le point $P_2$ est le point double \couleurnb{du niveau vert}{}, il n'y a pas de tangente en ce point (en fait on pourrait dire qu'il y a deux tangentes).




\begin{center}
  \includegraphics{figures/fig-gradient-04}
\end{center}

\end{exemple}






%----------------------------------------------------
\subsection{Lignes de plus forte pente}
 
Considérons les lignes de niveau $f(x,y)=k$ d'une fonction $f : \Rr^2 \to \Rr$.
On se place en un point $(x_0,y_0)$. On cherche dans quelle direction se déplacer pour augmenter le plus vite la valeur de $f$.
 
\begin{proposition}
Le vecteur gradient $\grad f(x_0,y_0)$ indique la direction de plus forte pente à partir du point $(x_0,y_0)$.
\end{proposition}


Autrement dit, si l'on veut passer le plus vite
possible du niveau $a$ au niveau $b>a$, à partir d'un point donné $(x_0,y_0)$ de niveau
$f(x_0,y_0)=a$, alors il faut démarrer en suivant la direction du gradient $\grad f(x_0,y_0)$. 

\myfigure{1}{
  \tikzinput{fig-gradient-05}
}

Comme illustration, un skieur voulant aller vite choisit 
la plus forte pente descendante en un point de la montagne, c'est la direction inverse du gradient. 
 
\begin{proof}
La dérivée suivant le vecteur non nul $v$ au point $(x_0,y_0)$ décrit la variation de $f$ autour de ce point lorsqu'on se déplace dans la direction $v$. 
La direction selon laquelle la croissance est la plus forte est celle du gradient de $f$. En effet,
$$D_{v}f(x_0,y_0)=\langle \grad f(x_0,y_0) \mid v\rangle=
\| \grad f(x_0,y_0) \| \cdot \| v \| \cdot \cos \theta$$
où $\theta$ est l'angle entre le vecteur $\grad f(x_0,y_0)$ et le vecteur $v$.
Le maximum est atteint lorsque l'angle $\theta=0$. C'est à dire lorsque $v$ pointe dans la même direction que $\grad f(x_0,y_0)$.
\end{proof}


%----------------------------------------------------
\subsection{Surface de niveau}

On a des résultats similaires pour les surfaces de niveau $f(x,y,z)=k$, d'une fonction différentiable.

Rappelons que le plan de $\Rr^3$ passant par $(x_0,y_0,z_0)$ et de vecteur normal 
$n=(a,b,c)$ a pour équation cartésienne :
$a(x-x_0)+b(y-y_0)+c(z-z_0)=0$.


De même qu'il existe une droite tangente au ligne de niveau, il existe un \defi{plan tangent} aux surfaces de niveau.


\begin{proposition}
Le vecteur gradient $\grad f(x_0,y_0,z_0)$ est orthogonal à la ligne de niveau de $f$ passant au point $(x_0,y_0,z_0)$. Autrement dit :
l'équation du plan tangent à la surface de niveau de $f$ en $(x_0,y_0,z_0)$ est 
\mybox{$\displaystyle
\frac{\partial f}{\partial x}(x_0,y_0,z_0)(x-x_0)+\frac{\partial f}{\partial y}(x_0,y_0,z_0)(y-y_0)
+\frac{\partial f}{\partial z}(x_0,y_0,z_0)(z-z_0)=0
$}
pourvu que le gradient de $f$ en ce point ne soit pas le vecteur nul.
\end{proposition}


\myfigure{1}{
  \tikzinput{fig-gradient-07}
}


Plus généralement pour $f : \Rr^n \to \Rr$, $\grad f (x)$ est orthogonal à l'espace tangent à
l'hypersurface de niveau $f=k$ passant par le point $x\in\Rr^n$.


\begin{exemple}
Pour quelles valeurs de $k$ la surface de niveau $x^2+y^2-z^2=k$ admet-elle un plan tangent horizontal (c'est-à-dire parallèle au plan $(z=0)$) ?

\emph{Solution.}
On pose $f(x,y,z) = x^2+y^2-z^2$.
\begin{itemize}
  \item \textbf{Calcul du gradient.}   
  $\grad f (x,y,z) = \begin{pmatrix}2x\\2y\\2z\end{pmatrix}$.
  
  \item \textbf{Gradient nul.} Le gradient est le vecteur nul uniquement au point $(0,0,0)$, donc au niveau $k=0$. En ce point il n'y a pas de plan tangent.
  
  \item \textbf{Plan tangent horizontal.}  
  Le plan tangent est horizontal exactement lorsque le gradient est un vecteur colinéaire à $\left(\begin{smallmatrix}0\\0\\1\end{smallmatrix}\right)$ (et n'est pas le vecteur nul).
  Il faut donc $\frac{\partial f}{\partial x} = 0$ et $\frac{\partial f}{\partial y} = 0$, ce qui implique ici $x=0$ et $y=0$.
  
  \item \textbf{Analyse.}
  En un tel point $(0,0,z)$ on a $f(x,y,z)=-z^2$. Donc le niveau $k$ est strictement négatif.
  
  \item \textbf{Synthèse.} Réciproquement, étant donné $k<0$. Alors aux points $(0,0,\pm\sqrt{|k|})$, le vecteur gradient est vertical, donc le plan tangent est horizontal.
  
  \item \textbf{Conclusion.}
  \begin{itemize} 
    \item Pour $k>0$, il n'y a pas de plan tangent horizontal. La surface $f=k$ est un \defi{hyperboloïde à une nappe}. 
    
    \item Pour $k=0$, il n'y a pas de plan tangent horizontal. Le point $(0,0,0)$ est singulier. La surface $f=0$ est un \defi{cone}. 
    
    \item Pour $k<0$, il y a deux points ayant un plan tangent horizontal. La surface $f=k$ est un \defi{hyperboloïde à deux nappes}.   
    
    \end{itemize}
    
\end{itemize}  

De gauche à droite : l'hyperboloïde à une nappe ; le cone, l'hyperboloïde à deux nappes.

\begin{center}
  \includegraphics[scale=0.25]{figures/fig-gradient-06d}
  \includegraphics[scale=0.23]{figures/fig-gradient-06c}  
  \includegraphics[scale=0.25]{figures/fig-gradient-06b}
\end{center}
Les trois surfaces ensembles, comme des surfaces de niveau de $f$ (avec une découpe pour voir l'intérieur). 
\begin{center}  
  \includegraphics[scale=0.4]{figures/fig-gradient-06a}   
\end{center}

  
\end{exemple}

%--------------------------------------------------
\subsection{Plan tangent au graphe de fonctions}

Soit $f : U \subset \Rr^2 \to \Rr$ différentiable.
On s'intéresse maintenant au graphe de $f$. Rappelons que c'est la surface
$$\mathcal{G}_f = \big\{ (x,y,z) \in \Rr^3 \mid (x,y)\in U \text{ et } z = f(x,y) \big\}.$$

Attention ! Il ne faut pas confondre le graphe d'une fonction $f : \Rr^2 \to \Rr$
avec les surfaces de niveau des fonctions $f : \Rr^3 \to \Rr$.


\begin{proposition}
Soit $f : U \subset \Rr^2 \to \Rr$ différentiable. Soit $(x_0,y_0) \in U$ et 
soit $M_0=(x_0,y_0,f(x_0,y_0))$ le point du graphe $\mathcal{G}_f$ de $f$.
Le plan tangent au graphe $\mathcal{G}_f$ en $M_0$ a pour équation :
\mybox{$\displaystyle 
z = f(x_0,y_0)+\frac{\partial f}{\partial x}(x_0,y_0)(x-x_0)
+\frac{\partial f}{\partial y}(x_0,y_0)(y-y_0).$}
\end{proposition}


\myfigure{1}{
  \tikzinput{fig-gradient-08}
}



\begin{proof}
On introduit la fonction $F$ définie par
$F(x,y,z)=z-f(x,y)$, pour tout $(x,y,z)\in U\times \Rr$.
Le graphe de $f$ est la surface $\mathcal{G}_f=\{(x,y,z)\in U\times \Rr\mid F(x,y,z)=0\}$.
Ainsi $\mathcal{G}_f$ est aussi une surface de niveau de $F$.
On calcule que 
$$\grad F (x,y,z)=\left(-\frac{\partial f}{\partial x}(x,y),-\frac{\partial f}{\partial x}(x,y),1\right)$$
Notez que ce vecteur n'est jamais nul et donc une équation du plan tangent en $(x_0,y_0,z_0)$ est :
$$\frac{\partial f}{\partial x}(x_0,y_0)(x-x_0)
+\frac{\partial f}{\partial y}(x_0,y_0)(y-y_0) - (z-z_0) = 0$$
où on a noté $z_0 = f(x_0,y_0)$.
\end{proof}


\begin{exemple}
Soit $f(x,y) = 3x^2-2y^3$. 

\begin{enumerate}
  \item Trouver l'équation du plan tangent au graphe de $f$ au-dessus de $(x_0,y_0)$.
  
  On a $\frac{\partial f}{\partial x}(x,y) = 6x$, $\frac{\partial f}{\partial y}(x,y) = 6y^2$. Donc l'équation du plan tangent au point $(x_0,y_0,f(x_0,y_0))$ est :
  $$z = 3x_0^2-2y_0^3 + 6x_0(x-x_0)+6y_0^2(y-y_0)$$
  ou encore :
  $$6x_0x+6y_0^2y-z = 3x_0^2+8y_0^3.$$
  
  \item Trouver l'équation du plan $\mathcal{P}_0$ tangent au-dessus du point $(x_0,y_0)=(1,2)$.
  
  On pose $z_0 = f(x_0,y_0) =  -5$. L'équation est alors
  $z = -5 + 6(x-1)+24(y-2)$, autrement dit $6x+24y-z=59$.
    
  \item Trouver les points pour lesquels le plan tangent est parallèle à $\mathcal{P}_0$.
  
  On cherche un point $(x_1,y_1)$ qui vérifie
  $(6,24,-1) = (6x_1,6y_1^2,-1)$  et $y_1 \neq y_0$. On trouve donc le seul point $(x_1,y_1)=(1,-2)$. 
  
\end{enumerate}

\end{exemple}

%----------------------------------------------------
\begin{miniexercices}
\sauteligne
\begin{enumerate}
  \item Calculer le gradient en tout point de la fonctions définie par $f(x,y) = xe^y$. Même question pour $f(x,y,z) = x^2y^3z^4$, puis $f(x_1,\ldots,x_n)= \sqrt{x_1^2+\cdots+x_n^2}$. 
  
  \item  Calculer le gradient de $f(x,y) = \ln(x+y^2)$ en tout point $(x_0,y_0)$.
  Exprimer la différentielle $d_f(x_0,y_0)(h,k)$. Calculer la dérivée directionnelle de $f$ en $(x_0,y_0)=(1,2)$ le long du vecteur $(2,3)$.
  
  \item Soient $f,g : \Rr^n \to \Rr$.
  Montrer que $\grad (f+g) (x) = \grad f(x) + \grad g(x)$. Que vaut $\grad (\lambda \cdot f) (x)$ (où $\lambda \in \Rr$) ?
  
  \item Soient $f : \Rr^n \to \Rr$, $x,y \in \Rr^n$, $\lambda \in \Rr$.  
  Est-ce que $\grad f (x+y) = \grad f (x) +\grad f (y)$ ? Est-ce que
  $\grad f (\lambda \cdot x) = \lambda \cdot \grad f (x)$ ?
     
  \item Soit l'hyperbole $\mathcal{H}$ d'équation $x^2-y^2=1$. Dessiner $\mathcal{H}$. Calculer l'équation cartésienne de la tangente en un point $(x_0,y_0) \in \mathcal{H}$. Trouver les points de $\mathcal{H}$ où la tangente est colinéaire au vecteur $(2,1)$.
   
  \item Trouver les points de la surface d'équation $x^2-y^2z=0$ où le gradient s'annule. Calculer l'équation du plan tangent en dehors de ces points.
  
  \item Soit $f(x,y) = \ln(x+y^2)$. Trouver l'équation du plant tangent au graphe de $f$ au-dessus du point $(-2,3)$.
  
   
  
\end{enumerate}
\end{miniexercices}



%%%%%%%%%%%%%%%%%%%%%%%%%%%%%%%%%%%%%%%%%%%%%%%%%%%%%
\section{Calcul d'incertitudes}

Pour les fonctions d'une variable, la dérivée permet de calculer un développement limité à l'ordre $1$ et donc d'approcher une fonction autour d'un point par une formule linéaire. Pour les fonctions de plusieurs variables, nous avons besoin des dérivées partielles pour obtenir une formule linéaire.

%----------------------------------------------------
\subsection{Calcul approché}

Motivation : comment estimer la valeur $\sqrt{1,01}$ sans calculatrice ?
On pose $f(x)=\sqrt{x}$, le développement limité en $1$ s'écrit alors
$f(1+h) \simeq f(1) + hf'(1)$. Autrement dit :
$$\sqrt{1+h} \simeq 1 + \frac{h}{2}.$$
Ce qui donne l'estimation  $\sqrt{1,01} \simeq 1,005$ (au lieu de 
$\sqrt{1,01} = 1,0498\ldots$).

Géométriquement, la tangente au graphe de $f$ en $1$ donne une bonne approximation des valeurs de $f$ autour de ce point.

\myfigure{1}{
  \tikzinput{fig-gradient-09}
}
\bigskip

Nous allons voir l'analogue pour les fonctions de deux variables.

Si $f$ est différentiable au point $A = (x_0,y_0)$, alors
$$f(x_0+h,y_0+k) = f(x_0,y_0) + \dd_f(x_0,y_0) (h,k) + o(\sqrt{h^2+k^2})$$
On propose alors comme approximation de $f(x_0+h,y_0+k)$, la quantité 
$f(x_0,y_0) + \dd_f(x_0,y_0) (h,k)$, appelée \defi{approximation linéaire}, c'est-à-dire
\mybox{$\displaystyle
f(x_0+h,y_0+k) \simeq f(x_0,y_0) + h\frac{\partial f}{\partial x}(x_0,y_0)
+k\frac{\partial f}{\partial y}(x_0,y_0)
$}
Approximation qui est valable pour $h$ et $k$ petits.

L'interprétation géométrique est la suivante : 
on approche le graphe de $f$ en $(x_0,y_0)$ par le plan tangent au graphe en ce point. Sur la figure ci-dessous sont représentés : le graphe de $f$\couleurnb{ (en rouge)}{}, le plan tangent au-dessus du point $(x_0,y_0)$\couleurnb{ (en bleu)}{}. La valeur $z_1 = f(x_0+h,y_0+k)$ est la valeur exacte donnée par le point de la surface au dessus de $(x_0+h,y_0+k)$. On approche cette valeur par la valeur $z_2 = f(x_0,y_0) + h\frac{\partial f}{\partial x}(x_0,y_0)
+k\frac{\partial f}{\partial y}(x_0,y_0)$ donnée par le point du plan tangent au dessus de $(x_0+h,y_0+k)$. 


\myfigure{1}{
  \tikzinput{fig-gradient-10}
}


\begin{exemple}
Valeur approchée de $f(1,002 ; 0,997)$ si $f(x,y) = x^2y$.
\bigskip

\emph{Solution.}
Ici $(x_0,y_0) = (1,1)$, $h = 2 \times 10^{-3}$, $k = -3 \times 10^{-3}$,
$\frac{\partial f}{\partial x} = 2xy$, $\frac{\partial f}{\partial y} = x^2$, donc $\frac{\partial f}{\partial x}(x_0,y_0) = 2$, $\frac{\partial f}{\partial x}(x_0,y_0) = 1$. Ainsi
$$f(1+h,1+k) \simeq f(1,1) + 2h + k$$
donc 
$$f(1,002 ; 0,997) \simeq 1 + 2 \times 2 \times 10^{-3} - 3 \times 10^{-3} \simeq 1,001$$
Avec une calculatrice on trouve $f(1,002 ; 0,997) = 1,000992$ : l'approximation est bonne.
\end{exemple}

\begin{exemple}
Deux résistances $R_1$ et $R_2$ sont connectées en parallèle. La résistance totale $R$ du circuit est donnée par la formule 
$$\frac{1}{R} = \frac{1}{R_1} + \frac{1}{R_2}.$$

La résistance $R_1$ vaut environ $1$ ; $R_2$ vaut environ $2$ (en kilo-ohms).
\'Ecrire l'approximation linéaire correspondante, puis donner la valeur approchée de $R$ lorsque $R_1 = 1,01$ et $R_2 = 1,98$.

\bigskip

\emph{Solution.}
Notons $f(x,y) = \frac{1}{\frac{1}{x} + \frac{1}{y}} = \frac{xy}{x+y}$, de sorte que $R = f(R_1,R_2)$.
Par exemple si $R_1 = 1$ et $R_2 = 2$ on trouve $R = 0,6666\ldots$


On calcule 
$$\frac{\partial f}{\partial x}(x,y) = \frac{y^2}{(x+y)^2} \qquad
\frac{\partial f}{\partial y}(x,y) = \frac{x^2}{(x+y)^2}.$$

Posons $(x_0,y_0)=(1,2)$. On a $f(x_0,y_0) = \frac 23$,
$\frac{\partial f}{\partial x}(x_0,y_0) = \frac 49$,
$\frac{\partial f}{\partial y}(x_0,y_0) = \frac 19$.

L'approximation linéaire de $f$ au voisinage de $(x_0,y_0)$ s'écrit
$$f(1+h,2+k) \simeq \frac 23 + \frac49 h + \frac19 k$$
Avec $h = 0,01$ et $k = -0,02$, on obtient
$f(1,01 ; 1,98) \simeq 0,6688$.
\end{exemple}


%----------------------------------------------------
\subsection{Calcul d'incertitudes}


Soit $f(x,y)$ une grandeur qui dépend de deux mesures $x$ et $y$.
La valeur de $x$ est proche d'une valeur fixe $x_0$, mais n'est connue qu'à une incertitude près, c'est-à-dire $x \in [x_0-\Delta x,x_0 + \Delta x]$ où $\Delta x$ est un nombre réel positif, appelé l'\defi{incertitude} sur $x$.
De même $y \in [y_0-\Delta y,y_0 + \Delta y]$.

Quelle va être l'erreur commise en approchant $f(x,y)$ par $f(x_0,y_0)$ ?



On note $(x,y) = (x_0+h,y_0+k)$.
\begin{align*}
  &\big| f(x_0+h,y_0+k) - f(x_0,y_0) \big| \\
  & \simeq \left| h\frac{\partial f}{\partial x}(x_0,y_0)
+k\frac{\partial f}{\partial y}(x_0,y_0) \right| \\
  & \le \Delta x \left|\frac{\partial f}{\partial x}(x_0,y_0) \right| 
+ \Delta y \left|\frac{\partial f}{\partial y}(x_0,y_0) \right| 
\end{align*}


On obtient ainsi une majoration de l'\defi{incertitude estimée} $\delta f$ :
$$\delta f \le \Delta x \left|\frac{\partial f}{\partial x}(x_0,y_0) \right| 
+ \Delta y \left|\frac{\partial f}{\partial y}(x_0,y_0) \right|$$


\begin{exemple}
Une usine produit des cylindres de rayon $r = 2 \pm 0,1$ et de hauteur $h = 10 \pm 0,2$. Quelle est l'incertitude estimée sur le volume du cylindre ?

\bigskip

\emph{Solution.}
Le volume est donné par la formule $V = \pi r^2 h$.
On pose $r_0 = 2$, $\Delta r = 0,1$, $h_0 = 10$, $\Delta h = 0,2$.
Ainsi l'incertitude estimée vérifie
$$\delta V \le \Delta r\left|\frac{\partial V}{\partial r}(r_0,h_0) \right| + \Delta h\left|\frac{\partial V}{\partial h} (r_0,h_0)\right| $$
On a 
$\frac{\partial V}{\partial r} = 2\pi r h$ 
et $\frac{\partial V}{\partial h} = \pi r^2$.
Donc ici :
$\delta V \le 0,1 \times 40\pi + 0,2 \times 4 \pi \simeq 15$.   
Le volume sans erreur est $V_0 = V(r_0,h_0) = 40 \pi \simeq 125$.
Au vu de notre estimation de l'incertitude on écrit 
$$V(r,h) = 125 \pm 15.$$

\end{exemple}

\begin{remarque*}
Sur cet exemple, $r$ est ici connu avec incertitude relative 
$\frac{\Delta r}{r_0} = \frac{0,1}{2} = 5\%$, et pour 
$h$ l'incertitude relative est $\frac{\Delta h}{h_0} = \frac{0,2}{10} = 2\%$.
L'estimation de l'incertitude relative du volume est 
$\frac{\delta V}{V_0} \simeq  \frac{15}{125} \simeq 12 \%$.
L'erreur relative du volume est donc bien plus élevée que les erreurs relatives sur $r$ et $h$.
\end{remarque*}

\begin{remarque*}
La formule de l'incertitude est seulement une estimation.
Pour une majoration exacte de l'erreur, il faut utiliser le théorème ou l'inégalité des accroissements finis.
\end{remarque*}


%----------------------------------------------------
\begin{miniexercices}
\sauteligne
\begin{enumerate}

  \item Donner l'approximation linéaire de $y^2e^x$ en $x_0=1$ et $y_0=2$. Sans calculatrice, en déduire une valeur approchée de $(1,99)^2 e^{1,03}$. Faire le même travail pour $\frac{\ln(9,99 \times 2,02)}{2,02}$.
  
  \item La tension $U$, la résistance $R$ et l'intensité $I$ sont reliées par la loi $U = RI$. \'Ecrire l'approximation linéaire de la résistance, pour $U_0 = 120$, $I_0 = 1$. Estimer la résistance lorsque $U = 118$ et $I = 0,9$.
  Estimer l'erreur commise lorsque l'on approche la résistance $R$ par $R_0 = U_0/I_0$ pour $U \in [118,122]$ et $I \in [0,9 ; 1,1]$.
  
  \item Soit $f(x,y) = \sqrt{x^2 + y^2}$. \'Ecrire l'approximation linéaire  pour $f(x,y)$ autour de $(x_0,y_0) = (4,3)$.
  Pour $x$ connu avec une incertitude relative de $5\%$ autour de $x_0$,
  et $y$ connu avec une incertitude relative de $10\%$ autour de $y_0$, estimer l'erreur relative commise l'on approche $f(x,y)$ par $f(x_0,y_0)$.
    
\end{enumerate}
\end{miniexercices}


%%%%%%%%%%%%%%%%%%%%%%%%%%%%%%%%%%%%%%%%%%%%%%%%%%%%%
\section{Théorème des accroissements finis}

Le théorème des accroissements finis est une façon exacte de mesurer l'écart entre deux valeurs de $f$. On peut aussi en tirer des inégalités. Cette section est beaucoup plus théorique que le reste du chapitre et peut être passée lors d'une première lecture.


%----------------------------------------------------
\subsection{Une variable (rappel)}

\begin{theoreme}[Théorème des accroissements finis (une variable)]
Soit $f:[a,b] \rightarrow \Rr$ continue sur $[a,b]$, dérivable sur $]a,b[$,  o\`u $a<b$. Il existe $c \in ]a,b[$ tel que
$$f(b)-f(a)=(b-a)f'(c).$$
\end{theoreme}
 
On renvoie au cours de première année pour la preuve.


%----------------------------------------------------
\subsection{Théorème des accroissements finis}


\begin{theoreme}[Théorème des accroissements finis (deux variables)]
Soit $f: U \to \Rr$ une fonction de classe $\mathcal{C}^1$ sur l'ouvert $U \subset \Rr^2$. Soient $a,b$ deux points de $U$. Si le segment $[a,b]$ est inclus dans $U$, alors il existe 
$c \in ]a,b[$ tel que
$$f(b)-f(a) = \left\langle \grad f (c) \mid b-a \right\rangle.$$
\end{theoreme}


\'Enoncé ainsi, le théorème est valable aussi pour $U \subset \Rr^n$.

Pour une reformulation, on note
$$a = (x_0,y_0), \quad b=(x_0+h,y_0+k), \quad c = a+\theta b = (x_0+\theta h,y_0+\theta k) \in ]a,b[$$
et
$$\grad f (c) = 
\begin{pmatrix}
\dfrac{\partial f}{\partial x}(c) \\
\dfrac{\partial f}{\partial y}(c) \end{pmatrix} 
\qquad
b-a = \begin{pmatrix} h \\ k \end{pmatrix} $$

\myfigure{0.7}{
  \tikzinput{fig-gradient-11}
}

\begin{theoreme}[Théorème des accroissements finis (deux variables)]
Soit $f: U \to \Rr$ une fonction de classe $\mathcal{C}^1$ sur l'ouvert $U \subset \Rr^2$. Soit $(x_0,y_0) \in U$. 
Pour tout $(h,k) \in \Rr^2$ tel que $(x_0+h,y_0+k)\in U$
et tel que $(x_0+th,y_0+tk)\in U$ pour tout $t\in[0,1]$,
 alors, il existe $\theta \in ]0,1[$ tel que
$$f(x_0+h,y_0+k) - f(x_0,y_0) = 
h\frac{\partial f}{\partial x}(x_0+\theta h,y_0+\theta k)
+k\frac{\partial f}{\partial y}(x_0+\theta h,y_0+\theta k).$$
\end{theoreme}


\begin{exemple}
Soit $f(x,y) = e^{x-y^2}$, $a = (0,0)$, $b=(2,1)$.
\begin{itemize} 
  \item Le théorème des accroissements finis nous dit qu'il existe $c \in ]a,b[$ tel que 
$$f(b)-f(a) = \left\langle \grad f (c) \mid b-a \right\rangle.$$

  \item On a   $\frac{\partial f}{\partial x} = e^{x-y^2}$, 
$\frac{\partial f}{\partial x} = -2ye^{x-y^2}$.
  Donc le théorème des accroissements finis affirme qu'il existe $\theta \in ]0,1[$, avec $c = (2\theta,\theta)$ tel que
  $$f(2,1)-f(0,0) = \left\langle (e^{2\theta-\theta^2},-2\theta e^{2\theta-\theta^2}) \mid (2,1) \right\rangle$$
Donc, il existe  $\theta \in ]0,1[$ tel que
 $$f(2,1)-f(0,0) = 2(1-\theta)e^{2\theta-\theta^2}.$$
  
Comme $f(2,1) = e$ et $f(0,0)=1$. On en déduit qu'il existe $\theta$ tel que $e-1= 2(1-\theta)e^{2\theta-\theta^2}$.
  
\end{itemize}


\end{exemple}

\begin{proof}
Soit $g : [0,1] \to \Rr$ la fonction définie par 
$$g(t) = f(x_0+th,y_0+tk).$$
Alors $g = f \circ \gamma$ où $\gamma : [0,1] \to \Rr^2$ est définie par $\gamma(t) = (x_0+th,y_0+tk)$.
La formule de différentiabilité d'une composition s'écrit 
$$J_g(t) = J_f (\gamma(t)) \times J_\gamma(t)$$
donc ici :
$$g'(t) = \left(\frac{\partial f}{\partial x}(\gamma(t)),
\frac{\partial f}{\partial y}(\gamma(t)) \right) \times 
\begin{pmatrix}
\gamma_1'(t)\\
\gamma_2'(t)
\end{pmatrix}
= \left(\frac{\partial f}{\partial x}(\gamma(t)),
\frac{\partial f}{\partial y}(\gamma(t)) \right) \times 
\begin{pmatrix}
h\\
k 
\end{pmatrix}
= h\frac{\partial f}{\partial x}(\gamma(t)) + k  \frac{\partial f}{\partial y}(\gamma(t))$$
Par le théorème des accroissements finis en une variable appliqué à la fonction $g : [0,1] \to \Rr$, il existe $\theta \in ]0,1[$ tel que $g(1)-g(0) = g'(\theta)(1-0)$, c'est-à-dire :
$$f(x_0+h,y_0+k) - f(x_0,y_0) = h\frac{\partial f}{\partial x}(x_0+\theta h,y_0+\theta k)
+k\frac{\partial f}{\partial y}(x_0+\theta h,y_0+\theta k),$$ 
autrement dit
$$f(b)-f(a) = \left\langle \grad f (c) \mid b-a \right\rangle.$$
\end{proof}


\bigskip


Nous allons voir quelques applications du théorème des accroissements finis.


%----------------------------------------------------
\subsection{Inégalités des accroissements finis}

\begin{corollaire}[Inégalité des accroissements finis]
Soit $f: U \to \Rr$ une fonction de classe $\mathcal{C}^1$ sur un ouvert \evidence{convexe} $U \subset \Rr^2$. 
On suppose qu'il existe $k>0$ tel que
$$\forall c \in U \qquad \| \grad f (c) \| \le k$$
alors
$$\forall a,b \in U \qquad  \left| f(b)-f(a)  \right| \le k   \| b -a \|$$
\end{corollaire}
L'hypothèse convexe permet de s'assurer que tous les points du segment $[a,b]$ appartiennent à $U$. Le corollaire est une conséquence immédiate du théorème des accroissements finis, à l'aide de l'inégalité de Cauchy-Schwarz.

\myfigure{0.6}{
  \tikzinput{fig-gradient-12}
}
\begin{exemple}
Soit $f(x,y) = \sin \left( \frac{x+\pi}{y+1} \right)$. 
Nous allons utiliser l'inégalité des accroissements finis pour majorer
$f(x,y)$ sur $[-\frac\pi2,+\frac\pi2] \times [-\frac12,+\frac12]$.

\begin{itemize}
  \item Soit $a=(0,0)$, alors $f(a) = f(0,0) = 0$.
  
  \item Soit $b=(x,y)$ avec $x \in [-\frac\pi2,+\frac\pi2]$ et $y \in [-\frac12,+\frac12]$.
  
  
  \item Calculons les dérivées partielles :
  $$\frac{\partial f}{\partial x}(x,y) =  \frac{1}{y+1}\cos \left( \frac{x+\pi}{y+1} \right) \qquad
\frac{\partial f}{\partial y}(x,y) = - \frac{x+\pi}{(y+1)^2} \cos \left( \frac{x+\pi}{y+1} \right) $$  

  \item On majore le cosinus par $1$. Pour  $x \in [-\frac\pi2,+\frac\pi2]$ et $y \in [-\frac12,+\frac12]$, on obtient
  $$ \left|\frac{\partial f}{\partial x}(x,y)\right| \le \frac{1}{-\frac12+1} = 2 \qquad
\left|\frac{\partial f}{\partial y}(x,y)\right| \le \frac{\frac\pi2+\pi}{(-\frac12+1)^2}  = 6\pi$$

  \item Ainsi pour ces $(x,y)$, $\|\grad f(x,y)\| \le \sqrt{ 2^2 + (6\pi)^2} = 2\sqrt{1+9\pi^2}$.
  On note $k = 2\sqrt{1+9\pi^2}$ et alors $\| \grad f (x,y) \| \le  k$, pour $(x,y) \in [-\frac\pi2,+\frac\pi2] \times [-\frac12,+\frac12]$.
  
  
  \item L'inégalité des accroissements finis s'écrit 
 $$ \left| f(b)-f(a)  \right| \le k   \| b -a \|$$
 Ici $b-a = (x,y)$ et $f(a)=0$. Alors, pour tout $(x,y) \in [-\frac\pi2,+\frac\pi2] \times [-\frac12,+\frac12]$, :
 $$\left| f(x,y) \right| \le k \| (x,y) \|$$
c'est-à-dire 
 $$\left|  \sin \left( \frac{x+\pi}{y+1} \right) \right| \le 2\sqrt{1+9\pi^2} \sqrt{x^2+y^2}.$$ 
  
\end{itemize}



\end{exemple}


%----------------------------------------------------
\subsection{Fonction dont la différentielle est nulle}
 
 
Il est clair que pour une fonction constante, ses dérivées partielles sont partout nulles. La réciproque est vraie sur un ensemble connexe.
 
\begin{corollaire}
Soit $f: U \subset \Rr^n \rightarrow \Rr$ de classe $\mathcal{C}^1$, o\`u $U$ est un ouvert \evidence{connexe} de $\Rr^2$.
Si $\grad f (x,y) = (0,0)$, pour tout $(x,y) \in U$, alors $f$ est constante sur $U$.
\end{corollaire}


Dans la pratique pour $f : \Rr^2 \to \Rr$ qui vérifie $\grad f (x,y) = (0,0)$ pour tout $(x,y)$, on écrit :
\begin{itemize}
  \item comme  $\frac{\partial f}{\partial x}(x,y) = 0$, alors $f$ ne dépend pas de la variable $x$, c'est-à-dire $f(x,y) = g(y)$ où $g : \Rr \to \Rr$ ;
  \item mais comme en plus $\frac{\partial f}{\partial y}(x,y) = 0$, alors $g'(y) = 0$, donc $f(x,y)= g(y) = \text{cst}$.
\end{itemize}


\begin{proof}
Par le théorème des accroissements finis, si $V$ est une boule ouverte incluse dans $U$, alors pour tout $a,b \in V$, $f(b) = f(a)$, car $\grad f (c) = 0$ pour tout $c\in [a,b] \subset V \subset U$. La fonction $f$ est donc localement constante, c'est-à-dire constante dans un voisinage de chaque point.

Sur un ensemble connexe, une fonction localement constante est constante. C'est une propriété purement topologique des ensembles connexes. Voici la preuve.
Fixons $a \in U$ et $z_0 = f(a)$. Notons $V_1 = f^{-1}(\{z_0\}) = \{ b \in U \mid f(b)= z_0\}$. Par le premier paragraphe de cette preuve, $V_1$ est un ensemble ouvert. Notons $V_2 = f^{-1} ( \Rr \setminus \{z_0\} )$. Comme $f$ est une fonction continue, alors l'image réciproque d'un ouvert est un ouvert, donc $V_2$ est un ouvert. Il est clair que $U = V_1 \cup V_2$ et que $V_1$ et $V_2$ sont disjoints.
Comme $U$ est connexe et que $V_1$ n'est pas vide (car $a \in V_1$) alors $V_2$ est vide, donc $U=V_1$ et pour tout $b\in U$, $f(b)= z_0$.
\end{proof}



%----------------------------------------------------
\begin{miniexercices}
\sauteligne
\begin{enumerate}

  \item Soit $f(x,y) = x + y^2$, $a=(0,0)$, $b=(h,k)$. 
  Appliquer le théorème des accroissement finis entre $a$ et $b$. Trouver explicitement la valeur de $\theta \in ]0,1[$ (ou bien le point $c \in ]a,b[$) du théorème.
  
  \item Soit $f(x,y) = \ln (1+x^2-y)$. Trouver une constante $k$ tel que $| f(x,y) \le k \sqrt{x^2+y^2}$ pour tout $(x,y) \in [-\frac12,+\frac12]\times [-\frac12,+\frac12]$.

  \item Trouver toutes les fonctions $f : \Rr^2 \to \Rr$, tel que $\grad f (x,y) = (1,-1)$, pour tout $(x,y) \in \Rr^2$. 
  
\end{enumerate}
\end{miniexercices}


\auteurs{
\\
D'après des cours de Abdellah Hanani (Lille), 
Goulwen Fichou et Stéphane Leborgne (Rennes),
Laurent Pujo-Menjouet (Lyon). 

Revu et augmenté par Arnaud Bodin.

Relu par Barbara Tumpach et [...].%Stéphanie Bodin et Vianney Combet.
}


\finchapitre 
\end{document}


