%%%%%%%%%%%%%%%%%% PREAMBULE %%%%%%%%%%%%%%%%%%

\documentclass[11pt,a4paper]{article}

\usepackage{amsfonts,amsmath,amssymb,amsthm}
\usepackage[utf8]{inputenc}
\usepackage[T1]{fontenc}
\usepackage[french]{babel}
\usepackage{fancybox}
\usepackage{graphicx}
%\usepackage{tikz}

%----- Ensembles : entiers, reels, complexes -----
\newcommand{\Nn}{\mathbb{N}} \newcommand{\N}{\mathbb{N}}
\newcommand{\Zz}{\mathbb{Z}} \newcommand{\Z}{\mathbb{Z}}
\newcommand{\Qq}{\mathbb{Q}} \newcommand{\Q}{\mathbb{Q}}
\newcommand{\Rr}{\mathbb{R}} \newcommand{\R}{\mathbb{R}}
\newcommand{\Cc}{\mathbb{C}} \newcommand{\C}{\mathbb{C}}
\newcommand{\Kk}{\mathbb{K}} \newcommand{\K}{\mathbb{K}}

%----- Modifications de symboles -----
\renewcommand{\epsilon}{\varepsilon}
\renewcommand{\Re}{\mathop{\mathrm{Re}}\nolimits}
\renewcommand{\Im}{\mathop{\mathrm{Im}}\nolimits}
\newcommand{\llbracket}{\left[\kern-0.15em\left[}
\newcommand{\rrbracket}{\right]\kern-0.15em\right]}
\renewcommand{\ge}{\geqslant} \renewcommand{\geq}{\geqslant}
\renewcommand{\le}{\leqslant} \renewcommand{\leq}{\leqslant}

%----- Fonctions usuelles -----
\newcommand{\ch}{\mathop{\mathrm{ch}}\nolimits}
\newcommand{\sh}{\mathop{\mathrm{sh}}\nolimits}
\renewcommand{\tanh}{\mathop{\mathrm{th}}\nolimits}
\newcommand{\cotan}{\mathop{\mathrm{cotan}}\nolimits}
\newcommand{\Arcsin}{\mathop{\mathrm{arcsin}}\nolimits}
\newcommand{\Arccos}{\mathop{\mathrm{arccos}}\nolimits}
\newcommand{\Arctan}{\mathop{\mathrm{arctan}}\nolimits}
\newcommand{\Argsh}{\mathop{\mathrm{argsh}}\nolimits}
\newcommand{\Argch}{\mathop{\mathrm{argch}}\nolimits}
\newcommand{\Argth}{\mathop{\mathrm{argth}}\nolimits}
\newcommand{\pgcd}{\mathop{\mathrm{pgcd}}\nolimits} 

%----- Structure des exercices ------
%\theoremstyle{definition}

\newtheoremstyle{exostyle}
{15pt}
{15pt}
{\normalfont}
{0pt}
{\bfseries}
{}
{\newline}
{\thmname{#1}~\thmnumber{#2} -- \thmnote{#3}}
\theoremstyle{exostyle} 

\newtheorem{exo}{Exercice}
\newtheorem{ind}{Indications}
\newtheorem{cor}{Correction}

\newcommand{\exercice}[1]{} \newcommand{\finexercice}{}
%\newcommand{\exercice}[1]{{\tiny\texttt{#1}}\vspace{-2ex}} % pour afficher le numero absolu, l'auteur...
\newcommand{\enonce}{\begin{exo}} \newcommand{\finenonce}{\end{exo}}
\newcommand{\indication}{\begin{ind}} \newcommand{\finindication}{\end{ind}}
\newcommand{\correction}{\begin{cor}} \newcommand{\fincorrection}{\end{cor}}

\newcommand{\noindication}{\stepcounter{ind}}
\newcommand{\nocorrection}{\stepcounter{cor}}

\newcommand{\fiche}[1]{} \newcommand{\finfiche}{}
\newcommand{\titre}[1]{\centerline{\large \bf #1}}
\newcommand{\addcommand}[1]{}
\newcommand{\video}[1]{}

%----- Presentation ------
\setlength{\parindent}{0cm}

\newcommand{\ExoSept}{\textbf{\textsf{Exo7}}}
\newcommand{\LogoExoSept}{\setlength{\unitlength}{0.6em}
	\begin{picture}(0,0)  \thicklines     \put(0,4){\line(0,1){3}}   \put(0,7){\line(1,0){3}}
		\put(3,7){\line(0,-1){7}}  \put(0,4){\line(1,0){7}}   \put(3,0){\line(1,0){4}}
		\put(7,0){\line(0,1){4}}   \put(3,7){\line(4,-3){4}}  \put(7,4){\line(3,4){3}}  
		\put(10,8){\line(-4,3){4}} \put(3,7){\line(3,4){3}}   \put(4.6,6.8){\mbox{\ExoSept}}
\end{picture}}

%----- Commandes supplementaires ------

\usepackage[a4paper, margin = 2cm]{geometry}
\usepackage[charter]{mathdesign}
%\usepackage{import}

\usepackage{tikz}
\usetikzlibrary{calc,shadows,arrows,shapes,patterns,matrix}
\usetikzlibrary{decorations.pathmorphing}
\usetikzlibrary{fadings}
\usetikzlibrary{external}
\usetikzlibrary{positioning}
\usetikzlibrary{arrows}
\usetikzlibrary{backgrounds}
\usepackage{tikz,tikz-3dplot}

\newcommand{\sauteligne}{\leavevmode\vspace{-\baselineskip}}

\newcommand{\grad}{\operatorname{grad}} % dans le document
\newcommand{\diver}{\operatorname{div}}
\newcommand{\rot}{\operatorname{rot}}

\begin{document}

	
	
%%%%%%%%%%%%%%%%%% ENTETE %%%%%%%%%%%%%%%%%%


%\kern-2em
\textsc{Feuille d'exercices 6}\hfill\textsc{Fonctions de plusieurs variables}

\vspace*{0.5ex}
\hrule\vspace*{1.5ex} 
\hfil{\textbf{\Large \textsc{Extremums}}}
\vspace*{1ex} \hrule 
\vspace*{5ex} 


%===========================================
\section{Dérivées partielles secondes}



\exercice{}  
% Modification de l'exercice 2638
\enonce[Dérivées partielles secondes]
Calculer les dérivées partielles secondes et la matrice hessienne des fonctions $f$ définies par les
expressions suivantes :
\[
\sin^2(y/x) \qquad \text{ et } \qquad \exp(xyz).
\]
\finenonce

\indication 
La matrice hessienne est la matrice
constituée des dérivées partielles secondes.
Utiliser $\sin(2u)=2\sin u \cos u$ afin de simplifier les calculs.
\finindication

\correction
\begin{enumerate}
	\item $f(x,y) = \sin^2(y/x)$.
	
	Les dérivées partielles sont : 
	\[
	\frac{\partial f}{\partial x}(x,y) = -\frac{2y}{x^2}\sin(y/x)\cos(y/x) = -\frac{y}{x^2}\sin(2y/x),
	\]
	et 
	\[
	\qquad
	\frac{\partial f}{\partial y}(x,y) = \frac{2}{x}\sin(y/x)\cos(y/x) = \frac{1}{x}\sin(2y/x).
	\]
	On a utilisé $\sin(2u)=2\sin u \cos u$.
	
	Désormais on omettra souvent les \og{}$(x,y)$\fg{}, c'est-à-dire qu'on écrira $\frac{\partial f}{\partial x}$ à la place de $\frac{\partial f}{\partial x}(x,y)$.
	
	On calcule les dérivées partielles secondes  :
	\[
	\frac{\partial^2 f}{\partial x^2} 
	= \frac{\partial }{\partial x}\frac{\partial f}{\partial x}
	= \frac{2y}{x^3}\sin(2y/x)+ \frac{2y^2}{x^4}\cos(2y/x).
	\]
	
	\[
	\frac{\partial^2 f}{\partial y^2} 
	= \frac{\partial }{\partial y}\frac{\partial f}{\partial y}
	= \frac{2}{x^2} \cos(2y/x).
	\]	
	
	On sait que les dérivées croisées sont égales : $\frac{\partial^2 f}{\partial x \partial y} =  \frac{\partial^2 f}{\partial y \partial x}$ car $f$ est $\mathcal{C}^2$. Soit on en calcule une seule pour aller plus vite, soit on calcule les deux afin de se rassurer (cela vous garantit quasiment qu'elles sont exactes).
	
	\[
	\frac{\partial^2 f}{\partial x \partial y}
	= \frac{\partial^2 f}{\partial y \partial x}
	= -\frac1{x^2}\sin(2y/x)  - \frac{2y}{x^3}\cos(2y/x)
	.\]
	
	La matrice hessienne est le tableau des dérivées partielles secondes, par le lemme de Schwarz c'est une matrice symétrique :
	\[
	H_f (x,y) =
	\begin{pmatrix}
		\frac{\partial^2f}{\partial x^2}(x,y) &\frac{\partial^2f}{\partial x\partial y}(x,y) \\  \frac{\partial^2f}{\partial y\partial x}(x,y)&\frac{\partial^2f}{\partial y^2}(x,y)
	\end{pmatrix}
	\]
	Ici :
	\[
	H_f (x,y) =	
	\begin{pmatrix}	
	\frac{2y}{x^3}\sin(2y/x)+ \frac{2y^2}{x^4}\cos(2y/x) &
	-\frac1{x^2}\sin(2y/x)  - \frac{2y}{x^3}\cos(2y/x) \\
	-\frac1{x^2}\sin(2y/x)  - \frac{2y}{x^3}\cos(2y/x) &
	\frac{2}{x^2} \cos(2y/x)
	\end{pmatrix}.
	\]
		

	
	\item $f(x,y,z) = \exp(xyz)$.
	

Les dérivées partielles sont :
\[
\frac{\partial f}{\partial x} = yz e^{xyz},
\qquad
\frac{\partial f}{\partial y} = xz e^{xyz},
\qquad
\frac{\partial f}{\partial z} = xy e^{xyz}.
\]
On calcule les dérivées partielles d'ordre 2 en se limitant à $6$ calculs par le lemme de Schwarz, on en déduit la matrice hessienne (qui est une matrice symétrique) :
\[
H_f=
\begin{pmatrix}
	\frac{\partial^2f}{\partial x^2}&\frac{\partial^2f}{\partial x\partial y}&\frac{\partial^2f}{\partial x\partial z}\\  \frac{\partial^2f}{\partial y\partial x}&\frac{\partial^2f}{\partial y^2}&\frac{\partial^2f}{\partial y\partial z}\\  \frac{\partial^2f}{\partial z\partial x}&\frac{\partial^2f}{\partial z\partial y}&\frac{\partial^2f}{\partial z^2}\\
\end{pmatrix}
=
\begin{pmatrix}
	y^2z^2 e^{xyz} & z(1 + xyz) e^{xyz} & y(1 + xyz) e^{xyz} \\
	z(1 + xyz) e^{xyz} & x^2z^2 e^{xyz} & x(1 + xyz) e^{xyz} \\
	y(1 + xyz) e^{xyz} & x(1 + xyz) e^{xyz} & x^2y^2 e^{xyz}
\end{pmatrix}.
\]

\end{enumerate}


\fincorrection
\finexercice


\exercice{4147, quercia, 2010/03/11}
\enonce[Contre-exemple au théorème de Schwarz]

Soit $f(x,y) = \dfrac{xy^3}{x^2+y^2}$ si $(x,y)\ne 0$ et $f(0,0)=0$.
\begin{enumerate}
	\item \'Etudier la continuité de~$f$ et de ses dérivées partielles premières sur~$\R^2$.
	
	\item Montrer que~$\frac{\partial^2 f}{\partial x \partial y}(0,0)\ne\frac{\partial^2 f}{\partial y \partial x}(0,0)$.
\end{enumerate}
\finenonce

\noindication

\correction
\begin{enumerate}
	\item 
	Il est facile de vérifier que $f(x,y) \to 0$ en $(0,0)$, par exemple en calculant $f(r\cos\theta,r\sin\theta)$.
	
	On a, pour $(x,y) \neq (0,0)$ :
	\[
	\frac{\partial f}{\partial x}(x,y) = \frac{y^5-x^2y^3}{(x^2+y^2)^2} \longrightarrow 0 = \frac{\partial f}{\partial x}(0,0),\]
	et
	\[
	\frac{\partial f}{\partial y}(x,y) = \frac{3x^3y^2 + xy^4}{(x^2+y^2)^2} \longrightarrow 0 = \frac{\partial f}{\partial y}(0,0).\]
	Ainsi les dérivées partielles existent et sont continue sur $\Rr^2$ : $f$ est de classe $\mathcal{C}^1$.
	
	\item 
	Calculons $\frac{\partial^2 f}{\partial x \partial y}(0,0)$.
	Il s'agit de calculer la dérivée partielle $\frac{\partial }{\partial x}$ de la fonction $\frac{\partial f}{\partial y}$ :
	On forme le taux d'accroissement et on calcule la limite :
	\[
	\frac{\frac{\partial f}{\partial y}(0+h,0) - \frac{\partial f}{\partial y}(0,0)}{h}
	= \frac{0-0}{h} = 0 \to 0\]
	Donc $\frac{\partial^2 f}{\partial x \partial y}(0,0)=0$.
	
	Pour $\frac{\partial^2 f}{\partial y \partial x}(0,0)$.
	Il s'agit de calculer la dérivée partielle $\frac{\partial }{\partial y}$ de la fonction $\frac{\partial f}{\partial x}$ :
	On forme le taux d'accroissement et on calcule la limite :
	\[
	\frac{\frac{\partial f}{\partial x}(0,0+k) - \frac{\partial f}{\partial x}(0,0)}{k}
	= \frac{\frac{k^5}{k^4}-0}{k} = 1 \to 1\]
	Donc $\frac{\partial^2 f}{\partial y \partial x}(0,0)=1$.
	
	Ainsi dans ce cas exceptionnel les dérivées croisées ne sont pas égales.
	L'explication est que la fonction est de classe $\mathcal{C}^1$, mais pas de classe $\mathcal{C}^2$ comme on en aurait besoin dans le lemme de Schwarz.
	
	Noter qu'en dehors de $(0,0)$, $f$ est de classe $\mathcal{C}^2$ (en tant que somme, produit, quotient de fonctions de classe $\mathcal{C}^2$) et donc, pour tout $(x,y) \neq (0,0)$, on a bien :
	\[
	\frac{\partial^2 f}{\partial x \partial y}(x,y) = \frac{\partial^2 f}{\partial y \partial x}(x,y).
	\]
\end{enumerate}
\fincorrection

\finexercice





%===========================================
\section{Minimums et maximums}


\exercice{2641, debievre, 2009/05/19}
\enonce[Étude d'un point critique à l'origine]
Pour chacune des fonctions suivantes étudier la nature du point
critique donné :
\begin{enumerate}
	\item $f(x,y)=x^2-xy+y^2$ au point critique $(0,0)$,
	\item $f(x,y)=x^2+2xy+y^2+6$ au point critique $(0,0)$,
	\item $f(x,y)=x^3+2xy^2-y^4+x^2+3xy+y^2+10$ au point critique
	$(0,0)$.
\end{enumerate}
\finenonce

\indication 
Calculer les dérivées partielles secondes au point critique et appliquer le critère de Monge.
\finindication

\correction
\begin{enumerate}  
	\item $f(x,y)=x^2-xy+y^2$ en $(0,0)$.
	
	On a $\frac{\partial f}{\partial x}(x,y) = 2x-y$ et $\frac{\partial f}{\partial y}(x,y) = 2y-x$ qui s'annulent simultanément en $(0,0)$.
	
	On calcule les dérivées partielles secondes en $(x,y)$ :
	\[
	\frac{\partial^2 f}{\partial x^2}(x,y) = 2
	\qquad	
	\frac{\partial^2 f}{\partial x \partial y}(x,y) = -1
	\qquad
	\frac{\partial^2 f}{\partial y^2}(x,y) = 2
	\]	
	Ce sont ici des fonctions constantes, on les évalue au point critique :
	\[
	r = \frac{\partial^2 f}{\partial x^2}(0,0) = 2
\qquad	
s = \frac{\partial^2 f}{\partial x \partial y}(0,0) = -1
\qquad
t = \frac{\partial^2 f}{\partial y^2}(0,0) = 2
\]	
	
	\[
	H_f (0,0)
	= \begin{pmatrix} r & s\\ s & t \end{pmatrix}
	= \begin{pmatrix} 2 & -1\\ -1 & 2 \end{pmatrix}
	\]
	D'où $\det H_f(0,0)= rt-s^2 = 3 > 0$ et $r>0$.
	Par le critère de Monge, $f$ atteint un minimum local en $(0,0)$.
	

	
	\item $f(x,y)=x^2+2xy+y^2+6$ en $(0,0)$.
	
	\[
	\frac{\partial f}{\partial x}(x,y) = 2x+2y
	\qquad
	\frac{\partial f}{\partial y}(x,y) = 2x+2y
	\]
	\[
	\frac{\partial^2 f}{\partial x^2}(x,y) = 2
	\qquad	
	\frac{\partial^2 f}{\partial x \partial y}(x,y) = 2
	\qquad
	\frac{\partial^2 f}{\partial y^2}(x,y) = 2
	\]	
	\[
	H_f (0,0)
	= \begin{pmatrix} r & s\\ s & t \end{pmatrix}
	= \begin{pmatrix} 2 & 2 \\ 2 & 2 \end{pmatrix}
	\]
	Comme $\det H_f(0,0)= rt-s^2 = 0$, le critère de Monge ne permet pas de conclure !
	
	Il faut alors trouver une autre méthode.
	Ici $f(x,y)=x^2+2xy+y^2+6 = (x+y)^2+6$ donc le point $(0,0)$
	présente un minimum local (qui n'est pas strict).
	
	
	\item $f(x,y)=x^3+2xy^2-y^4+x^2+3xy+y^2+10$ en $(0,0)$.
		
	\[
	\frac{\partial f}{\partial x}(x,y) = 3x^2+2x+2y^2 +3y
	\qquad
	\frac{\partial f}{\partial y}(x,y) = 4xy-4y^3 +3x+2y
	\]
	\[
	\frac{\partial^2 f}{\partial x^2}(x,y) = 6x +2
	\qquad	
	\frac{\partial^2 f}{\partial x \partial y}(x,y) = 4y+3
	\qquad
	\frac{\partial^2 f}{\partial y^2}(x,y) = -12y^2 +4x +2
	\]	
	\[
	H_f (0,0)
	= \begin{pmatrix} r & s\\ s & t \end{pmatrix}
	= \begin{pmatrix} 2 & 3\\ 3 &  2 \end{pmatrix}
	\]
	
	D'où $\det H_f(0,0)= rt-s^2 = -5 < 0$.
	Par le critère de Monge en $(0,0)$, $f$ est un point-selle.
	Ce n'est donc ni un minimum local, ni un maximum local.
	
	
\end{enumerate}
\fincorrection
\finexercice



\exercice{2642, debievre, 2009/05/19}
\enonce[Recherche de minimums et maximums]
Trouver les points critiques de la  fonction $f$  suivante et
déterminer si ce sont des minimums locaux, des maximums locaux ou
des points-selles :
\[
f(x,y)=\sin x+y^2-2y+1.
\] 
\finenonce

\indication 
Calculer les dérivées partielles secondes puis utiliser le critère de Monge
\finindication

\correction 
Puisque
\[
\frac{\partial f}{\partial x}(x,y) = \cos x 
\quad \text{ et } \quad
\frac{\partial f}{\partial y}(x,y) = 2y-2,
\]
alors les points critiques sont les points
\[ \big( (k+\tfrac12)\pi,1 \big)\,, \qquad k\in \Zz.\]
En plus,
\[H_f(x,y) =
\begin{pmatrix} 
	-\sin x & 0\\ 
	0 & 2 
\end{pmatrix}.
\] 
Comme $-\sin ((k+\tfrac12)\pi)=(-1)^{k+1}$, alors 
\[H_f \big( (k+1/2)\pi,1 \big)
=\begin{pmatrix} (-1)^{k+1} & 0\\ 0 & 2 \end{pmatrix}.\]
Autrement dit $r =  (-1)^{k+1}$, $s=0$ et $t=2$.

Par conséquent, 
\begin{itemize}
	\item si $k$ est impair, $rt-s^2 = 2$ et $r>0$, donc 
le point $((k+1/2)\pi,1)$ est un minimum local,
	\item si $k$ est pair, $rt-s^2 = -2$ et le point $((k+1/2)\pi,1)$ est un point-selle. 
\end{itemize}
\fincorrection
\finexercice




\exercice{4191, quercia, 2010/03/11}
\enonce[\'Etude de points critiques]

Chercher les extremums des fonctions $f(x,y)$ suivantes :
\begin{enumerate}
	\item $3xy - x^3 - y^3$
		
	\item $-2(x-y)^2+x^4+y^4$
		
	\item $2x+y-x^4-y^4$		

	\item $\frac {xy}{(x+y)(1+x)(1+y)},\ x,y> 0$
		
	\item $x^2y^2(1+3x+2y)$
	
	\item $xe^y + ye^x$
		
	\item $x(\ln^2x + y^2)$, $x>0$
		
	\item $\sqrt{x^2+(1-y)^2} + \sqrt{y^2+(1-x)^2}$
			
\end{enumerate}
\finenonce

\indication
Trouver les points critiques et appliquer le critère de Monge.
\finindication

\correction
On va noter $f_x$ et $f_y$ les dérivées partielles premières, ainsi que $f_{xx}$, $f_{yy}$ et $f_{xy}=f_{yx}$ les dérivées partielles secondes (toutes les fonctions considérées sont $\mathcal{C}^2$).

\begin{enumerate}
	
	%%%%%%%%%%%%
	\item $f = 3xy - x^3 - y^3$.
	
	\begin{enumerate}
		\item
	\[
	f_x = 3y-3x^2
	\qquad
	f_y = 3x-3y^2
	\]
	\[
	f_{xx} = -6x
	\qquad 
	f_{xy}= 3
	\qquad 
	f_{yy}= -6y
	\]
	
	\item Points critiques :
	\[
	\left\{
	\begin{array}{l}
	f_x = 0 \\
	f_y = 0 
	\end{array}
	\right.
	\iff
	\left\{
	\begin{array}{l}
	y-x^2 = 0 \\	
	x-y^2 = 0 
	\end{array}
	\right.	
	\iff
	\left\{
	\begin{array}{l}
	y-y^4 = 0 \\		
	x=y^2 
	\end{array}
	\right.	
	\iff
	\left\{
	\begin{array}{l}
		y(1-y^3) = 0 \\
		x=y^2
	\end{array}
	\right.	
	\iff
	\left\{
	\begin{array}{l}
		y = 0 \text{ ou } y = 1		\\
		x=y^2 
	\end{array}
	\right.			
	\]	
	On a obtenu l'ordonnée $y=0$ ou $y=1$ des points critiques, on obtient l'abscisse par la relation $x=y^2$.
	Ainsi les points critiques sont $(0,0)$ et $(1,1)$.
	
	\item Étude en $(0,0)$.
	On calcule $r=f_{xx}(0,0) = 0$, $s=f_{xy}(0,0) = 3$, $t=f_{yy}(0,0) = 0$.
	On a $rt-s^2 = -9$ et par le critère de Monge $(0,0)$ est un point-selle (ce n'est donc ni un minimum local, ni un maximum local).
	
	Autrement dit, on est dans le cas où $H_f(0,0) = \begin{pmatrix}r&s\\s&t\end{pmatrix}=\begin{pmatrix}0&3\\3&0\end{pmatrix}$
	a un déterminant négatif et donc deux valeurs propres de signe contraire.
	

	
	\item Étude en $(1,1)$.
	On calcule $r=f_{xx}(1,1) = -6$, $s=f_{xy}(1,1) = 3$, $t=f_{yy}(1,1) = -6$.
	On a $rt-s^2 = 27$ et $r<0$ donc par le critère de Monge $(0,0)$ est un maximum local.
	
	Autrement dit, on est dans le cas où $H_f(1,1) = \begin{pmatrix}r&s\\s&t\end{pmatrix}=\begin{pmatrix}-6&3\\3&-6\end{pmatrix}$
	a deux valeurs propres négatives.
	 
	\end{enumerate}
	
	%%%%%%%%%%%%
	\item $f= -2(x-y)^2+x^4+y^4$
	
	\begin{enumerate}
	\item	
	
	\[
	f_x = 4(x^3-x+y)
	\qquad
	f_y = 4(y^3-y+x)
	\]
	\[
	f_{xx} = 4(3x^2-1)
	\qquad 
	f_{xy}= 4
	\qquad 
	f_{yy}= 4(3y^2-1)
	\]
	
	\item Points critiques :
	\[
	\left\{
	\begin{array}{l}
		f_x = 0 \\
		f_y = 0 
	\end{array}
	\right.
	\iff
	\left\{
	\begin{array}{l}
		x^3-x+y = 0 \\
		y^3-y+x = 0 
	\end{array}
	\right.	
	\]
	Si on fait la somme de ces deux équations on obtient $x^3=-y^3$, donc $x=-y$ (les nombres sont des réels).
	On substitue $y=-x$ dans la première équation, pour obtenir : $x^3-2x=0$ c'est-à-dire $x(x^2-2)=0$.
	Ainsi $x=0$ ou $x=\pm\sqrt2$ et $y=-x$.
	Les trois points critiques sont :
	\[
	(0,0) \qquad (+\sqrt2,-\sqrt2) \qquad (-\sqrt2,+\sqrt2)
	\]
	
	\item Étude en $(0,0)$.
	$r= -4$, $s=4$, $t=-4$, $rt-s^2 = 0$ et le critère de Monge ne permet pas de conclure.
	Cependant sur $\gamma_1(t) = (t,t)$ on a $f(\gamma_1(t))= 2t^4 \ge f(0,0)$ ainsi $(0,0)$ ne peut pas être un maximum local ; mais d'autre part sur $\gamma_2(t) = (t,0)$ on a, lorsque $t\to0$, $f(\gamma_2(t))= -2t^2+t^4 \sim -2t^2 \le f(0,0)$ ainsi $(0,0)$ ne peut pas être un minimum local. Conclusion : $(0,0)$ est un point-selle.
	
	\item Étude en $\pm(\sqrt2,-\sqrt2)$.
	$r=20$, $s=4$, $t=20$, $rt-s^2 = 384>0$ et $r>0$, par le critère de Monge, $f$ admet un minimum local en $(\sqrt2,-\sqrt2)$ et en $(-\sqrt2,\sqrt2)$.
	
	\end{enumerate}
	

	%%%%%%%%%%%%
	\item $f = 2x+y-x^4-y^4$
	
	\[
	f_x = 2-4x^3
	\qquad
	f_y = 1-4y^3
	\]
	\[
	f_{xx} = -12x^2
	\qquad 
	f_{xy}= 0
	\qquad 
	f_{yy}= -12y^2
	\]
	
	Point critique : $(2^{-1/3},4^{-1/3})$, c'est un maximum local par le critère de Monge.	
	
	
	
	%%%%%%%%%%%%
	\item $f = \frac {xy}{(x+y)(1+x)(1+y)},\ x,y> 0$
	
	Des calculs un peu lourds donnent :
	\[
	f_x = \frac {y(y-x^2)}{(x+y)^2(1+x)^2(1+y)}
	\qquad
	f_y = \frac {x(x-y^2)}{(x+y)^2(1+x)(1+y)^2}
	\]
	
	Les points critiques sont les solutions qui vérifient à la fois $y-x^2=0$ et $x-y^2=0$.
	Donc $x=x^4$. La seule solution vérifiant $x>0,y>0$ est $(1,1)$.
	En ce point $r=-\frac1{16}$, $s=\frac1{32}$, $t=-\frac1{16}$.
	Donc $rt-s^2 > 0$ avec $r<0$. Il s'agit d'un maximum local.
	
	\textbf{Autre méthode.}
	L'idée est d'utiliser le logarithme pour simplifier le calcul de la dérivée d'un produit.
	En effet la dérivée de $\ln\big( u v \big)$ est simplement $\frac{u'}{u} + \frac{v'}{v}$.
		
	Comme on restreint l'étude au domaine $ D = \{ (x,y) \mid x>0, \, y>0\}$, tous les facteurs qui interviennent dans l'expression de $f$ sont strictement positifs. On peut donc écrire: 
	\[
	g(x,y) := \ln(f(x,y)) = \ln(x) + \ln(y) - \ln(x+y) - \ln(1+x) - \ln(1+y). 
	\]
	$g$ est de classe $\mathcal{C}^2$ comme composée de deux fonctions de classe $\mathcal{C}^2$. De plus, comme le logarithme est une fonction strictement croissante, les deux fonctions $f$ et $g$ atteignent leurs extremums locaux aux mêmes points de $D$. Nous allons donc chercher les extremums locaux de $g$, dont les dérivées partielles sont plus simples à calculer que celles de $f$.
	
	On a:
	\[
		g_x (x,y) =\frac{1}{x}-\frac{1}{x+y}-\frac{1}{1+x}  = \frac{y-x^2}{x(x+y)(1+x)}.
	\]
	
	En remarquant que $g(x,y) = g(y,x)$, on obtient:
	\[
		g_y (x,y) = g_x(y,x) = \frac{1}{y}-\frac{1}{y+x}-\frac{1}{1+y} = \frac{x-y^2}{y(x+y)(1+y)}.
	\]
	
	Comme précédemment, on trouve que l'unique point critique est $(1,1)$. 
	
	On a ensuite :
	\[
	g_{xx} = \frac{-1}{x^2}+\frac{1}{(x+y)^2} + \frac{1}{(1+x)^2}, \qquad g_{xy} = \frac{1}{(x+y)^2}, \qquad g_{yy}= \frac{-1}{y^2}+\frac{1}{(x+y)^2} + \frac{1}{(1+y)^2},
	\]		
	et donc $r=\frac{-1}{2}$, $s=\frac{1}{4}$ et $t=\frac{-1}{2}$. On a alors $rt-s^2 = \frac{1}{4} - \frac{1}{16} = \frac{3}{16} >0$ et $r<0$, ce qui implique que $g$ admet un maximum local en $(1,1)$. Par suite, $f$ admet également un maximum local en $(1,1)$.
	
	\emph{Remarque.} On peut montrer que $f(1,1) = \frac{1}{8}$ est même un maximum global (sur $D$) en utilisant l'inégalité
	\[
	ab \leq \frac{1}{2}(a^2+b^2), \quad \text{pour tous } a,b \in \mathbb{R}.
	\]
	Pour tout $(x,y) \in D$, on a en effet:
	\begin{align*}
		xy & = (\sqrt{x} \cdot \sqrt{y}) \cdot (1 \cdot \sqrt{x}) \cdot (1 \cdot \sqrt{y}) \\
		& \leq \frac{1}{2}(x+y) \cdot \frac{1}{2}(1+x) \cdot \frac{1}{2}(1+y)
	\end{align*}
	et donc $f(x,y) \leq \frac{1}{8}$.
	
	

	%%%%%%%%%%%%	
	\item $f = x^2y^2(1+3x+2y)$
	
	\begin{enumerate}
	\item	
	\[
	f_x = xy^2(9x+4y+2)
	\qquad
	f_y = 2x^2y(3x+3y+1)
	\]
	\[
	f_{xx} = 2y^2(9x+2y+1)
	\qquad 
	f_{xy}= 2xy(9x+6y+2)
	\qquad 
	f_{yy}= 2x^2(3x+6y+1)
	\]
	
	\item Points critiques : soit $x=0$ et alors n'importe quel $(0,y)$ est point critique, 
	soit $y=0$ et alors n'importe quel $(x,0)$ est point critique.
	Soit $x\neq0$ et $y\neq0$ et alors un point critique vérifie :
	\[
	\left\{
	\begin{array}{l}
		9x+4y+2 = 0 \\
		3x+3y+1 = 0 
	\end{array}
	\right.
	\iff
	\left\{
	\begin{array}{l}
		x = -\frac{2}{15} \\
		y = -\frac{1}{5} 
	\end{array}
	\right.	
	\]
	
	Les points critiques sont donc les $(x,0)$, les $(0,y)$ et $(-\frac{2}{15},-\frac{1}{5})$.
	
	\item Étude en $(-\frac{2}{15},-\frac{1}{5})$.
	$r = -\frac{6}{125}$, $s = -\frac{8}{375}$, $t = -\frac{8}{375}$. On a $rt-s^2>0$ et $r<0$ il s'agit donc d'un maximum local.
	
	\item Étude en $(x,0)$.	
	$r = f_{xx}(x,0)= 0$, 
	$s = f_{xy}(x,0)= 0$, 
	$t = f_{yy}(x,0)= 2x^2(3x+1)$.
	On a $rt-s^2=0$ et le critère de Monge ne permet pas de conclure.
	On étudie $f$ à la main autour de $(x,0)$. On sait que $x^2y^2 \ge 0$, il s'agit donc juste d'étudier 
	$1+3x+2y$ autour de $(x,y)=(x,0)$. Si $x > -\frac13$ alors $1+3x+2y > 0$ autour de $(x,0)$ donc $f(x,y) \sim kx^2y^2$ ($k\in\Rr_+^*$) et ainsi il s'agit d'un minimum local. De même si $x < -\frac13$ il s'agit d'un maximum local.
	En $(-\frac13,0)$ c'est un point-selle.
	
	\item Étude en $(0,y)$.
	De même : maximum local pour $y<-1/2$, minimum local pour $y > -1/2$, point-selle en $y=-1/2$.
	
	\end{enumerate}
	
	

	
	
	%%%%%%%%%%%%
	\item $f = xe^y + ye^x$
	
	\[
	f_x = e^y + ye^x
	\qquad
	f_y = xe^y + e^x
	\]

	\[
	f_{xx} = ye^x
	\qquad 
	f_{xy}= e^x + e^y
	\qquad 
	f_{yy}= xe^y
	\]
	
	Recherche des point critiques : comme la fonction est symétrique (c'est-à-dire $f(x,y)=f(y,x)$) alors
	un point critique $(x_0,y_0)$ vérifie $y_0=x_0$. Donc l'équation $f_x(x_0,y_0) = 0$ devient $e^{x_0}+x_0e^{x_0}=0$ d'où $x_0=-1$ et donc $y_0=-1$. Bilan : un seul point critique $(-1,-1)$.
	
	En ce point, $r=-1/e$, $s= +2/e$, $t=-1/e$. Donc $rt-s^2 <0$, il s'agit d'un point-selle.
	
	
	%%%%%%%%%%%%
	\item $f = x(\ln^2x + y^2)$, $x>0$
	
	\[
	f_x = \ln^2x + 2\ln x + y^2
	\qquad
	f_y = 2xy
	\]
	\[
	f_{xx} = 2\frac{\ln x+1}{x}
	\qquad 
	f_{xy}= 2y
	\qquad 
	f_{yy}= 2x
	\]
	
	Recherche du point critique. Par hypothèse $x>0$, donc $y=0$ et par suite $\ln^2x + 2\ln x=0$,
	d'où $\ln x(\ln x+2)=0$. Ainsi $x=1$ ou $x= e^{-2}$. Les points critiques sont $(1,0)$ et $(e^{-2},0)$.
	
	Le critère de Monge s'applique et indique que  $(1,0)$ est un minimum local alors que $(e^{-2},0)$ est un point-selle.
	
	
	%%%%%%%%%%%%
	\item $f = \sqrt{x^2+(1-y)^2} + \sqrt{y^2+(1-x)^2}$
	
	Le plus simple est d'interpréter géométriquement la fonction $f$ comme la somme des distances entre un point $M(x,y)$ et deux points $A(0,1)$ et $B(1,0)$ du plan.
	Cette somme est minimale (et vaut la longueur $AB$) si et seulement si $M$ appartient au segment $[A,B]$ : donc $x \in[0,1]$ et $y=1-x$.
	
\end{enumerate}

\fincorrection


\exercice{4182, quercia, 2010/03/11}
\enonce[Point non extrémal]
On pose pour $(x,y)\in\R^2$~:
$$f(x,y) = x^2+y^2-2x^2y - \frac{4x^6y^2}{(x^4+y^2)^2}\quad\text{si }(x,y)\ne 0,
\qquad f(0,0)=0.$$
\begin{enumerate}
	\item Montrer que $f$ est continue sur $\R^2$.
	
	\item Soit $\theta\in\R$ fixé et $g_\theta(r) = f(r\cos\theta,r\sin\theta)$.
	Montrer que $g_\theta$ admet un minimum local strict en~$r=0$.
	
	\item Calculer $f(x,x^2)$. Conclusion~?
	
\end{enumerate}
\finenonce

\indication
$2x^4y^2 \le (x^4+y^2)^2$.
\finindication

\correction
\begin{enumerate}
	\item $f$ est continue en dehors de l'origine ;
	$f$ est aussi continue en $(0,0)$ : 
	tout d'abord $2x^4y^2 \le (x^4+y^2)^2$ car $(x^4+y^2)^2 = x^8 + y^4 + 2x^4y^2 \ge 2x^4y^2$.
	Ainsi :
	\[
	0 \le \frac{4x^6y^2}{(x^4+y^2)^2} = 2x^2\frac{2x^4y^2}{(x^4+y^2)^2} \le 2x^2 \xrightarrow{} f(0,0) = 0.
	\]
	
	\item 
	
	\[
	g_\theta(r) = f(r\cos\theta,r\sin\theta) =  r^2 - r^3 c_{1,\theta} + r^4 c_{2,\theta}(r) 
	= r^2 +o(r^2).\]
	
	Donc le long d'un rayon d'angle $\theta$ fixé, $g_\theta(r) \sim r^2$. 
	Ainsi le long de ce rayon, $g_\theta$ admet un minimum local à l'origine, autrement dit sur chaque rayon les valeurs de $f$ sont supérieures à $f(0,0)$.
	
	
	\item $f(x,x^2)=-x^4$. Donc $(0,0)$ n'est pas minimum local de~$f$ car on trouve aussi des valeurs inférieures à $f(0,0)$.
\end{enumerate}


\fincorrection

\finexercice




%===========================================
\section{Applications}


\exercice{4159, quercia, 2010/03/11}
\enonce[Ajustement linéaire]

Étant donnés $n$ couples de réels  
$(x_i,y_i)$ avec $^1 \le i \le n$,
on cherche une droite $D$ d'équation $y = ax+b$ telle que
$E(a,b) = \sum_{i=1}^n (y_i-ax_i-b)^2$ soit minimal.

On note
$$\overline  x    = \frac 1n \sum_{i=1}^n x_i,\qquad
\overline  y    = \frac 1n \sum_{i=1}^n y_i,\qquad
\overline {x^2} = \frac 1n \sum_{i=1}^n x_i^2,\qquad
\overline {xy} = \frac 1n \sum_{i=1}^n x_iy_i,$$
et on suppose $\overline {x^2} \ne \overline x^2$.

\begin{enumerate}
	\item Résoudre le problème.
	\item Interpréter la relation $\overline {x^2} \ne \overline x^2$
	à l'aide de l'inégalité de Cauchy-Schwarz.
\end{enumerate}
\finenonce

\indication
Si la fonction $E$ atteint un minimum alors en ce ce minimum les deux dérivées partielles par rapport à $a$ et à $b$ s'annulent. Les variables sont bien $a$ et $b$ ; les $x_i$, $y_i$ sont des constantes.
\finindication

\correction
\begin{enumerate}
	\item 
	Comme $E(a,b) = \sum_{i=1}^n (y_i-ax_i-b)^2$ alors :
	\[
	\frac{\partial E}{\partial a}(a,b) 
	= \sum_{i=1}^n -2x_i(y_i-ax_i-b)
	= -2n \overline {xy} + 2n a \overline{x^2} + 2nb \overline{x}.\]
	Et 
	\[
	\frac{\partial E}{\partial b}(a,b) 
	= \sum_{i=1}^n -2(y_i-ax_i-b)
	= -2n \overline{y} + 2n a \overline{x} +2nb.\]
		
	La fonction $E$ est une somme de carrés, elle admet donc (au moins) un minimum global.
	Ce minimum global est nécessairement un point critique. Calculons-le !
	En un point critique les deux dérivées partielles s'annulent simultanément, donc
	\begin{equation}
		\label{Ea}
		\tag{$E_a$}
	\frac{\partial E}{\partial a}(a,b)	= 0 \iff - \overline {xy} +  a \overline{x^2} + b \overline{x} = 0
	\end{equation}
	\begin{equation}
	\label{Eb}
	\tag{$E_b$}
	\frac{\partial E}{\partial b}(a,b)	= 0 \iff  - \overline{y} +  a \overline{x} + b= 0
	\end{equation}	
	On calcule $\eqref{Ea} - \overline{x} \eqref{Eb}$ pour obtenir :
	\[
	- \overline {xy}+ \overline{x} \cdot \overline{y} + a\big( \overline{x^2} -\overline{x}^2 \big) = 0.
	\]
	D'où :
	\[
	a = \frac{\overline {xy} - \overline  x \cdot \overline  y}{  \overline{x^2} - \overline{x}^2} = \frac{\text{Covariance}(x,y)}{\text{Variance}(x)}.
	\]
	Puis par \eqref{Eb} on en déduit :

	\[b= \overline  y - a \overline  x.\]
	
	On a trouvé un seul point critique possible, c'est donc nécessairement le minimum global.
	
	\item  On applique Cauchy-Schwarz avec $X = (x_1,\ldots,x_n)$ et $Y= (1,\ldots,1)$.
	Alors $\langle X \mid Y \rangle \le \| X \| \cdot \| Y \|$, donne
	$\left( \sum x_i \right)^2 \le \left(\sum x_i^2\right) \ \times \ n$
	donc $\overline{x}^2 \le \overline{x^2}$ (i.e. la variance est toujours positive ou nulle).
	Le cas d'égalité de Cauchy-Schwarz, nous dit donc que $\overline{x}^2 = \overline{x^2}$, ssi $X = \lambda Y$ ($\lambda\in\Rr$), c'est-à-dire toutes les coordonnées de $X$ sont identiques, i.e. ssi $x_1=x_2=\cdots=x_n$.
	
\end{enumerate}
\fincorrection
\finexercice



\exercice{2640, debievre, 2009/05/19}
\enonce[L'équation des ondes]
L'équation des ondes est l'équation aux dérivées partielles :
\begin{equation}\label{eq:ondes}
	\frac{\partial^2 f}{\partial x^2}-\frac{\partial^2 f}{\partial t^2}=0.
\end{equation}
Il s'agit de trouver la solution générale 
$f\colon\R^2\to\R$, $(x,t) \mapsto f(x,t)$ (de classe $\mathcal{C}^2$) de cette équation.

\begin{enumerate}
	\item Grâce au changement de variables
	\[
	\Phi : \R^2 \longrightarrow \R^2,
	\ (u,v) \longmapsto (x,t)=\left(\frac{u-v}{2}, \frac{u+v}{2}\right),
	\]
	la fonction $f$ se transforme en 
	$F(u,v)= f \circ \Phi(u,v) = f(\frac{u-v}{2}, \frac{u+v}{2})$.
	Montrer que pour que $f$ soit solution de (\ref{eq:ondes}) 
	il faut et il suffit que
	\begin{equation}\label{eq:ondesbis}
		\frac{\partial^2F}{\partial u\partial v}=0 .
	\end{equation}


    \item Montrer que, si $F$ satisfait \`a (\ref{eq:ondesbis}), il existe deux fonctions $g_1,g_2\colon\R\to\R$ telles que
    \[
    F(u,v)=g_1(u)+g_2(v).
    \]
    
    \item  \'Ecrire la solution générale de (\ref{eq:ondes}) et expliquer la phrase: ``En une dimension d'espace, toute solution de l'équation des ondes s'écrit comme somme d'une onde qui se déplace vers la droite et une qui se déplace vers la gauche.''

	\item Trouver la solution unique de l'équation des ondes qui
satisfait aux conditions initiales
\begin{equation}
	f(x,0)=\sin x, \qquad \frac{\partial f}{\partial t}(x,0)=-\cos x.
	\label{in}
\end{equation}

\end{enumerate}

\finenonce

\noindication

\correction
\begin{enumerate}  
	\item 
	$F = f\circ \Phi$, on applique la formule \og{}$J_F = J_f \times J_\Phi$\fg{} où :
	\[
	J_F = \begin{pmatrix} \frac{\partial F}{\partial u} & \frac{\partial F}{\partial v} \end{pmatrix}
	\qquad
	J_f = \begin{pmatrix} \frac{\partial f}{\partial x} & \frac{\partial f}{\partial t}  \end{pmatrix}
	\qquad
	J_\Phi(u, v) =\begin{pmatrix} \frac12 & -\frac12 \\  \frac12 & \frac12 \end{pmatrix}
	\]
	Par la formule \og{}$J_F = J_f \times J_\Phi$\fg{} nous obtenons :
	\begin{align*}
		\frac{\partial F}{\partial u}&= 
		\frac 12\frac{\partial f}{\partial x}+\frac 12\frac{\partial f}{\partial t}
		\\
		\frac{\partial F}{\partial v}&= 
		-\frac 12\frac{\partial f}{\partial x}+\frac 12\frac{\partial f}{\partial t}
	\end{align*}
	Autrement dit, nous obtenons les identités d'opérateurs : 
	\begin{align*}
	\frac{\partial }{\partial u}
	&=\frac12\left(\frac{\partial}{\partial x}+\frac{\partial}{\partial t}\right) \\
	\frac{\partial }{\partial v}
	&=\frac12\left(-\frac{\partial}{\partial x}+\frac{\partial}{\partial t}\right)
	\end{align*}	
	Nous avons appliqué ces identités à la fonction $f$, mais on peut aussi les appliquer à 
	$\frac{\partial f}{\partial x}$ et $\frac{\partial f}{\partial x}$ afin de calculer $\frac{\partial^2 F}{\partial u \partial v}$ :
	\begin{align*}
	\frac{\partial^2 F}{\partial u \partial v}
	&= \frac{\partial }{\partial u}\frac{\partial F}{\partial v} \\
	&= \frac{\partial }{\partial u} \left( -\frac 12\frac{\partial f}{\partial x}+\frac 12\frac{\partial f}{\partial t} \right) \\
	&= -\frac 12\frac{\partial }{\partial u}\frac{\partial f}{\partial x}+\frac 12\frac{\partial }{\partial u}\frac{\partial f}{\partial t} \\
	&= -\frac 12\left(\frac12\frac{\partial}{\partial x}\frac{\partial f}{\partial x}+\frac12\frac{\partial}{\partial t}\frac{\partial f}{\partial x} \right) 
	+\frac 12\left( \frac12\frac{\partial}{\partial x}\frac{\partial f}{\partial t}+\frac12\frac{\partial}{\partial t}\frac{\partial f}{\partial t} \right) \\
	&= 	-\frac 14\frac{\partial^2 f}{\partial x^2}
		+\frac 14\frac{\partial^2 f}{\partial t^2}
	\end{align*}
	
	(On a utilisé $f$ de classe $\mathcal{C}^2$, pour simplifier les dérivées croisées.)
	D'où pour que $f$ satisfasse l'équation \eqref{eq:ondes} il faut et il suffit que $F$ satisfasse l'équation
	\eqref{eq:ondesbis}.
	
	\item Supposons que $F$ satisfasse l'équation \eqref{eq:ondesbis}.
	Alors la fonction $\frac{\partial F}{\partial u}$ est une fonction, disons $h_1$, dépendant seulement de la variable $u$ et la fonction $\frac{\partial F}{\partial v}$ est une fonction, disons $h_2$, dépendant seulement de la variable $v$.
	Par conséquent,
	$F(u,v)= g_1(u)+g_2(v)$ o\`u $g'_1=h_1$ et $g'_2=h_2$.
	
	\item On remarque que $u=x+t$ et $v=t-x$. La solution générale de
	\eqref{eq:ondes} s'écrit alors
	\[
	f(x,t)=g_1(u)+g_2(v)=g_1(x+t)+g_2(t-x) .
	\]
	En considérant $x$ comme la variable de position et $t$ la variable de temps, la fonction $g_1$ décrit
	une onde qui se déplace vers la droite et la fonction $g_2$ décrit une onde qui se déplace vers la gauche.
	
	\item Enfin, pour trouver la solution unique satisfaisant aux conditions initiales \eqref{in} nous constatons que les conditions initiales entraînent les égalités :
	\begin{align*}
		f(x,0)&=g_1(x)+g_2(-x)=\sin x
		\\
		\frac {\partial f}{\partial x}(x,0)&=g'_1(x)-g'_2(-x)=\cos x
		\\
		\frac {\partial f}{\partial t}(x,0)&=g'_1(x)+g'_2(-x)=-\cos x
	\end{align*}
	d'o\`u $g'_1(x)=0$ et $g'_2(-x)=-\cos x$, c'est-à-dire 
	$g_2(x)=\sin (-x)$. Par conséquent,
	la solution unique cherchée $f$ s'écrit
	\[
	f(x,t)= \sin(x-t).
	\]
\end{enumerate}
\fincorrection
\finexercice


%===========================================
\section{Laplacien \& co}

\exercice{}
\enonce[Laplacien, gradient, divergence, rotationnel]
Le \emph{laplacien} de $f : \Rr^n \to \Rr$ est :
$$\Delta f(x) =
\sum_{i=1}^n \frac{\partial^2 f}{\partial x_i^2}(x).$$

\begin{enumerate}
	\item Montrer $\Delta f(x) = \diver( \grad f(x) ) $.
	
	Cela justifie l'écriture $\Delta f(x) = \nabla \cdot \nabla f(x) =\nabla^2 f (x)$.
	
	\item Montrer $\diver( \rot F(x) ) = 0$ pour $F : \Rr^3 \to \Rr^3$ dont les composantes sont de classe $\mathcal{C}^2$.
	
	\item Montrer $\rot( \grad f(x) ) = (0,0,0)$ pour $f : \Rr^3 \to \Rr$ de classe $\mathcal{C}^2$.
\end{enumerate}

\finenonce

\indication
$f : \Rr^n \to \Rr$ :
$$\grad f (x) = \begin{pmatrix}
	\frac{\partial f}{\partial x_1}(x)\\
	\vdots \\
	\frac{\partial f}{\partial x_n}(x)
\end{pmatrix}.$$


$F = (f_1,\ldots,f_n) : \Rr^n \to \Rr^n$ 

$$\diver F (x) = \sum_{i=1}^n \frac{\partial f_i}{\partial x_i}(x).$$


$F = (f_1,f_2,f_3) : \Rr^3 \to \Rr^3$ 
$$\rot F (x,y,z)=
\begin{pmatrix}
	\dfrac{\partial f_3}{\partial y}(x,y,z)-\dfrac{\partial f_2}{\partial z}(x,y,z) \\[2ex]
	\dfrac{\partial f_1}{\partial z}(x,y,z)-\dfrac{\partial f_3}{\partial x}(x,y,z) \\[2ex]
	\dfrac{\partial f_2}{\partial x}(x,y,z)-\dfrac{\partial f_1}{\partial y}(x,y,z)
\end{pmatrix}.
$$
\finindication


\correction
\begin{enumerate}
\item On rappelle que pour $f : \Rr^n \to \Rr$ :
$$\grad f (x) = \begin{pmatrix}
	\frac{\partial f}{\partial x_1}(x)\\
	\vdots \\
	\frac{\partial f}{\partial x_n}(x)
\end{pmatrix}.$$

Et pour $F = (f_1,\ldots,f_n) : \Rr^n \to \Rr^n$ :
$$\diver F (x) = \sum_{i=1}^n \frac{\partial f_i}{\partial x_i}(x).$$

Donc la divergence du gradient est :
\begin{align*}
\diver( \grad f(x) ) 
&= \diver\begin{pmatrix}
	\frac{\partial f}{\partial x_1}(x)\\
	\vdots \\
	\frac{\partial f}{\partial x_n}(x) \end{pmatrix} \\
&= \frac{\partial }{\partial x_1}\frac{\partial f}{\partial x_1}(x) + \cdots + \frac{\partial }{\partial x_n}\frac{\partial f}{\partial x_n}(x) \\
&= \frac{\partial^2 f}{\partial x_1^2}(x) + \cdots + \frac{\partial^2 f}{\partial x_n^2}(x) \\
&= \Delta f(x).
\end{align*}

	

\item Soit $F = (f_1,f_2,f_3) : \Rr^3 \to \Rr^3$.
On rappelle que :
$$\rot F (x,y,z)=
\begin{pmatrix}
	\dfrac{\partial f_3}{\partial y}(x,y,z)-\dfrac{\partial f_2}{\partial z}(x,y,z) \\[2ex]
	\dfrac{\partial f_1}{\partial z}(x,y,z)-\dfrac{\partial f_3}{\partial x}(x,y,z) \\[2ex]
	\dfrac{\partial f_2}{\partial x}(x,y,z)-\dfrac{\partial f_1}{\partial y}(x,y,z)
\end{pmatrix}.
$$
\begin{align*}
\diver( \rot F(x) ) 
&=  \frac{\partial }{\partial x} \left( \frac{\partial f_3}{\partial y}-\frac{\partial f_2}{\partial z} \right)
+ \frac{\partial }{\partial y} \left( \frac{\partial f_1}{\partial z}-\frac{\partial f_3}{\partial x} \right)
+ \frac{\partial }{\partial z } \left( \frac{\partial f_2}{\partial x}-\frac{\partial f_1}{\partial y} \right) \\
&= \frac{\partial^2 f_3}{\partial x \partial y} - \frac{\partial^2 f_2}{\partial x \partial z}
\quad + \quad\frac{\partial^2 f_1}{\partial y \partial z} - \frac{\partial^2 f_3}{\partial y \partial x}
\quad +\quad \frac{\partial^2 f_2}{\partial z \partial x} - \frac{\partial^2 f_1}{\partial z \partial y} \\
&= 0.
\end{align*}
On a utilisé que les dérivées croisées sont égales car chaque $f_i$ est de classe $\mathcal{C}^2$.

\item Soit $f : \Rr^3 \to \Rr$ de classe $\mathcal{C}^2$.
\[
\rot( \grad f(x) ) 
= \rot \begin{pmatrix}
	\frac{\partial f}{\partial x} \\
	\frac{\partial f}{\partial y} \\
	\frac{\partial f}{\partial z} \\
	\end{pmatrix}  \\
= \begin{pmatrix}
	\frac{\partial }{\partial y}\frac{\partial f}{\partial z}-\frac{\partial }{\partial z}\frac{\partial f}{\partial y} \\[2ex]
	\frac{\partial }{\partial z}\frac{\partial f}{\partial x}-\frac{\partial }{\partial x}\frac{\partial f}{\partial z} \\[2ex]
	\frac{\partial }{\partial x}\frac{\partial f}{\partial y}-\frac{\partial }{\partial y}\frac{\partial f}{\partial x} 
\end{pmatrix} \\
= \begin{pmatrix} 0 \\ 0 \\0 \end{pmatrix}.
\]

\end{enumerate}
\fincorrection

\finexercice



\exercice{4145, quercia, 2010/03/11}
\enonce[Laplacien en dimension $n$]

Soit $f$ une application de classe $\mathcal{C}^2$ de $\Rr_+^*$ dans $\Rr$.
On définit une application $F$ de $\Rr^n\setminus\{0\}$ dans $\Rr$ par :
$F(x_1,\dots,x_n) = f \left(\sqrt{x_1^2 + \dots + x_n^2}\,\right)$.
Calculer le laplacien de $F$ en fonction de $f$.
\finenonce

\indication
Dériver la composition $F = f \circ r$ par rapport à $x_i$. Puis dériver $\frac{\partial F}{\partial x_i}$ une seconde fois.
Il faut trouver : $\Delta F = \frac{n-1}r f'(r) + f''(r)$.

\finindication 

\correction
\begin{enumerate}
	\item On note $r = \sqrt{x_1^2 + \dots + x_n^2}$ à la fois comme un nombre réel et une fonction $r : \Rr^n \to \Rr$. On a donc $F = f \circ r$.
	
	On va appliquer la formule \og{}$J_F = J_f \times J_r$\fg{}, sachant que 
	\[
	J_F = \begin{pmatrix} \frac{\partial F}{\partial x_1} &   \ldots & \frac{\partial F}{\partial x_n}\end{pmatrix}
	\qquad
	J_f = f'(r)
	\qquad 
	J_r = \begin{pmatrix} \frac{\partial r}{\partial x_1} &   \ldots & \frac{\partial r}{\partial x_n}\end{pmatrix}
	= \begin{pmatrix} \frac{x_1}{r} &   \ldots & \frac{x_n}{r}\end{pmatrix}
	\]
	Ainsi :
	\[
	\frac{\partial F}{\partial x_i} = \frac{x_i}{r} \cdot f'(r).
	\]
	Autrement dit, on a une identité d'opérateurs :
	\[
	\frac{\partial}{\partial x_i} = \frac{x_i}{r} \cdot  \frac{d}{dr}.
	\]	
	
	\item Dérivons $G = f'(r) = f' \circ r$ par rapport à $x_i$, en appliquant la même formule obtenue (avec $f'$ à la place de $f$) :
	\[
	\frac{\partial f'(r)}{\partial x_i} = \frac{x_i}{r} \cdot  f''(r) .
	\]	
	
	\item On peut maintenant dériver $\frac{\partial F}{\partial x_i} = \frac{x_i}{r} \cdot  f'(r)$ par rapport à $x_i$ :
	\begin{align*}
	\frac{\partial^2 F}{\partial x_i^2} 
	&= 	\frac{\partial }{\partial x_i}\frac{\partial F}{\partial x_i} \\
	&=  \frac{\partial }{\partial x_i} \left( \frac{x_i}{r} \cdot  f'(r) \right) \\
	&= \frac{\partial \frac{x_i}{r}}{\partial x_i} \cdot  f'(r) + \frac{x_i}{r} \cdot \frac{\partial f'(r)}{\partial x_i} 
	 \\
	&= 	\frac{r-\frac{x_i^2}{r}}{r^2} \cdot  f'(r) + \frac{x_i^2}{r^2}  \cdot f''(r) \\
	&= 	\frac{r^2-x_i^2}{r^3} \cdot f'(r) + \frac{x_i^2}{r^2} \cdot  f''(r)\\	
	\end{align*}
	
	\item On somme les équations précédentes :
	
	\begin{align*}	
	\Delta F 
	&= \frac{\partial^2 F}{\partial x_1^2} + \cdots + \frac{\partial^2 F}{\partial x_n^2} \\
	&= \frac{nr^2 - (x_1^2+\cdots+x_n^2)}{r^3}  \cdot  f'(r) + \frac{x_1^2+\cdots+x_n^2}{r^2}  \cdot  f''(r) \\
	&=  \frac{nr^2 - r^2}{r^3}  \cdot  f'(r)  + \frac{r^2}{r^2} \cdot  f''(r) \\	
	&= \frac{n-1}{r}  \cdot  f'(r) + f''(r).	
	\end{align*}	
\end{enumerate}

Conclusion :
\[\Delta F = \frac{n-1}{r} f'(r) + f''(r).\]
\fincorrection

\finexercice





\exercice{2639, debievre, 2009/05/19}
\enonce[Laplacien en coordonnées polaires]
Soit $f\colon \R^2\setminus\{(0,0)\}\longrightarrow \R$ 
une fonction
de classe $\mathcal{C}^2$ et soient $r$ et $\theta$ les coordonnées
polaires standard dans le plan de telle sorte que
l'association
\[
\Phi : {}]0,+\infty[ \times  [0,2\pi[ \longrightarrow  \R^2\setminus\{(0,0)\},
\quad
(r,\theta)\longmapsto (x,y)=(r\cos\theta, r\sin\theta),
\]
soit un changement de variables.
Soit $F$ la fonction définie par 
\[
F(r,\theta)=f(r\cos\theta, r\sin\theta).
\]
C'est \og{}l'expression de $f$ 
en coordonnées polaires\fg{}. 
Le but de l'exercice est de prouver :
\begin{equation*}
	\frac{\partial^2 f}{\partial x^2}(x,y)+\frac{\partial^2 f}{\partial y^2}(x,y)=
	\frac{\partial^2F}{\partial r^2}(r,\theta)+\frac{1}{r}\frac{\partial F}{\partial r}(r,\theta)+\frac{1}{r^2}\frac{\partial^2 F}{\partial \theta^2}(r,\theta).
	\label{laplace}
\end{equation*}

Cette formule s'appelle l'expression du \og{}laplacien en coordonnées polaires\fg{}. 

\begin{enumerate}
	\item Calculer les dérivées partielles de $F$ par rapport à $r$ et $\theta$ en fonction des dérivées partielles de $f$ par rapport à $x$ et $y$ et obtenir les identités d'opérateurs :
	\[
	\frac{\partial}{\partial r} = \frac xr \cdot  \frac{\partial }{\partial x}+ \frac yr \cdot  \frac{\partial }{\partial y}
	\qquad \text{ et } \qquad 
	\frac{\partial }{\partial \theta} = -y  \cdot  \frac{\partial }{\partial x}+x \cdot  \frac{\partial }{\partial y}.
	\]
	
	\item Calculer $\frac{\partial^2 F}{\partial r^2}$ en dérivant $\frac{\partial F}{\partial r}$  par rapport à $r$ et en utilisant que $x/r$ et $y/r$ ne dépendent pas de $r$.
	
	\item Calculer $\frac{\partial^2 F}{\partial \theta^2}$ en dérivant $\frac{\partial F}{\partial \theta}$  par rapport à $\theta$ et en montrant que $\frac{\partial x}{\partial \theta} = -y$ et $\frac{\partial y}{\partial \theta} = x$.
	
	\item Conclure.
\end{enumerate}

\finenonce

\indication
L'exercice ne dépend pas de la connaissance du Laplacien.
\begin{enumerate}
	\item On a $F = f \circ \Phi$, puis appliquer la formule \og{}$J_F = J_f \times J_\Phi$\fg{}.
	
	\item Il s'agir d’appliquer $\frac{\partial}{\partial r}$ à $\frac{\partial F}{\partial r} = \frac xr\frac{\partial f}{\partial x}+ \frac yr\frac{\partial f}{\partial y}$.
	
	\item Il s'agir d’appliquer $\frac{\partial}{\partial \theta}$ à $\frac{\partial F}{\partial \theta} = -y \frac{\partial f}{\partial x}+x\frac{\partial f}{\partial y}$.
\end{enumerate}

\finindication

\correction
\begin{enumerate}
	\item On a $F = f \circ \Phi$. 
	\[
	J_F = \begin{pmatrix} \frac{\partial F}{\partial r} & \frac{\partial F}{\partial \theta} \end{pmatrix}
	\qquad
	J_f = \begin{pmatrix} \frac{\partial f}{\partial x} & \frac{\partial f}{\partial y}  \end{pmatrix}
	\qquad
	J_\Phi
	= \begin{pmatrix}
		\cos\theta & -r\sin\theta \\
		\sin\theta & r\cos\theta
	\end{pmatrix}.
	\]
	
	Par la formule \og{}$J_F = J_f \times J_\Phi$\fg{}, on obtient :
	\[
	\frac{\partial F}{\partial r} = \cos\theta \cdot  \frac{\partial f}{\partial x}  + \sin\theta \cdot  \frac{\partial f}{\partial y} = \frac xr\frac{\partial f}{\partial x}+ \frac yr\frac{\partial f}{\partial y}.
	\]
	Ainsi on passe de l'opérateur $\frac{\partial }{\partial r}$ aux opérateurs $\frac{\partial }{\partial x}$, $\frac{\partial }{\partial y}$ par l'identité :
	\[
	\frac{\partial}{\partial r} = \frac xr\frac{\partial }{\partial x}+ \frac yr\frac{\partial }{\partial y}
	\]

	
	Et on obtient aussi :
	\[
	\frac{\partial F}{\partial \theta} = -r\sin\theta\frac{\partial f}{\partial x}+ r\cos\theta\frac{\partial f}{\partial y} = -y \frac{\partial f}{\partial x}+x\frac{\partial f}{\partial y},
	\]	
	d'où :	
	\[
	\frac{\partial }{\partial \theta} = -y \frac{\partial }{\partial x}+x\frac{\partial }{\partial y}.
	\]
	
	\item Pour calculer $\frac{\partial^2 F}{\partial r^2}$, on part de $\frac{\partial F}{\partial r} =  \frac xr f_x + \frac yr f_y$, et on dérive cette expression de nouveau par rapport à $r$.
	\begin{align*}
	\frac{\partial^2 F}{\partial r^2}
	&= 	\frac{\partial }{\partial r} \frac{\partial F}{\partial r} \\
	&= \frac{\partial }{\partial r} \left( \frac xr f_x + \frac yr f_y \right) \\
	&= \frac xr \frac{\partial f_x}{\partial r} + \frac yr \frac{\partial f_y}{\partial r} \quad \text{ car $\frac xr=\cos\theta$ et $\frac yr=\sin\theta$ ne dépendent pas de $r$}\\
	&= \frac xr \left( \frac xr \frac{\partial f_x}{\partial x}+ \frac yr\frac{\partial f_x}{\partial y} \right) +  \frac yr \left( \frac xr \frac{\partial f_y}{\partial x}+ \frac yr\frac{\partial f_y}{\partial y} \right) \\
	&= \frac{1}{r^2} \left( x^2 f_{xx} + 2xy f_{xy} + y^2 f_{yy} \right) 
	\end{align*}
	
	
	\item 
	Commençons par remarquer que :
	\[\frac{\partial x}{\partial \theta} = \frac{\partial r\cos\theta}{\partial \theta} = -r\sin\theta = -y\]
	et 
	\[\frac{\partial y}{\partial \theta} = \frac{\partial r\sin\theta}{\partial \theta} = r\cos\theta = x.\]
	
	
	
	Pour calculer $\frac{\partial^2 F}{\partial \theta^2}$, on part de $\frac{\partial F}{\partial \theta} = -y f_x+x f_y$, et on dérive cette expression par rapport à $\theta$.
	\begin{align*}
		\frac{\partial^2 F}{\partial \theta^2}
		&= 	\frac{\partial }{\partial \theta} \frac{\partial F}{\partial \theta} \\
		&= \frac{\partial }{\partial \theta} \left( -y f_x + x f_y  \right) \\
		&= \left( -\frac{\partial y}{\partial \theta} f_x  -y\frac{\partial f_x}{\partial \theta} \right) + 
		\left( \frac{\partial x}{\partial \theta} f_y  + x\frac{\partial f_y}{\partial \theta} \right) \\
		&= \left( -x f_x  -y\left( -y \frac{\partial f_x}{\partial x}+x\frac{\partial f_x}{\partial y} \right)\right) + 
\left( -y f_y  + x\left( -y \frac{\partial f_y}{\partial x}+x\frac{\partial f_y}{\partial y} \right) \right) \\		
		&= -x f_x -y f_y + x^2 f_{yy} - 2xyf_{xy} + y^2 f_{xx}
	\end{align*}
	
	

	
	\item Il s'agit de faire une somme à partir des trois égalités obtenues :
	\begin{align*}	
	r\frac{\partial F}{\partial r} + r^2 \frac{\partial^2F}{\partial r^2} + \frac{\partial^2 F}{\partial \theta^2} \\
	&= 
	xf_x + yf_y \\
	&\qquad + x^2 f_{xx} + 2xy f_{xy} + y^2 f_{yy} \\
	&\qquad \qquad -x f_x -y f_y + x^2 f_{yy} - 2xyf_{xy} + y^2 f_{xx} \\
	&= (x^2+y^2)(f_{xx}+f_{yy}) \\
	&= r^2 (f_{xx}+f_{yy})
	\end{align*}
	D'où l'égalité recherchée.
	
\end{enumerate}
\fincorrection
\finexercice


\bigskip

Corrections : Arnaud Bodin, Stephan de Bièvre. Relecture : Axel Renard.

\end{document}

