%%%%%%%%%%%%%%%%%% PREAMBULE %%%%%%%%%%%%%%%%%%

\documentclass[11pt,a4paper]{article}

\usepackage[utf8]{inputenc}
\usepackage[T1]{fontenc}
\usepackage[french]{babel}
\usepackage{amsfonts,amsmath,amssymb,amsthm}
\usepackage{fancybox}
\usepackage{graphicx}
%\usepackage{tikz}

%----- Ensembles : entiers, reels, complexes -----
\newcommand{\Nn}{\mathbb{N}} \newcommand{\N}{\mathbb{N}}
\newcommand{\Zz}{\mathbb{Z}} \newcommand{\Z}{\mathbb{Z}}
\newcommand{\Qq}{\mathbb{Q}} \newcommand{\Q}{\mathbb{Q}}
\newcommand{\Rr}{\mathbb{R}} \newcommand{\R}{\mathbb{R}}
\newcommand{\Cc}{\mathbb{C}} \newcommand{\C}{\mathbb{C}}
\newcommand{\Kk}{\mathbb{K}} \newcommand{\K}{\mathbb{K}}

%----- Modifications de symboles -----
\renewcommand{\epsilon}{\varepsilon}
\renewcommand{\Re}{\mathop{\mathrm{Re}}\nolimits}
\renewcommand{\Im}{\mathop{\mathrm{Im}}\nolimits}
\newcommand{\llbracket}{\left[\kern-0.15em\left[}
\newcommand{\rrbracket}{\right]\kern-0.15em\right]}
\renewcommand{\ge}{\geqslant} \renewcommand{\geq}{\geqslant}
\renewcommand{\le}{\leqslant} \renewcommand{\leq}{\leqslant}

%----- Fonctions usuelles -----
\newcommand{\ch}{\mathop{\mathrm{ch}}\nolimits}
\newcommand{\sh}{\mathop{\mathrm{sh}}\nolimits}
\renewcommand{\tanh}{\mathop{\mathrm{th}}\nolimits}
\newcommand{\cotan}{\mathop{\mathrm{cotan}}\nolimits}
\newcommand{\Arcsin}{\mathop{\mathrm{arcsin}}\nolimits}
\newcommand{\Arccos}{\mathop{\mathrm{arccos}}\nolimits}
\newcommand{\Arctan}{\mathop{\mathrm{arctan}}\nolimits}
\newcommand{\Argsh}{\mathop{\mathrm{argsh}}\nolimits}
\newcommand{\Argch}{\mathop{\mathrm{argch}}\nolimits}
\newcommand{\Argth}{\mathop{\mathrm{argth}}\nolimits}
\newcommand{\pgcd}{\mathop{\mathrm{pgcd}}\nolimits} 

%----- Structure des exercices ------
%\theoremstyle{definition}

\newtheoremstyle{exostyle}
{15pt}
{15pt}
{\normalfont}
{0pt}
{\bfseries}
{}
{\newline}
{\thmname{#1}~\thmnumber{#2} -- \thmnote{#3}}
\theoremstyle{exostyle} 

\newtheorem{exo}{Exercice}
\newtheorem{ind}{Indications}
\newtheorem{cor}{Correction}

\newcommand{\exercice}[1]{} \newcommand{\finexercice}{}
%\newcommand{\exercice}[1]{{\tiny\texttt{#1}}\vspace{-2ex}} % pour afficher le numero absolu, l'auteur...
\newcommand{\enonce}{\begin{exo}} \newcommand{\finenonce}{\end{exo}}
\newcommand{\indication}{\begin{ind}} \newcommand{\finindication}{\end{ind}}
\newcommand{\correction}{\begin{cor}} \newcommand{\fincorrection}{\end{cor}}

\newcommand{\noindication}{\stepcounter{ind}}
\newcommand{\nocorrection}{\stepcounter{cor}}

\newcommand{\fiche}[1]{} \newcommand{\finfiche}{}
\newcommand{\titre}[1]{\centerline{\large \bf #1}}
\newcommand{\addcommand}[1]{}
\newcommand{\video}[1]{}

%----- Presentation ------
\setlength{\parindent}{0cm}

\newcommand{\ExoSept}{\textbf{\textsf{Exo7}}}
\newcommand{\LogoExoSept}{\setlength{\unitlength}{0.6em}
	\begin{picture}(0,0)  \thicklines     \put(0,4){\line(0,1){3}}   \put(0,7){\line(1,0){3}}
		\put(3,7){\line(0,-1){7}}  \put(0,4){\line(1,0){7}}   \put(3,0){\line(1,0){4}}
		\put(7,0){\line(0,1){4}}   \put(3,7){\line(4,-3){4}}  \put(7,4){\line(3,4){3}}  
		\put(10,8){\line(-4,3){4}} \put(3,7){\line(3,4){3}}   \put(4.6,6.8){\mbox{\ExoSept}}
\end{picture}}

%----- Commandes supplementaires ------

\usepackage[a4paper, margin = 2cm]{geometry}
\usepackage[charter]{mathdesign}
%\usepackage{import}

\usepackage{tikz}
\usetikzlibrary{calc,shadows,arrows,shapes,patterns,matrix}
\usetikzlibrary{decorations.pathmorphing}
\usetikzlibrary{fadings}
\usetikzlibrary{external}
\usetikzlibrary{positioning}
\usetikzlibrary{arrows}
\usetikzlibrary{backgrounds}
\usepackage{tikz,tikz-3dplot}

\newcommand{\sauteligne}{\leavevmode\vspace{-\baselineskip}}
\newcommand{\grad}{\mathop{\mathrm{grad}}\nolimits} 

\begin{document}

	
	
%%%%%%%%%%%%%%%%%% ENTETE %%%%%%%%%%%%%%%%%%


%\kern-2em
\textsc{Feuille d'exercices 5}\hfill\textsc{Fonctions de plusieurs variables}

\vspace*{0.5ex}
\hrule\vspace*{1.5ex} 
\hfil{\textbf{\Large \textsc{Gradient -- Théorème des accroissements finis}}}
\vspace*{1ex} \hrule 
\vspace*{5ex} 




%===========================================
\section{Tangentes et plans tangents}

\exercice{}
\enonce[Tangentes]
\sauteligne
\begin{center}
	\begin{minipage}{0.4\textwidth}	
		Soit $\mathcal{C}$ la courbe définie par l'équation :
		\begin{equation*}
			y^2 - x^2 (x + 1) = 0.
		\end{equation*}	
	\end{minipage} 	
	\begin{minipage}{0.5\textwidth}	
		\center	
		\begin{tikzpicture}[scale=2]
			\draw[->,>=latex] (-1.2, 0) -- (2, 0) node[below] {$x$};
			\draw[->,>=latex] (0, -1.5) -- (0, 1.5) node[left] {$y$};
			\draw[very thick,  domain=-1:1, smooth, variable=\x, blue,smooth,samples=100] plot ({\x}, {sqrt(\x*\x*(\x+1))});
			\draw[very thick, domain=-1:1, smooth, variable=\x, blue,smooth,samples=100] plot ({\x}, {-sqrt(\x*\x*(\x+1))});
			
		\end{tikzpicture}
	\end{minipage}  
\end{center}
\begin{enumerate}
	\item Déterminer les points où les dérivées partielles de $f(x,y)=y^2-x^2(x+1)$ s'annulent simultanément. Est-ce que ces points appartiennent à $\mathcal{C}$ ?
	Ces points seront exclus dans la suite de l'exercice.
	\item Calculer l'équation de la tangente en un point de $\mathcal{C}$.
	\item Pour quels points la tangente est-elle horizontale ? Verticale ?
\end{enumerate}
\finenonce

\noindication

\correction
\begin{enumerate}
	\item 
	\[
	\frac{\partial f}{\partial x}(x,y) = -3x^2 - 2x \qquad
	\frac{\partial f}{\partial y}(x,y) = 2y.
	\]
	
	Si en $(x,y)$ les deux dérivées partielles s'annulent, alors d'une part $2y=0$ donc $y=0$ et d'autre part $-3x^2 - 2x = -x(3x + 2) = 0$ donc $x=0$ ou $x=-\frac23$.
	
	Ainsi $f$ admet deux points critiques $(0,0)$ et $(-\frac23,0)$.
	
	On a $f(0,0)=0$ donc $(0,0) \in \mathcal{C}$, par contre $f (-\frac23,0) \neq 0$ donc $(-\frac23,0)  \notin \mathcal{C}$.
	
	\item On fixe $(x_0,y_0) \in \mathcal{C} \setminus \{ (0,0) \}$.
	L'équation de la tangente en ce point est :
	\[
	\frac{\partial f}{\partial x}(x_0,y_0)(x-x_0)+\frac{\partial f}{\partial y}(x_0,y_0)(y-y_0)=0.
	\]
	Pour notre fonction $f$ cela donne :
	\[
	(-3x_0^2 - 2x_0)(x - x_0) + 2y_0(y - y_0) = 0,
	\]
	ou encore :
	\[
	(-3x_0^2 - 2x_0)x + 2y_0 y +3x_0^3 + 2x_0^2-2y_0^2 = 0.
	\]
	On peut encore simplifier un peu le terme constant en utilisant la relation $f(x_0,y_0)=0$ :
	\[
	-x_0(3x_0+2)x+2y_0y+x_0^3=0.
	\]
	
	
	\item 
	Rappelons que l'équation de la tangente en $(x_0,y_0)$ est :
	\[
	\frac{\partial f}{\partial x}(x_0,y_0)(x-x_0)+\frac{\partial f}{\partial y}(x_0,y_0)(y-y_0)=0.
	\]
	Une droite horizontale a une équation du type $y=\mathrm{cst}$, donc la tangente est horizontale si et seulement si $\frac{\partial f}{\partial x}(x_0,y_0) = 0$.
	Or 
	\[
	\frac{\partial f}{\partial x}(x,y) = 0
	\iff -3x^2 - 2x = 0 
	\iff x=0 \text{ ou } x=-\frac23.
	\]
	Si $x_0=0$ alors, comme $f(x_0,y_0)=0$, on obtient $y_0=0$, or le point $(0,0)$ est exclu de l'étude (la tangente n'y est pas bien définie).
	Si $x_0=-\frac23$, alors $f(x_0,y_0)=0$ implique $y_0^2-\frac{4}{27}=0$, donc $y_0=\pm\frac{2}{3\sqrt3}$.
	Bilan : il y a deux points où la tangente est horizontale : $(-\frac23,-\frac{2}{3\sqrt3})$, $(-\frac23,+\frac{2}{3\sqrt3})$.
	
	\medskip
	
	Une droite verticale a une équation du type $x=\mathrm{cst}$, donc la tangente est verticale si et seulement si $\frac{\partial f}{\partial y}(x_0,y_0) = 0$.
	Or $\frac{\partial f}{\partial y}(x,y) = 0$ si et seulement si $y=0$.
	Si $y_0=0$ alors la relation $f(x_0,y_0)=0$ implique $x_0=0$ ou $x_0=-1$.
	On a exclu $(0,0)$, donc la tangente est verticale à $\mathcal{C}$ uniquement au point $(-1,0)$.
\end{enumerate}
\fincorrection

\finexercice





% Cette fiche contient les exercices : 2628 2629 2630 2631 2632 2634 2652 

\exercice{2628, debievre, 2009/05/19}
\enonce[Plans tangents]
\sauteligne
\begin{enumerate}
	\item Trouver l'équation du plan tangent à la surface d'équation $z = \sin(\pi xy) \exp(2x^2y-1)$ au point $(1,\frac12,1)$.
	\item Trouver l'équation du plan tangent à la surface d'équation $z = \sqrt{19-x^2-y^2}$ au point $(1,3,3)$.	
    \item Trouver les points sur le paraboloïde d'équation $z=4x^2 +y^2$ où le plan tangent est parallèle au plan d'équation $x + 2y + z = 6$.
    % Même question avec le plan d'équation $3x + 5y - 2z = 3$. 
\end{enumerate}
\finenonce

\indication
Le plan tangent à la surface d'équation  $z = f(x, y)$ au point $(x_0, y_0, f(x_0,y_0))$  est donné par l'équation :
\begin{equation*} \label{tang1}
	z = f(x_0, y_0) + \frac{\partial f}{\partial x}(x_0, y_0) (x - x_0) + \frac{\partial f}{\partial y}(x_0, y_0) (y - y_0).
\end{equation*}
\finindication

\correction
On répond à la dernière question.

Le plan tangent à la surface d'équation  $z = f(x, y)$ au point $(x_0, y_0, f(x_0,y_0))$  est donné par l'équation :
\begin{equation*} \label{tang1}
	z = f(x_0, y_0) + \frac{\partial f}{\partial x}(x_0, y_0) (x - x_0) + \frac{\partial f}{\partial y}(x_0, y_0) (y - y_0).
\end{equation*}
Ainsi, le plan tangent à la surface d'équation 
$z=4x^2 +y^2$ au point $(x_0,y_0,z_0)$  a pour équation :
\begin{align*}
	     & z =(4x_0^2 + y_0^2) + 8x_0 (x-x_0) + 2y_0 (y-y_0)  \\
	\iff & z= 8x_0x +2y_0 y - (4x_0^2 + y_0^2).
\end{align*}
Donc l'équation est :
\begin{equation*} \label{tang2}
	8x_0 x + 2y_0 y - z = 4x_0^2 + y_0^2.
\end{equation*}
Pour que ce plan soit parallèle au plan d'équation $x + 2y + z = 6$,
il faut et il suffit que leurs vecteurs normaux $(8x_0, 2y_0, -1)$ et $(1, 2, 1)$ soient colinéaires. Le facteur de colinéarité est $\lambda=-1$, 
donc $x_0=-\frac18$ et $y_0=-1$.
Par conséquent, le point recherché sur le paraboloïde est le point $(-\frac18,-1,\frac{17}{16})$.

%De même, pour que le plan d'équation \eqref{tang2} soit parallèle au plan d'équation $3x + 5y - 2z = 3$,
%il faut et il suffit que $(8x_0, 2y_0, 1)$ et $(\frac32, \frac52, 1)$ soient colinéaires,
%donc que $x_0 = \frac3{16}$ et $y_0=\frac54$.
%Par conséquent, le point recherché sur la paraboloïde est le point $(\frac3{16}, \frac54, \frac{109}{64})$.
\fincorrection
\finexercice



\exercice{2629, debievre, 2009/05/19}
\enonce[Y'a comme un problème]
On demande à un étudiant de trouver l'équation du plan tangent à la surface d'équation
$z = x^4 - y^2$ au point $(2, 3, 7)$.
Sa réponse est
\begin{equation*}
	z = 4x^3 (x - 2) - 2y(y - 3).
\end{equation*}
\begin{enumerate}
	\item Expliquer, sans calcul, pourquoi cela ne peut en aucun  cas être la bonne réponse.
	\item Quelle est l'erreur commise par l'étudiant ?
	\item Donner la réponse correcte.
\end{enumerate}
\finenonce

\indication 
Ne pas confondre les variables pour l'équation de la surface,
les variables pour l'équation de la tangente en un point,
et les coordonnées du point de contact.
\finindication

\correction
\begin{enumerate} 
	\item L'équation d'un plan tangent doit être une équation linéaire en $x$, $y$ et $z$ !
	
	En plus le point $(2,3,7)$ ne vérifie pas l'équation proposée.
	
	\item Il a confondu les coordonnées du point de contact et les variables de l'équation du plan.
	\item L'équation du plan tangent à la surface $f(x,y,z)=k$ en $(x_0,y_0,z_0)$ est :
	\[
	\frac{\partial f}{\partial x}(x_0,y_0,z_0)(x-x_0)+\frac{\partial f}{\partial y}(x_0,y_0,z_0)(y-y_0)
	+\frac{\partial f}{\partial z}(x_0,y_0,z_0)(z-z_0)=0.
	\]
	Ici $f(x,y,z) = x^4-y^2-z$, on vérifie d'abord que $f(2,3,7)=0$. 
	Ensuite le plan tangent en $(2, 3, 7)$ a pour équation :
	\begin{align*}
		 &\frac{\partial f}{\partial x}(2,3,7) (x - 2) + \frac{\partial f}{\partial y}(2,3,7) (y - 3) + \frac{\partial f}{\partial z}(2,3,7) (z - 7) =0\\
		& \iff 32(x - 2) - 6(y - 3) -z + 7 = 0 \\
		&\iff z = 32x -6y - 39
	\end{align*}
	
	
	On aurait aussi pu considérer que la surface est le graphe de la fonction de deux variable $g(x,y) = x^4 - y^2$ et appliquer la formule adaptée :
	\[
	z = g(x_0, y_0) + \frac{\partial g}{\partial x}(x_0, y_0)(x - x_0) + \frac{\partial g}{\partial y}(x_0, y_0)(y - y_0).
	\]  
\end{enumerate}
\fincorrection
\finexercice



\exercice{2631, debievre, 2009/05/19}
\enonce[Cône]
Soit $\mathcal{C}$ le cône d'équation $z^2 = x^2 + y^2$.
On note $\mathcal{P}_{M_0}$ le plan tangent au cône $\mathcal{C}$ en $M_0 \in \mathcal{C} \setminus \{(0, 0, 0)\}$.
\begin{enumerate}
	\item  Déterminer un vecteur normal et l'équation du plan tangent $\mathcal{P}_{M_0}$
	en un point $M_0(x_0,y_0,z_0)$ du cône autre que l'origine.
	\item Déterminer les autres points du cône ayant le même plan tangent que $\mathcal{P}_{M_0}$.
\end{enumerate}
\finenonce

\indication
Un vecteur normal de la surface d'équation $f(x, y, z) = 0$ au point $(x_0, y_0, z_0)$  est le vecteur gradient en ce point.
\finindication

\correction
Un vecteur normal de la surface d'équation $f(x, y, z) = 0$ au point $(x_0, y_0, z_0)$  est le vecteur gradient :
\begin{equation*} \label{normal}
	\left(
	\frac{\partial f}{\partial x}(x_0, y_0, z_0),
	\frac{\partial f}{\partial y}(x_0, y_0, z_0),
	\frac{\partial f}{\partial z}(x_0, y_0, z_0)
	\right).
\end{equation*}

\begin{enumerate} 
	\item  Un vecteur normal du cône $\mathcal{C}$ au point $(x_0, y_0, z_0)$ de $\mathcal{C} \setminus \{(0, 0, 0)\}$
	est le vecteur $(x_0, y_0, -z_0)$ et le plan tangent au cône $\mathcal{C}$ en ce point est donné par l'équation :
	\begin{equation*}
		x_0 (x - x_0) + y_0 (y - y_0) - z_0 (z - z_0) = 0,
	\end{equation*}
	c'est-à-dire :
	\begin{equation*}
		x_0 x + y_0 y - z_0 z =  x_0^2 + y_0^2 - z_0^2  = 0 \quad \text{car } M_0 \in \mathcal{C}.
	\end{equation*}
	
	\item Pour que $M'_0 = (x'_0, y'_0, z'_0) \in \mathcal{C} \setminus \{(0, 0, 0)\}$ vérifie $\mathcal{P}_{M'_0} = \mathcal{P}_{M_0}$,
	il faut et il suffit que les vecteurs $(x_0, y_0, -z_0)$ et $(x'_0, y'_0, -z'_0)$ soient colinéaires,
	donc que $(x_0, y_0, z_0)$ et $(x'_0, y'_0, z'_0)$ soient colinéaires.
	On en conclut que l'ensemble des points du cône ayant le même plan tangent que $\mathcal{P}_{M_0}$ est constitué de la droite $(OM_0)$ privée du point $O$.
\end{enumerate}
\fincorrection
\finexercice





%===========================================
\section{Approximations -- Théorème des accroissements finis}

\exercice{2634, debievre, 2009/05/19}  % + 2652
\enonce[Approximations]
Utiliser une approximation affine bien choisie pour calculer une valeur approchée des nombres suivants :
\begin{equation*}
\exp[\sin(3.16)\cos(0.02)],
\qquad
\arctan[\sqrt{4.03}-2\exp(0.01)],
\qquad
\exp[-0.02\sqrt{4.03}].
\end{equation*}
\finenonce

\indication
On prend
\begin{align*}
	f(x, y) &= \exp[\sin(\pi + x) \cos(y)] = \exp[-\sin(x) \cos(y)],\ (x_0, y_0) = (0, 0),\ (h, k) = (0.02 , 0.02) \\
	f(x, y) &= \arctan\big[\sqrt{4 + x} - 2 \exp(y)\big],\ (x_0, y_0) = (4, 0),\ (h, k) = (0.03 , 0.01) \\
	f(x, y) &= \exp[-x \sqrt{y}],\ (x_0, y_0) = (0, 4),\ (h, k) = (0.02 , 0.03).
\end{align*}
\finindication

\correction 
La formule de l'approximation affine à l'ordre $1$ (DL1) est : 
\[
f(x_0+h,y_0+k) \simeq f(x_0,y_0) + h \cdot \frac{\partial f}{\partial x}(x_0,y_0)
+k \cdot \frac{\partial f}{\partial y}(x_0,y_0).
\]
\begin{enumerate}
	
  \item Pour la première approximation on considère $3.16 \simeq \pi + 0.02$.
On considère donc :
\[f(x, y) = \exp[\sin(\pi + x) \cos(y)] = \exp[-\sin(x) \cos(y)], \quad (x_0, y_0) = (0, 0),\quad  (h, k) = (0.02 , 0.02).\]

On calcule :
\begin{align*}
	f(x, y) &=\exp[\sin(\pi + x) \cdot \cos(y)]= \exp[-\sin (x) \cdot \cos (y)] \quad \implies \quad f(0, 0) = 1, \\
	\frac{\partial f}{\partial x}(x, y) &= -\cos x \cdot\cos y\cdot \exp[-\sin x \cdot \cos y] \quad \implies \quad \frac{\partial f}{\partial x}(0, 0) = -1, \\
	\frac{\partial f}{\partial y}(x, y) &= \sin x \cdot \sin y \cdot \exp[-\sin x \cdot \cos y] \quad \implies \quad \frac{\partial f}{\partial y}(0, 0) = 0.
\end{align*}
L'approximation affine de $f$ au voisinage de $(0, 0)$ s'écrit donc
\begin{equation*}
f(h, k) \simeq 1 - h.
\end{equation*}
Avec $h = k = 0.02$ on trouve $f(0.02, 0.02) \simeq 0.98$.

%N.B. On peut faire mieux si nécessaire : Avec
%\begin{align*}
%	\frac{\partial^2 f}{\partial x^2}&= (\sin x \cos y
%	+\cos^2 x \cos^2 y)\exp[-\sin x \cos y ]
%	\\
%	\frac{\partial^2 f}{\partial x \partial y}&
%	= (\cos x \sin y+\cos x \cos y \sin x \sin y)\exp[-\sin x \cos y ]
%	\\
%	\frac{\partial^2 f}{\partial y^2}&=(\sin x \cos y
%	+\sin^2 x \sin^2 y)\exp[-\sin x \cos y ]
%\end{align*}
%on trouve
%\[
%f(x,y)
%=
%f(0,0)+\frac{\partial f}{\partial x}(0,0)x+\frac{\partial f}{\partial y}(0,0)y
%+\ldots
%=1-x++\tfrac 12 x^2 +\ldots
%\]
%etc.

  \item De même avec 
  \[
  f(x, y) = \arctan\big[\sqrt{4 + x} - 2 \exp(y)\big],\quad (x_0, y_0) = (0, 0),\quad (h, k) = (0.03 , 0.01).
  \]
\begin{align*}
	\frac{\partial f}{\partial x}&= \frac 1 {2(1+(\sqrt{4+x}-2\exp(y))^2)\sqrt{4+x}}
	\\
	\frac{\partial f}{\partial y}&= \frac {-2\exp(y)} {1+(\sqrt{4+x}-2\exp(y))^2}
\end{align*}
etc. d'o\`u,  avec $\frac{\partial f}{\partial x}(0,0)=\tfrac 14$ et
$\frac{\partial f}{\partial y}(0,0)=-2$,
\[
f(0+h,0+k)
=
f(0,0)+h\frac{\partial f}{\partial x}(0,0)+k\frac{\partial f}{\partial y}(0,0)
+\cdots
=\tfrac 14 h-2k +\cdots 
\]
Avec $h=0.03$ et $k=0.01$ on trouve, pour
$ \arctan[\sqrt{4.03}-2\exp(0.01)]$,
la valeur approch\'ee $0.0075-0.02 =-0.0125$. 


  \item
  \[
  f(x, y) = \exp[-x \sqrt{y}],\quad (x_0, y_0) = (0, 4),\quad (h, k) = (0.02 , 0.03)
  \]
  On trouve :
  \[f(0+h,4+k) \simeq 1 -2 \cdot h + 0\cdot k.\]
\end{enumerate}

  \item 
\fincorrection
\finexercice


\exercice{}
\enonce[Résistances]
Deux résistances $R_1$ et $R_2$ sont connectées en parallèle. La résistance totale $R$ du circuit est donnée par la formule 
$$\frac{1}{R} = \frac{1}{R_1} + \frac{1}{R_2}.$$

La résistance $R_1$ vaut environ $1$ ; $R_2$ vaut environ $2$ (en kilo-ohms).
\'Ecrire l'approximation linéaire correspondante, puis donner une valeur approchée de $R$ lorsque $R_1 = 1.2$ et $R_2 = 1.9$.
\finenonce
\noindication
\correction

Posons
\begin{equation*}
	f(x, y) = \frac1{\frac1x + \frac1y} = \frac{xy}{x + y},
\end{equation*}
de sorte que $R = f(R_1,R_2)$. Par exemple, si $R_1 = 1$ et $R_2 = 2$, on trouve $R = \frac23 \simeq 0.666$.


On calcule :
\begin{equation*}
\frac{\partial f}{\partial x}(x, y) = \frac{y^2}{(x + y)^2}
\quad
\text{et}
\quad
\frac{\partial f}{\partial y}(x, y) = \frac{x^2}{(x + y)^2}.
\end{equation*}

Posons $(x_0,y_0)=(1,2)$. On a
\begin{equation*}
f(x_0, y_0) = \frac23,
\quad
\frac{\partial f}{\partial x}(x_0, y_0) = \frac49
\quad
\text{et}
\quad
\frac{\partial f}{\partial y}(x_0, y_0) = \frac19.
\end{equation*}
L'approximation linéaire de $f$ au voisinage de $(x_0, y_0)$ s'écrit donc
\begin{equation*}
f(1+h, 2+k) \simeq \frac 23 + \frac49 h + \frac19 k.
\end{equation*}
Avec $h = 0.2$ et $k = -0.1$, on trouve $f(1.2 , 1.9) \simeq 0.744$.
\fincorrection

\finexercice


\exercice{}

\enonce[Théorème des accroissements finis]
Démontrer les résultats suivants énoncés dans le cours.
\begin{enumerate}
	\item Soit $f: U \to \Rr$ une fonction de classe $\mathcal{C}^1$ sur un ouvert convexe $U \subset \Rr^2$
	muni de la norme euclidienne.
	On suppose qu'il existe $k > 0$ tel que
	\begin{equation*}
		\forall c \in U,\ \|\grad f (c)\| \leq k.
	\end{equation*}
	Alors
	\begin{equation*}
		\forall a, b \in U,\ \left| f(b)-f(a)  \right| \le k \| b -a \|.
	\end{equation*}
	\item Soit $f: U \to \Rr$ une fonction de classe $\mathcal{C}^1$ sur un ouvert convexe $U \subset \Rr^2$.
	Si $\grad f (x, y) = (0,0)$ pour tout $(x, y) \in U$, alors $f$ est constante sur $U$.
	\item Trouver toutes les fonctions $f : \Rr^2 \to \Rr$ de classe $\mathcal{C}^1$ telles que
	\begin{equation*}
		\forall (x, y) \in \Rr^2,\ \grad f (x, y) = (3x^2 + 2y, 2x - 2y).
	\end{equation*} 	 	
\end{enumerate}
\finenonce

\indication
\begin{enumerate}
	\item Appliquer le théorème des accroissements finis en une variable à la fonction $g(t) = f\big( (1-t)a + tb \big)$.
	
	\item Appliquer la question précédente.
	
	\item Intégrer d'abord $\frac{\partial f}{\partial x}$. Attention à la \og{}constante\fg{} !
\end{enumerate}
\finindication

\correction
\begin{enumerate}
	\item Considérons $g : [0,1] \to \Rr$ définie par :
	\[ g(t) = f\big( (1-t)a + tb \big). \]
	On a $g(0)=f(a)$ et $g(1)=f(b)$.
	Comme $f$ est $\mathcal{C}^1$ alors $f$ est continue et dérivable.
	On peut donc appliquer le théorème des accroissements finis en une variable à la fonction $g$. Il existe $s \in {}]0,1[$ tel que :
	\[g(1)-g(0) = g'(s) (1-0).\]
	On a déjà dit que $g(0) =f(a)$ et $g(1)=f(b)$.
	
	Calculons $g'(t)$ comme la dérivée d'une composition. On peut refaire les calculs sur cet exemple ou bien utiliser la formule plus générale de la dérivée de $g(t) = f\big( x(t), y(t) \big)$ :
	\[
	g'(t) 
	= 	x'(t) \frac{\partial f}{\partial x} + y'(t)\frac{\partial f}{\partial y}
	= \left\langle \begin{pmatrix} x'(t) \\ y'(t) \end{pmatrix} \mid \grad f \big( x(t), y(t) \big)\right\rangle
	\]
	Appliqué à notre exemple et en notant $a=(a_1,a_2)$, $b=(b_1,b_2)$ :
	\[
	\begin{pmatrix}	x(t) \\ y(t) \end{pmatrix}
	= (1-t)a + tb 
	= \begin{pmatrix} (1-t)a_1+tb_1 \\ (1-t)a_2+tb_2 \end{pmatrix}
	\qquad
	\begin{pmatrix}	x'(t) \\ y'(t) \end{pmatrix}
	= \begin{pmatrix} -a_1+b_1 \\ -a_2+b_2 \end{pmatrix}
	= b-a
	\]
	Ainsi :
	\[
	g'(t) 
	= \left\langle b-a \mid \grad f((1-t)a + tb) \right\rangle 
	\]
	Donc par l'inégalité de Cauchy-Schwarz :
	\[ |g'(t)| \le \| b-a \| \cdot \| \grad f \| \le  k\| b-a \|.\]
	
	Ainsi l'égalité $g(1)-g(0) = g'(s)(1-0)$ implique l'inégalité :
	\[ \left| f(b)-f(a)  \right| \le k \| b -a \|. \]
	
	\item Soient $a,b$ deux points de $U$. Comme $U$ est convexe on a le segment $[a,b]$ contenu dans $U$. Comme le gradient est partout nul, on peut choisir $k=0$ comme constante dans la formule de la question précédente, ce qui donne immédiatement :
	\[| f(b) - f(a) | \le 0.\]
	Et donc $f(a)=f(b)$. Ainsi la valeur de $f$ en deux points quelconque de $U$ est toujours la même, c'est exactement dire que $f$ est une fonction constante.
	
	\item Dans la pratique on intègre par rapport à une variable, puis par rapport à l'autre, en prenant bien soin d'expliquer les constantes d'intégration.
	
	Intégrer d'abord $\frac{\partial f}{\partial x}$. 
	Comme $\frac{\partial f}{\partial x}(x,y) = 3x^2 + 2y$ alors :
	\[
	f(x,y) = x^3+2xy + C(y).
	\]
	$C$ est une constante pour la variable $x$, mais peut dépendre de la variable $y$, c'est donc une fonction de $y$.
	
	Repartant de cette expression, l'équation $\frac{\partial f}{\partial y}(x,y) = 2x - 2y$
	devient :
	\[
	2x + C'(y) = 2x-2y. \]
	Donc $C'(y) = -2y$ d'où $C(y) = -y^2+K$, où $K\in \Rr$.
	Conclusion : les solutions cherchées sont de la forme  $f(x,y) = x^3+2xy-y^2+K$ et on vérifie qu'elles conviennent, quel que soit $K\in\Rr$.
\end{enumerate}
\fincorrection
\finexercice

\bigskip

Corrections : Arnaud Bodin, Stephan de Bièvre. Relecture : Axel Renard.

\end{document}
