
\documentclass[12pt, class=report,crop=false]{standalone}
\usepackage[screen]{../exo7book}


\begin{document}

%====================================================================
\chapitre{Fonctions de plusieurs variables}
%====================================================================




%%%%%%%%%%%%%%%%%%%%%%%%%%%%%%%%%%%%%%%%%%%%%%%%%%%%%
\section{Introduction}

En première année vous avez étudié les fonctions d'une variable, par exemple si $t\mapsto f(t)$ représente l'évolution d'une population en fonction du temps, vous savez déterminer ses caractéristiques (croissance, maximum, limite,...). 
Mais de nombreux phénomènes dépendent de plusieurs paramètres, par exemple le volume d'un gaz dépend de la température et de la pression ou bien les coordonnées $(x,y,z)$ d'une point à la surface de la Terre dépendent de la latitude et de la longitude. Le but de ce cours est de faire le même travail que pour les fonctions d'une variables : étudiant la croissance, les maximums, les limites,... Bien sûr la situation est plus délicates, mais aussi plus intéressante, du fait qu'il y ait plusieurs variables !






+ [[image ligne de niveau]]

%----------------------------------------------------
\subsection{Que sont les fonctions de plusieurs variables ?}

Dans ce chapitre nous allons étudier les fonctions de plusieurs variables dans le cadre particulier de $\Rr^2$ ou $\Rr^3$, mais également dans le cadre général de $\Rr^n$. Ces fonctions seront donc de la forme 
\begin{equation*}
f:E \subset \Rr^n \rightarrow \Rr,
\end{equation*}
où $n$ est un entier naturel $\ge 1$. 

Autrement dit, les éléments de l'ensemble de départ $E$ seront des vecteurs du type $x=(x_1,\ldots,x_n)$, et les éléments de l'ensemble d'arrivée seront des réels.


\begin{exemple}
\sauteligne
\begin{enumerate}
\item $n=1$. $f: I \subset \R \rightarrow \R$:  c'est le cas le plus simple,
$x \mapsto f(x)$, celui qui est connu depuis le lycée. Voici les graphes des fonctions $x \mapsto x\cos x$ et $x \mapsto \arccos x$ :

\myfigure{1.2}{
  \tikzinput{fig-plusvar-11-01}\qquad
  \tikzinput{fig-plusvar-11-02}
} 

\item $n=2$. $f: E \subset \Rr^2 \rightarrow \R$. 
On préfère noter les variables par $(x,y)$ (au lieu de $(x_1,x_2)$).
Ces fonctions $(x,y) \mapsto f(x,y)$, sont représentées par exemple par des surfaces :

[[Figure :
Courbe représentant la fonction $ (x,y) \mapsto -x \cdot y\cdot e^{-x^2-y^2}$. ]]

[[Autres exemples : à faire]]
\end{enumerate}
\end{exemple}

Dès que $n>2$, il est assez difficile d'avoir une vision graphique. 
  
\bigskip

Nous allons aussi étudier des fonctions, dites fonctions vectorielles, dont l'ensemble d'arrivée n'est pas $\Rr$, mais $\Rr^p$, donc de la forme 
\begin{equation*}
f:E \subset \Rr^n \rightarrow \Rr^p,
\end{equation*}
où $n$ et $p$ sont des entiers naturels $\ge 1$. 

Dans ce cas, pour $x=(x_1,\ldots,x_n)$ un vecteur de $\Rr^n$, alors $f(x)$ est un vecteur de $\Rr^p$, du type $f(x)=(f_1(x_1,\ldots,x_n), \ldots,f_p(x_1,\ldots,x_n))$.


\begin{exemple}
\sauteligne
\begin{enumerate}
 
\item $n=1$, $p=2$. $f: I \subset \Rr \rightarrow \Rr^2$ 
est représentée par une courbe paramétrée du plan.

[[figure à faire]]
[[ dire $\{ (x(t),y(t) ) \}$ de nature différente, on est dans l'ensemble d'arrivée]]

\item $n=1$, $p=3$. $f: I \subset \Rr \rightarrow \Rr^3$ 
est représentée par une courbe paramétrée de l'espace.

[[  $\{ (x(t),y(t),z(t) ) \}$ 
[[figure à faire]]


    [(cas $p=1$, $q=3$) Courbe représentant la fonction $ t \mapsto ((2+\cos(1.5t))\cos(t),  (2+\cos(1.5t))\sin(t),\sin(1.5t))$ (nœud de trèfle). ]

+ [[hélice]]
 %       \includegraphics[width=3in]{trefoil.jpg}

        
        
\item $n=2$, $p=3$. $f: E \subset \Rr^2 \rightarrow \Rr^3$: elles sont représentées par exemple par des surfaces paramétrées.

    [(cas $p=2$, $q=3$) Courbe représentant la fonction $ (u,v) \mapsto \left( (2+\sin(v))\cos(u), (2+\sin(v))\sin(u),u+\cos(v) \right)$. ]
    
[[figure à faire, à un élément $(u,v)$ on asocie]]

\item $n=2$, $p = 2$. $f: E \subset \Rr^2 \rightarrow \Rr^2$: elles sont représentées par exemple par des champs de vecteurs.

[[détails, figure à faire]]

[[Représentation du champ de vecteur donné par 
  $(x,y,z) \mapsto (y / z,-x /  z,z / 4 )$.]]

\end{enumerate}
\end{exemple}


Dans ce chapitre nous étudions surtout les fonctions $f : \Rr^n \to \Rr$, les exemples étant des fonctions $f : \Rr^2 \to \Rr$ où $f : \Rr^3 \to \Rr$.

Nous reviendrons plus tard sur les fonctions vectorielles $f : \Rr^n \to \Rr^p$.
%----------------------------------------------------
\subsection{Topologie de $\Rr^n$ (rappels/compléments)}

Voici quelques rappels de topologie dans l'espace vectoriel $\Rr^n$. 

\begin{itemize}
  \item Le \defi{produit scalaire} usuel de $x=(x_1,\ldots ,x_n)$ et $y=(y_1,\ldots ,y_n)$, noté $\langle x \mid y\rangle$, ou bien $x \cdot y$, est défini par
$$\langle x \mid y\rangle = x_1y_1+\dots +x_ny_n.$$

  \item La \defi{norme euclidienne} sur $\Rr^n$ est la norme associée à ce produit scalaire. Pour $x\in \Rr^n$, la norme euclidienne de $x$, notée $\| x\|$, est définie par
$$\| x \| = \sqrt{\langle x \mid x\rangle} = \sqrt{x_1^2+\cdots +x_n^2}.$$

  \item La \defi{distance} entre le point $A$ et le point $M=(x_1,\dots ,x_n)$ est $\|M-A\|=\sqrt{(x_1-a_1)^2+\cdots +(x_n-a_n)^2}$.
  
  \item La \defi{boule ouverte} de centre $A = (a_1,\ldots ,a_n) \in \Rr^n$ et de rayon $r>0$, notée $B_r(A)$, est l'ensemble suivant :
$$B_r(A)=\left\{M\in \Rr^n \mid \|M-A\|<r \right\}.$$


  \item Soient $U$ une partie de $\Rr^n$ et $A\in U$.
On dit que $U$ est un \defi{voisinage} de $A$, si $U$ contient une boule ouverte centrée en $A$.

  \item On dit que $U$ est un \defi{ouvert} de $\Rr^n$, si tout pour tout point $A \in U$, $U$ contient une boule ouverte centrée en $A$.
\end{itemize}

Dans le cas de $\Rr^2$, on note plutôt les coordonnées d'un point par $(x,y)$, alors :
\begin{itemize}
  \item $\langle (x,y) \mid (x',y') \rangle = xx'+yy'$
  \item $\| (x,y) \| = \sqrt{x^2+y^2}$
  \item $B_r(x_0,y_0) = \left\{ (x,y) \in \Rr^2 \mid (x-x_0)^2+(y-y_0)^2 <r^2 \right\}$ (On parle de \defi{disque} plutôt que de boule.)
\end{itemize}

De gauche à droite : la distance, un disque ouvert, un ouvert $U$.

\myfigure{1}{
  \tikzinput{fig-plusvar-12-01}
  \tikzinput{fig-plusvar-12-02}
  \tikzinput{fig-plusvar-12-03}
} 

Exemples : 
\begin{itemize}
  \item tout rectangle ouvert $]a,b[\times ]c,d[$ est un ouvert de $\Rr^2$ (à droite sur la figure),
  \item tout disque ouvert  de $\Rr^2$ sont des ouverts de $\Rr^2$ (à gauche sur la figure).
\end{itemize}
\myfigure{1}{
  \tikzinput{fig-plusvar-12-04}
} 



%%%%%%%%%%%%%%%%%%%%%%%%%%%%%%%%%%%%%%%%%%%%%%%%%%%%%
\section{Graphe}

%----------------------------------------------------
\subsection{Fonctions }

\begin{definition}
Soit $E$ une partie $\Rr^n$. 
Une \defi{fonction} $f : E \to \Rr$ associe à tout 
$(x_1,\ldots,x_n)$ de $E$, un seul nombre réel $f(x_1,\ldots,x_n)$.
\end{definition}

\begin{exemple}
\sauteligne
\begin{enumerate}
\item Distance d'un point à l'origine en fonction de ses coordonnées :
$$\begin{array}{ccccl}f&:&\Rr^2&\longrightarrow &\Rr\\ &&(x,y)&\longmapsto &\sqrt{x^2+y^2}.\end{array}$$
\item Surface d'un rectangle en fonction de sa longueur et sa largeur :
$$\begin{array}{ccccl}f&:&\Rr^2&\longrightarrow &\Rr\\ &&(x,y)&\mapsto &xy.\end{array}$$
\item Surface d'un parallélépipède en fonction de ses trois dimensions :
$$\begin{array}{ccccl}f&:&\Rr^3&\longrightarrow &\Rr\\ &&(x,y,z)&\mapsto &2(xy+yz+xz).\end{array}$$
\end{enumerate}
\end{exemple}



\begin{definition}
Si on nous donne d'abord une expression pour $f(x_1,\ldots,x_n)$,  alors le \defi{domaine de définition} de $f$ est le plus grand sous-ensemble $D_f$, tel que pour chaque $(x_1,\ldots,x_n)$ de $D_f$, $f(x_1,\ldots,x_n)$ soit bien définie. La fonction est alors $f : D_f \to \Rr$.
\end{definition}


\begin{exemple}
\sauteligne
\begin{enumerate}
  \item $f(x,y) = \ln(1 + x + y)$

  Il faut que $1+x+y$ soit strictement positif, afin de pouvoir calculer son logarithme. Donc :
  $$D_f = \left\{ (x,y)\in \Rr^2 \mid 1+x+y >0 \right\}$$
  
  Pour tracer cet ensemble, on trace d'abord la droite  d'équation $1+x+y=0$. On détermine ensuite de quel côté de la droite est l'ensemble $1+x+y>0$. Ici c'est au-dessus de la droite.
  
  [[dessin]]
  
  \item $f(x,y) = \exp\left(\frac{x+y}{x-y^2}\right)$
  
  Le dénominateur ne doit pas s'annuler :  
  $$D_f = \left\{ (x,y)\in \Rr^2 \mid x-y^2 \neq 0 \right\}$$
  
  Les points de l'ensemble de définition, sont tous les points du plan qui ne sont pas sur la parabole d'équation $(x-y^2=0)$.
  
  [[dessin]]
    
  \item $f(x,y,z) = \frac {1}{\sqrt{x^2 + y^2 + z^2 - 2}}$
  
   L'expression sous la racine doit être positive (pour pouvoir prendre la racine carrée) et ne doit pas s'annuler (pour pouvoir prendre l'inverse), ainsi
  $$D_f = \left\{ (x,y,z)\in \Rr^2 \mid x^2+y^2+z^2 > 2 \right\}$$
  Autrement dit ce sont tous les points en dehors de la boule fermée, centrée en $(0,0,0)$ et de rayon $\sqrt 2$.
      
  [[dessin]]     
 
  
\end{enumerate}
\end{exemple}



\begin{definition}
L'\defi{image} d'une fonction $f : E \to \Rr$ est l'ensemble des valeurs $f(x_1,\ldots,x_n)$ prises par $f$ lorsque $(x_1,\ldots,x_n)$ parcourt $E$ :
$$\Im f = \big\{ f(x_1,\ldots,x_n) \mid (x_1,\ldots,x_n) \in E \big\} \subset \Rr$$
\end{definition}

\begin{exemple}
\sauteligne
\begin{enumerate}
  \item $f(x,y) = \ln(1 + x + y)$

  L'image de $f$ est $\Rr$ tout entier : $\Im f = \Rr$.
  
  Preuve : soit $z\in\Rr$. Alors pour $(x,y)=(e^z,-1)$, on a 
  $$f(x,y)= f(e^z,-1) = \ln(e^z)=z.$$ 
  Donc tout $z\in\Rr$ est dans l'image de $f$.
  
  \item $f(x,y) = \exp\left(\frac{x+y}{x-y^2}\right)$
 
  $\Im f = ]0,+\infty[$.
  
  Preuve : soit $z\in ]0,+\infty[$. 
  Alors pour $(x,y)=(0,-\tfrac{1}{\ln z})$, on a 
  $$f(x,y)= f(0,-\tfrac{1}{\ln z}) = \exp\left(\frac{-\tfrac{1}{\ln z}}{-(\tfrac{1}{\ln z})^2}\right)= \exp(\ln z) = z.$$ 
    
  \item Pour $f(x,y,z) = \frac {1}{\sqrt{x^2 + y^2 + z^2 - 2}}$
  alors $\Im f = ]0,+ \infty[$.
  À vous de faire la preuve !
 
  
\end{enumerate}
\end{exemple}


\begin{definition}
Soit $E$ une partie $\Rr^n$. 
Une \defi{fonction vectorielle} $f : E \to \Rr^p$ associe à tout 
$(x_1,\ldots,x_n)$ de $E$, une suite de $p$ nombres réels.
On la note 
$$\begin{array}{ccccl}f&:&\Rr^n&\longrightarrow &\Rr^p\\ &&x&\longmapsto &\left(f_1(x),f_2(x),\ldots ,f_p(x)\right).\end{array}$$
\end{definition}


Nous nous limiterons souvent aux dimensions $\le 3$ pour $n$ et $p$.
La généralisation aux dimensions supérieures ne posant pas de problème particulier, sauf pour faire des dessins ! Voici quelques exemples simples.

\begin{exemple}
\sauteligne
\begin{enumerate}
\item Surface et volume d'un parallélépipède en fonction de ses trois dimensions :
$$\begin{array}{ccccl}f&:&\Rr^3& \longrightarrow &\Rr^2\\ &&(x,y,z)&\longmapsto &\big(2(xy+yz+xz),xyz\big).\end{array}$$

\item Coordonnées polaires d'un point du plan :
$$\begin{array}{ccccl}f&:&\Rr^+\times [0,2\pi[&\longrightarrow &\Rr^2\\ &&(r,\theta)&\longmapsto &(r\cos \theta ,r\sin \theta ).\end{array}$$
\end{enumerate}
\end{exemple}



%----------------------------------------------------
\subsection{Graphe et lignes de niveau}



\begin{definition}
Soit $f : D_f \subset \Rr^2 \to \Rr$ une fonction de $2$ variables. 
Le \defi{graphe} $\mathcal{G}_f$ de $f$ est le sous-ensemble de $\Rr^3$ formé des points de coordonnées $(x,y,f(x,y))$ avec $(x,y)\in D_f$ dans l'ensemble de définition. Le graphe est donc :
$$\mathcal{G}_f= \big\{ (x,y,z)\in \Rr^3 \mid (x,y)\in D_f \text{ et } z=f(x,y)\big\}.$$
\end{definition}

Représenter graphiquement le graphe n'est possible que pour les fonctions d'une seule variable ou deux variables. 
Pour les fonctions d'une variables $g: D_g \subset \Rr \to \Rr$, on rappelle que le graphe est 
$$\mathcal{G}_g= \big\{ (x,y)\in \Rr^2 \mid x \in D_g \text{ et } y=f(x)\big\}.$$


Dans le cas d'une variable (à gauche) le graphe est une courbe, dans le cas de deux variables qui nous intéresse ici, c'est une surface.

[[dessin 2 var, et 1 var]]

Afin de tracer le graphe d'une fonction de deux variables, on découpe la surface en morceaux.


\bigskip

\evidence{Tranches.}

Une première façon de faire : tracer, pour quelques valeurs de $a$, les graphe des fonctions partielles
$$f_1:x\mapsto f(x,a) \quad \text{ et } \quad f_2:y\mapsto f(a,y).$$
La première représente l'intersection du graphe $\mathcal{G}_f$ avec le plan $(y=a)$ et la seconde représente l'intersection du graphe avec le plan $(x=a)$.

[[dessin, expliquer couleurs]]

\bigskip

\evidence{Lignes de niveau.}

On va aussi s'intéresser à d'autres courbes tracées sur la surface : les courbes de niveau.


\begin{definition}

Soit $f : D_f \subset \Rr^2 \to \Rr$ une fonction de deux variables. 
\begin{itemize}
  \item La \defi{ligne de niveau} 
$z=c\in \Rr$ est 
$$L_c =\big\{(x,y)\in D_f \mid f(x,y)=a \big\}.$$
  \item La \defi{courbe de niveau} $z=c$ est la trace de $\mathcal{G}_f$ dans le plan $(z=c)$ : 
$$\mathcal{G}_f \cap \{z=c) = \big\{ (x,y,c)\in \Rr^3 \mid f(x,y)=c \big\}.$$
\end{itemize}
\end{definition}


La ligne de niveau $a$ est une courbe du plan $\Rr^2$, la courbe de niveau $a$
est une courbe de l'espace $\Rr^3$. On obtient la courbe de niveau $a$, en partant de la ligne de niveau $a$ et en remontant à l'altitude $a$.

\begin{exemple}
Soit $f : \Rr^2 \to \Rr$ définie par $f(x,y)=x^2+y^2$. 

\begin{itemize}
  \item
  \begin{itemize}
    \item Si $c<0$, la ligne de niveau $L_c$ est vide (aucun point n'a d'altitude négative).
    \item Si $c=0$, la ligne de niveau $L_0$ se réduit à $\{(0,0)\}$.
    \item Si $c>0$, la ligne de niveau $L_c$  est le cercle du plan de centre $(0,0)$ et de rayon $\sqrt{c}$. On remonte $L_c$ à l'altitude $z=c$ : la courbe de niveau est alors le cercle horizontal de l'espace de centre $(0,0,c)$ et de rayon $\sqrt{c}$. 
  \end{itemize}
      
Le graphe est alors une superposition de cercles horizontaux de l'espace de centre $(0,0,c)$ et de rayon $\sqrt{c}$, $c>0$.     
Il prend l'allure suivante :
[[dessins]]

  \item Les tranches sont ici des paraboles [[détails...]]
  
[[dessins]]  

\end{itemize}
\end{exemple}





\begin{exemple}
Sur une carte topographique, les lignes de niveau représente les courbes ayant la même altitude. 
\begin{center}
  \includegraphics[scale=0.4]{figures/fig-plusvar-topo}
\end{center}
\begin{itemize}
  \item Ici, une carte \emph{Open Street Map}, avec au centre, le mont Gerbier des Jonc (source de la Loire, 1551 m). 
  \item Les lignes de niveau correspondent à des altitudes par cran de $10$ m (par exemple, pour $c=1400$, $c=1410$, $c=1420$\ldots).
  \item Lorsque les lignes de niveau sont très espacées, le terrain est plutôt plat ; lorsque les lignes sont rapprochées le terrain est pentu.
  \item Par définition, si on se promène en suivant une ligne de niveau, on reste toujours à la même altitude !
\end{itemize}
\end{exemple}


\begin{exemple}
L'image qui a servi lors de l'introduction sont le graphe et les courbes de niveau de la fonction $f : \Rr^2 \to \Rr$ définie par
$$(x,y) \mapsto z = 
\frac{\sin \left( x^{2}+3y^{2} \right)}{\frac{1}{10}+r^{2}}+\left( x^{2}+5y^{2} \right)\cdot \frac{\exp \left( 1-r^{2} \right)}{2} \quad \text{ avec } \quad r=\sqrt{x^{2}+y^{2}}.$$
Les courbes de niveau, sont projetées sur les plans $z=0$ et $z=9$ pour donner les lignes de niveau.


[[figure faire]]

\end{exemple}   

 
 
\bigskip

\evidence{Surfaces de niveau.}

Pour les fonctions de $3$ variables, le graphe étant dans $\Rr^4$, on ne peut le dessiner. 
La notion analogue à la ligne de niveau est celle \defi{surface de niveau}, donnée par l'équation $f(x,y,z)=c$.


\begin{exemple}
$f(x,y,z) = x^2+y^2+z^2$. Les surfaces de niveaux sont données par l'équation $x^2+y^2+z^2=c$. 
Pour $c \ge 0$, ces surfaces sont des sphères, centrées à l'origine et de rayon $\sqrt{c}$. 
Voici ces surfaces pour $c=1,4,9$. Elles ont été découpées pour laisser entrevoir les surfaces des différents niveaux.

[[figure à faire]]

\end{exemple}


\begin{exemple}

~[[autre exemple : à faire]]

\end{exemple}


\subsection{Exemples de surfaces quadratiques}

Ce sont des exemples à connaître, car il seront fondamentaux pour la suite du cours.

\begin{exemple}
$$f(x,y) = \frac{x^2}{a^2} + \frac{y^2}{b^2}-1$$

[[dessin]]

\begin{itemize}
  \item Les tranches sont des paraboles.
  \item Les lignes de niveau sont des ellipses.
  \item Le graphe est donc un \defi{paraboloïde elliptique}.
\end{itemize}

\end{exemple}


\begin{exemple}
$$f(x,y) = x^2$$

[[dessin]]

\begin{itemize}
  \item Les tranches sont des paraboles.
  \item Les lignes de niveau sont des droites.
  \item Le graphe est donc un \defi{cylindre parabolique}.
\end{itemize}

\end{exemple}


\begin{exemple}
$$f(x,y) = x^2-y^2$$

[[dessin]]

\begin{itemize}
  \item Les tranches sont des paraboles.
  \item Les lignes de niveau sont des hyperboles.
  \item Le graphe est donc un \defi{paraboloïde hyperbolique}, que l'on appelle aussi la \defi{selle de cheval}.
  \item Un autre nom pour cette surface est un \defi{col} (du nom d'un col de montagne). 
  En effet le point $(0,0,0)$, est le point de passage le moins haut pour passer d'un versant à l'autre de la montagne. 
\end{itemize}

\end{exemple}



 
%----------------------------------------------------
\begin{miniexercices}
\sauteligne
\begin{enumerate}
  \item 
  
  \item 
  
  \item 
  
\end{enumerate}
\end{miniexercices}


%%%%%%%%%%%%%%%%%%%%%%%%%%%%%%%%%%%%%%%%%%%%%%%%%%%%%
\section{Limites}


La notion de limite et de continuité des fonctions d'une seule variable se généralise 
en plusieurs variables sans complexité supplémentaire : il suffit de remplacer la valeur absolue par la norme euclidienne.


%----------------------------------------------------
\subsection{Définition}



Soit $f$ une fonction $f: E \subset \Rr^n \to \R$ définie au voisinage de $x_0 \in \Rr^n$, sauf peut-être en $x_0$.

\begin{definition}
La fonction $f$ admet pour \defi{limite} le nombre réel $\ell$ lorsque $x$ tend vers $x_0$ si 
$$\forall \epsilon >0 \quad \exists \delta > 0 \qquad
\forall x\in E  \qquad 
0< \| x-x_0 \| <\delta \implies | f(x)-\ell | < \epsilon
$$
On écrit alors 
$$\lim_{x \to x_0} f(x) = \ell \qquad \text{ ou } \quad f(x) \underset{x\to x_0}{\longrightarrow} \ell$$
\end{definition}

On définirait de même $\lim_{x \to x_0} f(x) = +\infty$ par :
$$\forall A >0 \quad \exists \delta > 0 \qquad
\forall x\in E  \qquad 
0< \| x-x_0 \| <\delta \implies | f(x) | > A
$$

\begin{remarque*}
\sauteligne
\begin{itemize}
\item La notion de limite ne dépend pas ici des normes utilisées.
\item Si elle existe, la limite est unique.
\end{itemize}
\end{remarque*}


\begin{exemple}
\label{ex:plusvarex}
Soit $f$ la fonction définie par $f(x,y) = x^2+y\sin(x+y^2)$.
\begin{enumerate}
  \item Montrer que $f$ tend vers $0$ lorsque $(x,y) \to (0,0)$. 
  \item Trouver un ouvert $U$ contenant l'origine tel que, pour tout $(x,y) \in U$, on ait $| f(x,y) | < \frac{1}{100}$.
\end{enumerate}
  
\begin{center}
  \includegraphics[scale=0.4]{figures/fig-plusvar-31-01}
\end{center}
 

\bigskip
\emph{Solution.}

\begin{enumerate}
  \item On majore $f(x,y)$ en utilisant $|\sin(t)| \le 1$ :
  $$\big| f(x,y) \big|  = \big| x^2+y\sin(x+y^2) \big| \le
  x^2 + |y| \big| \sin(x+y^2) \big| \le x^2 +|y|$$ 
  
  Fixons $0<\epsilon<1$. Fixons $a = \sqrt{\frac{\epsilon}{2}}$ et $b=\frac\epsilon2$,
  alors pour $x \in ]-a,a[$, on a $x^2 < \frac\epsilon2$, pour $y \in ]-b,b[$ on a $|y| < \frac\epsilon2$. Donc pour $(x,y) \in ]-a,a[ \times ]-b,b[$ on a donc 
  $$\big| f(x,y) \big| \le x^2 +|y| < \frac\epsilon2 + \frac\epsilon2 = \epsilon$$
  
  Une valeur $\delta$ qui convient, est $\delta = \frac\epsilon2$, en effet
  si $\| (x,y) \| < \delta$ alors $|x| < \delta = \frac\epsilon2 \le  \sqrt{\frac{\epsilon}{2}} $ 
  et $|y| < \delta = \frac\epsilon2$ donc $|f(x,y)| < \epsilon$. Conclusion : $f$ admet pour limite $0$, lorsque $(x,y)$ tend vers $(0,0)$.
  
\myfigure{1}{
  \tikzinput{fig-plusvar-31-02}
} 
  
  
  \item Pour $\epsilon = \frac{1}{100}$ on a $a = \frac{1}{\sqrt{200}}$ et $b=\frac{1}{200}$. Pour chaque $(x,y)$ de l'ouvert $]-a,a[ \times ]-b,b[$, on a $|f(x,y)| < \frac{1}{100}$.
\end{enumerate}
  
\end{exemple}

%----------------------------------------------------
\subsection{Opérations sur les limites}


Pour calculer les limites, on ne recourt que rarement à cette définition. 
On utilise plutôt les théorèmes généraux : opérations sur les limites et encadrement. 
Ce sont les mêmes énoncés que pour les fonctions d'une variable, il n'y a aucune difficulté ni nouveauté. 


\begin{proposition}[Opérations sur les limites]
Soient $f,g:\Rr^n\to \Rr$ définies au voisinage de $x_0 \in \Rr^n$ et telles que $f$ et $g$ admettent des limites en $x_0$.
$$\begin{array}{ll}
\displaystyle \lim_{x_0}(f+g)=\lim _{x_0}f+\lim _{x_0}g 
\qquad & \qquad 
\displaystyle \lim _{x_0}(fg)=\lim _{x_0}f\times \lim _{x_0}g\\[3ex]
\displaystyle \lim _{x_0}\frac{1}{g}=\frac{1}{\lim _{x_0}g} 
\qquad & \qquad 
\displaystyle \lim _{x_0}\frac{f}{g}=\frac{\lim _{x_0}f}{\lim _{x_0}g}
\end{array}$$
\end{proposition}

\begin{remarque*}
\sauteligne
\begin{itemize}
  \item 
Les résultats ci-dessus sont aussi valables pour des limites infinies avec les conventions usuelles :
$$\ell+\infty =+\infty,\quad \ell-\infty =-\infty,\quad \frac{1}{0^+}=+\infty ,\quad \frac{1}{0^-}=-\infty ,\quad \frac{1}{\pm \infty }=0,$$
$$\ell\times \infty =\infty \; (\ell\neq 0),\; \infty \times \infty =\infty \text{ (avec règle de multiplication des signes).}$$

  \item 
Les formes indéterminées sont : $+\infty -\infty$, $\displaystyle \frac{0}{0}$, 
$\displaystyle \frac{\infty }{\infty}$, $\displaystyle 0\times \infty$, $\displaystyle 1^{\infty}$, et $0^0$.
\end{itemize}
\end{remarque*}


La composition est aussi souvent utile : 
\begin{itemize}
  \item soit $f : \Rr^n \to \Rr$ une fonction de plusieurs variables, telle que $\lim_{x \to x_0} f(x) = \ell$,
  \item soit $g : \Rr \to \Rr$ une fonction d'une seule variable, telle que 
$\lim_{t \to \ell} g(t) = \ell'$,
  \item alors la fonction de plusieurs variables $g \circ f : \Rr^n \to \Rr$ définie par $g \circ f (x) = g \big( f(x) \big)$ vérifie
  $\lim_{x \to x_0} g \circ f(x) = \ell'$.
\end{itemize}

Par exemple, grâce à l'exemple \ref{ex:plusvarex}, et comme $e^t \to 1$ lorsque $t\to 0$, on en déduit :
$$e^{x^2+y\sin(x+y^2)} \underset{(x,y) \to (0,0)}{\longrightarrow} 1$$


\bigskip

Il existe aussi un théorème \og{}des gendarmes\fg{}.
\begin{theoreme}[Théorème d'encadrement] 
Soient $f,g,h:\Rr^n\to \Rr$ trois fonctions définies dans un voisinage $U$ de $x_0 \in \Rr^n$.
\begin{itemize}
  \item Si pour tout $x \in U$, on a $f(x) \le  h(x) \le g(x)$,
  \item et $\lim _{x \to x_0}f = \lim_{x \to x_0}g = \ell$.
\end{itemize}
Alors $h$ admet une limite au point $x_0$ et $\displaystyle \lim _{x \to x_0} h=\ell$.
\end{theoreme}



\begin{exemple}
Soit $h$ définie par $h(x,y) = \cos(x+y^2)\big( x^2+y\sin(x+y^2) \big)$.
On majore la valeur absolue du cosinus par $1$ :
$$\big| h(x,y) \big| \le x^2+y\sin(x+y^2) .$$
On a vu lors de l'exemple \ref{ex:plusvarex} que la fonction définie par
$f(x,y) = x^2+y\sin(x+y^2)$ tend vers $0$ en $(0,0)$. 
Donc par le théorème des gendarmes $h(x,y)$ tend aussi vers $0$ lorsque $(x,y)$ tend vers $(0,0)$.
\end{exemple}



%----------------------------------------------------
\subsection{Limite le long d'un chemin}

L'unicité de la limite implique que, quelle que soit la façon dont arrive au point $x_0$, la valeur limite est toujours la même.

\begin{proposition} Soit $f:\Rr^n\to \Rr$ une fonction définie au voisinage de $x_0 \in \Rr^n$, sauf peut être en $x_0$.
\begin{enumerate}
\item Si $f$ admet une limite $\ell$ au point $x_0$, alors la restriction de $f$ à toute courbe passant par $x_0$ admet une limite en $x_0$ et cette limite est $\ell$. 
\item Par contraposée, si les restrictions de $f$ à deux courbes passant par $x_0$ ont des limites différentes au point $x_0$, alors $f$ n'admet pas de limite au point $x_0$.
\end{enumerate}
\end{proposition}

Détaillons dans les cas des fonctions de deux variables :
\begin{itemize}
  \item Une courbe passant par le point $(x_0,y_0) \in \Rr^2$ est une fonction continue 
$\gamma : \Rr \to \Rr^2$, $t \mapsto (x(t),y(t))$, telle que $\gamma(0) = (x_0,y_0)$.
  \item La restriction de $f$ le long de $\gamma$ est $f \circ \gamma$, $t \mapsto f \big( x(t),y(t) \big)$.
  \item Si $f$ a pour limite $\ell$ en $(x_0,y_0)$ alors, la première partie de la proposition affirme, 
  $f \big( x(t),y(t) \big) \underset{t \to 0}{\longrightarrow} \ell$.
\end{itemize}

\myfigure{1}{
  \tikzinput{fig-plusvar-33-01}
}

\begin{exemple}
Soit $f : \Rr^2 \to \Rr$ définie par 
$$f(x,y) = \frac{xy}{x^2+y^2} \text{ si } (x,y) \neq (0,0) \quad \text{ et }
f(0,0) =0.$$

La fonction $f$ admet-elle une limite en $(0,0)$ ?

\bigskip
\emph{Solution.}

\begin{itemize}
  \item Si on prend le chemin $\gamma_1(t) = (t,0)$, alors 
  $f \circ \gamma_1 (t) = f(t,0) = 0$. Donc lorsque $t \to 0$, $f \circ \gamma_1 (t) \to 0$.
  
  \item Si on prend le chemin $\gamma_2(t) = (t,t)$, alors 
  $f \circ \gamma_2 (t) = f(t,t) = \frac{t^2}{2t^2} = \frac12$. Donc lorsque $t \to 0$, $f \circ \gamma_2 (t) \to \frac 12$.


Ci-dessous, sur la figure de gauche, les deux chemins du plans ; 
sur les deux figures de droites, deux vues différentes des valeurs prises par $f$ le long de ces chemins.

\begin{minipage}{0.25\textwidth}
\myfigure{0.6}{
  \tikzinput{fig-plusvar-33-02}
}
\end{minipage}
\begin{minipage}{0.79\textwidth}
\begin{center}
\includegraphics[scale=0.20]{figures/fig-plusvar-33-03a} 
\includegraphics[scale=0.27]{figures/fig-plusvar-33-03b}  
\end{center}
\end{minipage}


   
  \item Si $f$ admettait une limite $\ell$, alors quel que soit le chemin $\gamma(t)$, 
  tel que $\gamma(t) \to (0,0)$ lorsque $t\to0$, on a  $f \circ \gamma(t) \to \ell$. 
  On aurait donc $\ell=0$ et $\ell=\frac12$. Ce qui est contredit l'unicité de la limite. Ainsi $f$ n'a pas de limite en $(0,0)$.
\end{itemize}
\end{exemple}

\bigskip
\bigskip

Une autre formulation possible : 

Si $f : \Rr^n\to \Rr$ a pour limite $\ell$ en $x_0 \in \Rr^n$ alors
pour tout suite $(u_n)$ d'élément de $\Rr^n$ telle que $u_n \to x_0$, on a $f(u_n) \to \ell$. 
Pour les fonctions de deux variables, cela s'écrit ainsi :
si $f$ a pour limite $\ell$ en $(a,b)$ alors pour toute suite
$(a_n,b_n) \to (a,b)$ on a $f(a_n,b_n) \to \ell$.


%----------------------------------------------------
\subsection{Fonctions continues}


\begin{definition}
\sauteligne
\begin{enumerate}
\item $f : E \subset \Rr^n \rightarrow \Rr$ est \defi{continue en $x_0$} $\in E$ si $\displaystyle \lim_{x \rightarrow x_0} f(x) = f(x_0)$.
\item $f$ est \defi{continue sur $E$} si elle est continue en tout point de $E$.
\end{enumerate}
\end{definition}

Par les propriétés des limites, si $f$ et $g$ deux fonctions continues en $x_0$. Alors : 
\begin{itemize}
\item la fonction $f + g$ est continue en $x_0$,
\item de même $fg$ et $f/g$ (avec $g(x) \neq 0$ sur un voisinage de $x_0$)  sont continues en $x_0$,
\item si $h : \Rr \to \Rr$ est continue, alors $h \circ f$ est continue en $x_0$.
\end{itemize}

\begin{exemple}
\sauteligne
\begin{itemize}
  \item Les applications définies par $(x,y)\mapsto x+y$, $(x,y)\mapsto xy$, puis 
  $(x,y)\mapsto x^2+3xy$ et toutes les fonctions polynômes en deux variables $x$ et $y$ 
  sont continues sur $\Rr^2$. De la même façon toutes les fractions rationnelles 
  en deux variables sont continues là où elles sont définies.
  
  \item Comme l'exponentielle est une fonction continue, alors $(x,y)\mapsto e^{xy}$ est continue sur $\Rr^2$.
  
  \item La fonction définie par $f(x,y) = \frac{1}{\sqrt{x^2 + y^2}}$ est continue sur $\Rr^2\setminus\{(0,0)\}$
\end{itemize}
\end{exemple}
  


\begin{definition}[Prolongement par continuité]
Soit $f: E \subset \Rr^n \to \Rr$. Soit $x_0$ un point adhérent à $E$ 
n'appartenant pas à $E$. Si $f$ a une limite $\ell$ lorsque $x \to x_0$,
on peut étendre le domaine de définition de $f$ à $E \cup \lbrace x_0 \rbrace$ en posant $f(x_0)=\ell$. 
La fonction étendue est continue en $x_0$. On dit que l'on a obtenu un \defi{prolongement de $f$ par continuité} au point $x_0$.
\end{definition}


\begin{exemple}
Soit $f : \Rr^2 \setminus \{ (0,0) \}$ définie par 
$$f(x,y) = \frac{xy}{\sqrt{x^2+y^2}}$$

Est-il possible de prolonger $f$ par continuité en $(0,0)$ ?

Sur la figure ci-dessous, la question devient simplement : est-il possible de 
boucher le trou au milieu de la surface en rajoutant juste un point ?
\begin{center}
\includegraphics[scale=0.3]{figures/fig-plusvar-34-01}  
\end{center}

\bigskip
\emph{Solution.}

\begin{itemize}
  \item \textbf{Limite à l'origine.}
  
  On utilise que $|x| \le \sqrt{x^2+y^2}$ et $|y| \le \sqrt{x^2+y^2}$.
  Donc
  $$| f(x,y) |  = \frac{|x| \cdot |y|}{\sqrt{x^2+y^2}}
  \le \sqrt{x^2+y^2} \underset{(x,y) \to (0,0)}{\longrightarrow} 0$$
  
  \item \textbf{Prolongement.}
  
  Pour prolonger $f$ en $(0,0)$, on choisit comme valeur la limite obtenue.  
  On pose donc $f(0,0) = 0$. (On note encore $f : \Rr^2 \to \Rr$ la fonction prolongée.) 
  
  \item \textbf{Continuité.}
  
  Par notre choix de $f(0,0)$, $f$ est continue en $(0,0)$.
  En dehors de l'origine, $f$ est continue comme somme, produit, composition, inverse de fonctions continue. 
  Conclusion : la fonction prolongée est continue sur $\Rr^2$ tout entier.
\end{itemize}  
\end{exemple}


%----------------------------------------------------
\begin{miniexercices}
\sauteligne
\begin{enumerate}
  \item Soit $f(x,y) = \frac{1+x}{1+y}$. Trouver un ouvert $U$ contenant l'origine, telle que 
  $0.999 < f(x,y) < 1.001$, pour tout $(x,y) \in U$.

  \item Soit $f : \Rr^n \rightarrow \Rr$ une fonction continue au point $(x_1,\ldots,x_n)$.
  Montrer que la fonction partielle $f_i : \Rr \longrightarrow \Rr$ définie par 
  $f_i(t) = f(x_1,\ldots,x_{i-1},t,x_{i+1},\ldots,x_n)$ est continue en $x_i$.
  
  \item Sachant que la limite de $f(x,y) = \frac{1+x}{1+y}$ en $(0,0)$ est $1$, calculer les limites des fonctions suivantes en $(0,0)$ :
 $\frac{1+y}{1+x}+ x^2+y^2$ ; $\frac{1+y}{1+x}$ ;  $\sin(xy)\frac{1+x}{1+y}$ ; $\ln\left(\frac{1+x}{1+y}\right)$.
  
  \item Sachant $\ln(t) \le t-1$, pour tout $t>0$, calculer la limite de $\frac{\ln(1+xy)}{1+x^2+y^4}$ en $(0,0)$. 
  
  \item Soit $f(x,y)= $. Soit $\gamma(t) = $ [avec paramètre]. Calculer la limite de $f \circ \gamma(t)$ lorsque $t\to0$. $f$ admet-elle une limite en $(0,0)$ ?  $f$ est-elle prolongeable par continuité ? 
  
  
  \item Soit $f$ définie sur $\Rr^2\setminus \{ (0,0) \}$ par 
  $f(x,y) = $. 
  $f$ admet-elle une limite en $(0,0)$ ? 
  $f$ est-elle prolongeable par continuité ? 
  Mêmes questions avec $f(x,y)=$.
  
\end{enumerate}
\end{miniexercices}


%%%%%%%%%%%%%%%%%%%%%%%%%%%%%%%%%%%%%%%%%%%%%%%%%%%%%
\section{Coordonnées polaires}


Plutôt que de repérer un point du plan $\Rr^2$ par ses coordonnées cartésiennes $(x,y)$, 
on peut le faire au moyen de sa distance à l'origine et de l'angle formé avec l'horizontale : ce sont les coordonnées polaires.


%----------------------------------------------------
\subsection{Définition}

Soit $M$ un point du plan $\Rr^2$. Soit $O=(0,0)$ l'origine. Soit $(O, \vec i, \vec j)$ un repère orthonormé direct.

\begin{itemize}
  \item On note $r = \| \overrightarrow{OM} \|$, la distance de $M$ à l'origine.
  \item On note $\theta$ l'angle entre $\vec i$ et $\vec{OM}$.
\end{itemize}

\myfigure{1}{
  \tikzinput{fig-plusvar-41-01}
}

On note $[r:\theta]$ les \defi{coordonnées polaires} du point $M$. Dans ce cours $r$ sera toujours positif.
L'angle n'est pas déterminé de manière unique. Plusieurs choix sont possibles. 
Pour avoir unicité, on peut limiter $\theta$ à l'intervalle $[0,2\pi[$, ou bien 
$]-\pi,+\pi]$. On n'attribue généralement pas de coordonnée polaire au point origine (l'angle n'aurait pas de sens). 

\bigskip
\evidence{Coordonnées polaires vers coordonnées cartésiennes.}

On retrouve les coordonnées cartésiennes $(x,y)$ à partir des coordonnées polaires $[r:\theta]$ par les formules
$$x = r\cos \theta \qquad \qquad y = r \sin \theta$$

Autrement dit, on a défini une application :
$$
]0,+\infty[\times[0,2\pi[\rightarrow \Rr^2 \qquad (r,\theta)\mapsto (r\cos\theta,r\sin\theta).
$$

\bigskip
\evidence{Coordonnées cartésiennes vers coordonnées polaires.}

On retrouve  $r$ et $\theta$ à partir de $(x,y)$ par les formules suivantes
$$r = \sqrt{x^2+y^2}$$
et dans le cas $x>0$ et $y\ge0$
$$\theta = \arctan\left(\frac yx\right)   \quad \text{ avec } x>0 \text{ et } y \ge 0$$

Pour les points dans les autres quadrants, on se ramène au quadrant principal où $x>0$ et $y\ge0$.



%----------------------------------------------------
\subsection{Limite et continuité}


Lorsque l'on considère des applications $f: E \subset \Rr^2 \rightarrow \R$, il est quelques fois plus facile de prouver des résultats de limite, continuité, etc. en passant par les coordonnées polaires.


\begin{proposition}
\label{prop:limrtheta}
Soit $f:\Rr^2\to \Rr$ une fonction définie au voisinage de $(0,0)\in \Rr^2$, sauf peut être en $(0,0)$. Si
$$\lim _{r\to 0}f(r\cos \theta ,r\sin \theta )=\ell \in \Rr$$
existe indépendamment de $\theta$, alors $\displaystyle \lim _{(x,y)\to (0,0)}f(x,y)=\ell$.
\end{proposition}


Pour clarifier cette proposition et expliquer les différents cas pratiques de la limite, voici comment faire. On exprime $f(x,y)$ en coordonnées polaires en calculant $f(r\cos\theta,r\sin\theta)$.
\begin{enumerate}
  \item Si $\lim_{r\to0} f(r\cos\theta,r\sin\theta)$ existe et si elle ne dépend pas de la variable $\theta$, alors cette limite est la limite de $f$ au point $(0,0)$.
  
  \item Si $\lim_{r\to0} f(r\cos\theta,r\sin\theta)$ n'existe pas, alors $f$ n'a pas de limite au  point $(0,0)$.
  
  \item Si $\lim_{r\to0} f(r\cos\theta,r\sin\theta) = \ell(\theta)$ dépend de $\theta$, alors $f$ n'a pas de limite au  point $(0,0)$. Pour le justifier, on donne deux valeurs $\theta_1$ et $\theta_2$ telles que $\ell(\theta_1) \neq \ell(\theta_2)$.
  
\end{enumerate}


Voyons un exemple de chaque situation.

\begin{exemple}
\sauteligne
\begin{enumerate}
  \item $f(x,y)=\dfrac{x^3}{x^2+y^2}$
  
  $$f(r\cos\theta,r\sin\theta) = \frac{r^3 \cos^3\theta}{r^2 (\cos^2\theta+\sin^2\theta)}
  = \frac{r^3 \cos^3\theta}{r^2} = r \cos^3 \theta$$
  Comme $\big| \cos^3 \theta \big| \le 1$ alors $r \cos^3 \theta \underset{r\to0}{\longrightarrow} 0$ (même si $\theta$ variait lorsque $r$ varie).
  Ce qui implique que $f(r\cos\theta,r\sin\theta) \underset{r\to0}{\longrightarrow} 0$.
  La limite existe (indépendamment des valeurs prises par $\theta$), donc 
  la fonction $f$ admet bien une limite en $(0,0)$ : $f(x,y) \underset{(x,y) \to (0,0)}{\longrightarrow} 0$.
  
  Pour ceux qui voudrait tout faire à la main avec plus de détails, on peut aussi écrire $\big|f(r\cos\theta,r\sin\theta)\big| \le r$ autrement dit $\big|f( x,y) \big| \le \sqrt{x^2+y^2}$. Donc $f(x,y) \underset{(x,y) \to (0,0)}{\longrightarrow} 0$.
  
   
  \item $f(x,y)=\dfrac{y}{x^2+y^3}$
  
  
 $$f(r\cos\theta,r\sin\theta) = \frac{r\sin\theta}{r^2(\cos^2\theta + r\sin^2\theta)} = \frac{1}{r} \frac{\sin\theta}{\cos^2\theta + r\sin^2\theta}$$
 
Fixons $\theta$ tel que $\sin \theta \neq 0$ (c'est-à-dire $\theta \neq 0 \pmod \pi$) alors, lorsque $r\to0$, $f(r\cos\theta,r\sin\theta)$ n'a pas de limite. En particulier la fonction $(x,y) \mapsto f(x,y)$ n'a pas de limite en $(0,0)$.
 
  
  \item  $f(x,y)=\dfrac{xy}{x^2+y^2}$
  
 $$f(r\cos\theta,r\sin\theta) = \frac{r^2 \cos\theta\sin\theta}{r^2}
  = \cos\theta \sin \theta = \frac12\sin(2\theta)$$ 

Pour $\theta$ fixé, la fonction $r \mapsto f(r\cos\theta,r\sin\theta)$ admet bien une limite $\ell(\theta) = \frac12\sin(2\theta)$. Mais cette limite dépend de l'angle $\theta$ :
si $\theta =0$, $\ell(\theta)=0$, par contre si $\theta = \frac\pi2$, $\ell(\theta) = 1$. Comme la limite dépend de l'angle, alors la fonction de deux variables $(x,y) \mapsto f(x,y)$ n'a pas de limite en $(0,0)$.
\end{enumerate}
\end{exemple}

 

%----------------------------------------------------
\subsection{Un exemple}

Cet exemple est assez subtil et peu être passé en première lecture.

\begin{remarque*}
Soit $\ell \in \Rr$. Soit $f: \Rr^2 \rightarrow \Rr$ une fonction telle que, pour chaque $\theta$ fixé, $\displaystyle \lim_{r \to0} f(r\cos\theta, r\sin\theta)=\ell$. Peut-on en conclure que $f$ est continue au point $(0,0)$ ? La réponse est non !

Autrement dit, regarder la limite de $f$ le long des rayons, ne permet pas de prouver la limite de $f$ à l'origine.
\end{remarque*}

Attention la différence entre cette remarque et la proposition \ref{prop:limrtheta} est subtile. Dans la proposition \ref{prop:limrtheta}, on a une hypothèse en terme de limite du type :
$$\forall \epsilon \quad \exists r_0 \quad \forall r<r_0 \quad \forall \theta \quad  \quad \cdots$$
alors quand dans la remarque, on note que l'hypothèse (plus faible) suivante est insuffisante :
$$ \forall \theta \quad \forall \epsilon \quad \exists r_0 \quad \forall r<r_0 \quad \ldots$$


\begin{exemple}
Soit la fonction $f$ définie sur $\Rr^2 \setminus \{(0,0)\}$ par
$$f(x,y)=\frac{xy^2}{x^2+y^4}.$$
\begin{enumerate}
  \item Le long de tous les rayons $f$ tend vers $0$, c'est-à-dire, pour $\theta$ fixé
$$f(r\cos\theta,r\sin\theta) \underset{r \to 0}{\longrightarrow} 0$$
  \item Cependant $f$ n'a pas de limite en $(0,0)$.
\end{enumerate}


\bigskip
\emph{Solution.}
\begin{enumerate}
  \item Calculons d'abord :
  $$f(r\cos\theta,r\sin\theta) 
  = \frac{r\cos\theta\sin^2\theta}{\cos^2\theta + r\sin^3\theta}$$
  
  Fixons $\theta$ et discutons selon sa valeur :
  \begin{itemize}
    \item Si $\cos\theta \neq 0$, alors le numérateur tend vers $0$, tandis que le dénominateur tend vers $\cos^2\theta \neq0$. Donc $f(r\cos\theta,r\sin\theta) \to 0$ lorsque $r\to0$.
    
    \item Si $\cos\theta = 0$, alors on se trouve sur des points $(x,y)$ où $x=0$ et donc 
    $f(r\cos\theta,r\sin\theta)=f(0,y)=0$.    
  \end{itemize}
  Dans tous les cas $f$ tend vers $0$ sur tous les rayons défini par un angle $\theta$ fixé.
  
  
  \item Considérons le chemin $\gamma(t) = (t^2,t)$, alors 
  $$f \circ \gamma(t) = \frac{t^4}{2t^4} = \frac12.$$
  Mais on a vu que le long des rayons $f$ tend vers $0$. 
  Cela contredit l'existence d'une limite pour $f(x,y)$ en $(0,0)$.
 
  \myfigure{1}{
  \tikzinput{fig-plusvar-43-01}
  }

\end{enumerate}  

\end{exemple}
 
%----------------------------------------------------
\begin{miniexercices}
\sauteligne
\begin{enumerate}

 \item Calculer l'angle $\theta$ des coordonnées polaires $[r:\theta]$ 
 d'un point $(x,y)$ dans le cas $x>0$, $y <0$. 
 Puis faire les cas où $x<0$.

  \item La fonction $f$ définie par $f(x,y)=\frac{??}{??}$  
  admet-elle une limite au point $(0,0)$ ?
  Même question avec $f(x,y)=$, puis $f(x,y)=$.

  \item Montrer que pour la fonction définie par $f(x,y)=...$ alors $f$ admet une    même limite en $(0,0)$ le long de chaque rayon, et que pourtant $f$ n'a pas de limite en $(0,0)$.
\end{enumerate}
\end{miniexercices}



\auteurs{
\\
D'après des cours de Abdellah Hanani (Lille), 
Goulwen Fichou et Stéphane Leborgne (Rennes),
Laurent Pujo-Menjouet (Lyon). 

Revu et augmenté par Arnaud Bodin.

Relu par Barbara Tumpach et Vianney Combet.
}

\finchapitre 
\end{document}


