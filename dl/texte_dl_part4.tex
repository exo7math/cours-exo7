
%%%%%%%%%%%%%%%%%% PREAMBULE %%%%%%%%%%%%%%%%%%


\documentclass[12pt]{article}

\usepackage{amsfonts,amsmath,amssymb,amsthm}
\usepackage[utf8]{inputenc}
\usepackage[T1]{fontenc}
\usepackage[francais]{babel}


% packages
\usepackage{amsfonts,amsmath,amssymb,amsthm}
\usepackage[utf8]{inputenc}
\usepackage[T1]{fontenc}
%\usepackage{lmodern}

\usepackage[francais]{babel}
\usepackage{fancybox}
\usepackage{graphicx}

\usepackage{float}

%\usepackage[usenames, x11names]{xcolor}
\usepackage{tikz}
\usepackage{datetime}

\usepackage{mathptmx}
%\usepackage{fouriernc}
%\usepackage{newcent}
\usepackage[mathcal,mathbf]{euler}

%\usepackage{palatino}
%\usepackage{newcent}


% Commande spéciale prompteur

%\usepackage{mathptmx}
%\usepackage[mathcal,mathbf]{euler}
%\usepackage{mathpple,multido}

\usepackage[a4paper]{geometry}
\geometry{top=2cm, bottom=2cm, left=1cm, right=1cm, marginparsep=1cm}

\newcommand{\change}{{\color{red}\rule{\textwidth}{1mm}\\}}

\newcounter{mydiapo}

\newcommand{\diapo}{\newpage
\hfill {\normalsize  Diapo \themydiapo \quad \texttt{[\jobname]}} \\
\stepcounter{mydiapo}}


%%%%%%% COULEURS %%%%%%%%%%

% Pour blanc sur noir :
%\pagecolor[rgb]{0.5,0.5,0.5}
% \pagecolor[rgb]{0,0,0}
% \color[rgb]{1,1,1}



%\DeclareFixedFont{\myfont}{U}{cmss}{bx}{n}{18pt}
\newcommand{\debuttexte}{
%%%%%%%%%%%%% FONTES %%%%%%%%%%%%%
\renewcommand{\baselinestretch}{1.5}
\usefont{U}{cmss}{bx}{n}
\bfseries

% Taille normale : commenter le reste !
%Taille Arnaud
%\fontsize{19}{19}\selectfont

% Taille Barbara
%\fontsize{21}{22}\selectfont

%Taille François
\fontsize{25}{30}\selectfont

%Taille Pascal
%\fontsize{25}{30}\selectfont

%Taille Laura
%\fontsize{30}{35}\selectfont


%\myfont
%\usefont{U}{cmss}{bx}{n}

%\Huge
%\addtolength{\parskip}{\baselineskip}
}


% \usepackage{hyperref}
% \hypersetup{colorlinks=true, linkcolor=blue, urlcolor=blue,
% pdftitle={Exo7 - Exercices de mathématiques}, pdfauthor={Exo7}}


%section
% \usepackage{sectsty}
% \allsectionsfont{\bf}
%\sectionfont{\color{Tomato3}\upshape\selectfont}
%\subsectionfont{\color{Tomato4}\upshape\selectfont}

%----- Ensembles : entiers, reels, complexes -----
\newcommand{\Nn}{\mathbb{N}} \newcommand{\N}{\mathbb{N}}
\newcommand{\Zz}{\mathbb{Z}} \newcommand{\Z}{\mathbb{Z}}
\newcommand{\Qq}{\mathbb{Q}} \newcommand{\Q}{\mathbb{Q}}
\newcommand{\Rr}{\mathbb{R}} \newcommand{\R}{\mathbb{R}}
\newcommand{\Cc}{\mathbb{C}} 
\newcommand{\Kk}{\mathbb{K}} \newcommand{\K}{\mathbb{K}}

%----- Modifications de symboles -----
\renewcommand{\epsilon}{\varepsilon}
\renewcommand{\Re}{\mathop{\text{Re}}\nolimits}
\renewcommand{\Im}{\mathop{\text{Im}}\nolimits}
%\newcommand{\llbracket}{\left[\kern-0.15em\left[}
%\newcommand{\rrbracket}{\right]\kern-0.15em\right]}

\renewcommand{\ge}{\geqslant}
\renewcommand{\geq}{\geqslant}
\renewcommand{\le}{\leqslant}
\renewcommand{\leq}{\leqslant}

%----- Fonctions usuelles -----
\newcommand{\ch}{\mathop{\mathrm{ch}}\nolimits}
\newcommand{\sh}{\mathop{\mathrm{sh}}\nolimits}
\renewcommand{\tanh}{\mathop{\mathrm{th}}\nolimits}
\newcommand{\cotan}{\mathop{\mathrm{cotan}}\nolimits}
\newcommand{\Arcsin}{\mathop{\mathrm{Arcsin}}\nolimits}
\newcommand{\Arccos}{\mathop{\mathrm{Arccos}}\nolimits}
\newcommand{\Arctan}{\mathop{\mathrm{Arctan}}\nolimits}
\newcommand{\Argsh}{\mathop{\mathrm{Argsh}}\nolimits}
\newcommand{\Argch}{\mathop{\mathrm{Argch}}\nolimits}
\newcommand{\Argth}{\mathop{\mathrm{Argth}}\nolimits}
\newcommand{\pgcd}{\mathop{\mathrm{pgcd}}\nolimits} 

\newcommand{\Card}{\mathop{\text{Card}}\nolimits}
\newcommand{\Ker}{\mathop{\text{Ker}}\nolimits}
\newcommand{\id}{\mathop{\text{id}}\nolimits}
\newcommand{\ii}{\mathrm{i}}
\newcommand{\dd}{\mathrm{d}}
\newcommand{\Vect}{\mathop{\text{Vect}}\nolimits}
\newcommand{\Mat}{\mathop{\mathrm{Mat}}\nolimits}
\newcommand{\rg}{\mathop{\text{rg}}\nolimits}
\newcommand{\tr}{\mathop{\text{tr}}\nolimits}
\newcommand{\ppcm}{\mathop{\text{ppcm}}\nolimits}

%----- Structure des exercices ------

\newtheoremstyle{styleexo}% name
{2ex}% Space above
{3ex}% Space below
{}% Body font
{}% Indent amount 1
{\bfseries} % Theorem head font
{}% Punctuation after theorem head
{\newline}% Space after theorem head 2
{}% Theorem head spec (can be left empty, meaning ‘normal’)

%\theoremstyle{styleexo}
\newtheorem{exo}{Exercice}
\newtheorem{ind}{Indications}
\newtheorem{cor}{Correction}


\newcommand{\exercice}[1]{} \newcommand{\finexercice}{}
%\newcommand{\exercice}[1]{{\tiny\texttt{#1}}\vspace{-2ex}} % pour afficher le numero absolu, l'auteur...
\newcommand{\enonce}{\begin{exo}} \newcommand{\finenonce}{\end{exo}}
\newcommand{\indication}{\begin{ind}} \newcommand{\finindication}{\end{ind}}
\newcommand{\correction}{\begin{cor}} \newcommand{\fincorrection}{\end{cor}}

\newcommand{\noindication}{\stepcounter{ind}}
\newcommand{\nocorrection}{\stepcounter{cor}}

\newcommand{\fiche}[1]{} \newcommand{\finfiche}{}
\newcommand{\titre}[1]{\centerline{\large \bf #1}}
\newcommand{\addcommand}[1]{}
\newcommand{\video}[1]{}

% Marge
\newcommand{\mymargin}[1]{\marginpar{{\small #1}}}



%----- Presentation ------
\setlength{\parindent}{0cm}

%\newcommand{\ExoSept}{\href{http://exo7.emath.fr}{\textbf{\textsf{Exo7}}}}

\definecolor{myred}{rgb}{0.93,0.26,0}
\definecolor{myorange}{rgb}{0.97,0.58,0}
\definecolor{myyellow}{rgb}{1,0.86,0}

\newcommand{\LogoExoSept}[1]{  % input : echelle
{\usefont{U}{cmss}{bx}{n}
\begin{tikzpicture}[scale=0.1*#1,transform shape]
  \fill[color=myorange] (0,0)--(4,0)--(4,-4)--(0,-4)--cycle;
  \fill[color=myred] (0,0)--(0,3)--(-3,3)--(-3,0)--cycle;
  \fill[color=myyellow] (4,0)--(7,4)--(3,7)--(0,3)--cycle;
  \node[scale=5] at (3.5,3.5) {Exo7};
\end{tikzpicture}}
}



\theoremstyle{definition}
%\newtheorem{proposition}{Proposition}
%\newtheorem{exemple}{Exemple}
%\newtheorem{theoreme}{Théorème}
\newtheorem{lemme}{Lemme}
\newtheorem{corollaire}{Corollaire}
%\newtheorem*{remarque*}{Remarque}
%\newtheorem*{miniexercice}{Mini-exercices}
%\newtheorem{definition}{Définition}




%definition d'un terme
\newcommand{\defi}[1]{{\color{myorange}\textbf{\emph{#1}}}}
\newcommand{\evidence}[1]{{\color{blue}\textbf{\emph{#1}}}}



 %----- Commandes divers ------

\newcommand{\codeinline}[1]{\texttt{#1}}

%%%%%%%%%%%%%%%%%%%%%%%%%%%%%%%%%%%%%%%%%%%%%%%%%%%%%%%%%%%%%
%%%%%%%%%%%%%%%%%%%%%%%%%%%%%%%%%%%%%%%%%%%%%%%%%%%%%%%%%%%%%



\begin{document}

\debuttexte

%%%%%%%%%%%%%%%%%%%%%%%%%%%%%%%%%%%%%%%%%%%%%%%%%%%%%%%%%%%
\diapo


\change

Les développements limités ont de nombreuses applications
en voici trois parmi les plus remarquables.

\change

Tout d'abord les DL sont extrêmement efficaces pour calculer
des limites.

\change

Ils permettent aussi de préciser le comportement local des fonctions
et en particulier la position d'une courbe par rapport à tangente.

\change 

Enfin les développement limités s'applique pour l'étude des asymptotes et le comportement 
en $+\infty$ des fonctions.


%%%%%%%%%%%%%%%%%%%%%%%%%%%%%%%%%%%%%%%%%%%%%%%%%%%%%%%%%%%
\diapo


Voici la première application : les DL sont un outils formidables 
pour calculer des limites ayant des formes indéterminées !

Le principe est simple si l'on a obtenu le DL d'une fonction, en $a$ par exemple,
alors la limite de $f$ en $a$ est tout simplement le terme constant du DL : $c_0$.

\change

Prenons l'exemple d'une fonction assez compliquée :
$\frac{\ln(1+x)-\tan x+\frac{1}{2}\sin^2x}{3x^2\sin^2x}$

On souhaite calculer la limite en $0$.

Remarquons que le numérateur vaut $0$ en $x=0$ et le dénominateur également.

Il s'agit donc d'une forme indéterminée du type ``$0/0$''.

Voyons comment les développements limités résolvent le problème.


\change

J'appelle $f(x)$ le numérateur 
et je calcule son DL en $0$ à l'ordre $4$.

\change

Tout d'abord $\ln(1+x)=x-\frac{x^2}{2}+\frac{x^3}{3}-\frac{x^4}{4}+o(x^4)$

\change

puis $\tan x=x+\frac{x^3}{3}+o(x^4)$

\change

Et enfin $\sin^2x = \big(x-\frac{x^3}{6}+o(x^3)\big)^2$

\change

on regroupe les deux premiers DL et on développe le carré.

\change

On obtient que 
$f(x)= -\frac{5}{12}x^4 + o(x^4)$

\change

Je fais la même chose avec le dénominateur 
$g(x)=3x^2\sin^2x$

\change

Son DL est $3x^2\big(x+o(x)\big)^2$

\change 

ce qui donne $3x^4+o(x^4)$

\change

Revenons à notre forme indéterminée $\frac{f(x)}{g(x)}$

elle s'écrit donc $\frac{-\frac{5}{12}x^4 + o(x^4)}{3x^4+o(x^4)}$

\change

On factorise en haut et en bas par $x^4$ pour obtenir
$\frac{-\frac{5}{12} + o(1)}{3+o(1)}$.

Ce n'est plus une forme indéterminée ! Le numérateur 
tend maintenant vers $-\frac{5}{12}$
et le dénominateur vers $3$.

Je vous rappelle que $o(1)$ désigne une fonction quelconque tendant vers $0$ quand $x \to 0$.

\change

Ainsi $\frac{f(x)}{g(x)}$ tends vers $-\frac{5}{12}$ divisé par $3$

Donc la limite recherchée est $=-\frac{5}{36}$.


Vous remarquez que l'on a du calculer des DL à l'ordre $4$ afin de trouver 
le premier terme non nul des DL de $f$ et $g$.

Si vous calculez seulement un DL à l'ordre $2$ par exemple, alors vous n'auriez pas levé l'indétermination.
C'est surtout la pratique qui vous dira jusqu'à quel ordre il faut écrire les développements.

Je vous encourage donc à faire de nombreux exemples.


%%%%%%%%%%%%%%%%%%%%%%%%%%%%%%%%%%%%%%%%%%%%%%%%%%%%%%%%%%%
\diapo


Soit $f$ une fonction admettant un DL en $a$ 
s'écrivant 
$f(x)=c_0+c_1(x-a)+c_k(x-a)^k+(x-a)^k\epsilon(x)$, 

ici $k$ est le plus petit entier $\ge2$ tel que le coefficient $c_k$ soit non nul.

\change

Alors le DL nous donne l'équation de la tangente à la courbe de $f$ en $a$ c'est  $y=c_0+c_1(x-a)$ 

\change

Mais il donne aussi 
la position de la courbe par rapport à la tangente pour $x$ proche de $a$ :

la position est donnée par le signe $f(x)-y$, c'est-à-dire le signe de $c_k(x-a)^k$.



Il y a trois types de position que peut avoir le graphe par rapport à sa tangente :

\change

Si le signe de $f(x)-y$ est positif alors la courbe est au-dessus de la tangente.

\change

  Si le signe est négatif alors la courbe est en dessous de la tangente. 

\change

Si le signe change (lorsque l'on passe de $x<a$ à $x>a$) alors la courbe traverse 
la tangente au point d'abscisse $a$. 

\change

Ce point s'appelle un \defi{point d'inflexion}.

\change

Notez que pour qu'il y ait un point d'inflexion 
il est nécessaire que le coefficient $c_2$ soit nul, c'est-à-dire $f''(a)=0$.


%%%%%%%%%%%%%%%%%%%%%%%%%%%%%%%%%%%%%%%%%%%%%%%%%%%%%%%%%%%
\diapo

Voici les dessins : dans le premier cas

$k$ est pair et $c_k$ positif donc $c_k (x-a)^k$ reste positif avant et après $a$ donc  la courbe est au-dessus de la tangente.

\change

si $k$ est pair et $c_k$ négatif alors $c_k (x-a)^k$ est négatif autour de $a$ la courbe reste sous de la tangente.

\change

Enfin si $k$ est impair alors $c_k (x-a)^k$ change de signe (positif puis négatif, ou négatif puis positif).
La courbe qui était au-dessus de la tangente passe en-dessous (ou inversement).

C'est un point d'inflexion.



%%%%%%%%%%%%%%%%%%%%%%%%%%%%%%%%%%%%%%%%%%%%%%%%%%%%%%%%%%%
\diapo


Soit $f(x)=x^4-2x^3+1$.

Déterminons la tangente du graphe de $f$ au point d'abscisse $\frac{1}{2}$  
et précisons la position du graphe par rapport à cette tangente.

\change

Pour calculer le DL on calcule d'abord les dérivées successives 

On a $f'(x)=4x^3-6x^2$

\change

$f''(x)=12x^2-12x$, donc $f''(\frac{1}{2}) = -3 \neq 0$ et donc le rang du premier coefficient non nul 
de la proposition précédente est ici $k=2$.

\change

la formule de Taylor-Young nous donne alors le le DL de $f$ en $\frac{1}{2}$ à l'ordre $2$

\change

$f(x)=f(\frac12)+f'(\frac12)(x-\frac12)+\frac{f''(\frac12)}{2!}(x-\frac12)^2 +(x-\frac12)^2 \epsilon(x)$

\change

ce qui donne ici 

$f(x)==\frac{13}{16} -(x- \frac12) -\frac{3}{2}(x-\frac12)^2 + (x-\frac12)^2 \epsilon(x)$.

\change

Les deux premiers termes du DL nous donne l'équation de la tangente :
$y= \frac{13}{16} -(x- \frac12)$

\change

Et le premier terme non nul suivant nous donne la position du graphe par rapport à la tangente :

Ici $f(x)-y$ (l'équation de la tangente) vaut $-\frac32(x-\frac12)^2$ + un petit reste.

Donc cette différence est négative autour de $x=\frac12$. 

\change

Ce qui signifie que le graphe de $f$ est en dessous de la tangente au voisinage de $x=\frac12$.

%%%%%%%%%%%%%%%%%%%%%%%%%%%%%%%%%%%%%%%%%%%%%%%%%%%%%%%%%%%
\diapo

Cela se vérifie sur le dessin :
voici le graphe de notre fonction $f$.

\change

Et lorsque l'on trace le graphe de la tangente au point d'abscisse $1/2$,
effectivement la graphe de la courbe reste sous la tangente,
ceci autour des $x=1/2$.


%%%%%%%%%%%%%%%%%%%%%%%%%%%%%%%%%%%%%%%%%%%%%%%%%%%%%%%%%%%
\diapo

Reprenons la même fonction $f$ et déterminons ses points d'inflexion.


Je vous rappelle qu'un point d'inflexion est un point où le graphe d'une fonction
traverse la tangente.

\change

Et que les points d'inflexion sont à chercher parmi les solutions de $f''(a)=0$. 

\change

Ici $f''(x)=12x^2-12x$,  $a=0$ ou $a=1$. 

Etudions chacun de ces cas.


\change

Tout d'abord en $0$
Le DL en $0$ est $f(x)= 1-2x^3+x^4$ 

Comme $f$ est un polynôme il s'agit juste 
d'écrire les monômes par degrés croissants !

\change

L'équation de la tangente au point d'abscisse $0$ est donc $y=1$ 

c'est une tangente horizontale

\change


Comme $f(x)-y=-2x^3$ + un reste,  alors le signe change en $0$ 

\change

Donc $0$ est bien un point d'inflexion. 

On peut même être plus précis : juste avant $0$, la différence est positive donc le graphe est au-dessus de la tangente
et juste après la différence devient négative et le graphe passe en-dessous de la tangente.

\change

On fait maintenant le même travail en $1$.

on calcule $f(1)$, $f'(1)$,...
pour trouver le DL en $1$

\change

La tangente en $1$ a donc pour équation $y=-2(x-1)$

\change

Maintenant $f(x)-y = 2(x-1)^3$ + un reste, change de signe en $1$, 

\change

$1$ est aussi un point d'inflexion de $f$. 

Ici le graphe est d'abord dessous puis dessus.

%%%%%%%%%%%%%%%%%%%%%%%%%%%%%%%%%%%%%%%%%%%%%%%%%%%%%%%%%%%
\diapo

Sur le graphe de $f$ vérifions graphiquement ce que l'on vient de démontrer :

\change

la courbe traverse bien la tangente $0$, c'est bien un point d'inflexion

\change

même chose en $1$.


%%%%%%%%%%%%%%%%%%%%%%%%%%%%%%%%%%%%%%%%%%%%%%%%%%%%%%%%%%%
\diapo

Nous savons ce qu'est un développement limité en un point $a$.
Voyons ce qu'est un développement limité en $+\infty$.

On parle aussi de développement asymptotique.

Prenons une fonction $f$ définie sur un intervalle qui va jusqu'à l'infini.

On dit que $f$ admet un \defi{DL en $+\infty$} à l'ordre $n$
s'il existe des réels $c_0,c_1,\ldots,c_n$ tels que 
$$f(x)=c_0+\frac{c_1}{x}+\frac{c_2}{x^2}+\cdots+\frac{c_n}{x^n} 
+\frac{1}{x^n}\epsilon\big(\frac{1}{x}\big)$$

\change

où $\epsilon\big(\frac{1}{x}\big)$ tend vers $0$ quand $x\to+\infty$.


\change

Notez que lorsque $x\to +\infty$ alors $\frac{1}{x} \to 0$.

Donc dire que la fonction $f(x)$ admet un DL en $+\infty$ à l'ordre $n$ est équivalent à 
dire que la fonction $f(\frac{1}{x})$ admet un DL en $0^+$ à l'ordre $n$.

\change

Comme exemple nous calculons le DL en $+\infty$ de la fonction $\ln\big(2+\frac{1}{x}\big)$.

\change

dont voici le graphe.

\change

On se ramène d'abord à une forme du type $\ln(1+u)$ en factorisant par $2$
 ce qui donne $\ln2+\ln\big(1+\frac{1}{2x}\big)$

\change

Maintenant lorsque $x\to+\infty$ alors $\frac{1}{2x} \to 0$ donc en posant
$u=\frac{1}{2x}$ on obtient une expression en $\ln(1+u)$
dont le DL pour $u$ proche de $0$ est $u-\frac{u^2}{2}+\frac{u^3}{3}...$

On remplace $u$ par $\frac{1}{2x}$

Pour obtenir ce développement en $+\infty$.


\change

Cela nous permet d'avoir une idée assez précise du comportement de $f$ au voisinage de $+\infty$.

Lorsque $x\to +\infty$ alors $f(x)\to \ln2$.

Et le second terme est $+\frac 12x$, donc est positif, cela signifie
que la fonction $f(x)$ tend vers $\ln 2$ tout en restant au-dessus de $\ln 2$.


%%%%%%%%%%%%%%%%%%%%%%%%%%%%%%%%%%%%%%%%%%%%%%%%%%%%%%%%%%%
\diapo

Approfondissons le comportement à l'infini vu dans l'exemple précédent.

On suppose qu'une fonction $\frac{f(x)}{x}$
admet un DL en  $+\infty$ qui s'écrit sous la forme :
$\frac{f(x)}{x}= a_0 +\frac{a_1}{x}+\frac{a_k}{x^k}+\frac{1}{x^k}\epsilon(\frac{1}{x}),$  

où $k$ est le plus petit entier $\ge 2$ tel que le coefficient de  $\frac{1}{x^k}$ soit
non nul. 

\change

Alors $f(x)-(a_0x+a_1) \to 0$ lorsque $x\to +\infty$

cela signifie que la droite d'équation $y= a_0x+a_1$ est une \defi{asymptote} 
à la courbe de $f$ en $+\infty$ 

\change

Voici une représentation graphique

\change

En plus on sait comment se comporte la courbe :

La position de la courbe par rapport à 
l'asymptote est donnée par le signe de $f(x)-y$, 
c'est-à-dire le signe de $\frac{a_k}{x^{k-1}}$.

%%%%%%%%%%%%%%%%%%%%%%%%%%%%%%%%%%%%%%%%%%%%%%%%%%%%%%%%%%%
\diapo

On termine avec le calcul de l'asymptote de  $f(x)=\exp{\frac1x} \cdot \sqrt{x^2-1}$
en $+\infty$.

\change

Dont voici le graphe


\change

Il s'agit donc de calculer le DL de $\frac{f(x)}{x}$

\change 

qui est $\exp{\frac1x} \cdot\frac{\sqrt{x^2-1}}{x}$

\change

et qui s'écrit ainsi après avoir passer le $x$ à l'intérieur de la racine.

\change

On pose $u=\frac1x$, comme $x\to +\infty$ alors $u\to 0$

on applique le DL de $\exp u$

et celui $\sqrt{1-u^2}$

\change

On multiplie les DL et on regroupe les termes pour obtenir que 

 $\frac{f(x)}{x} = 1+\frac{1}{x}-\frac{1}{3x^3} +\frac{1}{x^3}\epsilon(\frac{1}{x}) $


\change

Ainsi le graphe de $f$ admet un asymptote en $+\infty$ qui
la droite d`'équation $y=x+1$ : en effet avec cette équation $f(x)-y \to 0$.

\change

Voici son tracé.

\change

Et comme $f(x)-y=-\frac{1}{3x^2}$ + des termes négligeables 

\change

alors cette différence est négative (pour $x$ assez grand)
et donc le graphe de $f$ reste en dessous de l'asymptote
tout en s'en rapprochant.

\change

On montrerait de la même façon qu'il existe un asymptote en $-\infty$ qui est d'équation $y=-x-1$.

%%%%%%%%%%%%%%%%%%%%%%%%%%%%%%%%%%%%%%%%%%%%%%%%%%%%%%%%%%%
\diapo

Comme d'habitude voici les mini-exercices !


\end{document}