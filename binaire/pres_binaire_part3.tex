
%%%%%%%%%%%%%%%%%% PREAMBULE %%%%%%%%%%%%%%%%%%

\documentclass[aspectratio=169,utf8]{beamer}
%\documentclass[aspectratio=169,handout]{beamer}

\usetheme{Boadilla}
%\usecolortheme{seahorse}
\usecolortheme[RGB={245,66,24}]{structure}
\useoutertheme{infolines}

% packages
\usepackage{amsfonts,amsmath,amssymb,amsthm}
\usepackage[utf8]{inputenc}
\usepackage[T1]{fontenc}
\usepackage{lmodern}

\usepackage[francais]{babel}
\usepackage{fancybox}
\usepackage{graphicx}

\usepackage{float}
\usepackage{xfrac}

%\usepackage[usenames, x11names]{xcolor}
\usepackage{tikz}
\usepackage{pgfplots}
\usepackage{datetime}



%-----  Package unités -----
\usepackage{siunitx}
\sisetup{locale = FR,detect-all,per-mode = symbol}

%\usepackage{mathptmx}
%\usepackage{fouriernc}
%\usepackage{newcent}
%\usepackage[mathcal,mathbf]{euler}

%\usepackage{palatino}
%\usepackage{newcent}
% \usepackage[mathcal,mathbf]{euler}



% \usepackage{hyperref}
% \hypersetup{colorlinks=true, linkcolor=blue, urlcolor=blue,
% pdftitle={Exo7 - Exercices de mathématiques}, pdfauthor={Exo7}}


%section
% \usepackage{sectsty}
% \allsectionsfont{\bf}
%\sectionfont{\color{Tomato3}\upshape\selectfont}
%\subsectionfont{\color{Tomato4}\upshape\selectfont}

%----- Ensembles : entiers, reels, complexes -----
\newcommand{\Nn}{\mathbb{N}} \newcommand{\N}{\mathbb{N}}
\newcommand{\Zz}{\mathbb{Z}} \newcommand{\Z}{\mathbb{Z}}
\newcommand{\Qq}{\mathbb{Q}} \newcommand{\Q}{\mathbb{Q}}
\newcommand{\Rr}{\mathbb{R}} \newcommand{\R}{\mathbb{R}}
\newcommand{\Cc}{\mathbb{C}} 
\newcommand{\Kk}{\mathbb{K}} \newcommand{\K}{\mathbb{K}}

%----- Modifications de symboles -----
\renewcommand{\epsilon}{\varepsilon}
\renewcommand{\Re}{\mathop{\text{Re}}\nolimits}
\renewcommand{\Im}{\mathop{\text{Im}}\nolimits}
%\newcommand{\llbracket}{\left[\kern-0.15em\left[}
%\newcommand{\rrbracket}{\right]\kern-0.15em\right]}

\renewcommand{\ge}{\geqslant}
\renewcommand{\geq}{\geqslant}
\renewcommand{\le}{\leqslant}
\renewcommand{\leq}{\leqslant}
\renewcommand{\epsilon}{\varepsilon}

%----- Fonctions usuelles -----
\newcommand{\ch}{\mathop{\text{ch}}\nolimits}
\newcommand{\sh}{\mathop{\text{sh}}\nolimits}
\renewcommand{\tanh}{\mathop{\text{th}}\nolimits}
\newcommand{\cotan}{\mathop{\text{cotan}}\nolimits}
\newcommand{\Arcsin}{\mathop{\text{arcsin}}\nolimits}
\newcommand{\Arccos}{\mathop{\text{arccos}}\nolimits}
\newcommand{\Arctan}{\mathop{\text{arctan}}\nolimits}
\newcommand{\Argsh}{\mathop{\text{argsh}}\nolimits}
\newcommand{\Argch}{\mathop{\text{argch}}\nolimits}
\newcommand{\Argth}{\mathop{\text{argth}}\nolimits}
\newcommand{\pgcd}{\mathop{\text{pgcd}}\nolimits} 


%----- Commandes divers ------
\newcommand{\ii}{\mathrm{i}}
\newcommand{\dd}{\text{d}}
\newcommand{\id}{\mathop{\text{id}}\nolimits}
\newcommand{\Ker}{\mathop{\text{Ker}}\nolimits}
\newcommand{\Card}{\mathop{\text{Card}}\nolimits}
\newcommand{\Vect}{\mathop{\text{Vect}}\nolimits}
\newcommand{\Mat}{\mathop{\text{Mat}}\nolimits}
\newcommand{\rg}{\mathop{\text{rg}}\nolimits}
\newcommand{\tr}{\mathop{\text{tr}}\nolimits}


%----- Structure des exercices ------

\newtheoremstyle{styleexo}% name
{2ex}% Space above
{3ex}% Space below
{}% Body font
{}% Indent amount 1
{\bfseries} % Theorem head font
{}% Punctuation after theorem head
{\newline}% Space after theorem head 2
{}% Theorem head spec (can be left empty, meaning ‘normal’)

%\theoremstyle{styleexo}
\newtheorem{exo}{Exercice}
\newtheorem{ind}{Indications}
\newtheorem{cor}{Correction}


\newcommand{\exercice}[1]{} \newcommand{\finexercice}{}
%\newcommand{\exercice}[1]{{\tiny\texttt{#1}}\vspace{-2ex}} % pour afficher le numero absolu, l'auteur...
\newcommand{\enonce}{\begin{exo}} \newcommand{\finenonce}{\end{exo}}
\newcommand{\indication}{\begin{ind}} \newcommand{\finindication}{\end{ind}}
\newcommand{\correction}{\begin{cor}} \newcommand{\fincorrection}{\end{cor}}

\newcommand{\noindication}{\stepcounter{ind}}
\newcommand{\nocorrection}{\stepcounter{cor}}

\newcommand{\fiche}[1]{} \newcommand{\finfiche}{}
\newcommand{\titre}[1]{\centerline{\large \bf #1}}
\newcommand{\addcommand}[1]{}
\newcommand{\video}[1]{}

% Marge
\newcommand{\mymargin}[1]{\marginpar{{\small #1}}}

\def\noqed{\renewcommand{\qedsymbol}{}}


%----- Presentation ------
\setlength{\parindent}{0cm}

%\newcommand{\ExoSept}{\href{http://exo7.emath.fr}{\textbf{\textsf{Exo7}}}}

\definecolor{myred}{rgb}{0.93,0.26,0}
\definecolor{myorange}{rgb}{0.97,0.58,0}
\definecolor{myyellow}{rgb}{1,0.86,0}

\newcommand{\LogoExoSept}[1]{  % input : echelle
{\usefont{U}{cmss}{bx}{n}
\begin{tikzpicture}[scale=0.1*#1,transform shape]
  \fill[color=myorange] (0,0)--(4,0)--(4,-4)--(0,-4)--cycle;
  \fill[color=myred] (0,0)--(0,3)--(-3,3)--(-3,0)--cycle;
  \fill[color=myyellow] (4,0)--(7,4)--(3,7)--(0,3)--cycle;
  \node[scale=5] at (3.5,3.5) {Exo7};
\end{tikzpicture}}
}


\newcommand{\debutmontitre}{
  \author{} \date{} 
  \thispagestyle{empty}
  \hspace*{-10ex}
  \begin{minipage}{\textwidth}
    \titlepage  
  \vspace*{-2.5cm}
  \begin{center}
    \LogoExoSept{2.5}
  \end{center}
  \end{minipage}

  \vspace*{-0cm}
  
  % Astuce pour que le background ne soit pas discrétisé lors de la conversion pdf -> png
\begin{tikzpicture}
        \fill[opacity=0,green!60!black] (0,0)--++(0,0)--++(0,0)--++(0,0)--cycle; 
\end{tikzpicture}

% toc S'affiche trop tot :
% \tableofcontents[hideallsubsections, pausesections]
}

\newcommand{\finmontitre}{
  \end{frame}
  \setcounter{framenumber}{0}
} % ne marche pas pour une raison obscure

%----- Commandes supplementaires ------

% \usepackage[landscape]{geometry}
% \geometry{top=1cm, bottom=3cm, left=2cm, right=10cm, marginparsep=1cm
% }
% \usepackage[a4paper]{geometry}
% \geometry{top=2cm, bottom=2cm, left=2cm, right=2cm, marginparsep=1cm
% }

%\usepackage{standalone}


% New command Arnaud -- november 2011
\setbeamersize{text margin left=24ex}
% si vous modifier cette valeur il faut aussi
% modifier le decalage du titre pour compenser
% (ex : ici =+10ex, titre =-5ex

\theoremstyle{definition}
%\newtheorem{proposition}{Proposition}
%\newtheorem{exemple}{Exemple}
%\newtheorem{theoreme}{Théorème}
%\newtheorem{lemme}{Lemme}
%\newtheorem{corollaire}{Corollaire}
%\newtheorem*{remarque*}{Remarque}
%\newtheorem*{miniexercice}{Mini-exercices}
%\newtheorem{definition}{Définition}

% Commande tikz
\usetikzlibrary{calc}
\usetikzlibrary{patterns,arrows}
\usetikzlibrary{matrix}
\usetikzlibrary{fadings} 

%definition d'un terme
\newcommand{\defi}[1]{{\color{myorange}\textbf{\emph{#1}}}}
\newcommand{\evidence}[1]{{\color{blue}\textbf{\emph{#1}}}}
\newcommand{\assertion}[1]{\emph{\og#1\fg}}  % pour chapitre logique
%\renewcommand{\contentsname}{Sommaire}
\renewcommand{\contentsname}{}
\setcounter{tocdepth}{2}



%------ Figures ------

\def\myscale{1} % par défaut 
\newcommand{\myfigure}[2]{  % entrée : echelle, fichier figure
\def\myscale{#1}
\begin{center}
\footnotesize
{#2}
\end{center}}


%------ Encadrement ------

\usepackage{fancybox}


\newcommand{\mybox}[1]{
\setlength{\fboxsep}{7pt}
\begin{center}
\shadowbox{#1}
\end{center}}

\newcommand{\myboxinline}[1]{
\setlength{\fboxsep}{5pt}
\raisebox{-10pt}{
\shadowbox{#1}
}
}

%--------------- Commande beamer---------------
\newcommand{\beameronly}[1]{#1} % permet de mettre des pause dans beamer pas dans poly


\setbeamertemplate{navigation symbols}{}
\setbeamertemplate{footline}  % tiré du fichier beamerouterinfolines.sty
{
  \leavevmode%
  \hbox{%
  \begin{beamercolorbox}[wd=.333333\paperwidth,ht=2.25ex,dp=1ex,center]{author in head/foot}%
    % \usebeamerfont{author in head/foot}\insertshortauthor%~~(\insertshortinstitute)
    \usebeamerfont{section in head/foot}{\bf\insertshorttitle}
  \end{beamercolorbox}%
  \begin{beamercolorbox}[wd=.333333\paperwidth,ht=2.25ex,dp=1ex,center]{title in head/foot}%
    \usebeamerfont{section in head/foot}{\bf\insertsectionhead}
  \end{beamercolorbox}%
  \begin{beamercolorbox}[wd=.333333\paperwidth,ht=2.25ex,dp=1ex,right]{date in head/foot}%
    % \usebeamerfont{date in head/foot}\insertshortdate{}\hspace*{2em}
    \insertframenumber{} / \inserttotalframenumber\hspace*{2ex} 
  \end{beamercolorbox}}%
  \vskip0pt%
}


\definecolor{mygrey}{rgb}{0.5,0.5,0.5}
\setlength{\parindent}{0cm}
%\DeclareTextFontCommand{\helvetica}{\fontfamily{phv}\selectfont}

% background beamer
\definecolor{couleurhaut}{rgb}{0.85,0.9,1}  % creme
\definecolor{couleurmilieu}{rgb}{1,1,1}  % vert pale
\definecolor{couleurbas}{rgb}{0.85,0.9,1}  % blanc
\setbeamertemplate{background canvas}[vertical shading]%
[top=couleurhaut,middle=couleurmilieu,midpoint=0.4,bottom=couleurbas] 
%[top=fondtitre!05,bottom=fondtitre!60]



\makeatletter
\setbeamertemplate{theorem begin}
{%
  \begin{\inserttheoremblockenv}
  {%
    \inserttheoremheadfont
    \inserttheoremname
    \inserttheoremnumber
    \ifx\inserttheoremaddition\@empty\else\ (\inserttheoremaddition)\fi%
    \inserttheorempunctuation
  }%
}
\setbeamertemplate{theorem end}{\end{\inserttheoremblockenv}}

\newenvironment{theoreme}[1][]{%
   \setbeamercolor{block title}{fg=structure,bg=structure!40}
   \setbeamercolor{block body}{fg=black,bg=structure!10}
   \begin{block}{{\bf Th\'eor\`eme }#1}
}{%
   \end{block}%
}


\newenvironment{proposition}[1][]{%
   \setbeamercolor{block title}{fg=structure,bg=structure!40}
   \setbeamercolor{block body}{fg=black,bg=structure!10}
   \begin{block}{{\bf Proposition }#1}
}{%
   \end{block}%
}

\newenvironment{corollaire}[1][]{%
   \setbeamercolor{block title}{fg=structure,bg=structure!40}
   \setbeamercolor{block body}{fg=black,bg=structure!10}
   \begin{block}{{\bf Corollaire }#1}
}{%
   \end{block}%
}

\newenvironment{mydefinition}[1][]{%
   \setbeamercolor{block title}{fg=structure,bg=structure!40}
   \setbeamercolor{block body}{fg=black,bg=structure!10}
   \begin{block}{{\bf Définition} #1}
}{%
   \end{block}%
}

\newenvironment{lemme}[0]{%
   \setbeamercolor{block title}{fg=structure,bg=structure!40}
   \setbeamercolor{block body}{fg=black,bg=structure!10}
   \begin{block}{\bf Lemme}
}{%
   \end{block}%
}

\newenvironment{remarque}[1][]{%
   \setbeamercolor{block title}{fg=black,bg=structure!20}
   \setbeamercolor{block body}{fg=black,bg=structure!5}
   \begin{block}{Remarque #1}
}{%
   \end{block}%
}


\newenvironment{exemple}[1][]{%
   \setbeamercolor{block title}{fg=black,bg=structure!20}
   \setbeamercolor{block body}{fg=black,bg=structure!5}
   \begin{block}{{\bf Exemple }#1}
}{%
   \end{block}%
}


\newenvironment{miniexercice}[0]{%
   \setbeamercolor{block title}{fg=structure,bg=structure!20}
   \setbeamercolor{block body}{fg=black,bg=structure!5}
   \begin{block}{Mini-exercices}
}{%
   \end{block}%
}


\newenvironment{tp}[0]{%
   \setbeamercolor{block title}{fg=structure,bg=structure!40}
   \setbeamercolor{block body}{fg=black,bg=structure!10}
   \begin{block}{\bf Travaux pratiques}
}{%
   \end{block}%
}
\newenvironment{exercicecours}[1][]{%
   \setbeamercolor{block title}{fg=structure,bg=structure!40}
   \setbeamercolor{block body}{fg=black,bg=structure!10}
   \begin{block}{{\bf Exercice }#1}
}{%
   \end{block}%
}
\newenvironment{algo}[1][]{%
   \setbeamercolor{block title}{fg=structure,bg=structure!40}
   \setbeamercolor{block body}{fg=black,bg=structure!10}
   \begin{block}{{\bf Algorithme}\hfill{\color{gray}\texttt{#1}}}
}{%
   \end{block}%
}


\setbeamertemplate{proof begin}{
   \setbeamercolor{block title}{fg=black,bg=structure!20}
   \setbeamercolor{block body}{fg=black,bg=structure!5}
   \begin{block}{{\footnotesize Démonstration}}
   \footnotesize
   \smallskip}
\setbeamertemplate{proof end}{%
   \end{block}}
\setbeamertemplate{qed symbol}{\openbox}


\makeatother
\usecolortheme[RGB={245,66,24}]{structure}

% Commande spécifique à ce chapitre

\newcommand{\Python}{\texttt{Python}}

\usepackage{textcomp}

\usepackage{listings}
\lstset{
  literate={é}{{\'e}}1
           {è}{{\`e}}1
           {à}{{\`a}}1
}


\newcommand{\codeinline}[1]{\lstinline!#1!}

% Black and white
\lstset{
  upquote=true,
  columns=flexible,
  keepspaces=true,
  basicstyle=\ttfamily,
  commentstyle=\color{gray},
  language=Python,
  showstringspaces=false,
  aboveskip=0em,  
  belowskip=0em,
    literate={>>>}{\textcolor{red}{>\,\!>\,\!>}}{3},
  escapeinside=||
}


% Color
\lstset{
  language=Python,
  upquote=true,
  columns=flexible,
  keepspaces=true,
  basicstyle=\ttfamily,
  commentstyle=\color{gray},
  keywordstyle=\color{blue},
  %emph={bin,oct,hex},
  emphstyle=\color{blue},
  stringstyle=\color{green!60!black},
%  frame=single,  
  showspaces=false,
  showstringspaces=false,
  literate={>>>}{\textcolor{red}{>\,\!>\,\!>}}{3},
  escapeinside=||
}

\renewcommand{\debutmontitre}{
  \author{} \date{} 
  \thispagestyle{empty}
  \hspace*{-10ex}
  \begin{minipage}{\textwidth}
    \titlepage  
  \vspace*{-1.5cm}
%   \begin{center}
%     \LogoExoSept{2.5}
%   \end{center}
  \end{minipage}

  \vspace*{-0cm}
  
  % Astuce pour que le background ne soit pas discrétisé lors de la conversion pdf -> png
\begin{tikzpicture}
        \fill[opacity=0,green!60!black] (0,0)--++(0,0)--++(0,0)--++(0,0)--cycle; 
\end{tikzpicture}

% toc S'affiche trop tot :
% \tableofcontents[hideallsubsections, pausesections]
}

%%%%%%%%%%%%%%%%%%%%%%%%%%%%%%%%%%%%%%%%%%%%%%%%%%%%%%%%%%%%%
%%%%%%%%%%%%%%%%%%%%%%%%%%%%%%%%%%%%%%%%%%%%%%%%%%%%%%%%%%%%%


\begin{document}


\title{{\bf Représentation des nombres}}
\subtitle{Les bases $8$ et $16$}

\begin{frame}
  
  \debutmontitre

  \pause

{\footnotesize
\hfil\qquad\qquad\qquad\qquad
\setbeamercovered{transparent=50}
\begin{minipage}{0.6\textwidth}
  \begin{itemize}
    \item<3-> L'octal
    \item<4-> L'hexadécimal
    \item<5-> Intérêt de ces bases
  \end{itemize}
\end{minipage}
}

\end{frame}

\setcounter{framenumber}{0}

%%%%%%%%%%%%%%%%%%%%%%%%%%%%%%%%%%%%%%%%%%%%%%%%%%%%%%%%%%%%%%%%
\section{Motivation}

\begin{frame}

\evidence{Les bases $8$ et $16$}

\pause  
\bigskip

En informatique deux autres bases sont utilisées

\begin{itemize}
\pause
\item 
\begin{itemize}
\item la base $8$
\pause
\item les huit chiffres de \codeinline{0} à \codeinline{7}
\pause
\item écriture \defi{octale}
\end{itemize}


\pause  
\item 
\begin{itemize}
\item la base $16$
\pause
\item les dix chiffres usuels de \codeinline{0} à
  \codeinline{9} et les six premières lettres de l'alphabet de \codeinline{A} à
  \codeinline{F}
\pause
\item écriture \defi{hexadécimale}
\end{itemize}
\end{itemize}

\end{frame}


%%%%%%%%%%%%%%%%%%%%%%%%%%%%%%%%%%%%%%%%%%%%%%%%%%%%%%%%%%%%%%%%
\section{L'octal}

\begin{frame}

\evidence{Base $8$}
  
\medskip

\pause

\textbf{Exemple avec $n=47$}

  \begin{itemize}
    \pause
  \item Divisions successives par 8
    $$\begin{array}{rcl}
  47 &=& 8\times 5 + 7\\
  5  &=& 8\times 0 + 5
  \end{array}$$

\pause
  \item $47 = \textcolor{red}{5}\times 8^1 + \textcolor{red}{7}\times 8^0$
  
    \pause
  \item $47 = \overline{\textcolor{red}{\mathtt{57}}}_8$
  \end{itemize}
  
\medskip

\pause
\textbf{Exemple avec $n=3010$}
  \begin{itemize}
\pause
  \item
  $$\begin{array}{rcl}
  3010 &=& 8\times 376 + 2\\
  376  &=& 8\times 47 + 0\\
  47   &=& 8\times 5 + 7\\
  5    &=& 8\times 0 + 5
  \end{array}$$
\pause
  
  \item $3010 = \textcolor{red}{5}\times 8^3 + \textcolor{red}{7}\times 8^2 + \textcolor{red}{0}\times 8^1 + \textcolor{red}{2}\times 8^0$
  
\pause
  \item $3010 = \overline{\textcolor{red}{\mathtt{5702}}}_8$
  \end{itemize}

\end{frame}

\begin{frame}
  \evidence{Les nombres de 0 à 20 en octal}

  \bigskip
  
  \begin{center}
    \begin{tabular}[ht]{rrp{2cm}rr}
      0 & $\overline{0}_{8}$ & &10 & $\overline{12}_{8}$\\
      1 & $\overline{1}_{8}$ & &11 & $\overline{13}_{8}$\\
      2 & $\overline{2}_{8}$ & &12 & $\overline{14}_{8}$\\
      3 & $\overline{3}_{8}$ & &13 & $\overline{15}_{8}$\\
      4 & $\overline{4}_{8}$ & &14 & $\overline{16}_{8}$\\
      5 & $\overline{5}_{8}$ & &15 & $\overline{17}_{8}$\\
      6 & $\overline{6}_{8}$ & &16 & $\overline{20}_{8}$\\
      7 & $\overline{7}_{8}$ & &17 & $\overline{21}_{8}$\\
      8 & $\overline{10}_{8}$ & &18 & $\overline{22}_{8}$\\
      9 & $\overline{11}_{8}$ & &19 & $\overline{23}_{8}$\\
      &  & &20 & $\overline{24}_{8}$\\
    \end{tabular}
  \end{center}
\end{frame}

\begin{frame}[fragile]

\evidence{L'octal en \Python}

\begin{itemize}
  \pause
  \item L'analogue de \codeinline{bin} pour l'octal est \codeinline{oct}
  \pause
  \item Préfixe \codeinline{0o} ou \codeinline{-0o}
\end{itemize}
  \pause
  
\begin{lstlisting}
>>> oct(47)
'0o57'|\pause|
>>> oct(3010)
'0o5702'|\pause|
>>> oct(-3010)
'-0o5702'|\pause|
\end{lstlisting}

\begin{lstlisting}
>>> 0o57
47|\pause|
>>> 2 * 0o57 
94
\end{lstlisting}
\end{frame}

%%%%%%%%%%%%%%%%%%%%%%%%%%%%%%%%%%%%%%%%%%%%%%%%%%%%%%%%%%%%%%%%
\section{L'hexadécimal}


\begin{frame}

\evidence{Base $16$}


\pause
\textbf{Exemple avec $n=47$}

  \begin{itemize}
    \pause
  \item Divisions successives par 16
    $$\begin{array}{rcl}
  47 &=& 16\times 2 + 15\\
  2  &=& 16\times 0 + 2
  \end{array}$$
  \vspace*{-3ex}
  
  \pause
  \item $47 = \textcolor{red}{2}\times 16^1 + \textcolor{red}{15}\times 16^0$
  \pause
  \item $47 = \overline{\textcolor{red}{\mathtt{2F}}}_{16}$
  \quad (\codeinline{F} pour représenter $15$)
  \end{itemize}
  
\medskip

\pause
\textbf{Exemple avec $n=3010$}
  \begin{itemize}

  \item  $$\begin{array}{rcl}
  3010 &=& 16\times 188 + 2\\
  188  &=& 16\times 11 + 12\\
  11   &=& 16\times 0 + 11
  \end{array}$$  \vspace*{-1ex}
  
  \pause
  \item $3010 = \textcolor{red}{11}\times 16^2 + \textcolor{red}{12}\times 16^1 + \textcolor{red}{2}\times 16^0$
    \pause
  \item $3010 = \overline{\textcolor{red}{\mathtt{BC2}}}_{16}$  
 \quad (\codeinline{B} pour $11$, \codeinline{C} pour $12$)  
  \end{itemize}

\end{frame}

\begin{frame}
  \evidence{Les nombres de 0 à 20 en hexadécimal}

  \bigskip
  
  \begin{center}
    \begin{tabular}[ht]{rrp{2cm}rr}
      0 & $\overline{0}_{16}$ & &10 & $\overline{\mathtt{A}}_{16}$\\
      1 & $\overline{1}_{16}$ & &11 & $\overline{\mathtt{B}}_{16}$\\
      2 & $\overline{2}_{16}$ & &12 & $\overline{\mathtt{C}}_{16}$\\
      3 & $\overline{3}_{16}$ & &13 & $\overline{\mathtt{D}}_{16}$\\
      4 & $\overline{4}_{16}$ & &14 & $\overline{\mathtt{E}}_{16}$\\
      5 & $\overline{5}_{16}$ & &15 & $\overline{\mathtt{F}}_{16}$\\
      6 & $\overline{6}_{16}$ & &16 & $\overline{10}_{16}$\\
      7 & $\overline{7}_{16}$ & &17 & $\overline{11}_{16}$\\
      8 & $\overline{8}_{16}$ & &18 & $\overline{12}_{16}$\\
      9 & $\overline{9}_{16}$ & &19 & $\overline{13}_{16}$\\
      &  & &20 & $\overline{14}_{16}$\\
    \end{tabular}
  \end{center}
\end{frame}


\begin{frame}[fragile]

\evidence{L'hexadécimal en \Python}


\begin{itemize}
  \pause
  \item L'équivalent de \codeinline{bin} et \codeinline{oct} pour l'hexadécimal est \codeinline{hex}
    \pause
  \item Préfixe \codeinline{0x}
\end{itemize}
  
 \pause 
\begin{lstlisting}
>>> hex(47)
'0x2f'|\pause|
>>> hex(3010)
'0xbc2'|\pause|
>>> hex(-3010)
'-0xbc2'|\pause|  
\end{lstlisting}

\begin{lstlisting}
>>> 0x2f
47|\pause|
>>> 100 - 0x2f
53
\end{lstlisting}


\end{frame}

%%%%%%%%%%%%%%%%%%%%%%%%%%%%%%%%%%%%%%%%%%%%%%%%%%%%%%%%%%%%%%%%
\section{Intérêt de ces bases}


\begin{frame}


\evidence{Conversion binaire/octal}

 \begin{itemize}
   \pause
   \item $8 = 2^3$
     \pause
   \item Un entier entre $0$ et $7$ s'écrit en binaire à l'aide de $3$ bits
     \pause
   \item Pour passer du binaire à l'octal
     \begin{itemize}
       \pause
     \item regrouper les bits par paquets de $3$ en commençant par la droite
       \pause
     \item puis  convertir chacun de ces paquets en un chiffre octal
     \end{itemize}
     \pause
  \item $n = \overline{\mathtt{101111}}_2 =  \overbrace{\mathtt{101}}^{\textcolor{red}{\mathtt{5}}}\overbrace{\mathtt{111}}^{\textcolor{red}{\mathtt{7}}} = \overline{\textcolor{red}{\mathtt{57}}}_8$
 \end{itemize}



\end{frame}

\begin{frame}
  \evidence{Conversion binaire/hexadécimal}

 \begin{itemize}
   \pause
   \item $16 = 2^4$
     \pause
   \item Tout entier n’excédant pas $15$ a une écriture
binaire sur quatre bits
   \pause
   \item Pour passer du binaire à l'hexadécimal
     \begin{itemize}
       \pause
     \item  regrouper les bits par paquets de quatre en commençant par la droite
       \pause
     \item puis convertir chaque paquet en un chiffre hexadécimal
     \end{itemize}
     \pause
  \item $n = \overline{101111}_{2}= \overbrace{\mathtt{0010}}^{\textcolor{red}{\mathtt{2}}}\overbrace{\mathtt{1111}}^{\textcolor{red}{\mathtt{F}}} = \overline{\textcolor{red}{\mathtt{2F}}}_{16}$

   
 \end{itemize}

\end{frame}
\end{document}
