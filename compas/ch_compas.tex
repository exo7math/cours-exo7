
\documentclass[class=report,crop=false]{standalone}
\usepackage[screen]{../exo7book}


% Spiral of Theodorus  tex.stackexchange.com  "Manuel"
% Importation ne fonctionne plus !!
% \tikzinput{Fig-compas/fig_compas18}

\begin{document}


% Commandes specifiques a ce chapitre
\newcommand{\construc}{\mathcal{C}}
\newcommand{\plan}{\mathcal{P}}
\newcommand{\cercle}{\mathcal{C}}



%====================================================================
\chapitre{La règle et le compas}
%====================================================================


\insertvideo{cz2orZmwIqY}{partie 1. Constructions}

\insertvideo{lFHZWcIkcXY}{partie 2. Nombres constructibles}

\insertvideo{Ic4LCvrmCLw}{partie 3. \'Eléments de théorie des corps}

\insertvideo{pKcQKRLzpmA}{partie 4. Corps et nombres constructibles}

\insertvideo{WqOF9IHkD3M}{partie 5. Applications aux problèmes grecs}

\bigskip
\bigskip


Vous avez à votre disposition une règle et un compas et bien sûr du papier 
et un crayon ! Avec si peu de matériel s'ouvre à vous un monde merveilleux 
rempli de géométrie et d'algèbre.

%%%%%%%%%%%%%%%%%%%%%%%%%%%%%%%%%%%%%%%%%%%%%%%%%%%%%%%%%%%%%%%%
\section{Constructions et les trois problèmes grecs}

Nous allons voir dans cette première partie que tout un tas de constructions sont possibles.
Mais le but de ce cours est de répondre à trois problèmes qui datent des mathématiciens grecs :
la trisection des angles, la duplication du cube ainsi que le célèbre problème de la 
quadrature du cercle.

%---------------------------------------------------------------
\subsection{Premières constructions géométriques}
\label{ssec:premgeo}

Nous avons à notre disposition un compas et une règle (non graduée).
On démarre par des constructions élémentaires. 


\begin{itemize}
  \item Si $A,B$ sont deux points donnés du plan, alors 
  on peut construire, à la règle et au compas, 
  le \evidence{symétrique} de $B$ par rapport à $A$.
  Pour cela, il suffit juste de tracer la droite $(AB)$ et le 
  cercle de centre $A$ passant par $B$.
  Cette droite et ce cercle se coupent en $B$ bien sûr et aussi en $B' =s_A(B)$,
  le symétrique de $B$ par rapport à $A$. 
\myfigure{1}{
 \tikzinput{fig_compas01}\qquad\qquad
 \tikzinput{fig_compas02}
}

  \item Si $A,B$ sont deux points donnés du plan, alors on peut construire la 
  \evidence{médiatrice} de $[AB]$.
  Pour cela, tracer le cercle centré en $A$ passant par $B$ 
  et aussi le cercle centré en $B$ passant par~$A$. Ces deux cercles 
  s'intersectent en deux points $C$, $D$. 
  Les points $C$, $D$ appartiennent à la médiatrice de $[AB]$.
  Avec la règle on trace la droite $(CD)$ qui est la médiatrice de $[AB]$.
  
  
  \item En particulier cela permet de construire le \evidence{milieu} 
  $I$ du segment $[AB]$. En effet, c'est l'intersection 
de la droite $(AB)$ et de la médiatrice $(CD)$ 
que l'on vient de construire. 

  \item Si $A, B, C$ sont trois points donnés alors 
  on peut construire la \evidence{parallèle} à la droite $(AB)$ passant par $C$.
  Tout d'abord construire le milieu $I$ de $[AC]$. 
  Puis construire $D$ le symétrique de $B$ par rapport à $I$. 
  La figure $ABCD$ est un \evidence{parallélogramme}, 
  donc la droite $(CD)$ est bien la parallèle à la droite $(AB)$
  passant par $C$.

\myfigure{1.2}{
\tikzinput{fig_compas03}\qquad\qquad
\tikzinput{fig_compas04}
}
  
  \item Pour construire la \evidence{perpendiculaire} à $(AB)$ 
  passant par un point $C$, on construit d'abord deux points 
  de la médiatrice de $[AB]$, puis la parallèle 
  à cette médiatrice passant par $C$. 
\end{itemize}  

%---------------------------------------------------------------
\subsection{Règles du jeu}


Il est peut-être temps d'expliquer ce que l'on est autorisé à faire.
Voici les règles du jeu : partez de points sur une feuille.
Vous pouvez maintenant tracer d'autres points, à partir de cercles et de droites en respectant
les conditions suivantes :
\begin{itemize}
  \item vous pouvez tracer une droite entre deux points déjà construits,
  \item vous pouvez tracer un cercle dont le centre est un point construit et qui passe
  par un autre point construit,
  \item vous pouvez utiliser les points obtenus comme intersections de deux droites tracées, 
  ou bien intersections d'une droite et d'un cercle tracé,
  ou bien intersections de deux cercles tracés.
\end{itemize}

\myfigure{0.9}{
\tikzinput{fig_compas05}\qquad
\tikzinput{fig_compas06}
}

\myfigure{0.7}{
\tikzinput{fig_compas07}\qquad
\tikzinput{fig_compas08}\qquad
\tikzinput{fig_compas09}
}

\begin{itemize}
  \item Une remarque importante : la règle est une règle simple, qui n'est pas graduée.
  
  \item Convention pour les couleurs : les points donnés à l'avance sont les points bleus.
Les constructions se font en rouge (rouge pâle pour les constructions qui viennent en premier, 
rouge vif pour les constructions qui viennent en dernier).
\end{itemize}





%---------------------------------------------------------------
\subsection{Conserver l'écartement du compas}

\begin{itemize} 
  \item On peut \evidence{conserver l'écartement du compas}. 
  C'est une propriété importante qui simplifie les constructions.
  
  Si l'on a placé des points $A,B,A'$
  alors on peut placer la pointe en $A$ avec un écartement de longueur $AB$. 
  C'est-à-dire que l'on peut mesurer le segment $[AB]$, 
  puis soulever le compas en gardant l'écartement pour tracer le cercle 
  centré en $A'$ et d'écartement $AB$. 
  
  Cette opération se justifie de la façon suivante : on pourrait construire le point 
  $B'$ tel que $A'ABB'$ soit un parallélogramme et ensuite tracer le cercle centré en $A'$
  passant par $B'$.
  
 \myfigure{0.8}{
\tikzinput{fig_compas10}
}

  
  
  \item En conservant l'écartement du compas, nous pouvons plus facilement construire 
  les parallélogrammes, avec seulement deux traits de compas. Donnons-nous trois points $A,B,C$.  
  On mesure l'écartement $[AB]$, on trace le cercle centré en $C$ de rayon $AB$.
  Puis on mesure l'écartement $[BC]$ et on trace le cercle centré en $A$ de rayon $BC$.
  Ces deux cercles se recoupent en deux points, dont l'un est $D$,
  tel que $ABCD$ est un parallélogramme.
  
 \myfigure{0.8}{
\tikzinput{fig_compas11}
}  
\end{itemize}

 




 

%---------------------------------------------------------------
\subsection{Thalès et Pythagore}

Voyons comment le théorème de Thalès nous permet de diviser un segment en 
$n$ morceaux.

Fixons $n$ un entier.
Voici les étapes pour diviser un segment $[AB]$ en $n$ parts égales.

\begin{enumerate}
  \item Tracer une droite $\mathcal{D}$ quelconque, passant par $A$, autre que la droite $(AB)$.
  
  \item Prendre un écartement quelconque du compas. Sur la droite $\mathcal{D}$ et en partant de $A$,
  tracer $n$ segments de même longueur. On obtient des points $A_1,A_2,\ldots,A_n$.

  \item Tracer la droite $(A_nB)$. Tracer les parallèles à cette droite passant par $A_i$.
  Ces droites recoupent le segment $[AB]$ en des points $B_1,B_2,\ldots,B_{n-1}$ qui découpent l'intervalle
  $[AB]$ en $n$ segments égaux.
\end{enumerate}

Cette construction fonctionne grâce au théorème de Thalès.

\myfigure{0.6}{
\tikzinput{fig_compas12}\qquad
\tikzinput{fig_compas13}\qquad
\tikzinput{fig_compas14}

\bigskip

\tikzinput{fig_compas15}\qquad
\tikzinput{fig_compas16}\qquad
\tikzinput{fig_compas17}
}


\bigskip

Voyons maintenant comment le théorème de Pythagore va nous permettre de faire apparaître des racines carrées.
Supposons que l'on parte d'un segment de longueur $1$.
Il est facile de construire un segment de longueur $\sqrt2$ : 
c'est la longueur de la diagonale du carré de côté $1$.
Repartons du segment diagonal de longueur $\sqrt2$ : on construit un triangle rectangle 
avec un côté de longueur $1$, et l'hypoténuse a alors pour longueur $\sqrt{3}$ (voir le calcul plus bas).
Repartant de ce segment, on construit un << escargot >> avec des  segments de longueurs $\sqrt{4}$, $\sqrt{5}$...




% Depuis fig_compas18
% Spiral of Theodorus  tex.stackexchange.com  "Manuel"

% Problème à corriger !! L'importation figure 18 ne fonctionne plus

\newcommand*{\sqrtspiral}[2][scale=2]{
    \begin{tikzpicture}[#1]
        \def\sqrtlast{#2};
        \coordinate (A) at (0,0);
        \coordinate (B) at (1cm,0);
        \draw[thick, black] (A) edge node[auto, swap] {1} (B);
        \foreach \n in {1,...,\sqrtlast}{
            \pgfmathtruncatemacro{\currentsqrt}{\n+1};
            \coordinate (C) at ($(B)!1cm!-90:(A)$);           
%             \pgfdeclareradialshading{glow}{\pgfpoint{0cm}{0cm}}{
%                        color(0mm)=(white);
%                       color(3mm)=(white);
%                       color(7mm)=(black);
%                       color(10mm)=(black);
%                       }

%                       \begin{tikzfadingfrompicture}[name=glow fading]
%                               \shade [shading=glow] (0,0) circle (1);
%                       \end{tikzfadingfrompicture}
     %        \draw[thick, black] (A) edge node[fill=white, circle,inner sep=6pt,path fading=glow fading]{$\sqrt{\currentsqrt}$} (C);
            \draw[thick, black] (A) edge node[fill=white, circle,inner sep=6pt]{$\sqrt{\currentsqrt}$} (C);
            \draw[thick, black] (C) edge node[auto] {1} (B);
            \coordinate (w) at ($(B)!4pt!-90:(A)$);
            \coordinate (z) at ($(B)!4pt!0:(A)$);
            \coordinate (t) at ($(w)!4pt!-90:(B)$);
            \draw (w) -- (t) -- (z);
            \coordinate (B) at (C);
       };
    \end{tikzpicture}
}


\myfigure{1}{
  \sqrtspiral{1}
  \sqrtspiral{2}
  \sqrtspiral{3}
  \sqrtspiral[scale=1.5]{9}
}

Tout ceci se justifie par le théorème de Pythagore : dans un triangle rectangle
ayant un côté de longueur $\ell$ et un autre de longueur $1$, 
l'hypoténuse est de longueur $\sqrt{\ell^2 + 1}$.
En partant de $\ell_1 = 1$, on trouve $\ell_2 = \sqrt{\ell_1^2 + 1}=\sqrt{2}$,
puis $\ell_3 = \sqrt{\ell_2^2  + 1}=\sqrt{3}$, $\ell_4 = \sqrt{4}=2$, et plus généralement
$\ell_n = \sqrt{n}$.


\myfigure{1}{
\tikzinput{fig_compas19}
}

\bigskip

Voici maintenant trois questions qui datent de la Grèce antique 
et qui vont nous occuper le reste du chapitre.

%---------------------------------------------------------------
\subsection{La trisection des angles}
\label{ssec:trissec}

Considérons un angle $\theta$, c'est-à-dire la donnée d'un point $A$ et de deux demi-droites 
issues de ce point.
Nous savons diviser cet angle en deux à l'aide d'une règle et d'un compas :
il suffit de tracer la bissectrice. Pour cela on fixe un écartement de compas et on trace un cercle
centré en $A$ : il recoupe les demi-droites en des points $B$ et $C$. On trace maintenant 
deux cercles centrés en $B$ puis $C$ (avec le même rayon pour les deux cercles). Si $D$ est un point de l'intersection de ces
deux cercles alors la droite $(AD)$ est la bissectrice de l'angle.
\myfigure{1.3}{
\tikzinput{fig_compas20}\qquad
\tikzinput{fig_compas21}
}

\bigskip

\begin{center}
\shadowbox{
\begin{minipage}{0.7\textwidth}
\defi{Problème de la trisection.} Peut-on diviser un angle donné en trois 
angles égaux à l'aide de la règle et du compas ?
\end{minipage}
}
\end{center}

\myfigure{1}{
\tikzinput{fig_compas22}
}


%---------------------------------------------------------------
\subsection{La duplication du cube}

Commençons par un problème assez simple : étant donné un carré, construire (à la règle et au compas)
un carré dont l'aire est le double. C'est facile, car cela revient à savoir tracer un côté de longueur
$a\sqrt2$ à partir d'un côté de longueur $a$. En fait la diagonale de notre carré original 
a la longueur voulue $a\sqrt 2$. Partant de cette longueur, on construit un carré dont l'aire est
$(a\sqrt2 )^2 = 2 a^2$ : son aire est bien le double de celle du carré de départ.
\myfigure{1.2}{
\tikzinput{fig_compas23}
}


Posons nous la question dans l'espace : étant donné un cube, peut-on construire
un second cube dont le volume est le double de celui du premier ?
Si le premier cube a ses côtés de longueur $a$, alors le second doit
avoir ses côtés de longueur $a\sqrt[3]{2}$. La question se formule alors de la manière suivante :
\myfigure{1.3}
{
\tikzinput{fig_compas24}\qquad\qquad
\tikzinput{fig_compas25}
}
\begin{center}
\shadowbox{\begin{minipage}{0.7\textwidth}
\defi{Problème de la duplication du cube.} 
\'Etant donné un segment de longueur $1$, peut-on 
construire à la règle et au compas 
un segment de longueur $\sqrt[3]{2}$ ?
\end{minipage}}
\end{center}


%---------------------------------------------------------------
\subsection{La quadrature du cercle}

\begin{center}
\shadowbox{\begin{minipage}{0.7\textwidth}
\defi{Problème de la quadrature du cercle.} 
\'Etant donné un cercle, 
peut-on construire à la règle et au compas 
un carré de même aire ?
\end{minipage}}
\end{center}
\myfigure{1.2}{
\tikzinput{fig_compas26}
}
Cela revient à construire un segment de longueur $\sqrt{\pi}$ à la règle et au compas, à
partir d'un segment de longueur $1$.




%%%%%%%%%%%%%%%%%%%%%%%%%%%%%%%%%%%%%%%%%%%%%%%%%%%%%%%%%%%%%%%%
\section{Les nombres constructibles à la règle et au compas}

Pour résoudre les trois problèmes grecs, il va falloir les transformer complètement.
D'une question géométrique nous allons passer à une question algébrique.
Dans cette partie on ramène le problème de la construction de points dans le plan à la construction
de points sur la droite numérique réelle.


%---------------------------------------------------------------
\subsection{Nombre constructible}

On considère le plan euclidien $\plan$ muni d'un repère orthonormé,
que l'on identifiera à $\Rr^2$ (ou $\Cc$).
On définit des ensembles de points $\construc_i \subset \plan$
par récurrence.
\myfigure{1}{
\tikzinput{fig_compas27}
}
\begin{itemize}
  \item On se donne au départ seulement deux points : $\construc_0 = \{ O, I\}$ où $O=(0,0)$ et $I=(1,0)$. 
  
  \item Fixons $i\ge 0$, et supposons qu'un certain ensemble de points $\construc_i$
  soit déjà construit. Alors on définit $\construc_{i+1}$ par récurrence,
  comme l'ensemble des \defi{points élémentairement constructibles}
  à partir de $\construc_i$. C'est-à-dire : $P \in \construc_{i+1}$ si et seulement si
  \begin{enumerate}
    \setcounter{enumi}{-1}
    \item $P \in \construc_i$
    \item ou $P \in (AB) \cap (A'B')$ avec $A, B, A', B' \in \construc_i$,
    
    \item ou $P \in (AB) \cap \cercle(A',A'B')$ avec $A, B, A', B' \in \construc_i$,
    
    \item ou $P \in \cercle(A,AB) \cap \cercle(A',A'B')$ avec $A, B, A', B' \in \construc_i$.
  \end{enumerate}
\end{itemize}
On a noté $\cercle(A,r)$ le cercle de centre $A$ et de rayon $r$.

Il faut comprendre cette construction ainsi : si 
$A,B, A', B'$ ont été construits et sont dans $\construc_i$ 
alors, à partir de ces points, on peut tracer plusieurs objets à
la règle et au compas : par exemple la droite $(AB)$
--\,à l'aide de la règle\,--
ou le cercle de centre $A'$ et de rayon de longueur $A'B'$
en plaçant la pointe du compas en $A'$ avec un écartement faisant
passer le cercle par $B'$.
Si cette droite $(AB)$ et ce cercle $\cercle(A',A'B')$ s'intersectent alors
les points d'intersection sont par définition dans $\cercle_{i+1}$.


Voici les trois situations possibles. Les points $A,B,A',B'$ en bleu sont dans $\construc_i$,
et les points $P$ en rouge sont dans $\construc_{i+1}$.



\myfigure{1}{
\tikzinput{fig_compas28}
\tikzinput{fig_compas29}
\tikzinput{fig_compas30}
}


\bigskip

Voici la première étape.
Partant de $\construc_0$ (en bleu à gauche), on peut tracer une droite et deux cercles (au milieu),
ce qui donne pour $\construc_1$ quatre points supplémentaires (en rouge à droite).

\myfigure{0.7}{
\tikzinput{fig_compas31}
\tikzinput{fig_compas32}
\tikzinput{fig_compas33}
}
Pour $\construc_2$ on repartirait de tous les points (rouges ou bleus) 
de $\construc_1$, et on tracerait tous les cercles ou droites 
possibles (il y en a beaucoup !), et les points d'intersection 
formeraient l'ensemble $\construc_2$.




\bigskip

\begin{definition}
\sauteligne
\begin{itemize}
  \item $\construc = \bigcup_{i\ge 0} \construc_i$ est l'ensemble des \defi{points constructibles}.
 Autrement dit $\construc = \construc_0 \cup \construc_1 \cup \construc_2 \cup \cdots$
 De plus $P \in \construc$ si et seulement s'il existe $i\ge 0$ tel que $P \in \construc_i$.
  
  \item $\construc_\Rr \subset \Rr$ est l'ensemble des abscisses des points constructibles :
  ce sont les \defi{nombres (réels) constructibles}.  
  
  \item $\construc_\Cc \subset \Cc$ est l'ensemble des affixes des points constructibles : ce sont les 
  \defi{nombres complexes constructibles}.

\end{itemize}
\end{definition}


Attention ! Même si 
deux points $A$, $B$ sont constructibles et que l'on peut tracer la droite $(AB)$, 
pour autant les points de $(AB)$
ne sont pas tous constructibles. Seuls les points d'intersection  de $(AB)$
avec d'autres objets construits sont constructibles.


Déterminer les points constructibles $\construc$ ou déterminer 
les nombres constructibles $\construc_\Rr$ sont deux problèmes équivalents.
En effet, si $(x,y)$ est un point constructible alors par projection sur l'axe des abscisses
nous obtenons le réel constructible $x$, et de même pour $y$ projection sur l'axe des ordonnées, 
puis report sur l'axe des abscisses.
Réciproquement on peut passer de deux nombres constructibles $x,y \in \Rr$ à un point constructible 
$(x,y)$ dans le plan. Voici comment : partant du point $(y,0)$ on construit $(0,y)$ sur l'axe 
des ordonnées par un coup de compas en reportant $y$.
Une fois que $(x,0)$ et $(0,y)$ sont construits, il est facile de construire $(x,y)$.

\myfigure{1.5}{
\tikzinput{fig_compas34}
} 

%---------------------------------------------------------------
\subsection{Premières constructions algébriques}
\label{ssec:prem}


\begin{proposition}
\label{prop:cons}
Si $x, x'$ sont des réels constructibles alors :
\begin{enumerate}
 \item $x+x'$ est constructible,
 
 \item $-x$ est constructible,
 
 \item $x \cdot x'$ est constructible.
 
 \item Si $x' \neq 0$, alors $x/x'$ est constructible.
\end{enumerate}
Tous ces résultats sont valables si l'on remplace $x,x'$ 
par des nombres complexes $z,z'$.
\end{proposition}


\begin{proof}~
\begin{enumerate}
  \item La construction pour le réel $x+x'$ est facile en utilisant le report 
  du compas (on reporte la longueur $x'$ à partir de $x$). Une autre méthode
  est de construire d'abord le milieu $\frac{x+x'}{2}$ puis le symétrique de 
  $0$ par rapport à ce milieu : c'est $x+x'$.
  
  La somme de deux nombres complexes $z+z'$ correspond à la construction
  d'un parallélogramme de sommets $0,z,z',z+z'$ : les points d'affixes $0$, $z$, $z'$ 
  étant supposés constructibles, on construit un parallélogramme de sorte que 
  $z+z'$ soit le quatrième sommet.
  \myfigure{1}{
\tikzinput{fig_compas35}
} 
  \item L'opposé du réel $x$ (resp. du complexe $z$) s'obtient comme symétrique par rapport à l'origine : 
  tracez la droite passant par $0$ et $x$ (resp. $z$) ; tracez le cercle de centre 
  $0$ passant par $x$ (resp. $z$) ; ce cercle recoupe la droite en $-x$ (resp. $-z$). 
  
\myfigure{1}{
\tikzinput{fig_compas36}
}   
  \item Commençons par le produit de deux nombres réels $x\cdot x'$. On suppose construits
  les points $(x,0)$ et $(0,x')$ (dessin de gauche). On trace la droite  $\mathcal{D}$ passant par $(x,0)$ et 
  $(0,1)$. On construit ensuite --\,à la règle et au compas\,-- la droite $\mathcal{D}'$
  parallèle à $\mathcal{D}$ et passant par $(0,x')$. Le théorème de Thalès prouve que 
  $\mathcal{D}'$ recoupe l'axe des abscisses en $(x\cdot x',0)$. 
\myfigure{0.9}{
\tikzinput{fig_compas37}\quad
\tikzinput{fig_compas38}
}

  \item Pour le quotient la méthode est similaire (dessin de droite). 

  \item Il reste à s'occuper du produit et du quotient de deux nombres complexes.
  Tout d'abord, si $z= \rho e^{\ii \theta}$ est un nombre complexe constructible, alors
  $\rho$ est constructible (considérer le cercle centré à l'origine qui passe par $z$ ;
  il recoupe l'axe des abscisses en $(\rho,0)$). Le nombre $e^{\ii \theta}$
  est aussi constructible : c'est l'intersection de la droite passant par l'origine et $z$
  avec le cercle unité. Réciproquement avec $\rho$ et $e^{\ii \theta}$ on construit facilement 
  $z= \rho e^{\ii \theta}$.
\myfigure{0.9}{
  \tikzinput{fig_compas39}
  \tikzinput{fig_compas40}
}
 
  Maintenant si $z = \rho e^{\ii \theta}$ et $z' = \rho' e^{\ii \theta'}$ alors
  $z\cdot z' = (\rho \cdot \rho') e^{\ii (\theta+\theta')}$.
  Le réel $\rho \cdot \rho'$ est constructible comme nous l'avons vu au-dessus.
  Il reste à construire le nombre complexe $e^{\ii (\theta+\theta')}$, qui correspond à la somme de
  deux angles $\theta$ et $\theta'$. Cela se fait simplement, à partir du cercle unité,
  en reportant au compas la mesure d'un angle à partir de l'extrémité de l'autre. 

  Pour le quotient la méthode est similaire.
\end{enumerate}
\end{proof}


\begin{corollaire}
$$\Nn \subset \construc_\Rr \qquad\qquad \Zz \subset \construc_\Rr \qquad\qquad \Qq \subset \construc_\Rr$$
\end{corollaire}
Autrement dit,  tous les rationnels (et en particulier tous les entiers) sont des nombres réels constructibles.

\bigskip

La preuve découle facilement de la proposition :
\begin{proof}~
\begin{itemize}
 \item Puisque $1$ est un nombre constructible alors $2=1+1$ est constructible, mais alors 
$3=2+1$ est constructible et par récurrence tout entier $n\ge 0$ est un élément de $\construc_\Rr$.
 
 \item Comme tout entier $n\ge 0$ est constructible alors $-n$ l'est aussi ; donc tous les entiers
 $n\in \Zz$ sont constructibles.
 
 \item Enfin pour $\frac pq \in \Qq$, comme les entiers $p,q$ sont constructibles, alors le quotient 
$\frac pq$ est constructible et ainsi $\Qq \subset \construc_\Rr$.
\end{itemize}


\end{proof}


Nous allons voir que $\construc_\Rr$ contient davantage de nombres que les rationnels.
\begin{proposition}
\label{prop:rac}
Si $x\ge 0$ est un nombre constructible, alors $\sqrt x$ est constructible.
\end{proposition}

Remarques :
\begin{enumerate}
  \item La réciproque est vraie. En effet si $x'= \sqrt{x}$ est un nombre constructible,
alors par la proposition \ref{prop:cons} : $x'\cdot x'$ est constructible.
Or $x' \cdot x' = \sqrt x \cdot \sqrt x = x$, donc $x$ est constructible.
  
  \item On peut en déduire aussi que si $z \in \Cc$ est constructible alors
  les racines carrées (complexes) de $z$ sont constructibles. On utilise pour cela
  la racine carrée du module et la bissection de l'argument comme on l'a vue au paragraphe \ref{ssec:trissec}.
  
  \item En particulier $\sqrt 2$, $\sqrt 3$, \ldots\  sont des nombres constructibles (comme on l'avait vu en première partie).
\end{enumerate}

\begin{proof}



Soient les nombres constructibles $0,-1,x$ placés sur l'axe des abscisses.
Traçons le cercle dont le diamètre est $[-1,x]$ 
(cela revient à construire le centre du cercle $\frac{x-1}{2}$ ; voir la proposition \ref{prop:cons}).
Ce cercle recoupe l'axe des ordonnées en $y\ge0$.

\myfigure{1}{
  \tikzinput{fig_compas41}
}   

\myfigure{1.1}{
  \tikzinput{fig_compas42}
}

On applique le théorème de Pythagore dans trois triangles rectangles, pour obtenir :
$$\left\{\begin{array}{rcl} 
a^2+b^2 &=&(1+x)^2 \\
1+y^2   &=& a^2 \\
x^2+y^2 &=& b^2. \\
\end{array}\right.$$
On en déduit $a^2+b^2 = (1+x)^2 = 1 + x^2 + 2x$ d'une part et $a^2+b^2 = 1 + x^2 + 2y^2$ d'autre part.
Ainsi $1 + x^2+2x = 1+x^2+2y^2$ d'où $y^2=x$. Comme $y\ge0$ alors $y = \sqrt{x}$.




Une autre méthode consiste à remarquer que le triangle 
de sommets $(0,0), (-1,0), (0,y)$ et le triangle de sommets 
$(0,0), (x,0), (0,y)$ sont semblables
donc $\frac xy= \frac y1$, d'où $x=y^2$, donc $y=\sqrt x$.

\end{proof}

%---------------------------------------------------------------
\subsection{Retour sur les trois problèmes grecs}

Avec le langage des nombres constructibles les problèmes historiques s'énoncent 
ainsi :
\begin{itemize}  
  \item \evidence{La duplication du cube.} Est-ce que $\sqrt[3]{2}$ est un nombre constructible ?
  
  \item \evidence{La quadrature du cercle.} Est-ce que $\pi$ est un nombre constructible ?
  
  \item \evidence{La trisection des angles.} \'Etant donné un réel 
  constructible $\cos \theta$, est-ce que $\cos \frac\theta3$ est aussi constructible ?

\end{itemize}


Si vous n'êtes pas convaincu voici les preuves :
\begin{itemize}
  \item Si on a un cube de volume $a^3$, alors il faut construire 
  un cube de volume $2a^3$.
  Fixons $a$ un réel constructible. Que $\sqrt[3]{2}$ soit aussi constructible
  équivaut à $a\sqrt[3]{2}$ constructible. 
  Un segment de longueur $a\sqrt[3]{2}$ définit
  bien un cube de volume $2a^3$. On aurait résolu la duplication du cube.
  
  \item Soit construit un cercle de rayon $r$, donc d'aire $\pi r^2$. 
  Que $\pi$ soit constructible équivaut à $\sqrt\pi$ constructible. 
  Construire un segment de longueur $\sqrt\pi r$,
  correspond à un carré d'aire $\pi r^2$, donc de même aire que le cercle initial.
  Nous aurions construit un carré de même aire que le cercle !
  On aurait résolu la quadrature du cercle.
  
  \item Remarquons que construire un angle géométrique de mesure $\theta$
  est équivalent à construire le nombre réel $\cos \theta$ (voir la figure de gauche).
  Partons d'un angle géométrique $\theta$, c'est-à-dire partons d'un réel
  $\cos \theta$ constructible. Construire $\cos \frac\theta3$
  est équivalent à construire un angle géométrique de mesure $\frac\theta 3$.
  On aurait résolu la trisection des angles.
 
\myfigure{1}{
  \tikzinput{fig_compas43}\qquad
  \tikzinput{fig_compas44}
}  
\end{itemize}

%---------------------------------------------------------------
\subsection{Les ensembles}

Une dernière motivation à propos des nombres constructibles concerne les ensembles. Nous avons les inclusions d'ensembles:
$$\Nn \subset \Zz \subset \Qq \subset \Rr \subset \Cc.$$

Le passage d'un ensemble à un ensemble plus grand se justifie
par la volonté de résoudre davantage d'équations : 
\begin{itemize}
  \item passage de $\Nn$ à $\Zz$, pour résoudre des équations du type $x+7=0$,
  \item passage de $\Zz$ à $\Qq$, pour résoudre des équations du type $5x=4$,
  \item passage de $\Qq$ à $\Rr$, pour résoudre des équations du type $x^2=2$,
  \item passage de $\Rr$ à $\Cc$, pour résoudre des équations du type $x^2=-1$.
\end{itemize}


Mais en fait le passage de $\Qq$ à $\Rr$ est un saut beaucoup plus <<~grand~>> que les autres :
$\Qq$ est un ensemble dénombrable (il existe une bijection entre $\Zz$ et $\Qq$)
alors que $\Rr$ ne l'est pas. 

Nous allons définir et étudier deux ensembles intermédiaires :
$$\Qq \subset \construc_\Rr \subset \overline{\Qq} \subset \Rr$$
où 
\begin{itemize}
  \item $\construc_\Rr$ est l'ensemble des nombres réels constructibles à la règle et au compas,
  \item $\overline{\Qq}$ est l'ensemble des nombres algébriques : 
  ce sont les réels $x$ qui sont solutions d'une équation
  $P(x) = 0$, pour un polynôme $P$ à coefficients dans $\Qq$.
\end{itemize}



%%%%%%%%%%%%%%%%%%%%%%%%%%%%%%%%%%%%%%%%%%%%%%%%%%%%%%%%%%%%%%%%
\section{\'Eléments de théorie des corps}

La théorie des corps n'est pas évidente et mériterait un chapitre
entier. Nous résumons ici les grandes lignes utiles à nos fins.
Il est important de bien comprendre le paragraphe suivant ;
les autres paragraphes peuvent être sautés lors de la première lecture.

%---------------------------------------------------------------
\subsection{Les exemples à comprendre}


\textbf{Premier exemple.}
Soit l'ensemble 
$$\Qq(\sqrt 2) = \left\{ a+b\sqrt 2 \mid a,b \in \Qq \right\}.$$
C'est un sous-ensemble de $\Rr$, qui contient par exemple $0$, $1$, $\frac13$ et tous les éléments de $\Qq$,
mais aussi $\sqrt 2$ (qui n'est pas rationnel !), $\frac 12 - \frac 23 \sqrt 2$. 


Voici quelques propriétés :
\begin{itemize}
  \item Soient $a+b\sqrt 2$ et $a'+b'\sqrt 2$ deux éléments de $\Qq(\sqrt 2)$.
  Alors leur somme $(a+b\sqrt 2)+(a'+b'\sqrt 2)$ est encore un élément de $\Qq(\sqrt 2)$.
  De même $-(a+b\sqrt 2) \in \Qq(\sqrt 2)$.
    
  \item Plus surprenant, si $a+b\sqrt 2, a'+b'\sqrt 2 \in \Qq(\sqrt 2)$ alors
  $(a+b\sqrt 2)\times(a'+b'\sqrt 2) = aa'+2bb' + (ab'+a'b)\sqrt 2$ est aussi un élément de $\Qq(\sqrt 2)$.
  Enfin l'inverse d'un élément non nul $a+b\sqrt 2$ est $\frac{1}{a+b\sqrt 2} = \frac{1}{a^2-2b^2}(a-b\sqrt 2)$ :
c'est encore un élément de $\Qq(\sqrt 2)$.
\end{itemize}
Ces propriétés font de $\Qq(\sqrt 2)$ un \defi{corps}.
Comme ce corps contient $\Qq$ on parle d'une \defi{extension} de $\Qq$.
De plus, il est étendu avec un élément du type $\sqrt \delta$ : on parle alors d'une \defi{extension
quadratique}.
Notez que, même si $\delta \in \Qq$, $\sqrt{\delta}$ n'est généralement pas un élément de $\Qq$.


\bigskip

\textbf{Deuxième série d'exemples.}
On peut généraliser l'exemple précédent : si $K$ est lui-même
un corps et $\delta$ est un élément de $K$ alors 
$$K(\sqrt \delta) = \left\{ a+b\sqrt \delta \mid a,b \in K \right\}$$
est un corps. On vérifie comme ci-dessus que la somme et le produit de deux éléments restent
dans $K(\sqrt \delta)$, ainsi que l'opposé et l'inverse.

Cela permet de construire de nouveaux corps :
partant de $K_0 = \Qq$, on choisit un élément, disons $\delta_0 = 2$ et
on obtient le corps plus gros $K_1 = \Qq(\sqrt 2)$.
Si on prend $\delta_1 = 3$ alors $\sqrt 3 \notin \Qq(\sqrt 2)$ et
donc $K_2 = K_1(\sqrt 3)$ est un nouveau corps (qui contient $K_1$).
Le corps $K_2$ est :
$$K_2 = K_1(\sqrt 3) = \Qq(\sqrt2)(\sqrt 3) = \left\{ a+b\sqrt 2 + c\sqrt 3 + d\sqrt 2 \sqrt 3 \mid a,b,c,d \in \Qq \right\}.$$
On pourrait continuer avec $\delta_2 = 11$ et exprimer chaque élément de $\Qq(\sqrt2)(\sqrt 3)(\sqrt {11})$ comme une somme
de $8$ éléments $a_1+a_2\sqrt 2 + a_3\sqrt 3 + a_4\sqrt{11}+ a_5\sqrt 2 \sqrt 3 + a_6\sqrt 2 \sqrt {11}
+ a_7\sqrt 3 \sqrt {11} + a_8\sqrt 2 \sqrt 3 \sqrt {11}$ avec les $a_i\in \Qq$.


En partant de $K_1 = \Qq(\sqrt 2)$, on aurait pu considérer $\delta_1=1+\sqrt2$
et $K_2=K_1(\sqrt{1+\sqrt2}) = \Qq(\sqrt2)(\sqrt{1+\sqrt2})$. Chaque élément de $K_2$
peut s'écrire comme une somme de $4$ éléments $a+b\sqrt 2 + c\sqrt{1+\sqrt2} + d\sqrt 2\sqrt{1+\sqrt2}$.

\bigskip

\textbf{Une propriété.}
Il faut noter que chaque élément de $\Qq(\sqrt 2)$ est racine d'un polynôme de degré au plus $2$
à coefficients dans $\Qq$. Par exemple $3+\sqrt 2$ est annulé par 
$P(X) = (X-3)^2 -2 = X^2-6X+7$.
Les nombres qui sont annulés par un polynôme à coefficients rationnels sont les 
\defi{nombres algébriques}.
Plus généralement, si $K$ est un corps et $\delta \in K$, alors tout élément de $K(\sqrt \delta)$
est annulé par un polynôme de degré $1$ ou $2$ à coefficients dans $K$.
On en déduit que chaque élément de $\Qq(\sqrt2)(\sqrt 3)$ (ou de $\Qq(\sqrt2)(\sqrt{1+\sqrt2})$)
est racine d'un polynôme de $\Qq[X]$ de degré $1$, $2$ ou $4$.
Et chaque élément de $\Qq(\sqrt2)(\sqrt 3)(\sqrt {11})$ est racine d'un polynôme 
de $\Qq[X]$ de degré $1$, $2$, $4$ ou $8$, etc.


\bigskip

Nous allons maintenant reprendre ces exemples d'une manière plus théorique.

%---------------------------------------------------------------
\subsection{Corps}

Un corps est un ensemble sur lequel sont définies deux opérations : une addition et une multiplication.

\begin{definition}
 Un \defi{corps} $(K,+,\times)$ est un ensemble $K$
 muni des deux opérations $+$ et $\times$,
 qui vérifient :
 \begin{enumerate}
  \setcounter{enumi}{-1}
  \item $+$ et $\times$ sont des lois de composition interne, c'est à dire
  $x+y \in K$ et $x\times y \in K$ (pour tout $x,y \in K$).
  \item $(K,+)$ est un groupe commutatif, c'est-à-dire :
    \begin{itemize}
      \item Il existe $0 \in K$ tel que $0+x = x$ (pour tout $x\in K$).
      
      \item Pour tout $x\in K$ il existe $-x$ tel que $x+(-x)=0$.
      
      \item $+$ est associative : $(x+y)+z = x + (y+z)$ (pour tout $x,y,z \in K$).
      
      \item $x+y=y+x$ (pour tout $x,y \in K$).
    \end{itemize} 
  
  \item $(K\setminus\{0\},\times)$ est un groupe commutatif, c'est-à-dire :
     \begin{itemize} 
      \item Il existe $1 \in K\setminus\{0\}$ tel que $1 \times x = x$ (pour tout $x\in K$).
      
      \item Pour tout $x\in K\setminus\{0\}$, il existe $x^{-1}$ tel que $x\times x^{-1}=1$.
      
      \item $\times$ est associative : $(x\times y)\times z = x \times (y \times z)$ (pour tout $x,y,z \in K\setminus\{0\}$).
      
      \item $x \times y = y \times x$ (pour tout $x,y \in K\setminus\{0\}$).     

    \end{itemize} 
    
  \item $\times$ est distributive par rapport à $+$ : $(x+y)\times z= x\times z + y\times z$
  (pour tout $x,y,z \in K$).
 \end{enumerate}

\end{definition}


\bigskip

Voici des exemples classiques :
\begin{itemize}
  \item $\Qq$, $\Rr$, $\Cc$ sont des corps.
 L'addition et la multiplication sont les opérations usuelles.
  \item Par contre $(\Zz,+,\times)$ n'est pas un corps. (Pourquoi ?) 
\end{itemize}

Voici des exemples qui vont être importants pour la suite :
\begin{itemize}
 \item $\Qq(\sqrt 2) = \big\{ a+b\sqrt 2 \mid a,b \in \Qq \big\}$ est un corps
 (avec l'addition et la multiplication habituelles des nombres réels). Voir les exemples introductifs.
 
 \item  $\Qq(\ii) =  \big\{ a+\ii b \mid a,b \in \Qq  \big\}$ est un corps
 (avec l'addition et la multiplication habituelles des nombres complexes).
 
 \item Par contre  $ \big\{ a+b \pi \mid a,b \in \Qq  \big\}$ n'est pas un corps (où $\pi = 3,14\ldots$). 
 (C'est une conséquence du fait que $\pi$ n'est pas un nombre algébrique comme on le verra plus loin.)
\end{itemize}

% Enfin l'ensemble des nombres algébriques
% $$\overline{\Qq} = \left\{ x \in \Rr \mid \text{ il existe } P \in \Qq[x] \text{ tel que } P(x)=0 \right\}$$
% est un corps. La preuve sera vue plus loin.

\bigskip

La proposition \ref{prop:cons} de la première partie 
se reformule avec la notion de corps en :
\begin{proposition}
L'ensemble des nombre réels constructibles $(\construc_\Rr, +, \times)$ est un corps inclus dans $\Rr$.
\end{proposition}

On a aussi que $(\construc_\Cc, +, \times)$ est un corps inclus dans $\Cc$.


%---------------------------------------------------------------
\subsection{Extension de corps}

Nous cherchons des propositions qui lient deux corps, lorsque l'un est inclus dans l'autre.
Les résultats de ce paragraphe seront admis.

\begin{proposition}
Soient $K, L$ deux corps avec $K \subset L$. Alors $L$ est un espace vectoriel sur $K$.
\end{proposition}


\begin{definition}
$L$ est appelé une \defi{extension} de $K$.
Si la dimension de cet espace vectoriel est finie, alors
on l'appelle le \defi{degré} de l'extension, et on notera :
$$[L:K] = \dim_K L.$$
Si ce degré vaut $2$, nous parlerons d'une \defi{extension quadratique}.
\end{definition}

\begin{proposition}
\label{prop:KLM}
Si $K,L,M$ sont trois corps avec $K \subset L \subset M$
et si les extensions ont un degré fini alors :
$$[M:K] = [M:L] \times [L:K].$$
\end{proposition}


\begin{exemple}
\sauteligne
\begin{itemize}
  \item $\Qq(\sqrt 2)$ est une extension de $\Qq$.
  De plus, comme $\Qq(\sqrt 2) = \big\{ a+b\sqrt 2 \mid a,b \in \Qq \big\}$,
  alors $\Qq(\sqrt 2)$ est un espace vectoriel (sur $\Qq$) de dimension $2$ :
  en effet $(1,\sqrt 2)$ en est une base. Attention : ici $1$ est un vecteur 
  et $\sqrt 2$ est un autre vecteur. Le fait que $\sqrt{2} \notin \Qq$ se 
  traduit en : ces deux vecteurs sont linéairement indépendants sur $\Qq$. 
  C'est un peu déroutant au début !
 
  
  \item $\Cc$ est une extension de degré $2$ de $\Rr$ car tout élément
  de $\Cc$ s'écrit $a+\ii b$. Donc les vecteurs $1$ et $\ii$ forment une base de
  $\Cc$, vu comme un espace vectoriel sur $\Rr$.
  
  \item Notons $\Qq(\sqrt 2, \sqrt 3) = \Qq(\sqrt 2)(\sqrt 3) =
\big\{ a+ b \sqrt 3 \mid a,b \in \Qq(\sqrt 2) \big\}$.
Alors $\Qq \subset \Qq(\sqrt 2) \subset \Qq(\sqrt 2, \sqrt 3)$.
Calculer le degré des extensions. Expliciter une base sur $\Qq$ de 
$\Qq(\sqrt 2, \sqrt 3)$.
\end{itemize}
\end{exemple}

\bigskip


Pour $x\in\Rr$, on note $\Qq(x)$ le plus petit corps 
contenant $\Qq$ et $x$ : c'est le \defi{corps engendré} par $x$.
C'est cohérent avec la notation pour les 
extensions quadratiques $\Qq(\sqrt{\delta})$, qui est bien le plus petit 
corps contenant $\sqrt\delta$.

Par exemple, si $x = \sqrt[3]{2} = 2^{\frac13}$, alors il n'est pas 
dur de calculer que 
$$\Qq(\sqrt[3]{2}) = \left\{ a+b\sqrt[3]{2}+c\sqrt[3]{2}^2 \mid a,b,c \in \Qq \right\}.$$
En effet $\Qq(\sqrt[3]{2})$ contient $x,x^2,x^3,\ldots$ mais aussi $\frac1x, \frac 1{x^2},\ldots$
Mais comme $x^3 = 2 \in \Qq$ et $\frac1x = \frac{x^2}{2}$, alors $a + bx + cx^2$, avec $a,b,c \in \Qq$, engendrent 
tous les éléments de $\Qq(x)$.  Conclusion : $[\Qq(\sqrt[3]{2}):\Qq] = 3$.



%---------------------------------------------------------------
\subsection{Nombre algébrique}


L'ensemble des \defi{nombres algébriques} est
$$\overline{\Qq} = \big\{ x \in \Rr \mid \text{ il existe } P \in \Qq[X] \text{ non nul tel que } P(x)=0\big\}.$$

\begin{proposition}
$\overline{\Qq}$ est un corps.
\end{proposition}

\begin{proof}
L'addition et la multiplication définies sur $\overline{\Qq}$ sont celles du corps $(\Rr,+,\times)$.
Ainsi beaucoup de propriétés découlent du fait que l'ensemble des réels est un corps (on parle de sous-corps).

La première chose que l'on doit démontrer, c'est que $+$ et $\times$ sont des lois de composition interne,
c'est-à-dire que si $x$ et $y$ sont des nombres réels algébriques alors $x+y$ et $x\times y$ le sont aussi.
Ce sera prouvé dans le corollaire \ref{cor:algebrique}.

 \begin{enumerate}
  \item $(\overline{\Qq},+)$ est un groupe commutatif, car :
    \begin{itemize}
      \item $0 \in \overline{\Qq}$ (prendre $P(X)= X$) et $0+x = x$ (pour tout $x\in \overline{\Qq}$).
      
      \item Si $x\in \overline{\Qq}$ alors $-x \in \overline{\Qq}$ (si $P(X)$ est un polynôme qui annule $x$ 
      alors $P(-X)$ annule $-x$).
      
      \item $+$ est associative : cela découle de l'associativité sur $\Rr$.
      
      \item $x+y=y+x$ : idem.
    \end{itemize} 
  
  \item $(\overline{\Qq}\setminus\{0\},\times)$ est un groupe commutatif, car :
     \begin{itemize} 
      \item $1 \in \overline{\Qq}\setminus\{0\}$ et $1 \times x = x$ (pour tout $x\in \overline{\Qq}\setminus\{0\}$).
      
      \item Si $x\in \overline{\Qq}\setminus\{0\}$ alors $x^{-1} \in \overline{\Qq}\setminus\{0\}$ :
      en effet, si $P(X)$ est un polynôme de degré $n$ annulant $x$, alors $X^nP(\frac{1}{X})$ est un polynôme annulant $\frac 1 x$.
      
      \item $\times$ est associative :  cela découle de l'associativité sur $\Rr\setminus\{0\}$.
      
      \item $x\times y=y\times x$ : idem.     

    \end{itemize} 
    
  \item $\times$ est distributive par rapport à $+$ : cela découle de la distributivité sur $\Rr$.
 \end{enumerate}
\end{proof}


Si $x \in \overline{\Qq}$ est un nombre algébrique, alors le plus petit degré, 
parmi tous les degrés des polynômes $P \in \Qq[X]$ tels que $P(x)=0$,
est le \defi{degré algébrique} de $x$.
Par exemple, calculons le degré algébrique de $\sqrt 2$ : un polynôme annulant ce nombre est $P(X) = X^2-2$
et il n'est pas possible d'en trouver de degré $1$, donc le degré algébrique de $\sqrt2$ vaut $2$. 
Plus généralement $\sqrt \delta$ avec $\delta \in \Qq$
est de degré algébrique égal à $1$ ou $2$ 
(de degré algébrique $1$ si $\sqrt{\delta} \in \Qq$, de degré $2$ sinon).
Par contre $\sqrt[3]{2}$ est de degré $3$, car il est annulé par $P(X) = X^3-2$ 
mais pas par des polynômes de degré plus petit.

\begin{proposition}
\sauteligne
\label{prop:ext}

\begin{enumerate}
  \item Soit $L$ une extension finie du corps $\Qq$. Si $x\in L$, alors $x$ est un nombre algébrique.
  
  \item Si $x$ un nombre algébrique alors $\Qq(x)$ est une extension finie de $\Qq$. 
  
  \item Si $x$ est un nombre algébrique alors le degré de l'extension $[\Qq(x):\Qq]$ et le degré algébrique de $x$ coïncident.
\end{enumerate}
\end{proposition}

\begin{proof}~
\begin{enumerate}
  \item Soit $L$ une extension finie de $\Qq$, et soit $n=[L:\Qq]$.
  Fixons $x\in L$. Les $n+1$ éléments $(1,x,x^2,\ldots,x^n)$ forment une famille
  de $n+1$ vecteurs dans un espace vectoriel de dimension $n$. Donc cette famille est liée.
  Il existe donc une combinaison linéaire nulle non triviale, c'est-à-dire il existe $a_i\in \Qq$ non tous nuls tels que
  $\sum_{i=0}^n a_i x^i=0$. Si l'on définit $P(X) = \sum_{i=0}^n a_i X^i$, alors
  $P(X) \in \Qq[X]$, $P(X)$ n'est pas le polynôme nul et $P(x)=0$. C'est exactement dire que $x$ est un nombre algébrique.
  
  \item Soit $P(X) = \sum_{i=0}^n a_i X^i$ non nul qui vérifie $P(x)=0$. En écartant le le cas trivial $x=0$, 
  on peut donc supposer que $a_0\neq 0$ et $a_n\neq 0$.
  Alors $x^n = -\frac{1}{a_n} \sum_{i=0}^{n-1} a_i x^i$ et $\frac 1 x = \frac1{a_0} \sum_{i=1}^n a_i x^{i-1}$.
  Ce qui prouve que $x^n \in \text{Vect}(1,x,\ldots,x^{n-1})$ et $\frac 1 x \in \text{Vect}(1,x,\ldots,x^{n-1})$.
  De même pour tout $k \in \Zz$, $x^k \in \text{Vect}(1,x,\ldots,x^{n-1})$,
  donc $\Qq(x) \subset \text{Vect}(1,x,\ldots,x^{n-1})$. Ce qui prouve que $\Qq(x)$ est un espace vectoriel 
  de dimension finie sur $\Qq$.
  
  
  \item Ce sont à peu près les mêmes arguments. Si $m=[\Qq(x):\Qq]$ alors
  il existe $a_i\in\Qq$ non tous nuls tels que $\sum_{i=0}^m a_i x^i=0$. Donc il existe un polynôme non nul de degré
  $m$ annulant $x$. Donc le degré algébrique de $x$ est inférieur ou égal à $m$.
  
  Mais s'il existait un polynôme $P(X)=\sum_{i=0}^{m-1} b_iX^i$ non nul de degré strictement inférieur à $m$
  qui annulait $x$, alors nous aurions une combinaison linéaire nulle non triviale  
  $\sum_{i=0}^{m-1} b_i x^i=0$. Cela impliquerait que $x^{m-1} \in \text{Vect}(1,x,\ldots,x^{m-2})$
  et plus généralement que $\Qq(x) \subset \text{Vect}(1,x,\ldots,x^{m-2})$,
  ce qui contredirait le fait que $\Qq(x)$ soit un espace vectoriel   de dimension $m$ sur $\Qq$.
  
  Bilan : le degré algébrique de $x$ est exactement $[\Qq(x):\Qq]$.
\end{enumerate}
\end{proof}


\begin{corollaire}
\label{cor:algebrique}
Si $x$ et $y$ sont des nombres réels algébriques alors $x+y$ et $x\times y$ aussi.
\end{corollaire}

\begin{proof}
Comme $x$ est un nombre algébrique alors $L = \Qq(x)$ est une extension finie de $K=\Qq$.
Posons $M = \Qq(x,y) = \big(\Qq(x)\big) (y)$. Comme $y$ est un nombre algébrique alors
$M$ est une extension finie de $\Qq(x)$. Par la proposition \ref{prop:KLM}
$M=\Qq(x,y)$ est une extension finie de $K=\Qq$.

Comme $x+y \in \Qq(x+y) \subset \Qq(x,y)$ et que $\Qq(x,y)$ est une extension finie de $\Qq$ alors
par la proposition \ref{prop:ext}, $x+y$ est un nombre algébrique.

C'est la même preuve pour $x \times y \in \Qq(x \times y) \subset \Qq(x,y)$.
\end{proof}



%%%%%%%%%%%%%%%%%%%%%%%%%%%%%%%%%%%%%%%%%%%%%%%%%%%%%%%%%%%%%%%%
\section{Corps et nombres constructibles}

Cette partie est la charnière de ce chapitre. Nous expliquons à quoi 
correspondent algébriquement les opérations géométriques effectuées à la règle et au compas.

%---------------------------------------------------------------
\subsection{Nombre constructible et extensions quadratiques}

Voici le résultat théorique le plus important de ce chapitre.
C'est Pierre-Laurent Wantzel qui a démontré ce théorème en 1837, à l'âge de 23 ans.

\begin{theoreme}[Théorème de Wantzel]
\label{th:wantzel}
Un nombre réel $x$ est constructible si et seulement s'il existe 
des extensions quadratiques 
$$\Qq = K_0 \subset K_1 \subset \cdots \subset K_r$$
telles que $x \in K_r$.
\end{theoreme}



Chacune des extensions est quadratique, c'est-à-dire $[K_{i+1}:K_i]=2$.
Autrement dit, chaque extension est une extension quadratique de la précédente :
$K_{i+1} = K_i(\sqrt{\delta_i})$ pour un certain $\delta_i \in K_i$. 
Donc en partant de $K_0 = \Qq$, les extensions sont :
$$\Qq \subset \Qq(\sqrt{\delta_0}) \subset \Qq(\sqrt{\delta_0})(\sqrt{\delta_1}) \subset \cdots$$

\begin{proof}
Il y a un sens facile : comme on sait construire les racines carrées des nombres constructibles (voir la proposition 
\ref{prop:rac}) alors on sait construire tout élément d'une extension quadratique 
$K_1 = \Qq(\sqrt{\delta_0})$, puis par récurrence tout élément de $K_2$, $K_3$,\ldots


\medskip

Passons au sens difficile. Rappelons-nous que les points constructibles sont construits
par étapes $\construc_0$, $\construc_1$, $\construc_2$,\ldots

L'ensemble $\construc_{j+1}$ s'obtient à partir de $\construc_j$ en ajoutant 
les intersections des droites et des cercles
que l'on peut tracer à partir de $\construc_j$.
Nous allons voir que ce passage correspond à une suite d'extensions quadratiques.

Soit donc $K$ le plus petit corps contenant les coordonnées des points de $\construc_j$.
Nous considérons $P$ un point de $\construc_{j+1}$.
Ce point $P$ est l'intersection de deux objets (deux droites ; une droite et un cercle ;
deux cercles).
Distinguons les cas :
\begin{itemize}
  \item \emph{$P$ est l'intersection de deux droites.} Ces droites passent par des points de $\construc_j$
  donc elles ont pour équations $ax+by=c$ et $a'x+b'y=c'$ et il est important de noter que l'on peut prendre $a,b,c,a',b',c'$ comme étant des éléments de $K$.
Par exemple une équation de la droite passant par $A(x_A,y_A)$ et $B(x_B,y_B)$ 
(avec $x_A,y_A,x_B,y_B \in K$) est 
$y = \frac{y_B-y_A}{x_B-x_A} (x-x_A) + y_A$, ce qui donne bien une équation à coefficients dans $K$. 
  Les coordonnées de $P$ sont donc 
  $$\left(\frac{cb'-c'b}{ab'-a'b} , \frac{ac'-a'c}{ab'-a'b} \right).$$
  Comme $K$ est un corps alors l'abscisse et l'ordonnée de ce $P$ sont encore dans $K$.
  Dans ce cas il n'y a pas besoin d'extension : le plus petit corps contenant les coordonnées des points de $\construc_j$ et de $P$ est $K$.
  
  \item \emph{$P$ appartient à l'intersection d'une droite et d'un cercle.} Notons l'équation de la droite $ax+by=c$
  avec $a,b,c \in K$ et $(x-x_0)^2+(y-y_0)^2=r^2$ l'équation du cercle. On note que $x_0,y_0,r^2$ (mais pas nécessairement $r$) 
sont des éléments de $K$ car les coordonnées du centre et d'un point du cercle sont dans $K$.
  Il reste à calculer les intersections de la droite et du cercle : en posant
  $$\delta = {-2\,x_{{0
}}{a}^{3}by_{{0}}+2\,y_{{0}}{a}^{2}cb-{b}^{2}{y_{{0}}}^{2}{a}^{2}+{b}^
{2}{r}^{2}{a}^{2}+2\,{a}^{3}x_{{0}}c-{a}^{4}{x_{{0}}}^{2}-{a}^{2}{c}^{
2}+{a}^{4}{r}^{2}} \in K,$$
  on trouve deux points $(x,y)$, $(x',y')$ avec 
    $$x =- \frac ba \frac{1}{a^2+b^2} \left(-x_{{0}}ab+y_{{0}}{a}^{2}+cb - \frac cb (a^2+b^2) \mathbf{+} \sqrt \delta \right) \quad \text{ et } \quad y = \frac{c-ax}{b},$$
  $$x' =- \frac ba \frac{1}{a^2+b^2} \left(-x_{{0}}ab+y_{{0}}{a}^{2}+cb - \frac cb (a^2+b^2) \mathbf{-} \sqrt \delta \right) \quad \text{ et } \quad y' = \frac{c-ax'}{b}.$$
Les coordonnées sont bien de la forme $\alpha+\beta\sqrt{\delta}$ avec $\alpha,\beta \in K$
et c'est le même $\delta \in K$ pour $x,y,x',y'$.
Donc les coordonnées de $P$ sont bien dans l'extension quadratique $K(\sqrt \delta)$.
  
  \item \emph{$P$ appartient à l'intersection de deux cercles.} On trouve aussi deux points $(x,y)$, 
  $(x',y')$ et $x,y,x',y'$ sont aussi de la forme $\alpha+\beta\sqrt\delta$
  pour un certain $\delta \in K$ fixé et $\alpha,\beta\in K$. Les formules sont plus longues à écrire
  et on se contentera ici de faire un exemple (voir juste après).


%On pourrait donner un autre argument : l'intersection du cercle $\cercle$ centré en $O$ et du cercle $\cercle'$
%centré en $O'$ est aussi l'intersection du cercle $\cercle$ avec la médiatrice de $[OO']$. (Exercice :
%justifier que cette médiatrice est constructible sans étendre le corps.) 
%On se ramène donc au cas de l'intersection d'un cercle et d'une droite.
%
%\bigskip
%
%\myfigure{0.7}{
%\tikzinput{fig_compas45}
%}  

\end{itemize}

\bigskip

En résumé, dans tous les cas, les coordonnées de $P$ sont dans une extension quadratique du corps $K$,
qui contient les coefficients qui servent à construire $P$.


\medskip

Voici comment terminer la démonstration par une récurrence sur $j$.
Soit $K$ le plus petit corps contenant les coordonnées des points de $\construc_j$.
On suppose par récurrence que $K$ s'obtient par une suite d'extensions quadratiques de $\Qq$. 
Soit $P_1$ un point de $C_{j+1}$, alors nous venons de voir que le corps 
$K_1=K(d_1)$ correspondant est une extension quadratique de $K$.
Ensuite soit $P_2$ un autre point, toujours dans $C_{j+1}$, cela donne une autre extension quadratique $K_2=K(\delta_2)$ de $K$, mais on considère plutôt $K'_2 = K(d_1,d_2)=K_1(d_2)$
 comme une extension (au plus) quadratique de $K_1$.

On fait de même pour tous les points de $C_{j+1}$ et on construit une extension $K(d_1,d_2,...,d_p)$ de $K$
qui correspond à toutes les coordonnées des points de $C_{j+1}$. 
Par construction c'est bien une suite d'extension quadratique de $K$ donc de $\Qq$. 
\end{proof}


\begin{exemple}
Donnons les extensions nécessaires dans chacun des trois cas de la preuve sur un exemple concret.
\begin{enumerate}
  \item $P$ est l'intersection de deux droites $(AB)$ et $(A'B')$ avec par exemple
  $A(0,1)$, $B(1,-1)$, $A'(0,-2)$, $B'(1,1)$ dont les coordonnées sont dans $K=\Qq$. Les équations sont
  $2x+y=1$ et $3x-y=2$ ; 
  le point d'intersection $P$ a pour coordonnées $(\frac 35, -\frac 15)$,
  donc l'abscisse et l'ordonnée sont dans $\Qq$. Il n'y a pas besoin d'étendre le corps.
\myfigure{1}{
\tikzinput{fig_compas46}
}  
  
  \item $P$ et $P'$ sont les intersections de la droite passant par $A(0,1)$ et $B(1,-1)$ et du cercle 
  de centre $A'(2,1)$ passant par le point $B'(-1,1)$ (et donc de rayon $3$). Les équations sont 
  alors $2x+y=1$ et $(x-2)^2+(y-1)^2=9$. Les deux solutions sont les points :
  $$\left(\frac 1 5 \left(2- \sqrt{29}\right),\frac 1 5 \left(1+2\sqrt{29}\right) \right),
  \left(\frac 1 5 \left(2 + \sqrt{29}\right),\frac 1 5 \left(1-2\sqrt{29}\right) \right).$$
  Donc si on pose $\delta = 29$ (qui est bien un rationnel) alors les coordonnées des points d'intersection
  sont de la forme $\alpha+\beta\sqrt{\delta}$ ($\alpha,\beta \in \Qq$), c'est-à-dire appartiennent à l'extension quadratique $\Qq(\sqrt{29})$.
\myfigure{1}{
\tikzinput{fig_compas47}
}  
  
  \item Soient le cercle $\cercle((-1,0),2)$ (qui a pour centre $A(-1,0)$ et passe par $B(1,0)$) 
  et le cercle $\cercle((2,1),\sqrt{5})$ (qui a pour centre $A'(2,1)$ et passe par $B'(0,0)$). Les équations sont
  $(x+1)^2+y^2=4$, $(x-2)^2+(y-1)^2=5$. Les deux points d'intersection sont :
  $$\left(\frac{1}{20}\left( 7 - \sqrt{79}\right),\frac{3}{20}\left( 3 + \sqrt{79}\right)\right),
  \left(\frac{1}{20}\left( 7 + \sqrt{79}\right),\frac{3}{20}\left( 3 - \sqrt{79}\right)\right).$$
  Encore une fois, pour le rationnel $\delta = 79$, les abscisses et ordonnées des points d'intersection
  sont de la forme $\alpha+\beta\sqrt\delta$ avec $\alpha,\beta \in \Qq$ ;
  l'extension quadratique qui convient est donc $\Qq(\sqrt{79})$.
\myfigure{1}{
\tikzinput{fig_compas48}
}
\end{enumerate}

\end{exemple}


%---------------------------------------------------------------
\subsection{Corollaires}

La conséquence la plus importante du théorème de Wantzel
est donnée par l'énoncé suivant. C'est ce résultat que l'on utilisera 
dans la pratique.

\begin{corollaire}
Tout nombre réel constructible est un nombre algébrique
dont le degré algébrique est de la forme $2^n$, $n\ge 0$. 
\end{corollaire}

\begin{proof}
Soit $x$ un nombre constructible. Par le théorème de Wantzel,
il existe des extensions quadratiques 
$\Qq = K_0 \subset K_1 \subset \cdots \subset K_r$
telles que $x \in K_r$. 
Donc $x$ appartient à une extension de $\Qq$ de degré fini. Ainsi, par 
la proposition \ref{prop:ext}, $x$ est un nombre algébrique.


On sait de plus que $[K_{i+1}:K_i]=2$,
donc par la proposition \ref{prop:KLM},
nous avons $[K_r:\Qq]=2^r$.
Il nous reste à en déduire le degré algébrique $[\Qq(x):\Qq]$.
Comme $\Qq(x) \subset K_r$, alors nous avons toujours par la proposition 
\ref{prop:KLM} que : $[K_r:\Qq(x)] \times [\Qq(x):\Qq]=[K_r:\Qq]=2^r$.
Donc $[\Qq(x):\Qq]$ divise $2^r$ et est donc de la forme $2^n$.
\end{proof}

Voici une autre application plus théorique du théorème de Wantzel, 
qui caractérise les nombres constructibles.

\begin{corollaire}
$\construc_\Rr$ est le plus petit sous-corps de $\Rr$ stable
par racine carrée, c'est-à-dire tel que :
\begin{itemize}
  \item $(x \in \construc_\Rr \text{ et } x \ge 0) \Rightarrow \sqrt x \in \construc_\Rr$,
  \item si $K$ est un autre sous-corps de $\Rr$ stable par racine carrée alors $\construc_\Rr \subset K$.
\end{itemize}
\end{corollaire}

La preuve est à faire en exercice.

%%%%%%%%%%%%%%%%%%%%%%%%%%%%%%%%%%%%%%%%%%%%%%%%%%%%%%%%%%%%%%%%
\section{Applications aux problèmes grecs}

Nous allons pouvoir répondre aux problèmes 
de la trisection des angles, de la duplication du cube et de la quadrature du cercle, tout cela en même temps !
Il aura fallu près de 2\,000 ans pour répondre à ces questions. Mais pensez que, pour montrer qu'une construction
est possible, il suffit de l'exhiber (même si ce n'est pas toujours évident). Par contre pour montrer 
qu'une construction n'est pas possible, c'est complètement différent. Ce n'est pas parce que personne n'a réussi
une construction qu'elle n'est pas possible ! Ce sont les outils algébriques qui vont permettre de 
résoudre ces problèmes géométriques.


Rappelons le corollaire au théorème de Wantzel, qui va être la clé pour nos problèmes.

\begin{corollaire}
\sauteligne
\begin{enumerate}
  \item Si un nombre réel $x$ est constructible alors $x$ est un nombre algébrique. 
  C'est-à-dire qu'il existe un polynôme $P \in \Qq[X]$ tel que $P(x)=0$.
  
  \item De plus le degré algébrique de $x$ est de la forme $2^n$, $n\ge 0$. 
  C'est-à-dire que le plus petit degré, parmi tous les degrés des polynômes 
  $P \in \Qq[X]$ vérifiant $P(x)=0$, est une puissance de $2$.
\end{enumerate}





\end{corollaire}

%---------------------------------------------------------------
\subsection{L'impossibilité de la duplication du cube}


\begin{center}
\shadowbox{\begin{minipage}{0.7\textwidth}
La duplication du cube ne peut pas s'effectuer à la règle et au compas.
\end{minipage}}
\end{center}

Cela découle du fait suivant :
\begin{theoreme}
$\sqrt[3]{2}$ n'est pas un nombre constructible.
\end{theoreme}

\begin{proof}
$\sqrt[3]{2}$ est une racine du polynôme $P(X)=X^3-2$.
Ce polynôme est unitaire et irréductible dans $\Qq[X]$,
donc $\sqrt[3]{2}$ est un nombre algébrique de degré $3$.
Ainsi son degré algébrique n'est pas de la forme $2^n$. Bilan : 
$\sqrt[3]{2}$ n'est pas constructible.
\end{proof}

%---------------------------------------------------------------
\subsection{L'impossibilité de la quadrature du cercle}

\begin{center}
\shadowbox{\begin{minipage}{0.7\textwidth}
La quadrature du cercle ne peut pas s'effectuer à la règle et au compas.
\end{minipage}}
\end{center}

C'est une reformulation du théorème suivant, dû à Ferdinand von Lindemann (en 1882) :
\begin{theoreme}
$\pi$ n'est pas un nombre algébrique (donc n'est pas constructible).
\end{theoreme}

Comme $\pi$ n'est pas constructible, alors $\sqrt \pi$ n'est pas constructible non plus
(c'est la contraposée de $x \in \construc_\Rr \implies x^2 \in \construc_\Rr$).

Nous ne ferons pas ici la démonstration que $\pi$ n'est pas un nombre algébrique,
mais c'est une démonstration qui n'est pas si difficile et abordable en première année.


%---------------------------------------------------------------
\subsection{L'impossibilité de la trisection des angles}

\begin{center}
\shadowbox{\begin{minipage}{0.7\textwidth}
La trisection des angles ne peut pas s'effectuer à la règle et au compas.
\end{minipage}}
\end{center}

Plus précisément nous allons exhiber un angle que l'on ne peut pas couper en trois.
\begin{theoreme}
L'angle $\frac{\pi}{3}$ est constructible mais ne peut pas être coupé en trois car 
$\cos \frac \pi 9$ n'est pas un nombre constructible.
\end{theoreme}

Bien sûr l'angle $\frac\pi3$ est constructible car $\cos \frac\pi3 = \frac12$.
La preuve de la non constructibilité de l'angle $\frac \pi 9$ fait l'objet d'un exercice : 
$\cos \frac \pi 9$ est un nombre algébrique de degré 
algébrique $3$, donc il n'est pas constructible.

\myfigure{1}{
  \tikzinput{fig_compas49}
}  

La trisection n'est donc pas possible en général, mais attention,
pour certains angles particuliers c'est possible : par exemple les angles $\pi$ ou $\frac \pi 2$ !

%%%%%%%%%%%%%%%%%%%%%%%%%%%%%%%%%%%%%%%%%%%%%%%%%%%%%%%%%%%%%%%%


%%%%%%%%%%%%%%%%%%%%%%%%%%%%%%%%%%%%%%%%%%%%%%%%%%%%%%%%%%%%%%%%
\section{Exercices}

%%%%%%%%%%%%%%%%%%%%%%%%%%%%%%%%%%%%%%%%%%%%%%%%%%%%%%%%%%%%%%%%
\subsection{Constructions élémentaires}


\begin{exercicecours}[Constructions élémentaires]
\'Etant donné deux points $A,B$, dessiner à la règle et au compas :
\begin{enumerate}
    \item un triangle équilatéral de base $[AB]$,
    \item un carré de base $[AB]$,
    \item un rectangle dont l'un des côtés est $[AB]$ et l'autre est de longueur double,
    \item un losange dont l'une des diagonales est $[AB]$ 
    et l'autre est de longueur $\frac14 AB$,    
    \item un hexagone régulier dont l'un des côtés est $[AB]$.    
\end{enumerate} 
\end{exercicecours}


\begin{exercicecours}[Centres de cercles]
\sauteligne
\begin{itemize}
  \item \'Etant donné un cercle dont on a perdu le centre, retrouver 
  le centre à la règle et compas.
  
  \item \'Etant donné un triangle, tracer son cercle circonscrit.
  
  \item \'Etant donné un triangle, tracer son cercle inscrit.  
\end{itemize}
\end{exercicecours}



%%%%%%%%%%%%%%%%%%%%%%%%%%%%%%%%%%%%%%%%%%%%%%%%%%%%%%%%%%%%%%%%%%%%%%%%%%%%%%%%%%%%%%%%%%%%%%

\begin{exercicecours}[Construction au compas seul]
Construire au compas seulement :
\begin{enumerate}
 \item Le symétrique de $P$ par rapport à une droite $(AB)$. (Seuls les points $P, A, B$ sont tracés, 
 pas la droite.)

 \item Le symétrique d'un point $P$ par rapport à un point $O$.

 \item(*) Le milieu de deux points $A$, $B$. (La droite $(AB)$ n'est pas tracée !)
\end{enumerate}
\end{exercicecours}



\begin{exercicecours}[Pentagone régulier]
% \exercice{77, bodin, 1998/09/01}
% \video{wHlb0IsMB7Q}

Soit $(A_{0},A_{1},A_{2},A_{3},A_{4})$ un pentagone r\'egulier. On note $O$ son centre et on
choisit un rep\`ere orthonorm\'e $(O,\overrightarrow{u},\overrightarrow{v})$ avec
$\overrightarrow{u}=\overrightarrow{OA_{0}}$, qui nous permet d'identifier le plan avec
l'ensemble des nombres complexes $\Cc$.


\begin{enumerate}
\item
Donner les affixes $\omega_{0},\ldots,\omega_{4}$ des points $A_{0},\ldots,A_{4}$. Montrer
que  $\omega_{k}={\omega_{1}}^k$ pour $ k\in\{0,1,2,3,4\}$. Montrer que
$1+\omega_{1}+\omega_{1}^2+\omega_{1}^3+\omega_{1}^4=0$.

\item
En d\'eduire que $\cos(\frac{2\pi}{5})$ est l'une des solutions de l'\'equation $4z^2+2z-1=0$.
En d\'eduire la valeur de $\cos(\frac{2\pi}{5})$.

\item
On consid\`ere le point $B$ d'affixe $-1$. Calculer la longueur $BA_{2}$ en fonction de
$\cos\frac{2\pi}{5}$ puis de $\sqrt{5}$.

\item
On consid\`ere le point $I$ d'affixe $\frac{\ii}{2}$, le cercle $\mathcal{C}$ de centre $I$ de
rayon $\frac{1}{2}$ et enfin le point $J$ d'intersection de $\mathcal{C}$ avec le segment
$[BI]$. Calculer la longueur $BI$ puis  la longueur $BJ$.

\item
\textbf{Application:} Dessiner un pentagone r\'egulier \`a la r\`egle et au compas. Expliquer.
\end{enumerate}


\myfigure{1}{
  \tikzinput{fig_compas_exo_pentagone}
}


%\correction
%\begin{enumerate}
%\item
%Comme $(A_{0},\ldots,A_{4})$ est un pentagone r\'egulier, on a
%$OA_{0}=OA_{1}=OA_{2}=OA_{3}=OA_{4}=1$ et $
%  (\overrightarrow{OA_{0}},\overrightarrow{OA_{1}})=\frac{2\pi}{5}[2\pi],
%  (\overrightarrow{OA_{0}},\overrightarrow{OA_{2}})=\frac{4\pi}{5}[2\pi],
%  (\overrightarrow{OA_{0}},\overrightarrow{OA_{3}})=-\frac{4\pi}{5}[2\pi],
%  (\overrightarrow{OA_{0}},\overrightarrow{OA_{4}})=-\frac{2\pi}{5}[2\pi],
% $.
%On en d\'eduit:
% $
%  \omega_{0}=1,
%  \omega_{1}=e^{\frac{2\ii\pi}{5}},
%  \omega_{2}=e^{\frac{4\ii\pi}{5}},
%  \omega_{3}=e^{-\frac{4\ii\pi}{5}}=e^{\frac{6\ii\pi}{5}},
%  \omega_{4}=e^{-\frac{2\ii\pi}{5}}=e^{\frac{8\ii\pi}{5}},
% $.
%On a bien $\omega_{i}=\omega_{1}^i$. Enfin, comme
%$\omega_{1}\neq0$, $1+\omega_{1}+\ldots+\omega_{1}^4=
%\frac{1-\omega_{1}^5}{1-\omega_{1}}=\frac{1-1}{1-\omega_{1}}=0$.
%
%\item $\mathop{\mathrm{Re}}\nolimits(1+\omega_{1}+\ldots+\omega_{1}^4)=
%1+2\cos(\frac{2\pi}{5})+2\cos(\frac{4\pi}{5})$. Comme
%$\cos(\frac{4\pi}{5})=2\cos^2(\frac{2\pi}{5})-1$ on en d\'eduit:
%$4\cos^2(\frac{2\pi}{5})+2\cos(\frac{2\pi}{5})-1=0$.
%$\cos(\frac{2\pi}{5})$ est donc bien une solution de l'\'equation
%$4z^2+2z-1=0$. Etudions cette \'equation: $\Delta=20=2^2.5$. Les
%solutions sont donc $\frac{-1-\sqrt{5}}{4}$ et
%$\frac{-1+\sqrt{5}}{4}$. Comme $\cos(\frac{2\pi}{5})>0$, on en
%d\'eduit que $\cos(\frac{2\pi}{5})=\frac{\sqrt{5}-1}{4}$.
%
%\item
% $
%  BA_{2}^2=|\omega_{2}+1|^2
%          =|\cos(\frac{4\pi}{5})+i\sin(\frac{4\pi}{5})+1|^2
%          =1+2\cos(\frac{4\pi}{5})+\cos^2(\frac{4\pi}{5})+\sin^2(\frac{4\pi}{5})
%          =4\cos^2(\frac{2\pi}{5})
%  $. Donc $BA_{2}=\frac{\sqrt{5}-1}{2}$.
%
%\item
%$BI=|\ii/2+1|=\frac{\sqrt{5}}{2}$. $BJ=BI-1/2=\frac{\sqrt{5}-1}{2}$.
%
%\item
%Pour tracer un pentagone r\'egulier, on commence par tracer un
%cercle $C_{1}$ et deux diam\`etres orthogonaux, qui jouent le r\^ole
%du cercle passant par les sommets et des axes de coordonn\'ees. On
%trace ensuite le milieu d'un des rayons: on obtient le point I de
%la question 4. On trace le  cercle de centre $I$ passant par le
%centre de $C_{1}$: c'est le cercle $\mathcal{C}$. On trace le
%segment $[BI]$ pour obtenir son point $J$ d'intersection avec
%$\mathcal{C}$. On trace enfin le cercle de centre $B$ passant par
%$J$: il coupe $C_{1}$ en $A_{2}$ et $A_{3}$, deux sommets du
%pentagone. Il suffit pour obtenir tous les sommets de reporter la
%distance $A_{2}A_{3}$ sur $C_{1}$, une fois depuis $A_{2}$, une
%fois depuis $A_{3}$. (en fait le cercle de centre $B$ et passant
%par $J'$, le point de $\mathcal{C}$ diam\'etralement oppos\'e \`a
%$J$, coupe $C_{1}$ en $A_{1}$ et $A_{4}$, mais nous ne l'avons pas
%justifi\'e par le calcul : c'est un exercice !)
%\end{enumerate}
%% $$
%% \includegraphics[65mm,50mm]{penta.eps}
%% $$
%\fincorrection
\end{exercicecours}

%%%%%%%%%%%%%%%%%%%%%%%%%%%%%%%%%%%%%%%%%%%%%%%%%%%%%%%%%%%%%%%%
\subsection{Les nombres constructibles à la règle et au compas}


\begin{exercicecours}[Construction de l'axe des ordonnées]
\sauteligne
\begin{enumerate}
  \item Montrer qu'il possible de construire un point de l'axe des ordonnées 
  (autre que l'origine) dans l'ensemble des points constructibles $\mathcal{C}_3$.
  
  \item En déduire que $\mathcal{C}_4$ contient le point $(0,1)$.
\end{enumerate}
\end{exercicecours}


\begin{exercicecours}[Constructions approchant $\pi$]
\sauteligne
\begin{enumerate}
 \item Construire les approximations suivantes de $\pi$ : 
$\frac{22}{7}=3,1428\ldots$, $\sqrt{2}+\sqrt{3}=3,1462\ldots$
 
 \item Construire $\frac{\sqrt{5}}{\sqrt{3}}$, $10^{\frac{1}{4}}$, $\sqrt{10-\sqrt{3}}$. 
\end{enumerate}
\end{exercicecours}


\begin{exercicecours}[Approximation de Kochanski]
Une valeur approchée de $\pi$ avec quatre décimales exactes est donnée par
$$\phi = \sqrt{\frac{40}{3} - 2 \sqrt{3}} = 3,141533\ldots$$
Soient les points suivants $O(0,0)$, $I(1,0)$, $Q(2,0)$. 
On construit les points $P_1, P_2,\ldots$ ainsi:
\begin{itemize}
 \item $P_1$ est l'intersection des cercles centrés en $O$ et $I$ de rayon $1$ ayant une ordonnée positive.
 \item $P_2$ est l'intersection des cercles centrés en $O$ et $P_1$ de rayon $1$ (l'autre intersection est $I$).
 \item $P_3$ est l'intersection de  la droite $(P_2I)$ avec l'axe des ordonnées.
 \item $P_4$ est l'image de $P_3$ par une translation de vecteur $(0,-3)$.
\end{itemize}
Calculer les coordonnées de chacun des $P_i$.
Montrer que la longueur $P_4Q$ vaut $\phi$.
\end{exercicecours}

%%%%%%%%%%%%%%%%%%%%%%%%%%%%%%%%%%%%%%%%%%%%%%%%%%%%%%%%%%%%%%%%
\subsection{\'Eléments de théorie des corps}


\begin{exercicecours}[Extensions quadratiques]
\sauteligne
\begin{enumerate}
  \item Montrer que $\sqrt{3} \notin \Qq(\sqrt2)$.
  \item Montrer que $\sqrt{1+\sqrt2} \notin \Qq(\sqrt2)$.
  \item Montrer que $\sqrt 5 \notin \Qq(\sqrt2)(\sqrt 3)$.
  \item Trouver un polynôme $P \in \Qq[X]$ tel que que $P(\sqrt{1+\sqrt2}) = 0$.   
  \item Trouver un polynôme $P \in \Qq[X]$ tel que que $P(\sqrt2 + \sqrt 3) = 0$. 
  % C'est x4 − 10x2 + 1 = (x − √2 − √3)(x + √2 − √3)(x − √2 + √3)(x + √2 + √3)
  \item Montrer que $\Qq(\sqrt[3]{2})$ 
\end{enumerate}
\end{exercicecours}


\begin{exercicecours}[Degré algébrique]
\sauteligne
\begin{enumerate}
  \item Soit $x \in \Rr$ montrer que $\Qq(x)$, le plus petit corps contenant $x$, vérifie :
  $$\Qq(x) =\left\{ \frac{P(x)}{Q(x)} \mid P,Q \in \Qq[X] \text{ et } Q(x) \neq 0 \right\}.$$
  
  \item Montrer que $K = \left\{ a+b\sqrt[3]{2}+c\sqrt[3]{2}^2 \mid a,b,c \in \Qq \right\}$ est un corps.
  
  \item Montrer que  $K=\Qq(\sqrt[3]{2})$ ; c'est-à-dire que $K$ est le 
  plus petit corps contenant $\Qq$ et $\sqrt[3]{2}$.
  
  \item Vérifier sur cet exemple que le degré algébrique de $\sqrt[3]{2}$ égale 
  le degré de l'extension $[\Qq(\sqrt[3]{2}):\Qq]$.
  
  \item Expliciter une extension de $\Qq$ ayant le degré $4$.
\end{enumerate}

%\correction
%\begin{enumerate}
%  \item 
%  \begin{enumerate}
%    \item   
%    \item    
%  \end{enumerate} 
%  
%  \item Montrons que $K = \left\{ a+b\sqrt[3]{2}+c\sqrt[3]{2}^2 \mid a,b,c \in \Qq \right\}$ est un corps
%  L'addition et la multiplication définie sur $K$ sont celles du corps $(\Rr,+,\times)$.
%  Beaucoup de propriétés découlent du fait que l'ensemble des réels est un corps.
%  
%  [pb : mq $+$ et $\times$ lci]
%  
% \begin{enumerate}
%  \item $(K,+)$ est un groupe commutatif, car :
%    \begin{itemize}
%      
%      \item $0 \in K$ (prendre $a=b=c=0$) et $0+x = x$ (pour tout $x\in K$).
%      
%      \item Si $x\in K$ alors $-x \in K$.
%      
%      \item $+$ est associative : cela découle de l'associativité sur $\Rr$.
%      
%      \item $x+y=y+x$ : idem.
%    \end{itemize} 
%  
%  \item $(K\setminus\{0\},\times)$ est un groupe commutatif, en effet :
%     \begin{itemize} 
%      \item $1 \in K\setminus\{0\}$ et $1 \times x = x$ (pour tout $x\in K\setminus\{0\}$).
%      
%      \item Si $x\in K\setminus\{0\}$ alors $x^{-1} \in K\setminus\{0\}$ :
%      [à faire]
%      \item $\times$ est associative :  cela découle de l'associativité sur $\Rr\setminus\{0\}$.
%      
%      \item $x\times y=y\times x$ : .     
%
%    \end{itemize} 
%    
%  \item $\times$ est distributive par rapport à $+$ : cela découle de la distributivité sur $\Rr$.
% \end{enumerate}
%  
%  \begin{enumerate}
%    \item   
%    \item    
%  \end{enumerate} 
%  
%  \item Par définition $\Qq(\sqrt[3]{2})$ est le plus petit corps contenant $\Qq$ et $\sqrt[3]{2}$.
%  Mais il est clair que $K = \left\{ a+b\sqrt[3]{2}+c\sqrt[3]{2}^2 \mid a,b,c \in \Qq \right\}$ contient $\sqrt[3]{2}$ (prendre $a=0$, $b=1$, $c=0$) et que $\Qq \subset K$.
%  Donc comme $\Qq(\sqrt[3]{2}) \subset K$.
%  
%  Nous allons maintenant montrer la partie "le plus petit". Soit donc $K'$ un autre corps contenant
%  $\Qq$ et $\sqrt[3]{2}$. On veut montrer $K \subset K'$.
%  $K'$ doit contenir tout élément $a\in \Qq$ et contient $\sqrt[3]{2}$ donc il contient
%  tout élément de la forme $a+b \sqrt[3]{2}$ (avec $a,b \in \Qq$).
%  Mais comme $K'$ est un corps contenant $\sqrt[3]{2}$ il contient aussi $\sqrt[3]{2}^2$ et donc tout élément de la forme
%  $a+b\sqrt[3]{2}+c\sqrt[3]{2}^2$ avec ($a,b,c \in \Qq$). Ainsi $K \subset K'$.
%  
%  Conclusion : $K$ est bien le plus petit corps contenant  $\Qq$ et $\sqrt[3]{2}$, c'est-à-dire
%  $K = \Qq(\sqrt[3]{2})$.
%  
%  
%  \item
%  \begin{enumerate}
%    \item Le degré algébrique d'un réel $x$ est le plus petit degré 
%    d'un polynôme $P\in\Qq[X]$ tel que $P(x)=0$.
%    Pour $x = \sqrt[3]{2}$, un polynôme qui annule $x$ est $P(X) = X^3-2$.
%    Donc le degré algébrique de $x$ est $\le 3$. 
%    Mais le nombre algébrique de degré $1$ sont les rationnels, alors que le nombre réels
%    algébrique de degré $2$ sont de la forme $a+b \sqrt{\delta}$ (avec $a,b,\delta\in\Qq$).
%    Donc $x = \sqrt[3]{2}$ n'est ni de degré $1$, ni de degré $2$. Ainsi le degré algébrique de
%    $x = \sqrt[3]{2}$ est $3$.
%    
%    \item On a vu que $K=\Qq(\sqrt[3]{2}) = \left\{ a+b\sqrt[3]{2}+c\sqrt[3]{2}^2 \mid a,b,c \in \Qq \right\}$
%    c'est donc un $\Qq$-espace vectoriel dont une base est $(1,\sqrt[3]{2},\sqrt[3]{2}^2)$.
%    C'est donc un espace vectoriel sur $\Qq$ de dimension $3$.
%    Ainsi le degré de l'extension $[\Qq(\sqrt[3]{2}):\Qq] = \dim_\Qq K = 3$.
%  
%  \end{enumerate}  
%  C'est un résultat plus général (voir le cours) : le degré algébrique d'un réel $x$
%  égale le degré de l'extension $[\Qq(x):\Qq]$.
%  
%  \item 
%  \begin{enumerate}
%    \item 
%  Premier exemple avec $x = \sqrt{\sqrt{2}} = $, alors
%  $$\Qq\big(2^{\frac14}\big) = \big\{ a + b 2^{\frac14} + c 2^{\frac24} + d  2^{\frac34} \mid a,b,c,d \in \Qq \big\}.$$
%  C'est une extension de $\Qq$ degré $4$.
%    \item
%  Second exemple avec $x = \sqrt{2}\ii$, alors
%  $$\Qq\big(\sqrt{2}\ii\big) = \big\{ a + b \sqrt{2} + c \ii  + d \sqrt{2}\ii  \mid a,b,c,d \in \Qq   \big\}.$$
%  C'est aussi une extension de $\Qq$ de degré $4$.
%  \end{enumerate}
%\end{enumerate}

\end{exercicecours}


\begin{exercicecours}[Nombres transcendants]
\sauteligne
\begin{enumerate}
 \item Montrer que l'ensemble des nombres réels algébriques est un ensemble dénombrable.
 \item En déduire l'existence de nombres réels qui ne soient pas algébriques.
\end{enumerate}

%\indication
\emph{Indications :}
\begin{enumerate}
  \item Un nombre algébrique est par définition
  une racine d'un polynôme $P$ de $\Qq[X]$. 
  
  \item $\Rr$ n'est pas dénombrable.
\end{enumerate}
%\finindication
%
%\correction
%\begin{enumerate}
%  \item Un nombre $x \in \Rr$ est un \emph{nombre algébrique}
%  s'il existe un polynôme $P \in \Qq[X]$, tel que $P(x)=0$.
%  Pour un degré $n$ fixé, il y a un nombre dénombrable de polynômes
%  de degré $n$ à coefficients rationnels. En effet, un tel polynôme s'écrit :
%  $$P(X) = a_nX^n + a_{n-1}X^{n-1}+\cdots +a_1X+a_0$$
%  Et comme la liste des coefficients forment un sous-ensemble de $\Qq^{n+1}$, 
%  c'est un ensemble  dénombrable.
%  Chaque polynôme de degré $n$, admet au plus $n$ racines. Donc 
%  l'ensemble des racines des polynômes de degré $n$ est un ensemble dénombrable.
%  
%  Enfin on décompose l'ensemble $\overline{\Qq}$ des nombres algébriques comme
%  l'union (sur l'indice $n$) des racines des polynômes de degré $n$ :
%  $$\overline{\Qq} = \bigcup_{n=0}^{+\infty} \Big\{ x \in \Rr \mid 
%  \exists Q \in \Qq[X] \text{ avec } \deg Q = n \text{ et } Q(x)=0 \Big\}$$
%  Ainsi $\overline{\Qq}$ est l'union dénombrable d'ensembles dénombrables, ce qui implique
%  que $\overline{\Qq}$ est un ensemble dénombrable.
%  
%  \item $\overline{\Qq}$ est un sous-ensemble dénombrable de $\Rr$, qui lui n'est pas dénombrable.
%  Ainsi $\Rr \setminus \overline{\Qq}$ est un ensemble non dénombrable, 
%  et en particulier non vide !
%\end{enumerate}
%
%Remarque : un nombre qui n'est pas algébrique s'appelle un \emph{nombre transcendant}.
%Notre démonstration prouve l'existence de nombres transcendants, mais ne nous permet 
%pas d'en expliciter un seul. Par exemples $\pi$, $e$ sont des nombres transcendants.
%
%\fincorrection

\end{exercicecours}


%%%%%%%%%%%%%%%%%%%%%%%%%%%%%%%%%%%%%%%%%%%%%%%%%%%%%%%%%%%%%%%%
\subsection{Corps et nombres constructibles}

\begin{exercicecours}[Corps stables par racine carrée]
Un corps $K \subset \Rr$ est \emph{stable par racine carrée} s'il vérifie la propriété suivante:
$$\forall x \in K \qquad   x  \ge 0 \Rightarrow \sqrt x \in K.$$

Montrer que l'ensemble $\mathcal{C}_\Rr$ des nombres constructibles est 
le plus petit sous-corps de $\Rr$ stable par racine carrée.

%\indication
\emph{Indication :} Utiliser le théorème de Wantzel.
%\finindication
%
%\correction
%Il y a une étape facile : le sous-corps $\mathcal{C}_\Rr$ est stable par racine carrée.
%En effet on a vu que si $x \in \mathcal{C}_\Rr$ est un nombre constructible, alors
%$\sqrt x$ est aussi constructible, donc $\sqrt x \in \mathcal{C}_\Rr$.
%
%\bigskip
%
%Soit maintenant $K$ un corps stable par racine carrée, il s'agit de montrer que
%$\mathcal{C}_\Rr \subset K$.
%Par le théorème de Wantzel, si $x \in \mathcal{C}_\Rr$ est un nombre constructible alors
%il existe une suite d'extensions quadratiques :
%$$\Qq = K_0 \subset K_1 \subset \cdots \subset K_r$$
%telles que $x \in K_r$.
%
%\bigskip
%
%Montrons par récurrence que $K_i \subset K$.
%\begin{itemize}
%  \item Comme $K_0 = \Qq$ alors $K_0 \subset K$.
%  
%  \item Supposons $K_{i-1} \subset K$. Le fait que $K_i$ est une extension quadratique de 
%  $K_{i-1}$, signifie qu'il existe $\delta \in K_{i-1}$ tel que  $K_i = K_{i-1}(\sqrt{\delta})$.
%  Pour $x \in K_i$ alors il existe $a,b \in K_{i-1}$ tels que $x = a + b \sqrt{\delta}$.
%  Par hypothèse de récurrence $a,b \in K_{i-1}$, donc $a,b \in K$.
%  Mais on sait aussi que $\delta \in K_{i-1}$, donc $\delta \in K$. Par hypothèse
%  $K$ est stable par racine carrée donc $\sqrt\delta \in K$.
%  Ainsi $a,b, \sqrt\delta \in K$, donc $x  = a + b \sqrt{\delta} \in K$.
%  Bilan : $K_i \subset K$.
%  
%  \item Par le principe de récurrence, on a prouvé $K_r \subset K$.
%\end{itemize}
%
%\bigskip
%
%
%C'est presque terminé. Pour $x \in \mathcal{C}_\Rr$, le théorème de Wantzel, nous a
%dit $x \in K_r$, et par ce que l'on vient de prouver $x \in K$.
%
%\bigskip
%
%Conclusion : $\mathcal{C}_\Rr$ est corps stable par racine carrée et tout autre corps
%stable par racine carrée contient $\mathcal{C}_\Rr$.
%Ainsi $\mathcal{C}_\Rr$ est bien  
%le plus petit sous-corps de $\Rr$ stable par racine carrée.

\end{exercicecours}



%%%%%%%%%%%%%%%%%%%%%%%%%%%%%%%%%%%%%%%%%%%%%%%%%%%%%%%%%%%%%%%%
\subsection{Applications aux problèmes grecs}



\begin{exercicecours}[Trissection des angles]
Le but est de montrer que tous les angles ne sont pas \emph{trissectables} (divisibles en trois) à la règle et au compas.
Nous allons le prouver pour l'angle $\frac \pi 3$ : plus précisément le point de coordonnées 
$(\cos \frac \pi 3,\sin \frac \pi 3)$ est constructible mais le point $(\cos \frac \pi 9,\sin \frac \pi 9)$ ne l'est pas.
\begin{enumerate}
 \item Exprimer $\cos (3\theta)$ en fonction de $\cos \theta$.
 \item Soit $P(X) = 4X^3-3X-\frac 12$. Montrer que $P(\cos \frac\pi 9) = 0$.
 Montrer que $P(X)$ est irréductible dans $\Qq[X]$.
 \item Conclure.
\end{enumerate}


\emph{Indications :}
\begin{enumerate}
  \item \`A l'aide des nombres complexes, calculer $(e^{\ii\theta})^3$ de deux façons.
  \item Tout d'abord montrer que s'il était réductible alors il aurait racine dans $\Qq$.
Si $\frac ab$ est cette racine avec $\pgcd(a,b)=1$ alors à partir de $P(\frac ab)=0$ obtenir une équation d'entiers.
  \item Utiliser le théorème de Wantzel.
\end{enumerate}

\end{exercicecours}



%%%%%%%%%%%%%%%%%%%%%%%%%%%%%%%%%%%%%%%%%%%%%%%%%%%%%%%%%%%%%%%%
\subsection{Constructions assistées}

\begin{exercicecours}[Spirale d'Archimède]
Soit $(\mathcal{S})$ la \emph{spirale d'Archimède} paramétrée par 
$$M_t = \big(t \cos(2\pi t),t \sin(2\pi t)\big), \quad t \ge 0.$$
\begin{enumerate}
 \item Tracer $\mathcal{S}$.
 
 \item \emph{Trissection des angles}. \'Etant tracée la spirale d'Archimède, construire avec la règle et le compas la  trissection d'un angle donné. 
 
 \emph{Indications.} Supposer l'angle $\theta < \frac \pi 2$. \'Ecrire l'angle sous la forme $\theta = 2\pi t$ ; placer $M_t$ et tracer un cercle centré à l'origine du bon rayon.
 
 \item \emph{Quadrature du cercle.}
   \begin{enumerate}
     \item Calculer une équation de la tangente à $(\mathcal{S})$ en un point $M_t$.
     \item La tangente en $M_1$ (pour $t=1$) coupe l'axe des ordonnées en $N$. Calculer la longueur $ON$.
     \item En déduire qu'avec le tracé de la spirale d'Archimède et le tracé de la tangente en $M_1$ on peut résoudre la quadrature du cercle à la règle et au compas.
   \end{enumerate}
\end{enumerate}

\end{exercicecours}


\begin{exercicecours}[La règle tournante]
On souhaite trissecter les angles à l'aide d'un compas et d'une règle graduée.
Soit $\mathcal{C}$ le cercle de rayon $r>0$ centré en $O$.
Soient $A, B \in \mathcal{C}$ de telle sorte que 
l'angle en $O$ d'un triangle $OAB$ soit aigu. Notons $\theta$ cet angle.
\emph{Sur la règle} marquer deux points $O'$ et $C$ tel que $O'C=r$.
Faire pivoter et glisser la règle autour du point $B$ afin que 
le point $C$ appartienne à $\mathcal{C}$ et le point $O'$ appartiennent 
à la droite $(OA)$ (de sorte que $O$ soit dans le segment $[O'A]$).
Montrer que l'angle en $O'$ du triangle $AO'B$ vaut $\frac{\theta}{3}$.

\myfigure{1}{
  \tikzinput{fig_compas_exo_regle_tournante}
}

\end{exercicecours}



\begin{exercicecours}[Cissoïde de Dioclès]
Soient les points $O(0,0)$ et $I(1,0)$.
Soit $\mathcal{C}$ le cercle de diamètre $[OI]$
et $\mathcal{L}$ la droite d'équation $(x=1)$.
Pour un point $M$ de $\mathcal{C}$, soit $M'$ l'intersection de $(OM)$ avec $\mathcal{L}$.
Soit enfin $M''$ le point tel que $\overrightarrow{OM''}= \overrightarrow{MM'}$.
L'ensemble des points $M''$ lorsque $M$ parcourt $\mathcal{C}$ est \emph{la cissoïde
de Dioclès}, notée $\mathcal{D}$.

Le but de l'exercice est de montrer que la duplication du cube est possible à l'aide de la règle,
du compas et du tracé de la cissoïde.

\begin{enumerate}
 \item La droite $(OM)$ ayant pour équation $(y=tx)$, exprimer les coordonnées de $M$, $M'$ puis $M''$ en fonction de $t$.
% M(1/(1+t^2), t*x)   M'(1,t)  M'' (1-1/1+t^2, t*x)
\item En déduire une équation paramétrique de $\mathcal{D}$ :
 $$x(t) = \frac{t^2}{1+t^2}, \quad y(t) = \frac{t^3}{1+t^2}.$$

\item Montrer qu'une équation cartésienne de $\mathcal{D}$ est :
 $$x(x^2+y^2)-y^2 = 0.$$
 
\item \'Etudier et tracer $\mathcal{D}$.
 
\item Soit $P(0,2)$. La droite $(PI)$ coupe $\mathcal{D}$ en un point noté $Q$.
La droite $(OQ)$ coupe $\mathcal{L}$ en un point noté $R$. Calculer une équation de $(PI)$ ainsi que 
les coordonnées de $Q$ et $R$.
% R(1, \sqrt[3]{2})

\item Conclure.
\end{enumerate}

\myfigure{1}{
  \tikzinput{fig_compas_exo_diocles1}
  \hspace*{3cm}
  \tikzinput{fig_compas_exo_diocles2}  
}


\end{exercicecours}


\begin{exercicecours}[Lunules d'Hippocrate de Chios]
Montrer que l'aire des quatre lunules égale l'aire du carré. 

%\indication
\emph{Indication :} C'est un simple calcul d'aires : calculer l'aire totale formée 
par la figure de deux façons différentes.
%\finindication
%
\myfigure{1}{
  \tikzinput{fig_compas_exo_lunules}
}

%\correction
%
%Notons $\mathcal{A}$ l'aire totale formée par la figure.
%On décompose la figure totale de deux façon différente.
%
%\begin{center}
% \begin{tikzpicture}[scale=.6]
%      \useasboundingbox (0,-5) rectangle (10,5);
%      \def\maincircle{(0,0) circle (1.4142*2)}  
%      \fill[blue!50] (-2,2) rectangle (2,-2);
%      \fill[blue] (2,2) arc(0:180:2);     
%      \fill[blue] (-2,2) arc(90:270:2);
%      \fill[blue] (-2,-2) arc(180:360:2);     
%       \fill[blue] (2,2) arc(90:-90:2);   
%      
%      \draw (-2,2) rectangle (2,-2);
%      \draw \maincircle;
%      \draw (2,2) arc(0:180:2);
%      \draw (-2,2) arc(90:270:2);
%      \draw (-2,-2) arc(180:360:2);
%      \draw (2,2) arc(90:-90:2);     
% \end{tikzpicture} 
% \begin{tikzpicture}[scale=.6]
%      \useasboundingbox (0,-5) rectangle (0,5);
%      \def\maincircle{(0,0) circle (1.4142*2)}
%      
%      \begin{scope}
%        \begin{scope}[even odd rule]% first circle without the second
%            \clip \maincircle (-5,-5) rectangle (5,5);
%            \fill[green!60!black] (0,2) circle (2);
%            \fill[green!60!black] (0,-2) circle (2);
%            \fill[green!60!black] (2,0) circle (2);
%            \fill[green!60!black] (-2,0) circle (2);
%        \end{scope}
%      \end{scope}
%      
%      \fill[green!60] \maincircle;
%      \draw (-2,2) rectangle (2,-2);
%      \draw \maincircle;
%      \draw (2,2) arc(0:180:2);
%      \draw (-2,2) arc(90:270:2);
%      \draw (-2,-2) arc(180:360:2);
%      \draw (2,2) arc(90:-90:2);     
%   \end{tikzpicture}
%\end{center}
%
%Première façon, sur la figure de gauche.
%L'aire totale $\mathcal{A}$ se décompose en l'aire $\mathcal{A}_\text{carré}$ du carré
%(zone bleu clair, dont on veut calculer l'aire)
%et l'aire $\mathcal{A}_\text{demi-disques}$ formée par $4$ demi-disques (zone bleu foncé) :
%$$\mathcal{A} = \mathcal{A}_\text{carré} + \mathcal{A}_\text{demi-disques}$$
%Si on note $a$ la longueur d'un des côtés du carré alors 
%$$\mathcal{A}_\text{demi-disques} = 4 \times \frac12 \times \pi \left(\frac{a}{2}\right)^2 = \frac{\pi a^2}{2}$$
%
%
%\begin{center}
%   \begin{tikzpicture}[scale=0.5]
%      \def\maincircle{(0,0) circle (1.4142*2)} 
%      \begin{scope}
%        \begin{scope}[even odd rule]% first circle without the second
%            \clip \maincircle (-5,-5) rectangle (5,5);
%            \fill[orange] (0,2) circle (2);
%            \fill[orange] (0,-2) circle (2);
%            \fill[orange] (2,0) circle (2);
%            \fill[orange] (-2,0) circle (2);
%        \end{scope}
%      \end{scope}
%
%      \draw (-2,2) rectangle (2,-2);
%      \draw \maincircle;
%      \draw (2,2) arc(0:180:2);
%      \draw (-2,2) arc(90:270:2);
%      \draw (-2,-2) arc(180:360:2);
%      \draw (2,2) arc(90:-90:2);
%      
%      \draw[very thick, blue] (2,-2)--(2,2);
%      \node at (2,0) [left] {$a$};
%      \draw[very thick, blue] (0,0)--(45:2.82);
%      \node at (1.1,1) [left] {$r$};      
%   \end{tikzpicture}
%\end{center}
%
%Deuxième façon, sur la figure de droite.
%L'aire totale $\mathcal{A}$ se décompose cette fois en l'aire $\mathcal{A}_\text{disque}$ 
%du disque (zone vert clair)
%et l'aire $\mathcal{A}_\text{lunules}$ formée par $4$ lunules (zone vert foncé, dont 
%on veut aussi calculer l'aire) :
%$$\mathcal{A} = \mathcal{A}_\text{disque} + \mathcal{A}_\text{lunules}$$
%
%Le rayon du cercle est $r = \frac{\sqrt{2}}{2}a$.
%Ainsi 
%$$\mathcal{A}_\text{disque} = \pi r^2 = \pi \left(\frac{\sqrt{2}}{2}a\right)^2 = \frac{\pi a^2}{2}$$
%
%\begin{center}
%   \begin{tikzpicture}[scale=0.5]
%      \def\maincircle{(0,0) circle (1.4142*2)} 
%      \begin{scope}
%        \begin{scope}[even odd rule]% first circle without the second
%            \clip \maincircle (-5,-5) rectangle (5,5);
%            \fill[gray] (0,2) circle (2);
%            \fill[gray] (0,-2) circle (2);
%            \fill[gray] (2,0) circle (2);
%            \fill[gray] (-2,0) circle (2);
%        \end{scope}
%      \end{scope}
%
%      \draw (-2,2) rectangle (2,-2);
%      \draw \maincircle;
%      \draw (2,2) arc(0:180:2);
%      \draw (-2,2) arc(90:270:2);
%      \draw (-2,-2) arc(180:360:2);
%      \draw (2,2) arc(90:-90:2);
%      
%      \draw[very thick, blue] (2,-2)--(2,2);
%      \node at (2,0) [left] {$a$};
%      \draw[very thick, blue] (0,0)--(45:2.82);
%      \node at (1.1,1) [left] {$r$};      
%   \end{tikzpicture}
%\end{center}
%
%Conséquence $\mathcal{A}_\text{demi-disques} = \mathcal{A}_\text{disque}$
%donc
%$$\mathcal{A}_\text{lunules} 
%= \mathcal{A} - \mathcal{A}_\text{disque}
%= \mathcal{A} - \mathcal{A}_\text{demi-disques}
%= \mathcal{A}_\text{carré}$$
%L'aire des lunules égale l'aire du carrée ! C'est une variante résoluble de la quadrature du cercle.

\end{exercicecours}


\bigskip
\bigskip


Pour aller plus loin voici quelques références :
\begin{itemize}
 \item \emph{Théorie des corps. La règle et le compas.} 
 J.-L. Carrega, Hermann, 2001.

Un livre complet sur le sujet !


 \item \emph{Nombres constructibles.}
 V. Vassallo, Ph. Royer, IREM de Lille, 2002. 

Avec un point de vue pour le collège et le lycée.


 \item \emph{Sur les nombres algébriques constructibles à la règle et au compas.}
 A. Chambert-Loir, Gazette des mathématiciens 118, 2008.

Vous trouverez dans cet article une réciproque du corollaire au théorème de Wantzel,
prouvée de façon <<~élémentaire~>>, c'est-à-dire sans faire usage de la théorie de Galois.
\end{itemize}

%%%%%%%%%%%%%%%%%%%%%%%%%%%%%%%%%%%%%%%%%%%%%%%%%%%%%%%%%%%%%%%%
%%%%%%%%%%%%%%%%%%%%%%%%%%%%%%%%%%%%%%%%%%%%%%%%%%%%%%%%%%%%%%%%

\bigskip
\bigskip

\auteurs{
Arnaud Bodin. 

Relu par Vianney Combet.
}

\finchapitre
\end{document}

