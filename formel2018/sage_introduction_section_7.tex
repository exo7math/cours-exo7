\documentclass[class=report,crop=false]{standalone}
\usepackage[screen]{../exo7book}

% Commande ponctuelle
\newcommand{\alenvers}[1]{\rotatebox[origin=c]{180}{#1}}
\newcommand{\vect}{\overrightarrow}

\begin{document}

%====================================================================
\chapitre{Calcul formel}
%====================================================================

%\insertvideo{IPPCChtUNuU}{partie 7. Calculs d'intégrales}

%%%%%%%%%%%%%%%%%%%%%%%%%%%%%%%%%%%%%%%%%%%%%%%%%%%%%%%%%%%%%%%%
\setcounter{section}{6}
\section{Calculs d'intégrales}

%--------------------------------------------------------
\subsection{\Sage\ comme une super-calculatrice}



\begin{tp}
Calculer à la main les primitives des fonctions suivantes. 
Vérifier vos résultats à l'ordinateur.
\begin{enumerate}
  \item $f_1(x) = x^4 + \frac{1}{x^2} + \exp(x)+ \cos(x)$
  \item $f_2(x) = x \sin(x^2)$ 
  \item $f_3(x) = \frac{\alpha}{\sqrt{1-x^2}} + \frac{\beta}{1+x^2} + \frac{\gamma}{1+x}$
  \item $f_4(x) = \frac{x^4}{x^2-1}$ 
  \item $f_5(x) = x^n \ln x$ pour tout $n\ge0$
\end{enumerate}
\end{tp}

Pour une fonction $f$, par exemple \codeinline{f(x) = x*sin(x^2)} donnée, la commande
\codeinline{integral(f(x),x)} renvoie une primitive de $f$.
Le résultat obtenu ici est \codeinline{-1/2*cos(x^2)}.

Quelques remarques :
\begin{itemize}
  \item La machine renvoie une primitive. Pour avoir l'ensemble des primitives, il faut
  bien entendu ajouter une constante.

  \item La fonction \codeinline{log} est la fonction logarithme népérien usuel, habituellement
  notée $\ln$.
  
  \item \codeinline{integral(1/x,x)} renvoie \codeinline{log(x)} alors
  que $\int \frac{\dd x}{x} = \ln |x|$. \Sage\ omet les valeurs absolues, ce qui ne l'empêche pas de faire
  des calculs exacts même pour les valeurs négatives (en fait \Sage\ travaille avec le logarithme complexe).
  
  \item Souvent, pour intégrer des fractions rationnelles, on commence par écrire 
  la décomposition en éléments simples. C'est possible avec
  la commande \codeinline{partial_fraction}. Par exemple, pour
  \codeinline{f = x^4/(x^2-1)} la commande \codeinline{f.partial_fraction(x)} 
  renvoie la décomposition : %sur $\Qq$ : 
  $f(x) = x^2  + 1 - \frac12\frac{1}{x + 1} + \frac12\frac{1}{x - 1}$.
  
\end{itemize}



\begin{tp}
Calculer à la main et à l'ordinateur les intégrales suivantes.
\begin{enumerate}
  \item $\displaystyle I_1 = \int_1^3 x^2 + \sqrt{x} + \frac1x + \sqrt[3]{x} \; \dd x$
  \item $\displaystyle I_2 = \int_0^\frac{\pi}{2} \sin x (1+\cos x)^4 \; \dd x$
  \item $\displaystyle I_3 = \int_0^{\frac\pi6} \sin^3 x \; \dd x$
  \item $\displaystyle I_4 = \int_0^{\sqrt3} \frac{3x^3 - 2x^2 - 5x - 6}{(x + 1)^2(x^2 + 1)} \; \dd x$
\end{enumerate}
\end{tp}

Par exemple \codeinline{integral(sin(x)^3,x,0,pi/6)} renvoie 
$\frac23 - \frac{3\sqrt{3}}{8}$.



%--------------------------------------------------------
\subsection{Intégration par parties}

\begin{tp}
Pour $n$ entier positif ou nul, nous allons déterminer une primitive de la fonction 
$$f_n(x) = x^n \exp(x).$$

\begin{enumerate}
  \item \Sage\ sait-il répondre à cette question ?
  
  \item Calculer une primitive $F_n$ de $f_n$ pour les premières valeurs de $n$.
  
  \item 
  \begin{enumerate}
    \item \'Emettre une conjecture reliant $F_n$ et $F_{n-1}$.
    \item Prouver cette conjecture.
    \item En déduire une fonction récursive qui calcule $F_n$.
  \end{enumerate}
  
  \item 
  \begin{enumerate}
    \item \'Emettre une conjecture pour une formule directe de $F_n$.
    \item Prouver cette conjecture.
    \item En déduire une expression qui calcule $F_n$.
  \end{enumerate}  
\end{enumerate} 
\end{tp}

\begin{enumerate}
  \item La commande \codeinline{integral(x^n*exp(x),x)}
  renvoie le résultat \codeinline{(-1)^n*gamma(n + 1, -x)}. 
  Nous ne sommes pas tellement avancés ! \Sage\ renvoie une \emph{fonction spéciale} $\Gamma$ 
  que l'on ne connaît pas.  
  De plus, %(à l'heure actuelle, pour la version 7.6), 
  dériver cette primitive avec \Sage\ (pour la version courante) ne 
  redonne pas la même expression que $f_n$.
  
  \item Par contre, il n'y a aucun problème pour calculer les primitives $F_n$ 
  pour les premières valeurs de $n$. On obtient :
  $$\begin{array}{rcl}
    F_0(x) &=& \exp(x) \\
    F_1(x) &=& (x - 1)\exp(x) \\
    F_2(x) &=& (x^2 - 2x + 2)\exp(x) \\
    F_3(x) &=& (x^3 - 3x^2 + 6x - 6)\exp(x) \\
    F_4(x) &=& (x^4 - 4x^3 + 12x^2 - 24x + 24)\exp(x) \\
    F_5(x) &=& (x^5 - 5x^4 + 20x^3 - 60x^2 + 120x - 120)\exp(x) \\
  \end{array}$$  
   
    
  \item
  \begin{enumerate}
    \item Au vu des premiers termes, on conjecture $F_n(x) = x^n \exp(x) - nF_{n-1}(x)$.
    On vérifie facilement sur les premiers termes que cette conjecture est vraie.
    
     \Sage\ ne sait pas (encore) vérifier formellement cette identité 
     (avec $n$ une variable formelle). La commande suivante échoue à trouver $0$  : \\
    % Les versions récentes de \Sage\ peuvent vérifier formellement cette identité : la commande suivante retourne $0$ : \\
    \centerline{\codeinline{integral(x^n*exp(x),x) - x^n*exp(x) + n*integral(x^(n-1)*exp(x),x)}}
    
    \item Nous prouverons la validité de cette formule en faisant 
    une intégration par parties
    (on pose par exemple $u=x^n$, $v'=\exp(x)$ et donc $u'=nx^{n-1}$, $v = \exp(x)$) :
    
    $$F_n(x) 
    = \int x^n \exp(x)\; \dd x 
    = \big[x^n\exp(x)\big] - \int nx^{n-1}\exp(x)\; \dd x
    = x^n\exp(x) - n F_{n-1}(x).$$
    
    \item Voici l'algorithme récursif pour le calcul de $F_n$ utilisant 
    la relation trouvée ci-dessus.
    
    \insertcode{algos/integrales-ipp-tex1.sage}{integrales-ipp (1)}
    
  \end{enumerate} 
  
  \item Avec un peu de réflexion, on conjecture la formule 
  $$F_n(x) = \exp x \sum_{k=0}^n (-1)^{n-k} \frac{n!}{k!} x^k$$
  où $\frac{n!}{k!} =  n(n-1)(n-2)\cdots (k+1)$.
  
  Pour la preuve, on pose $G_n(x) = \exp(x)\sum_{k=0}^n (-1)^{n-k} \frac{n!}{k!} x^k$.
  On a 
  $$G_n(x) = \exp(x)\sum_{k=0}^{n-1} (-1)^{n-k} \frac{n!}{k!} x^k  + \exp(x)x^n
  = -n G_{n-1}(x)+ \exp(x)x^n.$$ 
  Ainsi 
  $G_n$ vérifie la même relation de récurrence que $F_n$.
  De plus $G_0(x)=\exp x = F_0(x)$. Donc pour tout $n$, $G_n=F_n$.
  
  On peut donc ainsi calculer directement notre intégrale par la formule : \\  
  \centerline{\codeinline{exp(x)*sum((-1)^(n-k)*x^k*factorial(n)/factorial(k),k,0,n)}}
 
\end{enumerate}



%--------------------------------------------------------
\subsection{Changement de variable}

Soit $f$ une fonction définie sur un intervalle $I$ et $\varphi : J \to I$ 
une bijection de classe $\mathcal{C}^1$.
La formule de changement de variable pour les primitives est :
$$\int f(x) \; \dd x = \int f\big(\varphi(u)\big)\cdot\varphi'(u) \; \dd u.$$

La formule de changement de variable pour les intégrales, pour tout $a,b\in I$, est :
$$\int_a^b f(x) \; \dd x 
= \int_{\varphi^{-1}(a)}^{\varphi^{-1}(b)} f\big(\varphi(u)\big)\cdot\varphi'(u) \; \dd u.$$

\begin{tp}
\sauteligne
\begin{enumerate}
  \item Calcul de la primitive $\displaystyle\int \sqrt{1-x^2} \; \dd x$.
  \begin{enumerate}
    \item Poser $f(x) = \sqrt{1-x^2}$ et le changement de variable $x = \sin u$, c'est-à-dire poser 
    $\varphi(u) = \sin u$.
    \item Calculer $f\big( \varphi(u) \big)$ et $\varphi'(u)$.
    \item Calculer la primitive de $g(u) = f\big(\varphi(u)\big)\cdot\varphi'(u)$.
    \item En déduire une primitive de $f(x) = \sqrt{1-x^2}$.
  \end{enumerate}
  
  \item Calcul de la primitive $\displaystyle\int \frac{\dd x}{1+\left(\frac{x+1}{x}\right)^{1/3}}$.
  \begin{enumerate}  
    \item Poser le changement de variable défini par l'équation $u^3 = \frac{x+1}{x}$.
    
    \item En déduire le changement de variable $\varphi(u)$ et $\varphi^{-1}(x)$.
  \end{enumerate}
  
  
  \item \'Ecrire une fonction \codeinline{integrale_chgtvar(f,eqn)} qui calcule
  une primitive de $f$ par le changement de variable défini par l'équation 
  \codeinline{eqn} reliant $u$ et $x$.
  
  En déduire $\displaystyle\int (\cos x + 1)^n \cdot \sin x \; \dd x$ (pour $n>0$) en posant $u= \cos x + 1$.
  
  \item Même travail avec \codeinline{integrale_chgtvar_bornes(f,a,b,eqn)}
  qui calcule l'intégrale de $f$ entre $a$ et $b$ par changement de variable.
  
  En déduire $\displaystyle\int_0^{\frac\pi4} \frac{\dd x}{2+\sin^2 x}$ en posant $u = \tan x$.
\end{enumerate}
\end{tp}


\begin{enumerate}
  \item 
  \begin{enumerate}
    \item La fonction est $f(x) = \sqrt{1-x^2}$ et 
    on pose $\varphi(u) = \sin u$, c'est-à-dire que l'on 
    espère que le changement de variable $x = \sin u$ va simplifier le calcul de l'intégrale :
    
    \centerline{\codeinline{f = sqrt(1-x^2)} \qquad \codeinline{phi = sin(u)}}
    
    \item 
    On obtient $f\big( \varphi(u) \big)$ en substituant la variable $x$ par $\sin u$ :
    cela donne $f\big( \varphi(u) \big) = \sqrt{1-\sin^2 u} = \sqrt{\cos^2 u} = |\cos u|$.
    Ce qui s'obtient par la commande \codeinline{frondphi = f(x=phi)} (puis en simplifiant).    
    Noter que \Sage\ \og oublie \fg{} les valeurs absolues, car il calcule en fait la 
    racine carrée complexe et non réelle.
    Ce sera heureusement sans conséquence pour la suite.
    En effet pour que $f(x)$ soit définie, il faut $x \in [-1,1]$, comme $x = \sin u$
    alors $u \in [-\frac\pi2,\frac\pi2]$ et donc $\cos u \ge 0$.    
    Enfin $\varphi'(u) = \cos u$ s'obtient par \codeinline{dphi = diff(phi,u)}.
    
    \item On pose $g(u) = f\big(\varphi(u)\big)\cdot\varphi'(u) = \cos^2 u = \frac{1+\cos(2u)}{2}$.
    Donc $\int g(u)\; \dd u = \frac{u}{2} + \frac{\sin(2u)}{4}$.
    
    \item Pour obtenir une primitive de $f$, on utilise la formule de changement de variable :
    $$\int \sqrt{1-x^2} \; \dd x = \int f(x)\; \dd x = \int g(u) \; \dd u =  \frac{u}{2} + \frac{\sin(2u)}{4}.$$
    Mais il ne faut pas oublier de revenir à la variable $x$. 
    Comme $x = \sin u$ alors $u = \arcsin x$, donc 
    $$\int \sqrt{1-x^2} \; \dd x = \frac{\arcsin x}{2} + \frac{\sin(2\arcsin x)}{4}.$$
    
    Les commandes sont donc \codeinline{G = integral(g,u)} puis
    \codeinline{F = G(u = arcsin(x))} pour obtenir la primitive recherchée.
    Après simplification de $\sin(2\arcsin x)$, on obtient :
    $$\int \sqrt{1-x^2} \; \dd x = \frac{\arcsin x}{2} + \frac x2 \sqrt{1-x^2}.$$
    
  \end{enumerate}
  
  \item Pour calculer la primitive $\int \frac{\dd x}{1+\left(\frac{x+1}{x}\right)^{1/3}}$, l'énoncé
    nous recommande le changement de variable $u^3 = \frac{x+1}{x}$.
  La seule difficulté supplémentaire est qu'il faut exprimer $x$ en fonction de $u$
  et aussi $u$ en fonction de $x$.
  Pour cela, on définit l'équation \codeinline{u^3 == (x+1)/x} reliant $u$ et $x$ : 
  \codeinline{eqn = u^3 == (x+1)/x}. On peut maintenant résoudre l'équation afin d'obtenir
  $\varphi(u)$ ($x$ en fonction de $u$) :  \codeinline{phi = solve(eqn,x)[0].rhs()} et $\varphi^{-1}(x)$ ($u$ en fonction de $x$) : \codeinline{phi_inv = solve(eqn,u)[0].rhs()}.
  Le reste se déroule comme précédemment.
 
  
  \item On automatise la méthode précédente :
  \insertcode{algos/integrales-chgtvar-tex1.sage}{integrales-chgtvar (1)}
  Ce qui s'utilise ainsi :
  \insertcode{algos/integrales-chgtvar-tex3.sage}{integrales-chgtvar (2)} 
  et montre que 
  $$\int (\cos x + 1)^n \cdot \sin x \; \dd x =  -\frac{1}{n + 1}(\cos x + 1)^{n + 1}.$$
 
  \item Pour la formule de changement de variable des intégrales, on adapte la procédure précédente
  en calculant l'intégrale $\int_{\varphi^{-1}(a)}^{\varphi^{-1}(b)} g(u) \; \dd u$.
  
  Pour cela, on calcule 
  $\varphi^{-1}(a)$ par  \codeinline{a_inv = phi_inv(x=a)} et 
  $\varphi^{-1}(b)$ par  \codeinline{b_inv = phi_inv(x=b)}.
 
  Pour calculer $\int_0^{\frac\pi4} \frac{\dd x}{2+\sin^2 x}$,
  on pose  $u = \tan x$.
  Donc $x = \varphi(u) = \arctan u$ et $u = \varphi^{-1}(x) = \tan x$.
  Nous avons $\varphi'(u)= \frac{1}{1+u^2}$. D'autre part,
  comme $a=0$ on a $\varphi^{-1}(a) = \tan 0 = 0$
  et comme $b = \frac\pi4$ on a $\varphi^{-1}(b) = \tan \frac\pi4 = 1$.
  Ainsi la formule de changement de variable :
  $$\int_a^b f(x) \;\dd x =
  \int_{\varphi^{-1}(a)}^{\varphi^{-1}(b)} f\big(\varphi(u)\big)\cdot\varphi'(u) \; \dd u$$
  devient 
  $$I = \int_0^{\frac\pi4} \frac{1}{2+\sin^2 x} \; \dd x = 
  \int_0^1 \frac{1}{2+\sin^2 (\arctan u)} \frac{1}{1+u^2}  \; \dd u.$$
  Mais $\sin^2(\arctan u ) = \frac{u^2}{1+u^2}$ donc
  $$I = \int_0^1 \frac{1}{2+\frac{u^2}{1+u^2}} \frac{1}{1+u^2}  \; \dd u
  = \int_0^1 \frac{\dd u}{2+3u^2}  \; \dd u = \left[\frac{1}{\sqrt6}\arctan\left(\frac{\sqrt6 u}{2}\right)\right]_0^1
  = \frac{1}{\sqrt6}\arctan\left(\frac{\sqrt6}{2}\right).$$
  Ce que l'ordinateur confirme !
\end{enumerate}

\finchapitre

\end{document}

