
%%%%%%%%%%%%%%%%%% PREAMBULE %%%%%%%%%%%%%%%%%%


\documentclass[12pt]{article}

\usepackage{amsfonts,amsmath,amssymb,amsthm}
\usepackage[utf8]{inputenc}
\usepackage[T1]{fontenc}
\usepackage[francais]{babel}


% packages
\usepackage{amsfonts,amsmath,amssymb,amsthm}
\usepackage[utf8]{inputenc}
\usepackage[T1]{fontenc}
%\usepackage{lmodern}

\usepackage[francais]{babel}
\usepackage{fancybox}
\usepackage{graphicx}

\usepackage{float}

%\usepackage[usenames, x11names]{xcolor}
\usepackage{tikz}
\usepackage{datetime}

\usepackage{mathptmx}
%\usepackage{fouriernc}
%\usepackage{newcent}
\usepackage[mathcal,mathbf]{euler}

%\usepackage{palatino}
%\usepackage{newcent}


% Commande spéciale prompteur

%\usepackage{mathptmx}
%\usepackage[mathcal,mathbf]{euler}
%\usepackage{mathpple,multido}

\usepackage[a4paper]{geometry}
\geometry{top=2cm, bottom=2cm, left=1cm, right=1cm, marginparsep=1cm}

\newcommand{\change}{{\color{red}\rule{\textwidth}{1mm}\\}}

\newcounter{mydiapo}

\newcommand{\diapo}{\newpage
\hfill {\normalsize  Diapo \themydiapo \quad \texttt{[\jobname]}} \\
\stepcounter{mydiapo}}


%%%%%%% COULEURS %%%%%%%%%%

% Pour blanc sur noir :
%\pagecolor[rgb]{0.5,0.5,0.5}
% \pagecolor[rgb]{0,0,0}
% \color[rgb]{1,1,1}



%\DeclareFixedFont{\myfont}{U}{cmss}{bx}{n}{18pt}
\newcommand{\debuttexte}{
%%%%%%%%%%%%% FONTES %%%%%%%%%%%%%
\renewcommand{\baselinestretch}{1.5}
\usefont{U}{cmss}{bx}{n}
\bfseries

% Taille normale : commenter le reste !
%Taille Arnaud
%\fontsize{19}{19}\selectfont

% Taille Barbara
%\fontsize{21}{22}\selectfont

%Taille François
\fontsize{25}{30}\selectfont

%Taille Pascal
%\fontsize{25}{30}\selectfont

%Taille Laura
%\fontsize{30}{35}\selectfont


%\myfont
%\usefont{U}{cmss}{bx}{n}

%\Huge
%\addtolength{\parskip}{\baselineskip}
}


% \usepackage{hyperref}
% \hypersetup{colorlinks=true, linkcolor=blue, urlcolor=blue,
% pdftitle={Exo7 - Exercices de mathématiques}, pdfauthor={Exo7}}


%section
% \usepackage{sectsty}
% \allsectionsfont{\bf}
%\sectionfont{\color{Tomato3}\upshape\selectfont}
%\subsectionfont{\color{Tomato4}\upshape\selectfont}

%----- Ensembles : entiers, reels, complexes -----
\newcommand{\Nn}{\mathbb{N}} \newcommand{\N}{\mathbb{N}}
\newcommand{\Zz}{\mathbb{Z}} \newcommand{\Z}{\mathbb{Z}}
\newcommand{\Qq}{\mathbb{Q}} \newcommand{\Q}{\mathbb{Q}}
\newcommand{\Rr}{\mathbb{R}} \newcommand{\R}{\mathbb{R}}
\newcommand{\Cc}{\mathbb{C}} 
\newcommand{\Kk}{\mathbb{K}} \newcommand{\K}{\mathbb{K}}

%----- Modifications de symboles -----
\renewcommand{\epsilon}{\varepsilon}
\renewcommand{\Re}{\mathop{\text{Re}}\nolimits}
\renewcommand{\Im}{\mathop{\text{Im}}\nolimits}
%\newcommand{\llbracket}{\left[\kern-0.15em\left[}
%\newcommand{\rrbracket}{\right]\kern-0.15em\right]}

\renewcommand{\ge}{\geqslant}
\renewcommand{\geq}{\geqslant}
\renewcommand{\le}{\leqslant}
\renewcommand{\leq}{\leqslant}

%----- Fonctions usuelles -----
\newcommand{\ch}{\mathop{\mathrm{ch}}\nolimits}
\newcommand{\sh}{\mathop{\mathrm{sh}}\nolimits}
\renewcommand{\tanh}{\mathop{\mathrm{th}}\nolimits}
\newcommand{\cotan}{\mathop{\mathrm{cotan}}\nolimits}
\newcommand{\Arcsin}{\mathop{\mathrm{Arcsin}}\nolimits}
\newcommand{\Arccos}{\mathop{\mathrm{Arccos}}\nolimits}
\newcommand{\Arctan}{\mathop{\mathrm{Arctan}}\nolimits}
\newcommand{\Argsh}{\mathop{\mathrm{Argsh}}\nolimits}
\newcommand{\Argch}{\mathop{\mathrm{Argch}}\nolimits}
\newcommand{\Argth}{\mathop{\mathrm{Argth}}\nolimits}
\newcommand{\pgcd}{\mathop{\mathrm{pgcd}}\nolimits} 

\newcommand{\Card}{\mathop{\text{Card}}\nolimits}
\newcommand{\Ker}{\mathop{\text{Ker}}\nolimits}
\newcommand{\id}{\mathop{\text{id}}\nolimits}
\newcommand{\ii}{\mathrm{i}}
\newcommand{\dd}{\mathrm{d}}
\newcommand{\Vect}{\mathop{\text{Vect}}\nolimits}
\newcommand{\Mat}{\mathop{\mathrm{Mat}}\nolimits}
\newcommand{\rg}{\mathop{\text{rg}}\nolimits}
\newcommand{\tr}{\mathop{\text{tr}}\nolimits}
\newcommand{\ppcm}{\mathop{\text{ppcm}}\nolimits}

%----- Structure des exercices ------

\newtheoremstyle{styleexo}% name
{2ex}% Space above
{3ex}% Space below
{}% Body font
{}% Indent amount 1
{\bfseries} % Theorem head font
{}% Punctuation after theorem head
{\newline}% Space after theorem head 2
{}% Theorem head spec (can be left empty, meaning ‘normal’)

%\theoremstyle{styleexo}
\newtheorem{exo}{Exercice}
\newtheorem{ind}{Indications}
\newtheorem{cor}{Correction}


\newcommand{\exercice}[1]{} \newcommand{\finexercice}{}
%\newcommand{\exercice}[1]{{\tiny\texttt{#1}}\vspace{-2ex}} % pour afficher le numero absolu, l'auteur...
\newcommand{\enonce}{\begin{exo}} \newcommand{\finenonce}{\end{exo}}
\newcommand{\indication}{\begin{ind}} \newcommand{\finindication}{\end{ind}}
\newcommand{\correction}{\begin{cor}} \newcommand{\fincorrection}{\end{cor}}

\newcommand{\noindication}{\stepcounter{ind}}
\newcommand{\nocorrection}{\stepcounter{cor}}

\newcommand{\fiche}[1]{} \newcommand{\finfiche}{}
\newcommand{\titre}[1]{\centerline{\large \bf #1}}
\newcommand{\addcommand}[1]{}
\newcommand{\video}[1]{}

% Marge
\newcommand{\mymargin}[1]{\marginpar{{\small #1}}}



%----- Presentation ------
\setlength{\parindent}{0cm}

%\newcommand{\ExoSept}{\href{http://exo7.emath.fr}{\textbf{\textsf{Exo7}}}}

\definecolor{myred}{rgb}{0.93,0.26,0}
\definecolor{myorange}{rgb}{0.97,0.58,0}
\definecolor{myyellow}{rgb}{1,0.86,0}

\newcommand{\LogoExoSept}[1]{  % input : echelle
{\usefont{U}{cmss}{bx}{n}
\begin{tikzpicture}[scale=0.1*#1,transform shape]
  \fill[color=myorange] (0,0)--(4,0)--(4,-4)--(0,-4)--cycle;
  \fill[color=myred] (0,0)--(0,3)--(-3,3)--(-3,0)--cycle;
  \fill[color=myyellow] (4,0)--(7,4)--(3,7)--(0,3)--cycle;
  \node[scale=5] at (3.5,3.5) {Exo7};
\end{tikzpicture}}
}



\theoremstyle{definition}
%\newtheorem{proposition}{Proposition}
%\newtheorem{exemple}{Exemple}
%\newtheorem{theoreme}{Théorème}
\newtheorem{lemme}{Lemme}
\newtheorem{corollaire}{Corollaire}
%\newtheorem*{remarque*}{Remarque}
%\newtheorem*{miniexercice}{Mini-exercices}
%\newtheorem{definition}{Définition}




%definition d'un terme
\newcommand{\defi}[1]{{\color{myorange}\textbf{\emph{#1}}}}
\newcommand{\evidence}[1]{{\color{blue}\textbf{\emph{#1}}}}



 %----- Commandes divers ------

\newcommand{\codeinline}[1]{\texttt{#1}}

%%%%%%%%%%%%%%%%%%%%%%%%%%%%%%%%%%%%%%%%%%%%%%%%%%%%%%%%%%%%%
%%%%%%%%%%%%%%%%%%%%%%%%%%%%%%%%%%%%%%%%%%%%%%%%%%%%%%%%%%%%%


\begin{document}

\debuttexte


%%%%%%%%%%%%%%%%%%%%%%%%%%%%%%%%%%%%%%%%%%%%%%%%%%%%%%%%%%%
\diapo

Bonjour,

Dans cette séquence, nous allons voir comment utiliser 
le logiciel Sage pour visualiser et étudier des suites récurrentes.

\change
Voici les étapes de notre étude.

\change
Nous allons commencer par expliquer comment représenter et visualiser une telle suite,

\change
nous en profiterons pour préciser la notion informatique de liste, 

\change
puis nous conclurons par l'étude d'une suite chaotique.



%%%%%%%%%%%%%%%%%%%%%%%%%%%%%%%%%%%%%%%%%%%%%%%%%%%%%%%%%%%
\diapo

Une suite récurrente est une suite définie par son premier terme et par une relation du type  $u_{n+1} = f(u_n)$, 
où $f$ est une fonction. 

Nous allons nous intéresser ici au cas particulier suivant : la fonction $f$ est définie par $f(x)=\exp(-x)$

et le premier terme constant vaut $a$.

\change

Dans un premier temps, on demande de calculer 
 les premiers termes de la suite lorsque le terme initial vaut $-1$, 
 et d'en constituer une liste.

\change

Le but de la deuxième question est de tracer le graphe de la fonction $\exp(-x)$, 
celui de la première bissectrice, et la ligne brisée qui représente les valeurs de la suite récurrente 
comme vous l'avez certainement déjà tracée à la main.

\change

Ensuite, on demande, à l'aide du graphique de la question précédente, 
d'émettre des conjectures quant au comportement et à la nature de la suite,

\change

et de les prouver.


%%%%%%%%%%%%%%%%%%%%%%%%%%%%%%%%%%%%%%%%%%%%%%%%%%%%%%%%%%%
\diapo

Voici le résultat que l'on cherche à obtenir.

On trace le graphe de la fonction $f$ et la première bissectrice $(y=x)$.

On part de la valeur $u_0$ (ici $-1$) sur l'axe des abscisses, 
le point de coordonnées $(u_0,f(u_0))$ est le point du graphe de $f$ situé à la verticale de notre premier point.

La valeur $u_1=f(u_0)$ se lit alors 
sur l'axe des ordonnées, mais on transforme cette ordonnée 
 en une abscisse à l'aide de la première bissectrice.

On recommence : $u_2=f(u_1)$ se lit sur l'axe des ordonnées et on le reporte sur l'axe des abscisses, etc.
On obtient ici une sorte d'escargot.


Graphiquement, on conjecture que 
la sous-suite des termes de rang pair est croissante et converge
alors que la sous-suite des termes de rang impair est décroissante et 
converge vers la même limite.


Cette limite à l'air d'être le réel $\ell$ vérifiant $f(\ell)=\ell$.




%%%%%%%%%%%%%%%%%%%%%%%%%%%%%%%%%%%%%%%%%%%%%%%%%%%%%%%%%%%
\diapo

Voyons comment obtenir ce graphe.

On définit la fonction $f$ par la commande \codeinline{f = exp(-x)}.

Voici le code définissant la liste des termes de la suite récurrente.

\change

C'est une fonction \codeinline{liste\_suite} en trois variables : la fonction, le terme intial et le nombre de termes à calculer.

On initialise \codeinline{maliste} (la liste des valeurs de la suite) comme la liste vide,

on introduit une autre variable $x$, qui va jouer le rôle du terme $u_k$ de la suite, 
initialisée à la valeur donnée par \codeinline{terme\_init},

puis une boucle sur \codeinline{n} termes qui 

ajoute \codeinline{x} à la liste 

et remplace \codeinline{x} par son image par $f$ : \codeinline{f(x)}.

Lorsque la boucle est terminée après \codeinline{n} itérations, la fonction renvoie la liste qui contient $n$ termes de la suite.

\change

Par exemple la commande
\codeinline{liste\_suite(f,-1,4)}
calcule, pour le terme initial $a=-1$, les $4$ premiers termes de la suite $(u_n)$ ;
on obtient la liste 
correspondant aux termes : 

$u_0=-1$, $u_1 = e$, $u_2 = e^{-e}$ et $u_3 = e^{-e^{-e}}$, 

où l'on note $e$ la valeur de la finction exponentielle en $1$.

%%%%%%%%%%%%%%%%%%%%%%%%%%%%%%%%%%%%%%%%%%%%%%%%%%%%%%%%%%%
\diapo

Nous allons maintenant calculer la liste des points de la ligne brisée demandée.

C'est une nouvelle fonction \codeinline{liste\_points} en les mêmes variables que la fonction \codeinline{liste\_suite}.

On définit $u$ comme la liste des valeurs de la suite définie par $f$ à l'aide de la fonctions précédente : \codeinline{liste\_suite}.

Le premier point de la liste est $(u_0,0)$,

puis de nouveau à l'aide d'une boucle et pour chaque rang, on calcule deux points :
$\big(u_k,f(u_k)\big)$ qui est un point du graphe de $f$ et $\big(u_{k+1},u_{k+1}\big)$ le point de la première bissectrice de même ordonnée,

et on les ajoute à la liste.

Notez que chaque élément de la liste est ici un couple $(x,y)$ de
coordonnées.


\change
Par exemple la commande \codeinline{liste\_points(f,-1,3)}
calcule, pour le terme initial $a=-1$, le point de départ $(-1,0)$ et les $3$ premiers 
points de la visualisation de la suite $(u_n)$ ;
on obtient la liste :

$$(-1, 0), (-1, e), (e, e), (e, e^(-e)), (e^(-e), e^(-e)) $$



%%%%%%%%%%%%%%%%%%%%%%%%%%%%%%%%%%%%%%%%%%%%%%%%%%%%%%%%%%%
\diapo

Nous avons tous les outils nécessaires pour obtenir le graphe demandé.

\change

Ceci est fait grâce à une nouvelle fonction 

\codeinline{dessine\_suite} toujours des trois mêmes variables, qui utilise la fonction \codeinline{liste\_points}.

Au graphe $G$ de la fonction $f$, 

on ajoute successivement le graphe de la première bissectrice, 

puis la ligne brisée joignant les points de la liste que nous venons de construire.

Il ne reste plus qu'à demander l'affichage du graphe.

\change

Par exemple, pour obtenir la figure présentée en debut de séquence, il suffit de faire appel à
\codeinline{dessine\_suite(f,-1,10)}.



%%%%%%%%%%%%%%%%%%%%%%%%%%%%%%%%%%%%%%%%%%%%%%%%%%%%%%%%%%%
\diapo

Une liste est, pour Sage, l'analogue de ce qu'on appellerait en mathématique une suite finie. 
Une liste peut-être constituée de n'importe quel type d'objets.

\change

La syntaxe de Sage veut qu'une liste soit présentée entre crochets.

\change

On accède au $i$-ème terme de la suite en faisant suivre son nom de $[i]$.

Attention, le premier terme est d'indice $0$.

Par exemple pour la liste  

\codeinline{mesprems = [2,3,5,7,11,13]},

\codeinline{mesprems[0]} vaut $2$,

\codeinline{mesprems[1]} vaut $3$...

\change 

Il est facile de parcourir une liste à l'aide une boucle. Par exemple 
\codeinline{for p in mesprems:}. 

%Tous les éléments de la liste sont atteints.
%La boucle se fait alors sur tous les éléments de la liste.


\change

Une façon de créer une liste est de partir de la liste vide, 
qui s'écrit \codeinline{[]},

\change
puis d'ajouter des éléments à l'aide de la méthode \codeinline{append}.

\change

 Par exemple 

à ce stade la liste \codeinline{mespoints} est vide, (\codeinline{mespoints=[]})

puis  l'élément (2,3) est ajouté avec la méthode \codeinline{append} (\codeinline{mespoints.append((2,3))})

puis un autre %  \codeinline{mespoints.append((7,-1))}. 


La liste \codeinline{mespoints} contient maintenant deux
  éléments : \codeinline{ (2,3), (7,-1)}.


%%%%%%%%%%%%%%%%%%%%%%%%%%%%%%%%%%%%%%%%%%%%%%%%%%%%%%%%%%%
\diapo

Ces petits énoncés présentent quelques exemples de constructions de listes.

Une utilisation intelligente du manuel vous permettra de produire un code très court et très lisible.
%Il faut potasser le manuel afin de manipuler les listes avec aisance.

PAUSE 3s

\change

La commande \codeinline{range} permet d'afficher les $n$ premiers entiers.

\change

On peut sélectionner les éléments d'une liste grâce à une condition que l'on introduit par un test \codeinline{if}.

Cet exemple fournit la liste des premiers nombres premiers. 


\change

On peut se servir d'une liste déjà définie, ici \codeinline{premiers}, pour en définir une autre.

\change

On peut tronquer une liste ou en extraire une sous-liste. 

Ici on ne garde que les $10$ premiers éléments.

\change

Voici la liste des nombres premiers auxquels on a ajouté leur rang.

\change

Il est également possible d'utiliser une fonction auxiliaire pour définir une liste. 

Ici \codeinline{zeroun(k)} est une fonction qui renvoie $1$ où $0$ selon que $k$ est premier ou pas. 

~

On peut également fusionner des listes, 
et faire de nombreuses autres opérations.

%%%%%%%%%%%%%%%%%%%%%%%%%%%%%%%%%%%%%%%%%%%%%%%%%%%%%%%%%%%
\diapo

Nous allons, pour le reste de cette séquence, considérer l'ensemble des fonctions $f$ définies 
%Dans ce tp, qui va nous occuper pour le reste de cette séquence, nous nous intéressons à
 sur $x \in [0,1]$ par 
$$f(x)=rx(1-x)$$  où le réel $r$, un paramètre, est compris entre $0$ et $4$.
%$ 0 \le  r \le 4$.

\change

Nous allons nous intéresser aux suites récurrentes de terme initial $u_0 \in [0,1]$ et définies par $u_{n+1} = f(u_n)$ pour $r$ fixé.

\change

Dans cette première question, on demande de visualiser les suites récurrentes pour différentes valeurs de $r$ et $u_0$. 

Vous pouvez pour cela suivre le procédé de construction que nous avons déjà élaboré.
%, en s'aidant du graphe de $f$ et de la diagonale principale, en suivant le procédé de construction que nous avons déjà décrit.

%%%%%%%%%%%%%%%%%%%%%%%%%%%%%%%%%%%%%%%%%%%%%%%%%%%%%%%%%%%
\diapo
Voici le comportement de différentes suites définies par le même premier terme $u_0=0,123$ et pour différentes valeurs de $r$.


Pour $r=2,2$, la suite semble être monotone et avoir une limite,

\change

Pour $r=3,2$, la suite semble avoir \og deux limites \fg\ , c'est-à-dire
  deux valeurs d'adhérence,

  \change
Pour $r=3,57$, la suite semble avoir $4$ valeurs d'adhérence.

 \change
 Pour  $r=4$ la situation semble chaotique.

%%%%%%%%%%%%%%%%%%%%%%%%%%%%%%%%%%%%%%%%%%%%%%%%%%%%%%%%%%%
\diapo

Dans cette deuxième question, et les suivantes, on conserve les mêmes notations pour la fonction $f$ et la suite $u_n$.

\change

On cherche à tracer, à l'aide de Sage, pour $u_0$ fixé et deux entiers $M$ et $N$ grands avec $M<N$ (par exemple $M=100$, $N=200$)

le \defi{diagramme de bifurcation} de $f$, c'est-à-dire l'ensemble des points
  $$(u_i,r) \qquad \text{ pour } \quad M \le i \le N \text{ et } 0 \le r \le 4$$

%%%%%%%%%%%%%%%%%%%%%%%%%%%%%%%%%%%%%%%%%%%%%%%%%%%%%%%%%%%
\diapo

Voici le résultat attendu.

En abscisse les réels de l'intervalle $[0,1]$  correspondant aux "limites" de la suites
et en ordonnée le paramètre $r$.


Lorsque l'on fixe une valeur de $r$, c'est-à-dire lorsqu'on regarde 
l'intersection de ce diagramme de bifurcation avec une droite horizontale.
alors cela donne une idée du nombre de valeurs d'adhérence de la suite 
construite à partir de la fonction $f$ correspondant à la valeur du paramètre $r$.

Pour $r\leq 3$ par exemple, on peut supposer que la suite n'admet qu'une seule valeur d'adhérence et donc qu'elle est convergente.

Dès que $r$ dépasse $3$, on constate expérimentalement que la suite admet deux valeurs d'adhérence, puis $4$,... et ensuite le comportement devient chaotique.


%%%%%%%%%%%%%%%%%%%%%%%%%%%%%%%%%%%%%%%%%%%%%%%%%%%%%%%%%%%
\diapo

Dans la troisième question, on demande de 

déterminer les points fixes de $f$, et d'étudier leurs caractères attractif ou répulsif.

On rappelle qu'un point fixe $x$ est attractif si $|f'(x)|<1$ et répulsif si $|f'(x)|>1$.

%%%%%%%%%%%%%%%%%%%%%%%%%%%%%%%%%%%%%%%%%%%%%%%%%%%%%%%%%%%
\diapo

Voici le code répondant à cette question. 

On commence par définir les variables, la fonction, puis on résout l'équation donnant les points fixes.

\change

\begin{enumerate}
    \item Les deux points fixes fournis dans la liste \codeinline{pts\_fixes}
    sont $0$ et $\frac{r-1}{r}$.

    \change 
    
    \item On calcule la dérivée \codeinline{ff}, 

    on l'évalue en $0$, on trouve $f'(0)=r$. Ainsi si $r>1$ alors
    $|f'(0)|>1$ et le point $0$ est répulsif. 
    
    \change

    \item Par contre lorsque l'on résout l'inéquation $|f'(\frac{r-1}{r})|<1$, 
    la machine renvoie les conditions $r>1$ et $r < 3$. 
    Ainsi pour $1<r<3$, le point fixe $\frac{r-1}{r}$ est attractif.
    \end{enumerate}


%%%%%%%%%%%%%%%%%%%%%%%%%%%%%%%%%%%%%%%%%%%%%%%%%%%%%%%%%%%
\diapo

Dans la quatrième question, on étudie le cas où $1 < r \le3$. 

Nous allons prouver mathématiquement que pour un terme initial $0<u_0<1$ 
la suite $(u_n)$ converge vers le point fixe de la fonction $\frac{r-1}{r}$. 

Nous nous limiterons au cas $r=2$.


%%%%%%%%%%%%%%%%%%%%%%%%%%%%%%%%%%%%%%%%%%%%%%%%%%%%%%%%%%%
\diapo

Nous n'allons par rentrer dans le détail de la solution mathématique, 
mais nous allons seulement illustrer la situation en traçant la 
ligne brisée associée à la suite à l'aide de la fonction 
\codeinline{dessine\_suite} que nous avons déjà définie. 

Nous constatons le phénomène de convergence vers le 
point fixe $\frac{r-1}{r}$, qui prend ici pour $r=2$ la valeur $\frac{1}{2}$.


%%%%%%%%%%%%%%%%%%%%%%%%%%%%%%%%%%%%%%%%%%%%%%%%%%%%%%%%%%%
\diapo

Dans cette cinquième question, on étudie le cas où $3<r<1+\sqrt{6}$.

\change

On commencera par chercher les points fixes $\ell_1$ et $\ell_2$ de $f\circ f$  
qui ne sont pas des points fixes de $f$.


\change

Puis on montrera que $f$ permute ces points fixes,

\change

et enfin on vérifiera sur un exemple que les suites $(u_{2n})$ et 
$(u_{2n+1})$ sont croissantes ou décroissantes à partir d'un certain rang
    et convergent, l'une vers $\ell_1$, l'autre vers $\ell_2$.



%%%%%%%%%%%%%%%%%%%%%%%%%%%%%%%%%%%%%%%%%%%%%%%%%%%%%%%%%%%
\diapo

Voici le code permettant de répondre à cette question.

La seule nouveauté par rapport au calcul des points 
fixes de $f$ est qu'il faut considérer, non plus $f$, mais maintenant $f\circ f$ ici notée $g$,
avant de chercher ses points fixes.

\change

On obtient alors deux points fixes de $g$ qui ne sont pas des points fixes de $f$. 
%Noter que l'on a du un peu aider Sage a simplifier les expressions.

\change

Pour finir, on demande de vérifier formellement l'égalité $f(\ell_1)=\ell_2$. 

La seconde égalité est alors immédiate.
%et la $f(\ell_2)=\ell_1$.

%%%%%%%%%%%%%%%%%%%%%%%%%%%%%%%%%%%%%%%%%%%%%%%%%%%%%%%%%%%
\diapo

Pour la vérification graphique, fixons les valeurs $r=1+\sqrt{5}$ pour définir la fonction $f$ et 
$u_0=\frac23$ comme premier terme de la suite.

On peut à nouveau  illustrer la situation en traçant les lignes 
brisées associées aux deux sous-suites des termes de rangs pairs 
et impairs à l'aide de la fonction \codeinline{dessine\_suite}, 
appliquée cette fois ci à $g=f\circ f$. 


%\change

A gauche, la sous-suite des termes de rangs pairs (en partant de $u_0$) décroît vers sa limite $\ell_1$, 

%\change

tandis qu'à droite, la sous-suite des termes de rangs impairs (en partant de $u_1$) croît vers sa limite $\ell_2$. 

%\change

%%%%%%%%%%%%%%%%%%%%%%%%%%%%%%%%%%%%%%%%%%%%%%%%%%%%%%%%%%%
%%%%%%%%%%%%%%%%%%%%%%%%%%%%%%%%%%%%%%%%%%%%%%%%%%%%%%%%%%%
%%%%%%%%%%%%%%%%%%%%%%%%%%%%%%%%%%%%%%%%%%%%%%%%%%%%%%%%%%%
%\diapo
%%%%%%%%%%%%%%%%%%%%%%%%%%%%%%%%%%%%%%%%%%%%%%%%%%%%%%%%%%%
%%%%%%%%%%%%%%%%%%%%%%%%%%%%%%%%%%%%%%%%%%%%%%%%%%%%%%%%%%%
%%%%%%%%%%%%%%%%%%%%%%%%%%%%%%%%%%%%%%%%%%%%%%%%%%%%%%%%%%%
\change

On peut également tracer la ligne brisée associée à tous les termes de la suite 
en revenant cette fois à la fonction $f$.

On observe nettement les deux valeurs d'adhérence. 
On dit que la suite  possède un cycle attracteur de période $2$.


%%%%%%%%%%%%%%%%%%%%%%%%%%%%%%%%%%%%%%%%%%%%%%%%%%%%%%%%%%%
\diapo
Dans les deux dernières questions, on s'intéresse aux cas restants.

Premier cas : $1+\sqrt{6}<r<3, 5699456\ldots$ dans lequel on souhaite observer un cycle attracteur de période une puissance de $2$.

Deuxième et dernier cas : $r > 3, 5699456\ldots$ dans lequel on souhaite mettre en évidence le caractère chaotique de la suite.

%%%%%%%%%%%%%%%%%%%%%%%%%%%%%%%%%%%%%%%%%%%%%%%%%%%%%%%%%%%
\diapo

On obtient par exemple un cycle de longueur $8$ en prenant $r=3,56$, $u_0=0,35$ :

%%%%%%%%%%%%%%%%%%%%%%%%%%%%%%%%%%%%%%%%%%%%%%%%%%%%%%%%%%%
%%%%%%%%%%%%%%%%%%%%%%%%%%%%%%%%%%%%%%%%%%%%%%%%%%%%%%%%%%%
%%%%%%%%%%%%%%%%%%%%%%%%%%%%%%%%%%%%%%%%%%%%%%%%%%%%%%%%%%%
%\diapo
%%%%%%%%%%%%%%%%%%%%%%%%%%%%%%%%%%%%%%%%%%%%%%%%%%%%%%%%%%%
%%%%%%%%%%%%%%%%%%%%%%%%%%%%%%%%%%%%%%%%%%%%%%%%%%%%%%%%%%%
%%%%%%%%%%%%%%%%%%%%%%%%%%%%%%%%%%%%%%%%%%%%%%%%%%%%%%%%%%%
\change

Le cas $r=4$ correspond à une situation chaotique.

Ce cas passionnant est détaillé dans le polycopié et je vous encourage vivement 
à l'étudier.

\end{document}
