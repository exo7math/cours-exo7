
%%%%%%%%%%%%%%%%%% PREAMBULE %%%%%%%%%%%%%%%%%%


\documentclass[12pt]{article}

\usepackage{amsfonts,amsmath,amssymb,amsthm}
\usepackage[utf8]{inputenc}
\usepackage[T1]{fontenc}
\usepackage[francais]{babel}


% packages
\usepackage{amsfonts,amsmath,amssymb,amsthm}
\usepackage[utf8]{inputenc}
\usepackage[T1]{fontenc}
%\usepackage{lmodern}

\usepackage[francais]{babel}
\usepackage{fancybox}
\usepackage{graphicx}

\usepackage{float}

%\usepackage[usenames, x11names]{xcolor}
\usepackage{tikz}
\usepackage{datetime}

\usepackage{mathptmx}
%\usepackage{fouriernc}
%\usepackage{newcent}
\usepackage[mathcal,mathbf]{euler}

%\usepackage{palatino}
%\usepackage{newcent}


% Commande spéciale prompteur

%\usepackage{mathptmx}
%\usepackage[mathcal,mathbf]{euler}
%\usepackage{mathpple,multido}

\usepackage[a4paper]{geometry}
\geometry{top=2cm, bottom=2cm, left=1cm, right=1cm, marginparsep=1cm}

\newcommand{\change}{{\color{red}\rule{\textwidth}{1mm}\\}}

\newcounter{mydiapo}

\newcommand{\diapo}{\newpage
\hfill {\normalsize  Diapo \themydiapo \quad \texttt{[\jobname]}} \\
\stepcounter{mydiapo}}


%%%%%%% COULEURS %%%%%%%%%%

% Pour blanc sur noir :
%\pagecolor[rgb]{0.5,0.5,0.5}
% \pagecolor[rgb]{0,0,0}
% \color[rgb]{1,1,1}



%\DeclareFixedFont{\myfont}{U}{cmss}{bx}{n}{18pt}
\newcommand{\debuttexte}{
%%%%%%%%%%%%% FONTES %%%%%%%%%%%%%
\renewcommand{\baselinestretch}{1.5}
\usefont{U}{cmss}{bx}{n}
\bfseries

% Taille normale : commenter le reste !
%Taille Arnaud
%\fontsize{19}{19}\selectfont

% Taille Barbara
%\fontsize{21}{22}\selectfont

%Taille François
\fontsize{25}{30}\selectfont

%Taille Pascal
%\fontsize{25}{30}\selectfont

%Taille Laura
%\fontsize{30}{35}\selectfont


%\myfont
%\usefont{U}{cmss}{bx}{n}

%\Huge
%\addtolength{\parskip}{\baselineskip}
}


% \usepackage{hyperref}
% \hypersetup{colorlinks=true, linkcolor=blue, urlcolor=blue,
% pdftitle={Exo7 - Exercices de mathématiques}, pdfauthor={Exo7}}


%section
% \usepackage{sectsty}
% \allsectionsfont{\bf}
%\sectionfont{\color{Tomato3}\upshape\selectfont}
%\subsectionfont{\color{Tomato4}\upshape\selectfont}

%----- Ensembles : entiers, reels, complexes -----
\newcommand{\Nn}{\mathbb{N}} \newcommand{\N}{\mathbb{N}}
\newcommand{\Zz}{\mathbb{Z}} \newcommand{\Z}{\mathbb{Z}}
\newcommand{\Qq}{\mathbb{Q}} \newcommand{\Q}{\mathbb{Q}}
\newcommand{\Rr}{\mathbb{R}} \newcommand{\R}{\mathbb{R}}
\newcommand{\Cc}{\mathbb{C}} 
\newcommand{\Kk}{\mathbb{K}} \newcommand{\K}{\mathbb{K}}

%----- Modifications de symboles -----
\renewcommand{\epsilon}{\varepsilon}
\renewcommand{\Re}{\mathop{\text{Re}}\nolimits}
\renewcommand{\Im}{\mathop{\text{Im}}\nolimits}
%\newcommand{\llbracket}{\left[\kern-0.15em\left[}
%\newcommand{\rrbracket}{\right]\kern-0.15em\right]}

\renewcommand{\ge}{\geqslant}
\renewcommand{\geq}{\geqslant}
\renewcommand{\le}{\leqslant}
\renewcommand{\leq}{\leqslant}

%----- Fonctions usuelles -----
\newcommand{\ch}{\mathop{\mathrm{ch}}\nolimits}
\newcommand{\sh}{\mathop{\mathrm{sh}}\nolimits}
\renewcommand{\tanh}{\mathop{\mathrm{th}}\nolimits}
\newcommand{\cotan}{\mathop{\mathrm{cotan}}\nolimits}
\newcommand{\Arcsin}{\mathop{\mathrm{Arcsin}}\nolimits}
\newcommand{\Arccos}{\mathop{\mathrm{Arccos}}\nolimits}
\newcommand{\Arctan}{\mathop{\mathrm{Arctan}}\nolimits}
\newcommand{\Argsh}{\mathop{\mathrm{Argsh}}\nolimits}
\newcommand{\Argch}{\mathop{\mathrm{Argch}}\nolimits}
\newcommand{\Argth}{\mathop{\mathrm{Argth}}\nolimits}
\newcommand{\pgcd}{\mathop{\mathrm{pgcd}}\nolimits} 

\newcommand{\Card}{\mathop{\text{Card}}\nolimits}
\newcommand{\Ker}{\mathop{\text{Ker}}\nolimits}
\newcommand{\id}{\mathop{\text{id}}\nolimits}
\newcommand{\ii}{\mathrm{i}}
\newcommand{\dd}{\mathrm{d}}
\newcommand{\Vect}{\mathop{\text{Vect}}\nolimits}
\newcommand{\Mat}{\mathop{\mathrm{Mat}}\nolimits}
\newcommand{\rg}{\mathop{\text{rg}}\nolimits}
\newcommand{\tr}{\mathop{\text{tr}}\nolimits}
\newcommand{\ppcm}{\mathop{\text{ppcm}}\nolimits}

%----- Structure des exercices ------

\newtheoremstyle{styleexo}% name
{2ex}% Space above
{3ex}% Space below
{}% Body font
{}% Indent amount 1
{\bfseries} % Theorem head font
{}% Punctuation after theorem head
{\newline}% Space after theorem head 2
{}% Theorem head spec (can be left empty, meaning ‘normal’)

%\theoremstyle{styleexo}
\newtheorem{exo}{Exercice}
\newtheorem{ind}{Indications}
\newtheorem{cor}{Correction}


\newcommand{\exercice}[1]{} \newcommand{\finexercice}{}
%\newcommand{\exercice}[1]{{\tiny\texttt{#1}}\vspace{-2ex}} % pour afficher le numero absolu, l'auteur...
\newcommand{\enonce}{\begin{exo}} \newcommand{\finenonce}{\end{exo}}
\newcommand{\indication}{\begin{ind}} \newcommand{\finindication}{\end{ind}}
\newcommand{\correction}{\begin{cor}} \newcommand{\fincorrection}{\end{cor}}

\newcommand{\noindication}{\stepcounter{ind}}
\newcommand{\nocorrection}{\stepcounter{cor}}

\newcommand{\fiche}[1]{} \newcommand{\finfiche}{}
\newcommand{\titre}[1]{\centerline{\large \bf #1}}
\newcommand{\addcommand}[1]{}
\newcommand{\video}[1]{}

% Marge
\newcommand{\mymargin}[1]{\marginpar{{\small #1}}}



%----- Presentation ------
\setlength{\parindent}{0cm}

%\newcommand{\ExoSept}{\href{http://exo7.emath.fr}{\textbf{\textsf{Exo7}}}}

\definecolor{myred}{rgb}{0.93,0.26,0}
\definecolor{myorange}{rgb}{0.97,0.58,0}
\definecolor{myyellow}{rgb}{1,0.86,0}

\newcommand{\LogoExoSept}[1]{  % input : echelle
{\usefont{U}{cmss}{bx}{n}
\begin{tikzpicture}[scale=0.1*#1,transform shape]
  \fill[color=myorange] (0,0)--(4,0)--(4,-4)--(0,-4)--cycle;
  \fill[color=myred] (0,0)--(0,3)--(-3,3)--(-3,0)--cycle;
  \fill[color=myyellow] (4,0)--(7,4)--(3,7)--(0,3)--cycle;
  \node[scale=5] at (3.5,3.5) {Exo7};
\end{tikzpicture}}
}



\theoremstyle{definition}
%\newtheorem{proposition}{Proposition}
%\newtheorem{exemple}{Exemple}
%\newtheorem{theoreme}{Théorème}
\newtheorem{lemme}{Lemme}
\newtheorem{corollaire}{Corollaire}
%\newtheorem*{remarque*}{Remarque}
%\newtheorem*{miniexercice}{Mini-exercices}
%\newtheorem{definition}{Définition}




%definition d'un terme
\newcommand{\defi}[1]{{\color{myorange}\textbf{\emph{#1}}}}
\newcommand{\evidence}[1]{{\color{blue}\textbf{\emph{#1}}}}



 %----- Commandes divers ------

\newcommand{\codeinline}[1]{\texttt{#1}}

%%%%%%%%%%%%%%%%%%%%%%%%%%%%%%%%%%%%%%%%%%%%%%%%%%%%%%%%%%%%%
%%%%%%%%%%%%%%%%%%%%%%%%%%%%%%%%%%%%%%%%%%%%%%%%%%%%%%%%%%%%%


\begin{document}

\debuttexte


%%%%%%%%%%%%%%%%%%%%%%%%%%%%%%%%%%%%%%%%%%%%%%%%%%%%%%%%%%%
\diapo

Nous allons voir comment le calcul formel peut nous aider à comprendre et résoudre des équations différentielles.


\change

\change

Nous commencerons par des équations différentielles du premier ordre du type 
 $y' = f(x,y)$

\change
Nous visualiserons ce que signifie résoudre des équations différentielles.

\change
Et nous terminerons par des équations différentielles du second ordre.


%%%%%%%%%%%%%%%%%%%%%%%%%%%%%%%%%%%%%%%%%%%%%%%%%%%%%%%%%%%
\diapo


Voici notre premier travail : résoudre plusieurs équations différentielles du premier ordre.

%Tout d'abord résoudre 
On s'intéresse d'abord à l'équation différentielle :
  $y' = y + x + 1.$
  
  Ici $y$ est une fonction inconnue de la variable $x$.

Si %on exige 
nous imposons une condition initiale $y(0)=1$, alors nous ne devrions trouver qu'une unique solution.
  

Nous ferons un travail similaire pour l'équation $x^2y' = (x-1)y$.

Enfin, en insistant un peu  nous obtiendrons une solution de 
l'équation différentielle $y'= y(1-y) - 1.$

\change
Vous pouvez bien sûr faire les calculs à la main ou demander à la machine de le faire !
C'est la commande \codeinline{desolve} qui permet de résoudre formellement des équations différentielles.


Consultez la documentation, accessible par \codeinline{help(desolve)}.



%%%%%%%%%%%%%%%%%%%%%%%%%%%%%%%%%%%%%%%%%%%%%%%%%%%%%%%%%%%
\diapo 


On commence par l'équation différentielle :
  $y' = y + x + 1.$

\change
On déclare la variable $x$. 

\change
Mais ici on doit aussi déclarer la fonction qui sera notre fonction inconnue $x \mapsto y(x)$ à déterminer.

\change
Pour définir l'équation différentielle nous aurons aussi besoin de la dérivée $y'(x)$, définie 
par  \codeinline{diff(y,x)}. 

Nous prendrons ici la convention de noter $y'$ par \codeinline{yy}.

\change
L'équation différentielle $y' = y + x + 1$
s'écrit donc \codeinline{yy == y+x+1}.

On peut maintenant demander à Sage de résoudre l'équation différentielle par la commande  
\codeinline{desolve}. Les arguments sont  
l'équation différentielle à résoudre ainsi que la fonction inconnue, qui est ici $y$. 

\change
Sage\ renvoie : $-\left((x + 1)*e^{(-x)} - \_C + e^{(-x)}\right)*e^x$

\change
\codeinline{$\_C$} désigne une constante.

\change
Après simplification évidente, les solutions sont les :
$$y(x) = -x-2 + C \exp(x) \quad \text{ où } \quad C \in \Rr.$$

\change
Il est également possible de préciser une condition 
initiale grâce à l'argument optionnel \codeinline{ics} 

[voir la dernière ligne du code], 

ici on 
spécifie donc que l'on cherche la solution vérifiant $y(0)=1$.


\change
L'unique solution est alors : $y(x) = -x-2 + 3\exp(x)$, c'est-à-dire $C=3$.

%%%%%%%%%%%%%%%%%%%%%%%%%%%%%%%%%%%%%%%%%%%%%%%%%%%%%%%%%%%
\diapo 

Voici l'équation différentielle $x^2y' = (x-1)y$.

\change
Et voici comment la résoudre sur les réels strictement positifs :


%\change

Ici l'équation différentielle, avec toujours notre notation $yy$ pour $y'(x)$

On fait l'hypothèse que les $x$ sont strictement positifs, puis on résout l'équation différentielle.

\change
La solution est $y(x) = Cx \exp(\frac1x)$, $C\in \Rr$.
  

\change 
La solution vérifiant $y(1)=2$...


\change
est $y(x) = 2x \exp(\frac1x - 1)$.

\change
Sur $\Rr_-^*$ c-à-d pour les $x<0$, 
on résout aussi l'équation différentielle.

\change
Les solutions sont aussi de la forme $y(x) = Cx \exp(\frac1x)$.

\change
    La seule solution définie sur $\Rr$ est donc la fonction nulle $y(x) = 0$ (c'est-à-dire $C=0$). 
    En effet, si $C \neq 0$ il n'y a pas de limite finie en $0$.
    
    
    
%%%%%%%%%%%%%%%%%%%%%%%%%%%%%%%%%%%%%%%%%%%%%%%%%%%%%%%%%%%
\diapo 

Résolvons l'équation différentielle $y'= y(1-y) - 1.$  

\change
On définit donc notre équation différentielle, notez encore une fois la différence entre l'affectation de la variable "equadiff" et l'équation qu'elle contient avec ici le double signe "=".

\change
Si on demande de résoudre cette équation, alors Sage\ ne donne pas directement la solution.

\change  
En effet, c'est une nouvelle équation vérifiée par $y$ qui est renvoyée : 
$$
  \frac{2}{3} \, \sqrt{3} \arctan\left(\frac{1}{3} \, \sqrt{3} {\big(2 \, y\left(x\right) - 1\big)}\right) = C + x.
$$
  
Même si ce n'est pas la réponse souhaitée, c'est un progrès. En effet, ce n'est plus une équation différentielle.
% (il n'y a plus $y'$).

\change
Il suffit alors de résoudre cette dernière équation 

\change
pour obtenir :
$$  y\left(x\right) = -\frac{\sqrt{3}}{2} \tan\left(\frac{\sqrt{3}}{2} C + \frac{\sqrt{3}}{2}  x\right) + \frac{1}{2}.
$$
  
  

Attention, notez que l'équation vérifiée par $y$ est valide pour tout $x$, tandis que la solution formulée ici ne l'est pas, il est nécessaire de préciser l'ensemble de définition.

%  L'équation (1) a l'avantage d'être valide pour tout $x$, alors 
%  que pour l'équation (2) il faudrait préciser l'intervalle de définition.
  
  



%%%%%%%%%%%%%%%%%%%%%%%%%%%%%%%%%%%%%%%%%%%%%%%%%%%%%%%%%%%
\diapo 


Pour bien comprendre les équations différentielles, il est utile de 
%bien comprendre
s'intéresser à leur l'interprétation géométrique.

Considérons une équation différentielle du type $y' = f(x,y)$.


%Le plus important sont
La notion la plus importante pour interpréter géométriquement cette équation est celle de courbe intégrale. 
Les \defi{courbes intégrales} 
%qui 
sont simplement les graphes des solutions.
%de cette équation.


\change
 \`A chaque point d'une courbe intégrale, on associe un vecteur tangent.
 
 
 Comme ici, la courbe intégrale en bleue et quelques vecteurs tangents en rouge.
 
Calculons les coordonnées de ces vecteurs tangents.  
  
Si $y(x)$ est une solution de l'équation différentielle, alors
au point de coordonnées $(x,y(x))$, la tangente est portée par le vecteur $(x',y'(x))$.
Mais, d'une part, $x' = 1$ et d'autre part, comme $y$ est solution de l'équation différentielle,
$y'(x) = f(x,y)$. Ainsi un \defi{vecteur tangent} en $(x,y)$ est $\big( 1 , f(x,y) \big)$.


\change
On peut également faire le cheminement inverse. On part toujours d'une équation différentielle $y' = f(x,y)$, et cette fois on définit d'abord les vecteurs tangents.

Plus exactement, le  \defi{champ de vecteurs} associé à l'équation différentielle est, en chaque point $(x,y)$ du plan, 
le vecteur $\big( 1 , f(x,y) \big)$.


\change
Suite à ces considérations, comment pouvons nous interpréter la résolution d'une équation différentielle ?

Eh bien, partant d'une équation différentielle, on lui associe 
%tout un tas de vecteurs.
un champ de vecteurs, 
%En fait 
c'est-à-dire qu'à chaque point du plan, on associe un vecteur. 

Il s'agit maintenant de trouver une courbe intégrale,
c'est-à-dire une courbe qui en chacun de ses points possède une tangente dirigée par un des vecteurs du champ.


%est tangente en chacun de ces points à l'un de ces vecteurs.

%à tous ces  vecteurs.




%%%%%%%%%%%%%%%%%%%%%%%%%%%%%%%%%%%%%%%%%%%%%%%%%%%%%%%%%%%
\diapo 



Nous allons faire l'étude graphique présentée précédemment sur l'équation différentielle suivante : 
  $$y' = -xy.$$
  
  
Nous commencerons par dessiner le champ de vecteurs associé à cette équation différentielle.



Puis après avoir résolu l'équation, nous tracerons des courbes intégrales, 


et enfin, quelques isoclines.


\change
Les \defi{isoclines} sont les courbes 
définies par l'équation $f(x,y) = c$.
%$\big\lbrace (x,y) \in \Rr^2 \mid f(x,y) = c \big\rbrace.$

Les courbes solutions de l'équation différentielle coupant une isocline donnée possèdent donc toutes en leur point d'intersection avec l'isocline un vecteur tangent de même direction.
%En chaque point d'une isocline, la 
%%solution 
%courbe intégrale passant par ce point croise cette isocline avec une pente $c$ fixée. 

%%%%%%%%%%%%%%%%%%%%%%%%%%%%%%%%%%%%%%%%%%%%%%%%%%%%%%%%%%%
\diapo 

Voici les graphes que vous devriez avoir obtenus.

Tout d'abord le champ de vecteurs : à chaque point du plan, on associe un vecteur. Bien sûr, on ne dessine pas tous les vecteurs, mais seulement  ceux partant de points d'une grille.

\change
Voici les courbes intégrales, chaque courbe est le graphe d'une solution de notre équation différentielle. Chaque courbe correspond à une condition initiale, par exemple si on impose $y(0)=1$ alors on obtient cette solution.

Si on impose $y(0)=0$ alors on obtient ici la solution identiquement nulle.

\change
%Les courbes intégrales (en bleu).
Voici le champ de vecteurs superposé aux courbes intégrales : 

%nous avons bien résolu l'équation différentielle: partant du champ de vecteur (en rouge) nous avons trouvé les courbes qui sont tangentes en chaque point à ces vecteurs. 

%Regardez 
Observez comme chaque vecteur est bien tangent à une courbe intégrale et réciproquement.

\change
Voici les isoclines (en vert). 

Pour une isocline fixée, on vérifie graphiquement que chaque intersection 
  avec les courbes intégrales se fait en des points où la tangente a toujours la même pente.
  
Par exemple en tout ces points [montrer] de cette isocline verte,

la pente des courbes intégrales bleues est toujours la même.

Sur cette isocline particulière, [la droite verticale] les courbes intégrales ont toute une tangente horizontale.


  
%%%%%%%%%%%%%%%%%%%%%%%%%%%%%%%%%%%%%%%%%%%%%%%%%%%%%%%%%%%
\diapo 
Voici comment tracer le champ de vecteurs associé à $y' = f(x,y)$.
  
On commence par définir une grille de points, les origines de nos vecteurs.

L'indice $i$ définit les abscisses, l'indice $j$ les ordonnées, 
espacés du pas $\delta$. 
%ce qui définit des points dans une grille avec un pas $\delta$.

Voici un point de la grille, considéré comme un vecteur pour faciliter les manipulations.

Voici le vecteur du champ en ce point, il est en effet défini par $\big( 1 , f(x,y) \big)$.

Pour chaque point, on trace ce vecteur : c'est une flèche d'origine notre point $pt$.
% et va jusqu'au point défini par notre point plus le vecteur tangent.

\change
On suppose que les variables utilisées pour définir la fonction $f$ sont ici $u$ et $v$, ceci afin de ne pas interférer avec la variable $x$ et la fonction $y$.


%%%%%%%%%%%%%%%%%%%%%%%%%%%%%%%%%%%%%%%%%%%%%%%%%%%%%%%%%%%
\diapo   

Voici le champ de vecteurs pour l'équation différentielle $y' = -xy$.

Il peut être nécessaire de renormaliser les vecteurs $v$ pour plus de lisibilité. % (par exemple en $v/10$) 

~

En fait, il existe déjà une fonction de Sage qui trace les champs de vecteurs : il s'agit de (\codeinline{plot\_vector\_field}).
  
%%%%%%%%%%%%%%%%%%%%%%%%%%%%%%%%%%%%%%%%%%%%%%%%%%%%%%%%%%%
\diapo   
 
\change
Les solutions de l'équation $y' = -xy$ sont définies par :
    $$y(x) = C \exp\left(-\frac{x^2}{2}\right) \quad \text{ où } \quad C \in \R.$$
    
\change
 Le code suivant résout l'équation différentielle et trace la solution 
    qui vérifie la condition initiale $y(0)=k$.
    
 La boucle permet de faire varier ce paramètre $k$ entre une valeur minimale et une valeur maximale   
    avec un pas de \codeinline{delta}.
 

%%%%%%%%%%%%%%%%%%%%%%%%%%%%%%%%%%%%%%%%%%%%%%%%%%%%%%%%%%%
\diapo    

Les courbes intégrales forment une partition 
de l'espace : par un point du plan, il passe une et une seule courbe intégrale. 

C'est la reformulation géométrique du théorème d'existence et d'unicité de Cauchy-Lipschitz. 


%%%%%%%%%%%%%%%%%%%%%%%%%%%%%%%%%%%%%%%%%%%%%%%%%%%%%%%%%%%
\diapo    

 Les isoclines sont définies par l'équation $f(x,y) = c$.

\change
Elles se tracent comme des courbes définies par une équation implicite par la commande \codeinline{implicit\_plot()} que nous avons déjà eu l'occasion d'utiliser.
 
\change 
  Ici les isoclines ont pour équation $-xy=c$, ce sont donc des hyperboles.
  
  

%%%%%%%%%%%%%%%%%%%%%%%%%%%%%%%%%%%%%%%%%%%%%%%%%%%%%%%%%%%
\diapo

Nous terminons par la résolution d'une équation différentielle linéaire du second ordre
à coefficients constants.

~

Cette équation modélisera la situation suivante : 

On considère une masse posée sur le sol et attachée à un ressort. On écarte la masse de sa position d'équilibre puis on la lâche.

Se pose alors la question de savoir quel est le mouvement de la masse.

%Une masse  est attachée à un ressort, on lâche la masse et nous allons étudier son mouvement.

On peut montrer que ce mouvement est décrit par l'équation différentielle
$$x''(t) +f x'(t)+ \frac{k}{m} x(t) = 0$$ 

où 

$x$, la fonction inconnue, représente la position de la masse en fonction du temps $t$.

[petit] $m$ est la masse,

$k0$ est une constante strictememnt positive liée au ressort,

et enfin $f$ est le coefficient de frottement de la masse sur le sol.

~


Nous allons résoudre cette équation en prenant par exemple $m=1$, $k=1$ et faire varier le coefficient de frottement en prenant par exemple quelques valeurs.




%%%%%%%%%%%%%%%%%%%%%%%%%%%%%%%%%%%%%%%%%%%%%%%%%%%%%%%%%%%
\diapo

Intéressons nous aux cas particuliers où le coefficient de frottements vaut $0$, $1$ et $2$.

~

Voici la courbe représentative de la solution dans le cas $f=0$, c'est-à-dire lorsqu'il n'y a pas de frottement.

En abscisse le temps $t$ et en ordonnée la position $x(t)$ de la masse.

Nous obtenons sans surprise un mouvement périodique.

\change

Maintenant $f=1$. Les frottements sont faibles.

Nous obtenons un mouvement oscillant amorti autour de la position d'origine $x=0$.
 
Notez que la solution vérifie bien la condition initiale $x(0)=1$ et $x'(0)=0$ car la tangente à l'origine est horizontale. 
 
\change

Enfin pour de plus grandes valeurs du coefficient de frottement,
%les frottements sont forts. 
il n'y a plus d'oscillations autour de la position d'origine.



%%%%%%%%%%%%%%%%%%%%%%%%%%%%%%%%%%%%%%%%%%%%%%%%%%%%%%%%%%%
\diapo


Pour une valeur de $f$, l'équation se résout par la commande suivante :

ici l'équation différentielle [montrer pas à pas] 

$$x''(t) +f x'(t)+ \frac{k}{m} x(t) = 0$$

ici la condition initiale 

en $0$, $x(0) = 1$ et aussi en $0$, $x'(0)=0$.

%on demande à Sage de résoudre cette équation.

\change
 Dans le cas où $f=0$, il n'y a pas de frottement. 

\change 
 L'équation différentielle s'écrit
  $x''+ \frac{k}{m} x = 0$. 
  
\change
On fixe par exemple  $k=1$, $m=1$. L'équation devient alors
  $x''+ x = 0$ ...
  
\change  
 ...  équation dont les solutions sont de la forme $\lambda \cos t + \mu \sin t$ avec $\lambda$ et $\mu$ réels.
  
\change
Si on impose de plus les conditions initiales $x(0)=1$ et $x'(0)=0$, 

\change 
l'unique solution est $x(t) = \cos t$.
 
\change 
Il s'agit bien du mouvement périodique annoncé.


\change
Pour de petit frottements, par exemple $f=1$.


\change 
La solution qui vérifie les conditions initiales est un peu plus compliquée
  $$\left(\frac{\sqrt{3}}{3} \sin\left(\frac{\sqrt{3}}{2} t\right) + 
   \cos\left(\frac{\sqrt{3}}{2} t\right)\right)
  e^{-\frac{1}{2} t}. $$

\change  
  L'exponentielle amortit le mouvement qui oscille autour de la position d'origine.
  
\change 
Pour des frottements plus importants, par exemple $f=2$,

\change 
la solution qui vérifie toujours les mêmes conditions initiales est 
  $$x(t) = (t + 1)e^{-t}.$$
  
\change 
  Il n'y a plus d'oscillations autour de la position d'origine.



\end{document}
