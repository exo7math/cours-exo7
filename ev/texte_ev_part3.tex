
%%%%%%%%%%%%%%%%%% PREAMBULE %%%%%%%%%%%%%%%%%%


\documentclass[12pt]{article}

\usepackage{amsfonts,amsmath,amssymb,amsthm}
\usepackage[utf8]{inputenc}
\usepackage[T1]{fontenc}
\usepackage[francais]{babel}


% packages
\usepackage{amsfonts,amsmath,amssymb,amsthm}
\usepackage[utf8]{inputenc}
\usepackage[T1]{fontenc}
%\usepackage{lmodern}

\usepackage[francais]{babel}
\usepackage{fancybox}
\usepackage{graphicx}

\usepackage{float}

%\usepackage[usenames, x11names]{xcolor}
\usepackage{tikz}
\usepackage{datetime}

\usepackage{mathptmx}
%\usepackage{fouriernc}
%\usepackage{newcent}
\usepackage[mathcal,mathbf]{euler}

%\usepackage{palatino}
%\usepackage{newcent}


% Commande spéciale prompteur

%\usepackage{mathptmx}
%\usepackage[mathcal,mathbf]{euler}
%\usepackage{mathpple,multido}

\usepackage[a4paper]{geometry}
\geometry{top=2cm, bottom=2cm, left=1cm, right=1cm, marginparsep=1cm}

\newcommand{\change}{{\color{red}\rule{\textwidth}{1mm}\\}}

\newcounter{mydiapo}

\newcommand{\diapo}{\newpage
\hfill {\normalsize  Diapo \themydiapo \quad \texttt{[\jobname]}} \\
\stepcounter{mydiapo}}


%%%%%%% COULEURS %%%%%%%%%%

% Pour blanc sur noir :
%\pagecolor[rgb]{0.5,0.5,0.5}
% \pagecolor[rgb]{0,0,0}
% \color[rgb]{1,1,1}



%\DeclareFixedFont{\myfont}{U}{cmss}{bx}{n}{18pt}
\newcommand{\debuttexte}{
%%%%%%%%%%%%% FONTES %%%%%%%%%%%%%
\renewcommand{\baselinestretch}{1.5}
\usefont{U}{cmss}{bx}{n}
\bfseries

% Taille normale : commenter le reste !
%Taille Arnaud
%\fontsize{19}{19}\selectfont

% Taille Barbara
%\fontsize{21}{22}\selectfont

%Taille François
\fontsize{25}{30}\selectfont

%Taille Pascal
%\fontsize{25}{30}\selectfont

%Taille Laura
%\fontsize{30}{35}\selectfont


%\myfont
%\usefont{U}{cmss}{bx}{n}

%\Huge
%\addtolength{\parskip}{\baselineskip}
}


% \usepackage{hyperref}
% \hypersetup{colorlinks=true, linkcolor=blue, urlcolor=blue,
% pdftitle={Exo7 - Exercices de mathématiques}, pdfauthor={Exo7}}


%section
% \usepackage{sectsty}
% \allsectionsfont{\bf}
%\sectionfont{\color{Tomato3}\upshape\selectfont}
%\subsectionfont{\color{Tomato4}\upshape\selectfont}

%----- Ensembles : entiers, reels, complexes -----
\newcommand{\Nn}{\mathbb{N}} \newcommand{\N}{\mathbb{N}}
\newcommand{\Zz}{\mathbb{Z}} \newcommand{\Z}{\mathbb{Z}}
\newcommand{\Qq}{\mathbb{Q}} \newcommand{\Q}{\mathbb{Q}}
\newcommand{\Rr}{\mathbb{R}} \newcommand{\R}{\mathbb{R}}
\newcommand{\Cc}{\mathbb{C}} 
\newcommand{\Kk}{\mathbb{K}} \newcommand{\K}{\mathbb{K}}

%----- Modifications de symboles -----
\renewcommand{\epsilon}{\varepsilon}
\renewcommand{\Re}{\mathop{\text{Re}}\nolimits}
\renewcommand{\Im}{\mathop{\text{Im}}\nolimits}
%\newcommand{\llbracket}{\left[\kern-0.15em\left[}
%\newcommand{\rrbracket}{\right]\kern-0.15em\right]}

\renewcommand{\ge}{\geqslant}
\renewcommand{\geq}{\geqslant}
\renewcommand{\le}{\leqslant}
\renewcommand{\leq}{\leqslant}

%----- Fonctions usuelles -----
\newcommand{\ch}{\mathop{\mathrm{ch}}\nolimits}
\newcommand{\sh}{\mathop{\mathrm{sh}}\nolimits}
\renewcommand{\tanh}{\mathop{\mathrm{th}}\nolimits}
\newcommand{\cotan}{\mathop{\mathrm{cotan}}\nolimits}
\newcommand{\Arcsin}{\mathop{\mathrm{Arcsin}}\nolimits}
\newcommand{\Arccos}{\mathop{\mathrm{Arccos}}\nolimits}
\newcommand{\Arctan}{\mathop{\mathrm{Arctan}}\nolimits}
\newcommand{\Argsh}{\mathop{\mathrm{Argsh}}\nolimits}
\newcommand{\Argch}{\mathop{\mathrm{Argch}}\nolimits}
\newcommand{\Argth}{\mathop{\mathrm{Argth}}\nolimits}
\newcommand{\pgcd}{\mathop{\mathrm{pgcd}}\nolimits} 

\newcommand{\Card}{\mathop{\text{Card}}\nolimits}
\newcommand{\Ker}{\mathop{\text{Ker}}\nolimits}
\newcommand{\id}{\mathop{\text{id}}\nolimits}
\newcommand{\ii}{\mathrm{i}}
\newcommand{\dd}{\mathrm{d}}
\newcommand{\Vect}{\mathop{\text{Vect}}\nolimits}
\newcommand{\Mat}{\mathop{\mathrm{Mat}}\nolimits}
\newcommand{\rg}{\mathop{\text{rg}}\nolimits}
\newcommand{\tr}{\mathop{\text{tr}}\nolimits}
\newcommand{\ppcm}{\mathop{\text{ppcm}}\nolimits}

%----- Structure des exercices ------

\newtheoremstyle{styleexo}% name
{2ex}% Space above
{3ex}% Space below
{}% Body font
{}% Indent amount 1
{\bfseries} % Theorem head font
{}% Punctuation after theorem head
{\newline}% Space after theorem head 2
{}% Theorem head spec (can be left empty, meaning ‘normal’)

%\theoremstyle{styleexo}
\newtheorem{exo}{Exercice}
\newtheorem{ind}{Indications}
\newtheorem{cor}{Correction}


\newcommand{\exercice}[1]{} \newcommand{\finexercice}{}
%\newcommand{\exercice}[1]{{\tiny\texttt{#1}}\vspace{-2ex}} % pour afficher le numero absolu, l'auteur...
\newcommand{\enonce}{\begin{exo}} \newcommand{\finenonce}{\end{exo}}
\newcommand{\indication}{\begin{ind}} \newcommand{\finindication}{\end{ind}}
\newcommand{\correction}{\begin{cor}} \newcommand{\fincorrection}{\end{cor}}

\newcommand{\noindication}{\stepcounter{ind}}
\newcommand{\nocorrection}{\stepcounter{cor}}

\newcommand{\fiche}[1]{} \newcommand{\finfiche}{}
\newcommand{\titre}[1]{\centerline{\large \bf #1}}
\newcommand{\addcommand}[1]{}
\newcommand{\video}[1]{}

% Marge
\newcommand{\mymargin}[1]{\marginpar{{\small #1}}}



%----- Presentation ------
\setlength{\parindent}{0cm}

%\newcommand{\ExoSept}{\href{http://exo7.emath.fr}{\textbf{\textsf{Exo7}}}}

\definecolor{myred}{rgb}{0.93,0.26,0}
\definecolor{myorange}{rgb}{0.97,0.58,0}
\definecolor{myyellow}{rgb}{1,0.86,0}

\newcommand{\LogoExoSept}[1]{  % input : echelle
{\usefont{U}{cmss}{bx}{n}
\begin{tikzpicture}[scale=0.1*#1,transform shape]
  \fill[color=myorange] (0,0)--(4,0)--(4,-4)--(0,-4)--cycle;
  \fill[color=myred] (0,0)--(0,3)--(-3,3)--(-3,0)--cycle;
  \fill[color=myyellow] (4,0)--(7,4)--(3,7)--(0,3)--cycle;
  \node[scale=5] at (3.5,3.5) {Exo7};
\end{tikzpicture}}
}



\theoremstyle{definition}
%\newtheorem{proposition}{Proposition}
%\newtheorem{exemple}{Exemple}
%\newtheorem{theoreme}{Théorème}
\newtheorem{lemme}{Lemme}
\newtheorem{corollaire}{Corollaire}
%\newtheorem*{remarque*}{Remarque}
%\newtheorem*{miniexercice}{Mini-exercices}
%\newtheorem{definition}{Définition}




%definition d'un terme
\newcommand{\defi}[1]{{\color{myorange}\textbf{\emph{#1}}}}
\newcommand{\evidence}[1]{{\color{blue}\textbf{\emph{#1}}}}



 %----- Commandes divers ------

\newcommand{\codeinline}[1]{\texttt{#1}}

%%%%%%%%%%%%%%%%%%%%%%%%%%%%%%%%%%%%%%%%%%%%%%%%%%%%%%%%%%%%%
%%%%%%%%%%%%%%%%%%%%%%%%%%%%%%%%%%%%%%%%%%%%%%%%%%%%%%%%%%%%%



\begin{document}

\debuttexte


%%%%%%%%%%%%%%%%%%%%%%%%%%%%%%%%%%%%%%%%%%%%%%%%%%%%%%%%%%%
\diapo

\change

Il est vite fatiguant de vérifier les $8$ axiomes qui font d'un ensemble un espace vectoriel.
Heureusement, il existe une manière rapide et efficace de prouver qu'un ensemble est un espace vectoriel :
grâce à la notion de sous-espace vectoriel.

\change

Nous allons voir qu'elle est la définition d'un sous-espace vectoriel

\change

Et nous verrons qu'un sous-espace vectoriel est un espace vectoriel


%%%%%%%%%%%%%%%%%%%%%%%%%%%%%%%%%%%%%%%%%%%%%%%%%%%%%%%%%%
\diapo

Partons d'un ensemble $E$ qui est un $\Kk$-espace vectoriel.

Considérons $F$ un sous-ensemble de $E$.

La partie $F$ sera un \defi{sous-espace vectoriel} de $E$ si 
les trois points suivants sont vérifiés :

(1) $0_E \in F$,

(2) pour tous $u,v \in F$, $u+v \in F$,
   
(3)  pour tout $\lambda \in \Kk$ et tout $u \in F$, $\lambda \cdot u \in F$. 



Expliquons chaque condition.
\begin{itemize}
  \item La première condition signifie que le vecteur nul de $E$ doit aussi être dans $F$.
  En fait il suffirait même de prouver que $F$ est non vide.
  
  \item La deuxième condition, c'est dire que $F$ est stable pour l'addition :
  la somme $u+v$ de deux vecteurs de $F$ est bien sûr un vecteur de $E$ 
  (car $E$ est un espace vectoriel),
  mais ici on exige que $u+v$ soit un élément de $F$. 
  
  \item  La troisième condition, c'est dire que $F$ est 
  stable pour la multiplication par un scalaire.
\end{itemize}



%%%%%%%%%%%%%%%%%%%%%%%%%%%%%%%%%%%%%%%%%%%%%%%%%%%%%%%%%%%
\diapo

Commençons par un exemple simple.

Soit $F=\big\{(x,y)\in \Rr^2\mid x+y=0\big\}$
  
Ici $E=\Rr^2$ est un $\Rr$-espace vectoriel.

Nous allons montrer que $F$ est un sous-espace vectoriel de $E$.

\change

Vérifions les trois points :

(1) tout d'abord $(0,0) \in F$,

\change

(2) si $u=(x_1,y_1)$ et $v=(x_2,y_2)$ appartiennent à $F$, 

\change

alors par définition de $F$, $x_1+y_1=0$ et $x_2+y_2=0$ 

\change

cela implique $(x_1+x_2)+(y_1+y_2)=0$

\change

et ainsi $u+v=(x_1+x_2,y_1+y_2)$ appartient à $F$,

$F$ est donc stable par addition

\change

(3) il reste à vérifier que $F$ est stable par multiplication par un scalaire.
    
si $u=(x,y) \in F$ et $\lambda \in \Rr$, 

alors $x+y=0$ donc aussi $\lambda x + \lambda y = 0$,


    d'où $\lambda u \in F$.

    
On a bien montrer que $F$ est un sous-espace vectoriel de $\Rr^2$.


%%%%%%%%%%%%%%%%%%%%%%%%%%%%%%%%%%%%%%%%%%%%%%%%%%%%%%%%%%
\diapo


L'ensemble des fonctions continues sur $\Rr$ est un sous-espace vectoriel 
de l'espace vectoriel des fonctions de $\Rr$ dans $\Rr$.

\change

  Voici la preuve :
  
  (1)  la fonction nulle est continue ;
  
\change
  
  (2) la somme de deux fonctions continues est continue ; 
 
\change

  (3)  une constante fois une fonction continue reste une fonction continue.
  

\change

Autre exemple, l'ensemble des suites réelles *convergentes* est un sous-espace vectoriel 
de l'espace vectoriel des suites.



%%%%%%%%%%%%%%%%%%%%%%%%%%%%%%%%%%%%%%%%%%%%%%%%%%%%%%%%%%
\diapo

Bien sûr tous les sous-ensembles \emph{ne sont pas} des sous-espaces vectoriels.

\change

L'ensemble $F_1=\big\{(x,y)\in \Rr^2\mid x+y=2\big\}$ n'est pas un sous-espace vectoriel de $\Rr^2$.


\change

Le premier axiome n'est pas vérifier : le vecteur nul $(0,0)$ n'appartient pas à $F_1$.

  \change
  
L'ensemble $F_2=\big\{(x,y)\in \Rr^2\mid x=0 \text{ ou } y=0 \big\}$ n'est pas non plus 
un sous-espace vectoriel de $\Rr^2$.

Il contient bien l'origine $(0,0)$,

\change
 
mais n'est pas stable par addition.

Par exemple les vecteurs $u=(1,0)$ et $v=(0,1)$ appartiennent à $F_2$, mais leur somme est 
le vecteur $u+v=(1,1)$, qui lui n'appartient pas à $F_2$.
  
  \change
  
L'ensemble $F_3=\big\{(x,y)\in \Rr^2\mid x \ge 0 \text{ et } y\ge 0\big\}$ n'est pas un sous-espace vectoriel de $\Rr^2$.

Cet ensemble contient l'origine, il est stable par addition, mais n'est pas stable 
par multiplication par un scalaire.

\change

 Par exemple le vecteur $u=(1,1)$ appartient à $F_3$ mais, pour $\lambda = -1$, 
 le vecteur $-u = (-1,-1)$ n'appartient pas à $F_3$.



%%%%%%%%%%%%%%%%%%%%%%%%%%%%%%%%%%%%%%%%%%%%%%%%%%%%%%%%%%%
\diapo

La notion de sous-espace vectoriel prend tout son intérêt avec le théorème suivant :
un sous-espace vectoriel est lui-même un espace vectoriel.


Théorème : "Soient $E$ un $\Kk$-espace vectoriel et $F$ un sous-espace vectoriel de $E$. 
Alors $F$ est lui-même un $\Kk$-espace vectoriel pour les lois induites par $E$." 

C'est ce théorème qui va nous fournir plein d'exemples d'espaces vectoriels.

\change


Pour répondre à une question du type \og L'ensemble $F$ est-il un espace vectoriel ? \fg,
voici une façon efficace de procéder

(1) trouver un espace vectoriel $E$ qui contient $F$, 


puis (2) prouver que $F$ est un sous-espace vectoriel de $E$.



Il y a seulement trois propriétés à vérifier au lieu de huit !



%%%%%%%%%%%%%%%%%%%%%%%%%%%%%%%%%%%%%%%%%%%%%%%%%%%%%%%%%%
\diapo

Appliquons immédiatement le théorème précédent "Un sous-espace vectoriel est lui-même un espace vectoriel"
pour répondre à la question :

Est-ce que l'ensemble des fonctions paires  est un espace vectoriel ?
 
(sur $\Rr$ avec les lois usuelles sur les fonctions) 

  Notons $\mathcal{P}$ l'ensemble des fonctions paires. 
  

\change

  
C'est un sous-ensemble de l'espace vectoriel $\mathcal{F}(\Rr,\Rr)$ des fonctions 

\change

  qui s'écrit
 
 $$ \mathcal{P}=\big\{ f \in \mathcal{F}(\Rr, \Rr ) \mid \forall x \in \Rr , f(-x)=f(x) \big\}$$
 
\change

 Montrons maintenant que $\mathcal{P}$ est un sous-espace vectoriel de l'espace vectoriel des fonctions. 
 C'est très simple à vérifier, :
 
 \change

 la fonction nulle est une fonction paire,   
 
 \change

 si $f,g \in \mathcal{P}$ alors $f+g \in\mathcal{P}$,
 
 \change
 
  si $f\in\mathcal{P}$ et si $\lambda \in \Rr$ alors $\lambda f\in\mathcal{P}$.
  
  \change
  
  Ainsi $\mathcal{P}$ est un sous-espace vectoriel 
  
  et par le théorème de la diapo précédente, $\mathcal{P}$ est un espace vectoriel.
  
  \change
  
  Je vous laisse écrire que l'ensemble $\mathcal{I}$ des fonctions impaires est aussi un espace vectoriel en tant 
  que sous-espace vectoriel.
  
  \change
  
  Il en va de même avec l'ensemble $\mathcal{S}_n$ des matrices symétriques, c'est-à-dire l'ensemble des matrices $A$
  telles que $A^T=A$. C'est un espace vectoriel en tant que sous-espace vectoriel de l'espace vectoriel des matrices $n \times n$.
  
%%%%%%%%%%%%%%%%%%%%%%%%%%%%%%%%%%%%%%%%%%%%%%%%%%%%%%%%%%%
\diapo

Un autre exemple d'espace vectoriel est donné par l'ensemble des solutions d'un système 
linéaire homogène. 

Soit $A$ une matrice à $n$ lignes et $p$ colonnes.

alors $AX = 0$ est un système de $n$ équations à $p$ inconnues,

que l'on peut écrire aussi ainsi.


\change

On a alors que l'ensemble des vecteurs $X$ qui sont solutions de cette équation $AX=0$
est un sous-espace vectoriel de $\Rr^p$.

\change

Pour la preuve on appelle $F$ l'ensemble des vecteurs $X \in \Rr^p$ solutions de l'équation $AX=0$. 
Vérifions que $F$ est un sous-espace vectoriel de $\Rr^p$.
\begin{itemize}
  \item Le vecteur $0$ est un élément de $F$.
  \item $F$ est stable par addition : si $X$ et $X'$ sont des vecteurs solutions, 
  alors $AX = 0$ et $AX' = 0$, donc $A(X + X') = AX + AX' = 0$, et ainsi $X+X'\in F$.
  \item  $F$ est stable par multiplication par un scalaire : si $X$ est un vecteur solution,
  alors $AX = 0$ et on a aussi $A (\lambda X)  = 0$.
  Donc $\lambda X \in F$.
\end{itemize}



%%%%%%%%%%%%%%%%%%%%%%%%%%%%%%%%%%%%%%%%%%%%%%%%%%%%%%%%%%
\diapo

Voyons un exemple concret avec ce système à 3 équations et 3 inconnues.

\change

Une fois que on l'a résolu on trouve que l'ensemble des solutions $F \subset \Rr^3$ de ce système est :
$$F = \big\{ (x =  2s - 3t, y  =  s, z  =  t) \mid s,t \in \Rr \big\}.$$


\change

Par le théorème précédent, $F$ est un sous-espace vectoriel de $\Rr^3$.

\change

et donc $F$ est aussi un espace vectoriel.

\change


Une autre façon de voir les choses est d'écrire que les éléments de $F$ sont ceux qui vérifient
l'équation $(x = 2y - 3z)$.

L'ensemble des solutions $F$ est donc un plan passant par l'origine. 
Nous avons déjà vu que ceci est un espace vectoriel. 



%%%%%%%%%%%%%%%%%%%%%%%%%%%%%%%%%%%%%%%%%%%%%%%%%%%%%%%%%%%
\diapo

Voici toute une liste d'ensembles, à vous de déterminer si ce sont des sous-espaces vectoriels ou pas.

\end{document}
