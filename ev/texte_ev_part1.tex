
%%%%%%%%%%%%%%%%%% PREAMBULE %%%%%%%%%%%%%%%%%%


\documentclass[12pt]{article}

\usepackage{amsfonts,amsmath,amssymb,amsthm}
\usepackage[utf8]{inputenc}
\usepackage[T1]{fontenc}
\usepackage[francais]{babel}


% packages
\usepackage{amsfonts,amsmath,amssymb,amsthm}
\usepackage[utf8]{inputenc}
\usepackage[T1]{fontenc}
%\usepackage{lmodern}

\usepackage[francais]{babel}
\usepackage{fancybox}
\usepackage{graphicx}

\usepackage{float}

%\usepackage[usenames, x11names]{xcolor}
\usepackage{tikz}
\usepackage{datetime}

\usepackage{mathptmx}
%\usepackage{fouriernc}
%\usepackage{newcent}
\usepackage[mathcal,mathbf]{euler}

%\usepackage{palatino}
%\usepackage{newcent}


% Commande spéciale prompteur

%\usepackage{mathptmx}
%\usepackage[mathcal,mathbf]{euler}
%\usepackage{mathpple,multido}

\usepackage[a4paper]{geometry}
\geometry{top=2cm, bottom=2cm, left=1cm, right=1cm, marginparsep=1cm}

\newcommand{\change}{{\color{red}\rule{\textwidth}{1mm}\\}}

\newcounter{mydiapo}

\newcommand{\diapo}{\newpage
\hfill {\normalsize  Diapo \themydiapo \quad \texttt{[\jobname]}} \\
\stepcounter{mydiapo}}


%%%%%%% COULEURS %%%%%%%%%%

% Pour blanc sur noir :
%\pagecolor[rgb]{0.5,0.5,0.5}
% \pagecolor[rgb]{0,0,0}
% \color[rgb]{1,1,1}



%\DeclareFixedFont{\myfont}{U}{cmss}{bx}{n}{18pt}
\newcommand{\debuttexte}{
%%%%%%%%%%%%% FONTES %%%%%%%%%%%%%
\renewcommand{\baselinestretch}{1.5}
\usefont{U}{cmss}{bx}{n}
\bfseries

% Taille normale : commenter le reste !
%Taille Arnaud
%\fontsize{19}{19}\selectfont

% Taille Barbara
%\fontsize{21}{22}\selectfont

%Taille François
\fontsize{25}{30}\selectfont

%Taille Pascal
%\fontsize{25}{30}\selectfont

%Taille Laura
%\fontsize{30}{35}\selectfont


%\myfont
%\usefont{U}{cmss}{bx}{n}

%\Huge
%\addtolength{\parskip}{\baselineskip}
}


% \usepackage{hyperref}
% \hypersetup{colorlinks=true, linkcolor=blue, urlcolor=blue,
% pdftitle={Exo7 - Exercices de mathématiques}, pdfauthor={Exo7}}


%section
% \usepackage{sectsty}
% \allsectionsfont{\bf}
%\sectionfont{\color{Tomato3}\upshape\selectfont}
%\subsectionfont{\color{Tomato4}\upshape\selectfont}

%----- Ensembles : entiers, reels, complexes -----
\newcommand{\Nn}{\mathbb{N}} \newcommand{\N}{\mathbb{N}}
\newcommand{\Zz}{\mathbb{Z}} \newcommand{\Z}{\mathbb{Z}}
\newcommand{\Qq}{\mathbb{Q}} \newcommand{\Q}{\mathbb{Q}}
\newcommand{\Rr}{\mathbb{R}} \newcommand{\R}{\mathbb{R}}
\newcommand{\Cc}{\mathbb{C}} 
\newcommand{\Kk}{\mathbb{K}} \newcommand{\K}{\mathbb{K}}

%----- Modifications de symboles -----
\renewcommand{\epsilon}{\varepsilon}
\renewcommand{\Re}{\mathop{\text{Re}}\nolimits}
\renewcommand{\Im}{\mathop{\text{Im}}\nolimits}
%\newcommand{\llbracket}{\left[\kern-0.15em\left[}
%\newcommand{\rrbracket}{\right]\kern-0.15em\right]}

\renewcommand{\ge}{\geqslant}
\renewcommand{\geq}{\geqslant}
\renewcommand{\le}{\leqslant}
\renewcommand{\leq}{\leqslant}

%----- Fonctions usuelles -----
\newcommand{\ch}{\mathop{\mathrm{ch}}\nolimits}
\newcommand{\sh}{\mathop{\mathrm{sh}}\nolimits}
\renewcommand{\tanh}{\mathop{\mathrm{th}}\nolimits}
\newcommand{\cotan}{\mathop{\mathrm{cotan}}\nolimits}
\newcommand{\Arcsin}{\mathop{\mathrm{Arcsin}}\nolimits}
\newcommand{\Arccos}{\mathop{\mathrm{Arccos}}\nolimits}
\newcommand{\Arctan}{\mathop{\mathrm{Arctan}}\nolimits}
\newcommand{\Argsh}{\mathop{\mathrm{Argsh}}\nolimits}
\newcommand{\Argch}{\mathop{\mathrm{Argch}}\nolimits}
\newcommand{\Argth}{\mathop{\mathrm{Argth}}\nolimits}
\newcommand{\pgcd}{\mathop{\mathrm{pgcd}}\nolimits} 

\newcommand{\Card}{\mathop{\text{Card}}\nolimits}
\newcommand{\Ker}{\mathop{\text{Ker}}\nolimits}
\newcommand{\id}{\mathop{\text{id}}\nolimits}
\newcommand{\ii}{\mathrm{i}}
\newcommand{\dd}{\mathrm{d}}
\newcommand{\Vect}{\mathop{\text{Vect}}\nolimits}
\newcommand{\Mat}{\mathop{\mathrm{Mat}}\nolimits}
\newcommand{\rg}{\mathop{\text{rg}}\nolimits}
\newcommand{\tr}{\mathop{\text{tr}}\nolimits}
\newcommand{\ppcm}{\mathop{\text{ppcm}}\nolimits}

%----- Structure des exercices ------

\newtheoremstyle{styleexo}% name
{2ex}% Space above
{3ex}% Space below
{}% Body font
{}% Indent amount 1
{\bfseries} % Theorem head font
{}% Punctuation after theorem head
{\newline}% Space after theorem head 2
{}% Theorem head spec (can be left empty, meaning ‘normal’)

%\theoremstyle{styleexo}
\newtheorem{exo}{Exercice}
\newtheorem{ind}{Indications}
\newtheorem{cor}{Correction}


\newcommand{\exercice}[1]{} \newcommand{\finexercice}{}
%\newcommand{\exercice}[1]{{\tiny\texttt{#1}}\vspace{-2ex}} % pour afficher le numero absolu, l'auteur...
\newcommand{\enonce}{\begin{exo}} \newcommand{\finenonce}{\end{exo}}
\newcommand{\indication}{\begin{ind}} \newcommand{\finindication}{\end{ind}}
\newcommand{\correction}{\begin{cor}} \newcommand{\fincorrection}{\end{cor}}

\newcommand{\noindication}{\stepcounter{ind}}
\newcommand{\nocorrection}{\stepcounter{cor}}

\newcommand{\fiche}[1]{} \newcommand{\finfiche}{}
\newcommand{\titre}[1]{\centerline{\large \bf #1}}
\newcommand{\addcommand}[1]{}
\newcommand{\video}[1]{}

% Marge
\newcommand{\mymargin}[1]{\marginpar{{\small #1}}}



%----- Presentation ------
\setlength{\parindent}{0cm}

%\newcommand{\ExoSept}{\href{http://exo7.emath.fr}{\textbf{\textsf{Exo7}}}}

\definecolor{myred}{rgb}{0.93,0.26,0}
\definecolor{myorange}{rgb}{0.97,0.58,0}
\definecolor{myyellow}{rgb}{1,0.86,0}

\newcommand{\LogoExoSept}[1]{  % input : echelle
{\usefont{U}{cmss}{bx}{n}
\begin{tikzpicture}[scale=0.1*#1,transform shape]
  \fill[color=myorange] (0,0)--(4,0)--(4,-4)--(0,-4)--cycle;
  \fill[color=myred] (0,0)--(0,3)--(-3,3)--(-3,0)--cycle;
  \fill[color=myyellow] (4,0)--(7,4)--(3,7)--(0,3)--cycle;
  \node[scale=5] at (3.5,3.5) {Exo7};
\end{tikzpicture}}
}



\theoremstyle{definition}
%\newtheorem{proposition}{Proposition}
%\newtheorem{exemple}{Exemple}
%\newtheorem{theoreme}{Théorème}
\newtheorem{lemme}{Lemme}
\newtheorem{corollaire}{Corollaire}
%\newtheorem*{remarque*}{Remarque}
%\newtheorem*{miniexercice}{Mini-exercices}
%\newtheorem{definition}{Définition}




%definition d'un terme
\newcommand{\defi}[1]{{\color{myorange}\textbf{\emph{#1}}}}
\newcommand{\evidence}[1]{{\color{blue}\textbf{\emph{#1}}}}



 %----- Commandes divers ------

\newcommand{\codeinline}[1]{\texttt{#1}}

%%%%%%%%%%%%%%%%%%%%%%%%%%%%%%%%%%%%%%%%%%%%%%%%%%%%%%%%%%%%%
%%%%%%%%%%%%%%%%%%%%%%%%%%%%%%%%%%%%%%%%%%%%%%%%%%%%%%%%%%%%%



\begin{document}

\debuttexte


%%%%%%%%%%%%%%%%%%%%%%%%%%%%%%%%%%%%%%%%%%%%%%%%%%%%%%%%%%%
\diapo

\change

Dans cette première leçon sur les espaces vectoriels

\change

nous commençons par aborder la définition d'un espace vectoriel

\change

et nous verrons ensuite des exemples simples.

\change

On termine par du vocabulaire spécifique aux espaces vectoriels.


%%%%%%%%%%%%%%%%%%%%%%%%%%%%%%%%%%%%%%%%%%%%%%%%%%%%%%%%%%
\diapo

La notion d'espace vectoriel est une structure fondamentale des mathématiques modernes.

\change

Il s'agit de dégager les propriétés communes que partagent des ensembles pourtant très différents.

\change

Par exemple, on peut additionner deux vecteurs du plan, et aussi multiplier un vecteur par un réel 
(pour l'agrandir ou le rétrécir). 

\change

Mais on peut aussi additionner deux fonctions, ou multiplier une fonction
par un réel. 

\change

Même chose avec les polynômes ou les matrices,...

\change

Le  but est d'obtenir des théorèmes généraux qui s'appliqueront aussi bien aux vecteurs du plan, 
de l'espace, aux espaces de fonctions, aux polynômes, aux matrices,...
La contrepartie de cette grande généralité de situations est que la notion 
d'espace vectoriel est difficile à appréhender et vous demandera une quantité conséquente de travail !

%%%%%%%%%%%%%%%%%%%%%%%%%%%%%%%%%%%%%%%%%%%%%%%%%%%%%%%%%%%
\diapo

Essayons d'abord de donner une définition d'espace vectoriel avec des mots :

"Un espace vectoriel est un ensemble formé de vecteurs, de sorte que l'on puisse
additionner (et soustraire) deux vecteurs $u,v$ pour en former un troisième $u+v$ (ou $u-v$)
et aussi afin que l'on puisse multiplier chaque vecteur $u$ d'un facteur $\lambda$ pour obtenir 
un vecteur $\lambda \cdot u$."

\change

Dans le reste de cette diapo et la suivante nous allons étudier la définition exacte.


Un \defi{$\Kk$-espace vectoriel} est un ensemble non vide $E$, sur lequel on a deux lois.

Tout d'abord une loi de composition interne, 
c'est-à-dire une application de $E \times E$ dans $E$ :
à un vecteur $u$ et un vecteur $v$, on associe un nouveau vecteur qui s'appelle $u+v$.

\change

Il y a aussi une loi de composition externe, 
  c'est-à-dire une application de $\Kk \times E$ dans $E$ : 
à un facteur $\lambda$ et à un vecteur $u$, on associe un nouveau vecteur qui s'appelle $\lambda \cdot u$.
[$\lambda$ *point* $u$]

  
Ces deux lois doivent vérifier toute une liste de propriétés, que nous allons maintenant découvrir.

\change 

Enfin dans ce chapitre, $\Kk$ désigne un corps. 
Dans la plupart des exemples, ce sera le corps des réels $\Rr$.

%%%%%%%%%%%%%%%%%%%%%%%%%%%%%%%%%%%%%%%%%%%%%%%%%%%%%%%%%%
\diapo

Les deux lois $+$ et $\cdot$ vérifient les propriétés suivantes :

Les quatre premières propriétés concernent la loi $+$.

$u + v = v + u$ \quad (pour tous $u,v \in E$)

\change

$u + (v+w) = (u+v) +w$ \quad (pour tous $u,v,w \in E$)

\change

Il existe un élément de $E$ que l'on appelle \defi{élément neutre} et que l'on note  $0_E$ 
tel que $u + 0_E = u$ \quad (pour tout $u \in E$)

\change

Pour tout vecteur $u \in E$, il existe un autre vecteur $u'$ tel que $u + u' = 0_E$. 
Ce vecteur $u'$ s'appelle le \defi{symétrique} de $u$ et se note $-u$.

\change

Les propriétés 5 et 6 concernent la loi $\cdot$.

Tout d'abord $1 \cdot u = u$ \quad (pour tout $u \in E$)

\change

et $\lambda \cdot (\mu \cdot u) = (\lambda\mu )\cdot u$ \quad (pour tous $\lambda, \mu \in \Kk$, $u \in E$)

\change

Les propriétés 7 et 8 définissent comment se comportent les lois entre elles :

$\lambda \cdot (u+v) = \lambda \cdot u + \lambda \cdot v$ \quad (pour tous $\lambda \in \Kk$, $u,v \in E$)

\change

$(\lambda + \mu ) \cdot u = \lambda \cdot u + \mu \cdot u$ \quad (pour tous $\lambda,\mu \in \Kk$, $u \in E$)

Cela fait beaucoup d'information d'un seul coup.

Mais vous allez voir que ces axiomes sont assez naturels
et nous reviendrons en détail sur chacune de ces propriétés dans la leçon suivante.

%%%%%%%%%%%%%%%%%%%%%%%%%%%%%%%%%%%%%%%%%%%%%%%%%%%%%%%%%%%
\diapo

Le premier exemple, vous le connaissez depuis longtemps !
$\Rr^2$ est un $\Rr$-espace vectoriel

On pose $E=\Rr^2$ : on considère donc l'ensemble des vecteurs du plan.

Et on pose $\Kk=\Rr$ : on agrandit ou rétrécit les vecteurs par des facteurs réels.

\change


  Un vecteur $u\in E$ est donc ici un
  couple $(x,y)$ avec $x$ réel et $y$ réel. 
  
\change
  
  Ceci s'écrit 
  $$\Rr^2=\big\{(x,y)\mid x \in \Rr, y \in \Rr\big\}.$$
 
\change

Définissons l'addition de deux vecteurs, c'est-à-dire définissons la loi interne :

    Si $(x,y)$ et $(x',y')$ sont deux vecteurs de $\Rr^2$, alors on pose
  $$(x,y)+(x',y')=(x+x',y+y').$$

\change

Définissons maintenant la multiplication par un scalaire, c'est-à-dire définissons la loi externe :

    Si $\lambda$ est un réel et $(x,y)$ est un vecteur de $\Rr^2$, alors on pose :
  $$\lambda \cdot (x,y)=(\lambda x, \lambda y).$$

  
\change

L'élément neutre de la loi interne est le vecteur nul $(0,0)$. 

En effet ajouter le vecteur nul à n'importe quel vecteur, ne change pas le vecteur.

\change

Le symétrique de $(x,y)$ est $(-x,-y)$, que l'on note aussi $-(x,y)$.

En effet $(x,y) + (-x,-y) = (0,0)$
 


%%%%%%%%%%%%%%%%%%%%%%%%%%%%%%%%%%%%%%%%%%%%%%%%%%%%%%%%%%
\diapo

Voici une figure,

chaque point du plan définit un vecteur, par exemple ici le vecteur $u$
et là le vecteur $v$. En mathématique un vecteur à toujours pour origine, l'origine.



Pour nos deux vecteurs $u$, $v$, 
on associe le nouveau vecteur $u+v$,
dans le plan cela ce fait par la construction d'un parallélogramme.

Pour un réel $\lambda$, et un vecteur $u$,
on associe un nouveau vecteur $\lambda \cdot u$.

Ici $u$ est ce vecteur et $\lambda =2$.

$u$ et $\lambda \cdot u$ sont colinéaire, mais n'ont pas la même longueur.

Si $\lambda$ avait été $<1$ alors $\lambda \cdot u$ aurait eu une longueur plus petite que celle de $u$.


L'élément neutre est le vecteur nul.

Et le symétrique de $u$ est ici, noté $-u$.

Il porte bien son nom car c'est le symétrique de $u$ par rapport à l'origine.

C'est un très bon exercice de faire un petit dessin pour chacun des 8 axiomes
qui font de $\Rr^2$ un espace vectoriel.

%%%%%%%%%%%%%%%%%%%%%%%%%%%%%%%%%%%%%%%%%%%%%%%%%%%%%%%%%%%
\diapo

[$x_1$ à $x_n$]

L'exemple suivant généralise le précédent. 
C'est aussi le bon moment pour lire ou regarder le chapitre \og L'espace vectoriel $\Rr^n$ \fg.
  

Soit $n$ un entier supérieur ou égal à $1$.   

On pose encore $\Kk=\Rr$ et cette fois $E=\Rr^n$.

\change

Un vecteur $u\in E$ est donc maintenant un $n$-uplet 
   $(x_1,x_2, \ldots , x_n)$ avec $x_1,x_2, \ldots , x_n$ des réels.

\change

  Si $(x_1, \dots , x_n)$ et $(x'_1, \dots , x'_n)$ sont deux vecteurs de $\Rr^n$, alors :
  on définit l'addition par 
  $$(x_1, \dots , x_n)+(x'_1, \dots , x'_n)=
  (x_1+x'_1, \dots , x_n+x'_n).$$    
  
\change

      Si $\lambda$ est un réel et $(x_1, \dots , x_n)$ est un vecteur de $\Rr^n$, alors :
      on définit la multiplication par un scalaire :
  $$\lambda \cdot (x_1, \dots , x_n)=(\lambda x_1,\dots ,  \lambda x_n).$$

 \change

L'élément neutre de la loi interne est le vecteur nul $(0,0, \dots, 0)$. 

\change

Le symétrique de  $(x_1, \dots , x_n)$ est $(-x_1, \dots , -x_n)$, que l'on note 
$-(x_1, \dots , x_n)$.
  
De manière analogue, pour les nombres complexes, on peut définir le $\Cc$-espace vectoriel $\Cc^n$,  
et plus généralement, pour un corps quelconque, le $\Kk$-espace vectoriel $\Kk^n$.



%%%%%%%%%%%%%%%%%%%%%%%%%%%%%%%%%%%%%%%%%%%%%%%%%%%%%%%%%%
\diapo

Voici un exemple un peu différent : tout plan passant par l'origine dans $\Rr^3$ est un espace vectoriel 

\change

Soient $\Kk=\Rr$ et $E$ un plan passant par l'origine. 

Les opérations $+$ et $\cdot$ sont les opérations habituelles sur les vecteurs. 

\change

Un plan passant par l'origine admet une équation de la forme :
$$ax + by + cz = 0$$
où $a$, $b$ et $c$ sont des réels non tous nuls.  



\change

Un vecteur $u\in E$ est donc un triplet, noté ici comme un vecteur colonne,
$\left(\begin{smallmatrix}x\\ y\\ z\end{smallmatrix}\right)$ tel que
$ax + by + cz = 0$.

\change

Soient $\left(\begin{smallmatrix}x\\ y\\ z\end{smallmatrix}\right)$ et 
$\left(\begin{smallmatrix}x'\\ y'\\ z'\end{smallmatrix}\right)$ deux vecteurs 
appartenant au plan $E$. 

\change

Cela signifie exactement que  
$$\begin{array}{rcl}
a x + b y + c z & = & 0,\\ 
\text{ et } \quad a x' + b y' + c z' & = & 0.  
\end{array}$$

\change 

On en déduit en particulier que 
$$a (x + x') + b(y + y') + c (z + z') = 0.$$

\change

c'est-à-dire que $u+v$ dont les coordonnées sont 
$\left(\begin{smallmatrix}x + x'\\ y + y'\\ z + z'\end{smallmatrix}\right)$ 
appartient lui aussi au plan $E$.

\change


Les autres propriétés sont aussi faciles à vérifier :
par exemple l'élément neutre est $\left(\begin{smallmatrix}0\\ 0\\ 0\end{smallmatrix}\right)$,

il appartient bien à notre plan.

\change

Enfin si $\left(\begin{smallmatrix}x\\ y\\ z\end{smallmatrix}\right)$ appartient  
au plan $E$, alors $a x + b y + c z  =  0$,
que l'on peut réécrire $a(-x)+b(-y)+c(-z)=0$ et ainsi le symétrique 
$-\left(\begin{smallmatrix}x\\ y\\ z\end{smallmatrix}\right)$ appartient aussi au plan.


Attention! Un plan ne contenant pas l'origine n'est pas un espace vectoriel,
car justement il ne contient pas le vecteur nul 
$\left(\begin{smallmatrix}0\\ 0\\ 0\end{smallmatrix}\right)$.



%%%%%%%%%%%%%%%%%%%%%%%%%%%%%%%%%%%%%%%%%%%%%%%%%%%%%%%%%%%
\diapo

Rassemblons les définitions déjà vues et précisons le vocabulaire.
 
On appelle les éléments de $E$ des \defi{vecteurs}.

\change
  
Les éléments de $\Kk$ seront appelés des \defi{scalaires}.

\change

L'\,\defi{élément neutre} $0_E$ s'appelle aussi le \defi{vecteur nul}.
Il ne doit pas être confondu avec l'élément $0$ du corps $\Kk$. 
Cependant lorsqu'il n'y aura pas de risque de confusion, 
  $0_{E}$ sera aussi noté $0$. 
    
\change

Le \defi{symétrique} $-u$ d'un vecteur $u \in E$ s'appelle aussi l'\defi{opposé}.

\change

La loi de composition interne sur $E$ (notée $+$) est appelée couramment 
l'addition et $u+v$ est appelée somme des vecteurs $u$ et $v$. 

\change

La loi de composition externe sur $E$ s'appelle la multiplication par un scalaire. 

%%%%%%%%%%%%%%%%%%%%%%%%%%%%%%%%%%%%%%%%%%%%%%%%%%%%%%%%%%%
\diapo

On termine par la somme de $n$ vecteurs.

\change

Par définition de la structure d'espace vectoriel, 

on a défini l'addition de deux vecteurs, ce qui initialise le processus. 

\change

Il est possible de définir, par récurrence, l'addition de $n$ vecteurs.

Si la somme de $n-1$ vecteurs est définie, alors la somme de $n$ vecteurs 
$v_1,v_2, \ldots, v_n$ est définie par 
$$(v_1+v_2+\cdots+v_{n-1})+v_n.$$  


L'axiome d'associativité de la loi $+$ nous permet de ne pas mettre de
parenthèses dans la somme.  

\change


On note aussi ${\displaystyle \sum_{i=1}^nv_{i}}$ cette somme.

%%%%%%%%%%%%%%%%%%%%%%%%%%%%%%%%%%%%%%%%%%%%%%%%%%%%%%%%%%%
\diapo

Il faut beaucoup de temps et de travail pour comprendre les espaces vectoriels,
mais c'est l'une des notions les plus importantes de la première année.



\end{document}
