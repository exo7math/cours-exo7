
%%%%%%%%%%%%%%%%%% PREAMBULE %%%%%%%%%%%%%%%%%%


\documentclass[12pt]{article}

\usepackage{amsfonts,amsmath,amssymb,amsthm}
\usepackage[utf8]{inputenc}
\usepackage[T1]{fontenc}
\usepackage[francais]{babel}


% packages
\usepackage{amsfonts,amsmath,amssymb,amsthm}
\usepackage[utf8]{inputenc}
\usepackage[T1]{fontenc}
%\usepackage{lmodern}

\usepackage[francais]{babel}
\usepackage{fancybox}
\usepackage{graphicx}

\usepackage{float}

%\usepackage[usenames, x11names]{xcolor}
\usepackage{tikz}
\usepackage{datetime}

\usepackage{mathptmx}
%\usepackage{fouriernc}
%\usepackage{newcent}
\usepackage[mathcal,mathbf]{euler}

%\usepackage{palatino}
%\usepackage{newcent}


% Commande spéciale prompteur

%\usepackage{mathptmx}
%\usepackage[mathcal,mathbf]{euler}
%\usepackage{mathpple,multido}

\usepackage[a4paper]{geometry}
\geometry{top=2cm, bottom=2cm, left=1cm, right=1cm, marginparsep=1cm}

\newcommand{\change}{{\color{red}\rule{\textwidth}{1mm}\\}}

\newcounter{mydiapo}

\newcommand{\diapo}{\newpage
\hfill {\normalsize  Diapo \themydiapo \quad \texttt{[\jobname]}} \\
\stepcounter{mydiapo}}


%%%%%%% COULEURS %%%%%%%%%%

% Pour blanc sur noir :
%\pagecolor[rgb]{0.5,0.5,0.5}
% \pagecolor[rgb]{0,0,0}
% \color[rgb]{1,1,1}



%\DeclareFixedFont{\myfont}{U}{cmss}{bx}{n}{18pt}
\newcommand{\debuttexte}{
%%%%%%%%%%%%% FONTES %%%%%%%%%%%%%
\renewcommand{\baselinestretch}{1.5}
\usefont{U}{cmss}{bx}{n}
\bfseries

% Taille normale : commenter le reste !
%Taille Arnaud
%\fontsize{19}{19}\selectfont

% Taille Barbara
%\fontsize{21}{22}\selectfont

%Taille François
\fontsize{25}{30}\selectfont

%Taille Pascal
%\fontsize{25}{30}\selectfont

%Taille Laura
%\fontsize{30}{35}\selectfont


%\myfont
%\usefont{U}{cmss}{bx}{n}

%\Huge
%\addtolength{\parskip}{\baselineskip}
}


% \usepackage{hyperref}
% \hypersetup{colorlinks=true, linkcolor=blue, urlcolor=blue,
% pdftitle={Exo7 - Exercices de mathématiques}, pdfauthor={Exo7}}


%section
% \usepackage{sectsty}
% \allsectionsfont{\bf}
%\sectionfont{\color{Tomato3}\upshape\selectfont}
%\subsectionfont{\color{Tomato4}\upshape\selectfont}

%----- Ensembles : entiers, reels, complexes -----
\newcommand{\Nn}{\mathbb{N}} \newcommand{\N}{\mathbb{N}}
\newcommand{\Zz}{\mathbb{Z}} \newcommand{\Z}{\mathbb{Z}}
\newcommand{\Qq}{\mathbb{Q}} \newcommand{\Q}{\mathbb{Q}}
\newcommand{\Rr}{\mathbb{R}} \newcommand{\R}{\mathbb{R}}
\newcommand{\Cc}{\mathbb{C}} 
\newcommand{\Kk}{\mathbb{K}} \newcommand{\K}{\mathbb{K}}

%----- Modifications de symboles -----
\renewcommand{\epsilon}{\varepsilon}
\renewcommand{\Re}{\mathop{\text{Re}}\nolimits}
\renewcommand{\Im}{\mathop{\text{Im}}\nolimits}
%\newcommand{\llbracket}{\left[\kern-0.15em\left[}
%\newcommand{\rrbracket}{\right]\kern-0.15em\right]}

\renewcommand{\ge}{\geqslant}
\renewcommand{\geq}{\geqslant}
\renewcommand{\le}{\leqslant}
\renewcommand{\leq}{\leqslant}

%----- Fonctions usuelles -----
\newcommand{\ch}{\mathop{\mathrm{ch}}\nolimits}
\newcommand{\sh}{\mathop{\mathrm{sh}}\nolimits}
\renewcommand{\tanh}{\mathop{\mathrm{th}}\nolimits}
\newcommand{\cotan}{\mathop{\mathrm{cotan}}\nolimits}
\newcommand{\Arcsin}{\mathop{\mathrm{Arcsin}}\nolimits}
\newcommand{\Arccos}{\mathop{\mathrm{Arccos}}\nolimits}
\newcommand{\Arctan}{\mathop{\mathrm{Arctan}}\nolimits}
\newcommand{\Argsh}{\mathop{\mathrm{Argsh}}\nolimits}
\newcommand{\Argch}{\mathop{\mathrm{Argch}}\nolimits}
\newcommand{\Argth}{\mathop{\mathrm{Argth}}\nolimits}
\newcommand{\pgcd}{\mathop{\mathrm{pgcd}}\nolimits} 

\newcommand{\Card}{\mathop{\text{Card}}\nolimits}
\newcommand{\Ker}{\mathop{\text{Ker}}\nolimits}
\newcommand{\id}{\mathop{\text{id}}\nolimits}
\newcommand{\ii}{\mathrm{i}}
\newcommand{\dd}{\mathrm{d}}
\newcommand{\Vect}{\mathop{\text{Vect}}\nolimits}
\newcommand{\Mat}{\mathop{\mathrm{Mat}}\nolimits}
\newcommand{\rg}{\mathop{\text{rg}}\nolimits}
\newcommand{\tr}{\mathop{\text{tr}}\nolimits}
\newcommand{\ppcm}{\mathop{\text{ppcm}}\nolimits}

%----- Structure des exercices ------

\newtheoremstyle{styleexo}% name
{2ex}% Space above
{3ex}% Space below
{}% Body font
{}% Indent amount 1
{\bfseries} % Theorem head font
{}% Punctuation after theorem head
{\newline}% Space after theorem head 2
{}% Theorem head spec (can be left empty, meaning ‘normal’)

%\theoremstyle{styleexo}
\newtheorem{exo}{Exercice}
\newtheorem{ind}{Indications}
\newtheorem{cor}{Correction}


\newcommand{\exercice}[1]{} \newcommand{\finexercice}{}
%\newcommand{\exercice}[1]{{\tiny\texttt{#1}}\vspace{-2ex}} % pour afficher le numero absolu, l'auteur...
\newcommand{\enonce}{\begin{exo}} \newcommand{\finenonce}{\end{exo}}
\newcommand{\indication}{\begin{ind}} \newcommand{\finindication}{\end{ind}}
\newcommand{\correction}{\begin{cor}} \newcommand{\fincorrection}{\end{cor}}

\newcommand{\noindication}{\stepcounter{ind}}
\newcommand{\nocorrection}{\stepcounter{cor}}

\newcommand{\fiche}[1]{} \newcommand{\finfiche}{}
\newcommand{\titre}[1]{\centerline{\large \bf #1}}
\newcommand{\addcommand}[1]{}
\newcommand{\video}[1]{}

% Marge
\newcommand{\mymargin}[1]{\marginpar{{\small #1}}}



%----- Presentation ------
\setlength{\parindent}{0cm}

%\newcommand{\ExoSept}{\href{http://exo7.emath.fr}{\textbf{\textsf{Exo7}}}}

\definecolor{myred}{rgb}{0.93,0.26,0}
\definecolor{myorange}{rgb}{0.97,0.58,0}
\definecolor{myyellow}{rgb}{1,0.86,0}

\newcommand{\LogoExoSept}[1]{  % input : echelle
{\usefont{U}{cmss}{bx}{n}
\begin{tikzpicture}[scale=0.1*#1,transform shape]
  \fill[color=myorange] (0,0)--(4,0)--(4,-4)--(0,-4)--cycle;
  \fill[color=myred] (0,0)--(0,3)--(-3,3)--(-3,0)--cycle;
  \fill[color=myyellow] (4,0)--(7,4)--(3,7)--(0,3)--cycle;
  \node[scale=5] at (3.5,3.5) {Exo7};
\end{tikzpicture}}
}



\theoremstyle{definition}
%\newtheorem{proposition}{Proposition}
%\newtheorem{exemple}{Exemple}
%\newtheorem{theoreme}{Théorème}
\newtheorem{lemme}{Lemme}
\newtheorem{corollaire}{Corollaire}
%\newtheorem*{remarque*}{Remarque}
%\newtheorem*{miniexercice}{Mini-exercices}
%\newtheorem{definition}{Définition}




%definition d'un terme
\newcommand{\defi}[1]{{\color{myorange}\textbf{\emph{#1}}}}
\newcommand{\evidence}[1]{{\color{blue}\textbf{\emph{#1}}}}



 %----- Commandes divers ------

\newcommand{\codeinline}[1]{\texttt{#1}}

%%%%%%%%%%%%%%%%%%%%%%%%%%%%%%%%%%%%%%%%%%%%%%%%%%%%%%%%%%%%%
%%%%%%%%%%%%%%%%%%%%%%%%%%%%%%%%%%%%%%%%%%%%%%%%%%%%%%%%%%%%%


\begin{document}

\debuttexte


%%%%%%%%%%%%%%%%%%%%%%%%%%%%%%%%%%%%%%%%%%%%%%%%%%%%%%%%%%%
\diapo

Bonjour.

Nous allons faire une courte introduction au calcul formel et commencer à nous familiariser avec le logiciel Sage.

\change

\change

Nous commencerons par des exemples élémentaires,

\change

puis nous décrirons brièvement ce qu'est le calcul formel,

\change

et nous le comparerons au calcul numérique.

\change

Nous verrons ensuite quelques possibilités graphiques de Sage,

\change

puis nous tenterons de cerner les limites du calcul formel.


%%%%%%%%%%%%%%%%%%%%%%%%%%%%%%%%%%%%%%%%%%%%%%%%%%%%%%%%%%%
\diapo

Dans un premier temps, servons nous de Sage comme d'une calculatrice. 

Effectuez les trois calculs élémentaires suivants.

Arrêtez ici la vidéo, lancez le logiciel Sage et faites ces calculs.

Ensuite continuer la vidéo.

PAUSE 3s

\change

Nous pouvons taper les formules mathématiques presque telles quelles à 
l'invite de commande de Sage. 

\change
La machine renvoie le résultat.

La multiplication se note par une étoile, 

la puissance avec un accent circonflexe ou deux étoiles successives.


\change

\change
Sage fait des calculs exacts, 
il simplifie $\frac{22}{33}$ en 
  $\frac{2}{3}$, contrairement à ce que faisait les anciennes calculatrices 
  qui auraient affiché $0.6666\ldots$
 
\change 

\change
Sage manipule les fonctions et les constantes usuelles. 
 
Afin d'alléger ls présentations à venir, nous ommetrons d'afficher l'invite de commande Sage.

%%%%%%%%%%%%%%%%%%%%%%%%%%%%%%%%%%%%%%%%%%%%%%%%%%%%%%%%%%%
\diapo


L'utilisation d'un système de calcul symbolique 
conduit le mathématicien que vous êtes à un type nouveau type 
de démarche  : 

l'expérimentation !

Nous pouvons décrire cette démarche comme suit : 

\change

J'ai une question, un énoncé, un problème mathématique.
  
\change
Je le traduis dans le langage de la machine et j'écris un algorithme afin de pouvoir faire des expériences.

\change
Les résultats permettent de valider la question si celle-ci est close ou bien si elle est ouverte de proposer une conjecture.
%Je valide ma question. C'est à dire que j'énonce
%une conjecture sur la base des résultats de mes expériences.

\change
Et bien sûr je prouve la conjecture ou démontre le résultat soit seul soit aidé de nouveau par la machine.

Notez bien que le calcul formel va nous être utile pour la partie
expérimentale, mais aussi pour la partie preuve !



%%%%%%%%%%%%%%%%%%%%%%%%%%%%%%%%%%%%%%%%%%%%%%%%%%%%%%%%%%%
\diapo

Appliquons cette démarche %à une nouveau tp.
aux deux questions suivantes.

Que vaut le nombre complexe $(1-\ii)^k$, en fonction 
de l'entier $k\ge 0$ ?

Les nombres de Fermat qui sont de la forme 
$F_n = 2^{(2^n)}+1$ sont-ils tous premiers ?


Encore une fois arrêtez ici la vidéo et appliquez vous-même la démarche.


PAUSE 3s

\change

Pour la première, nous pouvons faire des calculs avec les nombres complexes.
En posant $z= 1 - \ii$, on calcule pour $k=0,1,2$ jusqu'à $9$...

\change 
...$z^k$, sa partie réelle, sa partie imaginaire.

\change
On peut aussi calculer son module et son argument

Pour l'instant nous ne nous attardons pas sur la syntaxe, nous la reverrons en détail. 

Cependant, la lecture du code est ici suffisamment intelligible.

\change
Les résultats expérimentaux suggèrent que
$z_k = \sqrt{2}^k e^{-\frac{k\ii\pi}{4}}$. 

\change
Ici la preuve est très facile. Une fois que l'on a écrit $z$
sous la forme trigonométrique 
$z = \sqrt{2} e^{-\frac{\ii\pi}{4}}$ le résultat est immédiat.



%%%%%%%%%%%%%%%%%%%%%%%%%%%%%%%%%%%%%%%%%%%%%%%%%%%%%%%%%%%
\diapo

En ce qui concerne la seconde question. 

Pierre de Fermat pensait que tous les nombres de la forme 
$F_n = 2^{2^n}+1$ étaient des nombres premiers.

\change
Cent ans plus tard, Euler a calculé $F_5$ et a montré qu'il n'est pas un 
nombre premier. De nos jours, ce résultat s'obtient instantanément
avec deux lignes de code.



%%%%%%%%%%%%%%%%%%%%%%%%%%%%%%%%%%%%%%%%%%%%%%%%%%%%%%%%%%%
\diapo

Même si Sage est conçu pour le calcul symbolique, 
il offre également la possibilité d'effectuer des calculs numériques. 
Bien que non exact, le calcul numérique possède plusieurs avantages :
il est plus rapide, souvent suffisant pour les applications 
pratiques et peut aussi être plus lisible.  
Voyons ceci sur un exemple de calcul de limite 
d'une suite définie par récurrence.


\change

On écrit un algorithme pour calculer les termes un par un.
$u_0 = 1$.

\change
On initialise une variable $u$ à $1$.

\change
Puis on calcule successivement les termes.

\change
On remplace $u$ par $\sqrt{u+1}$.

C'est la formule de récurrence $u_{n+1} = \sqrt{1+u_n}$.

Et à chaque étape on affiche $u$.

Voyons le résultat de l'éxécution.

\change

\change

\change

\change

On trouve 
$$u_1 = \sqrt{2} \qquad u_2 = \sqrt{1+\sqrt{2}}
$$
$$ u_3 = \sqrt{1+\sqrt{1+\sqrt{2}}} \qquad 
u_4 = \sqrt{1+\sqrt{1+\sqrt{1+\sqrt{2}}}}$$

Ceci ne nous n'éclaire pas vraiment sur le comportement de la suite.

\change
Demandons maintenant à Sage de calculer, en plus de l'expression formelle, une approximation numérique de chacun des termes.

Ce qui donne les résultats suivants.

\change

\change

\change

\change

On peut alors émettre plusieurs conjectures : 

la suite est croissante, 

elle est majorée par $2$ 

et donc converge. 


\change

En poussant les calculs un peu plus loin, une approximation de la limite est
$1.618033\ldots$. 

\change
Les plus observateurs auront reconnu une approximation 
du nombre d'or $\phi = \frac{1+\sqrt{5}}{2}$. 

Nous renvoyons au cours d'analyse pour la preuve que 
cette suite $(u_n)$ converge effectivement vers $\phi$.


%%%%%%%%%%%%%%%%%%%%%%%%%%%%%%%%%%%%%%%%%%%%%%%%%%%%%%%%%%%
\diapo

Continuons notre tour d'horizon et l'étude des fonctions. 

Nous souhiatons préciser le comportement en $0$ de la fonction 
$f$ qui à $x$ associe $\sin(x) \cdot \exp(x)$ puis tracer son graphe, 
ainsi que celui de ses approximations en $0$.

\change
Commençons par définir la fonction $f$ à l'aide de 
la commande 

\codeinline{f = sin(x)*exp(x)}


\change
On utilise ensuite un outil de Sage

\codeinline{taylor(f,x,0,n)} 

qui permet d'obtenir le développement limité de  $f$ en $x=0$ à l'ordre $n$, 

\change
ce qui donne par exemple ici pour l'ordre $3$ :

  $$f(x) = x + x^2 + \frac13 x^3+ \epsilon(x) x^3$$

\change
Voici donc les trois premiers polynômes de Taylor.

\change
La commande \codeinline{plot(f,(a,b))} permet de 
tracer le graphe de $f$ sur l'intervalle $[a,b]$.


%%%%%%%%%%%%%%%%%%%%%%%%%%%%%%%%%%%%%%%%%%%%%%%%%%%%%%%%%%%
\diapo

Voici le tracé des graphes.

Ici le graphe de $f$ est en noir.

On note que 
\begin{itemize}
  \item Les polynômes de Taylor sont une bonne approximation de la fonction 
  au voisinage d'un point. 
  
  \item Plus l'ordre du développement limité est élevé, meilleure est l'approximation. 

  L'approximation linéaire en rouge est améliorée par 

  l'approximation de degré $2$ en vert, 

  elle même améliorée par l'approximation de degré $3$ en bleu.

  \item Enfin, l'approximation est uniquement \emph{locale} : 
  loin du point considéré (ici l'origine)
  les polynômes de Taylor n'approximent plus du tout la fonction.
\end{itemize}



%%%%%%%%%%%%%%%%%%%%%%%%%%%%%%%%%%%%%%%%%%%%%%%%%%%%%%%%%%%
\diapo



Il est facile de tracer une grande variété de graphiques.


On peut par exemple tracer une courbe de Lissajous
d'équation paramétrique $\big(\cos(3t),\sin(3t)\big)$

\change
La commande est \codeinline{parametric\_plot}

\change

\change
On peut visualiser des surfaces en dimension $3$

Comme ici la surface qui est le graphe 
de la fonction de deux variables $cos(x*y)$

\change
Attention ! Il faut, au préalable, avoir défini les variables utilisées comme de vraies variables : \codeinline{var('t')}
et \codeinline{var('x,y')}. (En fait seule la variable $x$ est définie par défaut dans Sage.)


%%%%%%%%%%%%%%%%%%%%%%%%%%%%%%%%%%%%%%%%%%%%%%%%%%%%%%%%%%%
\diapo

Abordons maintenant les limites de ce qu'on peut faire 
avec un logiciel de calcul formel. 


Tentons d'une part de calculer les solutions réelles de 
l'équation $x^k-x-1=0$, 

et d'autre part de montrer que les trois expressions 
ci-dessous sont nulles.

\change

Le logiciel n'est pas responsable de la première limitation, celle-ci est propre aux mathématiques : 
il est impossible d'écrire explicitement les solutions de toutes les équations.

Pour $x^2-x-1=0$, la commande
\codeinline{solve(x2-x-1==0,x)} renvoie les deux solutions
$\frac{1+\sqrt5}{2}$ et $\frac{1-\sqrt5}{2}$.

\change
Par contre la commande \codeinline{solve(x5-x-1==0,x)} 
ne renvoie pas les solutions, 
mais renvoie l'équation de départ.

Ce n'est pas ici un problème dû à Sage.
En effet, il n'est mathématiquement pas possible d'exprimer 
la solution réelle de $x^5-x-1=0$ 
à l'aide de radicaux. 
C'est seulement possible jusqu'au degré $4$.

Par contre Sage pourrait nous fournir une approximation 
de la ou des solutions réelles.

\change
En ce qui concerne les expressions, Sage simplifie la première sans problème. 

Pour les deux autres, par contre, il faut l'aider un peu.

En effet, chaque fois qu'un utilisateur interroge une machine, 
il doit être très attentif à la définition des objets, 
être informé de leur représentation interne ainsi que 
de leur contexte d'utilisation. 

De plus, la demande doit être la plus précise et la plus explicite.

\change

C'est à l'aide de la commande \codeinline{expand} 
que la deuxième expression peut être développé ;
ce qui permet de vérifier l'identité.

\change

Enfin, après avoir défini l'expression $\sin^2 x+\cos^2 x-1$, 
il est nécessaure de demander explicitement à Sage de 
simplifier l'expression trigonométrique pour obtenir $0$.



% %%%%%%%%%%%%%%%%%%%%%%%%%%%%%%%%%%%%%%%%%%%%%%%%%%%%%%%%%%%
% \diapo
% 
%  Il n'est pas du tout évident pour un ordinateur de reconnaître les identités formelles  comme $(a+b)^2 = a^2+2ab+b^2$.
% 
%  Lorsqu'il y a plusieurs écritures d'une même expression,  il n'est pas non plus évident pour l'ordinateur de savoir quelle forme est la plus adaptée à l'utilisateur.
% 
%   Pour représenter  la difficulté à identifier deux expressions, voici 
% une image de la façon dont les expressions $(a+b)^2$ et $a^2+2ab+b^2$ sont stockées  dans le logiciel.
% 

% 
% %%%%%%%%%%%%%%%%%%%%%%%%%%%%%%%%%%%%%%%%%%%%%%%%%%%%%%%%%%%
% \diapo
% 
% Concluons cette première partie par des remarques générales.
% 
% D'abord, il est possible de consulter la documentation de Sage en faisant suivre une commande d'un point d'interrogation ou bien en utilisant la commande \codeinline{help}.
% 
% \change
% 
% Ensuite, on peut utiliser Sage d'au moins deux manières : soit en ligne de commande,
% 
% \change soit dans un navigateur grâce à la commande \codeinline{notebook()}.



%%%%%%%%%%%%%%%%%%%%%%%%%%%%%%%%%%%%%%%%%%%%%%%%%%%%%%%%%%%
\diapo

Avant de terminer, quelques premières précisions sur le langage et la syntaxe de Sage.

La syntaxe de Sage est celle de Python,

mais il n'est pas nécessaire de connaître le langage Python.


\change

Reprenons le code utilisé pour l'étude des nombres de Fermat.

\change


Voici les éléments à retenir : 

 La boucle \codeinline{for n in range(N):} 
 signifie que l'indice $n$ parcourt les entiers de $0$ à  $N-1$.
 
  
\change

Le bloc d'instructions suivant, c'est-à-dire la ligne commençant par  
\codeinline{print} n,
  est donc exécuté successivement pour $n=0,1,\ldots,N-1$.
  
\change

Les espaces en début de ligne, (autrement dit \emph{l'indentation}) 
sont essentiels car ils permettent de caractériser le bloc d'instructions sui sera évalué à plusieurs reprise pour les différentes valeurs de $n$.

Même "invisibles", ces espaces font en quelque sorte partie du code.

\end{document}
