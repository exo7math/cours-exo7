
%%%%%%%%%%%%%%%%%% PREAMBULE %%%%%%%%%%%%%%%%%%

\documentclass[aspectratio=169,utf8]{beamer}
%\documentclass[aspectratio=169,handout]{beamer}

\usetheme{Boadilla}
%\usecolortheme{seahorse}
\usecolortheme[RGB={245,66,24}]{structure}
\useoutertheme{infolines}

% packages
\usepackage{amsfonts,amsmath,amssymb,amsthm}
\usepackage[utf8]{inputenc}
\usepackage[T1]{fontenc}
\usepackage{lmodern}

\usepackage[francais]{babel}
\usepackage{fancybox}
\usepackage{graphicx}

\usepackage{float}
\usepackage{xfrac}

%\usepackage[usenames, x11names]{xcolor}
\usepackage{tikz}
\usepackage{pgfplots}
\usepackage{datetime}



%-----  Package unités -----
\usepackage{siunitx}
\sisetup{locale = FR,detect-all,per-mode = symbol}

%\usepackage{mathptmx}
%\usepackage{fouriernc}
%\usepackage{newcent}
%\usepackage[mathcal,mathbf]{euler}

%\usepackage{palatino}
%\usepackage{newcent}
% \usepackage[mathcal,mathbf]{euler}



% \usepackage{hyperref}
% \hypersetup{colorlinks=true, linkcolor=blue, urlcolor=blue,
% pdftitle={Exo7 - Exercices de mathématiques}, pdfauthor={Exo7}}


%section
% \usepackage{sectsty}
% \allsectionsfont{\bf}
%\sectionfont{\color{Tomato3}\upshape\selectfont}
%\subsectionfont{\color{Tomato4}\upshape\selectfont}

%----- Ensembles : entiers, reels, complexes -----
\newcommand{\Nn}{\mathbb{N}} \newcommand{\N}{\mathbb{N}}
\newcommand{\Zz}{\mathbb{Z}} \newcommand{\Z}{\mathbb{Z}}
\newcommand{\Qq}{\mathbb{Q}} \newcommand{\Q}{\mathbb{Q}}
\newcommand{\Rr}{\mathbb{R}} \newcommand{\R}{\mathbb{R}}
\newcommand{\Cc}{\mathbb{C}} 
\newcommand{\Kk}{\mathbb{K}} \newcommand{\K}{\mathbb{K}}

%----- Modifications de symboles -----
\renewcommand{\epsilon}{\varepsilon}
\renewcommand{\Re}{\mathop{\text{Re}}\nolimits}
\renewcommand{\Im}{\mathop{\text{Im}}\nolimits}
%\newcommand{\llbracket}{\left[\kern-0.15em\left[}
%\newcommand{\rrbracket}{\right]\kern-0.15em\right]}

\renewcommand{\ge}{\geqslant}
\renewcommand{\geq}{\geqslant}
\renewcommand{\le}{\leqslant}
\renewcommand{\leq}{\leqslant}
\renewcommand{\epsilon}{\varepsilon}

%----- Fonctions usuelles -----
\newcommand{\ch}{\mathop{\text{ch}}\nolimits}
\newcommand{\sh}{\mathop{\text{sh}}\nolimits}
\renewcommand{\tanh}{\mathop{\text{th}}\nolimits}
\newcommand{\cotan}{\mathop{\text{cotan}}\nolimits}
\newcommand{\Arcsin}{\mathop{\text{arcsin}}\nolimits}
\newcommand{\Arccos}{\mathop{\text{arccos}}\nolimits}
\newcommand{\Arctan}{\mathop{\text{arctan}}\nolimits}
\newcommand{\Argsh}{\mathop{\text{argsh}}\nolimits}
\newcommand{\Argch}{\mathop{\text{argch}}\nolimits}
\newcommand{\Argth}{\mathop{\text{argth}}\nolimits}
\newcommand{\pgcd}{\mathop{\text{pgcd}}\nolimits} 


%----- Commandes divers ------
\newcommand{\ii}{\mathrm{i}}
\newcommand{\dd}{\text{d}}
\newcommand{\id}{\mathop{\text{id}}\nolimits}
\newcommand{\Ker}{\mathop{\text{Ker}}\nolimits}
\newcommand{\Card}{\mathop{\text{Card}}\nolimits}
\newcommand{\Vect}{\mathop{\text{Vect}}\nolimits}
\newcommand{\Mat}{\mathop{\text{Mat}}\nolimits}
\newcommand{\rg}{\mathop{\text{rg}}\nolimits}
\newcommand{\tr}{\mathop{\text{tr}}\nolimits}


%----- Structure des exercices ------

\newtheoremstyle{styleexo}% name
{2ex}% Space above
{3ex}% Space below
{}% Body font
{}% Indent amount 1
{\bfseries} % Theorem head font
{}% Punctuation after theorem head
{\newline}% Space after theorem head 2
{}% Theorem head spec (can be left empty, meaning ‘normal’)

%\theoremstyle{styleexo}
\newtheorem{exo}{Exercice}
\newtheorem{ind}{Indications}
\newtheorem{cor}{Correction}


\newcommand{\exercice}[1]{} \newcommand{\finexercice}{}
%\newcommand{\exercice}[1]{{\tiny\texttt{#1}}\vspace{-2ex}} % pour afficher le numero absolu, l'auteur...
\newcommand{\enonce}{\begin{exo}} \newcommand{\finenonce}{\end{exo}}
\newcommand{\indication}{\begin{ind}} \newcommand{\finindication}{\end{ind}}
\newcommand{\correction}{\begin{cor}} \newcommand{\fincorrection}{\end{cor}}

\newcommand{\noindication}{\stepcounter{ind}}
\newcommand{\nocorrection}{\stepcounter{cor}}

\newcommand{\fiche}[1]{} \newcommand{\finfiche}{}
\newcommand{\titre}[1]{\centerline{\large \bf #1}}
\newcommand{\addcommand}[1]{}
\newcommand{\video}[1]{}

% Marge
\newcommand{\mymargin}[1]{\marginpar{{\small #1}}}

\def\noqed{\renewcommand{\qedsymbol}{}}


%----- Presentation ------
\setlength{\parindent}{0cm}

%\newcommand{\ExoSept}{\href{http://exo7.emath.fr}{\textbf{\textsf{Exo7}}}}

\definecolor{myred}{rgb}{0.93,0.26,0}
\definecolor{myorange}{rgb}{0.97,0.58,0}
\definecolor{myyellow}{rgb}{1,0.86,0}

\newcommand{\LogoExoSept}[1]{  % input : echelle
{\usefont{U}{cmss}{bx}{n}
\begin{tikzpicture}[scale=0.1*#1,transform shape]
  \fill[color=myorange] (0,0)--(4,0)--(4,-4)--(0,-4)--cycle;
  \fill[color=myred] (0,0)--(0,3)--(-3,3)--(-3,0)--cycle;
  \fill[color=myyellow] (4,0)--(7,4)--(3,7)--(0,3)--cycle;
  \node[scale=5] at (3.5,3.5) {Exo7};
\end{tikzpicture}}
}


\newcommand{\debutmontitre}{
  \author{} \date{} 
  \thispagestyle{empty}
  \hspace*{-10ex}
  \begin{minipage}{\textwidth}
    \titlepage  
  \vspace*{-2.5cm}
  \begin{center}
    \LogoExoSept{2.5}
  \end{center}
  \end{minipage}

  \vspace*{-0cm}
  
  % Astuce pour que le background ne soit pas discrétisé lors de la conversion pdf -> png
\begin{tikzpicture}
        \fill[opacity=0,green!60!black] (0,0)--++(0,0)--++(0,0)--++(0,0)--cycle; 
\end{tikzpicture}

% toc S'affiche trop tot :
% \tableofcontents[hideallsubsections, pausesections]
}

\newcommand{\finmontitre}{
  \end{frame}
  \setcounter{framenumber}{0}
} % ne marche pas pour une raison obscure

%----- Commandes supplementaires ------

% \usepackage[landscape]{geometry}
% \geometry{top=1cm, bottom=3cm, left=2cm, right=10cm, marginparsep=1cm
% }
% \usepackage[a4paper]{geometry}
% \geometry{top=2cm, bottom=2cm, left=2cm, right=2cm, marginparsep=1cm
% }

%\usepackage{standalone}


% New command Arnaud -- november 2011
\setbeamersize{text margin left=24ex}
% si vous modifier cette valeur il faut aussi
% modifier le decalage du titre pour compenser
% (ex : ici =+10ex, titre =-5ex

\theoremstyle{definition}
%\newtheorem{proposition}{Proposition}
%\newtheorem{exemple}{Exemple}
%\newtheorem{theoreme}{Théorème}
%\newtheorem{lemme}{Lemme}
%\newtheorem{corollaire}{Corollaire}
%\newtheorem*{remarque*}{Remarque}
%\newtheorem*{miniexercice}{Mini-exercices}
%\newtheorem{definition}{Définition}

% Commande tikz
\usetikzlibrary{calc}
\usetikzlibrary{patterns,arrows}
\usetikzlibrary{matrix}
\usetikzlibrary{fadings} 

%definition d'un terme
\newcommand{\defi}[1]{{\color{myorange}\textbf{\emph{#1}}}}
\newcommand{\evidence}[1]{{\color{blue}\textbf{\emph{#1}}}}
\newcommand{\assertion}[1]{\emph{\og#1\fg}}  % pour chapitre logique
%\renewcommand{\contentsname}{Sommaire}
\renewcommand{\contentsname}{}
\setcounter{tocdepth}{2}



%------ Figures ------

\def\myscale{1} % par défaut 
\newcommand{\myfigure}[2]{  % entrée : echelle, fichier figure
\def\myscale{#1}
\begin{center}
\footnotesize
{#2}
\end{center}}


%------ Encadrement ------

\usepackage{fancybox}


\newcommand{\mybox}[1]{
\setlength{\fboxsep}{7pt}
\begin{center}
\shadowbox{#1}
\end{center}}

\newcommand{\myboxinline}[1]{
\setlength{\fboxsep}{5pt}
\raisebox{-10pt}{
\shadowbox{#1}
}
}

%--------------- Commande beamer---------------
\newcommand{\beameronly}[1]{#1} % permet de mettre des pause dans beamer pas dans poly


\setbeamertemplate{navigation symbols}{}
\setbeamertemplate{footline}  % tiré du fichier beamerouterinfolines.sty
{
  \leavevmode%
  \hbox{%
  \begin{beamercolorbox}[wd=.333333\paperwidth,ht=2.25ex,dp=1ex,center]{author in head/foot}%
    % \usebeamerfont{author in head/foot}\insertshortauthor%~~(\insertshortinstitute)
    \usebeamerfont{section in head/foot}{\bf\insertshorttitle}
  \end{beamercolorbox}%
  \begin{beamercolorbox}[wd=.333333\paperwidth,ht=2.25ex,dp=1ex,center]{title in head/foot}%
    \usebeamerfont{section in head/foot}{\bf\insertsectionhead}
  \end{beamercolorbox}%
  \begin{beamercolorbox}[wd=.333333\paperwidth,ht=2.25ex,dp=1ex,right]{date in head/foot}%
    % \usebeamerfont{date in head/foot}\insertshortdate{}\hspace*{2em}
    \insertframenumber{} / \inserttotalframenumber\hspace*{2ex} 
  \end{beamercolorbox}}%
  \vskip0pt%
}


\definecolor{mygrey}{rgb}{0.5,0.5,0.5}
\setlength{\parindent}{0cm}
%\DeclareTextFontCommand{\helvetica}{\fontfamily{phv}\selectfont}

% background beamer
\definecolor{couleurhaut}{rgb}{0.85,0.9,1}  % creme
\definecolor{couleurmilieu}{rgb}{1,1,1}  % vert pale
\definecolor{couleurbas}{rgb}{0.85,0.9,1}  % blanc
\setbeamertemplate{background canvas}[vertical shading]%
[top=couleurhaut,middle=couleurmilieu,midpoint=0.4,bottom=couleurbas] 
%[top=fondtitre!05,bottom=fondtitre!60]



\makeatletter
\setbeamertemplate{theorem begin}
{%
  \begin{\inserttheoremblockenv}
  {%
    \inserttheoremheadfont
    \inserttheoremname
    \inserttheoremnumber
    \ifx\inserttheoremaddition\@empty\else\ (\inserttheoremaddition)\fi%
    \inserttheorempunctuation
  }%
}
\setbeamertemplate{theorem end}{\end{\inserttheoremblockenv}}

\newenvironment{theoreme}[1][]{%
   \setbeamercolor{block title}{fg=structure,bg=structure!40}
   \setbeamercolor{block body}{fg=black,bg=structure!10}
   \begin{block}{{\bf Th\'eor\`eme }#1}
}{%
   \end{block}%
}


\newenvironment{proposition}[1][]{%
   \setbeamercolor{block title}{fg=structure,bg=structure!40}
   \setbeamercolor{block body}{fg=black,bg=structure!10}
   \begin{block}{{\bf Proposition }#1}
}{%
   \end{block}%
}

\newenvironment{corollaire}[1][]{%
   \setbeamercolor{block title}{fg=structure,bg=structure!40}
   \setbeamercolor{block body}{fg=black,bg=structure!10}
   \begin{block}{{\bf Corollaire }#1}
}{%
   \end{block}%
}

\newenvironment{mydefinition}[1][]{%
   \setbeamercolor{block title}{fg=structure,bg=structure!40}
   \setbeamercolor{block body}{fg=black,bg=structure!10}
   \begin{block}{{\bf Définition} #1}
}{%
   \end{block}%
}

\newenvironment{lemme}[0]{%
   \setbeamercolor{block title}{fg=structure,bg=structure!40}
   \setbeamercolor{block body}{fg=black,bg=structure!10}
   \begin{block}{\bf Lemme}
}{%
   \end{block}%
}

\newenvironment{remarque}[1][]{%
   \setbeamercolor{block title}{fg=black,bg=structure!20}
   \setbeamercolor{block body}{fg=black,bg=structure!5}
   \begin{block}{Remarque #1}
}{%
   \end{block}%
}


\newenvironment{exemple}[1][]{%
   \setbeamercolor{block title}{fg=black,bg=structure!20}
   \setbeamercolor{block body}{fg=black,bg=structure!5}
   \begin{block}{{\bf Exemple }#1}
}{%
   \end{block}%
}


\newenvironment{miniexercice}[0]{%
   \setbeamercolor{block title}{fg=structure,bg=structure!20}
   \setbeamercolor{block body}{fg=black,bg=structure!5}
   \begin{block}{Mini-exercices}
}{%
   \end{block}%
}


\newenvironment{tp}[0]{%
   \setbeamercolor{block title}{fg=structure,bg=structure!40}
   \setbeamercolor{block body}{fg=black,bg=structure!10}
   \begin{block}{\bf Travaux pratiques}
}{%
   \end{block}%
}
\newenvironment{exercicecours}[1][]{%
   \setbeamercolor{block title}{fg=structure,bg=structure!40}
   \setbeamercolor{block body}{fg=black,bg=structure!10}
   \begin{block}{{\bf Exercice }#1}
}{%
   \end{block}%
}
\newenvironment{algo}[1][]{%
   \setbeamercolor{block title}{fg=structure,bg=structure!40}
   \setbeamercolor{block body}{fg=black,bg=structure!10}
   \begin{block}{{\bf Algorithme}\hfill{\color{gray}\texttt{#1}}}
}{%
   \end{block}%
}


\setbeamertemplate{proof begin}{
   \setbeamercolor{block title}{fg=black,bg=structure!20}
   \setbeamercolor{block body}{fg=black,bg=structure!5}
   \begin{block}{{\footnotesize Démonstration}}
   \footnotesize
   \smallskip}
\setbeamertemplate{proof end}{%
   \end{block}}
\setbeamertemplate{qed symbol}{\openbox}


\makeatother
\usecolortheme[RGB={192,41,0}]{structure}

% Commande spécifique à ce chapitre
\newcommand{\Sage}{\texttt{Sage}}

\usepackage{textcomp}

\usepackage{listings}
\lstset{
  upquote=true,
  columns=flexible,
  keepspaces=true,
  basicstyle=\ttfamily,
  commentstyle=\color{gray},
  language=Python,
  showstringspaces=false,
  aboveskip=0em,  
  belowskip=0em,
  escapeinside=||,
  breaklines=true,
  postbreak=\raisebox{0ex}[0ex][0ex]{\qquad\ensuremath{\color{red}\hookrightarrow\space}},
}

\lstset{
  literate={é}{{\'e}}1
           {è}{{\`e}}1
           {à}{{\`a}}1
}

\newcommand{\codeinline}[1]{\lstinline!#1!}


%%%%%%%%%%%%%%%%%%%%%%%%%%%%%%%%%%%%%%%%%%%%%%%%%%%%%%%%%%%%%
%%%%%%%%%%%%%%%%%%%%%%%%%%%%%%%%%%%%%%%%%%%%%%%%%%%%%%%%%%%%%


\begin{document}

\renewcommand*{\theenumii}{\alph{enumii}}
%\setbeamertemplate{enumerate subitem}{\alph{enumii}}
    
\title{{\bf Calcul formel}}
\subtitle{Suites récurrentes et preuves formelles}

\begin{frame}
  
  \debutmontitre

  \pause

{\footnotesize
\hfill
\setbeamercovered{transparent=50}
\begin{minipage}{0.6\textwidth}
  \begin{itemize}
    \item<3-> La suite de Fibonacci
    \item<4-> L'identité de Cassini
  \end{itemize}
\end{minipage}
}

\end{frame}

\setcounter{framenumber}{0}



%%%%%%%%%%%%%%%%%%%%%%%%%%%%%%%%%%%%%%%%%%%%%%%%%%%%%%%%%%%%%%%%
\section{Motivation}

\begin{frame}


\evidence{Une preuve}
\pause

\begin{itemize}
  \item Deux variables $a$ et $b$ : \codeinline{var('a,b')}
  \pause
  \item \codeinline{expand( (a+b)^2-a^2-2*a*b-b^2 )} renvoie $0$
  \pause
  \item Preuve par ordinateur de l'identité $$(a+b)^2 = a^2+2ab+b^2$$
  \pause
  \item \Sage\ sait manipuler les expressions algébriques
\end{itemize}

\bigskip
\pause

\evidence{Démarche}
\pause

\begin{enumerate}
  \item Je me pose des questions
  \pause
  \item J'écris un algorithme pour tenter d'y répondre
  \pause
  \item Je trouve une conjecture sur la base des résultats expérimentaux
  \pause
  \item Je prouve la conjecture
\end{enumerate}
\end{frame}

%%%%%%%%%%%%%%%%%%%%%%%%%%%%%%%%%%%%%%%%%%%%%%%%%%%%%%%%%%%%%%%%
\section{La suite de Fibonacci}

\begin{frame}
\evidence{La suite de Fibonacci}
$$F_0 = 0 \qquad F_1 = 1 \qquad F_{n} = F_{n-1} + F_{n-2} \quad \text{ pour } n\ge 2$$
  \vspace*{-2ex}
\pause
\begin{tp}
\begin{enumerate}
  \item Calculer les $10$ premiers termes de la suite de Fibonacci.
  \item \textbf{Recherche d'une conjecture.} On cherche s'il peut exister 
  des constantes $\phi,\psi,c \in \Rr$ telles que
  
  \vspace*{-2ex}
  $$F_n = c \big(\phi^n-\psi^n\big)$$
  \vspace*{-4ex}
  

  Nous allons calculer les constantes qui conviendraient.
   \begin{enumerate}
    \item On pose $G_n = \phi^n$. Quelle équation doit satisfaire $\phi$
    afin que $G_{n} = G_{n-1}+G_{n-2}$ ?
    \item Résoudre cette équation.
    \item Calculer $c,\phi,\psi$.
  \end{enumerate}
  
  \item \textbf{Preuve.} On note $H_n = \frac{1}{\sqrt{5}}  \big(\phi^n-\psi^n\big)$
  où $\phi = \frac{1+\sqrt5}{2}$ et $\psi=\frac{1-\sqrt5}{2}$.
  \begin{enumerate}
    \item Vérifier que $F_n=H_n$ pour les premières valeurs de $n$.
    \item Montrer à l'aide du calcul formel que $H_{n} = H_{n-1}+H_{n-2}$. Conclure.
  \end{enumerate}  
\end{enumerate}
\end{tp}

\end{frame}


\begin{frame}[fragile]
\begin{enumerate}
  \item $0 \qquad 1 \qquad 1 \qquad 2 \qquad 3 \qquad 5 \qquad 8 \qquad 13 \qquad 21 
\qquad 34 $
  
\pause

%\insertcode{formel/Algos/fibonacci-tex1.sage}{fibonacci.sage (1)}
\begin{algo}[fibonacci.sage (1) \& (2)]
\begin{lstlisting}
def fibonacci(n):
    if n == 0: return 0
    if n == 1: return 1        
    F_n_2 = 0
    F_n_1 = 1
    for k in range(n-1):
        F_n = F_n_1 + F_n_2        
        F_n_2 = F_n_1
        F_n_1 = F_n
    return F_n|\pause|
    
for n in range(10):
    print fibonacci(n)    
\end{lstlisting}
\end{algo}

\end{enumerate}

\end{frame}


\begin{frame}[fragile]

\begin{enumerate}\setcounter{enumi}{1}
  \item 
  \begin{enumerate}
    \item ~
  \end{enumerate}    
\end{enumerate}  

  \begin{itemize}
    \item On pose $G_n = \phi^n$ et on suppose que $G_{n} = G_{n-1}+G_{n-2}$
    \pause
    \item On veut résoudre $\phi^n = \phi^{n-1} + \phi^{n-2}$
    \pause
    \item $X^n = X^{n-1} + X^{n-2} \pause\iff  X^{n-2}\big(X^2-X-1 \big) = 0$
    \pause
    \item $\phi$ solution de $X^2-X-1 = 0$
    \pause
    \item \codeinline{solve(x^2-x-1==0,x)}
    \pause
    \item \codeinline{[x == -1/2*sqrt(5) + 1/2, x == 1/2*sqrt(5) + 1/2]}
    
  \end{itemize}


\end{frame}


\begin{frame}[fragile]

\begin{enumerate}\setcounter{enumi}{1}
  \item 
  \begin{enumerate}\setcounter{enumii}{1}
  
    \item ~
    
  \end{enumerate}
  
%\insertcode{formel/Algos/fibonacci-tex3.sage}{fibonacci.sage (3)}
\begin{algo}[fibonacci.sage (3)]
\begin{lstlisting}
solutions = solve(x^2-x-1==0,x)
phi = solutions[1].rhs()
psi = solutions[0].rhs()
print 'phi : ',phi,'   psi : ',psi
\end{lstlisting}
\end{algo}

\pause
%  {\setbeamertemplate{itemize items}[ball]
  \begin{itemize}
  
    \item liste \codeinline{solutions}
    \pause
    \item \codeinline{solutions[1]} renvoie \codeinline{x == 1/2*sqrt(5) + 1/2}
    \pause
    \item \codeinline{rhs()} (\emph{right hand side}) partie droite d'une équation
  \end{itemize}
%  }
\pause
  $$\phi = \frac{1+\sqrt5}{2} \quad \mbox{et}  
  \quad \psi  = \frac{1-\sqrt5}{2}  
  \pause\qquad
  F_n = c  \big(\phi^n-\psi^n\big)\ ?$$
  
  \pause

  \begin{enumerate}\setcounter{enumii}{2}  
    \item Comme $F_1=1$ alors $c(\phi-\psi)=1$, donc $c = \frac{1}{\sqrt 5}$
  \end{enumerate}  
  
\pause
  Si une formule 
  $F_n = c(\phi^n-\psi^n)$ existe alors les constantes sont
  nécessairement $c= \frac{1}{\sqrt 5}$, $\phi = \frac{1+\sqrt5}{2}$, 
  $\psi  = \frac{1-\sqrt5}{2}$
\end{enumerate}  
\end{frame}


\begin{frame}[fragile]

\begin{enumerate}\setcounter{enumi}{2}
  \item  Soit $H_n = \frac{1}{\sqrt5}\left(\left(\tfrac{1+\sqrt5}{2}\right)^n- 
  \left(\tfrac{1-\sqrt5}{2}\right)^n\right)$
  
  \smallskip
  
  Montrons la conjecture $F_n = H_n$
  \pause
  
  \begin{enumerate}
    \item On définit une nouvelle fonction \codeinline{conjec} qui correspond au terme de droite de la conjecture que l'on souhaite montrer
    
    
    
    
%\insertcode{formel/Algos/fibonacci-tex4.sage}{fibonacci.sage (4)}
\begin{algo}[fibonacci.sage (4)]
\begin{lstlisting}
def conjec(n):
    return 1/sqrt(5)*(phi^n-psi^n)|\pause|      

for n in range(10):
    valeur = fibonacci(n)-conjec(n)
    print expand(valeur)
\end{lstlisting}
\end{algo}

L'égalité souhaitée est vérifiée pour les premières valeurs de $n$
  \end{enumerate} 
\end{enumerate}  
\end{frame}


\begin{frame}[fragile]

\begin{enumerate}\setcounter{enumi}{2}
  \item 
  \begin{enumerate}\setcounter{enumii}{1}
    \item Nous allons montrer que $F_n=H_n$ pour tout $n\ge0$
    \pause
    \begin{itemize}
      \item \textbf{Initialisation}
      
      \codeinline{fibonacci(0)-conjec(0)} vaut $0$ donc $F_0=H_0=0$
      \codeinline{fibonacci(1)-conjec(1)} vaut $0$ donc $F_1=H_1=1$
     
     \pause
      \item \textbf{Relation de récurrence}
      $H_n$ vérifie exactement la même relation de récurrence que les nombres de Fibonacci
      
      \pause
%\insertcode{formel/Algos/fibonacci-tex5.sage}{fibonacci.sage (5)}
\begin{algo}[fibonacci.sage (5)]
\begin{lstlisting}
def conjec(n):
    return 1/sqrt(5)*(phi^n-psi^n)|\pause| 

var('n')
test = conjec(n)-conjec(n-1)-conjec(n-2)
print(test.simplify_radical())
\end{lstlisting}
\end{algo}      
     
      \pause
      Le résultat est $0$
      
      \pause
      Le calcul est valable pour la variable formelle $n$
      
      \pause
      Ainsi, quel que soit $n\ge0$, $H_n=H_{n-1}+H_{n-2}$
     
      \pause
      \item \textbf{Conclusion} 
      Les suites $(F_n)$ et $(H_n)$ ont les mêmes termes initiaux et 
      vérifient la même relation de récurrence :
      pour tout $n\ge0$, $F_n=H_n$
      
      
    \end{itemize}

  \end{enumerate}
  
\end{enumerate}
\end{frame}




%%%%%%%%%%%%%%%%%%%%%%%%%%%%%%%%%%%%%%%%%%%%%%%%%%%%%%%%%%%%%%%%
\section{L'identité de Cassini}

\begin{frame}

\evidence{L'identité de Cassini}

\medskip

\begin{tp}
Calculer, pour différentes valeurs de $n$, la quantité
$$F_{n+1}F_{n-1}-F_n^2.$$

Que constatez-vous ? 

Proposez une formule générale.

Prouvez cette formule qui s'appelle l'identité de Cassini.   
\end{tp}
\end{frame}


\begin{frame}[fragile]

\begin{itemize}
  \item $C_n = F_{n+1}F_{n-1}-F_n^2 \qquad n \ge 1$
  \pause
  \item Est-ce que $C_n = (-1)^n$ ?
  \pause
  \item $F_n = \frac{1}{\sqrt5}\left(\left(\frac{1+\sqrt5}{2}\right)^n- 
  \left(\frac{1-\sqrt5}{2}\right)^n\right)$
  
\end{itemize}

\pause 
%\insertcode{formel/Algos/cassini-tex1.sage}{cassini.sage (1)}
{\footnotesize
\begin{algo}[cassini.sage]
\begin{lstlisting}
def fibonacci_bis(n):
    return 1/sqrt(5)*( ((1+sqrt(5))/2)^n - ((1-sqrt(5))/2)^n )|\pause|

var('n')|\pause|
cassini = fibonacci_bis(n+1)*fibonacci_bis(n-1)-fibonacci_bis(n)^2|\pause|
print(cassini.simplify_radical())
\end{lstlisting}
\end{algo}
}

\pause

$$C_n = (-1) ^n \frac{(\sqrt 5 -1)^n (\sqrt 5 + 1)^n}{2^{2n}} \pause= (-1)^n$$

\end{frame}


%%%%%%%%%%%%%%%%%%%%%%%%%%%%%%%%%%%%%%%%%%%%%%%%%%%%%%%%%%%%%%%%
\section{Une autre suite récurrente double}


\begin{frame}
\begin{tp}
Soit $(u_n)_{n \in \Nn}$ la suite définie par 
$$u_0 = \frac32, \qquad u_1 = \frac53 \qquad \mbox{et} \qquad u_n = 1 + 4\frac{u_{n-2}-1}{u_{n-1}u_{n-2}}
\quad \text{ pour } n\ge 2.$$
Calculer les premiers termes de cette suite. 

\'Emettre une conjecture.
 
La prouver par récurrence.

~

\emph{Indication.} Les puissances de $2$ seront utiles...
\end{tp}
\end{frame}

\end{document}
