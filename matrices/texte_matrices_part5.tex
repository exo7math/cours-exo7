
%%%%%%%%%%%%%%%%%% PREAMBULE %%%%%%%%%%%%%%%%%%


\documentclass[12pt]{article}

\usepackage{amsfonts,amsmath,amssymb,amsthm}
\usepackage[utf8]{inputenc}
\usepackage[T1]{fontenc}
\usepackage[francais]{babel}


% packages
\usepackage{amsfonts,amsmath,amssymb,amsthm}
\usepackage[utf8]{inputenc}
\usepackage[T1]{fontenc}
%\usepackage{lmodern}

\usepackage[francais]{babel}
\usepackage{fancybox}
\usepackage{graphicx}

\usepackage{float}

%\usepackage[usenames, x11names]{xcolor}
\usepackage{tikz}
\usepackage{datetime}

\usepackage{mathptmx}
%\usepackage{fouriernc}
%\usepackage{newcent}
\usepackage[mathcal,mathbf]{euler}

%\usepackage{palatino}
%\usepackage{newcent}


% Commande spéciale prompteur

%\usepackage{mathptmx}
%\usepackage[mathcal,mathbf]{euler}
%\usepackage{mathpple,multido}

\usepackage[a4paper]{geometry}
\geometry{top=2cm, bottom=2cm, left=1cm, right=1cm, marginparsep=1cm}

\newcommand{\change}{{\color{red}\rule{\textwidth}{1mm}\\}}

\newcounter{mydiapo}

\newcommand{\diapo}{\newpage
\hfill {\normalsize  Diapo \themydiapo \quad \texttt{[\jobname]}} \\
\stepcounter{mydiapo}}


%%%%%%% COULEURS %%%%%%%%%%

% Pour blanc sur noir :
%\pagecolor[rgb]{0.5,0.5,0.5}
% \pagecolor[rgb]{0,0,0}
% \color[rgb]{1,1,1}



%\DeclareFixedFont{\myfont}{U}{cmss}{bx}{n}{18pt}
\newcommand{\debuttexte}{
%%%%%%%%%%%%% FONTES %%%%%%%%%%%%%
\renewcommand{\baselinestretch}{1.5}
\usefont{U}{cmss}{bx}{n}
\bfseries

% Taille normale : commenter le reste !
%Taille Arnaud
%\fontsize{19}{19}\selectfont

% Taille Barbara
%\fontsize{21}{22}\selectfont

%Taille François
\fontsize{25}{30}\selectfont

%Taille Pascal
%\fontsize{25}{30}\selectfont

%Taille Laura
%\fontsize{30}{35}\selectfont


%\myfont
%\usefont{U}{cmss}{bx}{n}

%\Huge
%\addtolength{\parskip}{\baselineskip}
}


% \usepackage{hyperref}
% \hypersetup{colorlinks=true, linkcolor=blue, urlcolor=blue,
% pdftitle={Exo7 - Exercices de mathématiques}, pdfauthor={Exo7}}


%section
% \usepackage{sectsty}
% \allsectionsfont{\bf}
%\sectionfont{\color{Tomato3}\upshape\selectfont}
%\subsectionfont{\color{Tomato4}\upshape\selectfont}

%----- Ensembles : entiers, reels, complexes -----
\newcommand{\Nn}{\mathbb{N}} \newcommand{\N}{\mathbb{N}}
\newcommand{\Zz}{\mathbb{Z}} \newcommand{\Z}{\mathbb{Z}}
\newcommand{\Qq}{\mathbb{Q}} \newcommand{\Q}{\mathbb{Q}}
\newcommand{\Rr}{\mathbb{R}} \newcommand{\R}{\mathbb{R}}
\newcommand{\Cc}{\mathbb{C}} 
\newcommand{\Kk}{\mathbb{K}} \newcommand{\K}{\mathbb{K}}

%----- Modifications de symboles -----
\renewcommand{\epsilon}{\varepsilon}
\renewcommand{\Re}{\mathop{\text{Re}}\nolimits}
\renewcommand{\Im}{\mathop{\text{Im}}\nolimits}
%\newcommand{\llbracket}{\left[\kern-0.15em\left[}
%\newcommand{\rrbracket}{\right]\kern-0.15em\right]}

\renewcommand{\ge}{\geqslant}
\renewcommand{\geq}{\geqslant}
\renewcommand{\le}{\leqslant}
\renewcommand{\leq}{\leqslant}

%----- Fonctions usuelles -----
\newcommand{\ch}{\mathop{\mathrm{ch}}\nolimits}
\newcommand{\sh}{\mathop{\mathrm{sh}}\nolimits}
\renewcommand{\tanh}{\mathop{\mathrm{th}}\nolimits}
\newcommand{\cotan}{\mathop{\mathrm{cotan}}\nolimits}
\newcommand{\Arcsin}{\mathop{\mathrm{Arcsin}}\nolimits}
\newcommand{\Arccos}{\mathop{\mathrm{Arccos}}\nolimits}
\newcommand{\Arctan}{\mathop{\mathrm{Arctan}}\nolimits}
\newcommand{\Argsh}{\mathop{\mathrm{Argsh}}\nolimits}
\newcommand{\Argch}{\mathop{\mathrm{Argch}}\nolimits}
\newcommand{\Argth}{\mathop{\mathrm{Argth}}\nolimits}
\newcommand{\pgcd}{\mathop{\mathrm{pgcd}}\nolimits} 

\newcommand{\Card}{\mathop{\text{Card}}\nolimits}
\newcommand{\Ker}{\mathop{\text{Ker}}\nolimits}
\newcommand{\id}{\mathop{\text{id}}\nolimits}
\newcommand{\ii}{\mathrm{i}}
\newcommand{\dd}{\mathrm{d}}
\newcommand{\Vect}{\mathop{\text{Vect}}\nolimits}
\newcommand{\Mat}{\mathop{\mathrm{Mat}}\nolimits}
\newcommand{\rg}{\mathop{\text{rg}}\nolimits}
\newcommand{\tr}{\mathop{\text{tr}}\nolimits}
\newcommand{\ppcm}{\mathop{\text{ppcm}}\nolimits}

%----- Structure des exercices ------

\newtheoremstyle{styleexo}% name
{2ex}% Space above
{3ex}% Space below
{}% Body font
{}% Indent amount 1
{\bfseries} % Theorem head font
{}% Punctuation after theorem head
{\newline}% Space after theorem head 2
{}% Theorem head spec (can be left empty, meaning ‘normal’)

%\theoremstyle{styleexo}
\newtheorem{exo}{Exercice}
\newtheorem{ind}{Indications}
\newtheorem{cor}{Correction}


\newcommand{\exercice}[1]{} \newcommand{\finexercice}{}
%\newcommand{\exercice}[1]{{\tiny\texttt{#1}}\vspace{-2ex}} % pour afficher le numero absolu, l'auteur...
\newcommand{\enonce}{\begin{exo}} \newcommand{\finenonce}{\end{exo}}
\newcommand{\indication}{\begin{ind}} \newcommand{\finindication}{\end{ind}}
\newcommand{\correction}{\begin{cor}} \newcommand{\fincorrection}{\end{cor}}

\newcommand{\noindication}{\stepcounter{ind}}
\newcommand{\nocorrection}{\stepcounter{cor}}

\newcommand{\fiche}[1]{} \newcommand{\finfiche}{}
\newcommand{\titre}[1]{\centerline{\large \bf #1}}
\newcommand{\addcommand}[1]{}
\newcommand{\video}[1]{}

% Marge
\newcommand{\mymargin}[1]{\marginpar{{\small #1}}}



%----- Presentation ------
\setlength{\parindent}{0cm}

%\newcommand{\ExoSept}{\href{http://exo7.emath.fr}{\textbf{\textsf{Exo7}}}}

\definecolor{myred}{rgb}{0.93,0.26,0}
\definecolor{myorange}{rgb}{0.97,0.58,0}
\definecolor{myyellow}{rgb}{1,0.86,0}

\newcommand{\LogoExoSept}[1]{  % input : echelle
{\usefont{U}{cmss}{bx}{n}
\begin{tikzpicture}[scale=0.1*#1,transform shape]
  \fill[color=myorange] (0,0)--(4,0)--(4,-4)--(0,-4)--cycle;
  \fill[color=myred] (0,0)--(0,3)--(-3,3)--(-3,0)--cycle;
  \fill[color=myyellow] (4,0)--(7,4)--(3,7)--(0,3)--cycle;
  \node[scale=5] at (3.5,3.5) {Exo7};
\end{tikzpicture}}
}



\theoremstyle{definition}
%\newtheorem{proposition}{Proposition}
%\newtheorem{exemple}{Exemple}
%\newtheorem{theoreme}{Théorème}
\newtheorem{lemme}{Lemme}
\newtheorem{corollaire}{Corollaire}
%\newtheorem*{remarque*}{Remarque}
%\newtheorem*{miniexercice}{Mini-exercices}
%\newtheorem{definition}{Définition}




%definition d'un terme
\newcommand{\defi}[1]{{\color{myorange}\textbf{\emph{#1}}}}
\newcommand{\evidence}[1]{{\color{blue}\textbf{\emph{#1}}}}



 %----- Commandes divers ------

\newcommand{\codeinline}[1]{\texttt{#1}}

%%%%%%%%%%%%%%%%%%%%%%%%%%%%%%%%%%%%%%%%%%%%%%%%%%%%%%%%%%%%%
%%%%%%%%%%%%%%%%%%%%%%%%%%%%%%%%%%%%%%%%%%%%%%%%%%%%%%%%%%%%%



\begin{document}

\debuttexte


%%%%%%%%%%%%%%%%%%%%%%%%%%%%%%%%%%%%%%%%%%%%%%%%%%%%%%%%%%%
\diapo
    
\change
Dans cette partie nous allons aborder l'aspect technique de l'inversion de matrices.

\change
On commence par le lien entre les matrices et les systèmes linéaires

\change
spécialement lorsque la matrice est inversible.

\change
On définira les matrices élémentaires

\change
Qui permettront d'obtenir une matrice échelonnée

\change
On appliquera ceci pour l'inverse d'une matrice.


%%%%%%%%%%%%%%%%%%%%%%%%%%%%%%%%%%%%%%%%%%%%%%%%%%%%%%%%%%
\diapo

Considérons le système linéaire suivant :

Il y a $n$ équations et $p$ inconnues $x_1,x_2,...x_p$.

les coefficients sont les $a_{i,j}$ et les $b_i$ forment le second membre.


\change
De façon équivalente on peut réécrire  sous forme matricielle

$AX=B$.

$A$ est la matrice des coefficients du système, c'est une matrice à $n$ lignes et $p$ colonnes.

$X$ est l'inconnue, c'est un vecteur colonne, c-à-d une matrice de taille $p \times 1$.

$B$ est le vecteur colonne du second membre, c'est une matrice de taille $n \times 1$.


Le vecteur $X$ est une solution du système si et seulement si $AX = B.$

\change
Rappelons un résultat vu précédemment :
"Un système d'équations linéaires n'a soit aucune solution, 
soit une seule solution, soit une infinité de solutions."



%%%%%%%%%%%%%%%%%%%%%%%%%%%%%%%%%%%%%%%%%%%%%%%%%%%%%%%%%%%
\diapo

Considérons le cas important où le nombre d'équations est égal aux nombre d'inconnues :

Alors $A$ est une matrice carrée de taille $n\times n$.

\change
Pour tout second membre $B$, nous pouvons utiliser les matrices 
pour trouver la solution du système linéaire.

"Si la matrice $A$ est inversible, alors
la solution du système $AX=B$ est unique et, vaut :
$X = A^{-1}B$."

\change
Pour le prouver, il suffit de multiplier l'égalité $AX=B$
à gauche par la matrice $A^{-1}$, on obtient $X = A^{-1}B$.

Si par contre la matrice $A$ n'est pas inversible,
alors soit il n'y a pas de solution, soit il y a en a une infinité.


%%%%%%%%%%%%%%%%%%%%%%%%%%%%%%%%%%%%%%%%%%%%%%%%%%%%%%%%%%%
\diapo


Pour calculer l'inverse d'une matrice $A$, et aussi pour résoudre des systèmes linéaires,
nous avons utilisé trois opérations élémentaires sur les lignes qui sont :
 
\change
  On peut multiplier une ligne par un réel non nul (ou un élément non nul du corps de base $\Kk$).
    
  On peut ajouter à la ligne $L_i$ un multiple d'une autre ligne $L_j$.
  
  On peut échanger deux lignes.
  
\change
Nous allons définir trois matrices élémentaires correspondant à ces opérations. 

\change
Et le produit d'une matrice élémentaire $E$ par une matrice quelconque $A$ correspondra 
à l'opération élémentaire sur les lignes de $A$. 

\change
Un point important est que comme les opérations élémentaires sur les lignes sont réversibles,
alors nos matrices élémentaires sont aussi inversibles.

C'est ce qui permet de calculer l'inverse d'une matrice quelconque 
par des opérations élémentaires sur les lignes.



%%%%%%%%%%%%%%%%%%%%%%%%%%%%%%%%%%%%%%%%%%%%%%%%%%%%%%%%%%
\diapo

Reprenons chacune des opérations élémentaires en détails.

Tout d'abord $L_i \leftarrow \lambda L_i$ on peut multiplier une ligne par un réel non nul.


  
\change
 Définissons la matrice $E_{L_i \leftarrow \lambda L_i}$ : c'est la matrice obtenue en 
    multipliant par $\lambda$ la $i$-ème ligne de la matrice identité $I_n$, 
    où $\lambda$ est un nombre réel non nul.
    
\change
Voici par exemple la matrice $ E_{L_2 \leftarrow 5 L_2}$

obtenue à partir de la matrice identité en multipliant par $5$ la deuxième ligne.


\change
Si maintenant $A$ est une matrice quelconque
alors par un petit calcul, on voit que la matrice 
$E_{L_i \leftarrow \lambda L_i} \times A$ 
est la matrice obtenue en multipliant par $\lambda$ la $i$-ème ligne de $A$.


\change
Voyons l'exemple de $E_{L_2 \leftarrow \frac13 L_2}  \times A$

\change
c'est-à-dire $ \begin{pmatrix}
  1&0&0\\0&\frac13&0\\0&0&1  
  \end{pmatrix}
  \times
  \begin{pmatrix}
  x_1&x_2&x_3\\y_1&y_2&y_3\\z_1&z_2&z_3  
  \end{pmatrix}$

\change
qui donne bien la matrice $A$ avec la deuxième ligne multipliée par un facteur $1/3$.
  

%%%%%%%%%%%%%%%%%%%%%%%%%%%%%%%%%%%%%%%%%%%%%%%%%%%%%%%%%%%
\diapo

On fait le même travail pour l'opération élémentaire 
$L_i \leftarrow L_i+\lambda L_j$  :
  on peut ajouter à la ligne $L_i$ un multiple d'une autre ligne $L_j$.

\change
La matrice $E_{L_i \leftarrow L_i+\lambda L_j}$ est la  matrice
obtenue en ajoutant $\lambda$ fois la $j$-ème ligne de $I_n$ à la $i$-ème ligne de $I_n$.

\change
Par exemple 
 $$  E_{L_2 \leftarrow L_2 -3 L_1}=
    \begin{pmatrix}
    1 & 0 & 0 & 0\\
    -3 & 1 & 0 & 0\\
    0 & 0 & 1 & 0\\
    0 & 0 & 0 & 1
    \end{pmatrix}$$
 
 On a retranché à la deuxième ligne, $3$ fois la première.
 
 
\change
Et cette fois la matrice $E_{L_i \leftarrow L_i+\lambda L_j} \times A$ est la  matrice obtenue 
en ajoutant $\lambda$ fois la $j$-ème ligne de $A$ à la $i$-ème ligne de $A$.

\change
$E_{L_1 \leftarrow L_1-7 L_3}  \times A$

\change
$
= \begin{pmatrix}
  1&0&-7\\0&1&0\\0&0&1  
  \end{pmatrix}
  \times
  \begin{pmatrix}
  x_1&x_2&x_3\\y_1&y_2&y_3\\z_1&z_2&z_3  
  \end{pmatrix}$
  
\change
A la première ligne on a retranché $7$ fois la troisième.

%%%%%%%%%%%%%%%%%%%%%%%%%%%%%%%%%%%%%%%%%%%%%%%%%%%%%%%%%%
\diapo

L'opération $L_i \leftrightarrow L_j$ consiste à échanger deux lignes.

\change
La matrice  $E_{L_i \leftrightarrow L_j}$ est la matrice obtenue 
   en permutant les $i$-ème et $j$-ème lignes de l'identité. 

\change
Ici on à échanger les lignes $2$ et $4$.


\change
Et selon le même principe la matrice  $E_{L_i \leftrightarrow L_j} \times A$ 
est la matrice obtenue en permutant les $i$-ème et $j$-ème lignes de $A$. 

\change

\change

\change
Ici en multipliant à gauche par la matrice élémentaire $E_{L_2 \leftrightarrow L_3}$

on permute les lignes $2$ et $3$ de la matrice $A$.


%%%%%%%%%%%%%%%%%%%%%%%%%%%%%%%%%%%%%%%%%%%%%%%%%%%%%%%%%%%
\diapo
Nous allons maintenant formaliser le passage d'une matrice à une autre par des opérations élémentaires.


Deux matrices $A$ et $B$ sont dites \defi{équivalentes par lignes} si l'une 
peut être obtenue à partir de l'autre par une suite d'opérations 
élémentaires sur les lignes. On note alors $A \sim B$.

\change
Nous dirons d'une matrice quelconque qu'elle est \defi{échelonnée} si :

le nombre de zéros commençant une ligne croît strictement ligne par ligne 
(et s'arrête s'il n'y a plus que des zéros).
  
\change
Voici un exemple d'une matrice échelonnée

les $*$ désignent des coefficients quelconques, les $+$ des coefficients non nuls

chaque ligne commence par plus de $0$ que la ligne précédente.

\change
On dit d'une matrice qu'elle est \defi{échelonnée réduite} si elle est échelonnée et si en plus : 

le premier coefficient non nul de chaque ligne non nulle vaut $1$ ;

et c'est le seul élément non nul de sa colonne.

\change
Voici une matrice échelonnée réduite, les $*$ désignent toujours des coefficients quelconques.

elle est bien échelonnée car chaque ligne commence par plus de $0$ que la ligne précédente,

le premier  coefficient non nul de chaque ligne vaut $1$ ;

et au dessus de ces $1$ il n'y a que des zéros.




%%%%%%%%%%%%%%%%%%%%%%%%%%%%%%%%%%%%%%%%%%%%%%%%%%%%%%%%%%
\diapo

La définition de matrice échelonnée et échelonnée réduite prend tout son intérêt avec le théorème suivant.


Théorème :
"\'Etant donnée une matrice $A$, il existe une unique  
matrice échelonnée réduite $U$ obtenue à partir de $A$ 
par des opérations élémentaires sur les lignes. "

Ce théorème permet donc de se ramener par des opérations élémentaires
à des matrices dont la structure est beaucoup plus simple :
les matrices échelonnées réduites.


%%%%%%%%%%%%%%%%%%%%%%%%%%%%%%%%%%%%%%%%%%%%%%%%%%%%%%%%%%%
\diapo

Voyons quelques conséquence dans le cas des matrices carrées.


Soit $A\in M_{n}(\Kk)$.
La matrice $A$ est inversible si et seulement si sa forme échelonnée réduite 
est la matrice identité $I_n$.


C'est ce théorème qui justifie la méthode de calcul de l'inverse 
à l'aide de la matrice augmentée, que l'on a vu au chapitre précédent. Les opérations élémentaires étant inversibles, l'inverse correspond au produit des matrices élémentaires.

\change
Autre corollaire important

Les assertions suivantes sont équivalentes :

[(i)] La matrice $A$ est inversible.
  
[(ii)] Le système linéaire $AX=\left(\begin{smallmatrix} 0 \\ \vdots \\ 0\end{smallmatrix}\right)$ 
a une unique solution 
  $X=\left(\begin{smallmatrix} 0 \\ \vdots \\ 0\end{smallmatrix}\right)$.
             
[(iii)] Pour tout second membre $B$, le système linéaire $AX=B$
  a une unique solution $X$.
  
  Et on a même vu qu'alors la solution était $X=A^{-1}B$.

%%%%%%%%%%%%%%%%%%%%%%%%%%%%%%%%%%%%%%%%%%%%%%%%%%%%%%%%%%%
\diapo

Le reste de cette vidéo est consacré à démontrer le théorème principal :

"\'Etant donnée une matrice $A$, il existe une unique  
matrice échelonnée réduite $U$ obtenue à partir de $A$ 
par des opérations élémentaires sur les lignes. "


Nous admettons l'unicité.

L'existence se démontre grâce à l'algorithme de Gauss. 
L'idée générale consiste à utiliser des substitutions de
lignes pour placer des zéros là où il faut de façon à 
créer d'abord une forme échelonnée, puis une forme
échelonnée réduite. 


%%%%%%%%%%%%%%%%%%%%%%%%%%%%%%%%%%%%%%%%%%%%%%%%%%%%%%%%%%%
\diapo

Soit $A$ une matrice $n\times p$ quelconque.

On commence par transformer $A$ en une matrice échelonnée.

\change
La première étape est de choisir un pivot.

\change
On commence par observer la première colonne. 

Soit elle ne contient que des zéros, auquel cas on passe directement à l'étape A.3, 

\change
soit elle contient au moins un terme non nul, situé sur la ligne $i$. 
On choisit alors un tel terme, que l'on appelle le \defi{pivot}. 

\change
Par un échange de ligne on ramène le pivot sur la première ligne. Et on passe à l'étape suivante.

\change
Au terme de l'étape A.1, soit la matrice $A$ a sa première colonne nulle 
ou bien on obtient une matrice équivalente $A'$ dont le premier coefficient $a'_{11}$ est non nul.


\change
L'étape suivante consiste à faire apparaître des $0$ sous le pivot.

\change
On ne touche plus à la ligne $1$, 

\change
et on se sert du pivot $a'_{11}$
pour éliminer tous les termes $a'_{i1}$  situés sous le pivot.

\change
Pour cela, il suffit de remplacer la ligne $i$ par 
elle-même moins un facteur la ligne $1$ :
$L_2 \leftarrow L_2 - \frac{a'_{21}}{a'_{11}}L_1$, 
$L_3 \leftarrow L_3 - \frac{a'_{31}}{a'_{11}}L_1$,\ldots 

\change
Au terme de l'étape A.2, on a obtenu une matrice de cette forme
avec sur la première colonne un seul élément non nul.


%%%%%%%%%%%%%%%%%%%%%%%%%%%%%%%%%%%%%%%%%%%%%%%%%%%%%%%%%%
\diapo

Nous allons itérer nos étapes A.1 et A.2.

\change
Au début de l'étape A.3, on a obtenu dans tous les cas de figure une matrice de cette forme


\change
dont la première colonne est bien celle d'une matrice échelonnée. 

On va donc conserver cette première colonne. 

\change
Si ce coefficient $a^1_{11}\neq0$, 
on conserve aussi la première ligne, et l'on recommence l'étape A.1 
en l'appliquant cette fois à cette sous-matrice $(n-1)\times(p-1)$ 
obtenu en << oubliant >> la première ligne et la première colonne de $A$;

\change
Si par contre $a^1_{11}=0$, on repart avec l'étape A.1 en l'appliquant à cette sous-matrice 
$n\times(p-1)$ obtenue en << oubliant >> seulement la première colonne.


\change
Dans tous les cas, au terme de cette deuxième itération de la boucle, on aura obtenu une matrice de cette forme, dont les deux premières colonnes sont celles d'une matrice échelonnée. 


\change
On continue ainsi de suite.

Comme chaque itération de la boucle s'applique à une matrice qui a 
une colonne de moins que la précédente, 

\change
alors au bout d'au plus $p-1$ itérations de la boucle, on aura obtenu une 
matrice échelonnée.


%%%%%%%%%%%%%%%%%%%%%%%%%%%%%%%%%%%%%%%%%%%%%%%%%%%%%%%%%%%
\diapo

Notre matrice $A$ est maintenant équivalente à une matrice échelonnée.

Nous allons la transformer en une forme échelonnée et réduite.

\change
On commence par obtenir des pivots égaux à $1$.

\change
On repère le premier élément non nul de chaque ligne qui n'est pas nulle, c'est-à-dire le pivot

\change
et on multiplie cette ligne par l'inverse du pivot. 

\change
Exemple : si le pivot de la ligne $i$ est $\alpha$, alors
on effectue la transformation $L_i \leftarrow \frac1\alpha L_i$.

\change
Ceci crée une matrice échelonnée avec des $1$ en position de pivot.

\change
On fait de nouveau apparaître des $0$.

\change
On élimine les termes situés cette fois *au-dessus* des pivots comme précédemment.

Pour conserver le caractère échelonnée de la matrice on procède 
à partir du bas à droite de la matrice pour terminer en haut à gauche. 


\change
On obtient bien une matrice équivalente à $A$ qui est échelonnée et réduite.



%%%%%%%%%%%%%%%%%%%%%%%%%%%%%%%%%%%%%%%%%%%%%%%%%%%%%%%%%%%
\diapo


Reprenons toute cette méthode pas à pas sur un exemple avec la matrice
$$A=\begin{pmatrix}1&2&3&4\\
0&2&4&6\\
-1&0&1&0
\end{pmatrix}.$$

\change
Tout d'abord on va passer à une forme échelonnée.

\change
Le choix du premier pivot est tout fait, 
on garde $a_{11}^1=1$.

\change
On fait maintenant apparaître des zéros sous ce pivot.
On ne fait rien sur la ligne $2$ qui contient 
déjà un zéro en bonne position et on remplace la ligne 3 par $L_3 + L_1$. 
On obtient 
$$A\sim\begin{pmatrix}1&2&3&4\\
0&2&4&6\\
0&2&4&4
\end{pmatrix}.$$

\change
On passe à cette sous-matrice $2\times 3$.

Le choix du pivot est tout fait, on garde $a^2_{22}=2$.

\change
On remplace  la ligne $3$ avec
l'opération $L_3 \leftarrow L_3 - L_2$. On obtient
$$A\sim\begin{pmatrix}1&2&3&4\\
0&2&4&6\\
0&0&0&-2
\end{pmatrix}.$$

\change
Et on a obtenu une matrice équivalente par ligne avec $A$ et qui est bien échelonnée.

%%%%%%%%%%%%%%%%%%%%%%%%%%%%%%%%%%%%%%%%%%%%%%%%%%%%%%%%%%%
\diapo

Nous avons obtenu une matrice échelonnée équivalente à notre matrice $A$.

\change
Nous allons maintenant passer à une forme échelonnée et réduite.

\change
On ne veut que des coefficients $1$ en position de pivot.
 On multiplie la ligne $2$ par $\frac12$ 
et la ligne $3$ par $-\frac12$ et l'on obtient
$$A\sim\begin{pmatrix}1&2&3&4\\
0&1&2&3\\
0&0&0&1
\end{pmatrix}.$$

\change
On veut en plus des $0$ au-dessus des pivots.

Pour ce qui concerne ce pivot là,
on ne touche plus à la ligne $3$ et on remplace la ligne $2$ par
$ L_2-3L_3$ et la ligne 1 par $ L_1 - 4L_3$.
On obtient
$$A\sim\begin{pmatrix}1&2&3&0\\
0&1&2&0\\
0&0&0&1
\end{pmatrix}.$$

\change
Et pour ce pivot ci : on ne touche plus à la ligne $2$ 
et on remplace la ligne $1$ par $ L_1-2L_2$. On obtient
$$A\sim\begin{pmatrix}1&0&-1&0\\
0&1&2&0\\
0&0&0&1
\end{pmatrix}$$

\change

C'est bien une matrice échelonnée et réduite.



%%%%%%%%%%%%%%%%%%%%%%%%%%%%%%%%%%%%%%%%%%%%%%%%%%%%%%%%%%%
\diapo


Comme d'habitude terminez par des exercices pour assimiler les cours.


\end{document}
