
%%%%%%%%%%%%%%%%%% PREAMBULE %%%%%%%%%%%%%%%%%%

\documentclass[aspectratio=169,utf8]{beamer}
%\documentclass[aspectratio=169,handout]{beamer}

\usetheme{Boadilla}
%\usecolortheme{seahorse}
\usecolortheme[RGB={245,66,24}]{structure}
\useoutertheme{infolines}

% packages
\usepackage{amsfonts,amsmath,amssymb,amsthm}
\usepackage[utf8]{inputenc}
\usepackage[T1]{fontenc}
\usepackage{lmodern}

\usepackage[francais]{babel}
\usepackage{fancybox}
\usepackage{graphicx}

\usepackage{float}
\usepackage{xfrac}

%\usepackage[usenames, x11names]{xcolor}
\usepackage{tikz}
\usepackage{pgfplots}
\usepackage{datetime}



%-----  Package unités -----
\usepackage{siunitx}
\sisetup{locale = FR,detect-all,per-mode = symbol}

%\usepackage{mathptmx}
%\usepackage{fouriernc}
%\usepackage{newcent}
%\usepackage[mathcal,mathbf]{euler}

%\usepackage{palatino}
%\usepackage{newcent}
% \usepackage[mathcal,mathbf]{euler}



% \usepackage{hyperref}
% \hypersetup{colorlinks=true, linkcolor=blue, urlcolor=blue,
% pdftitle={Exo7 - Exercices de mathématiques}, pdfauthor={Exo7}}


%section
% \usepackage{sectsty}
% \allsectionsfont{\bf}
%\sectionfont{\color{Tomato3}\upshape\selectfont}
%\subsectionfont{\color{Tomato4}\upshape\selectfont}

%----- Ensembles : entiers, reels, complexes -----
\newcommand{\Nn}{\mathbb{N}} \newcommand{\N}{\mathbb{N}}
\newcommand{\Zz}{\mathbb{Z}} \newcommand{\Z}{\mathbb{Z}}
\newcommand{\Qq}{\mathbb{Q}} \newcommand{\Q}{\mathbb{Q}}
\newcommand{\Rr}{\mathbb{R}} \newcommand{\R}{\mathbb{R}}
\newcommand{\Cc}{\mathbb{C}} 
\newcommand{\Kk}{\mathbb{K}} \newcommand{\K}{\mathbb{K}}

%----- Modifications de symboles -----
\renewcommand{\epsilon}{\varepsilon}
\renewcommand{\Re}{\mathop{\text{Re}}\nolimits}
\renewcommand{\Im}{\mathop{\text{Im}}\nolimits}
%\newcommand{\llbracket}{\left[\kern-0.15em\left[}
%\newcommand{\rrbracket}{\right]\kern-0.15em\right]}

\renewcommand{\ge}{\geqslant}
\renewcommand{\geq}{\geqslant}
\renewcommand{\le}{\leqslant}
\renewcommand{\leq}{\leqslant}
\renewcommand{\epsilon}{\varepsilon}

%----- Fonctions usuelles -----
\newcommand{\ch}{\mathop{\text{ch}}\nolimits}
\newcommand{\sh}{\mathop{\text{sh}}\nolimits}
\renewcommand{\tanh}{\mathop{\text{th}}\nolimits}
\newcommand{\cotan}{\mathop{\text{cotan}}\nolimits}
\newcommand{\Arcsin}{\mathop{\text{arcsin}}\nolimits}
\newcommand{\Arccos}{\mathop{\text{arccos}}\nolimits}
\newcommand{\Arctan}{\mathop{\text{arctan}}\nolimits}
\newcommand{\Argsh}{\mathop{\text{argsh}}\nolimits}
\newcommand{\Argch}{\mathop{\text{argch}}\nolimits}
\newcommand{\Argth}{\mathop{\text{argth}}\nolimits}
\newcommand{\pgcd}{\mathop{\text{pgcd}}\nolimits} 


%----- Commandes divers ------
\newcommand{\ii}{\mathrm{i}}
\newcommand{\dd}{\text{d}}
\newcommand{\id}{\mathop{\text{id}}\nolimits}
\newcommand{\Ker}{\mathop{\text{Ker}}\nolimits}
\newcommand{\Card}{\mathop{\text{Card}}\nolimits}
\newcommand{\Vect}{\mathop{\text{Vect}}\nolimits}
\newcommand{\Mat}{\mathop{\text{Mat}}\nolimits}
\newcommand{\rg}{\mathop{\text{rg}}\nolimits}
\newcommand{\tr}{\mathop{\text{tr}}\nolimits}


%----- Structure des exercices ------

\newtheoremstyle{styleexo}% name
{2ex}% Space above
{3ex}% Space below
{}% Body font
{}% Indent amount 1
{\bfseries} % Theorem head font
{}% Punctuation after theorem head
{\newline}% Space after theorem head 2
{}% Theorem head spec (can be left empty, meaning ‘normal’)

%\theoremstyle{styleexo}
\newtheorem{exo}{Exercice}
\newtheorem{ind}{Indications}
\newtheorem{cor}{Correction}


\newcommand{\exercice}[1]{} \newcommand{\finexercice}{}
%\newcommand{\exercice}[1]{{\tiny\texttt{#1}}\vspace{-2ex}} % pour afficher le numero absolu, l'auteur...
\newcommand{\enonce}{\begin{exo}} \newcommand{\finenonce}{\end{exo}}
\newcommand{\indication}{\begin{ind}} \newcommand{\finindication}{\end{ind}}
\newcommand{\correction}{\begin{cor}} \newcommand{\fincorrection}{\end{cor}}

\newcommand{\noindication}{\stepcounter{ind}}
\newcommand{\nocorrection}{\stepcounter{cor}}

\newcommand{\fiche}[1]{} \newcommand{\finfiche}{}
\newcommand{\titre}[1]{\centerline{\large \bf #1}}
\newcommand{\addcommand}[1]{}
\newcommand{\video}[1]{}

% Marge
\newcommand{\mymargin}[1]{\marginpar{{\small #1}}}

\def\noqed{\renewcommand{\qedsymbol}{}}


%----- Presentation ------
\setlength{\parindent}{0cm}

%\newcommand{\ExoSept}{\href{http://exo7.emath.fr}{\textbf{\textsf{Exo7}}}}

\definecolor{myred}{rgb}{0.93,0.26,0}
\definecolor{myorange}{rgb}{0.97,0.58,0}
\definecolor{myyellow}{rgb}{1,0.86,0}

\newcommand{\LogoExoSept}[1]{  % input : echelle
{\usefont{U}{cmss}{bx}{n}
\begin{tikzpicture}[scale=0.1*#1,transform shape]
  \fill[color=myorange] (0,0)--(4,0)--(4,-4)--(0,-4)--cycle;
  \fill[color=myred] (0,0)--(0,3)--(-3,3)--(-3,0)--cycle;
  \fill[color=myyellow] (4,0)--(7,4)--(3,7)--(0,3)--cycle;
  \node[scale=5] at (3.5,3.5) {Exo7};
\end{tikzpicture}}
}


\newcommand{\debutmontitre}{
  \author{} \date{} 
  \thispagestyle{empty}
  \hspace*{-10ex}
  \begin{minipage}{\textwidth}
    \titlepage  
  \vspace*{-2.5cm}
  \begin{center}
    \LogoExoSept{2.5}
  \end{center}
  \end{minipage}

  \vspace*{-0cm}
  
  % Astuce pour que le background ne soit pas discrétisé lors de la conversion pdf -> png
\begin{tikzpicture}
        \fill[opacity=0,green!60!black] (0,0)--++(0,0)--++(0,0)--++(0,0)--cycle; 
\end{tikzpicture}

% toc S'affiche trop tot :
% \tableofcontents[hideallsubsections, pausesections]
}

\newcommand{\finmontitre}{
  \end{frame}
  \setcounter{framenumber}{0}
} % ne marche pas pour une raison obscure

%----- Commandes supplementaires ------

% \usepackage[landscape]{geometry}
% \geometry{top=1cm, bottom=3cm, left=2cm, right=10cm, marginparsep=1cm
% }
% \usepackage[a4paper]{geometry}
% \geometry{top=2cm, bottom=2cm, left=2cm, right=2cm, marginparsep=1cm
% }

%\usepackage{standalone}


% New command Arnaud -- november 2011
\setbeamersize{text margin left=24ex}
% si vous modifier cette valeur il faut aussi
% modifier le decalage du titre pour compenser
% (ex : ici =+10ex, titre =-5ex

\theoremstyle{definition}
%\newtheorem{proposition}{Proposition}
%\newtheorem{exemple}{Exemple}
%\newtheorem{theoreme}{Théorème}
%\newtheorem{lemme}{Lemme}
%\newtheorem{corollaire}{Corollaire}
%\newtheorem*{remarque*}{Remarque}
%\newtheorem*{miniexercice}{Mini-exercices}
%\newtheorem{definition}{Définition}

% Commande tikz
\usetikzlibrary{calc}
\usetikzlibrary{patterns,arrows}
\usetikzlibrary{matrix}
\usetikzlibrary{fadings} 

%definition d'un terme
\newcommand{\defi}[1]{{\color{myorange}\textbf{\emph{#1}}}}
\newcommand{\evidence}[1]{{\color{blue}\textbf{\emph{#1}}}}
\newcommand{\assertion}[1]{\emph{\og#1\fg}}  % pour chapitre logique
%\renewcommand{\contentsname}{Sommaire}
\renewcommand{\contentsname}{}
\setcounter{tocdepth}{2}



%------ Figures ------

\def\myscale{1} % par défaut 
\newcommand{\myfigure}[2]{  % entrée : echelle, fichier figure
\def\myscale{#1}
\begin{center}
\footnotesize
{#2}
\end{center}}


%------ Encadrement ------

\usepackage{fancybox}


\newcommand{\mybox}[1]{
\setlength{\fboxsep}{7pt}
\begin{center}
\shadowbox{#1}
\end{center}}

\newcommand{\myboxinline}[1]{
\setlength{\fboxsep}{5pt}
\raisebox{-10pt}{
\shadowbox{#1}
}
}

%--------------- Commande beamer---------------
\newcommand{\beameronly}[1]{#1} % permet de mettre des pause dans beamer pas dans poly


\setbeamertemplate{navigation symbols}{}
\setbeamertemplate{footline}  % tiré du fichier beamerouterinfolines.sty
{
  \leavevmode%
  \hbox{%
  \begin{beamercolorbox}[wd=.333333\paperwidth,ht=2.25ex,dp=1ex,center]{author in head/foot}%
    % \usebeamerfont{author in head/foot}\insertshortauthor%~~(\insertshortinstitute)
    \usebeamerfont{section in head/foot}{\bf\insertshorttitle}
  \end{beamercolorbox}%
  \begin{beamercolorbox}[wd=.333333\paperwidth,ht=2.25ex,dp=1ex,center]{title in head/foot}%
    \usebeamerfont{section in head/foot}{\bf\insertsectionhead}
  \end{beamercolorbox}%
  \begin{beamercolorbox}[wd=.333333\paperwidth,ht=2.25ex,dp=1ex,right]{date in head/foot}%
    % \usebeamerfont{date in head/foot}\insertshortdate{}\hspace*{2em}
    \insertframenumber{} / \inserttotalframenumber\hspace*{2ex} 
  \end{beamercolorbox}}%
  \vskip0pt%
}


\definecolor{mygrey}{rgb}{0.5,0.5,0.5}
\setlength{\parindent}{0cm}
%\DeclareTextFontCommand{\helvetica}{\fontfamily{phv}\selectfont}

% background beamer
\definecolor{couleurhaut}{rgb}{0.85,0.9,1}  % creme
\definecolor{couleurmilieu}{rgb}{1,1,1}  % vert pale
\definecolor{couleurbas}{rgb}{0.85,0.9,1}  % blanc
\setbeamertemplate{background canvas}[vertical shading]%
[top=couleurhaut,middle=couleurmilieu,midpoint=0.4,bottom=couleurbas] 
%[top=fondtitre!05,bottom=fondtitre!60]



\makeatletter
\setbeamertemplate{theorem begin}
{%
  \begin{\inserttheoremblockenv}
  {%
    \inserttheoremheadfont
    \inserttheoremname
    \inserttheoremnumber
    \ifx\inserttheoremaddition\@empty\else\ (\inserttheoremaddition)\fi%
    \inserttheorempunctuation
  }%
}
\setbeamertemplate{theorem end}{\end{\inserttheoremblockenv}}

\newenvironment{theoreme}[1][]{%
   \setbeamercolor{block title}{fg=structure,bg=structure!40}
   \setbeamercolor{block body}{fg=black,bg=structure!10}
   \begin{block}{{\bf Th\'eor\`eme }#1}
}{%
   \end{block}%
}


\newenvironment{proposition}[1][]{%
   \setbeamercolor{block title}{fg=structure,bg=structure!40}
   \setbeamercolor{block body}{fg=black,bg=structure!10}
   \begin{block}{{\bf Proposition }#1}
}{%
   \end{block}%
}

\newenvironment{corollaire}[1][]{%
   \setbeamercolor{block title}{fg=structure,bg=structure!40}
   \setbeamercolor{block body}{fg=black,bg=structure!10}
   \begin{block}{{\bf Corollaire }#1}
}{%
   \end{block}%
}

\newenvironment{mydefinition}[1][]{%
   \setbeamercolor{block title}{fg=structure,bg=structure!40}
   \setbeamercolor{block body}{fg=black,bg=structure!10}
   \begin{block}{{\bf Définition} #1}
}{%
   \end{block}%
}

\newenvironment{lemme}[0]{%
   \setbeamercolor{block title}{fg=structure,bg=structure!40}
   \setbeamercolor{block body}{fg=black,bg=structure!10}
   \begin{block}{\bf Lemme}
}{%
   \end{block}%
}

\newenvironment{remarque}[1][]{%
   \setbeamercolor{block title}{fg=black,bg=structure!20}
   \setbeamercolor{block body}{fg=black,bg=structure!5}
   \begin{block}{Remarque #1}
}{%
   \end{block}%
}


\newenvironment{exemple}[1][]{%
   \setbeamercolor{block title}{fg=black,bg=structure!20}
   \setbeamercolor{block body}{fg=black,bg=structure!5}
   \begin{block}{{\bf Exemple }#1}
}{%
   \end{block}%
}


\newenvironment{miniexercice}[0]{%
   \setbeamercolor{block title}{fg=structure,bg=structure!20}
   \setbeamercolor{block body}{fg=black,bg=structure!5}
   \begin{block}{Mini-exercices}
}{%
   \end{block}%
}


\newenvironment{tp}[0]{%
   \setbeamercolor{block title}{fg=structure,bg=structure!40}
   \setbeamercolor{block body}{fg=black,bg=structure!10}
   \begin{block}{\bf Travaux pratiques}
}{%
   \end{block}%
}
\newenvironment{exercicecours}[1][]{%
   \setbeamercolor{block title}{fg=structure,bg=structure!40}
   \setbeamercolor{block body}{fg=black,bg=structure!10}
   \begin{block}{{\bf Exercice }#1}
}{%
   \end{block}%
}
\newenvironment{algo}[1][]{%
   \setbeamercolor{block title}{fg=structure,bg=structure!40}
   \setbeamercolor{block body}{fg=black,bg=structure!10}
   \begin{block}{{\bf Algorithme}\hfill{\color{gray}\texttt{#1}}}
}{%
   \end{block}%
}


\setbeamertemplate{proof begin}{
   \setbeamercolor{block title}{fg=black,bg=structure!20}
   \setbeamercolor{block body}{fg=black,bg=structure!5}
   \begin{block}{{\footnotesize Démonstration}}
   \footnotesize
   \smallskip}
\setbeamertemplate{proof end}{%
   \end{block}}
\setbeamertemplate{qed symbol}{\openbox}


\makeatother
\usecolortheme[RGB={191,146,10}]{structure}

%%%%%%%%%%%%%%%%%%%%%%%%%%%%%%%%%%%%%%%%%%%%%%%%%%%%%%%%%%%%%
%%%%%%%%%%%%%%%%%%%%%%%%%%%%%%%%%%%%%%%%%%%%%%%%%%%%%%%%%%%%%

\begin{document}


\title{{\bf Matrices}}
\subtitle{Inverse d'une matrice : définition}

\begin{frame}
  
  \debutmontitre

  \pause

{\footnotesize
\hfill
\setbeamercovered{transparent=50}
\begin{minipage}{0.6\textwidth}
  \begin{itemize}
    \item<3-> Définition
    \item<4-> Exemples
    \item<5-> Propriétés
  \end{itemize}
\end{minipage}
}

\end{frame}

\setcounter{framenumber}{0}


%%%%%%%%%%%%%%%%%%%%%%%%%%%%%%%%%%%%%%%%%%%%%%%%%%%%%%%%%%%%%%%%
\section{Définition}

\begin{frame}
\begin{mydefinition}%[Matrice inverse]
Soit $A\in M_n(\Kk)$

\pause
S'il existe $B\in M_n(\Kk)$ telle que
 $$ AB = \Id \qquad \text{et} \qquad BA = \Id $$
alors $A$ est dite \defi{inversible}. \pause On appelle $B$ l'\defi{inverse de $A$}, noté $A^{-1}$
\end{mydefinition}

\begin{itemize}
\item\pause Il suffit en fait de vérifier $AB=\Id$ ou bien $BA=\Id$
\item\pause Quand $A$ est inversible, pour $p\in \Nn$, on note  
$$A^{-p}=(A^{-1})^p = \underbrace{A^{-1} A^{-1} \cdots A^{-1}}_{{p \text{ facteurs}}}$$
\item\pause L'ensemble des matrices inversibles de $M_{n}(\Kk)$ est noté $GL_{n}(\Kk)$
\end{itemize}

\end{frame}

%%%%%%%%%%%%%%%%%%%%%%%%%%%%%%%%%%%%%%%%%%%%%%%%%%%%%%%%%%%%%%%%
\section{Exemples}

\begin{frame}
\begin{exemple} 
\begin{itemize}
  \item Soit $A=\left(\begin{smallmatrix}
1 & 2 \cr
0 & 3 \cr
\end{smallmatrix}\right)$

  \item\pause
Cherchons $B= \left(\begin{smallmatrix}
a & b \cr
c & d\cr
\end{smallmatrix}\right)$ telle que $AB=I_2$ et $BA=I_2$
  
  \item\pause Or $AB=\begin{pmatrix}
1 & 2 \cr
0 & 3\cr
\end{pmatrix}
\begin{pmatrix}
a & b \cr
c & d\cr
\end{pmatrix}
=\begin{pmatrix}
a+2c & b+2d \cr
3c & 3d\cr
\end{pmatrix}$
  
  \item\pause 
\vspace{-.2cm}
{\small
$
AB=I_2 \iff
\begin{pmatrix}
a+2c & b+2d \cr
3c & 3d\cr
\end{pmatrix}
=\begin{pmatrix}
1 & 0 \cr
0 & 1\cr
\end{pmatrix}
\pause
 \iff 
\left\{ \begin{array}{l}
a+2c=1\\        
b+2d=0\\
3c=0\\
3d=1\\
\end{array} \right.
$}
  
  \item\pause Il y a une unique solution $B=\left(\begin{smallmatrix}
1 & - \sfrac{2}{3} \cr
0 & \sfrac{1}{3}\cr
\end{smallmatrix}\right)$
  
  \item\pause Pour prouver que $B=A^{-1}$, il faut aussi montrer l'égalité $BA=I_2$
  
  \item\pause La matrice $A$ est donc inversible et $A^{-1}= \begin{pmatrix}
1 & -\frac23 \cr
0 & \frac13\cr
\end{pmatrix}$
\end{itemize}


\end{exemple}
\end{frame}

%--------------------------------------------------------------

\begin{frame}
\begin{exemple}
$ A = \left(\begin{smallmatrix}
3 & 0\\
5 & 0\end{smallmatrix}\right)$
n'est pas inversible

\pause
En effet, pour 
$B= \left( \begin{smallmatrix}
a & b \cr
c & d\cr
\end{smallmatrix} \right)$ 
$$
BA = 
\begin{pmatrix}
a & b \cr
c & d\cr
\end{pmatrix}
\begin{pmatrix}
3 & 0\\
5 & 0   
\end{pmatrix}
=
\begin{pmatrix}
3a+5b & 0\\
3c+5d      & 0
\end{pmatrix} \pause \quad \neq \quad
\begin{pmatrix}
1 & 0 \cr
0 & 1\cr
\end{pmatrix}$$
\end{exemple}

\pause
\begin{exemple}
\begin{itemize}
  \item $I_{n}$ est inversible et $I_n^{-1}=I_n$, par l'égalité 
 $I_{n}I_{n}=I_{n}$
 
  \item\pause La matrice nulle $0_n$ n'est pas inversible : 
  
  pour tout $B\in M_{n}(\Kk)$, on a $B0_n=0_n \ \neq \ I_n$
\end{itemize}
  \end{exemple}

\end{frame}


%%%%%%%%%%%%%%%%%%%%%%%%%%%%%%%%%%%%%%%%%%%%%%%%%%%%%%%%%%%%%%%%
\section{Propriétés}

\begin{frame}

\begin{proposition}%[Unicité]
Si $A$ est inversible, alors son inverse est unique
\end{proposition}

\pause
\begin{proof} 
Supposons qu'il existe $B_{1}$ telle que $AB_{1}=B_{1}A=I_{n}$ et $B_{2}$ telle que 
$AB_{2}=B_{2}A=I_{n}$

\begin{itemize}
\item\pause Calculons $B_{2}(AB_{1})$
\item\pause D'une part, $B_{2}(AB_{1})=B_{2} I_n \pause = B_2$
\item\pause D'autre part, $B_{2}(AB_{1})=(B_{2}A)B_{1}\pause=I_{n}B_{1}=B_{1}$
\item\pause Donc $B_{1}=B_{2}$
\end{itemize}
\end{proof}
\end{frame}

%--------------------------------------------------------------

\begin{frame}

\begin{proposition}%[Inverse de l'inverse]
Soit $A$ une matrice inversible. Alors 
$A^{-1}$ est aussi inversible 
\pause
et
\mybox{$(A^{-1})^{-1}=A$}
\end{proposition}

% \pause
% \bigskip

% Cohérent avec la notation puissance:
% \[
% \big( A^{-1} \big)^{-1} = A^{(-1) \times (-1)} = A
% \]

\end{frame}

%--------------------------------------------------------------

\begin{frame}

\begin{proposition}%[Inverse d'un produit]
 $A$ et $B$ deux matrices inversibles de même taille. Alors
 $AB$ est inversible 
 \pause
 et  \vspace*{-2ex}
 \mybox{$\displaystyle (AB)^{-1} = B^{-1} A^{-1}$}
\end{proposition}

\medskip
\pause
\evidence{Attention à l'inversion de l'ordre !}
\medskip

\pause
\begin{proof}
\begin{itemize}
\item Montrons que $(B^{-1}A^{-1}) (AB) = \Id$

\pause
\centerline{$
   (B^{-1}A^{-1}) (AB) = B^{-1}(A^{-1}A)B \pause= B^{-1}\Id B\pause=B^{-1}B\pause=\Id
$}
\item\pause De même $(AB) (B^{-1} A^{-1})=A(BB^{-1})A^{-1} = A A^{-1} =  \Id$ \qedhere
 \end{itemize}
\end{proof}
 
\pause
De façon analogue, si $A_1, \dots , A_m$ sont inversibles, alors
\[
(A_1 A_2 \cdots A_m)^{-1} = A_m^{-1} A_{m-1}^{-1} \cdots A^{-1}_1
\]
\end{frame}


%--------------------------------------------------------------

\begin{frame}

Rappel : en général  $AC=BC$ n'implique pas $A=B$
\medskip

\pause
\begin{proposition}%[Simplification par une matrice inversible]
Soient $A, B\in M_{n}(\Kk)$ et $C$ une matrice \evidence{inversible} de $M_{n}(\Kk)$

\smallskip
\pause
\centerline{Si $AC=BC$ alors $A=B$}
\end{proposition}

\pause
\begin{proof} 
$$\begin{array}{rcl}
AC=BC 
\pause&\implies& (AC)C^{-1}=(BC)C^{-1}\\
\pause&\implies& A(CC^{-1})=B(CC^{-1})\\
\pause&\implies& A\Id=B\Id\\
\pause&\implies& A=B
\end{array}$$
\end{proof} 

\end{frame}

%%%%%%%%%%%%%%%%%%%%%%%%%%%%%%%%%%%%%%%%%%%%%%%%%%%%%%%%%%%%%%%%
\section{Mini-exercices}

\begin{frame}

\begin{miniexercice}
\begin{enumerate}
  \item Soient $A=\left(\begin{smallmatrix}-1 & -2 \\ 3  & 4 \end{smallmatrix}\right)$
  et $B = \left(\begin{smallmatrix}2 & 1 \\ 5  & 3 \end{smallmatrix}\right)$. Calculer
  $A^{-1}$, $B^{-1}$ , $(AB)^{-1}$, $(BA)^{-1}$, $A^{-2}$.
  
  \item Calculer l'inverse de $\left(\begin{smallmatrix}1&0&0\\0&2&0\\1&0&3\end{smallmatrix}\right)$.
  
  \item Soit $A=\left(\begin{smallmatrix}-1&-2&0\\2&3&0\\0&0&1\end{smallmatrix}\right)$.
  Calculer $2A-A^2$. Sans calculs, en déduire $A^{-1}$.

\end{enumerate}
\end{miniexercice}

\end{frame}

\end{document}