
%%%%%%%%%%%%%%%%%% PREAMBULE %%%%%%%%%%%%%%%%%%

\documentclass[aspectratio=169,utf8]{beamer}
%\documentclass[aspectratio=169,handout]{beamer}

\usetheme{Boadilla}
%\usecolortheme{seahorse}
\usecolortheme[RGB={245,66,24}]{structure}
\useoutertheme{infolines}

% packages
\usepackage{amsfonts,amsmath,amssymb,amsthm}
\usepackage[utf8]{inputenc}
\usepackage[T1]{fontenc}
\usepackage{lmodern}

\usepackage[francais]{babel}
\usepackage{fancybox}
\usepackage{graphicx}

\usepackage{float}
\usepackage{xfrac}

%\usepackage[usenames, x11names]{xcolor}
\usepackage{tikz}
\usepackage{pgfplots}
\usepackage{datetime}



%-----  Package unités -----
\usepackage{siunitx}
\sisetup{locale = FR,detect-all,per-mode = symbol}

%\usepackage{mathptmx}
%\usepackage{fouriernc}
%\usepackage{newcent}
%\usepackage[mathcal,mathbf]{euler}

%\usepackage{palatino}
%\usepackage{newcent}
% \usepackage[mathcal,mathbf]{euler}



% \usepackage{hyperref}
% \hypersetup{colorlinks=true, linkcolor=blue, urlcolor=blue,
% pdftitle={Exo7 - Exercices de mathématiques}, pdfauthor={Exo7}}


%section
% \usepackage{sectsty}
% \allsectionsfont{\bf}
%\sectionfont{\color{Tomato3}\upshape\selectfont}
%\subsectionfont{\color{Tomato4}\upshape\selectfont}

%----- Ensembles : entiers, reels, complexes -----
\newcommand{\Nn}{\mathbb{N}} \newcommand{\N}{\mathbb{N}}
\newcommand{\Zz}{\mathbb{Z}} \newcommand{\Z}{\mathbb{Z}}
\newcommand{\Qq}{\mathbb{Q}} \newcommand{\Q}{\mathbb{Q}}
\newcommand{\Rr}{\mathbb{R}} \newcommand{\R}{\mathbb{R}}
\newcommand{\Cc}{\mathbb{C}} 
\newcommand{\Kk}{\mathbb{K}} \newcommand{\K}{\mathbb{K}}

%----- Modifications de symboles -----
\renewcommand{\epsilon}{\varepsilon}
\renewcommand{\Re}{\mathop{\text{Re}}\nolimits}
\renewcommand{\Im}{\mathop{\text{Im}}\nolimits}
%\newcommand{\llbracket}{\left[\kern-0.15em\left[}
%\newcommand{\rrbracket}{\right]\kern-0.15em\right]}

\renewcommand{\ge}{\geqslant}
\renewcommand{\geq}{\geqslant}
\renewcommand{\le}{\leqslant}
\renewcommand{\leq}{\leqslant}
\renewcommand{\epsilon}{\varepsilon}

%----- Fonctions usuelles -----
\newcommand{\ch}{\mathop{\text{ch}}\nolimits}
\newcommand{\sh}{\mathop{\text{sh}}\nolimits}
\renewcommand{\tanh}{\mathop{\text{th}}\nolimits}
\newcommand{\cotan}{\mathop{\text{cotan}}\nolimits}
\newcommand{\Arcsin}{\mathop{\text{arcsin}}\nolimits}
\newcommand{\Arccos}{\mathop{\text{arccos}}\nolimits}
\newcommand{\Arctan}{\mathop{\text{arctan}}\nolimits}
\newcommand{\Argsh}{\mathop{\text{argsh}}\nolimits}
\newcommand{\Argch}{\mathop{\text{argch}}\nolimits}
\newcommand{\Argth}{\mathop{\text{argth}}\nolimits}
\newcommand{\pgcd}{\mathop{\text{pgcd}}\nolimits} 


%----- Commandes divers ------
\newcommand{\ii}{\mathrm{i}}
\newcommand{\dd}{\text{d}}
\newcommand{\id}{\mathop{\text{id}}\nolimits}
\newcommand{\Ker}{\mathop{\text{Ker}}\nolimits}
\newcommand{\Card}{\mathop{\text{Card}}\nolimits}
\newcommand{\Vect}{\mathop{\text{Vect}}\nolimits}
\newcommand{\Mat}{\mathop{\text{Mat}}\nolimits}
\newcommand{\rg}{\mathop{\text{rg}}\nolimits}
\newcommand{\tr}{\mathop{\text{tr}}\nolimits}


%----- Structure des exercices ------

\newtheoremstyle{styleexo}% name
{2ex}% Space above
{3ex}% Space below
{}% Body font
{}% Indent amount 1
{\bfseries} % Theorem head font
{}% Punctuation after theorem head
{\newline}% Space after theorem head 2
{}% Theorem head spec (can be left empty, meaning ‘normal’)

%\theoremstyle{styleexo}
\newtheorem{exo}{Exercice}
\newtheorem{ind}{Indications}
\newtheorem{cor}{Correction}


\newcommand{\exercice}[1]{} \newcommand{\finexercice}{}
%\newcommand{\exercice}[1]{{\tiny\texttt{#1}}\vspace{-2ex}} % pour afficher le numero absolu, l'auteur...
\newcommand{\enonce}{\begin{exo}} \newcommand{\finenonce}{\end{exo}}
\newcommand{\indication}{\begin{ind}} \newcommand{\finindication}{\end{ind}}
\newcommand{\correction}{\begin{cor}} \newcommand{\fincorrection}{\end{cor}}

\newcommand{\noindication}{\stepcounter{ind}}
\newcommand{\nocorrection}{\stepcounter{cor}}

\newcommand{\fiche}[1]{} \newcommand{\finfiche}{}
\newcommand{\titre}[1]{\centerline{\large \bf #1}}
\newcommand{\addcommand}[1]{}
\newcommand{\video}[1]{}

% Marge
\newcommand{\mymargin}[1]{\marginpar{{\small #1}}}

\def\noqed{\renewcommand{\qedsymbol}{}}


%----- Presentation ------
\setlength{\parindent}{0cm}

%\newcommand{\ExoSept}{\href{http://exo7.emath.fr}{\textbf{\textsf{Exo7}}}}

\definecolor{myred}{rgb}{0.93,0.26,0}
\definecolor{myorange}{rgb}{0.97,0.58,0}
\definecolor{myyellow}{rgb}{1,0.86,0}

\newcommand{\LogoExoSept}[1]{  % input : echelle
{\usefont{U}{cmss}{bx}{n}
\begin{tikzpicture}[scale=0.1*#1,transform shape]
  \fill[color=myorange] (0,0)--(4,0)--(4,-4)--(0,-4)--cycle;
  \fill[color=myred] (0,0)--(0,3)--(-3,3)--(-3,0)--cycle;
  \fill[color=myyellow] (4,0)--(7,4)--(3,7)--(0,3)--cycle;
  \node[scale=5] at (3.5,3.5) {Exo7};
\end{tikzpicture}}
}


\newcommand{\debutmontitre}{
  \author{} \date{} 
  \thispagestyle{empty}
  \hspace*{-10ex}
  \begin{minipage}{\textwidth}
    \titlepage  
  \vspace*{-2.5cm}
  \begin{center}
    \LogoExoSept{2.5}
  \end{center}
  \end{minipage}

  \vspace*{-0cm}
  
  % Astuce pour que le background ne soit pas discrétisé lors de la conversion pdf -> png
\begin{tikzpicture}
        \fill[opacity=0,green!60!black] (0,0)--++(0,0)--++(0,0)--++(0,0)--cycle; 
\end{tikzpicture}

% toc S'affiche trop tot :
% \tableofcontents[hideallsubsections, pausesections]
}

\newcommand{\finmontitre}{
  \end{frame}
  \setcounter{framenumber}{0}
} % ne marche pas pour une raison obscure

%----- Commandes supplementaires ------

% \usepackage[landscape]{geometry}
% \geometry{top=1cm, bottom=3cm, left=2cm, right=10cm, marginparsep=1cm
% }
% \usepackage[a4paper]{geometry}
% \geometry{top=2cm, bottom=2cm, left=2cm, right=2cm, marginparsep=1cm
% }

%\usepackage{standalone}


% New command Arnaud -- november 2011
\setbeamersize{text margin left=24ex}
% si vous modifier cette valeur il faut aussi
% modifier le decalage du titre pour compenser
% (ex : ici =+10ex, titre =-5ex

\theoremstyle{definition}
%\newtheorem{proposition}{Proposition}
%\newtheorem{exemple}{Exemple}
%\newtheorem{theoreme}{Théorème}
%\newtheorem{lemme}{Lemme}
%\newtheorem{corollaire}{Corollaire}
%\newtheorem*{remarque*}{Remarque}
%\newtheorem*{miniexercice}{Mini-exercices}
%\newtheorem{definition}{Définition}

% Commande tikz
\usetikzlibrary{calc}
\usetikzlibrary{patterns,arrows}
\usetikzlibrary{matrix}
\usetikzlibrary{fadings} 

%definition d'un terme
\newcommand{\defi}[1]{{\color{myorange}\textbf{\emph{#1}}}}
\newcommand{\evidence}[1]{{\color{blue}\textbf{\emph{#1}}}}
\newcommand{\assertion}[1]{\emph{\og#1\fg}}  % pour chapitre logique
%\renewcommand{\contentsname}{Sommaire}
\renewcommand{\contentsname}{}
\setcounter{tocdepth}{2}



%------ Figures ------

\def\myscale{1} % par défaut 
\newcommand{\myfigure}[2]{  % entrée : echelle, fichier figure
\def\myscale{#1}
\begin{center}
\footnotesize
{#2}
\end{center}}


%------ Encadrement ------

\usepackage{fancybox}


\newcommand{\mybox}[1]{
\setlength{\fboxsep}{7pt}
\begin{center}
\shadowbox{#1}
\end{center}}

\newcommand{\myboxinline}[1]{
\setlength{\fboxsep}{5pt}
\raisebox{-10pt}{
\shadowbox{#1}
}
}

%--------------- Commande beamer---------------
\newcommand{\beameronly}[1]{#1} % permet de mettre des pause dans beamer pas dans poly


\setbeamertemplate{navigation symbols}{}
\setbeamertemplate{footline}  % tiré du fichier beamerouterinfolines.sty
{
  \leavevmode%
  \hbox{%
  \begin{beamercolorbox}[wd=.333333\paperwidth,ht=2.25ex,dp=1ex,center]{author in head/foot}%
    % \usebeamerfont{author in head/foot}\insertshortauthor%~~(\insertshortinstitute)
    \usebeamerfont{section in head/foot}{\bf\insertshorttitle}
  \end{beamercolorbox}%
  \begin{beamercolorbox}[wd=.333333\paperwidth,ht=2.25ex,dp=1ex,center]{title in head/foot}%
    \usebeamerfont{section in head/foot}{\bf\insertsectionhead}
  \end{beamercolorbox}%
  \begin{beamercolorbox}[wd=.333333\paperwidth,ht=2.25ex,dp=1ex,right]{date in head/foot}%
    % \usebeamerfont{date in head/foot}\insertshortdate{}\hspace*{2em}
    \insertframenumber{} / \inserttotalframenumber\hspace*{2ex} 
  \end{beamercolorbox}}%
  \vskip0pt%
}


\definecolor{mygrey}{rgb}{0.5,0.5,0.5}
\setlength{\parindent}{0cm}
%\DeclareTextFontCommand{\helvetica}{\fontfamily{phv}\selectfont}

% background beamer
\definecolor{couleurhaut}{rgb}{0.85,0.9,1}  % creme
\definecolor{couleurmilieu}{rgb}{1,1,1}  % vert pale
\definecolor{couleurbas}{rgb}{0.85,0.9,1}  % blanc
\setbeamertemplate{background canvas}[vertical shading]%
[top=couleurhaut,middle=couleurmilieu,midpoint=0.4,bottom=couleurbas] 
%[top=fondtitre!05,bottom=fondtitre!60]



\makeatletter
\setbeamertemplate{theorem begin}
{%
  \begin{\inserttheoremblockenv}
  {%
    \inserttheoremheadfont
    \inserttheoremname
    \inserttheoremnumber
    \ifx\inserttheoremaddition\@empty\else\ (\inserttheoremaddition)\fi%
    \inserttheorempunctuation
  }%
}
\setbeamertemplate{theorem end}{\end{\inserttheoremblockenv}}

\newenvironment{theoreme}[1][]{%
   \setbeamercolor{block title}{fg=structure,bg=structure!40}
   \setbeamercolor{block body}{fg=black,bg=structure!10}
   \begin{block}{{\bf Th\'eor\`eme }#1}
}{%
   \end{block}%
}


\newenvironment{proposition}[1][]{%
   \setbeamercolor{block title}{fg=structure,bg=structure!40}
   \setbeamercolor{block body}{fg=black,bg=structure!10}
   \begin{block}{{\bf Proposition }#1}
}{%
   \end{block}%
}

\newenvironment{corollaire}[1][]{%
   \setbeamercolor{block title}{fg=structure,bg=structure!40}
   \setbeamercolor{block body}{fg=black,bg=structure!10}
   \begin{block}{{\bf Corollaire }#1}
}{%
   \end{block}%
}

\newenvironment{mydefinition}[1][]{%
   \setbeamercolor{block title}{fg=structure,bg=structure!40}
   \setbeamercolor{block body}{fg=black,bg=structure!10}
   \begin{block}{{\bf Définition} #1}
}{%
   \end{block}%
}

\newenvironment{lemme}[0]{%
   \setbeamercolor{block title}{fg=structure,bg=structure!40}
   \setbeamercolor{block body}{fg=black,bg=structure!10}
   \begin{block}{\bf Lemme}
}{%
   \end{block}%
}

\newenvironment{remarque}[1][]{%
   \setbeamercolor{block title}{fg=black,bg=structure!20}
   \setbeamercolor{block body}{fg=black,bg=structure!5}
   \begin{block}{Remarque #1}
}{%
   \end{block}%
}


\newenvironment{exemple}[1][]{%
   \setbeamercolor{block title}{fg=black,bg=structure!20}
   \setbeamercolor{block body}{fg=black,bg=structure!5}
   \begin{block}{{\bf Exemple }#1}
}{%
   \end{block}%
}


\newenvironment{miniexercice}[0]{%
   \setbeamercolor{block title}{fg=structure,bg=structure!20}
   \setbeamercolor{block body}{fg=black,bg=structure!5}
   \begin{block}{Mini-exercices}
}{%
   \end{block}%
}


\newenvironment{tp}[0]{%
   \setbeamercolor{block title}{fg=structure,bg=structure!40}
   \setbeamercolor{block body}{fg=black,bg=structure!10}
   \begin{block}{\bf Travaux pratiques}
}{%
   \end{block}%
}
\newenvironment{exercicecours}[1][]{%
   \setbeamercolor{block title}{fg=structure,bg=structure!40}
   \setbeamercolor{block body}{fg=black,bg=structure!10}
   \begin{block}{{\bf Exercice }#1}
}{%
   \end{block}%
}
\newenvironment{algo}[1][]{%
   \setbeamercolor{block title}{fg=structure,bg=structure!40}
   \setbeamercolor{block body}{fg=black,bg=structure!10}
   \begin{block}{{\bf Algorithme}\hfill{\color{gray}\texttt{#1}}}
}{%
   \end{block}%
}


\setbeamertemplate{proof begin}{
   \setbeamercolor{block title}{fg=black,bg=structure!20}
   \setbeamercolor{block body}{fg=black,bg=structure!5}
   \begin{block}{{\footnotesize Démonstration}}
   \footnotesize
   \smallskip}
\setbeamertemplate{proof end}{%
   \end{block}}
\setbeamertemplate{qed symbol}{\openbox}


\makeatother
\usecolortheme[RGB={191,146,10}]{structure}

%%%%%%%%%%%%%%%%%%%%%%%%%%%%%%%%%%%%%%%%%%%%%%%%%%%%%%%%%%%%%
%%%%%%%%%%%%%%%%%%%%%%%%%%%%%%%%%%%%%%%%%%%%%%%%%%%%%%%%%%%%%

\begin{document}


\title{{\bf Matrices}}
\subtitle{Inverse d'une matrice : calcul}

\begin{frame}
  
  \debutmontitre

  \pause

{\footnotesize
\hfill
\setbeamercovered{transparent=50}
\begin{minipage}{0.6\textwidth}
  \begin{itemize}
    \item<3-> Matrices $2 \times 2$
    \item<4-> Méthode de Gauss
    \item<5-> Un exemple
  \end{itemize}
\end{minipage}
}

\end{frame}

\setcounter{framenumber}{0}


%%%%%%%%%%%%%%%%%%%%%%%%%%%%%%%%%%%%%%%%%%%%%%%%%%%%%%%%%%%%%%%%
\section{Matrices $2 \times 2$}

\begin{frame}
Soit $A = \begin{pmatrix}
 a & b\\
 c & d       
     \end{pmatrix}
$ 
une matrice $2\times 2$

\pause

\begin{proposition}
Si $ad - bc \not= 0$,  alors $A$ est inversible et
\mybox{$A^{-1} = \frac{1}{ad-bc} \begin{pmatrix}
d & -b\\
 -c & a
\end{pmatrix}$}
\end{proposition}

\pause

\begin{proof}
\begin{itemize}
\item on pose $B=\frac{1}{ad-bc}  \left(\begin{smallmatrix}
d & -b\\
 -c & a
\end{smallmatrix}\right)$
\item\pause on vérifie alors $AB = \left(\begin{smallmatrix}
 1 & 0\\
 0 & 1
 \end{smallmatrix}\right)$
\item\pause de même $BA=\left(\begin{smallmatrix}
 1 & 0\\
 0 & 1
 \end{smallmatrix}\right)$ \qedhere
\end{itemize}
\end{proof}
\end{frame}


%%%%%%%%%%%%%%%%%%%%%%%%%%%%%%%%%%%%%%%%%%%%%%%%%%%%%%%%%%%%%%%%
\section{Méthode de Gauss pour inverser les matrices}
 
 
\begin{frame}
\textbf{Méthode pour inverser $A$}
\begin{itemize}
\item\pause par des opérations élémentaires sur les lignes de $A$, transformer $A$ en la matrice identité $\Id$
\item\pause faire simultanément les mêmes opérations sur $\Id$
\item\pause à la fin, $A$ est transformée en $\Id$, et $\Id$ est transformée en $A^{-1}$
\end{itemize}

\pause
On forme le tableau
\vspace{-.1cm}
\[
(A\ |\ \Id) \pause \longrightarrow (\Id\ |\ B) \pause =  (\Id\ |\ A^{-1}) 
\]

\pause
\textbf{Opérations élémentaires sur les lignes}

\begin{enumerate}
  \item\pause \evidence{$L_i \leftarrow \lambda L_i$} avec $\lambda \neq 0$ : 
  multiplier une ligne par un réel non nul 
      
  \item\pause \evidence{$L_i \leftarrow L_i+\lambda L_j$} avec $\lambda \in \Kk$ ($j\neq i$) :
  ajouter à la ligne $L_i$ un multiple de $L_j$
  
  \item\pause \evidence{$L_i \leftrightarrow L_j$} : échanger deux lignes
\end{enumerate}
\end{frame}


%%%%%%%%%%%%%%%%%%%%%%%%%%%%%%%%%%%%%%%%%%%%%%%%%%%%%%%%%%%%%%%%
\section{Un exemple}
 
 
\begin{frame}
Calculons l'inverse de
$
 A = \begin{pmatrix}
 1 & 2 & 1\\
 4 & 0 & -1\\
 -1 & 2 & 2       
\end{pmatrix}
$
\pause

Voici la matrice augmentée
%
$$\begin{array}{rcl}
(A\ |\ I_3) &=&  \left(\begin{array}{ccc|ccc}
1 & 2 & 1 & 1 & 0 & 0\\
4 & 0 & -1 & 0 & 1 & 0\\
-1 & 2 & 2 & 0 & 0 & 1
\end{array}\right)
\begin{array}{l} {\scriptstyle L_1}  \\ {\scriptstyle L_2} \\  {\scriptstyle L_3}\end{array}\\[7mm]
%
\pause \pause
\onslide<3->&&\left(\begin{array}{ccc|ccc}
1 & 2 & 1 & 1 & 0 & 0\\
\uncover<5->{0 & -8 & -5} &\uncover<6->{ -4 & 1 & 0}\\
-1 & 2 & 2 & 0 & 0 & 1
\end{array}\right)
\onslide<4->\begin{array}{l} ~ \\  {\scriptstyle L_2 \leftarrow L_2 - 4 L_1} \\ ~ \end{array}\\[7mm]
%
\pause \pause
\onslide<7->&&\left(\begin{array}{ccc|ccc}
1 & 2 & 1 & 1 & 0 & 0\\
0 & -8 & -5 & -4 & 1 & 0\\
\uncover<9->{0 & 4 & 3} &\uncover<10->{ 1 & 0 & 1}
\end{array}\right)
\onslide<8->\begin{array}{l} ~ \\ ~ \\ {\scriptstyle L_3 \leftarrow L_3 + L_1}   \end{array}
\end{array}$$

\end{frame}

%--------------------------------------------------------------

\begin{frame}
$$\begin{array}{l}
\left(\begin{array}{ccc|ccc}
1 & 2 & 1 & 1 & 0 & 0\\
0 & -8 & -5 & -4 & 1 & 0\\
0 & 4 & 3 & 1 & 0 & 1
\end{array}\right)\\[9mm]
%
\pause
\left(\begin{array}{ccc|ccc}
1 & 2 & 1 & 1 & 0 & 0\\
\uncover<3->{0 & 1 & \frac58} & \uncover<3->{\frac12 & -\frac18 & 0}\\
0 & 4 & 3 & 1 & 0 & 1
\end{array}\right)
\begin{array}{l} ~ \\  {\scriptstyle L_2 \leftarrow -\frac{1}{8} L_2} \\ ~ \end{array}\\[9mm]
%
\pause\pause 
\left(\begin{array}{ccc|ccc}
1 & 2 & 1 & 1 & 0 & 0\\
0 & 1 & \frac58 & \frac12 & -\frac18 & 0\\
\uncover<5->{0 & 0 & \frac12} & \uncover<5->{-1 & \frac12 & 1}
\end{array}\right)
\begin{array}{l} ~ \\ ~ \\ {\scriptstyle L_3 \leftarrow L_3 -4L_2}   \end{array}\\[9mm]
%
\pause\pause
\left(\begin{array}{ccc|ccc}
1 & 2 & 1 & 1 & 0 & 0\\
0 & 1 & \frac58 & \frac12 & -\frac18 & 0\\
\uncover<7->{0 & 0 & 1} & \uncover<7->{-2 & 1 & 2}
\end{array}\right)
\begin{array}{l} ~ \\ ~ \\ {\scriptstyle L_3 \leftarrow 2L_3}   \end{array}
\end{array}$$

\end{frame}



\begin{frame}
\vspace*{-2ex}
$$
\begin{array}{l}
\left(\begin{array}{ccc|ccc}
1 & 2 & 1 & 1 & 0 & 0\\
0 & 1 & \frac58 & \frac12 & -\frac18 & 0\\
0 & 0 & 1 & -2 & 1 & 2
\end{array}\right)\\[9mm]
%
\pause
\left(\begin{array}{ccc|ccc}
1 & 2 & 1 & 1 & 0 & 0\\
0 & 1 & 0 & \frac{7}{4} & -\frac{3}{4} & -\frac{5}{4}\\
0 & 0 & 1 & -2 & 1 & 2
\end{array}\right)
\begin{array}{l} ~ \\  {\scriptstyle L_2 \leftarrow L_2-\frac{5}{8} L_3} \\ ~ \end{array}\\[9mm]
%
\pause 
\left(\begin{array}{ccc|ccc}
1 & 0 & 0 & -\frac{1}{2} & \frac{1}{2} & \frac{1}{2}\\
0 & 1 & 0 &\frac{7}{4} & -\frac{3}{4} & -\frac{5}{4}\\
0 & 0 & 1 & -2 & 1 & 2
\end{array}\right)
\begin{array}{l}  {\scriptstyle L_1 \leftarrow L_1-2L_2-L_3} \\ ~ \\ ~ \end{array}
\end{array}$$

\medskip
\pause
$\text{Donc}
\quad 
A^{-1} = \displaystyle\frac{1}{4} 
\begin{pmatrix}
-2 & 2 & 2\\
7 & -3 & -5\\
-8 & 4 & 8                       
\end{pmatrix}
$

\medskip\smallskip
\pause
N'oubliez pas de vérifier $A \times A^{-1} = I_3$

\end{frame}


 



%%%%%%%%%%%%%%%%%%%%%%%%%%%%%%%%%%%%%%%%%%%%%%%%%%%%%%%%%%%%%%%%
\section{Mini-exercices}

\begin{frame}

\begin{miniexercice}
\begin{enumerate}
  \item Si possible calculer l'inverse des matrices :
  $\left(\begin{smallmatrix}3&1\\7&2\end{smallmatrix}\right)$,
  $\left(\begin{smallmatrix}2&-3\\-5&4\end{smallmatrix}\right)$,
  $\left(\begin{smallmatrix}0&2\\3&0\end{smallmatrix}\right)$,  
  $\left(\begin{smallmatrix}\alpha+1&1\\2&\alpha\end{smallmatrix}\right)$.
  
  \item Soit 
  $A(\theta)=\left(\begin{smallmatrix} \cos \theta & -\sin\theta \\ \sin\theta & \cos\theta\end{smallmatrix}\right)$.
  Calculer $A(\theta)^{-1}$.
  
  \item Calculer l'inverse des matrices :
  $\left(\begin{smallmatrix}1&3&0\\2&1&-1\\-2&1&1\end{smallmatrix}\right)$, 
  $\left(\begin{smallmatrix}2&-2&1\\3&0&5\\1&1&2\end{smallmatrix}\right)$, 
  $\left(\begin{smallmatrix}1&0&1&0\\0&2&-2&0\\-1&2&0&1\\0&2&1&3\end{smallmatrix}\right)$, 
  $\left(\begin{smallmatrix}2&1&1&1\\1&0&0&1\\0&1&-1&2\\0&1&1&0\end{smallmatrix}\right)$, 
  $\left(\begin{smallmatrix}1&1&1&0&0\\0&1&2&0&0\\-1&1&2&0&0\\0&0&0&2&1\\0&0&0&5&3\end{smallmatrix}\right)$.

\end{enumerate}
\end{miniexercice}

\end{frame}

\end{document}