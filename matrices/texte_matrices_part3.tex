
%%%%%%%%%%%%%%%%%% PREAMBULE %%%%%%%%%%%%%%%%%%


\documentclass[12pt]{article}

\usepackage{amsfonts,amsmath,amssymb,amsthm}
\usepackage[utf8]{inputenc}
\usepackage[T1]{fontenc}
\usepackage[francais]{babel}


% packages
\usepackage{amsfonts,amsmath,amssymb,amsthm}
\usepackage[utf8]{inputenc}
\usepackage[T1]{fontenc}
%\usepackage{lmodern}

\usepackage[francais]{babel}
\usepackage{fancybox}
\usepackage{graphicx}

\usepackage{float}

%\usepackage[usenames, x11names]{xcolor}
\usepackage{tikz}
\usepackage{datetime}

\usepackage{mathptmx}
%\usepackage{fouriernc}
%\usepackage{newcent}
\usepackage[mathcal,mathbf]{euler}

%\usepackage{palatino}
%\usepackage{newcent}


% Commande spéciale prompteur

%\usepackage{mathptmx}
%\usepackage[mathcal,mathbf]{euler}
%\usepackage{mathpple,multido}

\usepackage[a4paper]{geometry}
\geometry{top=2cm, bottom=2cm, left=1cm, right=1cm, marginparsep=1cm}

\newcommand{\change}{{\color{red}\rule{\textwidth}{1mm}\\}}

\newcounter{mydiapo}

\newcommand{\diapo}{\newpage
\hfill {\normalsize  Diapo \themydiapo \quad \texttt{[\jobname]}} \\
\stepcounter{mydiapo}}


%%%%%%% COULEURS %%%%%%%%%%

% Pour blanc sur noir :
%\pagecolor[rgb]{0.5,0.5,0.5}
% \pagecolor[rgb]{0,0,0}
% \color[rgb]{1,1,1}



%\DeclareFixedFont{\myfont}{U}{cmss}{bx}{n}{18pt}
\newcommand{\debuttexte}{
%%%%%%%%%%%%% FONTES %%%%%%%%%%%%%
\renewcommand{\baselinestretch}{1.5}
\usefont{U}{cmss}{bx}{n}
\bfseries

% Taille normale : commenter le reste !
%Taille Arnaud
%\fontsize{19}{19}\selectfont

% Taille Barbara
%\fontsize{21}{22}\selectfont

%Taille François
\fontsize{25}{30}\selectfont

%Taille Pascal
%\fontsize{25}{30}\selectfont

%Taille Laura
%\fontsize{30}{35}\selectfont


%\myfont
%\usefont{U}{cmss}{bx}{n}

%\Huge
%\addtolength{\parskip}{\baselineskip}
}


% \usepackage{hyperref}
% \hypersetup{colorlinks=true, linkcolor=blue, urlcolor=blue,
% pdftitle={Exo7 - Exercices de mathématiques}, pdfauthor={Exo7}}


%section
% \usepackage{sectsty}
% \allsectionsfont{\bf}
%\sectionfont{\color{Tomato3}\upshape\selectfont}
%\subsectionfont{\color{Tomato4}\upshape\selectfont}

%----- Ensembles : entiers, reels, complexes -----
\newcommand{\Nn}{\mathbb{N}} \newcommand{\N}{\mathbb{N}}
\newcommand{\Zz}{\mathbb{Z}} \newcommand{\Z}{\mathbb{Z}}
\newcommand{\Qq}{\mathbb{Q}} \newcommand{\Q}{\mathbb{Q}}
\newcommand{\Rr}{\mathbb{R}} \newcommand{\R}{\mathbb{R}}
\newcommand{\Cc}{\mathbb{C}} 
\newcommand{\Kk}{\mathbb{K}} \newcommand{\K}{\mathbb{K}}

%----- Modifications de symboles -----
\renewcommand{\epsilon}{\varepsilon}
\renewcommand{\Re}{\mathop{\text{Re}}\nolimits}
\renewcommand{\Im}{\mathop{\text{Im}}\nolimits}
%\newcommand{\llbracket}{\left[\kern-0.15em\left[}
%\newcommand{\rrbracket}{\right]\kern-0.15em\right]}

\renewcommand{\ge}{\geqslant}
\renewcommand{\geq}{\geqslant}
\renewcommand{\le}{\leqslant}
\renewcommand{\leq}{\leqslant}

%----- Fonctions usuelles -----
\newcommand{\ch}{\mathop{\mathrm{ch}}\nolimits}
\newcommand{\sh}{\mathop{\mathrm{sh}}\nolimits}
\renewcommand{\tanh}{\mathop{\mathrm{th}}\nolimits}
\newcommand{\cotan}{\mathop{\mathrm{cotan}}\nolimits}
\newcommand{\Arcsin}{\mathop{\mathrm{Arcsin}}\nolimits}
\newcommand{\Arccos}{\mathop{\mathrm{Arccos}}\nolimits}
\newcommand{\Arctan}{\mathop{\mathrm{Arctan}}\nolimits}
\newcommand{\Argsh}{\mathop{\mathrm{Argsh}}\nolimits}
\newcommand{\Argch}{\mathop{\mathrm{Argch}}\nolimits}
\newcommand{\Argth}{\mathop{\mathrm{Argth}}\nolimits}
\newcommand{\pgcd}{\mathop{\mathrm{pgcd}}\nolimits} 

\newcommand{\Card}{\mathop{\text{Card}}\nolimits}
\newcommand{\Ker}{\mathop{\text{Ker}}\nolimits}
\newcommand{\id}{\mathop{\text{id}}\nolimits}
\newcommand{\ii}{\mathrm{i}}
\newcommand{\dd}{\mathrm{d}}
\newcommand{\Vect}{\mathop{\text{Vect}}\nolimits}
\newcommand{\Mat}{\mathop{\mathrm{Mat}}\nolimits}
\newcommand{\rg}{\mathop{\text{rg}}\nolimits}
\newcommand{\tr}{\mathop{\text{tr}}\nolimits}
\newcommand{\ppcm}{\mathop{\text{ppcm}}\nolimits}

%----- Structure des exercices ------

\newtheoremstyle{styleexo}% name
{2ex}% Space above
{3ex}% Space below
{}% Body font
{}% Indent amount 1
{\bfseries} % Theorem head font
{}% Punctuation after theorem head
{\newline}% Space after theorem head 2
{}% Theorem head spec (can be left empty, meaning ‘normal’)

%\theoremstyle{styleexo}
\newtheorem{exo}{Exercice}
\newtheorem{ind}{Indications}
\newtheorem{cor}{Correction}


\newcommand{\exercice}[1]{} \newcommand{\finexercice}{}
%\newcommand{\exercice}[1]{{\tiny\texttt{#1}}\vspace{-2ex}} % pour afficher le numero absolu, l'auteur...
\newcommand{\enonce}{\begin{exo}} \newcommand{\finenonce}{\end{exo}}
\newcommand{\indication}{\begin{ind}} \newcommand{\finindication}{\end{ind}}
\newcommand{\correction}{\begin{cor}} \newcommand{\fincorrection}{\end{cor}}

\newcommand{\noindication}{\stepcounter{ind}}
\newcommand{\nocorrection}{\stepcounter{cor}}

\newcommand{\fiche}[1]{} \newcommand{\finfiche}{}
\newcommand{\titre}[1]{\centerline{\large \bf #1}}
\newcommand{\addcommand}[1]{}
\newcommand{\video}[1]{}

% Marge
\newcommand{\mymargin}[1]{\marginpar{{\small #1}}}



%----- Presentation ------
\setlength{\parindent}{0cm}

%\newcommand{\ExoSept}{\href{http://exo7.emath.fr}{\textbf{\textsf{Exo7}}}}

\definecolor{myred}{rgb}{0.93,0.26,0}
\definecolor{myorange}{rgb}{0.97,0.58,0}
\definecolor{myyellow}{rgb}{1,0.86,0}

\newcommand{\LogoExoSept}[1]{  % input : echelle
{\usefont{U}{cmss}{bx}{n}
\begin{tikzpicture}[scale=0.1*#1,transform shape]
  \fill[color=myorange] (0,0)--(4,0)--(4,-4)--(0,-4)--cycle;
  \fill[color=myred] (0,0)--(0,3)--(-3,3)--(-3,0)--cycle;
  \fill[color=myyellow] (4,0)--(7,4)--(3,7)--(0,3)--cycle;
  \node[scale=5] at (3.5,3.5) {Exo7};
\end{tikzpicture}}
}



\theoremstyle{definition}
%\newtheorem{proposition}{Proposition}
%\newtheorem{exemple}{Exemple}
%\newtheorem{theoreme}{Théorème}
\newtheorem{lemme}{Lemme}
\newtheorem{corollaire}{Corollaire}
%\newtheorem*{remarque*}{Remarque}
%\newtheorem*{miniexercice}{Mini-exercices}
%\newtheorem{definition}{Définition}




%definition d'un terme
\newcommand{\defi}[1]{{\color{myorange}\textbf{\emph{#1}}}}
\newcommand{\evidence}[1]{{\color{blue}\textbf{\emph{#1}}}}



 %----- Commandes divers ------

\newcommand{\codeinline}[1]{\texttt{#1}}

%%%%%%%%%%%%%%%%%%%%%%%%%%%%%%%%%%%%%%%%%%%%%%%%%%%%%%%%%%%%%
%%%%%%%%%%%%%%%%%%%%%%%%%%%%%%%%%%%%%%%%%%%%%%%%%%%%%%%%%%%%%



\begin{document}

\debuttexte


%%%%%%%%%%%%%%%%%%%%%%%%%%%%%%%%%%%%%%%%%%%%%%%%%%%%%%%%%%
\diapo

\change
Nous poursuivons cette leçon sur les matrices par un chapitre consacré à la définition de l'inverse.

\change
Nous commencerons par définir ce qu'est l'inverse d'une matrice,

\change
puis nous en verrons des exemples de calculs,

\change
et enfin, nous établirons les propriétés essentielles de l'inverse d'une matrice.

%%%%%%%%%%%%%%%%%%%%%%%%%%%%%%%%%%%%%%%%%%%%%%%%%%%%%%%%%%%
\diapo

Commençons par définir ce qu'est l'inverse d'une matrice, quand il existe.

Soit $A$ une matrice  carrée de taille $n \times n$. 

\change
S'il existe une matrice carrée $B$ de même taille $n \times n$ telle que les produits
 $AB$ et $ BA$ soient égaux à la matrice identité,
 alors on dit que $A$ est \defi{inversible}. 
 
 
\change
On appelle $B$ la matrice inverse de $A$ et on note cette matrice $B$ par $A^{-1}$.

\change
On verra plus tard qu'il suffit en fait de vérifier une seule des conditions $AB=I$ ou bien $BA=I$.

\change
Plus généralement, quand $A$ est inversible, pour tout entier $p$, on note : 
$A^{-p}=(A^{-1})^p =A^{-1} A^{-1} \cdots$ $ p$ fois.

\change
Terminons avec avec une notation : l'ensemble des matrices inversibles de $M_{n}(\Kk)$ est noté  $GL_{n}(\Kk)$.


%%%%%%%%%%%%%%%%%%%%%%%%%%%%%%%%%%%%%%%%%%%%%%%%%%%%%%%%%%
\diapo

Soit $A$ la matrice de coefficients $\left(\begin{smallmatrix}
1 & 2 \cr
0 & 3 \cr
\end{smallmatrix}\right)$. \'Etudier si $A$ est inversible...

\change
... c'est étudier l'existence d'une matrice 
$B$ à coefficients $a, b, c, d$ dans $\Kk$, telle que $AB=$ identité et $BA=$ identité.

\change
Or le produit $AB$ est égal, après calcul, à la matrice  $\begin{pmatrix}
a+2c & b+2d \cr
3c & 3d\cr
\end{pmatrix}$

\change
Et donc $AB=I$ si et seulement si cette matrice vaut 
$\begin{pmatrix}
1 & 0 \cr
0 & 1\cr
\end{pmatrix}
$

\change
Cette égalité équivaut au système  : 
$$\left \{ \begin{array}{l}
a+2c=1\\
b+2d=0\\
3c=0\\
3d=1\\
\end{array} \right .$$

Sa résolution est immédiate : $d=\frac13$ , $c=0$, $b=-\frac23$, $a=1$.

\change

Il n'y a donc qu'une seule matrice possible, à savoir 
$B=\left(\begin{smallmatrix}
1 & -\frac23 \cr
0 & \frac13\cr
\end{smallmatrix}\right)$.

\change
Pour prouver qu'elle convient, il faudrait aussi montrer l'égalité $BA=I$,
dont la vérification est laissée au lecteur.

\change

En conclusion la matrice $A$ est donc inversible et $A^{-1}= \begin{pmatrix}
1 & -\frac23 \cr
0 & \frac13\cr
\end{pmatrix}$.



%%%%%%%%%%%%%%%%%%%%%%%%%%%%%%%%%%%%%%%%%%%%%%%%%%%%%%%%%%%
\diapo

Toutes les matrices ne sont pas inversibles.
Voyons l'exemple de la matrice 
$ A = \left(\begin{smallmatrix}
3 & 0\\
5 & 0\end{smallmatrix}\right)$
qui n'est pas inversible. 

\change
En effet, soit
$B= \begin{pmatrix}
a & b \cr
c & d\cr
\end{pmatrix}$ une matrice quelconque. 
Alors le produit $ BA$ vaut  $
\begin{pmatrix}
3a+5b & 0\\
3c+5d      & 0
\end{pmatrix}$

\change
qui ne peut jamais être égal à la matrice identité,

car ici le coefficient nul ne pourra jamais être $1$.

\change

Par contre $I_{n}$, la matrice carrée identité de taille $n\times n$,
est toujours une matrice inversible. Son inverse est elle-même par l'égalité $I_{n}I_{n}=I_{n}$.
 
\change
A l'opposé, la matrice nulle $0$ de taille $n \times n$ n'est jamais inversible. 
En effet on sait que, pour toute matrice $B$ de $M_{n}(\Kk)$, on a $B0=0$, qui ne peut jamais être la matrice identité.


%%%%%%%%%%%%%%%%%%%%%%%%%%%%%%%%%%%%%%%%%%%%%%%%%%%%%%%%%%
\diapo

Passons à présent à une propriété  :  l'unicité de l'inverse.

"Si $A$ est inversible, alors son inverse est unique."

\change
La méthode classique pour mener à bien une telle démonstration est de 
supposer l'existence de deux matrices $B_{1}$ et $B_{2}$ 
satisfaisant aux conditions imposées et de démontrer que l'on a nécessairement $B_{1}=B_{2}$.

Soient donc $B_{1}$ une matrice telle que $AB_{1}=B_{1}A=I_{n}$ et $B_{2}$ une matrice telle que 
$AB_{2}=B_{2}A=I_{n}$.

\change
Calculons le produit $B_{2}(AB_{1})$ de deux façons différentes.

\change
D'une part, comme $AB_{1}=I_{n}$, 

on a  $B_{2}(AB_{1})=B_{2}\times I_n$

\change
$=B_2.$

\change
D'autre part, comme le produit des matrices est associatif, on a 
$B_{2}(AB_{1})=(B_{2}A)B_{1}$

\change
qui vaut donc $I_{n}B_{1}=B_{1}$.

\change
Donc $B_{1}=B_{2}$. 


%%%%%%%%%%%%%%%%%%%%%%%%%%%%%%%%%%%%%%%%%%%%%%%%%%%%%%%%%%%
\diapo

Une autre propriété de l'inverse.

Soit $A$ une matrice inversible. Alors $A^{-1}$ est aussi inversible 

\change
et son inverse est $A$, c'est-à-dire que $(A^{-1})^{-1}=A$

En d'autres termes l'inverse de l'inverse est la matrice elle même.
% 
% \change
% Ce résultat est cohérent avec la notion puissance $-1$ pour l'inverse, puisqu'on a bien $(A^{-1})^{-1}=A^{(-1)\times(-1)}=A$


%%%%%%%%%%%%%%%%%%%%%%%%%%%%%%%%%%%%%%%%%%%%%%%%%%%%%%%%%%
\diapo

Voyons à présent comment s'exprime l'inverse d'un produit.

Soient $A$ et $B$ deux matrices carrées ayant la même taille et toutes deux inversibles. 
Alors leur produit $AB$ est également inversible 

\change
et on a la formule $\displaystyle (AB)^{-1} = B^{-1} A^{-1}$.

\change
Il faut bien faire attention à l'inversion de l'ordre !

\change
Il suffit de montrer que les deux produits $(B^{-1}A^{-1}) (AB) $ et $(AB) (B^{-1} A^{-1}) $ sont égaux à l'identité. 

Commençons par montrer que $(B^{-1}A^{-1}) (AB) =I$.

\change
Par associativité, ce produit vaut $B^{-1}(AA^{-1})B$

\change
et par définition de l'inverse de $A$, on obtient donc $ B^{-1}IB$

\change
c'est-à-dire $B^{-1}B$,

\change
ce qui, encore une fois par définition de l'inverse, fait l'identité.

\change
On montre exactement de la même manière que  $ (AB)(B^{-1} A^{-1})  =I$.
 
\change
De façon analogue, si on a plusieurs matrices $A_1, \dots , A_m$ toutes inversibles, alors l'inverse du produit
$A_1 A_2 \cdots A_m$ est le produit $A_m^{-1} A_{m-1}^{-1} \cdots A^{-1}_1$.


%%%%%%%%%%%%%%%%%%%%%%%%%%%%%%%%%%%%%%%%%%%%%%%%%%%%%%%%%%%
\diapo

Si $C$ est une matrice quelconque, nous savons que la relation  $AC=BC$
 n'entraîne *pas* forcément l'égalité $A=B$.

En revanche, si $C$ est une matrice *inversible*, nous allons voir que c'est vérifié.

\change
Proposition. Soient $A$ et $B$ deux matrices de taille $n\times n$ et $C$ une matrice
*inversible* de taille $n\times n$.

\change
Alors l'égalité $AC=BC$ implique l'égalité $A=B$.

\change
La preuve est immédiate : si on multiplie à droite l'égalité $AC=BC$ par l'inverse de $C$, 

\change
on obtient l'égalité : $(AC)C^{-1}=(BC)C^{-1}$. 

\change
En utilisant l'associativité on a $A(CC^{-1})=B(CC^{-1})$,

\change
ce qui donne $AI=BI$, 

\change
d'où $A=B$.



%%%%%%%%%%%%%%%%%%%%%%%%%%%%%%%%%%%%%%%%%%%%%%%%%%%%%%%%%%%
\diapo

Quelques petits calculs d'inverse pour terminer !


\end{document}
