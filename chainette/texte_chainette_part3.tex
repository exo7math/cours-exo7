
%%%%%%%%%%%%%%%%%% PREAMBULE %%%%%%%%%%%%%%%%%%


\documentclass[12pt]{article}

\usepackage{amsfonts,amsmath,amssymb,amsthm}
\usepackage[utf8]{inputenc}
\usepackage[T1]{fontenc}
\usepackage[francais]{babel}


% packages
\usepackage{amsfonts,amsmath,amssymb,amsthm}
\usepackage[utf8]{inputenc}
\usepackage[T1]{fontenc}
%\usepackage{lmodern}

\usepackage[francais]{babel}
\usepackage{fancybox}
\usepackage{graphicx}

\usepackage{float}

%\usepackage[usenames, x11names]{xcolor}
\usepackage{tikz}
\usepackage{datetime}

\usepackage{mathptmx}
%\usepackage{fouriernc}
%\usepackage{newcent}
\usepackage[mathcal,mathbf]{euler}

%\usepackage{palatino}
%\usepackage{newcent}


% Commande spéciale prompteur

%\usepackage{mathptmx}
%\usepackage[mathcal,mathbf]{euler}
%\usepackage{mathpple,multido}

\usepackage[a4paper]{geometry}
\geometry{top=2cm, bottom=2cm, left=1cm, right=1cm, marginparsep=1cm}

\newcommand{\change}{{\color{red}\rule{\textwidth}{1mm}\\}}

\newcounter{mydiapo}

\newcommand{\diapo}{\newpage
\hfill {\normalsize  Diapo \themydiapo \quad \texttt{[\jobname]}} \\
\stepcounter{mydiapo}}


%%%%%%% COULEURS %%%%%%%%%%

% Pour blanc sur noir :
%\pagecolor[rgb]{0.5,0.5,0.5}
% \pagecolor[rgb]{0,0,0}
% \color[rgb]{1,1,1}



%\DeclareFixedFont{\myfont}{U}{cmss}{bx}{n}{18pt}
\newcommand{\debuttexte}{
%%%%%%%%%%%%% FONTES %%%%%%%%%%%%%
\renewcommand{\baselinestretch}{1.5}
\usefont{U}{cmss}{bx}{n}
\bfseries

% Taille normale : commenter le reste !
%Taille Arnaud
%\fontsize{19}{19}\selectfont

% Taille Barbara
%\fontsize{21}{22}\selectfont

%Taille François
\fontsize{25}{30}\selectfont

%Taille Pascal
%\fontsize{25}{30}\selectfont

%Taille Laura
%\fontsize{30}{35}\selectfont


%\myfont
%\usefont{U}{cmss}{bx}{n}

%\Huge
%\addtolength{\parskip}{\baselineskip}
}


% \usepackage{hyperref}
% \hypersetup{colorlinks=true, linkcolor=blue, urlcolor=blue,
% pdftitle={Exo7 - Exercices de mathématiques}, pdfauthor={Exo7}}


%section
% \usepackage{sectsty}
% \allsectionsfont{\bf}
%\sectionfont{\color{Tomato3}\upshape\selectfont}
%\subsectionfont{\color{Tomato4}\upshape\selectfont}

%----- Ensembles : entiers, reels, complexes -----
\newcommand{\Nn}{\mathbb{N}} \newcommand{\N}{\mathbb{N}}
\newcommand{\Zz}{\mathbb{Z}} \newcommand{\Z}{\mathbb{Z}}
\newcommand{\Qq}{\mathbb{Q}} \newcommand{\Q}{\mathbb{Q}}
\newcommand{\Rr}{\mathbb{R}} \newcommand{\R}{\mathbb{R}}
\newcommand{\Cc}{\mathbb{C}} 
\newcommand{\Kk}{\mathbb{K}} \newcommand{\K}{\mathbb{K}}

%----- Modifications de symboles -----
\renewcommand{\epsilon}{\varepsilon}
\renewcommand{\Re}{\mathop{\text{Re}}\nolimits}
\renewcommand{\Im}{\mathop{\text{Im}}\nolimits}
%\newcommand{\llbracket}{\left[\kern-0.15em\left[}
%\newcommand{\rrbracket}{\right]\kern-0.15em\right]}

\renewcommand{\ge}{\geqslant}
\renewcommand{\geq}{\geqslant}
\renewcommand{\le}{\leqslant}
\renewcommand{\leq}{\leqslant}

%----- Fonctions usuelles -----
\newcommand{\ch}{\mathop{\mathrm{ch}}\nolimits}
\newcommand{\sh}{\mathop{\mathrm{sh}}\nolimits}
\renewcommand{\tanh}{\mathop{\mathrm{th}}\nolimits}
\newcommand{\cotan}{\mathop{\mathrm{cotan}}\nolimits}
\newcommand{\Arcsin}{\mathop{\mathrm{Arcsin}}\nolimits}
\newcommand{\Arccos}{\mathop{\mathrm{Arccos}}\nolimits}
\newcommand{\Arctan}{\mathop{\mathrm{Arctan}}\nolimits}
\newcommand{\Argsh}{\mathop{\mathrm{Argsh}}\nolimits}
\newcommand{\Argch}{\mathop{\mathrm{Argch}}\nolimits}
\newcommand{\Argth}{\mathop{\mathrm{Argth}}\nolimits}
\newcommand{\pgcd}{\mathop{\mathrm{pgcd}}\nolimits} 

\newcommand{\Card}{\mathop{\text{Card}}\nolimits}
\newcommand{\Ker}{\mathop{\text{Ker}}\nolimits}
\newcommand{\id}{\mathop{\text{id}}\nolimits}
\newcommand{\ii}{\mathrm{i}}
\newcommand{\dd}{\mathrm{d}}
\newcommand{\Vect}{\mathop{\text{Vect}}\nolimits}
\newcommand{\Mat}{\mathop{\mathrm{Mat}}\nolimits}
\newcommand{\rg}{\mathop{\text{rg}}\nolimits}
\newcommand{\tr}{\mathop{\text{tr}}\nolimits}
\newcommand{\ppcm}{\mathop{\text{ppcm}}\nolimits}

%----- Structure des exercices ------

\newtheoremstyle{styleexo}% name
{2ex}% Space above
{3ex}% Space below
{}% Body font
{}% Indent amount 1
{\bfseries} % Theorem head font
{}% Punctuation after theorem head
{\newline}% Space after theorem head 2
{}% Theorem head spec (can be left empty, meaning ‘normal’)

%\theoremstyle{styleexo}
\newtheorem{exo}{Exercice}
\newtheorem{ind}{Indications}
\newtheorem{cor}{Correction}


\newcommand{\exercice}[1]{} \newcommand{\finexercice}{}
%\newcommand{\exercice}[1]{{\tiny\texttt{#1}}\vspace{-2ex}} % pour afficher le numero absolu, l'auteur...
\newcommand{\enonce}{\begin{exo}} \newcommand{\finenonce}{\end{exo}}
\newcommand{\indication}{\begin{ind}} \newcommand{\finindication}{\end{ind}}
\newcommand{\correction}{\begin{cor}} \newcommand{\fincorrection}{\end{cor}}

\newcommand{\noindication}{\stepcounter{ind}}
\newcommand{\nocorrection}{\stepcounter{cor}}

\newcommand{\fiche}[1]{} \newcommand{\finfiche}{}
\newcommand{\titre}[1]{\centerline{\large \bf #1}}
\newcommand{\addcommand}[1]{}
\newcommand{\video}[1]{}

% Marge
\newcommand{\mymargin}[1]{\marginpar{{\small #1}}}



%----- Presentation ------
\setlength{\parindent}{0cm}

%\newcommand{\ExoSept}{\href{http://exo7.emath.fr}{\textbf{\textsf{Exo7}}}}

\definecolor{myred}{rgb}{0.93,0.26,0}
\definecolor{myorange}{rgb}{0.97,0.58,0}
\definecolor{myyellow}{rgb}{1,0.86,0}

\newcommand{\LogoExoSept}[1]{  % input : echelle
{\usefont{U}{cmss}{bx}{n}
\begin{tikzpicture}[scale=0.1*#1,transform shape]
  \fill[color=myorange] (0,0)--(4,0)--(4,-4)--(0,-4)--cycle;
  \fill[color=myred] (0,0)--(0,3)--(-3,3)--(-3,0)--cycle;
  \fill[color=myyellow] (4,0)--(7,4)--(3,7)--(0,3)--cycle;
  \node[scale=5] at (3.5,3.5) {Exo7};
\end{tikzpicture}}
}



\theoremstyle{definition}
%\newtheorem{proposition}{Proposition}
%\newtheorem{exemple}{Exemple}
%\newtheorem{theoreme}{Théorème}
\newtheorem{lemme}{Lemme}
\newtheorem{corollaire}{Corollaire}
%\newtheorem*{remarque*}{Remarque}
%\newtheorem*{miniexercice}{Mini-exercices}
%\newtheorem{definition}{Définition}




%definition d'un terme
\newcommand{\defi}[1]{{\color{myorange}\textbf{\emph{#1}}}}
\newcommand{\evidence}[1]{{\color{blue}\textbf{\emph{#1}}}}



 %----- Commandes divers ------

\newcommand{\codeinline}[1]{\texttt{#1}}

%%%%%%%%%%%%%%%%%%%%%%%%%%%%%%%%%%%%%%%%%%%%%%%%%%%%%%%%%%%%%
%%%%%%%%%%%%%%%%%%%%%%%%%%%%%%%%%%%%%%%%%%%%%%%%%%%%%%%%%%%%%



\begin{document}

\debuttexte


%%%%%%%%%%%%%%%%%%%%%%%%%%%%%%%%%%%%%%%%%%%%%%%%%%%%%%%%%%%
\diapo

\change
Dans cette partie nous allons approfondir notre étude de la chaînette.

\change
Nous calculerons sa longueur

\change
Nous verrons comment on peut facilement calculer le paramètre $a$ d'une chaînette

\change
On en profitera pour donner une autre représentation de la chaînette,
par une équation paramétrique

\change
Puis ce qui est important dans la pratique c'est de calculer
quelles sont les forces de tension en chaque point.

\change
Enfin, nous terminerons par l'énoncé de deux problèmes pour aller plus loin.


%%%%%%%%%%%%%%%%%%%%%%%%%%%%%%%%%%%%%%%%%%%%%%%%%%%%%%%%%%%
\diapo

On commence par le calcul de la longueur.

Les données sont les suivantes, on se donne une chaînette
de paramètre $a$.


Alors la longueur de la portion de la chaînette  
entre le point le plus bas $(0,a)$ et le point d'abscisse $x_0$ est :
$$\ell = a \sh \frac{x_0}{a}.$$

Je vous rappelle que $sh$ est le sinus hyperbolique.




%%%%%%%%%%%%%%%%%%%%%%%%%%%%%%%%%%%%%%%%%%%%%%%%%%%%%%%%%%%
\diapo

Calculons donc cette longueur $\ell$.

\change
Je vous rappelle l'équation de la chaînette : $y(x) =a \ch \frac x a$.

\change
Par définition la longueur vaut
$\ell = \int_0^{x_0} \sqrt{1+y'(x)^2} dx.$

\change
Ainsi :
 $\ell = \int_0^{x_0} \sqrt{1+\sh^2 \tfrac x a} dx$
 car la dérivée de $\ch$ est $\sh$.
 
 \change
 En utilisant la relation $1+\sh^2 u = \ch^2 u$
 on a $\ell = \int_0^{x_0} \sqrt{\ch^2 \tfrac x a} dx$
 
 \change
 Cela donne $\int_0^{x_0} \ch \tfrac x a dx$
 
 \change
 et comme une primitive de $\ch x$ est $\sh x$ alors
 
 la longueur vaut
 le crochet $\left[ a \sh \tfrac x a \right]_0^{x_0}$
 
 \change
 Ce qui conduit bien à la formule
 $\ell =  a \sh \tfrac{x_0}{a}.$
 
%%%%%%%%%%%%%%%%%%%%%%%%%%%%%%%%%%%%%%%%%%%%%%%%%%%%%%%%%%%
\diapo

 
 \change
La chaînette ne dépend que du seul paramètre $a$.

\change
Ce paramètre $a$ vaut $a = \frac{T_h}{\mu g}$ et est fonction de la masse $\mu$  
du fil par unité de longueur, de la constante de gravitation $g$ et 
de la tension horizontale $T_h$, qui elle dépend de l'écartement
de deux points par lesquels passe la chaînette.
Ce qui fait qu'il n'est pas facile de calculer $a$ ainsi.

\change
Fixons deux points, pour simplifier nous supposerons qu'ils sont à la même 
hauteur (même ordonnée). Prenons une chaînette de longueur $2\ell$ fixée (et connue !).
Nous allons calculer le paramètre $a$ en fonction de la longueur $2\ell$
et de la flèche $h$. 

\change
La \defi{flèche} est la hauteur $h$ entre les deux points d'accroche
et le point le plus bas de la chaînette.


%%%%%%%%%%%%%%%%%%%%%%%%%%%%%%%%%%%%%%%%%%%%%%%%%%%%%%%%%%%
\diapo

Voici comment calculer simplement le paramètre $a$ à partir de la longueur
de la chaînette et de la flèche.

Proposition :
Pour une chaînette de longueur $2\ell$ et de flèche $h$ alors
$$a=\frac{\ell^2-h^2}{2h}.$$


\change
La preuve n'est pas compliquée.

On note $(+x_0, y_0)$ et $(-x_0, y_0)$ 
les coordonnées des points d'accroche.

\change
L'équation de la chaînette étant $y= a \ch \frac x a$,
alors $y_0 = a \ch \frac{x_0}{a}$ 

et par définition de la flèche cela vaut aussi $y_0 = a + h$.

\change
donc la flèche vaut $h = a \ch \frac{x_0}{a} -a$

\change
Quant à la longueur nous avons vu qu'elle vaut $2\ell =  2a \sh \left( \frac{x_0}{a} \right)$.

\change

On va calculer :
$\ell^2 - h^2$

\change
On remplace par les expressions  de $\ell$ et $h$ :

$\ell^2 - h^2 = a^2 \sh^2 \tfrac{x_0}{a} - \left( a \ch \tfrac{x_0}{a} - a \right)^2$

\change
On développe le carré :
    
\change
Puis on utilise la formule $\ch^2 u - \sh^2 u = 1$

qui permet de simplifier 
$\ell^2 - h^2 = 2a\left(-a+a \ch \tfrac{x_0}{a} \right)$
  
\change
Et on reconnaît l'expression de la flèche,
donc $\ell^2 - h^2 =2ah$. 

\change

Ainsi $\displaystyle a = \frac{\ell^2-h^2}{2h}$.


%%%%%%%%%%%%%%%%%%%%%%%%%%%%%%%%%%%%%%%%%%%%%%%%%%%%%%%%%%%
\diapo

Nous avons obtenu une équation cartésienne $y= a \ch \frac x a$

voici une équation paramétrique de la chaînette :
$$\left\{
\begin{array}{rcl}
x(t) &=& a \ln t \\
y(t) &=& \frac a 2 \left(t+\frac 1 t\right)
\end{array}
\right.
$$
ceci pour les $t>0$.

La preuve utilise la forme logarithmique de la fonction $\Argch$.


%%%%%%%%%%%%%%%%%%%%%%%%%%%%%%%%%%%%%%%%%%%%%%%%%%%%%%%%%%%
\diapo

Nous pouvons calculer la tension en un point $(x_0,y_0)$ de la chaînette.
On note $h$ la flèche correspondante et $\ell$ la longueur entre le point le plus bas et $(x_0,y_0)$.

\change

La \emph{tension horizontale} $T_h$ est constante et vaut :
  $$T_h = a \mu g = \frac{\ell^2-h^2}{2h} \mu g.$$

 
\change 
Le \emph{tension verticale} dépend du point où l'on se trouve,
par exemple en $(x_0,y_0)$ elle est : $$T_v= T_h \cdot \sh \frac{x_0}{a} = T_h \cdot \frac \ell a.$$
  
\change
Enfin la \emph{tension totale} au point $(x_0,y_0)$ est :
  $$T = \sqrt{T_h^2+T_v^2} = T_h \cdot \ch \frac{x_0}{a} 
  = T_h \cdot \frac{a+h}{a}.$$

La tension totale croît donc avec la hauteur du point.



%%%%%%%%%%%%%%%%%%%%%%%%%%%%%%%%%%%%%%%%%%%%%%%%%%%%%%%%%%%
\diapo


Voici les calculs des tensions.

\change

\change
On a vu dans une partie précédente que la tension horizontale est constante. 

\change
Et la formule $T_h = a \mu g$ provient de la définition même de la constante $a$.

\change
Enfin, la dernière égalité est donnée par la proposition qui permet le calcul du paramètre en fonction de la flèche.


\change
Pour la tension verticale 

\change
nous avons cette relation 
  $T_v(x_0) = T_h \cdot y'(x_0) =
  T_h \cdot \sh \frac{x_0}{a} = T_h \cdot \frac \ell a$
  démontrée lors du calcul de l'équation de la chaînette.
  
\change
On termine avec la tension totale.

\change
Le vecteur tension est $\vec T(x) = -T_h(x)\vec i - T_v(x) \vec j$,

\change
La tension totale est la norme de ce vecteur, donc 
au point d'abscisse $x_0$ c'est $T(x_0) = \|\vec T(x_0) \| 
  = \sqrt{T_h^2+T_v^2}$
  
 \change
 En remplaçant par les expression précédentes on trouve
 
 $T(x_0) = = T_h \sqrt{1+\sh^2 \frac{x_0}{a}}
  = T_h \cdot \ch \frac{x_0}{a} = T_h \cdot \frac{a+h}{a}.$
  
  La dernière égalité est juste le fait que 
  $y_0 = a+h = a \ch \frac{x_0}{a}$.



%%%%%%%%%%%%%%%%%%%%%%%%%%%%%%%%%%%%%%%%%%%%%%%%%%%%%%%%%%%
\diapo

Voici un premier problème que vous allez résoudre.

\change
On se donne deux poteaux distants d'une longueur $2x_0$ fixée 
et d'une hauteur suffisante.

\change
Parmi toutes les chaînettes passant par les sommets de ces poteaux, on cherche celle
qui a les forces de tensions minimales.

\change
Nous avons calculé que la tension totale vaut 
$$T_x(a) = a \mu g \cdot \ch \frac{x}{a}.$$

\change
Pour une chaînette donnée, c'est-à-dire pour une valeur de $a$ fixée, 
la tension est donc maximale au point d'accroche, ici en $x=x_0$.

car le cosinus hyperbolique est une fonction croissante sur $[0,+\infty[$. 

\change
Pour un $a$
fixé, la tension maximale est donc $T_{x_0}(a)$. 

\change
Notre problème, $x_0$ étant fixé, est de trouver
le $a$ qui minimise $T_{x_0}(a)$.


%%%%%%%%%%%%%%%%%%%%%%%%%%%%%%%%%%%%%%%%%%%%%%%%%%%%%%%%%%%
\diapo

Voici la liste des questions qui vont vous permettent de savoir quelle
forme doit avoir une chaînette afin d'avoir des tensions minimales.



%%%%%%%%%%%%%%%%%%%%%%%%%%%%%%%%%%%%%%%%%%%%%%%%%%%%%%%%%%%
\diapo

\change
Vous allez maintenant calculer que la courbe du câble 
d'un pont suspendu est une parabole.

\change
Un pont suspendu est composé d'un tablier sur lequel passe les voitures par exemples.
Ici sa longueur est grand $L$ et de masse grand $M$.

\change
Comment tient se pont ?

Un gros câble est accroché entre deux pylônes. 

\change
Et à ce gros câble sont accrochés plein de petits câbles verticaux,
appelés câbles de suspension, reliant le gros câble au tablier.

Le problème est de calculer l'équation $y(x)$ du câble. 



%%%%%%%%%%%%%%%%%%%%%%%%%%%%%%%%%%%%%%%%%%%%%%%%%%%%%%%%%%%
\diapo

Voici maintenant la liste des questions qui se termine par 
le calcul de l'équation du câble.

La démarche est très similaire aux calculs que 
l'on a faits pour la chaînette.

%%%%%%%%%%%%%%%%%%%%%%%%%%%%%%%%%%%%%%%%%%%%%%%%%%%%%%%%%%%
\diapo

Vous avez tous les ingrédients pour calculer 
une équation du câble du \emph{Golden Bridge} qui surplombe
la baie de San Francisco.

\end{document}
