
%%%%%%%%%%%%%%%%%% PREAMBULE %%%%%%%%%%%%%%%%%%

\documentclass[aspectratio=169,utf8]{beamer}
%\documentclass[aspectratio=169,handout]{beamer}

\usetheme{Boadilla}
%\usecolortheme{seahorse}
\usecolortheme[RGB={245,66,24}]{structure}
\useoutertheme{infolines}

% packages
\usepackage{amsfonts,amsmath,amssymb,amsthm}
\usepackage[utf8]{inputenc}
\usepackage[T1]{fontenc}
\usepackage{lmodern}

\usepackage[francais]{babel}
\usepackage{fancybox}
\usepackage{graphicx}

\usepackage{float}
\usepackage{xfrac}

%\usepackage[usenames, x11names]{xcolor}
\usepackage{tikz}
\usepackage{pgfplots}
\usepackage{datetime}



%-----  Package unités -----
\usepackage{siunitx}
\sisetup{locale = FR,detect-all,per-mode = symbol}

%\usepackage{mathptmx}
%\usepackage{fouriernc}
%\usepackage{newcent}
%\usepackage[mathcal,mathbf]{euler}

%\usepackage{palatino}
%\usepackage{newcent}
% \usepackage[mathcal,mathbf]{euler}



% \usepackage{hyperref}
% \hypersetup{colorlinks=true, linkcolor=blue, urlcolor=blue,
% pdftitle={Exo7 - Exercices de mathématiques}, pdfauthor={Exo7}}


%section
% \usepackage{sectsty}
% \allsectionsfont{\bf}
%\sectionfont{\color{Tomato3}\upshape\selectfont}
%\subsectionfont{\color{Tomato4}\upshape\selectfont}

%----- Ensembles : entiers, reels, complexes -----
\newcommand{\Nn}{\mathbb{N}} \newcommand{\N}{\mathbb{N}}
\newcommand{\Zz}{\mathbb{Z}} \newcommand{\Z}{\mathbb{Z}}
\newcommand{\Qq}{\mathbb{Q}} \newcommand{\Q}{\mathbb{Q}}
\newcommand{\Rr}{\mathbb{R}} \newcommand{\R}{\mathbb{R}}
\newcommand{\Cc}{\mathbb{C}} 
\newcommand{\Kk}{\mathbb{K}} \newcommand{\K}{\mathbb{K}}

%----- Modifications de symboles -----
\renewcommand{\epsilon}{\varepsilon}
\renewcommand{\Re}{\mathop{\text{Re}}\nolimits}
\renewcommand{\Im}{\mathop{\text{Im}}\nolimits}
%\newcommand{\llbracket}{\left[\kern-0.15em\left[}
%\newcommand{\rrbracket}{\right]\kern-0.15em\right]}

\renewcommand{\ge}{\geqslant}
\renewcommand{\geq}{\geqslant}
\renewcommand{\le}{\leqslant}
\renewcommand{\leq}{\leqslant}
\renewcommand{\epsilon}{\varepsilon}

%----- Fonctions usuelles -----
\newcommand{\ch}{\mathop{\text{ch}}\nolimits}
\newcommand{\sh}{\mathop{\text{sh}}\nolimits}
\renewcommand{\tanh}{\mathop{\text{th}}\nolimits}
\newcommand{\cotan}{\mathop{\text{cotan}}\nolimits}
\newcommand{\Arcsin}{\mathop{\text{arcsin}}\nolimits}
\newcommand{\Arccos}{\mathop{\text{arccos}}\nolimits}
\newcommand{\Arctan}{\mathop{\text{arctan}}\nolimits}
\newcommand{\Argsh}{\mathop{\text{argsh}}\nolimits}
\newcommand{\Argch}{\mathop{\text{argch}}\nolimits}
\newcommand{\Argth}{\mathop{\text{argth}}\nolimits}
\newcommand{\pgcd}{\mathop{\text{pgcd}}\nolimits} 


%----- Commandes divers ------
\newcommand{\ii}{\mathrm{i}}
\newcommand{\dd}{\text{d}}
\newcommand{\id}{\mathop{\text{id}}\nolimits}
\newcommand{\Ker}{\mathop{\text{Ker}}\nolimits}
\newcommand{\Card}{\mathop{\text{Card}}\nolimits}
\newcommand{\Vect}{\mathop{\text{Vect}}\nolimits}
\newcommand{\Mat}{\mathop{\text{Mat}}\nolimits}
\newcommand{\rg}{\mathop{\text{rg}}\nolimits}
\newcommand{\tr}{\mathop{\text{tr}}\nolimits}


%----- Structure des exercices ------

\newtheoremstyle{styleexo}% name
{2ex}% Space above
{3ex}% Space below
{}% Body font
{}% Indent amount 1
{\bfseries} % Theorem head font
{}% Punctuation after theorem head
{\newline}% Space after theorem head 2
{}% Theorem head spec (can be left empty, meaning ‘normal’)

%\theoremstyle{styleexo}
\newtheorem{exo}{Exercice}
\newtheorem{ind}{Indications}
\newtheorem{cor}{Correction}


\newcommand{\exercice}[1]{} \newcommand{\finexercice}{}
%\newcommand{\exercice}[1]{{\tiny\texttt{#1}}\vspace{-2ex}} % pour afficher le numero absolu, l'auteur...
\newcommand{\enonce}{\begin{exo}} \newcommand{\finenonce}{\end{exo}}
\newcommand{\indication}{\begin{ind}} \newcommand{\finindication}{\end{ind}}
\newcommand{\correction}{\begin{cor}} \newcommand{\fincorrection}{\end{cor}}

\newcommand{\noindication}{\stepcounter{ind}}
\newcommand{\nocorrection}{\stepcounter{cor}}

\newcommand{\fiche}[1]{} \newcommand{\finfiche}{}
\newcommand{\titre}[1]{\centerline{\large \bf #1}}
\newcommand{\addcommand}[1]{}
\newcommand{\video}[1]{}

% Marge
\newcommand{\mymargin}[1]{\marginpar{{\small #1}}}

\def\noqed{\renewcommand{\qedsymbol}{}}


%----- Presentation ------
\setlength{\parindent}{0cm}

%\newcommand{\ExoSept}{\href{http://exo7.emath.fr}{\textbf{\textsf{Exo7}}}}

\definecolor{myred}{rgb}{0.93,0.26,0}
\definecolor{myorange}{rgb}{0.97,0.58,0}
\definecolor{myyellow}{rgb}{1,0.86,0}

\newcommand{\LogoExoSept}[1]{  % input : echelle
{\usefont{U}{cmss}{bx}{n}
\begin{tikzpicture}[scale=0.1*#1,transform shape]
  \fill[color=myorange] (0,0)--(4,0)--(4,-4)--(0,-4)--cycle;
  \fill[color=myred] (0,0)--(0,3)--(-3,3)--(-3,0)--cycle;
  \fill[color=myyellow] (4,0)--(7,4)--(3,7)--(0,3)--cycle;
  \node[scale=5] at (3.5,3.5) {Exo7};
\end{tikzpicture}}
}


\newcommand{\debutmontitre}{
  \author{} \date{} 
  \thispagestyle{empty}
  \hspace*{-10ex}
  \begin{minipage}{\textwidth}
    \titlepage  
  \vspace*{-2.5cm}
  \begin{center}
    \LogoExoSept{2.5}
  \end{center}
  \end{minipage}

  \vspace*{-0cm}
  
  % Astuce pour que le background ne soit pas discrétisé lors de la conversion pdf -> png
\begin{tikzpicture}
        \fill[opacity=0,green!60!black] (0,0)--++(0,0)--++(0,0)--++(0,0)--cycle; 
\end{tikzpicture}

% toc S'affiche trop tot :
% \tableofcontents[hideallsubsections, pausesections]
}

\newcommand{\finmontitre}{
  \end{frame}
  \setcounter{framenumber}{0}
} % ne marche pas pour une raison obscure

%----- Commandes supplementaires ------

% \usepackage[landscape]{geometry}
% \geometry{top=1cm, bottom=3cm, left=2cm, right=10cm, marginparsep=1cm
% }
% \usepackage[a4paper]{geometry}
% \geometry{top=2cm, bottom=2cm, left=2cm, right=2cm, marginparsep=1cm
% }

%\usepackage{standalone}


% New command Arnaud -- november 2011
\setbeamersize{text margin left=24ex}
% si vous modifier cette valeur il faut aussi
% modifier le decalage du titre pour compenser
% (ex : ici =+10ex, titre =-5ex

\theoremstyle{definition}
%\newtheorem{proposition}{Proposition}
%\newtheorem{exemple}{Exemple}
%\newtheorem{theoreme}{Théorème}
%\newtheorem{lemme}{Lemme}
%\newtheorem{corollaire}{Corollaire}
%\newtheorem*{remarque*}{Remarque}
%\newtheorem*{miniexercice}{Mini-exercices}
%\newtheorem{definition}{Définition}

% Commande tikz
\usetikzlibrary{calc}
\usetikzlibrary{patterns,arrows}
\usetikzlibrary{matrix}
\usetikzlibrary{fadings} 

%definition d'un terme
\newcommand{\defi}[1]{{\color{myorange}\textbf{\emph{#1}}}}
\newcommand{\evidence}[1]{{\color{blue}\textbf{\emph{#1}}}}
\newcommand{\assertion}[1]{\emph{\og#1\fg}}  % pour chapitre logique
%\renewcommand{\contentsname}{Sommaire}
\renewcommand{\contentsname}{}
\setcounter{tocdepth}{2}



%------ Figures ------

\def\myscale{1} % par défaut 
\newcommand{\myfigure}[2]{  % entrée : echelle, fichier figure
\def\myscale{#1}
\begin{center}
\footnotesize
{#2}
\end{center}}


%------ Encadrement ------

\usepackage{fancybox}


\newcommand{\mybox}[1]{
\setlength{\fboxsep}{7pt}
\begin{center}
\shadowbox{#1}
\end{center}}

\newcommand{\myboxinline}[1]{
\setlength{\fboxsep}{5pt}
\raisebox{-10pt}{
\shadowbox{#1}
}
}

%--------------- Commande beamer---------------
\newcommand{\beameronly}[1]{#1} % permet de mettre des pause dans beamer pas dans poly


\setbeamertemplate{navigation symbols}{}
\setbeamertemplate{footline}  % tiré du fichier beamerouterinfolines.sty
{
  \leavevmode%
  \hbox{%
  \begin{beamercolorbox}[wd=.333333\paperwidth,ht=2.25ex,dp=1ex,center]{author in head/foot}%
    % \usebeamerfont{author in head/foot}\insertshortauthor%~~(\insertshortinstitute)
    \usebeamerfont{section in head/foot}{\bf\insertshorttitle}
  \end{beamercolorbox}%
  \begin{beamercolorbox}[wd=.333333\paperwidth,ht=2.25ex,dp=1ex,center]{title in head/foot}%
    \usebeamerfont{section in head/foot}{\bf\insertsectionhead}
  \end{beamercolorbox}%
  \begin{beamercolorbox}[wd=.333333\paperwidth,ht=2.25ex,dp=1ex,right]{date in head/foot}%
    % \usebeamerfont{date in head/foot}\insertshortdate{}\hspace*{2em}
    \insertframenumber{} / \inserttotalframenumber\hspace*{2ex} 
  \end{beamercolorbox}}%
  \vskip0pt%
}


\definecolor{mygrey}{rgb}{0.5,0.5,0.5}
\setlength{\parindent}{0cm}
%\DeclareTextFontCommand{\helvetica}{\fontfamily{phv}\selectfont}

% background beamer
\definecolor{couleurhaut}{rgb}{0.85,0.9,1}  % creme
\definecolor{couleurmilieu}{rgb}{1,1,1}  % vert pale
\definecolor{couleurbas}{rgb}{0.85,0.9,1}  % blanc
\setbeamertemplate{background canvas}[vertical shading]%
[top=couleurhaut,middle=couleurmilieu,midpoint=0.4,bottom=couleurbas] 
%[top=fondtitre!05,bottom=fondtitre!60]



\makeatletter
\setbeamertemplate{theorem begin}
{%
  \begin{\inserttheoremblockenv}
  {%
    \inserttheoremheadfont
    \inserttheoremname
    \inserttheoremnumber
    \ifx\inserttheoremaddition\@empty\else\ (\inserttheoremaddition)\fi%
    \inserttheorempunctuation
  }%
}
\setbeamertemplate{theorem end}{\end{\inserttheoremblockenv}}

\newenvironment{theoreme}[1][]{%
   \setbeamercolor{block title}{fg=structure,bg=structure!40}
   \setbeamercolor{block body}{fg=black,bg=structure!10}
   \begin{block}{{\bf Th\'eor\`eme }#1}
}{%
   \end{block}%
}


\newenvironment{proposition}[1][]{%
   \setbeamercolor{block title}{fg=structure,bg=structure!40}
   \setbeamercolor{block body}{fg=black,bg=structure!10}
   \begin{block}{{\bf Proposition }#1}
}{%
   \end{block}%
}

\newenvironment{corollaire}[1][]{%
   \setbeamercolor{block title}{fg=structure,bg=structure!40}
   \setbeamercolor{block body}{fg=black,bg=structure!10}
   \begin{block}{{\bf Corollaire }#1}
}{%
   \end{block}%
}

\newenvironment{mydefinition}[1][]{%
   \setbeamercolor{block title}{fg=structure,bg=structure!40}
   \setbeamercolor{block body}{fg=black,bg=structure!10}
   \begin{block}{{\bf Définition} #1}
}{%
   \end{block}%
}

\newenvironment{lemme}[0]{%
   \setbeamercolor{block title}{fg=structure,bg=structure!40}
   \setbeamercolor{block body}{fg=black,bg=structure!10}
   \begin{block}{\bf Lemme}
}{%
   \end{block}%
}

\newenvironment{remarque}[1][]{%
   \setbeamercolor{block title}{fg=black,bg=structure!20}
   \setbeamercolor{block body}{fg=black,bg=structure!5}
   \begin{block}{Remarque #1}
}{%
   \end{block}%
}


\newenvironment{exemple}[1][]{%
   \setbeamercolor{block title}{fg=black,bg=structure!20}
   \setbeamercolor{block body}{fg=black,bg=structure!5}
   \begin{block}{{\bf Exemple }#1}
}{%
   \end{block}%
}


\newenvironment{miniexercice}[0]{%
   \setbeamercolor{block title}{fg=structure,bg=structure!20}
   \setbeamercolor{block body}{fg=black,bg=structure!5}
   \begin{block}{Mini-exercices}
}{%
   \end{block}%
}


\newenvironment{tp}[0]{%
   \setbeamercolor{block title}{fg=structure,bg=structure!40}
   \setbeamercolor{block body}{fg=black,bg=structure!10}
   \begin{block}{\bf Travaux pratiques}
}{%
   \end{block}%
}
\newenvironment{exercicecours}[1][]{%
   \setbeamercolor{block title}{fg=structure,bg=structure!40}
   \setbeamercolor{block body}{fg=black,bg=structure!10}
   \begin{block}{{\bf Exercice }#1}
}{%
   \end{block}%
}
\newenvironment{algo}[1][]{%
   \setbeamercolor{block title}{fg=structure,bg=structure!40}
   \setbeamercolor{block body}{fg=black,bg=structure!10}
   \begin{block}{{\bf Algorithme}\hfill{\color{gray}\texttt{#1}}}
}{%
   \end{block}%
}


\setbeamertemplate{proof begin}{
   \setbeamercolor{block title}{fg=black,bg=structure!20}
   \setbeamercolor{block body}{fg=black,bg=structure!5}
   \begin{block}{{\footnotesize Démonstration}}
   \footnotesize
   \smallskip}
\setbeamertemplate{proof end}{%
   \end{block}}
\setbeamertemplate{qed symbol}{\openbox}


\makeatother
\usecolortheme[RGB={142,35,35}]{structure}

%%%%%%%%%%%%%%%%%%%%%%%%%%%%%%%%%%%%%%%%%%%%%%%%%%%%%%%%%%%%%
%%%%%%%%%%%%%%%%%%%%%%%%%%%%%%%%%%%%%%%%%%%%%%%%%%%%%%%%%%%%%

\begin{document}


\title{{\bf Polynômes}}
\subtitle{Fractions rationnelles}

\begin{frame}
  
  \debutmontitre

  \pause

{\footnotesize
\hfill
\setbeamercovered{transparent=50}
\begin{minipage}{0.6\textwidth}
  \begin{itemize}
    \item<3-> Définition
    \item<4-> Décomposition en éléments simples sur $\Cc$
    \item<5-> Décomposition en éléments simples sur $\Rr$
  \end{itemize}
\end{minipage}
}

\end{frame}

\setcounter{framenumber}{0}


%%%%%%%%%%%%%%%%%%%%%%%%%%%%%%%%%%%%%%%%%%%%%%%%%%%%%%%%%%%%%%%%


%---------------------------------------------------------------
\section{Définition}

\begin{frame}
\begin{mydefinition}
$$F=\frac{P}{Q} \qquad \text{est une \defi{fraction rationnelle}}$$

\medskip

\centerline{$P,Q \in \Kk[X]$ sont deux polynômes et $Q \neq 0$}
\end{mydefinition}
\end{frame}


%---------------------------------------------------------------
\section{Décomposition en éléments simples sur $\Cc$}

\begin{frame}
\begin{theoreme}[Décomposition en éléments simples sur $\Cc$]
Soit $P/Q$ une fraction rationnelle avec $P,Q \in \Cc[X]$, $\pgcd(P,Q)=1$ et 
$Q=(X-\alpha_1)^{k_1}\cdots(X-\alpha_r)^{k_r}$. 
Alors il existe une et une seule écriture :
\pause
$$\begin{array}{rl}
\displaystyle \frac{P}{Q} \  =  \ \  E  & + \ 
 \displaystyle  \frac{a_{1,1}}{(X-\alpha_1)^{k_1}}+\frac{a_{1,2}}{(X-\alpha_1)^{k_1-1}}+\cdots
+\ \frac{a_{1,k_1}}{(X-\alpha_1)} \\[4mm]
  & \displaystyle+ \frac{a_{2,1}}{(X-\alpha_2)^{k_2}}+\cdots
+\ \frac{a_{2,k_2}}{(X-\alpha_2)} \\[3mm]
 & + \ \cdots
\end{array}$$
\end{theoreme}

\pause

Le polynôme $E$ s'appelle la \defi{partie polynomiale}


\pause

Les termes $\frac{a}{(X-\alpha)^i}$ sont les \defi{éléments simples} sur $\Cc$

\pause

\begin{exemple}
\begin{itemize}
  \item $\frac{1}{X^2+1} = \frac{a}{X+\ii} + \frac{b}{X-\ii}$ \quad avec $a=\tfrac12 \ii$, $b=-\tfrac12\ii$

\pause 
  
  \item $\frac{X^4-8X^2+9X-7}{(X-2)^2(X+3)}= X+1 + \frac{-1}{(X-2)^2} + \frac{2}{X-2} + \frac{-1}{X+3}$
\end{itemize}
\end{exemple}
\end{frame}


\begin{frame}

\begin{exemple}
Décomposons la fraction $\frac{P}{Q}= \frac{X^5-2X^3+4X^2-8X+11}{X^3-3X+2}$

\pause

\begin{enumerate}
  \item \textbf{Partie polynomiale}
  \begin{itemize}
    \item La partie polynomiale est donc le quotient de la division euclidienne de $P$ par $Q$
\pause
    \item $P(X) = (X^2+1)Q(X)+ 2X^2-5X+9$
\pause
    \item $E(X)=X^2+1$
\pause
    \item $\frac{P(X)}{Q(X)} =X^2+1 + \frac{2X^2-5X+9}{Q(X)}$
  \end{itemize}
\pause
  \item \textbf{Factorisation du dénominateur} \pause $Q(X)= (X-1)^2(X+2)$
\pause
  \item \textbf{Décomposition théorique}
\pause
$$\frac{P(X)}{Q(X)}= E(X)+ \frac{a}{(X-1)^2} + \frac{b}{X-1} + \frac{c}{X+2}$$

\end{enumerate}
\end{exemple}
\end{frame}



\begin{frame}
\begin{exemple}
\begin{enumerate}
\setcounter{enumi}{3}
  \item \textbf{Détermination des coefficients}
  \begin{itemize}
  \setlength{\itemsep}{5pt} 
    \item $\displaystyle \frac{P(X)}{Q(X)} =X^2+1 + \frac{2X^2-5X+9}{X^3-3X+2}$
    \item $\displaystyle \frac{P(X)}{Q(X)}= E(X)+ \frac{a}{(X-1)^2} + \frac{b}{X-1} + \frac{c}{X+2}$
\pause
    \item $\displaystyle \frac{2X^2-5X+9}{(X-1)^2(X+2)} = \frac{(b+c)X^2+(a+b-2c)X+2a-2b+c}{(X-1)^2(X+2)}$
\pause
    \item $b+c=2$, \quad  $a+b-2c=-5$, \quad $2a-2b+c=9$
\pause
    \item $\ldots$ $a=2$, $b=-1$, $c=3$
  \end{itemize}
\pause
{\small
$$\frac{P}{Q} = \frac{X^5-2X^3+4X^2-8X+11}{X^3-3X+2} = X^2+1 + \frac{2}{(X-1)^2} + \frac{-1}{X-1} + \frac{3}{X+2}$$
}
\end{enumerate}
\end{exemple}
\end{frame}

\begin{frame}
\begin{exemple}
\begin{enumerate}
\setcounter{enumi}{3}
  \item \textbf{Détermination des coefficients (bis)}
  \begin{itemize}
  \setlength{\itemsep}{5pt} 
    \item $\frac{P_1(X)}{Q(X)} =  \frac{2X^2-5X+9}{(X-1)^2(X+2)} = \frac{a}{(X-1)^2} + \frac{b}{X-1} + \frac{c}{X+2}$
\pause
    \item $F_1(X)= (X-1)^2 \frac{P_1(X)}{Q(X)} \pause = a + b(X-1) + c\frac{(X-1)^2}{X+2}$ \pause donc $F_1(1)=a$
\pause
    \item $F_1(X) = \frac{2X^2-5X+9}{X+2}$ \pause donc $F_1(1)=2$ \pause ainsi $a=2$
\pause
    \item $F_2(X) = (X+2)\frac{P_1(X)}{Q(X)} \pause = \frac{2X^2-5X+9}{(X-1)^2} \pause = a\frac{X+2}{(X-1)^2} + b\frac{X+2}{X-1} + c$
\pause
    \item $F_2(-2) = c = 3$ ainsi $c=3$
\pause    
    \item $\frac{P_1(0)}{Q(0)} = a - b + \frac c2 = \frac{9}{2}$ donc $b=-1$
  \end{itemize}    
\pause  
{\small
$$\frac{P}{Q} =  X^2+1 + \frac{2}{(X-1)^2} + \frac{-1}{X-1} + \frac{3}{X+2}$$
}

\end{enumerate}
\end{exemple}
\end{frame}


%---------------------------------------------------------------
\section{Décomposition en éléments simples sur $\Rr$}

\begin{frame}

Soit $P/Q$ une fraction rationnelle avec $P,Q \in \Rr[X]$, $\pgcd(P,Q)=1$
\begin{theoreme}[Décomposition en éléments simples sur $\Rr$]
$P/Q$ s'écrit de manière unique comme somme :
\begin{itemize}
  \item d'une partie polynomiale $E(X)$
  \item d'éléments simples $\frac{a}{(X-\alpha)^i}$
  \item d'éléments simples $\frac{aX+b}{(X^2+\alpha X + \beta)^i}$

  \pause
  
  \begin{itemize}
    \item $X-\alpha$ et $X^2 + \alpha X + \beta$ sont les facteurs irréductibles de $Q(X)$
    
    \pause
    
    \item les exposants $i$ sont inférieurs ou égaux à la puissance correspondante dans cette factorisation
  \end{itemize}
\end{itemize}

\end{theoreme}
\end{frame}



\begin{frame}

\begin{exemple} 
$$\frac{P(X)}{Q(X)}=\frac{3X^4+5X^3+8X^2+5X+3}{(X^2+X+1)^2(X-1)}$$

\pause

\begin{itemize}[<+->]
\setlength{\itemsep}{7pt} 
  \item $\deg P < \deg Q$ donc $E(X)=0$
  \item $X^2+X+1$ est irréductible sur $\Rr$
  \item $\displaystyle \frac{P(X)}{Q(X)} = \frac{aX+b}{(X^2+X+1)^2}+\frac{cX+d}{X^2+X+1}+\frac{e}{X-1}$
  \item $\displaystyle \frac{P(X)}{Q(X)} = \frac{2X+1}{(X^2+X+1)^2}+\frac{-1}{X^2+X+1}+\frac{3}{X-1}$
\end{itemize}
\end{exemple}

\end{frame}



%%%%%%%%%%%%%%%%%%%%%%%%%%%%%%%%%%%%%%%%%%%%%%%%%%%%%%%%%%%%%%%%
\section{Mini-exercices}

\begin{frame}

\begin{miniexercice}
\begin{enumerate}
  \item Soit $Q(X)=(X-2)^2(X^2-1)^3(X^2+1)^4$. Pour $P \in \Rr[X]$ quelle est la forme théorique 
de la décomposition en éléments simples sur $\Cc$ de $\frac PQ$ ? Et sur $\Rr$ ?

  \item Décomposer les fractions suivantes en éléments simples sur $\Rr$ et $\Cc$ :
$\frac{1}{X^2-1}$ ; $\frac{X^2+1}{(X-1)^2}$ ; $\frac{X}{X^3-1}$.

  \item Décomposer les fractions suivantes en éléments simples sur $\Rr$ :
$\frac{X^2+X+1}{(X-1)(X+2)^2}$ ; $\frac{2X^2-X}{(X^2+2)^2}$ ; $\frac{X^6}{(X^2+1)^2}$.

  \item Soit $F(X) = \frac{2X^2+7X-20}{X+2}$. 
  Déterminer l'équation de l'asymptote oblique en $\pm \infty$. 
  \'Etudier la position du graphe de $F$ par rapport à cette droite.

\end{enumerate}
\end{miniexercice}

\end{frame}

\end{document}