
%%%%%%%%%%%%%%%%%% PREAMBULE %%%%%%%%%%%%%%%%%%

\documentclass[aspectratio=169,utf8]{beamer}
%\documentclass[aspectratio=169,handout]{beamer}

\usetheme{Boadilla}
%\usecolortheme{seahorse}
\usecolortheme[RGB={245,66,24}]{structure}
\useoutertheme{infolines}

% packages
\usepackage{amsfonts,amsmath,amssymb,amsthm}
\usepackage[utf8]{inputenc}
\usepackage[T1]{fontenc}
\usepackage{lmodern}

\usepackage[francais]{babel}
\usepackage{fancybox}
\usepackage{graphicx}

\usepackage{float}
\usepackage{xfrac}

%\usepackage[usenames, x11names]{xcolor}
\usepackage{tikz}
\usepackage{pgfplots}
\usepackage{datetime}



%-----  Package unités -----
\usepackage{siunitx}
\sisetup{locale = FR,detect-all,per-mode = symbol}

%\usepackage{mathptmx}
%\usepackage{fouriernc}
%\usepackage{newcent}
%\usepackage[mathcal,mathbf]{euler}

%\usepackage{palatino}
%\usepackage{newcent}
% \usepackage[mathcal,mathbf]{euler}



% \usepackage{hyperref}
% \hypersetup{colorlinks=true, linkcolor=blue, urlcolor=blue,
% pdftitle={Exo7 - Exercices de mathématiques}, pdfauthor={Exo7}}


%section
% \usepackage{sectsty}
% \allsectionsfont{\bf}
%\sectionfont{\color{Tomato3}\upshape\selectfont}
%\subsectionfont{\color{Tomato4}\upshape\selectfont}

%----- Ensembles : entiers, reels, complexes -----
\newcommand{\Nn}{\mathbb{N}} \newcommand{\N}{\mathbb{N}}
\newcommand{\Zz}{\mathbb{Z}} \newcommand{\Z}{\mathbb{Z}}
\newcommand{\Qq}{\mathbb{Q}} \newcommand{\Q}{\mathbb{Q}}
\newcommand{\Rr}{\mathbb{R}} \newcommand{\R}{\mathbb{R}}
\newcommand{\Cc}{\mathbb{C}} 
\newcommand{\Kk}{\mathbb{K}} \newcommand{\K}{\mathbb{K}}

%----- Modifications de symboles -----
\renewcommand{\epsilon}{\varepsilon}
\renewcommand{\Re}{\mathop{\text{Re}}\nolimits}
\renewcommand{\Im}{\mathop{\text{Im}}\nolimits}
%\newcommand{\llbracket}{\left[\kern-0.15em\left[}
%\newcommand{\rrbracket}{\right]\kern-0.15em\right]}

\renewcommand{\ge}{\geqslant}
\renewcommand{\geq}{\geqslant}
\renewcommand{\le}{\leqslant}
\renewcommand{\leq}{\leqslant}
\renewcommand{\epsilon}{\varepsilon}

%----- Fonctions usuelles -----
\newcommand{\ch}{\mathop{\text{ch}}\nolimits}
\newcommand{\sh}{\mathop{\text{sh}}\nolimits}
\renewcommand{\tanh}{\mathop{\text{th}}\nolimits}
\newcommand{\cotan}{\mathop{\text{cotan}}\nolimits}
\newcommand{\Arcsin}{\mathop{\text{arcsin}}\nolimits}
\newcommand{\Arccos}{\mathop{\text{arccos}}\nolimits}
\newcommand{\Arctan}{\mathop{\text{arctan}}\nolimits}
\newcommand{\Argsh}{\mathop{\text{argsh}}\nolimits}
\newcommand{\Argch}{\mathop{\text{argch}}\nolimits}
\newcommand{\Argth}{\mathop{\text{argth}}\nolimits}
\newcommand{\pgcd}{\mathop{\text{pgcd}}\nolimits} 


%----- Commandes divers ------
\newcommand{\ii}{\mathrm{i}}
\newcommand{\dd}{\text{d}}
\newcommand{\id}{\mathop{\text{id}}\nolimits}
\newcommand{\Ker}{\mathop{\text{Ker}}\nolimits}
\newcommand{\Card}{\mathop{\text{Card}}\nolimits}
\newcommand{\Vect}{\mathop{\text{Vect}}\nolimits}
\newcommand{\Mat}{\mathop{\text{Mat}}\nolimits}
\newcommand{\rg}{\mathop{\text{rg}}\nolimits}
\newcommand{\tr}{\mathop{\text{tr}}\nolimits}


%----- Structure des exercices ------

\newtheoremstyle{styleexo}% name
{2ex}% Space above
{3ex}% Space below
{}% Body font
{}% Indent amount 1
{\bfseries} % Theorem head font
{}% Punctuation after theorem head
{\newline}% Space after theorem head 2
{}% Theorem head spec (can be left empty, meaning ‘normal’)

%\theoremstyle{styleexo}
\newtheorem{exo}{Exercice}
\newtheorem{ind}{Indications}
\newtheorem{cor}{Correction}


\newcommand{\exercice}[1]{} \newcommand{\finexercice}{}
%\newcommand{\exercice}[1]{{\tiny\texttt{#1}}\vspace{-2ex}} % pour afficher le numero absolu, l'auteur...
\newcommand{\enonce}{\begin{exo}} \newcommand{\finenonce}{\end{exo}}
\newcommand{\indication}{\begin{ind}} \newcommand{\finindication}{\end{ind}}
\newcommand{\correction}{\begin{cor}} \newcommand{\fincorrection}{\end{cor}}

\newcommand{\noindication}{\stepcounter{ind}}
\newcommand{\nocorrection}{\stepcounter{cor}}

\newcommand{\fiche}[1]{} \newcommand{\finfiche}{}
\newcommand{\titre}[1]{\centerline{\large \bf #1}}
\newcommand{\addcommand}[1]{}
\newcommand{\video}[1]{}

% Marge
\newcommand{\mymargin}[1]{\marginpar{{\small #1}}}

\def\noqed{\renewcommand{\qedsymbol}{}}


%----- Presentation ------
\setlength{\parindent}{0cm}

%\newcommand{\ExoSept}{\href{http://exo7.emath.fr}{\textbf{\textsf{Exo7}}}}

\definecolor{myred}{rgb}{0.93,0.26,0}
\definecolor{myorange}{rgb}{0.97,0.58,0}
\definecolor{myyellow}{rgb}{1,0.86,0}

\newcommand{\LogoExoSept}[1]{  % input : echelle
{\usefont{U}{cmss}{bx}{n}
\begin{tikzpicture}[scale=0.1*#1,transform shape]
  \fill[color=myorange] (0,0)--(4,0)--(4,-4)--(0,-4)--cycle;
  \fill[color=myred] (0,0)--(0,3)--(-3,3)--(-3,0)--cycle;
  \fill[color=myyellow] (4,0)--(7,4)--(3,7)--(0,3)--cycle;
  \node[scale=5] at (3.5,3.5) {Exo7};
\end{tikzpicture}}
}


\newcommand{\debutmontitre}{
  \author{} \date{} 
  \thispagestyle{empty}
  \hspace*{-10ex}
  \begin{minipage}{\textwidth}
    \titlepage  
  \vspace*{-2.5cm}
  \begin{center}
    \LogoExoSept{2.5}
  \end{center}
  \end{minipage}

  \vspace*{-0cm}
  
  % Astuce pour que le background ne soit pas discrétisé lors de la conversion pdf -> png
\begin{tikzpicture}
        \fill[opacity=0,green!60!black] (0,0)--++(0,0)--++(0,0)--++(0,0)--cycle; 
\end{tikzpicture}

% toc S'affiche trop tot :
% \tableofcontents[hideallsubsections, pausesections]
}

\newcommand{\finmontitre}{
  \end{frame}
  \setcounter{framenumber}{0}
} % ne marche pas pour une raison obscure

%----- Commandes supplementaires ------

% \usepackage[landscape]{geometry}
% \geometry{top=1cm, bottom=3cm, left=2cm, right=10cm, marginparsep=1cm
% }
% \usepackage[a4paper]{geometry}
% \geometry{top=2cm, bottom=2cm, left=2cm, right=2cm, marginparsep=1cm
% }

%\usepackage{standalone}


% New command Arnaud -- november 2011
\setbeamersize{text margin left=24ex}
% si vous modifier cette valeur il faut aussi
% modifier le decalage du titre pour compenser
% (ex : ici =+10ex, titre =-5ex

\theoremstyle{definition}
%\newtheorem{proposition}{Proposition}
%\newtheorem{exemple}{Exemple}
%\newtheorem{theoreme}{Théorème}
%\newtheorem{lemme}{Lemme}
%\newtheorem{corollaire}{Corollaire}
%\newtheorem*{remarque*}{Remarque}
%\newtheorem*{miniexercice}{Mini-exercices}
%\newtheorem{definition}{Définition}

% Commande tikz
\usetikzlibrary{calc}
\usetikzlibrary{patterns,arrows}
\usetikzlibrary{matrix}
\usetikzlibrary{fadings} 

%definition d'un terme
\newcommand{\defi}[1]{{\color{myorange}\textbf{\emph{#1}}}}
\newcommand{\evidence}[1]{{\color{blue}\textbf{\emph{#1}}}}
\newcommand{\assertion}[1]{\emph{\og#1\fg}}  % pour chapitre logique
%\renewcommand{\contentsname}{Sommaire}
\renewcommand{\contentsname}{}
\setcounter{tocdepth}{2}



%------ Figures ------

\def\myscale{1} % par défaut 
\newcommand{\myfigure}[2]{  % entrée : echelle, fichier figure
\def\myscale{#1}
\begin{center}
\footnotesize
{#2}
\end{center}}


%------ Encadrement ------

\usepackage{fancybox}


\newcommand{\mybox}[1]{
\setlength{\fboxsep}{7pt}
\begin{center}
\shadowbox{#1}
\end{center}}

\newcommand{\myboxinline}[1]{
\setlength{\fboxsep}{5pt}
\raisebox{-10pt}{
\shadowbox{#1}
}
}

%--------------- Commande beamer---------------
\newcommand{\beameronly}[1]{#1} % permet de mettre des pause dans beamer pas dans poly


\setbeamertemplate{navigation symbols}{}
\setbeamertemplate{footline}  % tiré du fichier beamerouterinfolines.sty
{
  \leavevmode%
  \hbox{%
  \begin{beamercolorbox}[wd=.333333\paperwidth,ht=2.25ex,dp=1ex,center]{author in head/foot}%
    % \usebeamerfont{author in head/foot}\insertshortauthor%~~(\insertshortinstitute)
    \usebeamerfont{section in head/foot}{\bf\insertshorttitle}
  \end{beamercolorbox}%
  \begin{beamercolorbox}[wd=.333333\paperwidth,ht=2.25ex,dp=1ex,center]{title in head/foot}%
    \usebeamerfont{section in head/foot}{\bf\insertsectionhead}
  \end{beamercolorbox}%
  \begin{beamercolorbox}[wd=.333333\paperwidth,ht=2.25ex,dp=1ex,right]{date in head/foot}%
    % \usebeamerfont{date in head/foot}\insertshortdate{}\hspace*{2em}
    \insertframenumber{} / \inserttotalframenumber\hspace*{2ex} 
  \end{beamercolorbox}}%
  \vskip0pt%
}


\definecolor{mygrey}{rgb}{0.5,0.5,0.5}
\setlength{\parindent}{0cm}
%\DeclareTextFontCommand{\helvetica}{\fontfamily{phv}\selectfont}

% background beamer
\definecolor{couleurhaut}{rgb}{0.85,0.9,1}  % creme
\definecolor{couleurmilieu}{rgb}{1,1,1}  % vert pale
\definecolor{couleurbas}{rgb}{0.85,0.9,1}  % blanc
\setbeamertemplate{background canvas}[vertical shading]%
[top=couleurhaut,middle=couleurmilieu,midpoint=0.4,bottom=couleurbas] 
%[top=fondtitre!05,bottom=fondtitre!60]



\makeatletter
\setbeamertemplate{theorem begin}
{%
  \begin{\inserttheoremblockenv}
  {%
    \inserttheoremheadfont
    \inserttheoremname
    \inserttheoremnumber
    \ifx\inserttheoremaddition\@empty\else\ (\inserttheoremaddition)\fi%
    \inserttheorempunctuation
  }%
}
\setbeamertemplate{theorem end}{\end{\inserttheoremblockenv}}

\newenvironment{theoreme}[1][]{%
   \setbeamercolor{block title}{fg=structure,bg=structure!40}
   \setbeamercolor{block body}{fg=black,bg=structure!10}
   \begin{block}{{\bf Th\'eor\`eme }#1}
}{%
   \end{block}%
}


\newenvironment{proposition}[1][]{%
   \setbeamercolor{block title}{fg=structure,bg=structure!40}
   \setbeamercolor{block body}{fg=black,bg=structure!10}
   \begin{block}{{\bf Proposition }#1}
}{%
   \end{block}%
}

\newenvironment{corollaire}[1][]{%
   \setbeamercolor{block title}{fg=structure,bg=structure!40}
   \setbeamercolor{block body}{fg=black,bg=structure!10}
   \begin{block}{{\bf Corollaire }#1}
}{%
   \end{block}%
}

\newenvironment{mydefinition}[1][]{%
   \setbeamercolor{block title}{fg=structure,bg=structure!40}
   \setbeamercolor{block body}{fg=black,bg=structure!10}
   \begin{block}{{\bf Définition} #1}
}{%
   \end{block}%
}

\newenvironment{lemme}[0]{%
   \setbeamercolor{block title}{fg=structure,bg=structure!40}
   \setbeamercolor{block body}{fg=black,bg=structure!10}
   \begin{block}{\bf Lemme}
}{%
   \end{block}%
}

\newenvironment{remarque}[1][]{%
   \setbeamercolor{block title}{fg=black,bg=structure!20}
   \setbeamercolor{block body}{fg=black,bg=structure!5}
   \begin{block}{Remarque #1}
}{%
   \end{block}%
}


\newenvironment{exemple}[1][]{%
   \setbeamercolor{block title}{fg=black,bg=structure!20}
   \setbeamercolor{block body}{fg=black,bg=structure!5}
   \begin{block}{{\bf Exemple }#1}
}{%
   \end{block}%
}


\newenvironment{miniexercice}[0]{%
   \setbeamercolor{block title}{fg=structure,bg=structure!20}
   \setbeamercolor{block body}{fg=black,bg=structure!5}
   \begin{block}{Mini-exercices}
}{%
   \end{block}%
}


\newenvironment{tp}[0]{%
   \setbeamercolor{block title}{fg=structure,bg=structure!40}
   \setbeamercolor{block body}{fg=black,bg=structure!10}
   \begin{block}{\bf Travaux pratiques}
}{%
   \end{block}%
}
\newenvironment{exercicecours}[1][]{%
   \setbeamercolor{block title}{fg=structure,bg=structure!40}
   \setbeamercolor{block body}{fg=black,bg=structure!10}
   \begin{block}{{\bf Exercice }#1}
}{%
   \end{block}%
}
\newenvironment{algo}[1][]{%
   \setbeamercolor{block title}{fg=structure,bg=structure!40}
   \setbeamercolor{block body}{fg=black,bg=structure!10}
   \begin{block}{{\bf Algorithme}\hfill{\color{gray}\texttt{#1}}}
}{%
   \end{block}%
}


\setbeamertemplate{proof begin}{
   \setbeamercolor{block title}{fg=black,bg=structure!20}
   \setbeamercolor{block body}{fg=black,bg=structure!5}
   \begin{block}{{\footnotesize Démonstration}}
   \footnotesize
   \smallskip}
\setbeamertemplate{proof end}{%
   \end{block}}
\setbeamertemplate{qed symbol}{\openbox}


\makeatother
\usecolortheme[RGB={142,35,35}]{structure}

% Commande spécifique à ce chapitre
\newcommand{\ppcm}{\mathop{\text{ppcm}}\nolimits}

%%%%%%%%%%%%%%%%%%%%%%%%%%%%%%%%%%%%%%%%%%%%%%%%%%%%%%%%%%%%%
%%%%%%%%%%%%%%%%%%%%%%%%%%%%%%%%%%%%%%%%%%%%%%%%%%%%%%%%%%%%%


\begin{document}


\title{{\bf Polynômes}}
\subtitle{Arithmétique des polynômes}

\begin{frame}
  
  \debutmontitre

  \pause

{\footnotesize
\hfill
\setbeamercovered{transparent=50}
\begin{minipage}{0.6\textwidth}
  \begin{itemize}
    \item<3-> Division euclidienne
    \item<4-> pgcd
    \item<5-> Théorème de Bézout
%    \item<6-> ppcm    
  \end{itemize}
\end{minipage}
}

\end{frame}

\setcounter{framenumber}{0}


%%%%%%%%%%%%%%%%%%%%%%%%%%%%%%%%%%%%%%%%%%%%%%%%%%%%%%%%%%%%%%%%


%---------------------------------------------------------------
\section{Division euclidienne}

\begin{frame}
Soient $A,B \in\Kk[X]$
\begin{mydefinition}
$B$ \defi{divise} $A$ s'il existe  $Q\in\Kk[X]$ tel que $A=BQ$
\end{mydefinition}

\pause

\begin{itemize}
  \item On note $B|A$
  \item $A$ est multiple de $B$
  \item $A$ est divisible par $B$
\end{itemize}

\pause

\begin{proposition}
\begin{enumerate}[<+->]
  \item $A|A$, $1|A$ et $A|0$
  \item Si $A|B$ et $B|A$, alors il existe $\lambda \in\Kk^*$ tel que $A=\lambda B$
  \item Si $A|B$ et $B|C$ alors $A|C$
  \item Si $C|A$ et $C|B$ alors  $C|(AU+BV)$, pour tout $U,V \in\Kk[X]$
\end{enumerate}  
\end{proposition}
\end{frame}


\begin{frame}
\begin{theoreme}[Division euclidienne]
Soient $A,B \in\Kk[X]$, avec $B \neq 0$

Il \evidence{existe} des polynômes $Q$ et $R$ \evidence{uniques} tels que :
\mybox{$A=BQ+R \quad \text{ et } \quad \deg R < \deg B$}
\end{theoreme}

\pause
\bigskip

\begin{itemize}[<+->]
  \item $Q$ est le \defi{quotient} et $R$ le \defi{reste} de la \defi{division euclidienne}
  \item $\deg R < \deg B \iff R=0 \text{ ou } 0 \le \deg R < \deg B$
  \item $R=0 \iff  B|A$
\end{itemize}

\end{frame}

\begin{frame}
\begin{exemple}
\begin{itemize}
  \item $A=2X^4-X^3-2X^2+3X-1$ \quad et \quad  $B=X^2-X+1$
\pause  
  \item $Q= 2X^2+X-3$ \quad et \quad $R=-X+2$
\pause  
  \item $A=BQ+R$ et $\deg R < \deg B$
\end{itemize}
\pause  
\myfigure{1.2}{
\tikzinput{fig_polynome01-pres}
}

\end{exemple}
\end{frame}


\begin{frame}
\begin{exemple}
\begin{itemize}
  \item $A=X^4-3X^3+X+1$ \quad et \quad  $B=X^2+2$
 
\uncover<10->{  \item $Q=X^2-3X-2$ \quad et \quad $R=7X+5$}
\end{itemize}
\pause
\pause
\myfigure{1.2}{
\tikzinput{fig_polynome02-pres}
}
\end{exemple}
\end{frame}



%---------------------------------------------------------------
\section{pgcd}

\begin{frame}
Soient  $A,B \in\Kk[X]$, avec $A \neq 0$ ou $B \neq 0$
\begin{proposition} 
\label{prop_pgcd1}
Il existe un unique polynôme unitaire de plus grand degré qui divise à la fois $A$ et $B$.
\pause
On le note \defi{$\pgcd(A,B)$}
\end{proposition}


\bigskip
\pause

\begin{remarque}
\begin{itemize}
  \item $\pgcd(A,B)$ est un polynôme unitaire
\pause 
  \item Si $A=BQ+R$ alors $\pgcd(A,B) = \pgcd(B,R)$
\end{itemize}
\end{remarque}


\end{frame}

\begin{frame}
\textbf{Algorithme d'Euclide}

\pause

$$\begin{array}{ll}
A = B Q_1+R_1 \quad \quad & \deg R_1 < \deg B \\
\pause
B = R_1 Q_2 + R_2  & \deg R_2 < \deg R_1 \\
\pause
R_1=R_2Q_3+R_3 & \deg R_3 < \deg R_2 \\
\pause
\vdots & \\
\pause
R_{k-2}=R_{k-1}Q_{k}+R_k  & \deg R_k < \deg R_{k-1} \\
\pause
R_{k-1}=R_kQ_{k+1} & \\
\end{array}$$

\pause

\begin{itemize}[<+->]
  \item Le pgcd est le dernier reste non nul $R_k$ (rendu unitaire)
  \item Divisions euclidiennes successives
  \item Le degré du reste diminue à chaque division
  \item L'algorithme se termine lorsque le reste est nul
  \item {\small $\pgcd(A,B)=\pgcd(B,R_1)=\pgcd(R_1,R_2)=\cdots =\pgcd(R_k,0)=R_k$}
\end{itemize}

\end{frame}


\begin{frame}
\begin{exemple}
Calculons le pgcd de  $A=X^4-1$ et $B=X^3-1$

\pause
\medskip

On applique l'algorithme d'Euclide
$$\begin{array}{rcl}
X^4-1 & = & (X^3-1) \times X + X-1 \\
\pause
X^3-1 & = & (X-1)\times (X^2+X+1) + 0 \\
\end{array}$$

\pause
\medskip

Le pgcd est le dernier reste non nul

$$\pgcd(X^4-1, X^3-1)=X-1$$
\end{exemple}
\end{frame}


\begin{frame}
\begin{exemple}
Pgcd de  $A=X^5+X^4+2X^3+X^2+X+2$ et $B=X^4+2X^3+X^2-4$

\medskip
\pause

{\footnotesize
$$\begin{array}{rcl}
\!\!\!\!\!\!X^5+X^4+2X^3+X^2+X+2 & \!\!\!\! = \!\!\!\! & (X^4+2X^3+X^2-4)  (X\!-\!1)  + 3X^3+2X^2+5X-2\\
\pause
X^4+2X^3+X^2-4 & \!\!\!\! = \!\!\!\! & (3X^3+2X^2+5X-2) \frac19(3X+4) - \frac{14}{9}(X^2+X+2) \\
\pause
3X^3+2X^2+5X-2 & \!\!\!\! = \!\!\!\!  & (X^2+X+2) (3X-1) + 0 \\
\end{array}$$
}

\medskip
\pause

Ainsi $\pgcd(A,B)=X^2+X+2$
\end{exemple}
\end{frame}

%---------------------------------------------------------------
\section{Théorème de Bézout}

\begin{frame}

Soit $D=\pgcd(A,B)$

\begin{theoreme}[de Bézout]
\label{thm_Bezout}
Il existe $U, V\in \Kk[X]$ tels que \myboxinline{$AU+BV=D$}
\end{theoreme}

\pause

\begin{exemple}
\begin{itemize}[<+->]
  \item $\pgcd(X^4-1, X^3-1) = X-1$
  \item $X^4-1  =  (X^3-1) \times X + X-1$
  \item $X-1 = (X^4-1)\times 1 + (X^3-1) \times (-X)$
  \item $U=1$ et $V=-X$
\end{itemize}
\end{exemple}
\end{frame}


\begin{frame}
\begin{exemple}
\begin{itemize}  
\setlength{\itemsep}{7pt} 
  \item $A=X^5+X^4+2X^3+X^2+X+2$ et $B=X^4+2X^3+X^2-4$
  \item $D= \pgcd(A,B)=X^2+X+2$
\end{itemize}
\pause
{\footnotesize
$$\begin{array}{rcl}
A & =  & B \times  (X\!-\!1)  + 3X^3+2X^2+5X-2\\
B &  =  & (3X^3+2X^2+5X-2) \frac19(3X+4) - \frac{14}{9}(X^2+X+2) \\
3X^3+2X^2+5X-2 &  =   & (X^2+X+2) (3X-1) + 0 \\
\end{array}$$
}
\pause
\begin{itemize}
\setlength{\itemsep}{7pt}
  \item $- \frac{14}{9} D = B - (3X^3+2X^2+5X-2)\times \frac19(3X+4)$
\pause
  \item $- \frac{14}{9} D = B - \big(A-B\times(X-1)\big)\times \frac19(3X+4)$
\pause
  \item $- \frac{14}{9}D = -A\times \frac19(3X+4) + B\big(1+(X-1)\times \frac19(3X+4)\big)$  
\pause
  \item {\small $U=\frac{1}{14}(3X+4)$  et  $V= -\frac{1}{14}\big(9+(X-1)(3X+4)\big)=-\frac{1}{14}(3X^2+X+5)$}
\pause
  \item $AU+BV=D$
\end{itemize}
\end{exemple}
\end{frame}


\begin{frame}

\begin{mydefinition}
$A$ et $ B$ sont \defi{premiers entre eux} si  $\pgcd(A,B)=1$
\end{mydefinition}

\pause
\bigskip

Si  $\pgcd(A,B)=D$ alors $A=DA'$, $B=DB'$ avec $\pgcd(A',B')=1$

\end{frame}


\begin{frame}

\begin{corollaire}
$A$ et $B$ sont premiers entre eux si et seulement s'il existe deux polynômes $U$ et $V$ tels que 
\mybox{$AU+BV=1$}
\end{corollaire}

\pause

\begin{corollaire}
Si $C|A$ et $C|B$ alors $C|\pgcd(A,B)$
\end{corollaire}

\pause

\begin{corollaire}[Lemme de Gauss]
Si $A|BC$ et $\pgcd(A,B)=1$ alors $A|C$
\end{corollaire}

\end{frame}

% %---------------------------------------------------------------
% \section{ppcm}
% 
% 
% \begin{frame}
% 
% $A, B\in \Kk[X]$ des polynômes non nuls
% \begin{proposition}
% Il existe un unique polynôme unitaire $M$ de plus petit degré tel que $A|M$ et $B|M$
% \end{proposition}
% 
% \pause
% 
% Ce polynôme est le \defi{ppcm} de $A$ et $B$ 
% 
% \pause
% 
% \begin{exemple}
%  $\ppcm  \big( X(X-2)^2(X^2+1)^4, (X+1)(X-2)^3(X^2+1)^3 \big)$ \\
%  \hfill $=  X(X+1)(X-2)^3(X^2+1)^4$
% \end{exemple}
% 
% \pause
% 
% \begin{proposition}
% Si $C\in\Kk[X]$ est un polynôme tel que $A|C$ et $B|C$, alors $\ppcm(A,B)|C$
% \end{proposition}
% 
% \end{frame}


%%%%%%%%%%%%%%%%%%%%%%%%%%%%%%%%%%%%%%%%%%%%%%%%%%%%%%%%%%%%%%%%
\section{Mini-exercices}

\begin{frame}

\begin{miniexercice}
\begin{enumerate}
  \item Trouver les diviseurs de $X^4+2X^2+1$ dans $\Rr[X]$, puis dans $\Cc[X]$.

  \item Montrer que $X-1|X^n-1$ (pour $n\ge 1$).

  \item Calculer les divisions euclidiennes de $A$ par $B$ avec $A=X^4-1$, $B=X^3-1$.
Puis $A = 4X^3+2X^2-X-5$ et $B = X^2+X$ ; $A= 2X^4-9X^3+18X^2-21X+2$ et $B=X^2-3X+1$ ;
$A=X^ 5-2X^4+6X^3$ et $B=2X^3+1$.

  \item Déterminer le pgcd de $A=X^5+X^3+X^2+1$ et $B=2X^3+3X^2+2X+3$.
Trouver les coefficients de Bézout $U,V$.
Mêmes questions avec $A=X^5-1$ et $B=X^4+X+1$.

  \item Montrer que si $AU+BV=1$ avec $\deg U < \deg B$ et $\deg V < \deg A$ 
  alors les polynômes $U,V$ sont uniques.
\end{enumerate}
\end{miniexercice}

\end{frame}

\end{document}