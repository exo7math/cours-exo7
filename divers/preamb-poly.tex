%%%%%%%%%%%%%%%%%% PREAMBULE %%%%%%%%%%%%%%%%%%

\documentclass[11pt]{article}

%----- Principaux packages -----
\usepackage{amsfonts,amsmath,amssymb,amsthm}
\usepackage[utf8]{inputenc}
\usepackage[T1]{fontenc}
\usepackage[francais]{babel}
\usepackage{fancybox}
\usepackage{graphicx}
\usepackage{float}
\usepackage[usenames, x11names]{xcolor}
\usepackage{fouriernc}


%----- Ensembles : entiers, reels, complexes -----
\newcommand{\Nn}{\mathbb{N}} \newcommand{\N}{\mathbb{N}}
\newcommand{\Zz}{\mathbb{Z}} \newcommand{\Z}{\mathbb{Z}}
\newcommand{\Qq}{\mathbb{Q}} \newcommand{\Q}{\mathbb{Q}}
\newcommand{\Rr}{\mathbb{R}} \newcommand{\R}{\mathbb{R}}
\newcommand{\Cc}{\mathbb{C}} \newcommand{\C}{\mathbb{C}}
\newcommand{\Kk}{\mathbb{K}} \newcommand{\K}{\mathbb{K}}

%----- Modifications de symboles -----
\renewcommand{\epsilon}{\varepsilon}
\renewcommand{\Re}{\mathop{\text{Re}}\nolimits}
\renewcommand{\Im}{\mathop{\text{Im}}\nolimits}
\renewcommand{\ge}{\geqslant} \renewcommand{\geq}{\geqslant}
\renewcommand{\le}{\leqslant} \renewcommand{\leq}{\leqslant}


%----- Fonctions usuelles -----
\newcommand{\ch}{\mathop{\mathrm{ch}}\nolimits}
\newcommand{\sh}{\mathop{\mathrm{sh}}\nolimits}
\renewcommand{\tanh}{\mathop{\mathrm{th}}\nolimits}
\newcommand{\cotan}{\mathop{\mathrm{cotan}}\nolimits}
\newcommand{\Arcsin}{\mathop{\mathrm{arcsin}}\nolimits}
\newcommand{\Arccos}{\mathop{\mathrm{arccos}}\nolimits}
\newcommand{\Arctan}{\mathop{\mathrm{arctan}}\nolimits}
\newcommand{\Argsh}{\mathop{\mathrm{argsh}}\nolimits}
\newcommand{\Argch}{\mathop{\mathrm{argch}}\nolimits}
\newcommand{\Argth}{\mathop{\mathrm{argth}}\nolimits}
\newcommand{\pgcd}{\mathop{\mathrm{pgcd}}\nolimits} 

 %----- Commandes divers ------
\newcommand{\ii}{\mathrm{i}}
\newcommand{\dd}{\mathrm{d}}
\newcommand{\Ker}{\mathop{\text{Ker}}\nolimits}
\newcommand{\id}{\mathop{\text{id}}\nolimits}
\newcommand{\Card}{\mathop{\text{Card}}\nolimits}
\newcommand{\Vect}{\mathop{\text{Vect}}\nolimits}
\newcommand{\Mat}{\mathop{\mathrm{Mat}}\nolimits}
\newcommand{\rg}{\mathop{\text{rg}}\nolimits}
\newcommand{\tr}{\mathop{\text{tr}}\nolimits}
\newcommand{\ppcm}{\mathop{\text{ppcm}}\nolimits}


%----- Definition d'un terme -----
\newcommand{\defi}[1]{{\color{myorange}\textbf{\emph{#1}}}}
\newcommand{\evidence}[1]{{\color{blue}\textbf{\emph{#1}}}}
\newcommand{\assertion}[1]{{\og\emph{#1}\fg}} % pour chapitre logique


%-----  Package liens hypertexts ----- 
\usepackage{hyperref}
\hypersetup{colorlinks=true, linkcolor=blue, urlcolor=blue,
pdftitle={Exo7 - Cours de mathématiques}, pdfauthor={Exo7}}

%----- Commandes tikz -----
\usepackage{tikz}
\usepackage{pgfplots}
\usetikzlibrary{calc}
\usetikzlibrary{shadows}
\usetikzlibrary{arrows}
\usetikzlibrary{patterns}
\usetikzlibrary{matrix}

%-----  Multi colonnes- ---- 
\usepackage{multicol}
\setlength{\columnseprule}{0.2mm}

%-----  Package d'importation ----- 
\usepackage{import}


%----- Style sections ----- 
\usepackage{sectsty}
%\allsectionsfont{\color{blue}\itshape\underline}
\chapterfont{\selectfont}
\sectionfont{\color{Tomato3}\upshape\selectfont}
\subsectionfont{\color{Tomato4}\upshape\selectfont}
\usepackage{titlesec}
\titleformat{\chapter}{\color{Red3}\titlerule[3pt]\vspace{2pt}\titlerule[1pt]\normalfont\sffamily\Large\bfseries\center\vspace{0.8em}}{}{20pt}{}[\vspace{1em}{\titlerule[1pt]}\vspace{2pt}\titleline{\color{Red3}\newline\titlerule[3pt]}]




%----- Structure des exercices ------

\newtheoremstyle{styleexo}% name
{2ex}% Space above
{3ex}% Space below
{}% Body font
{}% Indent amount 1
{\bfseries} % Theorem head font
{}% Punctuation after theorem head
{\newline}% Space after theorem head 2
{}% Theorem head spec (can be left empty, meaning ‘normal’)

\theoremstyle{styleexo}
\newtheorem{exo}{Exercice}
\newtheorem{ind}{Indications}
\newtheorem{cor}{Correction}

\newcommand{\exercice}[1]{} \newcommand{\finexercice}{}
%\newcommand{\exercice}[1]{{\tiny\texttt{#1}}\vspace{-2ex}} % pour afficher le numero absolu, l'auteur...
\newcommand{\enonce}{\begin{exo}} \newcommand{\finenonce}{\end{exo}}
\newcommand{\indication}{\begin{ind}} \newcommand{\finindication}{\end{ind}}
\newcommand{\correction}{\begin{cor}} \newcommand{\fincorrection}{\end{cor}}
\newcommand{\noindication}{\stepcounter{ind}}
\newcommand{\nocorrection}{\stepcounter{cor}}
\newcommand{\fiche}[1]{} \newcommand{\finfiche}{}
\newcommand{\titre}[1]{\centerline{\large \bf #1}}
\newcommand{\addcommand}[1]{}


%Link to video Youtube

% variable myvideo : 0 no video, otherwise youtube reference
\newcommand{\video}[1]{\def\myvideo{#1}}
\newcommand{\insertvideo}[2]{\video{#1}%
{\small\texttt{\href{http://www.youtube.com/watch?v=\myvideo}{Vidéo $\blacksquare$ #2}}}}

% Liens vers les fiches d'exercices
\newcommand{\mafiche}[1]{\def\mymafiche{#1}}
\newcommand{\insertfiche}[2]{\mafiche{#1}%
{\small\texttt{\href{http://exo7.emath.fr/ficpdf/\mymafiche}{Fiche d'exercices $\blacklozenge$ #2}}}}

%----- Sommaire et mini-sommaires ------
% \usepackage[french]{minitoc}
 \setcounter{tocdepth}{2}
% \setcounter{minitocdepth}{2}
% \renewcommand{\contentsname}{}
% \mtcsettitle{minitoc}{}
% \mtcsetrules{minitoc}{off}
% \mtcsetfeature{minitoc}{before}{\vspace{-5pt}}% à ajuster
% \makeatletter
% \renewcommand{\thesection}{\@arabic\c@section}
% % \renewcommand{\thechapter}{}
% % \renewcommand{\chaptername}{}
% \makeatother



%----- Logo Exo7 ------
\definecolor{myred}{rgb}{0.93,0.26,0}
\definecolor{myorange}{rgb}{0.97,0.58,0}
\definecolor{myyellow}{rgb}{1,0.86,0}

\newcommand{\LogoExoSept}[1]{  % input : echelle
{\usefont{U}{cmss}{bx}{n}
\begin{tikzpicture}[scale=0.1*#1,transform shape]
  \fill[color=myorange] (0,0)--(4,0)--(4,-4)--(0,-4)--cycle;
  \fill[color=myred] (0,0)--(0,3)--(-3,3)--(-3,0)--cycle;
  \fill[color=myyellow] (4,0)--(7,4)--(3,7)--(0,3)--cycle;
  \node[scale=5] at (3.5,3.5) {Exo7};
\end{tikzpicture}}
}

%------ Titre livre -------------
\newcommand{\montitre}[1]{
\hfill\LogoExoSept{3}

{\footnotesize % \renewcommand{\contentsname}{\small Sommaire}
\dominitoc 
\renewcommand{\contentsname}{Cours de mathématiques\\ Première année}
\begingroup
\let\clearpage\relax
\tableofcontents
\endgroup}\finsommaire}

\newcommand{\finsommaire}{
\vfill\par\href{http://www.unisciel.fr/}{\includegraphics[scale=1]{logo_unisciel.png}}
\hfill\hspace*{9ex}\begin{minipage}{0.5\textwidth}\vspace*{-5.5ex}\center Cours et exercices 
de maths \\ \texttt{\href{http://exo7.emath.fr}{exo7.emath.fr}}\end{minipage}
\hfill\href{http://www.univ-lille1.fr/}{\includegraphics[scale=0.33]{logo_lille1.jpg}}
\vspace*{-3.5ex}\newpage}


%------ Chapitre -------------
% \newcommand{\chapitre}[1]{          % pour chapitre dans livre
% \chapter{#1}\vspace*{-21.5ex}
% \LogoExoSept{1.6} \vspace*{5ex}
% {\footnotesize % \renewcommand{\contentsname}{\small Sommaire}
% \renewcommand{\contentsname}{}
% {\small\minitoc}}
% \vspace*{2ex}} 

\newcommand{\chapitre}[1]{          % pour chapitre seul
\begin{document}
 \vspace*{-4ex}
\LogoExoSept{3} \vspace*{-2ex}

\vspace*{3.0ex}{\color{Red3}
{\hrule height 3pt}\vspace{2pt}{\hrule height 1pt}\vspace*{1.8ex} 
\hfil\textsf{\textbf{\Large {#1}}}
\vspace*{2ex}{\hrule height 1pt}\vspace{2pt}{\hrule height 3pt}}
\vspace*{2ex}{\footnotesize 
\renewcommand{\contentsname}{}
\tableofcontents}\vspace*{2ex}} 

\newcommand{\finchapitre}{\end{document}}

%----- Style des pages ------
\usepackage[a4paper]{geometry}
\geometry{top=2cm, bottom=2cm, left=2cm, right=2cm, marginparsep=1cm}
\setlength{\parindent}{0cm}

%----- Personnalisation tiret itemize -----
\renewcommand{\FrenchLabelItem}{{\bf--}}  % Evite confusion avec signe -

%----- Personnalisation pour les theoremes,... -----
\makeatletter
\newtheoremstyle{propthm}{\topsep}{\topsep}{}{}{\bfseries}{}{\newline}{%
\thmname{#1}\thmnumber{ #2}\@ifnotempty{#3}{ (\thmnote{#3})}.}

\newtheoremstyle{definit}{\topsep}{\topsep}{}{}{\bfseries}{}{ }{%
\thmname{#1}\thmnumber{ #2}\@ifnotempty{#3}{ (\thmnote{#3})}.}
\makeatother

\theoremstyle{propthm}
\newtheorem{proposition}{Proposition}
\newtheorem{theoreme}{Théorème}
\newtheorem*{propriete*}{Propriété}
\theoremstyle{definit}
\newtheorem{definition}{Définition}
\newtheorem{corollaire}{Corollaire}

\theoremstyle{definition}
\newtheorem{exemple}{Exemple}
\newtheorem{lemme}{Lemme}
\newtheorem*{remarque*}{Remarque}
\newtheorem{tp}{Travaux pratiques}
\newtheorem{exercicecours}{Exercice}
\newtheorem*{miniexercices}{Mini-exercices}

%----- Commandes anti-beamer -----
\newcommand{\pause}{}  % permet de mettre des \pause dans beamer pas dans poly
\newcommand{\beameronly}[1]{}


%------ Figures ------
\def\myscale{1} % par défaut 
\newcommand{\myfigure}[2]{  % entrée : echelle, fichier figure
\def\myscale{#1}\begin{center}\footnotesize{#2}\end{center}}


%------ Encadrement des formules ------
\usepackage{fancybox}
%\setlength{\fboxsep}{7pt}
\newcommand{\mybox}[1]{\begin{center}\shadowbox{#1}\end{center}}
\newcommand{\myboxinline}[1]{\raisebox{-2ex}{\shadowbox{#1}}}


%------ Encadrement auteurs ------
\usepackage[tikz]{bclogo}
\newcommand{\auteurs}[1]{
\renewcommand\bcStyleTitre[1]{\large\textbf{Auteurs}}
\vfill{\color{black!80}
\begin{bclogo}[logo=\bcplume, couleurBord=black!70, couleurBarre=gray, couleurOmbre=gray, arrondi=0.2, ombre=true, blur]{Auteurs}
 #1 \end{bclogo}}}

%------ Algorithmes ------

\newcommand{\Python}{\texttt{Python}}
\newcommand{\Sage}{\texttt{Sage}}

% Pour afficher du code
\usepackage{listingsutf8}

% \lstset{
%   language=Python,
%   upquote=true,
%   columns=flexible,
%   keepspaces=true,
%   basicstyle=\ttfamily,
%   commentstyle=\color{gray},
%   showspaces=false
%   showstringspaces=false
% } 
\lstset{
  language=Python,
  upquote=true,
  columns=flexible,
  keepspaces=true,
  basicstyle=\ttfamily,
  commentstyle=\color{gray},
  showspaces=false
  showstringspaces=false
}

% \makeatletter
% \def\lst@outputspace{\lst@bkgcolor\empty\color{white}}
% \makeatother
\makeatletter
\def\lst@outputspace{{\ifx\lst@bkgcolor\empty\color{white}\else
\lst@bkgcolor\fi\lst@visiblespace}}
\lst@keepspacestrue
\lst@keepspacestrue 
\makeatother

% \newcommand{\insertcode}[1]{\hrulefill\quad \texttt{#1}\newline
% \lstinputlisting[inputencoding=utf8/latin1]{Algos/#1}\hrulefill}
\newcommand{\insertcode}[2]{
\renewcommand\bcStyleTitre[1]{\hfill\texttt{\color{gray}#2}}
\begin{center}
\begin{minipage}{0.9\textwidth}
\begin{bclogo}[logo=\bctrombone, couleurBord=black!80, couleurBarre=gray, 
couleurOmbre=gray, arrondi=0.2, ombre=true, blur]{#1}
\lstinputlisting[inputencoding=utf8/latin1]{../#1} 
\end{bclogo}  
\end{minipage}  
\end{center}
}

\newcommand{\codeinline}[1]{\lstinline!#1!}
