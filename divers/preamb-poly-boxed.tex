%%%%%%%%%%%%%%%%%% PREAMBULE %%%%%%%%%%%%%%%%%%

\documentclass[11pt]{article}

\usepackage{amsfonts,amsmath,amssymb,amsthm}
\usepackage[utf8]{inputenc}
\usepackage[T1]{fontenc}
\usepackage[francais]{babel}
\usepackage{fancybox}
\usepackage{graphicx}

\usepackage{float}

\usepackage[usenames, x11names]{xcolor}
\usepackage{tikz}
\usepackage{datetime}
%\usepackage{mathptmx}
\usepackage{newcent}
\usepackage[mathcal,mathbf]{euler}


\usepackage{hyperref}
\hypersetup{colorlinks=true, linkcolor=blue, urlcolor=blue,
pdftitle={Exo7 - Exercices de mathématiques}, pdfauthor={Exo7}}


% section
\usepackage{sectsty}
%\allsectionsfont{\color{blue}\itshape\underline}
\sectionfont{\color{Tomato3}\upshape\selectfont}
\subsectionfont{\color{Tomato4}\upshape\selectfont}

%----- Ensembles : entiers, reels, complexes -----
\newcommand{\Nn}{\mathbb{N}} \newcommand{\N}{\mathbb{N}}
\newcommand{\Zz}{\mathbb{Z}} \newcommand{\Z}{\mathbb{Z}}
\newcommand{\Qq}{\mathbb{Q}} \newcommand{\Q}{\mathbb{Q}}
\newcommand{\Rr}{\mathbb{R}} \newcommand{\R}{\mathbb{R}}
\newcommand{\Cc}{\mathbb{C}} \newcommand{\C}{\mathbb{C}}
\newcommand{\Kk}{\mathbb{K}} \newcommand{\K}{\mathbb{K}}

%----- Modifications de symboles -----
\renewcommand{\epsilon}{\varepsilon}
\renewcommand{\Re}{\mathop{\text{Re}}\nolimits}
\renewcommand{\Im}{\mathop{\text{Im}}\nolimits}
\newcommand{\llbracket}{\left[\kern-0.15em\left[}
\newcommand{\rrbracket}{\right]\kern-0.15em\right]}

\renewcommand{\ge}{\geqslant}
\renewcommand{\geq}{\geqslant}
\renewcommand{\le}{\leqslant}
\renewcommand{\leq}{\leqslant}

%----- Fonctions usuelles -----
\newcommand{\ch}{\mathop{\mathrm{ch}}\nolimits}
\newcommand{\sh}{\mathop{\mathrm{sh}}\nolimits}
\renewcommand{\tanh}{\mathop{\mathrm{th}}\nolimits}
\newcommand{\cotan}{\mathop{\mathrm{cotan}}\nolimits}
\newcommand{\Arcsin}{\mathop{\mathrm{Arcsin}}\nolimits}
\newcommand{\Arccos}{\mathop{\mathrm{Arccos}}\nolimits}
\newcommand{\Arctan}{\mathop{\mathrm{Arctan}}\nolimits}
\newcommand{\Argsh}{\mathop{\mathrm{Argsh}}\nolimits}
\newcommand{\Argch}{\mathop{\mathrm{Argch}}\nolimits}
\newcommand{\Argth}{\mathop{\mathrm{Argth}}\nolimits}
\newcommand{\pgcd}{\mathop{\mathrm{pgcd}}\nolimits} 

%----- Structure des exercices ------

\newtheoremstyle{styleexo}% name
{2ex}% Space above
{3ex}% Space below
{}% Body font
{}% Indent amount 1
{\bfseries} % Theorem head font
{}% Punctuation after theorem head
{\newline}% Space after theorem head 2
{}% Theorem head spec (can be left empty, meaning ‘normal’)

\theoremstyle{styleexo}
\newtheorem{exo}{Exercice}
\newtheorem{ind}{Indications}
\newtheorem{cor}{Correction}


\newcommand{\exercice}[1]{} \newcommand{\finexercice}{}
%\newcommand{\exercice}[1]{{\tiny\texttt{#1}}\vspace{-2ex}} % pour afficher le numero absolu, l'auteur...
\newcommand{\enonce}{\begin{exo}} \newcommand{\finenonce}{\end{exo}}
\newcommand{\indication}{\begin{ind}} \newcommand{\finindication}{\end{ind}}
\newcommand{\correction}{\begin{cor}} \newcommand{\fincorrection}{\end{cor}}

\newcommand{\noindication}{\stepcounter{ind}}
\newcommand{\nocorrection}{\stepcounter{cor}}

\newcommand{\fiche}[1]{} \newcommand{\finfiche}{}
\newcommand{\titre}[1]{\centerline{\large \bf #1}}
\newcommand{\addcommand}[1]{}
\newcommand{\video}[1]{}

% Marge
\newcommand{\mymargin}[1]{\marginpar{{\small #1}}}



%----- Presentation ------
\setlength{\parindent}{0cm}

\definecolor{myred}{rgb}{0.93,0.26,0}
\definecolor{myorange}{rgb}{0.97,0.58,0}
\definecolor{myyellow}{rgb}{1,0.86,0}

\newcommand{\LogoExoSept}[1]{  % input : echelle
{\usefont{U}{cmss}{bx}{n}
\begin{tikzpicture}[scale=0.1*#1,transform shape]
  \fill[color=myorange] (0,0)--(4,0)--(4,-4)--(0,-4)--cycle;
  \fill[color=myred] (0,0)--(0,3)--(-3,3)--(-3,0)--cycle;
  \fill[color=myyellow] (4,0)--(7,4)--(3,7)--(0,3)--cycle;
  \node[scale=5] at (3.5,3.5) {Exo7};
\end{tikzpicture}}
}

% \newcommand{\ExoSept}{\href{http://exo7.emath.fr}{\textbf{\textsf{Exo7}}}}
% \newcommand{\LogoExoSept}{\setlength{\unitlength}{0.6em}
% \begin{picture}(0,0) \thicklines  \put(0,4){\line(0,1){3}}   \put(0,7){\line(1,0){3}}
%   \put(3,7){\line(0,-1){7}}  \put(0,4){\line(1,0){7}}   \put(3,0){\line(1,0){4}}
%   \put(7,0){\line(0,1){4}}   \put(3,7){\line(4,-3){4}}  \put(7,4){\line(3,4){3}}
%   \put(10,8){\line(-4,3){4}} \put(3,7){\line(3,4){3}}   \put(4.6,6.8){\mbox{\ExoSept}}
% \end{picture}}


\newcommand{\montitre}[1]{
 \vspace*{-4ex}
\LogoExoSept{3} \vspace*{-2ex}

%\hfill\texttt{\color{Wheat4}Math47 -- Université Lille 1} \\ 
\hfill\texttt{\color{Wheat4}le \today\ à \currenttime} \\ 
\vspace*{0.0ex}
{\color{Red3}
{\hrule height 3pt}\vspace*{1.8ex} 
\hfil\textsf{\textbf{\Large {#1}}}
\vspace*{2ex}
{\hrule height 3pt} 
}
\vspace*{2ex} 
% {\marginpar
% {\hfill
% \begin{minipage}{7cm}
% \color{darkgray}
{\footnotesize % \renewcommand{\contentsname}{\small Sommaire}
\renewcommand{\contentsname}{}
\tableofcontents
}
\vspace*{2ex}
% \end{minipage}}
%}
} 

%----- Commandes supplementaires ------

% \usepackage[landscape]{geometry}
% \geometry{top=1cm, bottom=3cm, left=2cm, right=10cm, marginparsep=1cm
% }
\usepackage[a4paper]{geometry}
\geometry{top=2cm, bottom=2cm, left=2cm, right=2cm, marginparsep=1cm}

%\usepackage{standalone}

\usepackage{fancybox}
\setlength{\fboxsep}{7pt}
\newcommand{\mabox}{\shadowbox}


% Commande tikz
\usetikzlibrary{calc}


%--- personnalisation theoremes ---%

%--- tentative de personnalisation pour les theoremes, definitions, etc.
%\makeatletter
%\newtheoremstyle{propthm}{\topsep}{\topsep}{\itshape}{}{\bfseries}{}{\newline}{%
%
%\hspace{-0.75em}%
%\begin{tikzpicture}
%%\fill[myred] (0pt,-0.5pt) rectangle (0.25\linewidth,0.5pt);
%%\fill[myred] (0,0) rectangle (1pt,-2.5em);
%\draw[myred,very thick,rounded corners] (0,-2.5em) -- (0,0) -- (0.25\linewidth,0);
%%\fill[myred] (\linewidth+10pt,-0.5pt) rectangle (\linewidth+10pt-0.25\linewidth,0.5pt);
%%\fill[myred] (\linewidth+10pt,0) rectangle (\linewidth+10pt-1pt,-2.5em);
%\node[inner sep=0em,outer sep=0pt,right,fill=white] at (0.75em,0em) {
%\thmname{#1}\thmnumber{ #2}\@ifnotempty{#3}{ (\thmnote{#3})}.
%};
%\end{tikzpicture}
%\vspace{-2.em}
%}
%
%\newtheoremstyle{definit}{\topsep}{\topsep}{\itshape}{}{\bfseries}{}{\newline}{%
%\hspace{-0.75em}%
%\begin{tikzpicture}
%%\fill[myred] (0pt,-0.5pt) rectangle (0.25\linewidth,0.5pt);
%%\fill[myred] (0,0) rectangle (1pt,-2.5em);
%\draw[myorange,very thick,rounded corners] (0,-2.5em) -- (0,0) -- (2.0em,0);
%%\fill[myred] (\linewidth+10pt,-0.5pt) rectangle (\linewidth+10pt-0.25\linewidth,0.5pt);
%%\fill[myred] (\linewidth+10pt,0) rectangle (\linewidth+10pt-1pt,-2.5em);
%\node[inner sep=0em,outer sep=0pt,right,fill=white] at (0.75em,0em) {
%\thmname{#1}\thmnumber{ #2}\@ifnotempty{#3}{ (\thmnote{#3})}.
%};
%\end{tikzpicture}
%\vspace{-2.em}
%}
%\makeatother


\theoremstyle{propthm}
%\newtheorem{proposition}{Proposition}
%\newtheorem{theoreme}{Théorème}

\theoremstyle{definit}
%\newtheorem{definition}{Définition}
%\newtheorem{corollaire}{Corollaire}

\theoremstyle{definition}
\newtheorem{exemple}{Exemple}
\newtheorem{lemme}{Lemme}
\newtheorem*{remarque*}{Remarque}




%--- personnalisation des theoremes avec boites arrondies colorees
%--- demande deux compilations

\makeatletter

\tikzstyle{every picture}+=[remember picture]


\newcounter{numdef}	% definition du compteur
\setcounter{numdef}{0}	% attribution de la valeur 0 au compteur

\newenvironment{definition}[1][]{%
\refstepcounter{numdef}	% +1 au compteur
\vskip\baselineskip		% nouvelle ligne avant de debuter l'environnement
% on place un node en haut a gauche de la boite, et l'intitule de la boite
\begin{tikzpicture}[remember picture]
\coordinate (n1);
\node[below right,inner sep=0] at (n1)
	{\textbf{D\'efinition \thenumdef\@ifnotempty{#1}{ (#1)}.} };
\end{tikzpicture}
}{%
\vskip\baselineskip		% nouvelle ligne pour placer le second coin de la boite
\hfill					% tout a droite !
% on place le node en bas a droite et on trace enfin le rectangle (avec la bonne position des la seconde compilation)
\begin{tikzpicture}[remember picture,overlay]
\coordinate (n2);
\draw[myorange,rounded corners,very thick]
	($(n1)+(-0.5em,0.5em)$) rectangle ($(n2)+(1em,1.5em)$);
\end{tikzpicture}
\par
}



\newcounter{numthm}
\setcounter{numthm}{0}

\newenvironment{theoreme}[1][]{%
\refstepcounter{numthm}%
\vskip\baselineskip
\begin{tikzpicture}[remember picture]
\coordinate (n1) {};
\node[below right,inner sep=0] at (n1)
	{\textbf{Th\'eor\`eme \thenumthm\@ifnotempty{#1}{ (#1)}.} };
\end{tikzpicture}
\newline
}{%
\vskip\baselineskip\hfill
\begin{tikzpicture}[remember picture,overlay]
\coordinate (n2) {};
\draw[myred,rounded corners,very thick]
	($(n1)+(-0.5em,0.5em)$) rectangle ($(n2)+(1em,1.5em)$);
\end{tikzpicture}
}


\newcounter{numprop}
\setcounter{numprop}{0}

\newenvironment{proposition}[1][]{%
\refstepcounter{numprop}%
\vskip\baselineskip
\begin{tikzpicture}[remember picture]
\coordinate (n1) {};
\node[below right,inner sep=0] at (n1)
	{\textbf{Proposition \thenumprop\@ifnotempty{#1}{ (#1)}.} };
\end{tikzpicture}
\newline
}{%
\vskip\baselineskip\hfill
\begin{tikzpicture}[remember picture,overlay]
\coordinate (n2) {};
\draw[myred,rounded corners,very thick]
	($(n1)+(-0.5em,0.5em)$) rectangle ($(n2)+(1em,1.5em)$);
\end{tikzpicture}
}

\newcounter{numcor}
\setcounter{numcor}{0}

\newenvironment{corollaire}[1][]{%
\refstepcounter{numcor}%
\vskip\baselineskip
\begin{tikzpicture}[remember picture]
\coordinate (n1) {};
\node[below right,inner sep=0] at (n1)
	{\textbf{Corollaire \thenumcor\@ifnotempty{#1}{ (#1)}.} };
\end{tikzpicture}
\newline
}{%
\vskip\baselineskip\hfill
\begin{tikzpicture}[remember picture,overlay]
\coordinate (n2) {};
\draw[myorange,rounded corners,very thick]
	($(n1)+(-0.5em,0.5em)$) rectangle ($(n2)+(1em,1.5em)$);
\end{tikzpicture}
}


\makeatother







%----- Commandes geometrie ------

\newcommand{\droite}[1]{(#1)}
\newcommand{\segment}[1]{[#1]}
\newcommand{\aire}[1]{\mathcal{A}(#1)}
\newcommand{\milieu}[1]{\mathop{\mathrm{mil}}[#1]}
\newcommand{\cercle}{\mathcal{C}}
\newcommand{\plan}{\mathcal{P}}
\newcommand{\arc}[1]{\overset{\frown}{#1}}

%----- Commandes geometrie ------

%\nom{le prénom}{le nom}
\newcommand{\nom}[2]{\textsc{#1 #2}}

% \triangle, \angle

%----- Commandes ifs ------
\newcommand{\myvec}[2]{\begin{pmatrix}#1 \\ #2\end{pmatrix}}


%definition d'un terme
\newcommand{\defi}[1]{{\color{myorange}\textbf{\emph{#1}}}}
\newcommand{\evidence}[1]{{\color{blue}\textbf{\emph{#1}}}}

%\renewcommand{\contentsname}{Sommaire}
\renewcommand{\contentsname}{}
\setcounter{tocdepth}{2}

 %----- Commandes divers ------

\newcommand{\ii}{\mathrm{i}}
\newcommand{\dd}{\mathrm{d}}
\newcommand{\pause}{}
\newcommand{\Ker}{\mathop{\text{Ker}}\nolimits}
\newcommand{\id}{\mathop{\text{id}}\nolimits}
\newcommand{\Card}{\mathop{\text{Card}}\nolimits}

%------ Figures ------

\def\myscale{1} % par défaut 
\newcommand{\myfigure}[2]{  % entrée : echelle, fichier figure
\def\myscale{#1}
\begin{center}
\footnotesize
{#2}
\end{center}}

%------ Encadrement ------
\newcommand{\mybox}[1]{
\begin{center}
\shadowbox{#1}
\end{center}}

\newcommand{\myboxinline}[1]{
\raisebox{-2ex}{
\shadowbox{#1}
}
}
