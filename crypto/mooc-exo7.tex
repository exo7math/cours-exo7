
\input{../preamb-poly-book.tex}

% Ne pas oublier de commenter les préambules de chaque chapitre !



% Pour inclure les fiches exercices
%\usepackage{pdfpages}

% Autre méthode pdftk : garde les liens hypertextes
% pdftk mooc-exo7.pdf exo-arithmetique.pdf exo-logique-ensemble.pdf cat output livre-mooc-exo7.pdf


\begin{document}

% Titre + sommaire
\montitre{Arithmétique : en route pour la cryptographie\\ Un MOOC}


% Les bulles
% Occasionne des erreurs 
~
% \vfil
% \myfigure{0.8}{
% %\usetikzlibrary{shadows,arrows}

% Define the layers to draw the diagram
\pgfdeclarelayer{background}
\pgfdeclarelayer{foreground}
\pgfsetlayers{background,main,foreground}

% Define block styles
\tikzstyle{materia}=[draw, fill=blue!20, text width=6.0em, text centered,
  minimum height=1.5em,drop shadow]
\tikzstyle{practica} = [materia, text width=8em, minimum width=10em,
  minimum height=3em, rounded corners, drop shadow]
\tikzstyle{texto} = [above, text width=6em, text centered]
\tikzstyle{linepart} = [draw, thick, color=black!50, -latex', dashed]
\tikzstyle{line} = [draw, very thick, shorten <=3pt,shorten >=3pt, color=black!70, -latex']
\tikzstyle{biline} = [draw, very thick, shorten <=3pt, shorten >=3pt, color=black!70, <->, >=latex]
\tikzstyle{ur}=[draw, text centered, minimum height=0.01em]

% Define distances for bordering
\newcommand{\blockdist}{1.3}
\newcommand{\edgedist}{1.5}

\newcommand{\practica}[2]{node (p#1) [practica] {#2}}

% Mes modules
\newcommand{\logique}[2]{node (p#1) [practica, fill=blue!30] {#2}}
\newcommand{\algebreun}[2]{node (p#1) [practica, fill=green!30] {#2}}
\newcommand{\analyseun}[2]{node (p#1) [practica, fill=red!30] {#2}}
\newcommand{\algebredeux}[2]{node (p#1) [practica, fill=green!60] {#2}}
\newcommand{\analysedeux}[2]{node (p#1) [practica, fill=red!60] {#2}}
\newcommand{\geometrie}[2]{node (p#1) [practica, fill=orange!60] {#2}}

% Draw background
\newcommand{\background}[5]{%
  \begin{pgfonlayer}{background}
    % Left-top corner of the background rectangle
    \path (#1.west |- #2.north)+(-0.5,0.5) node (a1) {};
    % Right-bottom corner of the background rectanle
    \path (#3.east |- #4.south)+(+0.5,-0.25) node (a2) {};
    % Draw the background
    \path[fill=yellow!20,rounded corners, draw=black!50, dashed]
      (a1) rectangle (a2);
    \path (a1.east |- a1.south)+(0.8,-0.3) node (u1)[texto]
      {\scriptsize\textit{Unidad #5}};
  \end{pgfonlayer}}

\newcommand{\transreceptor}[3]{%
  \path [linepart] (#1.east) -- node [above]
    {\scriptsize Transreceptor #2} (#3);}

\begin{tikzpicture}[scale=1.4]
  % Draw diagram elements

% Logique
  \path \logique{1}{Logique \& Raisonnements};

% Algebre 1
  \path (p1.east)+(2.5,1.5) \algebreun{2}{Ensembles \& Applications};
 % \path (p2.east)+(1,-1.5) \algebreun{3}{};
  \path (p2.east)+(1,1.5) \algebreun{4}{Arithm\'etique};
  \path (p2.east)+(-2,3.5) \algebreun{5}{Nombres complexes};
  \path (p4.east)+(-0.5,1.5) \algebreun{6}{Polyn\^omes};

% Algebre 2

  \path (p6.east)+(3.5,-.5) \algebredeux{21}{Espaces vectoriels};
  \path (p21.east)+(-3,1.5) \algebredeux{22}{Groupes};
  \path (p21.east)+(-3,-1.5) \algebredeux{23}{Syst\`emes lin\'eaires};
  \path (p21.east)+(1,2) \algebredeux{24}{Dimension finie};
  \path (p21.east)+(2.5,-1.5) \algebredeux{25}{Matrices};
  \path (p21.east)+(2.5,0.5) \algebredeux{26}{Applications lin\'eaires};
  \path (p25.east)+(2,1) \algebredeux{27}{D\'eterminants};


% Geometrie

  \path (p6.east)+(3.5,-4.5) \geometrie{31}{Droites et plans};
  \path (p31.east)+(4,-1) \geometrie{32}{Courbes param\'etr\'es};
  \path (p31.east)+(3,0.5) \geometrie{33}{G\'eom\'etrie affine et euclidienne};

% Analyse 1
   \path (p1.east)+(1.0,-1.5) \analyseun{12}{Nombres r\'eels};
   \path (p12.east)+(0.5,-2) \analyseun{13}{Suites I};
   \path (p12.east)+(3,0) \analyseun{14}{Fonctions continues};
   \path (p14.east)+(-1,-4) \analyseun{15}{Z\'eros de fonctions};
   \path (p14.east)+(2.5,-2.5) \analyseun{16}{D\'eriv\'ees};
   \path (p14.east)+(2.5,-0.2) \analyseun{17}{Trigonom\'etrie \\ Fonctions usuelles};
% Analyse 2
   \path (p16.east)+(2.5,1.5) \analysedeux{41}{D\'eveloppements limit\'es};
   \path (p16.east)+(2.5,-1) \analysedeux{42}{Int\'egrales I};
   \path (p42.east)+(1,-1.5) \analysedeux{43}{Int\'egrales II};
   \path (p41.east)+(2.5,-0.5) \analysedeux{44}{Suites II};
   \path (p42.east)+(3.5,0) \analysedeux{45}{\'Equations diff\'erentielles};


  % Draw arrows between elements

% Algebre 1
  \path[line] (p1.north) -- node [above] {} (p2);
%  \path[line] (p2.south east) -- node [above] {} (p3);
  \path[line] (p4.north) -- node [above] {} (p6);
  \path[line] (p5.east) -- node [above] {} (p6);

% Analyse 1
  \path[line] (p1.south) -- node [above] {} (p12.north west);
  \path[line] (p12.south) -- node [above] {} (p13.north);
  \path[line] (p2.south) -- node [above] {} (p14);
  \path[biline] (p13.north) -- node [above] {} (p14);
  \path[line] (p14.south) -- node [above] {} (p15.north);
  \path[line] (p13.south) -- node [above] {} (p15.north);
  \path[line] (p16.west) -- node [above] {} (p15.north);
  \path[line] (p14.south) -- node [above] {} (p16);
  \path[line] (p14.east) -- node [above] {} (p17);

% Algebre 2
  \path[line] (p23.north) -- node [above] {} (p21);
  \path[line] (p22.south) -- node [above] {} (p21);
  \path[line] (p21.north) -- node [above] {} (p24);
  \path[biline] (p21.south east) -- node [above] {} (p25.north west);
  \path[line] (p21.east) -- node [above] {} (p26);
  \path[line] (p25.east) -- node [above] {} (p27);
  \path[biline] (p25.north) -- node [above] {} (p26);
% Analyse 2
  \path[line] (p16.north) -- node [above] {} (p41.west);
  \path[line] (p41.north) -- node [above] {} (p32.south west);
  \path[line] (p16.south east) -- node [above] {} (p42.west);
  \path[line] (p42.south) -- node [above] {} (p43.west);
  \path[line] (p42.east) -- node [above] {} (p45.west);
  \path[line] (p41.east) -- node [above] {} (p44.west);

% Geometrie
  \path[line] (p31.east) -- node [above] {} (p33);
  \path[line] (p31.north) -- node [above] {} (p23);

\end{tikzpicture}

% }

\vfil

\centerline{\textbf{\color{myred} \Large  \textsf{Arithmétique : en route pour la cryptographie}}}

\bigskip
\bigskip
{\sf
Vous voulez comprendre l'arithmétique ? Vous souhaitez découvrir 
une application des mathématiques à la vie quotidienne ? Ce cours 
est fait pour vous ! De niveau première année d'université, vous 
apprendrez les bases de l'arithmétique (division euclidienne, 
théorème de Bézout, nombres premiers, congruences). 

\bigskip

Vous vous êtes déjà demandé comment sont sécurisées les transactions sur Internet  ? 
Vous découvrirez les bases de la cryptographie, en commençant par 
les codes les plus simples pour aboutir au code RSA. Le code RSA 
est le code utilisé pour crypter les communications sur internet. 
Il est basé sur de l'arithmétique assez simple que l'on comprendra 
en détails. Vous pourrez en plus mettre en pratique vos connaissances par 
l'apprentissage de notions sur le langage de programmation Python.

\bigskip

Comment utiliser ce support de cours ? 
Le but est de comprendre en détails les deux premiers chapitres ! 
Les autres chapitres sont là pour vous aider si vous en avez besoin.

\begin{itemize}
  \item Chapitre 1. \textbf{\color{myred} Arithmétique.} Le c\oe ur de nos préoccupations, nous étudierons tout pas à pas et 
  en détails .
  
  \item Chapitre 2. \textbf{\color{myred} Cryptographie.} Notre motivation : comprendre le chiffrement RSA.
  
  \item Chapitre 3. \textbf{\color{myred} Algorithme.} Nous aurons besoin d'un petit peu 
  de programmation pour <<casser>> des codes secrets. Nous n'étudierons pas ce chapitre en entier.
  
  \item Chapitre 4 et  5. Pour ceux qui ont besoin d'une petite remise à niveau, 
  des rappels sur la \textbf{\color{myred} logique} et les \textbf{\color{myred} ensembles}.
  
  \item Des \textbf{\color{myred} exercices} pour l'arithmétique que l'on travaillera en profondeur.
  Et aussi pour ceux qui le souhaitent des exercices de remise à niveau pour la logique et les ensembles.
  
\end{itemize}


\bigskip

Plan de travail
\begin{itemize}
  \item \textbf{\color{blue} Semaine 0} 
  
  Remise à niveau sur les chapitres <<Logique>> et <<Ensembles>>.
  
  \item \textbf{\color{blue} Semaine 1} 
  
  Cryptographie : Le chiffrement de César
  
  Mathématiques : Division euclidienne
  
  \item \textbf{\color{blue} Semaine 2} 
  
  Cryptographie : Le chiffre de Vigenère
  
  Mathématiques : pgcd
  
  \item \textbf{\color{blue} Semaine 3}
  
  Cryptographie : La machine Enigma et les clés secrètes
  
  Mathématiques : Théorème de Bézout 
  
  \item \textbf{\color{blue} Semaine 4}
  
  Cryptographie : La cryptographie à clé publique
  
  Mathématiques : Nombres premiers  
  
  \item \textbf{\color{blue} Semaine 5}
  
  Cryptographie : L'arithmétique pour RSA
  
  Mathématiques : Congruences  
  
  \item \textbf{\color{blue} Semaine 6}
  
  Cryptographie : Le chiffrement RSA

\end{itemize}

\bigskip
\bigskip
\bigskip

\hfil\hfil Bon courage ! 

\hfil\hfil Arnaud Bodin \& François Recher

}% end sf

\vfill
\centerline{Licence Creative Commons - BY-NC-SA}

\newpage

% Les chapitres


\part{Le cours du MOOC}

\import{../arithmetique/}{ch_arithmetique.tex}

\import{../crypto/}{ch_crypto.tex}

\import{../algo/}{ch_algo.tex}

\part{Les rappels de cours}

\import{../logique/}{ch_logique.tex}

\import{../ensembles/}{ch_ensembles.tex}

\part{Les exercices}

% A ajouter à la main avec la commande pdftk (voir en haut de ce fichier)
% % 
% % \includepdf[pages=1-10]{exo-arithmetique.pdf}
% % 
% % \includepdf{exo-logique-ensemble.pdf}

\end{document}
