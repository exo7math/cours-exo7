
%%%%%%%%%%%%%%%%%% PREAMBULE %%%%%%%%%%%%%%%%%%


\documentclass[12pt]{article}

\usepackage{amsfonts,amsmath,amssymb,amsthm}
\usepackage[utf8]{inputenc}
\usepackage[T1]{fontenc}
\usepackage[francais]{babel}


% packages
\usepackage{amsfonts,amsmath,amssymb,amsthm}
\usepackage[utf8]{inputenc}
\usepackage[T1]{fontenc}
%\usepackage{lmodern}

\usepackage[francais]{babel}
\usepackage{fancybox}
\usepackage{graphicx}

\usepackage{float}

%\usepackage[usenames, x11names]{xcolor}
\usepackage{tikz}
\usepackage{datetime}

\usepackage{mathptmx}
%\usepackage{fouriernc}
%\usepackage{newcent}
\usepackage[mathcal,mathbf]{euler}

%\usepackage{palatino}
%\usepackage{newcent}


% Commande spéciale prompteur

%\usepackage{mathptmx}
%\usepackage[mathcal,mathbf]{euler}
%\usepackage{mathpple,multido}

\usepackage[a4paper]{geometry}
\geometry{top=2cm, bottom=2cm, left=1cm, right=1cm, marginparsep=1cm}

\newcommand{\change}{{\color{red}\rule{\textwidth}{1mm}\\}}

\newcounter{mydiapo}

\newcommand{\diapo}{\newpage
\hfill {\normalsize  Diapo \themydiapo \quad \texttt{[\jobname]}} \\
\stepcounter{mydiapo}}


%%%%%%% COULEURS %%%%%%%%%%

% Pour blanc sur noir :
%\pagecolor[rgb]{0.5,0.5,0.5}
% \pagecolor[rgb]{0,0,0}
% \color[rgb]{1,1,1}



%\DeclareFixedFont{\myfont}{U}{cmss}{bx}{n}{18pt}
\newcommand{\debuttexte}{
%%%%%%%%%%%%% FONTES %%%%%%%%%%%%%
\renewcommand{\baselinestretch}{1.5}
\usefont{U}{cmss}{bx}{n}
\bfseries

% Taille normale : commenter le reste !
%Taille Arnaud
%\fontsize{19}{19}\selectfont

% Taille Barbara
%\fontsize{21}{22}\selectfont

%Taille François
\fontsize{25}{30}\selectfont

%Taille Pascal
%\fontsize{25}{30}\selectfont

%Taille Laura
%\fontsize{30}{35}\selectfont


%\myfont
%\usefont{U}{cmss}{bx}{n}

%\Huge
%\addtolength{\parskip}{\baselineskip}
}


% \usepackage{hyperref}
% \hypersetup{colorlinks=true, linkcolor=blue, urlcolor=blue,
% pdftitle={Exo7 - Exercices de mathématiques}, pdfauthor={Exo7}}


%section
% \usepackage{sectsty}
% \allsectionsfont{\bf}
%\sectionfont{\color{Tomato3}\upshape\selectfont}
%\subsectionfont{\color{Tomato4}\upshape\selectfont}

%----- Ensembles : entiers, reels, complexes -----
\newcommand{\Nn}{\mathbb{N}} \newcommand{\N}{\mathbb{N}}
\newcommand{\Zz}{\mathbb{Z}} \newcommand{\Z}{\mathbb{Z}}
\newcommand{\Qq}{\mathbb{Q}} \newcommand{\Q}{\mathbb{Q}}
\newcommand{\Rr}{\mathbb{R}} \newcommand{\R}{\mathbb{R}}
\newcommand{\Cc}{\mathbb{C}} 
\newcommand{\Kk}{\mathbb{K}} \newcommand{\K}{\mathbb{K}}

%----- Modifications de symboles -----
\renewcommand{\epsilon}{\varepsilon}
\renewcommand{\Re}{\mathop{\text{Re}}\nolimits}
\renewcommand{\Im}{\mathop{\text{Im}}\nolimits}
%\newcommand{\llbracket}{\left[\kern-0.15em\left[}
%\newcommand{\rrbracket}{\right]\kern-0.15em\right]}

\renewcommand{\ge}{\geqslant}
\renewcommand{\geq}{\geqslant}
\renewcommand{\le}{\leqslant}
\renewcommand{\leq}{\leqslant}

%----- Fonctions usuelles -----
\newcommand{\ch}{\mathop{\mathrm{ch}}\nolimits}
\newcommand{\sh}{\mathop{\mathrm{sh}}\nolimits}
\renewcommand{\tanh}{\mathop{\mathrm{th}}\nolimits}
\newcommand{\cotan}{\mathop{\mathrm{cotan}}\nolimits}
\newcommand{\Arcsin}{\mathop{\mathrm{Arcsin}}\nolimits}
\newcommand{\Arccos}{\mathop{\mathrm{Arccos}}\nolimits}
\newcommand{\Arctan}{\mathop{\mathrm{Arctan}}\nolimits}
\newcommand{\Argsh}{\mathop{\mathrm{Argsh}}\nolimits}
\newcommand{\Argch}{\mathop{\mathrm{Argch}}\nolimits}
\newcommand{\Argth}{\mathop{\mathrm{Argth}}\nolimits}
\newcommand{\pgcd}{\mathop{\mathrm{pgcd}}\nolimits} 

\newcommand{\Card}{\mathop{\text{Card}}\nolimits}
\newcommand{\Ker}{\mathop{\text{Ker}}\nolimits}
\newcommand{\id}{\mathop{\text{id}}\nolimits}
\newcommand{\ii}{\mathrm{i}}
\newcommand{\dd}{\mathrm{d}}
\newcommand{\Vect}{\mathop{\text{Vect}}\nolimits}
\newcommand{\Mat}{\mathop{\mathrm{Mat}}\nolimits}
\newcommand{\rg}{\mathop{\text{rg}}\nolimits}
\newcommand{\tr}{\mathop{\text{tr}}\nolimits}
\newcommand{\ppcm}{\mathop{\text{ppcm}}\nolimits}

%----- Structure des exercices ------

\newtheoremstyle{styleexo}% name
{2ex}% Space above
{3ex}% Space below
{}% Body font
{}% Indent amount 1
{\bfseries} % Theorem head font
{}% Punctuation after theorem head
{\newline}% Space after theorem head 2
{}% Theorem head spec (can be left empty, meaning ‘normal’)

%\theoremstyle{styleexo}
\newtheorem{exo}{Exercice}
\newtheorem{ind}{Indications}
\newtheorem{cor}{Correction}


\newcommand{\exercice}[1]{} \newcommand{\finexercice}{}
%\newcommand{\exercice}[1]{{\tiny\texttt{#1}}\vspace{-2ex}} % pour afficher le numero absolu, l'auteur...
\newcommand{\enonce}{\begin{exo}} \newcommand{\finenonce}{\end{exo}}
\newcommand{\indication}{\begin{ind}} \newcommand{\finindication}{\end{ind}}
\newcommand{\correction}{\begin{cor}} \newcommand{\fincorrection}{\end{cor}}

\newcommand{\noindication}{\stepcounter{ind}}
\newcommand{\nocorrection}{\stepcounter{cor}}

\newcommand{\fiche}[1]{} \newcommand{\finfiche}{}
\newcommand{\titre}[1]{\centerline{\large \bf #1}}
\newcommand{\addcommand}[1]{}
\newcommand{\video}[1]{}

% Marge
\newcommand{\mymargin}[1]{\marginpar{{\small #1}}}



%----- Presentation ------
\setlength{\parindent}{0cm}

%\newcommand{\ExoSept}{\href{http://exo7.emath.fr}{\textbf{\textsf{Exo7}}}}

\definecolor{myred}{rgb}{0.93,0.26,0}
\definecolor{myorange}{rgb}{0.97,0.58,0}
\definecolor{myyellow}{rgb}{1,0.86,0}

\newcommand{\LogoExoSept}[1]{  % input : echelle
{\usefont{U}{cmss}{bx}{n}
\begin{tikzpicture}[scale=0.1*#1,transform shape]
  \fill[color=myorange] (0,0)--(4,0)--(4,-4)--(0,-4)--cycle;
  \fill[color=myred] (0,0)--(0,3)--(-3,3)--(-3,0)--cycle;
  \fill[color=myyellow] (4,0)--(7,4)--(3,7)--(0,3)--cycle;
  \node[scale=5] at (3.5,3.5) {Exo7};
\end{tikzpicture}}
}



\theoremstyle{definition}
%\newtheorem{proposition}{Proposition}
%\newtheorem{exemple}{Exemple}
%\newtheorem{theoreme}{Théorème}
\newtheorem{lemme}{Lemme}
\newtheorem{corollaire}{Corollaire}
%\newtheorem*{remarque*}{Remarque}
%\newtheorem*{miniexercice}{Mini-exercices}
%\newtheorem{definition}{Définition}




%definition d'un terme
\newcommand{\defi}[1]{{\color{myorange}\textbf{\emph{#1}}}}
\newcommand{\evidence}[1]{{\color{blue}\textbf{\emph{#1}}}}



 %----- Commandes divers ------

\newcommand{\codeinline}[1]{\texttt{#1}}

%%%%%%%%%%%%%%%%%%%%%%%%%%%%%%%%%%%%%%%%%%%%%%%%%%%%%%%%%%%%%
%%%%%%%%%%%%%%%%%%%%%%%%%%%%%%%%%%%%%%%%%%%%%%%%%%%%%%%%%%%%%


\begin{document}

\debuttexte


%%%%%%%%%%%%%%%%%%%%%%%%%%%%%%%%%%%%%%%%%%%%%%%%%%%%%%%%%%%
\diapo


Dans cette le\c{c}on nous abordons la notion \defi{fondamentale} de base d'un espace vectoriel. \\

\change
Le plan de la le\c{c}on est le suivant :

\change
Dans un premier temps nous donnerons la définition de base

\change
ensuite nous l'illustrerons par des exemples

\change
nous donnerons un théorème d'existence de bases

\change
puis le théorème de la base incomplète et ses variantes.
% 
% \change
% et enfin nous donnerons les schémas de preuves de ces résultats.

%%%%%%%%%%%%%%%%%%%%%%%%%%%%%%%%%%%%%%%%%%%%%%%%%%%%%%%%%%%
\diapo

Dans les deux dernières le\c{c}ons nous avons introduit la notion de 
famille libre d'une part et de famille génératrice d'autre part. C'est le moment de combiner ces deux notions pour définir la notion de base.\\

Une famille $\mathcal{B}$ de vecteurs d'un $\Kk$-espace vectoriel $E$ 
est une \defi{base} de $E$
si c'est une famille libre \evidence{et} génératrice.\\

%%%%%
C'est une \defi{généralisation} de la notion de repère. \\

Dans $\Rr^2$, un repère est donné par un couple de vecteurs 
non colinéaires. Dans $\Rr^ 3$, un repère est donné par un 
triplet de vecteurs non coplanaires.  
Dans un repère, 
un vecteur se décompose suivant un ensemble de vecteurs donnés. 
Il en est de même pour une base d'un espace vectoriel. 

%%%%%


Dans ce cas, $\mathcal{B}$ joue le rôle de repère de l'espace $E$ 
comme le montre le théorème suivant.


\change
Soit $\mathcal{B}$ une base de l'espace vectoriel $E$.
Tout vecteur $v$ de $E$ s'exprime de façon unique comme combinaison 
linéaire des vecteurs  $v_1, v_2, \dots , v_n$ de $\mathcal{B}$.
Autrement dit, il \evidence{existe}  $\lambda_1,\ldots,\lambda_n$ 
\evidence{uniques} tels que $v$ est combinaison linéaire de $v_1, \dots , v_n$ avec pour coefficients $\lambda_1,\ldots,\lambda_n$.

\change
Les coefficients $\lambda_1,\ldots,\lambda_n$ sont appelés \defi{coordonnées} de $v$ dans la base $\mathcal{B}$. \\

\change
Remarquons que la base $v$ est ordonnée, ce qui veut dire que l'on peut 
parler du coefficient devant le premier vecteur comme de la \defi{première} coordonnées, 

\change
du coefficient devant le second vecteur comme de la \defi{deuxième} coordonnée et ainsi de suite.

% \change
% Notons également que l'application qui à $\lambda_1,\ldots,\lambda_n$ associe le vecteur $v$ de $E$ de coordonnées $\lambda_1,\ldots,\lambda_n$ est un isomorphisme de $\Kk^n$ vers $E$. Ainsi, se donner une base d'un $\Kk$-espace vectoriel c'est se ramener au cas de l'espace vectoriel connu $\Kk^n$.



%%%%%%%%%%%%%%%%%%%%%%%%%%%%%%%%%%%%%%%%%%%%%%%%%%%%%%%%%%%
\diapo
Passons maintenant aux exemples.\\

Tout d'abord, dans $\Rr^2$, les vecteurs $e_1$ et $e_2$ 
suivants forment la base dite \defi{canonique} de $\Rr^2$. 
C'est la base que nous utilisons implicitement  
lorsque nous parlons des coordonnées $x$ et $y$ d'un vecteur du plan qui s'écrit
$xe_1+ye_2$. 
Mais $(e_1,e_2)$ n'est pas la seule base.

\change
Les vecteurs $v_1$ et $v_2$ suivants forment \defi{tout autant} une base de $\Rr^2$ 
Ainsi le \defi{m\^eme} vecteur $v$ 
se décompose sous la forme $v=\lambda_1 v_1 + \lambda_2 v_2$
par rapport à cette base $(v_1, v_2)$.

\change
De même dans $\Rr^3$, les vecteurs $e_1$, $e_2$, $e_3$ 
suivants forment la base \defi{canonique} de $\Rr^3$.

\change
car tout vecteur $v$ de $\Rr^3$ s'écrit comme combinaison linéaire de $e_1$, $e_2$, $e_3$.

%%%%%%%%%%%%%%%%%%%%%%%%%%%%%%%%%%%%%%%%%%%%%%%%%%%%%%%%%%%
\diapo


Considérons maintenant les vecteurs $v_1$, $v_2$ et $v_3$ suivants. 
La famille constituée de ces 3 vecteurs est-elle une base de $\Rr^3$? 

\change
Pour répondre à cette question il faut savoir 
(1) si c'est famille est génératrice ?

\change
et (2) si c'est une famille libre?


\change
Pour répondre à la première question nous allons \^etre ramenés à  

l'étude d'un système linéaire. 


\change
En effet, $\mathcal{B}$ engendre $\Rr^3$ si tout vecteur $v$ %de coordonnées $a_1$, $a_2$, $a_3$
peut s'écrire comme combinaison linéaire de $v_1$, $v_2$, $v_3$.

\change
Une combinaison linéaire de $v_1$, $v_2$ et $v_3$ avec pour 
coefficients $\lambda_1$, $\lambda_2$ et $\lambda_3$ s'écrit de la manière suivante

\change
que l'on regroupe en un seul vecteur.


\change
Ainsi en comparant avec les coordonnées de $v$ nous obtenons le système linéaire suivant.
Il nous restera à montrer que ce système a une solution $\lambda_1, \lambda_2, \lambda_3$
quelque soit les coordonnées $a_1$, $a_2$, $a_3$ du vecteur $v$.

En résumé, lorsqu'on veut répondre à la première question, 
on aboutit à l'étude de ce premier système.


%%%%%%%%%%%%%%%%%%%%%%%%%%%%%%%%%%%%%%%%%%%%%%%%%%%%%%%%%%%
\diapo

Pour savoir si la famille $\mathcal{B}$ formée 
de $v_1$, $v_2$ et $v_3$ est une famille libre de $\Rr^3$ nous 
devons également étudier un système linéaire. Quel est-il? 

\change
Reprenons la définition, la famille $\mathcal{B}$ est libre 
si cette équation $ \lambda_1 v_1+ \lambda_2 v_2 + \lambda_3 v_3= 0$ 
admet comme unique solution celle o\`u tous les coefficients sont nuls.


\change
On réécrit cette équation coordonnées par coordonnées ce qui donne le système linéaire  de 3 équations à 3 inconnues suivant, 
et il s'agit de savoir s'il possède une solution unique $(0,0,0)$.

Recapitulons :

La famille $\mathcal{B}$ est \defi{génératrice} 
si le système de la diapo précédente avait une solution

et elle est \defi{libre}
si le ce second système admet l'\defi{unique} solution $(0,0,0)$.

\change
Remarquez que les deux systèmes ont la même matrice de coefficients. 
En fait le second système est le système homogène associé au premier.
  On peut donc montrer simultanément que $\mathcal{B}$ est une famille génératrice 
  et une famille libre de $\Rr^3$...
  
\change
...en montrant que la matrice des coefficients est inversible.

\change
En effet, si la matrice des coefficients est inversible, alors le premier système
admet une solution unique $(\lambda_1,\lambda_2,\lambda_3)$ 
quel que soit les valeurs $(a_1,a_2,a_3)$ du second membre, 

\change
en particulier c'est aussi le cas lorsque le second membre est nul 
comme c'est le cas dans le second système.\\
  
Pour montrer que la matrice du système est inversible, 
on peut calculer son inverse ou seulement montrer que son déterminant est non nul. 

\change
En conclusion, la famille $\mathcal{B}$ est à la fois une famille génératrice 
et une famille libre, c'est bien une base de $\Rr^3$ !


%%%%%%%%%%%%%%%%%%%%%%%%%%%%%%%%%%%%%%%%%%%%%%%%%%%%%%%%%%%
\diapo
Passons maintenant à un exemple dans $\Kk^n$. 
De m\^eme que dans $\Rr^2$ ou $\Rr^3$, on définit la \defi{base canonique} 
de $\Kk^n$ comme la famille de vecteurs suivants.

\change
Considérons maintenant la famille $v_1$, $v_2$, $v_n$ suivante. 

C'est un bon exercice de montrer que c'est une base de $\Kk^n$. 

Pour cela il faut montrer que c'est une famille libre puis une famille génératrice,
ce qui se fait à chaque fois en résolvant un système linéaire, 
mais qui est ici triangulaire donc facile à résoudre.





%%%%%%%%%%%%%%%%%%%%%%%%%%%%%%%%%%%%%%%%%%%%%%%%%%%%%%%%%%%
\diapo
La base canonique de l'espace vectoriel $\Rr_n[X]$ des polynômes 
de degré inférieur ou égal à $n$ est la famille formé des polynômes $1,X,X^2, \ldots , X^n$. 
  Attention, il y a bien $n+1$ éléments dans cette base!
  
\change
Voici une autre base de $\Rr_n[X]$ :
  $(1,1+X,1+X+X^2,\ldots,1+X+X^2+\cdots+X^n)$.
  
%%%%%%%%%%%%%%%%%%%%%%%%%%%%%%%%%%%%%%%%%%%%%%%%%%%%%%%%%%%
\diapo
Voyons maintenant le théorème d'existence d'une base. 

L'énoncé est tout simple :

Théorème :  Tout espace vectoriel admettant une famille finie génératrice
admet une base.



%%%%%%%%%%%%%%%%%%%%%%%%%%%%%%%%%%%%%%%%%%%%%%%%%%%%%%%%%%%%%
\diapo
Une version importante et plus générale de ce qui précède est le théorème suivant :\\

Soit $E$ un $\Kk$-espace vectoriel admettant une famille génératrice finie.\\

\change
Premier point :
 \defi{Toute famille libre $\mathcal{L}$ peut être complétée en une base.}
  C'est-à-dire qu'il existe une famille $\mathcal{F}$ telle que 
  $\mathcal{L} \cup \mathcal{F}$ soit une famille libre et génératrice de $E$.\\
  
  
\change
Deuxième point :
 \defi{De toute famille génératrice $\mathcal{G}$ on peut extraire une base de $E$.}
  C'est-à-dire qu'il existe une famille $\mathcal{B} \subset \mathcal{G}$ telle que 
  $\mathcal{B}$ soit une famille libre et génératrice de $E$.

%%%%%%%%%%%%%%%%%%%%%%%%%%%%%%%%%%%%%%%%%%%%%%%%%%%%%%%%%%%
\diapo
Les deux théorèmes précédents sont la conséquence d'un résultat encore plus général :\\

Soit $\mathcal{G}$ une famille génératrice finie de $E$ 
et $\mathcal{L}$ une famille libre de $E$. \\

Alors il existe une famille $\mathcal{F}$ de $\mathcal{G}$ telle que
$\mathcal{L} \cup \mathcal{F}$ soit une base de $E$\\

c'est-à-dire que l'on peut compléter la famille libre $\mathcal{L}$ en une base de $E$, et ce en n'utilisant que des éléments de $\mathcal{G}$.
La démonstration de ce théorème est un algorithme que nous allons appliquer à un exemple concret.



%%%%%%%%%%%%%%%%%%%%%%%%%%%%%%%%%%%%%%%%%%%%%%%%%%%%%%%%%%%
\diapo

Dans cet exemple, nous considérons les 5 polynômes suivants.

\change
$E$ est le sous-espace de l'espace des polynômes engendré par les polynômes  
$P_1, P_2, P_3, P_4, P_5$.


\change
 La famille $\mathcal{G}$ est ici formée par ces 5 polynômes particuliers.

\change
De plus dans cet exemple, nous allons prendre comme famille libre $\mathcal{L}$ 
l'ensemble vide. 

\change
Présentons un algorithme qui permet de compléter $\mathcal{L}$ en une base de $E$
à partir d'éléments de $\mathcal{G}$.

\change
A l'étape $0$ on remarque que la famille $\mathcal{L}$ n'est pas génératrice vu que c'est l'ensemble vide. On passe alors à l'étape suivante.

\change
A l'étape $1$, on ajoute le polynôme $P_1$ à $\mathcal{L}$. La liste obtenue est donc constituée du polynôme $P_1$ qui est non nul donc forme une famille libre. Néanmoins $P_1$ n'est pas une famille génératrice de $E$. On passe donc à l'étape suivante.

\change
Considérons $P_2$. 
  Comme les éléments $P_1$ et $P_2$ sont linéairement indépendants, la famille
  $\{ P_1, P_2 \}$ est une famille libre. Néanmoins ce n'est pas une famille génératrice de $E$. On passe donc à l'étape suivante.
  
  \change
Considérons $P_3$ : 
  ce vecteur est combinaison linéaire des vecteurs  $P_1$ et $P_2$ car 
  $P_3 = P_1 + P_2$ donc $\{P_1, P_2, P_3 \}$ est une famille liée. Le vecteur $P_3$ n'est pas ajouté à la liste.
  
  \change
    Considérons alors $P_4$.  Un calcul rapide prouve que les vecteurs $P_1$, $P_2$  et 
  $P_4$ sont linéairement indépendants. Alors  la famille $ \{P_1, P_2, P_4\}$ 
  est une famille libre. 
  
  \change
  C'est aussi une famille génératrice de $E$ car le vecteur $P_5$, dernier vecteur de $\mathcal{G}$ à considérer, s'écrit $P_1+P_2-P_4$.
 
 \change
  Donc l'algorithme s'arr\^ete et la famille $\mathcal{F} = \{P_1, P_2, P_4\}$ est une base de $E$.

Voici un résumé de l'algorithme : on part de la famille libre $\mathcal{L}$ et 
à chaque étape on ajoute un vecteur
de la famille $\mathcal{G}$ à condition qu'il forme 
une famille libre avec les éléments déjà dans la liste.

% Il va donc s'agir d'extraire de la famille $\mathcal{G}$ une base de $E$.
% L'algorithme consiste à créer une liste de polynôme qui résulte, 
% à la fin de l'algorithme, en une base de $E$. 
% \\
% 
% Pour cela on initialise la liste par la famille 
% donnée (l'ensemble vide dans notre cas) et on parcourir les 
% polynômes de $\mathcal{G}$  les uns après les autres. \\
% 
% Un polynôme de $\mathcal{G}$ sera ajouté à la liste  si et 
% seulement si il forme avec les éléments déjà dans la liste une famille libre. \\
% 
% Il sera exclu de la liste si et seulement si il est 
% combinaison linéaire des éléments qui se trouve déjà dans la liste. \\
% 
% Lorsqu'on a parcouru tous les éléments de $\mathcal{G}$, 
% la liste obtenue est une base de $E$. Mais il est possible 
% de s'arr\^eter avant : pour cela on teste à chaque étape si 
% la liste obtenue est une famille génératrice de $E$. 
% Si c'est la cas, on s'arr\^ete, et comme la liste est une 
% famille libre par construction, c'est une base de $E$.
% 
% Voyons ce que cela donne étape par étape sur notre exemple :
  
%%%%%%%%%%%%%%%%%%%%%%%%%%%%%%%%%%%%%%%%%%%%%%%%%%%%%%%%%%%
\diapo
C'est le moment de vous entraîner !

\end{document}
