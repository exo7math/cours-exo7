
%%%%%%%%%%%%%%%%%% PREAMBULE %%%%%%%%%%%%%%%%%%

\documentclass[aspectratio=169,utf8]{beamer}
%\documentclass[aspectratio=169,handout]{beamer}

\usetheme{Boadilla}
%\usecolortheme{seahorse}
\usecolortheme[RGB={245,66,24}]{structure}
\useoutertheme{infolines}

% packages
\usepackage{amsfonts,amsmath,amssymb,amsthm}
\usepackage[utf8]{inputenc}
\usepackage[T1]{fontenc}
\usepackage{lmodern}

\usepackage[francais]{babel}
\usepackage{fancybox}
\usepackage{graphicx}

\usepackage{float}
\usepackage{xfrac}

%\usepackage[usenames, x11names]{xcolor}
\usepackage{tikz}
\usepackage{pgfplots}
\usepackage{datetime}



%-----  Package unités -----
\usepackage{siunitx}
\sisetup{locale = FR,detect-all,per-mode = symbol}

%\usepackage{mathptmx}
%\usepackage{fouriernc}
%\usepackage{newcent}
%\usepackage[mathcal,mathbf]{euler}

%\usepackage{palatino}
%\usepackage{newcent}
% \usepackage[mathcal,mathbf]{euler}



% \usepackage{hyperref}
% \hypersetup{colorlinks=true, linkcolor=blue, urlcolor=blue,
% pdftitle={Exo7 - Exercices de mathématiques}, pdfauthor={Exo7}}


%section
% \usepackage{sectsty}
% \allsectionsfont{\bf}
%\sectionfont{\color{Tomato3}\upshape\selectfont}
%\subsectionfont{\color{Tomato4}\upshape\selectfont}

%----- Ensembles : entiers, reels, complexes -----
\newcommand{\Nn}{\mathbb{N}} \newcommand{\N}{\mathbb{N}}
\newcommand{\Zz}{\mathbb{Z}} \newcommand{\Z}{\mathbb{Z}}
\newcommand{\Qq}{\mathbb{Q}} \newcommand{\Q}{\mathbb{Q}}
\newcommand{\Rr}{\mathbb{R}} \newcommand{\R}{\mathbb{R}}
\newcommand{\Cc}{\mathbb{C}} 
\newcommand{\Kk}{\mathbb{K}} \newcommand{\K}{\mathbb{K}}

%----- Modifications de symboles -----
\renewcommand{\epsilon}{\varepsilon}
\renewcommand{\Re}{\mathop{\text{Re}}\nolimits}
\renewcommand{\Im}{\mathop{\text{Im}}\nolimits}
%\newcommand{\llbracket}{\left[\kern-0.15em\left[}
%\newcommand{\rrbracket}{\right]\kern-0.15em\right]}

\renewcommand{\ge}{\geqslant}
\renewcommand{\geq}{\geqslant}
\renewcommand{\le}{\leqslant}
\renewcommand{\leq}{\leqslant}
\renewcommand{\epsilon}{\varepsilon}

%----- Fonctions usuelles -----
\newcommand{\ch}{\mathop{\text{ch}}\nolimits}
\newcommand{\sh}{\mathop{\text{sh}}\nolimits}
\renewcommand{\tanh}{\mathop{\text{th}}\nolimits}
\newcommand{\cotan}{\mathop{\text{cotan}}\nolimits}
\newcommand{\Arcsin}{\mathop{\text{arcsin}}\nolimits}
\newcommand{\Arccos}{\mathop{\text{arccos}}\nolimits}
\newcommand{\Arctan}{\mathop{\text{arctan}}\nolimits}
\newcommand{\Argsh}{\mathop{\text{argsh}}\nolimits}
\newcommand{\Argch}{\mathop{\text{argch}}\nolimits}
\newcommand{\Argth}{\mathop{\text{argth}}\nolimits}
\newcommand{\pgcd}{\mathop{\text{pgcd}}\nolimits} 


%----- Commandes divers ------
\newcommand{\ii}{\mathrm{i}}
\newcommand{\dd}{\text{d}}
\newcommand{\id}{\mathop{\text{id}}\nolimits}
\newcommand{\Ker}{\mathop{\text{Ker}}\nolimits}
\newcommand{\Card}{\mathop{\text{Card}}\nolimits}
\newcommand{\Vect}{\mathop{\text{Vect}}\nolimits}
\newcommand{\Mat}{\mathop{\text{Mat}}\nolimits}
\newcommand{\rg}{\mathop{\text{rg}}\nolimits}
\newcommand{\tr}{\mathop{\text{tr}}\nolimits}


%----- Structure des exercices ------

\newtheoremstyle{styleexo}% name
{2ex}% Space above
{3ex}% Space below
{}% Body font
{}% Indent amount 1
{\bfseries} % Theorem head font
{}% Punctuation after theorem head
{\newline}% Space after theorem head 2
{}% Theorem head spec (can be left empty, meaning ‘normal’)

%\theoremstyle{styleexo}
\newtheorem{exo}{Exercice}
\newtheorem{ind}{Indications}
\newtheorem{cor}{Correction}


\newcommand{\exercice}[1]{} \newcommand{\finexercice}{}
%\newcommand{\exercice}[1]{{\tiny\texttt{#1}}\vspace{-2ex}} % pour afficher le numero absolu, l'auteur...
\newcommand{\enonce}{\begin{exo}} \newcommand{\finenonce}{\end{exo}}
\newcommand{\indication}{\begin{ind}} \newcommand{\finindication}{\end{ind}}
\newcommand{\correction}{\begin{cor}} \newcommand{\fincorrection}{\end{cor}}

\newcommand{\noindication}{\stepcounter{ind}}
\newcommand{\nocorrection}{\stepcounter{cor}}

\newcommand{\fiche}[1]{} \newcommand{\finfiche}{}
\newcommand{\titre}[1]{\centerline{\large \bf #1}}
\newcommand{\addcommand}[1]{}
\newcommand{\video}[1]{}

% Marge
\newcommand{\mymargin}[1]{\marginpar{{\small #1}}}

\def\noqed{\renewcommand{\qedsymbol}{}}


%----- Presentation ------
\setlength{\parindent}{0cm}

%\newcommand{\ExoSept}{\href{http://exo7.emath.fr}{\textbf{\textsf{Exo7}}}}

\definecolor{myred}{rgb}{0.93,0.26,0}
\definecolor{myorange}{rgb}{0.97,0.58,0}
\definecolor{myyellow}{rgb}{1,0.86,0}

\newcommand{\LogoExoSept}[1]{  % input : echelle
{\usefont{U}{cmss}{bx}{n}
\begin{tikzpicture}[scale=0.1*#1,transform shape]
  \fill[color=myorange] (0,0)--(4,0)--(4,-4)--(0,-4)--cycle;
  \fill[color=myred] (0,0)--(0,3)--(-3,3)--(-3,0)--cycle;
  \fill[color=myyellow] (4,0)--(7,4)--(3,7)--(0,3)--cycle;
  \node[scale=5] at (3.5,3.5) {Exo7};
\end{tikzpicture}}
}


\newcommand{\debutmontitre}{
  \author{} \date{} 
  \thispagestyle{empty}
  \hspace*{-10ex}
  \begin{minipage}{\textwidth}
    \titlepage  
  \vspace*{-2.5cm}
  \begin{center}
    \LogoExoSept{2.5}
  \end{center}
  \end{minipage}

  \vspace*{-0cm}
  
  % Astuce pour que le background ne soit pas discrétisé lors de la conversion pdf -> png
\begin{tikzpicture}
        \fill[opacity=0,green!60!black] (0,0)--++(0,0)--++(0,0)--++(0,0)--cycle; 
\end{tikzpicture}

% toc S'affiche trop tot :
% \tableofcontents[hideallsubsections, pausesections]
}

\newcommand{\finmontitre}{
  \end{frame}
  \setcounter{framenumber}{0}
} % ne marche pas pour une raison obscure

%----- Commandes supplementaires ------

% \usepackage[landscape]{geometry}
% \geometry{top=1cm, bottom=3cm, left=2cm, right=10cm, marginparsep=1cm
% }
% \usepackage[a4paper]{geometry}
% \geometry{top=2cm, bottom=2cm, left=2cm, right=2cm, marginparsep=1cm
% }

%\usepackage{standalone}


% New command Arnaud -- november 2011
\setbeamersize{text margin left=24ex}
% si vous modifier cette valeur il faut aussi
% modifier le decalage du titre pour compenser
% (ex : ici =+10ex, titre =-5ex

\theoremstyle{definition}
%\newtheorem{proposition}{Proposition}
%\newtheorem{exemple}{Exemple}
%\newtheorem{theoreme}{Théorème}
%\newtheorem{lemme}{Lemme}
%\newtheorem{corollaire}{Corollaire}
%\newtheorem*{remarque*}{Remarque}
%\newtheorem*{miniexercice}{Mini-exercices}
%\newtheorem{definition}{Définition}

% Commande tikz
\usetikzlibrary{calc}
\usetikzlibrary{patterns,arrows}
\usetikzlibrary{matrix}
\usetikzlibrary{fadings} 

%definition d'un terme
\newcommand{\defi}[1]{{\color{myorange}\textbf{\emph{#1}}}}
\newcommand{\evidence}[1]{{\color{blue}\textbf{\emph{#1}}}}
\newcommand{\assertion}[1]{\emph{\og#1\fg}}  % pour chapitre logique
%\renewcommand{\contentsname}{Sommaire}
\renewcommand{\contentsname}{}
\setcounter{tocdepth}{2}



%------ Figures ------

\def\myscale{1} % par défaut 
\newcommand{\myfigure}[2]{  % entrée : echelle, fichier figure
\def\myscale{#1}
\begin{center}
\footnotesize
{#2}
\end{center}}


%------ Encadrement ------

\usepackage{fancybox}


\newcommand{\mybox}[1]{
\setlength{\fboxsep}{7pt}
\begin{center}
\shadowbox{#1}
\end{center}}

\newcommand{\myboxinline}[1]{
\setlength{\fboxsep}{5pt}
\raisebox{-10pt}{
\shadowbox{#1}
}
}

%--------------- Commande beamer---------------
\newcommand{\beameronly}[1]{#1} % permet de mettre des pause dans beamer pas dans poly


\setbeamertemplate{navigation symbols}{}
\setbeamertemplate{footline}  % tiré du fichier beamerouterinfolines.sty
{
  \leavevmode%
  \hbox{%
  \begin{beamercolorbox}[wd=.333333\paperwidth,ht=2.25ex,dp=1ex,center]{author in head/foot}%
    % \usebeamerfont{author in head/foot}\insertshortauthor%~~(\insertshortinstitute)
    \usebeamerfont{section in head/foot}{\bf\insertshorttitle}
  \end{beamercolorbox}%
  \begin{beamercolorbox}[wd=.333333\paperwidth,ht=2.25ex,dp=1ex,center]{title in head/foot}%
    \usebeamerfont{section in head/foot}{\bf\insertsectionhead}
  \end{beamercolorbox}%
  \begin{beamercolorbox}[wd=.333333\paperwidth,ht=2.25ex,dp=1ex,right]{date in head/foot}%
    % \usebeamerfont{date in head/foot}\insertshortdate{}\hspace*{2em}
    \insertframenumber{} / \inserttotalframenumber\hspace*{2ex} 
  \end{beamercolorbox}}%
  \vskip0pt%
}


\definecolor{mygrey}{rgb}{0.5,0.5,0.5}
\setlength{\parindent}{0cm}
%\DeclareTextFontCommand{\helvetica}{\fontfamily{phv}\selectfont}

% background beamer
\definecolor{couleurhaut}{rgb}{0.85,0.9,1}  % creme
\definecolor{couleurmilieu}{rgb}{1,1,1}  % vert pale
\definecolor{couleurbas}{rgb}{0.85,0.9,1}  % blanc
\setbeamertemplate{background canvas}[vertical shading]%
[top=couleurhaut,middle=couleurmilieu,midpoint=0.4,bottom=couleurbas] 
%[top=fondtitre!05,bottom=fondtitre!60]



\makeatletter
\setbeamertemplate{theorem begin}
{%
  \begin{\inserttheoremblockenv}
  {%
    \inserttheoremheadfont
    \inserttheoremname
    \inserttheoremnumber
    \ifx\inserttheoremaddition\@empty\else\ (\inserttheoremaddition)\fi%
    \inserttheorempunctuation
  }%
}
\setbeamertemplate{theorem end}{\end{\inserttheoremblockenv}}

\newenvironment{theoreme}[1][]{%
   \setbeamercolor{block title}{fg=structure,bg=structure!40}
   \setbeamercolor{block body}{fg=black,bg=structure!10}
   \begin{block}{{\bf Th\'eor\`eme }#1}
}{%
   \end{block}%
}


\newenvironment{proposition}[1][]{%
   \setbeamercolor{block title}{fg=structure,bg=structure!40}
   \setbeamercolor{block body}{fg=black,bg=structure!10}
   \begin{block}{{\bf Proposition }#1}
}{%
   \end{block}%
}

\newenvironment{corollaire}[1][]{%
   \setbeamercolor{block title}{fg=structure,bg=structure!40}
   \setbeamercolor{block body}{fg=black,bg=structure!10}
   \begin{block}{{\bf Corollaire }#1}
}{%
   \end{block}%
}

\newenvironment{mydefinition}[1][]{%
   \setbeamercolor{block title}{fg=structure,bg=structure!40}
   \setbeamercolor{block body}{fg=black,bg=structure!10}
   \begin{block}{{\bf Définition} #1}
}{%
   \end{block}%
}

\newenvironment{lemme}[0]{%
   \setbeamercolor{block title}{fg=structure,bg=structure!40}
   \setbeamercolor{block body}{fg=black,bg=structure!10}
   \begin{block}{\bf Lemme}
}{%
   \end{block}%
}

\newenvironment{remarque}[1][]{%
   \setbeamercolor{block title}{fg=black,bg=structure!20}
   \setbeamercolor{block body}{fg=black,bg=structure!5}
   \begin{block}{Remarque #1}
}{%
   \end{block}%
}


\newenvironment{exemple}[1][]{%
   \setbeamercolor{block title}{fg=black,bg=structure!20}
   \setbeamercolor{block body}{fg=black,bg=structure!5}
   \begin{block}{{\bf Exemple }#1}
}{%
   \end{block}%
}


\newenvironment{miniexercice}[0]{%
   \setbeamercolor{block title}{fg=structure,bg=structure!20}
   \setbeamercolor{block body}{fg=black,bg=structure!5}
   \begin{block}{Mini-exercices}
}{%
   \end{block}%
}


\newenvironment{tp}[0]{%
   \setbeamercolor{block title}{fg=structure,bg=structure!40}
   \setbeamercolor{block body}{fg=black,bg=structure!10}
   \begin{block}{\bf Travaux pratiques}
}{%
   \end{block}%
}
\newenvironment{exercicecours}[1][]{%
   \setbeamercolor{block title}{fg=structure,bg=structure!40}
   \setbeamercolor{block body}{fg=black,bg=structure!10}
   \begin{block}{{\bf Exercice }#1}
}{%
   \end{block}%
}
\newenvironment{algo}[1][]{%
   \setbeamercolor{block title}{fg=structure,bg=structure!40}
   \setbeamercolor{block body}{fg=black,bg=structure!10}
   \begin{block}{{\bf Algorithme}\hfill{\color{gray}\texttt{#1}}}
}{%
   \end{block}%
}


\setbeamertemplate{proof begin}{
   \setbeamercolor{block title}{fg=black,bg=structure!20}
   \setbeamercolor{block body}{fg=black,bg=structure!5}
   \begin{block}{{\footnotesize Démonstration}}
   \footnotesize
   \smallskip}
\setbeamertemplate{proof end}{%
   \end{block}}
\setbeamertemplate{qed symbol}{\openbox}


\makeatother
\usecolortheme[RGB={150,93,42}]{structure}
   
%%%%%%%%%%%%%%%%%%%%%%%%%%%%%%%%%%%%%%%%%%%%%%%%%%%%%%%%%%%%%
%%%%%%%%%%%%%%%%%%%%%%%%%%%%%%%%%%%%%%%%%%%%%%%%%%%%%%%%%%%%%


\begin{document}


\title{{\bf Dimension finie}}
\subtitle{Dimension d'un espace vectoriel}

\begin{frame}
  
  \debutmontitre

  \pause

{\footnotesize
\hfill
\setbeamercovered{transparent=50}
\begin{minipage}{0.6\textwidth}
  \begin{itemize}
    \item<3-> Définition
    \item<4-> Exemples
    \item<5-> Compléments
    %\item<6-> Preuve (??)
  \end{itemize}
\end{minipage}
}

\end{frame}

\setcounter{framenumber}{0}


%%%%%%%%%%%%%%%%%%%%%%%%%%%%%%%%%%%%%%%%%%%%%%%%%%%%%%%%%%%%%%%%
\section{Définition}

\begin{frame}
\begin{mydefinition}
Un $\Kk$-espace vectoriel $E$ admettant une base ayant un nombre fini
d'éléments est dit de \defi{dimension finie}
\end{mydefinition}
\pause
\begin{theoreme}[de la dimension]
Toutes les bases d'un espace vectoriel $E$ de dimension 
finie ont le même nombre d'éléments 
\end{theoreme}

\pause
\begin{mydefinition}
La \defi{dimension} d'un espace vectoriel de dimension finie $E$, 
notée \defi{$\dim E$}, est par définition le nombre d'éléments d'une base de $E$
\end{mydefinition}


\end{frame}



%%%%%%%%%%%%%%%%%%%%%%%%%%%%%%%%%%%%%%%%%%%%%%%%%%%%%%%%%%%%%%%%
\section{Exemples}

\begin{frame}
\begin{exemple}
\begin{itemize}
\setlength{\itemsep}{7pt}
  \item La base canonique de $\Rr^2$ est 
$\left( 
\left(\begin{smallmatrix} 1\\0 \end{smallmatrix}\right),
\left(\begin{smallmatrix} 0\\1 \end{smallmatrix}\right)
\right)$ : dim$\Rr^2=2$
  
  
  \pause
  \item Les vecteurs $\left(
  \left(\begin{smallmatrix}2\\1\end{smallmatrix}\right),
  \left(\begin{smallmatrix}1\\1\end{smallmatrix}\right) \right)$
  forment aussi une base de $\Rr^2$  
  
  
  \pause
  \item $\dim \Kk^n = n$
  
    \pause 
  \item $\dim \Rr_n[X] = n+1\quad$ car $\quad\Card \{1,X,X^2,\ldots,X^n\} = n+1$
  
\end{itemize}
\end{exemple}
\end{frame}


\begin{frame}
\begin{exemple}[Système homogène]
\[ \left\{
\begin{array}{ccccccccccc}
2x_1 &+ &2x_2 &- &x_3 &&&+ &x_5 & = & 0\\
-x_1 &- &x_2 &+ &2x_3 &- &3x_4 &+ &x_5 & = &0\\
x_1 &+ &x_2 &- &2x_3 &&&- &x_5 & = & 0\\
&&&&x_3 &+ &x_4 &+ &x_5 & = & 0
\end{array} \right.
\]
\pause

$$ x_1 = -s-t\qquad x_2 = s\qquad x_3 = -t\qquad x_4 = 0\qquad x_5 = t\, $$ 

\pause

$$ \left( \begin{smallmatrix} x_1\\x_2\\x_3\\x_4\\x_5\end{smallmatrix}\right) 
=  \left( \begin{smallmatrix} -s-t\\s\\-t\\ 0\\t\end{smallmatrix}\right) 
\pause
=  \left( \begin{smallmatrix} -s\\s\\0\\0\\0\end{smallmatrix}\right) 
+  \left( \begin{smallmatrix} -t\\0\\-t\\0\\t\end{smallmatrix}\right)
\pause
= s  \left( \begin{smallmatrix} -1\\1\\0\\0\\0\end{smallmatrix}\right) 
+ t \left( \begin{smallmatrix} -1\\0\\-1\\0\\1\end{smallmatrix}\right)\, $$
\pause
$$ v_1 = \left( \begin{smallmatrix} -1\\1\\0\\0\\0\end{smallmatrix}\right) 
\quad \text{ et } \quad 
v_2 =  \left( \begin{smallmatrix} -1\\0\\-1\\0\\1\end{smallmatrix}\right)$$ 
\end{exemple}
\end{frame}




%%%%%%%%%%%%%%%%%%%%%%%%%%%%%%%%%%%%%%%%%%%%%%%%%%%%%%%%%%%%%%%%
\section{Compléments}

\begin{frame}
Soit $E$ un espace vectoriel de dimension finie
\begin{lemme}
Soit $\mathcal{L}$ une famille libre et soit
$\mathcal{G}$ une famille génératrice $E$. Alors \\
\centerline{$\Card \mathcal{L} \le \Card \mathcal{G}$}
\end{lemme}

\bigskip
\pause

\begin{proposition}
Soit $E$ un $\Kk$-espace vectoriel de dimension $n$. Alors :
\pause
\begin{enumerate}
  \item Toute famille libre de $E$ a au plus $n$ éléments
\pause  
  \item Toute famille génératrice de $E$ a au moins $n$ éléments
\end{enumerate}
\end{proposition}
\end{frame}

\begin{frame}
Soit $E$ un $\Kk$-espace vectoriel de dimension \evidence{$n$}
\begin{theoreme}
$\mathcal{F}=(v_1,\ldots,v_n)$ une famille de \evidence{$n$} vecteurs de $E$ \\
Il y a équivalence entre :
\begin{itemize}
  \item[(i)] $\mathcal{F}$ est une base de $E$
  
  \item[(ii)] $\mathcal{F}$ est une famille libre de $E$
  
  \item[(iii)] $\mathcal{F}$ est une famille génératrice de $E$
\end{itemize}
\end{theoreme}
\pause
\begin{proof}
\begin{itemize}  
  \item (ii) $\implies$ (i) Soit $\mathcal{F}$ une famille libre à $n$ éléments.
  Il existe une famille $\mathcal{F}'$ telle que
$\mathcal{F} \cup \mathcal{F}'$  base de $E$ (th\'eor\`eme de la base incompl\`ete). Alors
 $\Card \big(\mathcal{F} \cup \mathcal{F}'\big)=n$. Or
$\Card \mathcal{F}=n$. Donc $\Card \mathcal{F}'=0$, ce qui implique que  $\mathcal{F}'=\varnothing$ et donc que 
$\mathcal{F}$ est déjà une base de $E$
  \pause
  \item (iii) $\implies$ (i) Soit $\mathcal{F}$ une famille génératrice \`a $n$ \'el\'ements.
On peut extraire de $\mathcal{F}$ une base $\mathcal{B} \subset \mathcal{F}$ (th\'eor\`eme de la base incompl\`ete).
Alors  $\Card \mathcal{B}=n$ (th\'eor\`eme de la dimension), donc
$n = \Card \mathcal{B} \le \Card \mathcal{F}=n$. Donc $\mathcal{B} = \mathcal{F}$ et 
$\mathcal{F}$ est bien une base \qedhere
\end{itemize}
\end{proof}
\end{frame}


\begin{frame}

\begin{exemple}
\begin{itemize}
\item
$v_1 = \begin{pmatrix}1\\1\\4\end{pmatrix} \qquad
v_2 = \begin{pmatrix}1\\3\\t\end{pmatrix} \qquad
v_3 = \begin{pmatrix}1\\1\\t\end{pmatrix}$

\pause
\item
Pour quelles valeurs de $t\in\Rr$, $(v_1,v_2,v_3)$ est  une base de $\Rr^3$ ?

\pause
  \item $\dim \Rr^3 = 3$ et $\Card \{v_1,v_2,v_3\} = 3$  
  \pause
  \item \`A quelle condition la famille $\{ v_1,v_2,v_3 \}$ est libre ?
  \pause
  \item 
  Soient $\lambda_1,\lambda_2,\lambda_3 \in \Rr$ tels que $\lambda_1 v_1 + \lambda_2 v_2 + \lambda_3 v_3 = 0$
  \pause
  $\Leftrightarrow\left\{ \begin{array}{rcl}
  \lambda_1 + \lambda_2 + \lambda_3 &=& 0 \\
  \lambda_1 + 3 \lambda_2 + \lambda_3 &=& 0 \\
  4\lambda_1 + t\lambda_2 + t \lambda_3 &=& 0
  \end{array}  \right.$
  \pause  $\Leftrightarrow\left\{ \begin{array}{rcl}
  \lambda_1 + \lambda_2 + \lambda_3 &=& 0 \\
  2 \lambda_2 &=& 0 \\
  (t-4)\lambda_2 + (t-4) \lambda_3 &=& 0
  \end{array} \right.$
  \pause
  $
  \Leftrightarrow  \left\{ \begin{array}{rcl}
  \lambda_1 + \lambda_3 &=& 0 \\
  \lambda_2 &=& 0 \\
  (t-4) \lambda_3 &=& 0
  \end{array} \right.$
  
  \pause
  \item $\{ v_1,v_2,v_3 \}$ est une base si et seulement si $t\neq 4$  
\end{itemize}

\end{exemple}


\end{frame}

%%%%%%%%%%%%%%%%%%%%%%%%%%%%%%%%%%%%%%%%%%%%%%%%%%%%%%%%%%%%%%%%%
%\section{Preuve (??)}
%
%\begin{frame}
%
%\pause
%
%\end{frame}
%
%
%\begin{frame}
%
%\end{frame}



%%%%%%%%%%%%%%%%%%%%%%%%%%%%%%%%%%%%%%%%%%%%%%%%%%%%%%%%%%%%%%%%
\section{Mini-exercices}

\begin{frame}
\begin{miniexercice}
Dire si les assertions suivantes sont vraies ou fausses.
Justifier votre réponse par un résultat du cours ou un contre-exemple :
\begin{enumerate}
  \item Une famille de $p \ge n$ vecteurs dans un espace vectoriel 
  de dimension $n$ est génératrice.
  
  \item Une famille de $p > n$ vecteurs dans un espace vectoriel 
  de dimension $n$ est liée.  
  
  \item Une famille de $p < n$ vecteurs dans un espace vectoriel 
  de dimension $n$ est libre.   
  
  \item Une famille génératrice de $p \le n$ vecteurs dans un espace vectoriel 
  de dimension $n$ est libre.
   
  \item Une famille de $p \neq n$ vecteurs dans un espace vectoriel 
  de dimension $n$ n'est pas une base.
  
  \item Toute famille libre à $p$ éléments d'un espace vectoriel de dimension $n$
  se complète par une famille ayant exactement $n-p$ éléments en une base de $E$.
\end{enumerate}
\end{miniexercice}
\end{frame}

\end{document}