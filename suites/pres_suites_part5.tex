
%%%%%%%%%%%%%%%%%% PREAMBULE %%%%%%%%%%%%%%%%%%

\documentclass[aspectratio=169,utf8]{beamer}
%\documentclass[aspectratio=169,handout]{beamer}

\usetheme{Boadilla}
%\usecolortheme{seahorse}
\usecolortheme[RGB={245,66,24}]{structure}
\useoutertheme{infolines}

% packages
\usepackage{amsfonts,amsmath,amssymb,amsthm}
\usepackage[utf8]{inputenc}
\usepackage[T1]{fontenc}
\usepackage{lmodern}

\usepackage[francais]{babel}
\usepackage{fancybox}
\usepackage{graphicx}

\usepackage{float}
\usepackage{xfrac}

%\usepackage[usenames, x11names]{xcolor}
\usepackage{tikz}
\usepackage{pgfplots}
\usepackage{datetime}



%-----  Package unités -----
\usepackage{siunitx}
\sisetup{locale = FR,detect-all,per-mode = symbol}

%\usepackage{mathptmx}
%\usepackage{fouriernc}
%\usepackage{newcent}
%\usepackage[mathcal,mathbf]{euler}

%\usepackage{palatino}
%\usepackage{newcent}
% \usepackage[mathcal,mathbf]{euler}



% \usepackage{hyperref}
% \hypersetup{colorlinks=true, linkcolor=blue, urlcolor=blue,
% pdftitle={Exo7 - Exercices de mathématiques}, pdfauthor={Exo7}}


%section
% \usepackage{sectsty}
% \allsectionsfont{\bf}
%\sectionfont{\color{Tomato3}\upshape\selectfont}
%\subsectionfont{\color{Tomato4}\upshape\selectfont}

%----- Ensembles : entiers, reels, complexes -----
\newcommand{\Nn}{\mathbb{N}} \newcommand{\N}{\mathbb{N}}
\newcommand{\Zz}{\mathbb{Z}} \newcommand{\Z}{\mathbb{Z}}
\newcommand{\Qq}{\mathbb{Q}} \newcommand{\Q}{\mathbb{Q}}
\newcommand{\Rr}{\mathbb{R}} \newcommand{\R}{\mathbb{R}}
\newcommand{\Cc}{\mathbb{C}} 
\newcommand{\Kk}{\mathbb{K}} \newcommand{\K}{\mathbb{K}}

%----- Modifications de symboles -----
\renewcommand{\epsilon}{\varepsilon}
\renewcommand{\Re}{\mathop{\text{Re}}\nolimits}
\renewcommand{\Im}{\mathop{\text{Im}}\nolimits}
%\newcommand{\llbracket}{\left[\kern-0.15em\left[}
%\newcommand{\rrbracket}{\right]\kern-0.15em\right]}

\renewcommand{\ge}{\geqslant}
\renewcommand{\geq}{\geqslant}
\renewcommand{\le}{\leqslant}
\renewcommand{\leq}{\leqslant}
\renewcommand{\epsilon}{\varepsilon}

%----- Fonctions usuelles -----
\newcommand{\ch}{\mathop{\text{ch}}\nolimits}
\newcommand{\sh}{\mathop{\text{sh}}\nolimits}
\renewcommand{\tanh}{\mathop{\text{th}}\nolimits}
\newcommand{\cotan}{\mathop{\text{cotan}}\nolimits}
\newcommand{\Arcsin}{\mathop{\text{arcsin}}\nolimits}
\newcommand{\Arccos}{\mathop{\text{arccos}}\nolimits}
\newcommand{\Arctan}{\mathop{\text{arctan}}\nolimits}
\newcommand{\Argsh}{\mathop{\text{argsh}}\nolimits}
\newcommand{\Argch}{\mathop{\text{argch}}\nolimits}
\newcommand{\Argth}{\mathop{\text{argth}}\nolimits}
\newcommand{\pgcd}{\mathop{\text{pgcd}}\nolimits} 


%----- Commandes divers ------
\newcommand{\ii}{\mathrm{i}}
\newcommand{\dd}{\text{d}}
\newcommand{\id}{\mathop{\text{id}}\nolimits}
\newcommand{\Ker}{\mathop{\text{Ker}}\nolimits}
\newcommand{\Card}{\mathop{\text{Card}}\nolimits}
\newcommand{\Vect}{\mathop{\text{Vect}}\nolimits}
\newcommand{\Mat}{\mathop{\text{Mat}}\nolimits}
\newcommand{\rg}{\mathop{\text{rg}}\nolimits}
\newcommand{\tr}{\mathop{\text{tr}}\nolimits}


%----- Structure des exercices ------

\newtheoremstyle{styleexo}% name
{2ex}% Space above
{3ex}% Space below
{}% Body font
{}% Indent amount 1
{\bfseries} % Theorem head font
{}% Punctuation after theorem head
{\newline}% Space after theorem head 2
{}% Theorem head spec (can be left empty, meaning ‘normal’)

%\theoremstyle{styleexo}
\newtheorem{exo}{Exercice}
\newtheorem{ind}{Indications}
\newtheorem{cor}{Correction}


\newcommand{\exercice}[1]{} \newcommand{\finexercice}{}
%\newcommand{\exercice}[1]{{\tiny\texttt{#1}}\vspace{-2ex}} % pour afficher le numero absolu, l'auteur...
\newcommand{\enonce}{\begin{exo}} \newcommand{\finenonce}{\end{exo}}
\newcommand{\indication}{\begin{ind}} \newcommand{\finindication}{\end{ind}}
\newcommand{\correction}{\begin{cor}} \newcommand{\fincorrection}{\end{cor}}

\newcommand{\noindication}{\stepcounter{ind}}
\newcommand{\nocorrection}{\stepcounter{cor}}

\newcommand{\fiche}[1]{} \newcommand{\finfiche}{}
\newcommand{\titre}[1]{\centerline{\large \bf #1}}
\newcommand{\addcommand}[1]{}
\newcommand{\video}[1]{}

% Marge
\newcommand{\mymargin}[1]{\marginpar{{\small #1}}}

\def\noqed{\renewcommand{\qedsymbol}{}}


%----- Presentation ------
\setlength{\parindent}{0cm}

%\newcommand{\ExoSept}{\href{http://exo7.emath.fr}{\textbf{\textsf{Exo7}}}}

\definecolor{myred}{rgb}{0.93,0.26,0}
\definecolor{myorange}{rgb}{0.97,0.58,0}
\definecolor{myyellow}{rgb}{1,0.86,0}

\newcommand{\LogoExoSept}[1]{  % input : echelle
{\usefont{U}{cmss}{bx}{n}
\begin{tikzpicture}[scale=0.1*#1,transform shape]
  \fill[color=myorange] (0,0)--(4,0)--(4,-4)--(0,-4)--cycle;
  \fill[color=myred] (0,0)--(0,3)--(-3,3)--(-3,0)--cycle;
  \fill[color=myyellow] (4,0)--(7,4)--(3,7)--(0,3)--cycle;
  \node[scale=5] at (3.5,3.5) {Exo7};
\end{tikzpicture}}
}


\newcommand{\debutmontitre}{
  \author{} \date{} 
  \thispagestyle{empty}
  \hspace*{-10ex}
  \begin{minipage}{\textwidth}
    \titlepage  
  \vspace*{-2.5cm}
  \begin{center}
    \LogoExoSept{2.5}
  \end{center}
  \end{minipage}

  \vspace*{-0cm}
  
  % Astuce pour que le background ne soit pas discrétisé lors de la conversion pdf -> png
\begin{tikzpicture}
        \fill[opacity=0,green!60!black] (0,0)--++(0,0)--++(0,0)--++(0,0)--cycle; 
\end{tikzpicture}

% toc S'affiche trop tot :
% \tableofcontents[hideallsubsections, pausesections]
}

\newcommand{\finmontitre}{
  \end{frame}
  \setcounter{framenumber}{0}
} % ne marche pas pour une raison obscure

%----- Commandes supplementaires ------

% \usepackage[landscape]{geometry}
% \geometry{top=1cm, bottom=3cm, left=2cm, right=10cm, marginparsep=1cm
% }
% \usepackage[a4paper]{geometry}
% \geometry{top=2cm, bottom=2cm, left=2cm, right=2cm, marginparsep=1cm
% }

%\usepackage{standalone}


% New command Arnaud -- november 2011
\setbeamersize{text margin left=24ex}
% si vous modifier cette valeur il faut aussi
% modifier le decalage du titre pour compenser
% (ex : ici =+10ex, titre =-5ex

\theoremstyle{definition}
%\newtheorem{proposition}{Proposition}
%\newtheorem{exemple}{Exemple}
%\newtheorem{theoreme}{Théorème}
%\newtheorem{lemme}{Lemme}
%\newtheorem{corollaire}{Corollaire}
%\newtheorem*{remarque*}{Remarque}
%\newtheorem*{miniexercice}{Mini-exercices}
%\newtheorem{definition}{Définition}

% Commande tikz
\usetikzlibrary{calc}
\usetikzlibrary{patterns,arrows}
\usetikzlibrary{matrix}
\usetikzlibrary{fadings} 

%definition d'un terme
\newcommand{\defi}[1]{{\color{myorange}\textbf{\emph{#1}}}}
\newcommand{\evidence}[1]{{\color{blue}\textbf{\emph{#1}}}}
\newcommand{\assertion}[1]{\emph{\og#1\fg}}  % pour chapitre logique
%\renewcommand{\contentsname}{Sommaire}
\renewcommand{\contentsname}{}
\setcounter{tocdepth}{2}



%------ Figures ------

\def\myscale{1} % par défaut 
\newcommand{\myfigure}[2]{  % entrée : echelle, fichier figure
\def\myscale{#1}
\begin{center}
\footnotesize
{#2}
\end{center}}


%------ Encadrement ------

\usepackage{fancybox}


\newcommand{\mybox}[1]{
\setlength{\fboxsep}{7pt}
\begin{center}
\shadowbox{#1}
\end{center}}

\newcommand{\myboxinline}[1]{
\setlength{\fboxsep}{5pt}
\raisebox{-10pt}{
\shadowbox{#1}
}
}

%--------------- Commande beamer---------------
\newcommand{\beameronly}[1]{#1} % permet de mettre des pause dans beamer pas dans poly


\setbeamertemplate{navigation symbols}{}
\setbeamertemplate{footline}  % tiré du fichier beamerouterinfolines.sty
{
  \leavevmode%
  \hbox{%
  \begin{beamercolorbox}[wd=.333333\paperwidth,ht=2.25ex,dp=1ex,center]{author in head/foot}%
    % \usebeamerfont{author in head/foot}\insertshortauthor%~~(\insertshortinstitute)
    \usebeamerfont{section in head/foot}{\bf\insertshorttitle}
  \end{beamercolorbox}%
  \begin{beamercolorbox}[wd=.333333\paperwidth,ht=2.25ex,dp=1ex,center]{title in head/foot}%
    \usebeamerfont{section in head/foot}{\bf\insertsectionhead}
  \end{beamercolorbox}%
  \begin{beamercolorbox}[wd=.333333\paperwidth,ht=2.25ex,dp=1ex,right]{date in head/foot}%
    % \usebeamerfont{date in head/foot}\insertshortdate{}\hspace*{2em}
    \insertframenumber{} / \inserttotalframenumber\hspace*{2ex} 
  \end{beamercolorbox}}%
  \vskip0pt%
}


\definecolor{mygrey}{rgb}{0.5,0.5,0.5}
\setlength{\parindent}{0cm}
%\DeclareTextFontCommand{\helvetica}{\fontfamily{phv}\selectfont}

% background beamer
\definecolor{couleurhaut}{rgb}{0.85,0.9,1}  % creme
\definecolor{couleurmilieu}{rgb}{1,1,1}  % vert pale
\definecolor{couleurbas}{rgb}{0.85,0.9,1}  % blanc
\setbeamertemplate{background canvas}[vertical shading]%
[top=couleurhaut,middle=couleurmilieu,midpoint=0.4,bottom=couleurbas] 
%[top=fondtitre!05,bottom=fondtitre!60]



\makeatletter
\setbeamertemplate{theorem begin}
{%
  \begin{\inserttheoremblockenv}
  {%
    \inserttheoremheadfont
    \inserttheoremname
    \inserttheoremnumber
    \ifx\inserttheoremaddition\@empty\else\ (\inserttheoremaddition)\fi%
    \inserttheorempunctuation
  }%
}
\setbeamertemplate{theorem end}{\end{\inserttheoremblockenv}}

\newenvironment{theoreme}[1][]{%
   \setbeamercolor{block title}{fg=structure,bg=structure!40}
   \setbeamercolor{block body}{fg=black,bg=structure!10}
   \begin{block}{{\bf Th\'eor\`eme }#1}
}{%
   \end{block}%
}


\newenvironment{proposition}[1][]{%
   \setbeamercolor{block title}{fg=structure,bg=structure!40}
   \setbeamercolor{block body}{fg=black,bg=structure!10}
   \begin{block}{{\bf Proposition }#1}
}{%
   \end{block}%
}

\newenvironment{corollaire}[1][]{%
   \setbeamercolor{block title}{fg=structure,bg=structure!40}
   \setbeamercolor{block body}{fg=black,bg=structure!10}
   \begin{block}{{\bf Corollaire }#1}
}{%
   \end{block}%
}

\newenvironment{mydefinition}[1][]{%
   \setbeamercolor{block title}{fg=structure,bg=structure!40}
   \setbeamercolor{block body}{fg=black,bg=structure!10}
   \begin{block}{{\bf Définition} #1}
}{%
   \end{block}%
}

\newenvironment{lemme}[0]{%
   \setbeamercolor{block title}{fg=structure,bg=structure!40}
   \setbeamercolor{block body}{fg=black,bg=structure!10}
   \begin{block}{\bf Lemme}
}{%
   \end{block}%
}

\newenvironment{remarque}[1][]{%
   \setbeamercolor{block title}{fg=black,bg=structure!20}
   \setbeamercolor{block body}{fg=black,bg=structure!5}
   \begin{block}{Remarque #1}
}{%
   \end{block}%
}


\newenvironment{exemple}[1][]{%
   \setbeamercolor{block title}{fg=black,bg=structure!20}
   \setbeamercolor{block body}{fg=black,bg=structure!5}
   \begin{block}{{\bf Exemple }#1}
}{%
   \end{block}%
}


\newenvironment{miniexercice}[0]{%
   \setbeamercolor{block title}{fg=structure,bg=structure!20}
   \setbeamercolor{block body}{fg=black,bg=structure!5}
   \begin{block}{Mini-exercices}
}{%
   \end{block}%
}


\newenvironment{tp}[0]{%
   \setbeamercolor{block title}{fg=structure,bg=structure!40}
   \setbeamercolor{block body}{fg=black,bg=structure!10}
   \begin{block}{\bf Travaux pratiques}
}{%
   \end{block}%
}
\newenvironment{exercicecours}[1][]{%
   \setbeamercolor{block title}{fg=structure,bg=structure!40}
   \setbeamercolor{block body}{fg=black,bg=structure!10}
   \begin{block}{{\bf Exercice }#1}
}{%
   \end{block}%
}
\newenvironment{algo}[1][]{%
   \setbeamercolor{block title}{fg=structure,bg=structure!40}
   \setbeamercolor{block body}{fg=black,bg=structure!10}
   \begin{block}{{\bf Algorithme}\hfill{\color{gray}\texttt{#1}}}
}{%
   \end{block}%
}


\setbeamertemplate{proof begin}{
   \setbeamercolor{block title}{fg=black,bg=structure!20}
   \setbeamercolor{block body}{fg=black,bg=structure!5}
   \begin{block}{{\footnotesize Démonstration}}
   \footnotesize
   \smallskip}
\setbeamertemplate{proof end}{%
   \end{block}}
\setbeamertemplate{qed symbol}{\openbox}


\makeatother
\usecolortheme[RGB={34,139,34}]{structure}

%%%%%%%%%%%%%%%%%%%%%%%%%%%%%%%%%%%%%%%%%%%%%%%%%%%%%%%%%%%%%
%%%%%%%%%%%%%%%%%%%%%%%%%%%%%%%%%%%%%%%%%%%%%%%%%%%%%%%%%%%%%

\begin{document}


\title{{\bf Suites}}
\subtitle{Suites récurrentes}

\begin{frame}
  
  \debutmontitre

  \pause

{\footnotesize
\hfill
\setbeamercovered{transparent=50}
\begin{minipage}{0.6\textwidth}
  \begin{itemize}
    \item<3-> Suite récurrente définie par une fonction
    \item<4-> Cas d'une fonction croissante
    \item<5-> Cas d'une fonction décroissante
  \end{itemize}
\end{minipage}
}



\end{frame}

\setcounter{framenumber}{0}


%%%%%%%%%%%%%%%%%%%%%%%%%%%%%%%%%%%%%%%%%%%%%%%%%%%%%%%%%%%%%%%%
\section{Suite récurrente définie par une fonction}

\begin{frame}
Une \defi{suite récurrente} est définie par 
\pause
\begin{itemize}
  \item une fonction $f : \Rr \to \Rr$ 
\pause
  \item un terme initial $u_0 \in \Rr$
\pause
  \item une relation de récurrence $u_{n+1}=f(u_n)$ pour $n \ge 0$
\end{itemize}
\pause
\medskip
$$u_0 \quad\pause u_1 = f(u_0) \quad\pause u_2 = f(u_1) =f(f(u_0)) \quad\pause u_3 = f(u_2) =f(f(f(u_0)))\ldots$$
\pause
\vspace*{-2ex}
\mybox{$u_0 \in \Rr \quad \text{ et } \quad u_{n+1} = f(u_n) \ \ \text{ pour } n \ge 0$}
\pause
\begin{exemple}
\begin{itemize}
  \item $f(x)=1+\sqrt{x}$
\pause  
  \item $u_0= 2$
\pause  
  \item $u_{n+1}=f(u_n)$ c'est-à-dire $u_{n+1}=1+\sqrt{u_n}$, pour $n\ge0$
\end{itemize}
\pause
{\footnotesize
$$2 \quad\pause 1+\sqrt{2} \quad\pause 1+\sqrt{1+\sqrt{2}}  \quad\pause 1+\sqrt{1+\sqrt{1+\sqrt{2}}}  \quad\pause 1+\sqrt{1+\sqrt{1+\sqrt{1+\sqrt{2}}}}\ldots$$
}
\end{exemple}

\end{frame}


\begin{frame}

\begin{proposition}
\label{prop:fll}
Si $f$ est une fonction continue
et si la suite récurrente $(u_n)$ converge vers $\ell$
alors $\ell$ est une solution de l'équation :
\myboxinline{$f(\ell)=\ell$}  
\end{proposition}

\pause

Une valeur $\ell$, vérifiant $f(\ell)=\ell$ est un \defi{point fixe} de $f$

\myfigure{1}{
\tikzinput{fig_suites15}
}


\end{frame}



\begin{frame}
\begin{proposition}
\label{prop:fll}
Si $f$ est une fonction continue et la suite récurrente $(u_n)$ converge vers $\ell$, alors
$\ell$ est une solution de l'équation :
\myboxinline{$f(\ell)=\ell$}  
\end{proposition}


\begin{proof}
\pause
\begin{itemize}
  \item Lorsque $n\to +\infty$, $u_n\to \ell$ et donc aussi $u_{n+1} \to \ell$
\pause  
  \item Comme $u_n\to \ell$ et que $f$ est continue alors la suite $(f(u_n)) \to f(\ell)$
\pause  
  \item La relation $u_{n+1} = f(u_n)$ devient à la limite : $\ell=f(\ell)$
\end{itemize}

\end{proof}
\end{frame}




%%%%%%%%%%%%%%%%%%%%%%%%%%%%%%%%%%%%%%%%%%%%%%%%%%%%%%%%%%%%%%%%
\section{Cas d'une fonction croissante}


\begin{frame}


\begin{proposition}
\label{prop:fcroissante}
Si $f : [a,b] \to [a,b]$ une fonction continue et \evidence{croissante}, 
alors quelque soit $u_0 \in [a,b]$, la suite récurrente $(u_n)$ est
monotone et converge vers $\ell \in [a,b]$ vérifiant \myboxinline{$f(\ell)=\ell$} 
\end{proposition}

\pause
\medskip


\begin{minipage}{0.6\textwidth}
\begin{itemize}
  \item $f([a,b]) \subset [a,b]$
  \pause
\uncover<4->{  \item Si $u_1\ge u_0$ on a $u_2 = f(u_1)\ge f(u_0)=u_1$, on en déduit $u_3 \ge u_2$,...  $(u_n)$ est croissante}
\uncover<5->{  \item Si $u_1 \le u_0$ alors $(u_n)$ est décroissante}
\uncover<6->{  \item Preuve : si $u_1\ge u_0$ alors la suite $(u_n)$ est croissante, elle est majorée par 
$b$, donc elle converge vers un réel $\ell$}
\uncover<7->{  \item $f(\ell)=\ell$}
\end{itemize}  
\end{minipage}
\begin{minipage}{0.3\textwidth}
\uncover<3->{
\myfigure{0.7}{
\tikzinput{fig_suites16}
}
}
\end{minipage}


\end{frame}



\begin{frame}
\begin{exemple}
Soit $f : \Rr \to \Rr$ définie par $f(x)=\frac14(x^2-1)(x-2)+x$ et $u_0 \in [0,2]$

\'Etude de la suite $(u_n)$ définie : $u_{n+1}=f(u_n)$ (pour tout $n\ge0$)

\pause

\begin{enumerate}

  \item \textbf{\'Etude de $f$}
\pause  
  \begin{itemize}[<+->]
    \item $f$ est continue
    \item $f$ est dérivable et $f'(x)>0$
    \item $f$ est strictement croissante
    \item Comme $f(0)=\frac12$ et $f(2)=2$ alors $f([0,2]) \subset [0,2]$
  \end{itemize}
\end{enumerate}


\end{exemple}
\end{frame}


\begin{frame}
\myfigure{1}{
\tikzinput{fig_suites17}
}  
\end{frame}



\begin{frame}
\begin{exemple}
\begin{enumerate}
  \setcounter{enumi}{1}
  \item \textbf{Calcul des points fixes}
  \begin{itemize}
\pause
    \item Cherchons les $x$ qui vérifient $(f(x)=x)$
\pause    
    \item $f(x)-x=\frac14 (x^2-1)(x-2) \qquad \color{blue}{(\star)}$
\pause    
    \item Donc les points fixes sont les $\{-1,1,2\}$
\pause    
    \item La limite de $(u_n)$ est donc à chercher parmi ces $3$ valeurs
  \end{itemize}

\bigskip
\pause
   \item \textbf{Premier cas :} $u_0=1$ ou $u_0=2$
\pause  

  Alors $u_1=f(u_0)=u_0$ et la suite $(u_n)$ est constante
\end{enumerate}


\end{exemple}
\end{frame}




\begin{frame}
\begin{exemple}
\begin{enumerate}
  \setcounter{enumi}{3}
  \item \textbf{Deuxième cas :} $0 \le u_0 <1$
  \pause
  \begin{itemize}[<+->]
    \item $f([0,1])\subset [0,1]$, on considère la restriction $f : [0,1] \to [0,1]$
    
    \item De plus $\color{blue}{(\star)}$ implique sur $[0,1]$, $f(x)-x\ge0$ 
    
    \item Pour $u_0 \in [0,1[$, $u_1 = f(u_0) \ge u_0$. Comme $f$ est croissante, $(u_n)$ est croissante
    
    \item $(u_n)$ est croissante et majorée par $1$, donc elle converge. Notons $\ell$ sa limite
    
    \item D'une part $\ell$ est un point fixe de $f$ : $f(\ell)=\ell$. Donc $\ell \in \{-1,1,2\}$
    
    \item D'autre part $(u_n)$ étant croissante avec $u_0\ge0$ et majorée par $1$, donc $\ell \in [0,1]$
    
    \item Conclusion : si $0\le u_0<1$ alors $(u_n)$ converge vers $\ell = 1$
  \end{itemize}
  
\medskip
\pause

  \item \textbf{Troisième cas :} $1 < u_0 < 2$
  
   \pause 
  \begin{itemize}[<+->]
      \item $f : [1,2] \to [1,2]$
      \item $(u_n)$ est décroissante et minorée par $1$
      \item $(u_n)$ converge vers $\ell=1$
  \end{itemize}

\end{enumerate}

\end{exemple}
\end{frame}


  

 

%%%%%%%%%%%%%%%%%%%%%%%%%%%%%%%%%%%%%%%%%%%%%%%%%%%%%%%%%%%%%%%%
\section{Cas d'une fonction décroissante}


\begin{frame}
\begin{proposition}
Soit $f : [a,b] \to [a,b]$ une fonction continue et \evidence{décroissante}.
Soit $u_0 \in [a,b]$ et la suite récurrente $(u_n)$ définie par $u_{n+1}=f(u_n)$. Alors :
\begin{itemize}
  \item $(u_{2n})$ converge vers une limite $\ell$  vérifiant $f\circ f(\ell)=\ell$
  \item $(u_{2n+1})$ converge vers une limite $\ell'$ vérifiant $f\circ f(\ell')=\ell'$
\end{itemize}
\end{proposition}

\end{frame}


\begin{frame}
\begin{exemple}
$$f(x)=1+\frac1x \qquad u_0 >0 \qquad u_{n+1} = f(u_n)= 1 + \frac{1}{u_n}$$

\pause

\begin{enumerate}
  \item \textbf{\'Etude de $f$} \quad  $f : ]0,+\infty[\to ]0,+\infty[$  
  continue et strictement décroissante

\pause

  \item \textbf{Points fixes de $f\circ f$} 
{\small  $$f\circ f(x)= f\big( f(x)\big)= f\big(1+\frac1x\big)= 1+ \frac{1}{1+\frac1x}= \frac{2x+1}{x+1}$$}
\pause
{\footnotesize    $$f\circ f(x)=x \ \Leftrightarrow \  \frac{2x+1}{x+1} = x  \ \Leftrightarrow \ x^2-x-1=0  \ \Leftrightarrow \  x \in 
\left\{\frac{1-\sqrt{5}}{2},\frac{1+\sqrt{5}}{2}\right\}$$}  

\pause
 $$\ell=\frac{1+\sqrt{5}}{2}$$

\end{enumerate}

\end{exemple}
\end{frame}
  
  

\begin{frame}

 \myfigure{1}{
\tikzinput{fig_suites18}
}   


\end{frame}

\begin{frame}
\begin{exemple}
$$f(x)=1+\tfrac1x \qquad u_0 >0 \qquad u_{n+1} = f(u_n)= 1 + \tfrac{1}{u_n}$$
\vspace*{-3.5ex}
\begin{enumerate}
 \setcounter{enumi}{2}
  \item \textbf{Premier cas :} $0 < u_0 \le \ell = \frac{1+\sqrt{5}}{2}$
\pause  
  \begin{itemize}[<+->]
    \item $u_1 = f(u_0) \ge f(\ell)=\ell$ ; et par une étude de $f\circ f(x)-x$, 
  on obtient que  :  $u_2 = f\circ f(u_0) \ge u_0$ 
  
    \item Comme $u_2 \ge u_0$ et $f\circ f$ est croissante, la suite $(u_{2n})$ est croissante
    
    \item $u_1  \ge f\circ f(u_1)=u_3$, la suite $(u_{2n+1})$ est décroissante
    
    \item Comme  $u_0 \le u_1$, en itérant $f\circ f$, on obtient $u_{2n} \le u_{2n+1}$
    
    \item $u_0 \le u_2 \le \cdots \le u_{2n} \le \cdots \le u_{2n+1} \le \cdots \le u_3 \le u_1$
    
     \item $(u_{2n})$ est croissante et majorée par $u_1$
     
     \item Sa limite est l'unique point fixe $\ell$ de $f\circ f$
     
     \item $(u_{2n+1})$ est décroissante et minorée par $u_0$, donc converge vers $\ell$
     
     \item La suite $(u_{n})$ converge vers $\ell = \frac{1+\sqrt{5}}{2}$
  \end{itemize}

 
 \pause
 
  \item \textbf{Deuxième cas :} $u_0 \ge \ell = \frac{1+\sqrt{5}}{2}$

  \pause
  
  $(u_{2n})$ est décroissante, $(u_{2n+1})$ est croissante, toutes deux convergent vers $\frac{1+\sqrt{5}}{2}$
    
\end{enumerate}  
\end{exemple}
\end{frame}






%%%%%%%%%%%%%%%%%%%%%%%%%%%%%%%%%%%%%%%%%%%%%%%%%%%%%%%%%%%%%%%%
\section{Mini-exercices}

\begin{frame}

\begin{miniexercice}
\begin{enumerate}

  \item Soit $f(x)=\frac19x^3+1$, $u_0=0$ et pour $n\ge0$ :
  $u_{n+1}=f(u_n)$. \'Etudier en détails la suite $(u_n)$ :
(a) montrer que $u_n\ge0$ ;
(b) étudier et tracer le graphe de $g$ ; 
(c) tracer les premiers termes de $(u_n)$ ;
(d) montrer que $(u_n)$ est croissante ; 
(e) étudier la fonction $g(x)=f(x)-x$ ;
(f) montrer que $f$ admet deux points fixes sur $\Rr_+$, $0 < \ell < \ell'$ ;
(g) montrer que $f([0,\ell]) \subset [0,\ell]$ ;
(h) en déduire que $(u_n)$ converge vers $\ell$.


  \item Soit $f(x)=1+\sqrt{x}$, $u_0= 2$ et pour $n\ge0$ :
  $u_{n+1}=f(u_n)$. \'Etudier en détail la suite $(u_n)$.


  \item Soit $(u_n)_{n\in\Nn}$ la suite définie par :
$u_0 \in[0,1]$ et $u_{n+1} = u_n - u_n^2$. \'Etudier en détail la suite $(u_n)$.

  \item \'Etudier la suite définie par $u_0=4$ et $u_{n+1}=\frac{4}{u_n+2}$.
  
\end{enumerate}
\end{miniexercice}

\end{frame}

\end{document}