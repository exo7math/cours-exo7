
%%%%%%%%%%%%%%%%%% PREAMBULE %%%%%%%%%%%%%%%%%%


\documentclass[12pt]{article}

\usepackage{amsfonts,amsmath,amssymb,amsthm}
\usepackage[utf8]{inputenc}
\usepackage[T1]{fontenc}
\usepackage[francais]{babel}


% packages
\usepackage{amsfonts,amsmath,amssymb,amsthm}
\usepackage[utf8]{inputenc}
\usepackage[T1]{fontenc}
%\usepackage{lmodern}

\usepackage[francais]{babel}
\usepackage{fancybox}
\usepackage{graphicx}

\usepackage{float}

%\usepackage[usenames, x11names]{xcolor}
\usepackage{tikz}
\usepackage{datetime}

\usepackage{mathptmx}
%\usepackage{fouriernc}
%\usepackage{newcent}
\usepackage[mathcal,mathbf]{euler}

%\usepackage{palatino}
%\usepackage{newcent}


% Commande spéciale prompteur

%\usepackage{mathptmx}
%\usepackage[mathcal,mathbf]{euler}
%\usepackage{mathpple,multido}

\usepackage[a4paper]{geometry}
\geometry{top=2cm, bottom=2cm, left=1cm, right=1cm, marginparsep=1cm}

\newcommand{\change}{{\color{red}\rule{\textwidth}{1mm}\\}}

\newcounter{mydiapo}

\newcommand{\diapo}{\newpage
\hfill {\normalsize  Diapo \themydiapo \quad \texttt{[\jobname]}} \\
\stepcounter{mydiapo}}


%%%%%%% COULEURS %%%%%%%%%%

% Pour blanc sur noir :
%\pagecolor[rgb]{0.5,0.5,0.5}
% \pagecolor[rgb]{0,0,0}
% \color[rgb]{1,1,1}



%\DeclareFixedFont{\myfont}{U}{cmss}{bx}{n}{18pt}
\newcommand{\debuttexte}{
%%%%%%%%%%%%% FONTES %%%%%%%%%%%%%
\renewcommand{\baselinestretch}{1.5}
\usefont{U}{cmss}{bx}{n}
\bfseries

% Taille normale : commenter le reste !
%Taille Arnaud
%\fontsize{19}{19}\selectfont

% Taille Barbara
%\fontsize{21}{22}\selectfont

%Taille François
\fontsize{25}{30}\selectfont

%Taille Pascal
%\fontsize{25}{30}\selectfont

%Taille Laura
%\fontsize{30}{35}\selectfont


%\myfont
%\usefont{U}{cmss}{bx}{n}

%\Huge
%\addtolength{\parskip}{\baselineskip}
}


% \usepackage{hyperref}
% \hypersetup{colorlinks=true, linkcolor=blue, urlcolor=blue,
% pdftitle={Exo7 - Exercices de mathématiques}, pdfauthor={Exo7}}


%section
% \usepackage{sectsty}
% \allsectionsfont{\bf}
%\sectionfont{\color{Tomato3}\upshape\selectfont}
%\subsectionfont{\color{Tomato4}\upshape\selectfont}

%----- Ensembles : entiers, reels, complexes -----
\newcommand{\Nn}{\mathbb{N}} \newcommand{\N}{\mathbb{N}}
\newcommand{\Zz}{\mathbb{Z}} \newcommand{\Z}{\mathbb{Z}}
\newcommand{\Qq}{\mathbb{Q}} \newcommand{\Q}{\mathbb{Q}}
\newcommand{\Rr}{\mathbb{R}} \newcommand{\R}{\mathbb{R}}
\newcommand{\Cc}{\mathbb{C}} 
\newcommand{\Kk}{\mathbb{K}} \newcommand{\K}{\mathbb{K}}

%----- Modifications de symboles -----
\renewcommand{\epsilon}{\varepsilon}
\renewcommand{\Re}{\mathop{\text{Re}}\nolimits}
\renewcommand{\Im}{\mathop{\text{Im}}\nolimits}
%\newcommand{\llbracket}{\left[\kern-0.15em\left[}
%\newcommand{\rrbracket}{\right]\kern-0.15em\right]}

\renewcommand{\ge}{\geqslant}
\renewcommand{\geq}{\geqslant}
\renewcommand{\le}{\leqslant}
\renewcommand{\leq}{\leqslant}

%----- Fonctions usuelles -----
\newcommand{\ch}{\mathop{\mathrm{ch}}\nolimits}
\newcommand{\sh}{\mathop{\mathrm{sh}}\nolimits}
\renewcommand{\tanh}{\mathop{\mathrm{th}}\nolimits}
\newcommand{\cotan}{\mathop{\mathrm{cotan}}\nolimits}
\newcommand{\Arcsin}{\mathop{\mathrm{Arcsin}}\nolimits}
\newcommand{\Arccos}{\mathop{\mathrm{Arccos}}\nolimits}
\newcommand{\Arctan}{\mathop{\mathrm{Arctan}}\nolimits}
\newcommand{\Argsh}{\mathop{\mathrm{Argsh}}\nolimits}
\newcommand{\Argch}{\mathop{\mathrm{Argch}}\nolimits}
\newcommand{\Argth}{\mathop{\mathrm{Argth}}\nolimits}
\newcommand{\pgcd}{\mathop{\mathrm{pgcd}}\nolimits} 

\newcommand{\Card}{\mathop{\text{Card}}\nolimits}
\newcommand{\Ker}{\mathop{\text{Ker}}\nolimits}
\newcommand{\id}{\mathop{\text{id}}\nolimits}
\newcommand{\ii}{\mathrm{i}}
\newcommand{\dd}{\mathrm{d}}
\newcommand{\Vect}{\mathop{\text{Vect}}\nolimits}
\newcommand{\Mat}{\mathop{\mathrm{Mat}}\nolimits}
\newcommand{\rg}{\mathop{\text{rg}}\nolimits}
\newcommand{\tr}{\mathop{\text{tr}}\nolimits}
\newcommand{\ppcm}{\mathop{\text{ppcm}}\nolimits}

%----- Structure des exercices ------

\newtheoremstyle{styleexo}% name
{2ex}% Space above
{3ex}% Space below
{}% Body font
{}% Indent amount 1
{\bfseries} % Theorem head font
{}% Punctuation after theorem head
{\newline}% Space after theorem head 2
{}% Theorem head spec (can be left empty, meaning ‘normal’)

%\theoremstyle{styleexo}
\newtheorem{exo}{Exercice}
\newtheorem{ind}{Indications}
\newtheorem{cor}{Correction}


\newcommand{\exercice}[1]{} \newcommand{\finexercice}{}
%\newcommand{\exercice}[1]{{\tiny\texttt{#1}}\vspace{-2ex}} % pour afficher le numero absolu, l'auteur...
\newcommand{\enonce}{\begin{exo}} \newcommand{\finenonce}{\end{exo}}
\newcommand{\indication}{\begin{ind}} \newcommand{\finindication}{\end{ind}}
\newcommand{\correction}{\begin{cor}} \newcommand{\fincorrection}{\end{cor}}

\newcommand{\noindication}{\stepcounter{ind}}
\newcommand{\nocorrection}{\stepcounter{cor}}

\newcommand{\fiche}[1]{} \newcommand{\finfiche}{}
\newcommand{\titre}[1]{\centerline{\large \bf #1}}
\newcommand{\addcommand}[1]{}
\newcommand{\video}[1]{}

% Marge
\newcommand{\mymargin}[1]{\marginpar{{\small #1}}}



%----- Presentation ------
\setlength{\parindent}{0cm}

%\newcommand{\ExoSept}{\href{http://exo7.emath.fr}{\textbf{\textsf{Exo7}}}}

\definecolor{myred}{rgb}{0.93,0.26,0}
\definecolor{myorange}{rgb}{0.97,0.58,0}
\definecolor{myyellow}{rgb}{1,0.86,0}

\newcommand{\LogoExoSept}[1]{  % input : echelle
{\usefont{U}{cmss}{bx}{n}
\begin{tikzpicture}[scale=0.1*#1,transform shape]
  \fill[color=myorange] (0,0)--(4,0)--(4,-4)--(0,-4)--cycle;
  \fill[color=myred] (0,0)--(0,3)--(-3,3)--(-3,0)--cycle;
  \fill[color=myyellow] (4,0)--(7,4)--(3,7)--(0,3)--cycle;
  \node[scale=5] at (3.5,3.5) {Exo7};
\end{tikzpicture}}
}



\theoremstyle{definition}
%\newtheorem{proposition}{Proposition}
%\newtheorem{exemple}{Exemple}
%\newtheorem{theoreme}{Théorème}
\newtheorem{lemme}{Lemme}
\newtheorem{corollaire}{Corollaire}
%\newtheorem*{remarque*}{Remarque}
%\newtheorem*{miniexercice}{Mini-exercices}
%\newtheorem{definition}{Définition}




%definition d'un terme
\newcommand{\defi}[1]{{\color{myorange}\textbf{\emph{#1}}}}
\newcommand{\evidence}[1]{{\color{blue}\textbf{\emph{#1}}}}



 %----- Commandes divers ------

\newcommand{\codeinline}[1]{\texttt{#1}}

%%%%%%%%%%%%%%%%%%%%%%%%%%%%%%%%%%%%%%%%%%%%%%%%%%%%%%%%%%%%%
%%%%%%%%%%%%%%%%%%%%%%%%%%%%%%%%%%%%%%%%%%%%%%%%%%%%%%%%%%%%%

%\usepackage{mathptmx}

\begin{document}

\debuttexte

%%%%%%%%%%%%%%%%%%%%%%%%%%%%%%%%%%%%%%%%%%%%%%%%%%%%%%%%%%
\diapo

\change

Dans cette partie du chapitre sur les suites, 
nous allons étudier des exemples classiques de suites numériques.

\change

Nous commençons par les suites géométriques,

\change

puis nous examinons les \emph{séries} géométriques, qui leur sont étroitement liées.

\change

Nous passons ensuite aux suites vérifiant le critère 
$$\left|\frac{u_{n+1}}{u_n}\right|<\ell<1 \; ,$$ ce qui garantit leur convergence.

\change 

Puis nous conclurons par une application des suites à l'approximation 
des nombres réels par des nombres décimaux.

%%%%%%%%%%%%%%%%%%%%%%%%%%%%%%%%%%%%%%%%%%%%%%%%%%%%%%%%%%
\diapo

Nous commençons par les suites géométriques, 
dont le terme général est du type $a^n$, 
pour un nombre réel $a$ fixé (qui s'appelle la raison).


Cette proposition affirme que la raison détermine la convergence de la suite.

\change

Si $a=1$, alors la suite est constante égale à $1$, et donc converge.

\change

Si $a>1$, alors la suite tend vers $+\infty$.

\change 

Si $a$ est dans $]-1,+1[$, alors la suite tend vers $0$.

\change

Si $a\le-1$, alors la suite diverge.

\change

Donnons la preuve du deuxième point. On écrit
$a=1+b$ avec $b>0$. 

\change

Alors la formule du binôme de Newton s'écrit

$a^n=(1+b)^n=1+nb+\binom{n}{2}b^2+\cdots+\binom{n}{k}b^k+\cdots+b^n$. 

\change

Tous les termes sont positifs,  donc  $a^n\geq 1+nb$. 

\change
 
Or $1+nb\to+\infty$ car $b>0$. 


\change

On en déduit que 

$\lim_{n\to +\infty} a^n=+\infty$. 

Les troisième et quatrième points se déduisent du deuxième.







%%%%%%%%%%%%%%%%%%%%%%%%%%%%%%%%%%%%%%%%%%%%%%%%%%%%%%%%%%
\diapo

Passons aux séries géométriques. Ce sont des suites dont le terme général 
est la somme des termes d'une suite géométrique.

Cette somme se calcule facilement, on va montrer que

$\sum_{k=0}^na^k=\frac{1-a^{n+1}}{1-a} $

si $a\neq 1$,

\change


Autrement dit $1+a+a^2+\cdots+a^n=\frac{1-a^{n+1}}{1-a} $



\change


La preuve est simple : on développe l'expression 
\[(1-a)\big(1+a+a^2+\cdots+a^n \big)\; ,\] 

\change

ce qui fait apparaître la différence des deux sommes 

$\big(1+a+a^2+\cdots+a^n \big)$ et 

$\big(a+a^2+\cdots+a^{n+1} \big)$ qui se \og{} télescopent \fg{}:

\change

presque tous les termes disparaissent et il reste $1-a^{n+1}$, ce qui conclut la preuve.





%%%%%%%%%%%%%%%%%%%%%%%%%%%%%%%%%%%%%%%%%%%%%%%%%%%%%%%%%%
 \diapo


Lorsque $a\in ]-1,1[$, on peut déduire des deux dernières propositions 
un résultat de convergence de la série géométrique de raison $a$, 

c'est-à-dire la suite $(u_n)_{n\in \Nn}$ de terme général : 
$u_n= \sum_{k=0}^na^k$, 


\change

Cette série vaut $\frac{1-a^{n+1}}{1-a}$ 

\change

elle converge donc vers $\frac{1}{1-a}$ car $a^{n+1} \to 0$. 

\change

On pourrait écrire ce résultat de la manière suivante :
\[1+a+a^2+a^3 +\cdots = \frac{1}{1-a}\]

\change

Le nombre $a$ peut en fait être complexe, la formule de la somme d'une suite géométrique reste valable pour $a$ nombre complexe 
$\neq 1$.

\change

Si $a=1$ alors la formule précédente n'est plus valable mais
la somme
est alors simplement $n+1$.

\change

Si par exemple $a=\frac{1}{2}$ on obtient
  
  \[1+\frac{1}{2} +\frac{1}{4} +\frac{1}{8} +\cdots = 2 .\]


Une formule qui se comprend bien sur un dessin :
on coupe un segment de longueur $2$ en son milieu, la moitié de gauche est de longueur $1$,

on coupe la partie restante en deux, on garde la moitié gauche de longueur $1/2$ et on recoupe la partie droite,...
Ce processus continue indéfiniment et correspond à l'égalité 
 \[1+\frac{1}{2} +\frac{1}{4} +\frac{1}{8} +\cdots = 2 .\]





%%%%%%%%%%%%%%%%%%%%%%%%%%%%%%%%%%%%%%%%%%%%%%%%%%%%%%%%%%
\diapo

Voici un critère pratique permettant d'affirmer qu'une suite tend vers $0$.

S'il existe un nombre réel $\ell<1$ indépendant de $n$ tel que pour tout les $n$, on a
\[ \left | \frac{u_{n+1}}{u_n}\right |  <\ell<1.\]


alors la suite $(u_n)_{n\in \Nn}$ tend vers $0$.

\change

On peut aussi seulement supposer que l'inégalité est vraie à partir d'un certain rang
et la conclusion reste valide.

\change

Pour prouver ceci, on écrit le rapport \(\frac{u_n}{u_0}\) comme un \og{} produit télescopique \fg{} :
 \[ \frac{u_n}{u_0}=\frac{u_1}{u_0} \times\frac{u_2}{u_1} \times\frac{u_3}{u_2} \times\cdots\times \frac{u_n}{u_{n-1}}\] 

\change

on majore en valeur absolue,  chaque facteur étant $<\ell$.

\change

ce qui permet de majorer $|u_n|$, vu l'hypothèse sur la suite, par $|u_0|\ell^n$.

\change 

Comme la raison $\ell$ est $<1$, on sait que 
$\lim_{n\to +\infty} \ell^n= 0$, et il s'ensuit que $\lim_{n\to +\infty} u_n= 0$.  


%%%%%%%%%%%%%%%%%%%%%%%%%%%%%%%%%%%%%%%%%%%%%%%%%%%%%%%%%%
\diapo

Voici un corollaire immédiat de ce critère qui est très utile .


Si $\frac{u_{n+1}}{u_n}$ tend vers $0$,

alors $\lim_{n\to +\infty} u_n= 0$.

\change


Par exemple nous allons montrer que, pour $a$ un réel fixé, alors $\frac{a^n}{n!}$ tend vers $0$.

\change

Tout d'abord si $a=0$, le résultat est évident. 

\change

Supposons donc $a\neq 0$,

et définissons $u_n= \frac{a^n}{n!}$. 

\change

Calculons le quotient $\frac{u_{n+1}}{u_n}$

$\frac{u_{n+1}}{u_n}= \frac{a^{n+1}}{(n+1)!}\cdot \frac{n!}{a^n}$

\change

$=\frac{a}{n+1} $

\change

Comme $a$ est fixe alors $\frac{u_{n+1}}{u_n} \to 0$

\change 

Et par ce corollaire on en déduit que $u_n \to 0$

c-a-d  $\frac{a^n}{n!} \to 0$ lorsque $n\to +\infty$.


%%%%%%%%%%%%%%%%%%%%%%%%%%%%%%%%%%%%%%%%%%%%%%%%%%%%%%%%%%
\diapo

Nous terminons cette partie sur les suites en parlant 
de l'approximation des réels par des décimaux.



Pour chaque réel $a$, on peut, à l'aide de la 
fonction partie entière, construire une suite de rationnels tendant vers $a$.

Explicitement, la suite de terme général

 \[u_n = \frac{E(10^na)}{10^n} \]

 convient.

 Cette suite est une suite de nombres rationnels et tend vers $a$.
 
 
\change

Pour prouver que cette suite tend vers $a$, on écrit la définition de la partie entière pour $10^n a$ :

  \[E(10^na)\leq 10^na < E(10^na)+1\]

\change
  
  On en déduit en divisant par $10^n$ que 
\( u_n \leq a < u_n+\frac{1}{10^n} \)

\change

ce qui revient à dire \( 0 \leq a -u_n< \frac{1}{10^n} .\)

\change

Comme la suite de terme général $\frac{1}{10^n}$ tend vers $0$ 
(car c'est une suite géométrique de raison $\frac{1}{10}<1$), 
on conclut que $\lim_{n\to +\infty} u_n=a$.


%%%%%%%%%%%%%%%%%%%%%%%%%%%%%%%%%%%%%%%%%%%%%%%%%%%%%%%%%%
\diapo


Voici ce que vaut cette suite de rationnels lorsque $a=\pi$.

\change

Tout d'abord $u_0$  vaut simplement $E(\pi)$ donc $3$.

\change

Puis $u_1$ vaut $\frac{E(10^1\pi)}{10^1}$ donc $\frac{E(31,415)\ldots}{10}$, soit $3,1$

\change

Pour $u_2$ cela donne $3,14$

\change

Et en général $u_n$ est la troncation du développement de $\pi$ à $n$ décimales.

\change

Deux remarques pour finir :

d'une part, pour tout $a$, les nombres $u_n$ sont en fait décimaux, donc en particulier ce sont des nombres rationnels.

\change

d'autre part ceci fournit une nouvelle démonstration de la densité de $\Qq$ dans $\Rr$. 

Pour un réel $a$ fixé et pour tout $\epsilon $ aussi petit que l'on veut
alors pour $n$ assez grand les éléments de notre suite $(u_n)$
sont des rationnels de l'intervalle $]a-\epsilon,a+\epsilon[$.


%%%%%%%%%%%%%%%%%%%%%%%%%%%%%%%%%%%%%%%%%%%%%%%%%%%%%%%%%%
\diapo

Voici quelques exercices pour tester votre compréhension du cours.

\end{document}
