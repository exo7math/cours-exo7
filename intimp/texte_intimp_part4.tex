
%%%%%%%%%%%%%%%%%% PREAMBULE %%%%%%%%%%%%%%%%%%


\documentclass[12pt]{article}

\usepackage{amsfonts,amsmath,amssymb,amsthm}
\usepackage[utf8]{inputenc}
\usepackage[T1]{fontenc}
\usepackage[francais]{babel}


% packages
\usepackage{amsfonts,amsmath,amssymb,amsthm}
\usepackage[utf8]{inputenc}
\usepackage[T1]{fontenc}
%\usepackage{lmodern}

\usepackage[francais]{babel}
\usepackage{fancybox}
\usepackage{graphicx}

\usepackage{float}

%\usepackage[usenames, x11names]{xcolor}
\usepackage{tikz}
\usepackage{datetime}

\usepackage{mathptmx}
%\usepackage{fouriernc}
%\usepackage{newcent}
\usepackage[mathcal,mathbf]{euler}

%\usepackage{palatino}
%\usepackage{newcent}


% Commande spéciale prompteur

%\usepackage{mathptmx}
%\usepackage[mathcal,mathbf]{euler}
%\usepackage{mathpple,multido}

\usepackage[a4paper]{geometry}
\geometry{top=2cm, bottom=2cm, left=1cm, right=1cm, marginparsep=1cm}

\newcommand{\change}{{\color{red}\rule{\textwidth}{1mm}\\}}

\newcounter{mydiapo}

\newcommand{\diapo}{\newpage
\hfill {\normalsize  Diapo \themydiapo \quad \texttt{[\jobname]}} \\
\stepcounter{mydiapo}}


%%%%%%% COULEURS %%%%%%%%%%

% Pour blanc sur noir :
%\pagecolor[rgb]{0.5,0.5,0.5}
% \pagecolor[rgb]{0,0,0}
% \color[rgb]{1,1,1}



%\DeclareFixedFont{\myfont}{U}{cmss}{bx}{n}{18pt}
\newcommand{\debuttexte}{
%%%%%%%%%%%%% FONTES %%%%%%%%%%%%%
\renewcommand{\baselinestretch}{1.5}
\usefont{U}{cmss}{bx}{n}
\bfseries

% Taille normale : commenter le reste !
%Taille Arnaud
%\fontsize{19}{19}\selectfont

% Taille Barbara
%\fontsize{21}{22}\selectfont

%Taille François
\fontsize{25}{30}\selectfont

%Taille Pascal
%\fontsize{25}{30}\selectfont

%Taille Laura
%\fontsize{30}{35}\selectfont


%\myfont
%\usefont{U}{cmss}{bx}{n}

%\Huge
%\addtolength{\parskip}{\baselineskip}
}


% \usepackage{hyperref}
% \hypersetup{colorlinks=true, linkcolor=blue, urlcolor=blue,
% pdftitle={Exo7 - Exercices de mathématiques}, pdfauthor={Exo7}}


%section
% \usepackage{sectsty}
% \allsectionsfont{\bf}
%\sectionfont{\color{Tomato3}\upshape\selectfont}
%\subsectionfont{\color{Tomato4}\upshape\selectfont}

%----- Ensembles : entiers, reels, complexes -----
\newcommand{\Nn}{\mathbb{N}} \newcommand{\N}{\mathbb{N}}
\newcommand{\Zz}{\mathbb{Z}} \newcommand{\Z}{\mathbb{Z}}
\newcommand{\Qq}{\mathbb{Q}} \newcommand{\Q}{\mathbb{Q}}
\newcommand{\Rr}{\mathbb{R}} \newcommand{\R}{\mathbb{R}}
\newcommand{\Cc}{\mathbb{C}} 
\newcommand{\Kk}{\mathbb{K}} \newcommand{\K}{\mathbb{K}}

%----- Modifications de symboles -----
\renewcommand{\epsilon}{\varepsilon}
\renewcommand{\Re}{\mathop{\text{Re}}\nolimits}
\renewcommand{\Im}{\mathop{\text{Im}}\nolimits}
%\newcommand{\llbracket}{\left[\kern-0.15em\left[}
%\newcommand{\rrbracket}{\right]\kern-0.15em\right]}

\renewcommand{\ge}{\geqslant}
\renewcommand{\geq}{\geqslant}
\renewcommand{\le}{\leqslant}
\renewcommand{\leq}{\leqslant}

%----- Fonctions usuelles -----
\newcommand{\ch}{\mathop{\mathrm{ch}}\nolimits}
\newcommand{\sh}{\mathop{\mathrm{sh}}\nolimits}
\renewcommand{\tanh}{\mathop{\mathrm{th}}\nolimits}
\newcommand{\cotan}{\mathop{\mathrm{cotan}}\nolimits}
\newcommand{\Arcsin}{\mathop{\mathrm{Arcsin}}\nolimits}
\newcommand{\Arccos}{\mathop{\mathrm{Arccos}}\nolimits}
\newcommand{\Arctan}{\mathop{\mathrm{Arctan}}\nolimits}
\newcommand{\Argsh}{\mathop{\mathrm{Argsh}}\nolimits}
\newcommand{\Argch}{\mathop{\mathrm{Argch}}\nolimits}
\newcommand{\Argth}{\mathop{\mathrm{Argth}}\nolimits}
\newcommand{\pgcd}{\mathop{\mathrm{pgcd}}\nolimits} 

\newcommand{\Card}{\mathop{\text{Card}}\nolimits}
\newcommand{\Ker}{\mathop{\text{Ker}}\nolimits}
\newcommand{\id}{\mathop{\text{id}}\nolimits}
\newcommand{\ii}{\mathrm{i}}
\newcommand{\dd}{\mathrm{d}}
\newcommand{\Vect}{\mathop{\text{Vect}}\nolimits}
\newcommand{\Mat}{\mathop{\mathrm{Mat}}\nolimits}
\newcommand{\rg}{\mathop{\text{rg}}\nolimits}
\newcommand{\tr}{\mathop{\text{tr}}\nolimits}
\newcommand{\ppcm}{\mathop{\text{ppcm}}\nolimits}

%----- Structure des exercices ------

\newtheoremstyle{styleexo}% name
{2ex}% Space above
{3ex}% Space below
{}% Body font
{}% Indent amount 1
{\bfseries} % Theorem head font
{}% Punctuation after theorem head
{\newline}% Space after theorem head 2
{}% Theorem head spec (can be left empty, meaning ‘normal’)

%\theoremstyle{styleexo}
\newtheorem{exo}{Exercice}
\newtheorem{ind}{Indications}
\newtheorem{cor}{Correction}


\newcommand{\exercice}[1]{} \newcommand{\finexercice}{}
%\newcommand{\exercice}[1]{{\tiny\texttt{#1}}\vspace{-2ex}} % pour afficher le numero absolu, l'auteur...
\newcommand{\enonce}{\begin{exo}} \newcommand{\finenonce}{\end{exo}}
\newcommand{\indication}{\begin{ind}} \newcommand{\finindication}{\end{ind}}
\newcommand{\correction}{\begin{cor}} \newcommand{\fincorrection}{\end{cor}}

\newcommand{\noindication}{\stepcounter{ind}}
\newcommand{\nocorrection}{\stepcounter{cor}}

\newcommand{\fiche}[1]{} \newcommand{\finfiche}{}
\newcommand{\titre}[1]{\centerline{\large \bf #1}}
\newcommand{\addcommand}[1]{}
\newcommand{\video}[1]{}

% Marge
\newcommand{\mymargin}[1]{\marginpar{{\small #1}}}



%----- Presentation ------
\setlength{\parindent}{0cm}

%\newcommand{\ExoSept}{\href{http://exo7.emath.fr}{\textbf{\textsf{Exo7}}}}

\definecolor{myred}{rgb}{0.93,0.26,0}
\definecolor{myorange}{rgb}{0.97,0.58,0}
\definecolor{myyellow}{rgb}{1,0.86,0}

\newcommand{\LogoExoSept}[1]{  % input : echelle
{\usefont{U}{cmss}{bx}{n}
\begin{tikzpicture}[scale=0.1*#1,transform shape]
  \fill[color=myorange] (0,0)--(4,0)--(4,-4)--(0,-4)--cycle;
  \fill[color=myred] (0,0)--(0,3)--(-3,3)--(-3,0)--cycle;
  \fill[color=myyellow] (4,0)--(7,4)--(3,7)--(0,3)--cycle;
  \node[scale=5] at (3.5,3.5) {Exo7};
\end{tikzpicture}}
}



\theoremstyle{definition}
%\newtheorem{proposition}{Proposition}
%\newtheorem{exemple}{Exemple}
%\newtheorem{theoreme}{Théorème}
\newtheorem{lemme}{Lemme}
\newtheorem{corollaire}{Corollaire}
%\newtheorem*{remarque*}{Remarque}
%\newtheorem*{miniexercice}{Mini-exercices}
%\newtheorem{definition}{Définition}




%definition d'un terme
\newcommand{\defi}[1]{{\color{myorange}\textbf{\emph{#1}}}}
\newcommand{\evidence}[1]{{\color{blue}\textbf{\emph{#1}}}}



 %----- Commandes divers ------

\newcommand{\codeinline}[1]{\texttt{#1}}

%%%%%%%%%%%%%%%%%%%%%%%%%%%%%%%%%%%%%%%%%%%%%%%%%%%%%%%%%%%%%
%%%%%%%%%%%%%%%%%%%%%%%%%%%%%%%%%%%%%%%%%%%%%%%%%%%%%%%%%%%%%



\begin{document}

\debuttexte


%%%%%%%%%%%%%%%%%%%%%%%%%%%%%%%%%%%%%%%%%%%%%%%%%%%%%%%%%%%
\diapo

Dans cette section, nous allons nous intéresser aux intégrales 
impropres dont le domaine d'intégration est un intervalle borné.

  \change

  \change

Nous commencerons par préciser le problème de la 
convergence dans le cas des fonctions de signe constant

  \change

nous énoncerons le théorème de comparaison
 
\change

et le théorème des équivalents.

\change
 
Nous conclurons par l'étude des fonctions oscillantes.

 
%%%%%%%%%%%%%%%%%%%%%%%%%%%%%%%%%%%%%%%%%%%%%%%%%%%%%%%%%%%
 \diapo

Nous commençons par le cas d'une fonction positive qui tend vers
l'infini en l'une des bornes de l'intervalle d'intégration. 
Le traitement est tout à fait analogue au cas d'une fonction positive 
sur un intervalle non borné et l'on omettra les démonstrations.

Nous supposerons que la fonction  tend vers $+\infty$ en $a$.  

\change

Rappelons que, par définition,
$$\int_a^b f(t)\;\dd t = \lim_{x\rightarrow a^+} \int_x^b f(t)\;\dd t\;.$$

Observons que si la fonction $f$ est positive, alors,
% $\int_x^b f(t)\;\dd t$ est décroissante sur $]a,b]$, donc 
soit $\int_x^b f(t)\;\dd t$ est bornée, et l'intégrale $\int_a^b
f(t)\;\dd t$ est convergente, soit $\int_x^b f(t)\;\dd t$ tend vers
$+\infty$.


%%%%%%%%%%%%%%%%%%%%%%%%%%%%%%%%%%%%%%%%%%%%%%%%%%%%%%%%%%%
\diapo

Énonçons le théorème de comparaison : soient $f$ et $g$ 
deux fonctions positives et continues sur $]a,b]$, 
telles que $f$ soit majorée par $g$ au 
voisinage de $a$ :

c-à-d : $\exists \epsilon>0 \quad \forall t\in ]a,a+\epsilon] \qquad f(t)\le g(t)$

d'une part, si $\int_a^b g(t)\;\dd t$ \  converge 
alors \  $\int_a^b f(t)\;\dd t$ \  converge aussi. 

et d'autre part, la contraposée dit que :
Si \  $\int_a^{b} f(t)\;\dd t$ \  diverge 
alors \  $\int_a^{b} g(t)\;\dd t$ \  diverge aussi.



\change
Voici un exemple d'application :

Étant donné un réel $\alpha$, étudions la convergence de 

$$\int_0^1 \frac{(-\ln t)^\alpha}{\sqrt{t}}\;\dd t$$

\change
On commence par comparer notre fonction à une fonction plus 
simple en écrivant 

$$\frac{(-\ln t)^\alpha}{\sqrt{t}} 
= \big((-\ln t)^\alpha t^{1/4}\big)\,t^{-3/4}\;. $$

Le choix de décomposé $1/\sqrt{t}$  en $t^{1/4}*t^{-3/4}$
est un peu arbitraire mais pas trop comme on le verra !

\change
On sait que $\lim_{t\rightarrow 0^+}  (-\ln t)^\alpha t^{1/4}=0$

 pour tout $\alpha$, car les puissances 
 l'emportent sur le logarithme.
 

 \change 
 en particulier il existe $\epsilon>0$ tel que :
sur $]0,\epsilon]$ on ait $(-\ln t)^\alpha t^{1/4}\le 1\;.$ 

 \change 
En multipliant les deux membres de l'inégalité 
par $t^{-3/4}$ on obtient :
$\frac{(-\ln t)^\alpha}{\sqrt{t}} \le t^{-3/4}$
sur $]0,\epsilon]$.


\change
Or une primitive de $t^{-3/4}$ est $t^{1/4}$, donc
l'intégrale du type Riemann $\int_0^1 t^{-3/4}\;\dd t$ 
vaut la limite du crochet $[t^{1/4}]_\epsilon^1$ 
qui vaut $1$ et donc converge.

C'est ici où il était important d'avoir bien 
choisi notre découpage, afin d'avoir
une intégrale convergente.

\change
On peut donc appliquer le théorème comparaison ci-dessus : 
puisque $\int_0^1 t^{-3/4}\;\dd t$ converge, 
alors $\int_0^1 \frac{(-\ln t)^\alpha}{\sqrt{t}}\;\dd t$ 
converge aussi, quelle que soit la valeur de $\alpha$.  

%%%%%%%%%%%%%%%%%%%%%%%%%%%%%%%%%%%%%%%%%%%%%%%%%%%%%%%%%%%
\diapo

Gr\^ace au théorème des équivalents, 
on peut remplacer la fonction à intégrer 
par un équivalent au voisinage de $a$ pour
étudier la convergence d'une intégrale. Énonçons le :

Si $f$ et $g$ sont deux fonctions continues et strictement positives sur
$]a,b]$ qui sont équivalentes au voisinage de $a$, 
alors l'intégrale $\int_a^b f(t)\;\dd t$ 
converge si et seulement si  $\int_a^b g(t)\;\dd t$ converge. \\

On rappelle que deux fonctions sont équivalentes 
en un point si leur rapport tend vers $1$ en ce point.
Le théorème dit donc que, lorsque c'est le cas, les deux intégrales sont de même nature.

Attention : il est important dans le théorème 
que $f$ et $g$ soient positives.

\change
Voici un exemple :

$$\int_0^1 \sqrt{\frac{-\ln t+1}{\sin t}}\;\dd t\qquad\text{ converge.}$$

\change
En effet on montre que 
$\sqrt{\frac{-\ln t+1}{\sin t}} \quad\mathop{\sim}_{0^+}\quad
\frac{(-\ln t)^{1/2}}{\sqrt{t}}\;,$ 
et on vient de voir dans l'exemple précédent avec ici $\alpha=1/2$,
que 
$\int_0^1 \frac{(-\ln t)^{1/2}}{\sqrt{t}}\;\dd t$ converge.   


%%%%%%%%%%%%%%%%%%%%%%%%%%%%%%%%%%%%%%%%%%%%%%%%%%%%%%%%%%%
\diapo

L'utilisation des équivalents permet ainsi d'étudier la convergence d'une intégrale 
pour laquelle on n'a pas de primitive en la comparant à un catalogue d'intégrales 
dont la convergence est connue. Les plus classiques sont les intégrales de Riemann 
du type $\int_0^1 \frac{1}{t^\alpha}\;\dd t$, 

$$\text{Si } *\alpha < 1* \quad \text{ alors } \quad \int_0^1 \frac{1}{t^\alpha}\;\dd t
\quad\text{ converge.}$$
$$\text{Si } *\alpha \ge 1* \quad \text{ alors } \quad \int_0^1 \frac{1}{t^\alpha}\;\dd t
\quad\text{ diverge.}$$  

\change
Faites bien attention, la convergence en fonction du paramètre $\alpha$ est
*inversée* par rapport aux intégrales de Riemann entre $1$ et $+\infty$ :
Ici, si $\alpha < 1$ l'intégrale converge,
si $\alpha \ge 1$ l'intégrale diverge.


%%%%%%%%%%%%%%%%%%%%%%%%%%%%%%%%%%%%%%%%%%%%%%%%%%%%%%%%%%%
\diapo

Le dernier cas à traiter est celui où la fonction à intégrer
oscille au voisinage d'une des bornes, prenant des valeurs
arbitrairement proches de $+\infty$ ou $-\infty$, lorsque $x$ tend $a$.  

\change
Rappelons que, la définition de l'intégrale impropre reste inchangée :
$$\int_a^b f(t)\;\dd t = \lim_{x\rightarrow a^+} \int_x^b f(t)\;\dd t\;.$$

\change

Le changement de variable $u=\frac{1}{t-a}$ permet de se ramener au cas,
vu dans une autre section, d'une fonction oscillante sur un intervalle non borné, 
ce qui nous dispensera de donner autant de détails.

%%%%%%%%%%%%%%%%%%%%%%%%%%%%%%%%%%%%%%%%%%%%%%%%%%%%%%%%%%%
\diapo

La notion importante est toujours la convergence absolue.

Comme dans le cas d'une fonction sur un intervalle non borné, 
on dit que $\int f$  est \defi{absolument convergente} 
si $\int \big|f\big|$ est convergente.

\change

Si l'intégrale $\int_a^b f(t)\;\dd t$ est absolument convergente,
alors elle est convergente : on dit que la convergence absolue 
implique la convergence.

La preuve est la même qu'auparavant, c'est-à-dire utilise 
le critère de Cauchy pour les fonctions.

\change

Voici deux exemples.

L'intégrale $\int_0^1 \frac{\sin\frac1t}{\sqrt{t}}\;\dd t
\quad\text{ est absolument convergente,}$. 

\change
En effet, on peut majorer $\frac{\big|\sin\frac1t\big|}{\sqrt{t}}$ 
par $\frac{1}{\sqrt{t}}$, 

\change
et on a vu que l'intégrale $\int_0^1 \frac{1}{\sqrt{t}}\;\dd t$ 
converge.

\change

Deuxième exemple : $\int_0^1 \frac{\sin\frac1t}{t}\;\dd t$ 
n'est pas absolument convergente mais elle est convergente.

\change
En effet  le changement de variable $t\mapsto \frac1u$ donne transforme 
$\int_x^1 \frac{\sin\frac1t}{t}\;\dd t$

en ceci, qui n'est autre que $\int_1^{1/x} \frac{\sin u}{u}\;\dd u$

Quand $x \to 0^+$, $\frac1x \to +\infty$, 
on est donc ramené au cas de l'intégrale 
$\int_1^{\infty} \frac{\sin u}{u}\;\dd u$ dont 
on a vu qu'elle converge, mais pas absolument.


%%%%%%%%%%%%%%%%%%%%%%%%%%%%%%%%%%%%%%%%%%%%%%%%%%%%%%%%%%%
\diapo

Voici quelques exercices sur les intégrales 
impropres sur un intervalle borné.

\end{document}
