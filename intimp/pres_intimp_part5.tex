
%%%%%%%%%%%%%%%%%% PREAMBULE %%%%%%%%%%%%%%%%%%

\documentclass[aspectratio=169,utf8]{beamer}
%\documentclass[aspectratio=169,handout]{beamer}

\usetheme{Boadilla}
%\usecolortheme{seahorse}
\usecolortheme[RGB={245,66,24}]{structure}
\useoutertheme{infolines}

% packages
\usepackage{amsfonts,amsmath,amssymb,amsthm}
\usepackage[utf8]{inputenc}
\usepackage[T1]{fontenc}
\usepackage{lmodern}

\usepackage[francais]{babel}
\usepackage{fancybox}
\usepackage{graphicx}

\usepackage{float}
\usepackage{xfrac}

%\usepackage[usenames, x11names]{xcolor}
\usepackage{tikz}
\usepackage{pgfplots}
\usepackage{datetime}



%-----  Package unités -----
\usepackage{siunitx}
\sisetup{locale = FR,detect-all,per-mode = symbol}

%\usepackage{mathptmx}
%\usepackage{fouriernc}
%\usepackage{newcent}
%\usepackage[mathcal,mathbf]{euler}

%\usepackage{palatino}
%\usepackage{newcent}
% \usepackage[mathcal,mathbf]{euler}



% \usepackage{hyperref}
% \hypersetup{colorlinks=true, linkcolor=blue, urlcolor=blue,
% pdftitle={Exo7 - Exercices de mathématiques}, pdfauthor={Exo7}}


%section
% \usepackage{sectsty}
% \allsectionsfont{\bf}
%\sectionfont{\color{Tomato3}\upshape\selectfont}
%\subsectionfont{\color{Tomato4}\upshape\selectfont}

%----- Ensembles : entiers, reels, complexes -----
\newcommand{\Nn}{\mathbb{N}} \newcommand{\N}{\mathbb{N}}
\newcommand{\Zz}{\mathbb{Z}} \newcommand{\Z}{\mathbb{Z}}
\newcommand{\Qq}{\mathbb{Q}} \newcommand{\Q}{\mathbb{Q}}
\newcommand{\Rr}{\mathbb{R}} \newcommand{\R}{\mathbb{R}}
\newcommand{\Cc}{\mathbb{C}} 
\newcommand{\Kk}{\mathbb{K}} \newcommand{\K}{\mathbb{K}}

%----- Modifications de symboles -----
\renewcommand{\epsilon}{\varepsilon}
\renewcommand{\Re}{\mathop{\text{Re}}\nolimits}
\renewcommand{\Im}{\mathop{\text{Im}}\nolimits}
%\newcommand{\llbracket}{\left[\kern-0.15em\left[}
%\newcommand{\rrbracket}{\right]\kern-0.15em\right]}

\renewcommand{\ge}{\geqslant}
\renewcommand{\geq}{\geqslant}
\renewcommand{\le}{\leqslant}
\renewcommand{\leq}{\leqslant}
\renewcommand{\epsilon}{\varepsilon}

%----- Fonctions usuelles -----
\newcommand{\ch}{\mathop{\text{ch}}\nolimits}
\newcommand{\sh}{\mathop{\text{sh}}\nolimits}
\renewcommand{\tanh}{\mathop{\text{th}}\nolimits}
\newcommand{\cotan}{\mathop{\text{cotan}}\nolimits}
\newcommand{\Arcsin}{\mathop{\text{arcsin}}\nolimits}
\newcommand{\Arccos}{\mathop{\text{arccos}}\nolimits}
\newcommand{\Arctan}{\mathop{\text{arctan}}\nolimits}
\newcommand{\Argsh}{\mathop{\text{argsh}}\nolimits}
\newcommand{\Argch}{\mathop{\text{argch}}\nolimits}
\newcommand{\Argth}{\mathop{\text{argth}}\nolimits}
\newcommand{\pgcd}{\mathop{\text{pgcd}}\nolimits} 


%----- Commandes divers ------
\newcommand{\ii}{\mathrm{i}}
\newcommand{\dd}{\text{d}}
\newcommand{\id}{\mathop{\text{id}}\nolimits}
\newcommand{\Ker}{\mathop{\text{Ker}}\nolimits}
\newcommand{\Card}{\mathop{\text{Card}}\nolimits}
\newcommand{\Vect}{\mathop{\text{Vect}}\nolimits}
\newcommand{\Mat}{\mathop{\text{Mat}}\nolimits}
\newcommand{\rg}{\mathop{\text{rg}}\nolimits}
\newcommand{\tr}{\mathop{\text{tr}}\nolimits}


%----- Structure des exercices ------

\newtheoremstyle{styleexo}% name
{2ex}% Space above
{3ex}% Space below
{}% Body font
{}% Indent amount 1
{\bfseries} % Theorem head font
{}% Punctuation after theorem head
{\newline}% Space after theorem head 2
{}% Theorem head spec (can be left empty, meaning ‘normal’)

%\theoremstyle{styleexo}
\newtheorem{exo}{Exercice}
\newtheorem{ind}{Indications}
\newtheorem{cor}{Correction}


\newcommand{\exercice}[1]{} \newcommand{\finexercice}{}
%\newcommand{\exercice}[1]{{\tiny\texttt{#1}}\vspace{-2ex}} % pour afficher le numero absolu, l'auteur...
\newcommand{\enonce}{\begin{exo}} \newcommand{\finenonce}{\end{exo}}
\newcommand{\indication}{\begin{ind}} \newcommand{\finindication}{\end{ind}}
\newcommand{\correction}{\begin{cor}} \newcommand{\fincorrection}{\end{cor}}

\newcommand{\noindication}{\stepcounter{ind}}
\newcommand{\nocorrection}{\stepcounter{cor}}

\newcommand{\fiche}[1]{} \newcommand{\finfiche}{}
\newcommand{\titre}[1]{\centerline{\large \bf #1}}
\newcommand{\addcommand}[1]{}
\newcommand{\video}[1]{}

% Marge
\newcommand{\mymargin}[1]{\marginpar{{\small #1}}}

\def\noqed{\renewcommand{\qedsymbol}{}}


%----- Presentation ------
\setlength{\parindent}{0cm}

%\newcommand{\ExoSept}{\href{http://exo7.emath.fr}{\textbf{\textsf{Exo7}}}}

\definecolor{myred}{rgb}{0.93,0.26,0}
\definecolor{myorange}{rgb}{0.97,0.58,0}
\definecolor{myyellow}{rgb}{1,0.86,0}

\newcommand{\LogoExoSept}[1]{  % input : echelle
{\usefont{U}{cmss}{bx}{n}
\begin{tikzpicture}[scale=0.1*#1,transform shape]
  \fill[color=myorange] (0,0)--(4,0)--(4,-4)--(0,-4)--cycle;
  \fill[color=myred] (0,0)--(0,3)--(-3,3)--(-3,0)--cycle;
  \fill[color=myyellow] (4,0)--(7,4)--(3,7)--(0,3)--cycle;
  \node[scale=5] at (3.5,3.5) {Exo7};
\end{tikzpicture}}
}


\newcommand{\debutmontitre}{
  \author{} \date{} 
  \thispagestyle{empty}
  \hspace*{-10ex}
  \begin{minipage}{\textwidth}
    \titlepage  
  \vspace*{-2.5cm}
  \begin{center}
    \LogoExoSept{2.5}
  \end{center}
  \end{minipage}

  \vspace*{-0cm}
  
  % Astuce pour que le background ne soit pas discrétisé lors de la conversion pdf -> png
\begin{tikzpicture}
        \fill[opacity=0,green!60!black] (0,0)--++(0,0)--++(0,0)--++(0,0)--cycle; 
\end{tikzpicture}

% toc S'affiche trop tot :
% \tableofcontents[hideallsubsections, pausesections]
}

\newcommand{\finmontitre}{
  \end{frame}
  \setcounter{framenumber}{0}
} % ne marche pas pour une raison obscure

%----- Commandes supplementaires ------

% \usepackage[landscape]{geometry}
% \geometry{top=1cm, bottom=3cm, left=2cm, right=10cm, marginparsep=1cm
% }
% \usepackage[a4paper]{geometry}
% \geometry{top=2cm, bottom=2cm, left=2cm, right=2cm, marginparsep=1cm
% }

%\usepackage{standalone}


% New command Arnaud -- november 2011
\setbeamersize{text margin left=24ex}
% si vous modifier cette valeur il faut aussi
% modifier le decalage du titre pour compenser
% (ex : ici =+10ex, titre =-5ex

\theoremstyle{definition}
%\newtheorem{proposition}{Proposition}
%\newtheorem{exemple}{Exemple}
%\newtheorem{theoreme}{Théorème}
%\newtheorem{lemme}{Lemme}
%\newtheorem{corollaire}{Corollaire}
%\newtheorem*{remarque*}{Remarque}
%\newtheorem*{miniexercice}{Mini-exercices}
%\newtheorem{definition}{Définition}

% Commande tikz
\usetikzlibrary{calc}
\usetikzlibrary{patterns,arrows}
\usetikzlibrary{matrix}
\usetikzlibrary{fadings} 

%definition d'un terme
\newcommand{\defi}[1]{{\color{myorange}\textbf{\emph{#1}}}}
\newcommand{\evidence}[1]{{\color{blue}\textbf{\emph{#1}}}}
\newcommand{\assertion}[1]{\emph{\og#1\fg}}  % pour chapitre logique
%\renewcommand{\contentsname}{Sommaire}
\renewcommand{\contentsname}{}
\setcounter{tocdepth}{2}



%------ Figures ------

\def\myscale{1} % par défaut 
\newcommand{\myfigure}[2]{  % entrée : echelle, fichier figure
\def\myscale{#1}
\begin{center}
\footnotesize
{#2}
\end{center}}


%------ Encadrement ------

\usepackage{fancybox}


\newcommand{\mybox}[1]{
\setlength{\fboxsep}{7pt}
\begin{center}
\shadowbox{#1}
\end{center}}

\newcommand{\myboxinline}[1]{
\setlength{\fboxsep}{5pt}
\raisebox{-10pt}{
\shadowbox{#1}
}
}

%--------------- Commande beamer---------------
\newcommand{\beameronly}[1]{#1} % permet de mettre des pause dans beamer pas dans poly


\setbeamertemplate{navigation symbols}{}
\setbeamertemplate{footline}  % tiré du fichier beamerouterinfolines.sty
{
  \leavevmode%
  \hbox{%
  \begin{beamercolorbox}[wd=.333333\paperwidth,ht=2.25ex,dp=1ex,center]{author in head/foot}%
    % \usebeamerfont{author in head/foot}\insertshortauthor%~~(\insertshortinstitute)
    \usebeamerfont{section in head/foot}{\bf\insertshorttitle}
  \end{beamercolorbox}%
  \begin{beamercolorbox}[wd=.333333\paperwidth,ht=2.25ex,dp=1ex,center]{title in head/foot}%
    \usebeamerfont{section in head/foot}{\bf\insertsectionhead}
  \end{beamercolorbox}%
  \begin{beamercolorbox}[wd=.333333\paperwidth,ht=2.25ex,dp=1ex,right]{date in head/foot}%
    % \usebeamerfont{date in head/foot}\insertshortdate{}\hspace*{2em}
    \insertframenumber{} / \inserttotalframenumber\hspace*{2ex} 
  \end{beamercolorbox}}%
  \vskip0pt%
}


\definecolor{mygrey}{rgb}{0.5,0.5,0.5}
\setlength{\parindent}{0cm}
%\DeclareTextFontCommand{\helvetica}{\fontfamily{phv}\selectfont}

% background beamer
\definecolor{couleurhaut}{rgb}{0.85,0.9,1}  % creme
\definecolor{couleurmilieu}{rgb}{1,1,1}  % vert pale
\definecolor{couleurbas}{rgb}{0.85,0.9,1}  % blanc
\setbeamertemplate{background canvas}[vertical shading]%
[top=couleurhaut,middle=couleurmilieu,midpoint=0.4,bottom=couleurbas] 
%[top=fondtitre!05,bottom=fondtitre!60]



\makeatletter
\setbeamertemplate{theorem begin}
{%
  \begin{\inserttheoremblockenv}
  {%
    \inserttheoremheadfont
    \inserttheoremname
    \inserttheoremnumber
    \ifx\inserttheoremaddition\@empty\else\ (\inserttheoremaddition)\fi%
    \inserttheorempunctuation
  }%
}
\setbeamertemplate{theorem end}{\end{\inserttheoremblockenv}}

\newenvironment{theoreme}[1][]{%
   \setbeamercolor{block title}{fg=structure,bg=structure!40}
   \setbeamercolor{block body}{fg=black,bg=structure!10}
   \begin{block}{{\bf Th\'eor\`eme }#1}
}{%
   \end{block}%
}


\newenvironment{proposition}[1][]{%
   \setbeamercolor{block title}{fg=structure,bg=structure!40}
   \setbeamercolor{block body}{fg=black,bg=structure!10}
   \begin{block}{{\bf Proposition }#1}
}{%
   \end{block}%
}

\newenvironment{corollaire}[1][]{%
   \setbeamercolor{block title}{fg=structure,bg=structure!40}
   \setbeamercolor{block body}{fg=black,bg=structure!10}
   \begin{block}{{\bf Corollaire }#1}
}{%
   \end{block}%
}

\newenvironment{mydefinition}[1][]{%
   \setbeamercolor{block title}{fg=structure,bg=structure!40}
   \setbeamercolor{block body}{fg=black,bg=structure!10}
   \begin{block}{{\bf Définition} #1}
}{%
   \end{block}%
}

\newenvironment{lemme}[0]{%
   \setbeamercolor{block title}{fg=structure,bg=structure!40}
   \setbeamercolor{block body}{fg=black,bg=structure!10}
   \begin{block}{\bf Lemme}
}{%
   \end{block}%
}

\newenvironment{remarque}[1][]{%
   \setbeamercolor{block title}{fg=black,bg=structure!20}
   \setbeamercolor{block body}{fg=black,bg=structure!5}
   \begin{block}{Remarque #1}
}{%
   \end{block}%
}


\newenvironment{exemple}[1][]{%
   \setbeamercolor{block title}{fg=black,bg=structure!20}
   \setbeamercolor{block body}{fg=black,bg=structure!5}
   \begin{block}{{\bf Exemple }#1}
}{%
   \end{block}%
}


\newenvironment{miniexercice}[0]{%
   \setbeamercolor{block title}{fg=structure,bg=structure!20}
   \setbeamercolor{block body}{fg=black,bg=structure!5}
   \begin{block}{Mini-exercices}
}{%
   \end{block}%
}


\newenvironment{tp}[0]{%
   \setbeamercolor{block title}{fg=structure,bg=structure!40}
   \setbeamercolor{block body}{fg=black,bg=structure!10}
   \begin{block}{\bf Travaux pratiques}
}{%
   \end{block}%
}
\newenvironment{exercicecours}[1][]{%
   \setbeamercolor{block title}{fg=structure,bg=structure!40}
   \setbeamercolor{block body}{fg=black,bg=structure!10}
   \begin{block}{{\bf Exercice }#1}
}{%
   \end{block}%
}
\newenvironment{algo}[1][]{%
   \setbeamercolor{block title}{fg=structure,bg=structure!40}
   \setbeamercolor{block body}{fg=black,bg=structure!10}
   \begin{block}{{\bf Algorithme}\hfill{\color{gray}\texttt{#1}}}
}{%
   \end{block}%
}


\setbeamertemplate{proof begin}{
   \setbeamercolor{block title}{fg=black,bg=structure!20}
   \setbeamercolor{block body}{fg=black,bg=structure!5}
   \begin{block}{{\footnotesize Démonstration}}
   \footnotesize
   \smallskip}
\setbeamertemplate{proof end}{%
   \end{block}}
\setbeamertemplate{qed symbol}{\openbox}


\makeatother
\usecolortheme[RGB={102,51,102}]{structure}

%%%%%%%%%%%%%%%%%%%%%%%%%%%%%%%%%%%%%%%%%%%%%%%%%%%%%%%%%%%%%
%%%%%%%%%%%%%%%%%%%%%%%%%%%%%%%%%%%%%%%%%%%%%%%%%%%%%%%%%%%%%


\begin{document}


\title{{\bf Intégrales impropres}}
\subtitle{Intégration par parties - Changement de variables}

\begin{frame}
  
  \debutmontitre

  \pause

{\footnotesize
\hfill
\setbeamercovered{transparent=50}
\begin{minipage}{0.6\textwidth}
  \begin{itemize}
    \item<3-> Intégration par parties
    \item<4-> Changement de variable
    \item<5-> Plan d'étude
        \end{itemize}
\end{minipage}
}

\end{frame}

\setcounter{framenumber}{0}

%%%%%%%%%%%%%%%%%%%%%%%%%%%%%%%%%%%%%%%%%%%%%%%%%%%%%%%%%%%%%%%%
\section*{Intégration par parties}

\begin{frame}

\begin{theoreme}[Intégration par parties]
\uncover<2->{
\begin{itemize}
  \item Soient $u$ et $v$ deux fonctions de classe $\mathcal{C}^1$ sur l'intervalle $[a,+\infty[$

  \item Supposons que $\lim_{t\to+\infty} u(t)v(t)$ existe et soit finie
\end{itemize}

Les intégrales $\displaystyle\int_a^{+\infty} u(t) \, v'(t)\;\dd t$
et $\displaystyle\int_a^{+\infty} u'(t) \, v(t)\;\dd t$ sont de même nature.
En cas de convergence on a :
}
\alt<1>{
$$\int u(t) \, v'(t)\;\dd t = \big[uv\big] - \int u'(t) \, v(t)\;\dd t$$
}{
$$\int_{a}^{+\infty} u(t) \, v'(t)\;\dd t 
= \big[uv\big]_a^{+\infty} - \int_a^{+\infty} u'(t) \, v(t)\;\dd t$$
}
\vspace{-0.5cm}
\end{theoreme}

\pause\pause
\centerline{$\big[uv\big]_a^{+\infty} = \lim_{t\to+\infty} (uv)(t) - (uv)(a)$}
\pause
\begin{proof}
C'est la formule usuelle d'intégration par parties 
$$\int_a^{x} u(t) \, v'(t)\;\dd t 
= \big[uv\big]_a^{x} - \int_a^{x} u'(t) \, v(t)\;\dd t$$
\end{proof}
\end{frame}

\begin{frame}
\begin{exemple}
Que vaut l'espérance $\int_0^{+\infty} \lambda t e^{-\lambda t}\;\dd t$ de la loi exponentielle
 ($\lambda > 0$) ?

\pause

Intégration par parties : $u = \lambda t$, $v' = e^{-\lambda t}$
\pause
\vspace*{-1ex}
\begin{align*}
\int_0^{x} \lambda t e^{-\lambda t}\;\dd t 
 & = \int_0^{x} u(t) \, v'(t)\;\dd t  \\
\uncover<4->{ & = \big[uv\big]_0^{x} - \int_0^{x} u'(t) \, v(t)\;\dd t \\}
\uncover<5->{ & = \Big[ \lambda t \cdot \frac{-1}{\lambda}e^{-\lambda t}\Big]_0^{x} 
 - \int_0^{x} \lambda \cdot  \frac{-1}{\lambda}e^{-\lambda t}\;\dd t \\}
\uncover<6->{ & = -xe^{-\lambda x} + \int_0^{x} e^{-\lambda t} \;\dd t\\}
\uncover<7->{ & = -xe^{-\lambda x} + \Big[\frac{-1}{\lambda}e^{-\lambda t}\Big]_0^{x} \\}
\uncover<8->{ & = -xe^{-\lambda x}-\frac{1}{\lambda} \left(e^{-\lambda x} -1\right)} \uncover<9->{\xrightarrow{x \mapsto+\infty}  \frac{1}{\lambda}}
% & \longrightarrow \ \  \frac{1}{\lambda} \qquad \text{ lorsque } x \to +\infty \\
\end{align*}
\vspace*{-3ex}

\uncover<10->{
Ainsi $\int_0^{+\infty} \lambda t e^{-\lambda t}\;\dd t = \frac{1}{\lambda}$
}
\end{exemple}	
\end{frame}


%%%%%%%%%%%%%%%%%%%%%%%%%%%%%%%%%%%%%%%%%%%%%%%%%%%%%%%%%%%%%%%%
\section*{Changement de variable}

\begin{frame}
\begin{theoreme}[Changement de variable]
\uncover<2->{
\begin{itemize}
  \item Soit $f$ une fonction définie sur un intervalle $I = [a,+\infty[$
  \item Soit $J= [\alpha,\beta[$ un intervalle avec $\alpha\in \Rr$ et $\beta\in \Rr$ ou $\beta=+\infty$
  \item Soit $\varphi : J \to I$ un difféomorphisme de classe $\mathcal{C}^1$
\end{itemize}
Les intégrales $\int_{a}^{+\infty} f(x) \;\dd x$ et
$\int_\alpha^\beta f\big(\varphi(t)\big)\cdot\varphi'(t) \;\dd t$ sont de même nature.
En cas de convergence, on a :
}
$$\int_{a}^{+\infty} f(x) \;\dd x = \int_\alpha^\beta f\big(\varphi(t)\big)\cdot\varphi'(t) \;\dd t$$
\end{theoreme} 

\bigskip
\pause
\pause

Un \defi{difféomorphisme de classe $\mathcal{C}^1$}
est une application $\mathcal{C}^1$, bijective, 
dont la bijection réciproque est aussi $\mathcal{C}^1$





\end{frame}


\begin{frame}[t]

\begin{exemple}
L'intégrale de Fresnel  $\int_1^{+\infty} \sin (t^2) \; \dd t$ converge
\end{exemple}

\uncover<1->{
\only<1>{
\vspace*{1cm}
\myfigure{1.1}{
\tikzinput{fig_intimp06} 
} 
}

\pause\pause
\begin{itemize}
  \item Changement de variable $u=t^2$, qui induit 
$t=\sqrt{u}$,  $\dd t=\frac{\dd u}{2\sqrt u}$
\pause  
  \item $\varphi \!:\! u \mapsto t = \sqrt{u}$ est un difféomorphisme 
  entre $u\!\in\![1,\!x^2]$ et $t\!\in\![1,\!x]$
\pause  
 \item Théorème : $\int f(t) \;\dd t 
= \int f\big(\varphi(u)\big)\cdot\varphi'(u) \;\dd u$
\pause 
  \item D'où $\int_1^x \sin (t^2) \; \dd t = \int_1^{x^2} \sin (u) \; \frac{\dd u}{2\sqrt u}$
\pause    
  \item Par le théorème d'Abel $\int_1^{+\infty} \frac{\sin u}{\sqrt u} \; \dd u$
  converge
\pause
  \item Donc $\int_1^{x^2} \sin (u) \; \frac{\dd u}{2\sqrt u}$ admet une limite finie (lorsque 
$x\to+\infty$)
\pause
  \item Ce qui prouve que $\int_1^x \sin (t^2) \; \dd t$
admet aussi une limite finie
\pause
  \item Conclusion : $\int_1^{+\infty} \sin (t^2) \; \dd t$ converge
\end{itemize}
}

\end{frame}

%%%%%%%%%%%%%%%%%%%%%%%%%%%%%%%%%%%%%%%%%%%%%%%%%%%%%%%%%%%%%%%%
\section*{Plan d'étude}

\begin{frame}

\evidence{Plan d'étude}

\vspace*{-3ex}

\begin{minipage}{0.29\textwidth}
$$I = \int_{-\infty}^{+\infty} \frac{\sin |t|}{|t|^{3/2}}\;\dd t$$  
\end{minipage}
\begin{minipage}{0.69\textwidth}
\myfigure{0.5}{
\tikzinput{fig_intimp01} 
}  
\end{minipage}

	\pause

\evidence{Étape 1. Identifier les points incertains}\\
\pause
\begin{itemize}
  \item La fonction $|t|^{-3/2}\sin |t|$ est définie sur $\Rr^*$
  \pause
  \item La fonction est paire, elle tend vers $0$ en
oscillant quand $t$ tend vers $-\infty$ et $+\infty$
\pause
  \item Elle n'est pas définie en $0$ et
tend vers $+\infty$ en $0^-$ et en $0^+$
\pause
  \item $4$ points incertains : $-\infty$, $0^-$, $0^+$ et $+\infty$
\end{itemize}
\end{frame}



\begin{frame}

\evidence{Étape 2. Isoler les points incertains}\\
\pause
\begin{itemize}
  \item 4 intervalles : $]-\infty,-1]$, $[-1,0[$, $]0,1]$ et $[1,+\infty[$
  \pause
  \item 4 intégrales $I_1$, $I_2$, $I_3$ et $I_4$
  \pause
  \item L'intégrale $I$ ne converge que si chacun des
morceaux converge
\pause
  \item $I=I_1+I_2+I_3+I_4$
\end{itemize}

\bigskip
\pause 
\evidence{Étape 3. Se ramener à une intégrale sur $[a,+\infty[$ ou sur $]a,b]$}\\
\pause
\begin{itemize}
  \item Changement de variable $t\mapsto -t$
  \pause
  \item Notre exemple : puisque la fonction est paire,
$I_1=I_4$ et $I_2=I_3$
\end{itemize}

\end{frame}



\begin{frame}

\evidence{Étape 4.1. Si c'est possible, calculer une primitive}\\

\pause
\evidence{Étape 4.2. Si la fonction est de signe constant}\\
\pause
\vspace*{-1ex}
\begin{itemize}
  \item Utiliser le théorème des équivalents ou le théorème de comparaison
  \pause
  \item Exemple 
  \pause
  \begin{itemize}
    \item $t^{-3/2} \sin |t|  \;\underset{0^+}{\sim}\; t^{-1/2}$
    \pause
    \item $\int_0^1 \frac{1}{t^{1/2}}\;\dd t$ converge (intégrale de Riemann)
    \pause
    \item donc $I_3 = \int_0^1 t^{-3/2} \sin |t| \;\dd t$ converge
  \end{itemize}

\end{itemize}

\pause 

\evidence{Étape 4.3. Si la fonction n'est pas de signe constant}\\
\vspace*{-1ex}
\pause
\begin{itemize}
  \item \'Etude de l'intégrale de $|f|$. S'il y a convergence absolue, il y a convergence
  \pause
  \item Si l'intégrale n'est pas absolument convergente, il
faut essayer d'appliquer le théorème d'Abel
\pause
  \item Exemple 
  \pause
  \begin{itemize}
    \item $t^{-3/2} \big|\sin |t| \big|\le t^{-3/2}$
    \pause
    \item $\int_1^{+\infty} \frac{1}{t^{3/2}}\;\dd t$ converge (intégrale de Riemann)
    \pause
    \item donc $I_4=\int_{1}^{+\infty} |t|^{-3/2} \sin |t| \;\dd t$ est absolument convergente
  \end{itemize} 
\end{itemize}
\pause
Conclusion : $I = \int_{-\infty}^{+\infty} \frac{\sin |t|}{|t|^{3/2}}\;\dd t$ converge
\end{frame}



%%%%%%%%%%%%%%%%%%%%%%%%%%%%%%%%%%%%%%%%%%%%%%%%%%%%%%%%%%%%%%%
\section*{Mini-exercices}

\begin{frame}
\begin{miniexercice}
\begin{enumerate}
  \item Soit $I_n = \int_0^{+\infty} t^n e^{-t}\;\dd t$. 
  Calculer $I_0$. Par intégration par parties, trouver une relation de 
  récurrence entre $I_{n+1}$ et $I_n$. En déduire que $I_n = n!$\ .
  
  \item Par intégration par parties, montrer que l'intégrale
  $\int_0^1 \frac{\ln t}{(1+t)^2}\; \dd t$ converge et vaut $-\ln 2$.
  
  \item Soit $I= \int_0^{+\infty} \ln(1+\frac{1}{t^2}) \;\dd t$. Identifier les points incertains.
  Faire une intégration par parties en écrivant $1\cdot \ln(1+\frac{1}{t^2})$. Montrer que $I$ converge
  et que $I = \pi$. 
  
  \item Soit $I=\int_0^1 \left(-\ln t\right)^\beta \;\dd t$.
  Identifier les points incertains. 
  \`A l'aide du changement de variable $t=\frac 1u$, montrer que cette intégrale
  converge quel que soit $\beta \in \Rr$.
  
  \item Soit $I=\int_0^{+\infty}\frac{\dd t}{t^2+\sqrt t}$.
  Identifier les points incertains. 
  \`A l'aide du changement de variable $t=u^2$, montrer que l'intégrale
  converge et que $I = \frac{4\sqrt3\pi}{9}$.


\end{enumerate}
\end{miniexercice}	
\end{frame}




\end{document}

