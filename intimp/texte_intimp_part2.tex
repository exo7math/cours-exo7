
%%%%%%%%%%%%%%%%%% PREAMBULE %%%%%%%%%%%%%%%%%%


\documentclass[12pt]{article}

\usepackage{amsfonts,amsmath,amssymb,amsthm}
\usepackage[utf8]{inputenc}
\usepackage[T1]{fontenc}
\usepackage[francais]{babel}


% packages
\usepackage{amsfonts,amsmath,amssymb,amsthm}
\usepackage[utf8]{inputenc}
\usepackage[T1]{fontenc}
%\usepackage{lmodern}

\usepackage[francais]{babel}
\usepackage{fancybox}
\usepackage{graphicx}

\usepackage{float}

%\usepackage[usenames, x11names]{xcolor}
\usepackage{tikz}
\usepackage{datetime}

\usepackage{mathptmx}
%\usepackage{fouriernc}
%\usepackage{newcent}
\usepackage[mathcal,mathbf]{euler}

%\usepackage{palatino}
%\usepackage{newcent}


% Commande spéciale prompteur

%\usepackage{mathptmx}
%\usepackage[mathcal,mathbf]{euler}
%\usepackage{mathpple,multido}

\usepackage[a4paper]{geometry}
\geometry{top=2cm, bottom=2cm, left=1cm, right=1cm, marginparsep=1cm}

\newcommand{\change}{{\color{red}\rule{\textwidth}{1mm}\\}}

\newcounter{mydiapo}

\newcommand{\diapo}{\newpage
\hfill {\normalsize  Diapo \themydiapo \quad \texttt{[\jobname]}} \\
\stepcounter{mydiapo}}


%%%%%%% COULEURS %%%%%%%%%%

% Pour blanc sur noir :
%\pagecolor[rgb]{0.5,0.5,0.5}
% \pagecolor[rgb]{0,0,0}
% \color[rgb]{1,1,1}



%\DeclareFixedFont{\myfont}{U}{cmss}{bx}{n}{18pt}
\newcommand{\debuttexte}{
%%%%%%%%%%%%% FONTES %%%%%%%%%%%%%
\renewcommand{\baselinestretch}{1.5}
\usefont{U}{cmss}{bx}{n}
\bfseries

% Taille normale : commenter le reste !
%Taille Arnaud
%\fontsize{19}{19}\selectfont

% Taille Barbara
%\fontsize{21}{22}\selectfont

%Taille François
\fontsize{25}{30}\selectfont

%Taille Pascal
%\fontsize{25}{30}\selectfont

%Taille Laura
%\fontsize{30}{35}\selectfont


%\myfont
%\usefont{U}{cmss}{bx}{n}

%\Huge
%\addtolength{\parskip}{\baselineskip}
}


% \usepackage{hyperref}
% \hypersetup{colorlinks=true, linkcolor=blue, urlcolor=blue,
% pdftitle={Exo7 - Exercices de mathématiques}, pdfauthor={Exo7}}


%section
% \usepackage{sectsty}
% \allsectionsfont{\bf}
%\sectionfont{\color{Tomato3}\upshape\selectfont}
%\subsectionfont{\color{Tomato4}\upshape\selectfont}

%----- Ensembles : entiers, reels, complexes -----
\newcommand{\Nn}{\mathbb{N}} \newcommand{\N}{\mathbb{N}}
\newcommand{\Zz}{\mathbb{Z}} \newcommand{\Z}{\mathbb{Z}}
\newcommand{\Qq}{\mathbb{Q}} \newcommand{\Q}{\mathbb{Q}}
\newcommand{\Rr}{\mathbb{R}} \newcommand{\R}{\mathbb{R}}
\newcommand{\Cc}{\mathbb{C}} 
\newcommand{\Kk}{\mathbb{K}} \newcommand{\K}{\mathbb{K}}

%----- Modifications de symboles -----
\renewcommand{\epsilon}{\varepsilon}
\renewcommand{\Re}{\mathop{\text{Re}}\nolimits}
\renewcommand{\Im}{\mathop{\text{Im}}\nolimits}
%\newcommand{\llbracket}{\left[\kern-0.15em\left[}
%\newcommand{\rrbracket}{\right]\kern-0.15em\right]}

\renewcommand{\ge}{\geqslant}
\renewcommand{\geq}{\geqslant}
\renewcommand{\le}{\leqslant}
\renewcommand{\leq}{\leqslant}

%----- Fonctions usuelles -----
\newcommand{\ch}{\mathop{\mathrm{ch}}\nolimits}
\newcommand{\sh}{\mathop{\mathrm{sh}}\nolimits}
\renewcommand{\tanh}{\mathop{\mathrm{th}}\nolimits}
\newcommand{\cotan}{\mathop{\mathrm{cotan}}\nolimits}
\newcommand{\Arcsin}{\mathop{\mathrm{Arcsin}}\nolimits}
\newcommand{\Arccos}{\mathop{\mathrm{Arccos}}\nolimits}
\newcommand{\Arctan}{\mathop{\mathrm{Arctan}}\nolimits}
\newcommand{\Argsh}{\mathop{\mathrm{Argsh}}\nolimits}
\newcommand{\Argch}{\mathop{\mathrm{Argch}}\nolimits}
\newcommand{\Argth}{\mathop{\mathrm{Argth}}\nolimits}
\newcommand{\pgcd}{\mathop{\mathrm{pgcd}}\nolimits} 

\newcommand{\Card}{\mathop{\text{Card}}\nolimits}
\newcommand{\Ker}{\mathop{\text{Ker}}\nolimits}
\newcommand{\id}{\mathop{\text{id}}\nolimits}
\newcommand{\ii}{\mathrm{i}}
\newcommand{\dd}{\mathrm{d}}
\newcommand{\Vect}{\mathop{\text{Vect}}\nolimits}
\newcommand{\Mat}{\mathop{\mathrm{Mat}}\nolimits}
\newcommand{\rg}{\mathop{\text{rg}}\nolimits}
\newcommand{\tr}{\mathop{\text{tr}}\nolimits}
\newcommand{\ppcm}{\mathop{\text{ppcm}}\nolimits}

%----- Structure des exercices ------

\newtheoremstyle{styleexo}% name
{2ex}% Space above
{3ex}% Space below
{}% Body font
{}% Indent amount 1
{\bfseries} % Theorem head font
{}% Punctuation after theorem head
{\newline}% Space after theorem head 2
{}% Theorem head spec (can be left empty, meaning ‘normal’)

%\theoremstyle{styleexo}
\newtheorem{exo}{Exercice}
\newtheorem{ind}{Indications}
\newtheorem{cor}{Correction}


\newcommand{\exercice}[1]{} \newcommand{\finexercice}{}
%\newcommand{\exercice}[1]{{\tiny\texttt{#1}}\vspace{-2ex}} % pour afficher le numero absolu, l'auteur...
\newcommand{\enonce}{\begin{exo}} \newcommand{\finenonce}{\end{exo}}
\newcommand{\indication}{\begin{ind}} \newcommand{\finindication}{\end{ind}}
\newcommand{\correction}{\begin{cor}} \newcommand{\fincorrection}{\end{cor}}

\newcommand{\noindication}{\stepcounter{ind}}
\newcommand{\nocorrection}{\stepcounter{cor}}

\newcommand{\fiche}[1]{} \newcommand{\finfiche}{}
\newcommand{\titre}[1]{\centerline{\large \bf #1}}
\newcommand{\addcommand}[1]{}
\newcommand{\video}[1]{}

% Marge
\newcommand{\mymargin}[1]{\marginpar{{\small #1}}}



%----- Presentation ------
\setlength{\parindent}{0cm}

%\newcommand{\ExoSept}{\href{http://exo7.emath.fr}{\textbf{\textsf{Exo7}}}}

\definecolor{myred}{rgb}{0.93,0.26,0}
\definecolor{myorange}{rgb}{0.97,0.58,0}
\definecolor{myyellow}{rgb}{1,0.86,0}

\newcommand{\LogoExoSept}[1]{  % input : echelle
{\usefont{U}{cmss}{bx}{n}
\begin{tikzpicture}[scale=0.1*#1,transform shape]
  \fill[color=myorange] (0,0)--(4,0)--(4,-4)--(0,-4)--cycle;
  \fill[color=myred] (0,0)--(0,3)--(-3,3)--(-3,0)--cycle;
  \fill[color=myyellow] (4,0)--(7,4)--(3,7)--(0,3)--cycle;
  \node[scale=5] at (3.5,3.5) {Exo7};
\end{tikzpicture}}
}



\theoremstyle{definition}
%\newtheorem{proposition}{Proposition}
%\newtheorem{exemple}{Exemple}
%\newtheorem{theoreme}{Théorème}
\newtheorem{lemme}{Lemme}
\newtheorem{corollaire}{Corollaire}
%\newtheorem*{remarque*}{Remarque}
%\newtheorem*{miniexercice}{Mini-exercices}
%\newtheorem{definition}{Définition}




%definition d'un terme
\newcommand{\defi}[1]{{\color{myorange}\textbf{\emph{#1}}}}
\newcommand{\evidence}[1]{{\color{blue}\textbf{\emph{#1}}}}



 %----- Commandes divers ------

\newcommand{\codeinline}[1]{\texttt{#1}}

%%%%%%%%%%%%%%%%%%%%%%%%%%%%%%%%%%%%%%%%%%%%%%%%%%%%%%%%%%%%%
%%%%%%%%%%%%%%%%%%%%%%%%%%%%%%%%%%%%%%%%%%%%%%%%%%%%%%%%%%%%%



\begin{document}

\debuttexte


%%%%%%%%%%%%%%%%%%%%%%%%%%%%%%%%%%%%%%%%%%%%%%%%%%%%%%%%%%%
\diapo

Dans cette partie, nous nous intéressons aux intégrales 
impropres des fonctions de signe constant. \\
On se concentrera sur le cas des fonctions positives, les fonctions négatives 
se traitant de la même manière.

  \change
  
  \change

  Nous commençons par un théorème de comparaison entre 
  deux intégrales impropres de fonctions positives,

  \change

puis nous démontrons un autre théorème qui liera 
les intégrales de deux fonctions équivalentes.
  
\change

  Enfin, nous examinons un exemple de référence :
  les intégrales impropres dites de Riemann

\change

et une variante : les intégrales impropres de Bertrand.

%%%%%%%%%%%%%%%%%%%%%%%%%%%%%%%%%%%%%%%%%%%%%%%%%%%%%%%%%%%
\diapo

Si on ne peut pas calculer une primitive de $f$, 
on étudie la convergence en
comparant avec des intégrales dont la convergence est connue,
grâce au théorème suivant.

 \change

 Dans ce théorème, on considère deux fonctions $f$ et $g$
 continues et *positives*, telles que $f$ est majorée par $g$ 
 au voisinage de $+\infty$, 
 ce qui signifie $\exists A\ge a \quad \forall t>A \qquad f(t)\le g(t)$
 
 \change
 
 On s'intéresse aux intégrales impropres de ces fonctions 
 sur un intervalle du type $[a,+\infty[$.

 Alors si l'intégrale de $g$ converge, 
 il en est de même de l'intégrale de $f$

 \change
 on a aussi que si l'intégrale de $f$ diverge, 
 il en est de même de l'intégrale de $g$.

 \change
La preuve est assez simple.
Tout d'abord, la convergence des intégrales ne dépend pas de la borne 
de gauche de l'intervalle, et nous pouvons nous contenter d'étudier 
$\int_A^x f(t)\;\dd t$ et $\int_A^x g(t)\;\dd t$.

Enfin d'après la positivité de l'intégrale, l'intégrale de $A$ à $x$ 
de $f$ est majorée par celle de $g$.

\change
On remarque aussi que l'intégrale de $A$ à $x$ de $f$ est une 
fonction croissante de $x$, par positivité de la fonction $f(x)$
et positivité de l'intégrale.

Donc si l'intégrale de $g$ converge, l'intégrale de $A$ à $x$ 
de $f$ est une fonction croissante et majorée par 
l'intégrale de $g$ de $A$ à $+\infty$, et donc converge. 
Ce qui signifie précisément que  l'intégrale de $f$ converge

\change
Inversement, si $\int_A^{x} f(t)\;\dd t$ tend vers $+\infty$, 
alors $\int_A^{x} g(t)\;\dd t$, qui lui est supérieure,
tend aussi vers $+\infty$.
 
%%%%%%%%%%%%%%%%%%%%%%%%%%%%%%%%%%%%%%%%%%%%%%%%%%%%%%%%%%%
\diapo

Voici une application typique du théorème de comparaison.

Montrons que l'intégrale 
$$\int_1^{+\infty} t^\alpha e^{-t}\;\dd t\quad\text{ converge,}$$
quel que soit le réel $\alpha$. 
\\

On ne connait pas de primitive de $t^\alpha e^{-t}$, on va donc 
majorer cette fonction positive par une autre plus simple à 
intégrer, et dont l'intégrale converge. 

\change
On choisit de majorer par la fonction $e^{-t/2}$, pour cela on 
commence par écrire : $t^\alpha e^{-t} = t^\alpha e^{-t/2}\,e^{-t/2}$.
  
\change

On sait que $\lim_{t\rightarrow+\infty} t^\alpha e^{-t/2} =0$, 
pour tout $\alpha$,  car l'exponentielle "l'emporte" sur les 
puissances.

\change

Ainsi $t^\alpha e^{-t/2}$ est aussi petit que l'on veut et par exemple 
il existe un réel $A>0$ tel que :
$$\forall t>A\qquad t^\alpha e^{-t/2}\le 1\;.$$ 

\change

En multipliant par $e^{-t/2}$ on obtient :
$$\forall t>A\qquad t^\alpha e^{-t}\le e^{-t/2}\;.$$

\change

Mais l'intégrale $\int_1^{+\infty} e^{-t/2}\;\dd t$ converge.

En effet :
$\int_1^x e^{-t/2}\;\dd t$

\change
se calcule par la primitive :
$= \left[-2 e^{-t/2}\right]_1^x$

\change
donc $= 2e^{-1/2} -2e^{-x/2}$

\change
Lorsque $x\rightarrow+\infty$
la limite est $2e^{-1/2}$ 
\hfill (.../...)

\change
Cela prouve que, comme on l'a dit, l'intégrale 
$\int_1^{+\infty} e^{-t/2}\;\dd t$ converge.

\change  
On peut à présent appliquer le théorème  de comparaison :

(1) la fonction $t^\alpha e^{-t}$ est plus petite que
$e^{-t/2}$ pour $t$ assez grand ;

(2) $\int_1^{+\infty} e^{-t/2}\;\dd t$ converge

Conclusion : 
$\int_1^{+\infty} t^\alpha e^{-t}\;\dd t$ converge aussi.

%%%%%%%%%%%%%%%%%%%%%%%%%%%%%%%%%%%%%%%%%%%%%%%%%%%%%%%%%%%
\diapo



Pour le théorème des équivalents, considérons $f$ et $g$ 
deux fonctions continues strictement positives sur $[a,+\infty[$.
On suppose que $f$ et $g$  sont équivalentes en $+\infty$, 
ce qui signifie simplement que $\lim_{t\rightarrow+\infty}\frac{f(t)}{g(t)} = 1$.

\change
Alors les intégrales de $f$ et $g$ sont de même nature.

C'est un théorème très utile qui permet de remplacer la fonction 
à intégrer par un équivalent, souvent lus simple, au voisinage de $+\infty$ 
pour étudier la convergence d'une intégrale.

\change

La preuve consiste à traduire l'hypothèse que $f$ et $g$ 
sont équivalentes en $+\infty$, 

ainsi $\lim_{t\rightarrow+\infty}\frac{f(t)}{g(t)} = 1$, 

\change
cela fournit un l'encadrement 

$$(1-\epsilon)g(t)<f(t)<(1+\epsilon)g(t)\;.$$

valable pour $t$ assez grand.

\change
On a le choix de $\epsilon>0$, et on prend $0<\epsilon<1$ arbitraire. 

Alors cette double inégalité et le théorème de comparaison permettent de conclure : 

si $\int g$ converge alors par cette inégalité $\int f$ converge aussi.
Par contre si $\int g$ diverge alors par cette inégalité $\int f$ diverge. 
\\
Les deux intégrales ont donc la même nature.
\hfill (.../...)

\change
\newpage

Voici un exemple : 

$\frac{t^5+3t+1}{t^3+4}e^{-t}$

\change

le numérateur de la fraction est équivalent à $t^5$, le dénominateur à $t^3$,
donc la fonction équivaut à $t^2e^{-t}$ au voisinage de $+\infty$.

\change
Mais on a déjà vu que l'intégrale 
$\int_1^{+\infty} t^2e^{-t}\;\dd t$ converge, 

\change
on conclut que $\int_1^{+\infty} \frac{t^5+3t+1}{t^3+4}e^{-t}\;\dd t$ 
converge aussi.


%%%%%%%%%%%%%%%%%%%%%%%%%%%%%%%%%%%%%%%%%%%%%%%%%%%%%%%%%%%
\diapo

Pour l'étude de la convergence d'une intégrale pour laquelle on n'a pas de primitive, 
l'utilisation des équivalents permet de se ramener 
à un catalogue d'intégrales dont la nature est connue. 

Commençons par l'une des plus importante : les intégrales de Riemann.

Une \defi{intégrale de Riemann} est une intégrale de la forme 
$$\int_1^{+\infty} \frac{1}{t^{\alpha}}\;\dd t$$\\
où $\alpha$ est un paramètre strictement positif.

\change
La convergence dépend de $\alpha$.
$$\text{Si } \quad \alpha > 1\quad \text{ alors }\quad 
\int_1^{+\infty} \frac{1}{t^{\alpha}}\;\dd t \quad\text{ converge.}$$

\change
$$\text{Si } \quad \alpha\le 1\quad \text{ alors }\quad 
\int_1^{+\infty} \frac{1}{t^{\alpha}}\;\dd t \quad\text{ diverge.}$$


\change
Il est essentiel de retenir ce résultat.

\change
La preuve est très simple : on dispose d'une primitive 
explicite de $t^\alpha$, 

\change
Pour $\alpha\neq 1$, une primitive de $\frac{1}{t^{\alpha}}$ est $\tfrac{1}{-\alpha+1}\frac{1}{t^{\alpha-1}}$,
donc si $\alpha >1$, le crochet admet une limite finie, donc l'intégrale de Riemann converge.
Par contre si $\alpha<1$  le crochet a une limite infinie, donc l'intégrale diverge.

\change
Le cas $\alpha =1$ se traite à part :
une primitive $1/t$ est $\ln t$, et le crochet tend vers $+\infty$, donc 
l'intégrale diverge.



%%%%%%%%%%%%%%%%%%%%%%%%%%%%%%%%%%%%%%%%%%%%%%%%%%%%%%%%%%%
\diapo

Poursuivons avec les intégrales de Bertrand.

Une \defi{intégrale de Bertrand} est  du type
$$\int_2^{+\infty} \frac{1}{t\;(\ln t)^{\beta}}\;\dd t$$
où $\beta \in \Rr$.

\change
Si $\beta > 1$ alors  l'intégrale converge.

\change
Si $\beta \le 1$ alors  l'intégrale diverge.

\change

\change
Encore une fois on dispose d'une primitive explicite de 
$t\;(\ln t)^{\beta}$, 

\change
en distinguant le cas $\beta \neq 1$,

\change
et le cas $\beta=1$,

ce qui aboutit à la règle que je viens d'énoncer.

%%%%%%%%%%%%%%%%%%%%%%%%%%%%%%%%%%%%%%%%%%%%%%%%%%%%%%%%%%%
\diapo

Voici un exemple d'application :
on souhaite déterminer la nature de 
cette intégrale, pour laquelle le seul point incertain est $+\infty$.

Pour répondre au problème, calculons un équivalent de la fonction au voisinage de $+\infty$.
Pour cela, on calcule un équivalent de chacun des trois facteurs :
%donc du produit, car .

\change
$\sqrt{t^2+3t}$

\change
$= t\sqrt{1+\frac{3}{t}}$

\change
$\mathop{\sim}_{+\infty}\quad   t$


\change
Pour $\ln\left(\cos\frac{1}{t}\right)$ on fait un DL du cosinus

\change
ce qui donne $\ln\left(1-\frac{1}{2t^2}+o\left(\frac{1}{t^2}\right)\right)$

\change
et par un DL du logarithme
$\mathop{\sim}_{+\infty}\quad  -\frac{1}{2t^2}$

\change
Enfin 
$\sin^2\left(\frac{1}{\ln t}\right)$

\change
$\mathop{\sim}_{+\infty}\quad \left(\frac{1}{\ln t}\right)^2$

\change
Les équivalents sont multiplicatifs
donc la fonction à étudier équivaut en $+\infty$ à $-\frac{1}{2t\;(\ln t)^2}$


\change
Remarquons que dans cette équivalence les deux fonctions sont négatives 
au voisinage de $+\infty$. 
D'après le théorème des équivalents,
les deux intégrales associées sont de même nature.
Mais comme cette intégrale est du type intégrale de Bertrand 
avec $\beta=2$ donc converge, 
alors notre intégrale initiale est aussi convergente.


%%%%%%%%%%%%%%%%%%%%%%%%%%%%%%%%%%%%%%%%%%%%%%%%%%%%%%%%%%%
\diapo

Voici quelques exercices pour vous entraîner 
sur les intégrales impropres des fonctions de signe constant.

\end{document}
