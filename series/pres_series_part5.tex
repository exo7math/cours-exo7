
%%%%%%%%%%%%%%%%%% PREAMBULE %%%%%%%%%%%%%%%%%%

\documentclass[aspectratio=169,utf8]{beamer}
%\documentclass[aspectratio=169,handout]{beamer}

\usetheme{Boadilla}
%\usecolortheme{seahorse}
\usecolortheme[RGB={245,66,24}]{structure}
\useoutertheme{infolines}

% packages
\usepackage{amsfonts,amsmath,amssymb,amsthm}
\usepackage[utf8]{inputenc}
\usepackage[T1]{fontenc}
\usepackage{lmodern}

\usepackage[francais]{babel}
\usepackage{fancybox}
\usepackage{graphicx}

\usepackage{float}
\usepackage{xfrac}

%\usepackage[usenames, x11names]{xcolor}
\usepackage{tikz}
\usepackage{pgfplots}
\usepackage{datetime}



%-----  Package unités -----
\usepackage{siunitx}
\sisetup{locale = FR,detect-all,per-mode = symbol}

%\usepackage{mathptmx}
%\usepackage{fouriernc}
%\usepackage{newcent}
%\usepackage[mathcal,mathbf]{euler}

%\usepackage{palatino}
%\usepackage{newcent}
% \usepackage[mathcal,mathbf]{euler}



% \usepackage{hyperref}
% \hypersetup{colorlinks=true, linkcolor=blue, urlcolor=blue,
% pdftitle={Exo7 - Exercices de mathématiques}, pdfauthor={Exo7}}


%section
% \usepackage{sectsty}
% \allsectionsfont{\bf}
%\sectionfont{\color{Tomato3}\upshape\selectfont}
%\subsectionfont{\color{Tomato4}\upshape\selectfont}

%----- Ensembles : entiers, reels, complexes -----
\newcommand{\Nn}{\mathbb{N}} \newcommand{\N}{\mathbb{N}}
\newcommand{\Zz}{\mathbb{Z}} \newcommand{\Z}{\mathbb{Z}}
\newcommand{\Qq}{\mathbb{Q}} \newcommand{\Q}{\mathbb{Q}}
\newcommand{\Rr}{\mathbb{R}} \newcommand{\R}{\mathbb{R}}
\newcommand{\Cc}{\mathbb{C}} 
\newcommand{\Kk}{\mathbb{K}} \newcommand{\K}{\mathbb{K}}

%----- Modifications de symboles -----
\renewcommand{\epsilon}{\varepsilon}
\renewcommand{\Re}{\mathop{\text{Re}}\nolimits}
\renewcommand{\Im}{\mathop{\text{Im}}\nolimits}
%\newcommand{\llbracket}{\left[\kern-0.15em\left[}
%\newcommand{\rrbracket}{\right]\kern-0.15em\right]}

\renewcommand{\ge}{\geqslant}
\renewcommand{\geq}{\geqslant}
\renewcommand{\le}{\leqslant}
\renewcommand{\leq}{\leqslant}
\renewcommand{\epsilon}{\varepsilon}

%----- Fonctions usuelles -----
\newcommand{\ch}{\mathop{\text{ch}}\nolimits}
\newcommand{\sh}{\mathop{\text{sh}}\nolimits}
\renewcommand{\tanh}{\mathop{\text{th}}\nolimits}
\newcommand{\cotan}{\mathop{\text{cotan}}\nolimits}
\newcommand{\Arcsin}{\mathop{\text{arcsin}}\nolimits}
\newcommand{\Arccos}{\mathop{\text{arccos}}\nolimits}
\newcommand{\Arctan}{\mathop{\text{arctan}}\nolimits}
\newcommand{\Argsh}{\mathop{\text{argsh}}\nolimits}
\newcommand{\Argch}{\mathop{\text{argch}}\nolimits}
\newcommand{\Argth}{\mathop{\text{argth}}\nolimits}
\newcommand{\pgcd}{\mathop{\text{pgcd}}\nolimits} 


%----- Commandes divers ------
\newcommand{\ii}{\mathrm{i}}
\newcommand{\dd}{\text{d}}
\newcommand{\id}{\mathop{\text{id}}\nolimits}
\newcommand{\Ker}{\mathop{\text{Ker}}\nolimits}
\newcommand{\Card}{\mathop{\text{Card}}\nolimits}
\newcommand{\Vect}{\mathop{\text{Vect}}\nolimits}
\newcommand{\Mat}{\mathop{\text{Mat}}\nolimits}
\newcommand{\rg}{\mathop{\text{rg}}\nolimits}
\newcommand{\tr}{\mathop{\text{tr}}\nolimits}


%----- Structure des exercices ------

\newtheoremstyle{styleexo}% name
{2ex}% Space above
{3ex}% Space below
{}% Body font
{}% Indent amount 1
{\bfseries} % Theorem head font
{}% Punctuation after theorem head
{\newline}% Space after theorem head 2
{}% Theorem head spec (can be left empty, meaning ‘normal’)

%\theoremstyle{styleexo}
\newtheorem{exo}{Exercice}
\newtheorem{ind}{Indications}
\newtheorem{cor}{Correction}


\newcommand{\exercice}[1]{} \newcommand{\finexercice}{}
%\newcommand{\exercice}[1]{{\tiny\texttt{#1}}\vspace{-2ex}} % pour afficher le numero absolu, l'auteur...
\newcommand{\enonce}{\begin{exo}} \newcommand{\finenonce}{\end{exo}}
\newcommand{\indication}{\begin{ind}} \newcommand{\finindication}{\end{ind}}
\newcommand{\correction}{\begin{cor}} \newcommand{\fincorrection}{\end{cor}}

\newcommand{\noindication}{\stepcounter{ind}}
\newcommand{\nocorrection}{\stepcounter{cor}}

\newcommand{\fiche}[1]{} \newcommand{\finfiche}{}
\newcommand{\titre}[1]{\centerline{\large \bf #1}}
\newcommand{\addcommand}[1]{}
\newcommand{\video}[1]{}

% Marge
\newcommand{\mymargin}[1]{\marginpar{{\small #1}}}

\def\noqed{\renewcommand{\qedsymbol}{}}


%----- Presentation ------
\setlength{\parindent}{0cm}

%\newcommand{\ExoSept}{\href{http://exo7.emath.fr}{\textbf{\textsf{Exo7}}}}

\definecolor{myred}{rgb}{0.93,0.26,0}
\definecolor{myorange}{rgb}{0.97,0.58,0}
\definecolor{myyellow}{rgb}{1,0.86,0}

\newcommand{\LogoExoSept}[1]{  % input : echelle
{\usefont{U}{cmss}{bx}{n}
\begin{tikzpicture}[scale=0.1*#1,transform shape]
  \fill[color=myorange] (0,0)--(4,0)--(4,-4)--(0,-4)--cycle;
  \fill[color=myred] (0,0)--(0,3)--(-3,3)--(-3,0)--cycle;
  \fill[color=myyellow] (4,0)--(7,4)--(3,7)--(0,3)--cycle;
  \node[scale=5] at (3.5,3.5) {Exo7};
\end{tikzpicture}}
}


\newcommand{\debutmontitre}{
  \author{} \date{} 
  \thispagestyle{empty}
  \hspace*{-10ex}
  \begin{minipage}{\textwidth}
    \titlepage  
  \vspace*{-2.5cm}
  \begin{center}
    \LogoExoSept{2.5}
  \end{center}
  \end{minipage}

  \vspace*{-0cm}
  
  % Astuce pour que le background ne soit pas discrétisé lors de la conversion pdf -> png
\begin{tikzpicture}
        \fill[opacity=0,green!60!black] (0,0)--++(0,0)--++(0,0)--++(0,0)--cycle; 
\end{tikzpicture}

% toc S'affiche trop tot :
% \tableofcontents[hideallsubsections, pausesections]
}

\newcommand{\finmontitre}{
  \end{frame}
  \setcounter{framenumber}{0}
} % ne marche pas pour une raison obscure

%----- Commandes supplementaires ------

% \usepackage[landscape]{geometry}
% \geometry{top=1cm, bottom=3cm, left=2cm, right=10cm, marginparsep=1cm
% }
% \usepackage[a4paper]{geometry}
% \geometry{top=2cm, bottom=2cm, left=2cm, right=2cm, marginparsep=1cm
% }

%\usepackage{standalone}


% New command Arnaud -- november 2011
\setbeamersize{text margin left=24ex}
% si vous modifier cette valeur il faut aussi
% modifier le decalage du titre pour compenser
% (ex : ici =+10ex, titre =-5ex

\theoremstyle{definition}
%\newtheorem{proposition}{Proposition}
%\newtheorem{exemple}{Exemple}
%\newtheorem{theoreme}{Théorème}
%\newtheorem{lemme}{Lemme}
%\newtheorem{corollaire}{Corollaire}
%\newtheorem*{remarque*}{Remarque}
%\newtheorem*{miniexercice}{Mini-exercices}
%\newtheorem{definition}{Définition}

% Commande tikz
\usetikzlibrary{calc}
\usetikzlibrary{patterns,arrows}
\usetikzlibrary{matrix}
\usetikzlibrary{fadings} 

%definition d'un terme
\newcommand{\defi}[1]{{\color{myorange}\textbf{\emph{#1}}}}
\newcommand{\evidence}[1]{{\color{blue}\textbf{\emph{#1}}}}
\newcommand{\assertion}[1]{\emph{\og#1\fg}}  % pour chapitre logique
%\renewcommand{\contentsname}{Sommaire}
\renewcommand{\contentsname}{}
\setcounter{tocdepth}{2}



%------ Figures ------

\def\myscale{1} % par défaut 
\newcommand{\myfigure}[2]{  % entrée : echelle, fichier figure
\def\myscale{#1}
\begin{center}
\footnotesize
{#2}
\end{center}}


%------ Encadrement ------

\usepackage{fancybox}


\newcommand{\mybox}[1]{
\setlength{\fboxsep}{7pt}
\begin{center}
\shadowbox{#1}
\end{center}}

\newcommand{\myboxinline}[1]{
\setlength{\fboxsep}{5pt}
\raisebox{-10pt}{
\shadowbox{#1}
}
}

%--------------- Commande beamer---------------
\newcommand{\beameronly}[1]{#1} % permet de mettre des pause dans beamer pas dans poly


\setbeamertemplate{navigation symbols}{}
\setbeamertemplate{footline}  % tiré du fichier beamerouterinfolines.sty
{
  \leavevmode%
  \hbox{%
  \begin{beamercolorbox}[wd=.333333\paperwidth,ht=2.25ex,dp=1ex,center]{author in head/foot}%
    % \usebeamerfont{author in head/foot}\insertshortauthor%~~(\insertshortinstitute)
    \usebeamerfont{section in head/foot}{\bf\insertshorttitle}
  \end{beamercolorbox}%
  \begin{beamercolorbox}[wd=.333333\paperwidth,ht=2.25ex,dp=1ex,center]{title in head/foot}%
    \usebeamerfont{section in head/foot}{\bf\insertsectionhead}
  \end{beamercolorbox}%
  \begin{beamercolorbox}[wd=.333333\paperwidth,ht=2.25ex,dp=1ex,right]{date in head/foot}%
    % \usebeamerfont{date in head/foot}\insertshortdate{}\hspace*{2em}
    \insertframenumber{} / \inserttotalframenumber\hspace*{2ex} 
  \end{beamercolorbox}}%
  \vskip0pt%
}


\definecolor{mygrey}{rgb}{0.5,0.5,0.5}
\setlength{\parindent}{0cm}
%\DeclareTextFontCommand{\helvetica}{\fontfamily{phv}\selectfont}

% background beamer
\definecolor{couleurhaut}{rgb}{0.85,0.9,1}  % creme
\definecolor{couleurmilieu}{rgb}{1,1,1}  % vert pale
\definecolor{couleurbas}{rgb}{0.85,0.9,1}  % blanc
\setbeamertemplate{background canvas}[vertical shading]%
[top=couleurhaut,middle=couleurmilieu,midpoint=0.4,bottom=couleurbas] 
%[top=fondtitre!05,bottom=fondtitre!60]



\makeatletter
\setbeamertemplate{theorem begin}
{%
  \begin{\inserttheoremblockenv}
  {%
    \inserttheoremheadfont
    \inserttheoremname
    \inserttheoremnumber
    \ifx\inserttheoremaddition\@empty\else\ (\inserttheoremaddition)\fi%
    \inserttheorempunctuation
  }%
}
\setbeamertemplate{theorem end}{\end{\inserttheoremblockenv}}

\newenvironment{theoreme}[1][]{%
   \setbeamercolor{block title}{fg=structure,bg=structure!40}
   \setbeamercolor{block body}{fg=black,bg=structure!10}
   \begin{block}{{\bf Th\'eor\`eme }#1}
}{%
   \end{block}%
}


\newenvironment{proposition}[1][]{%
   \setbeamercolor{block title}{fg=structure,bg=structure!40}
   \setbeamercolor{block body}{fg=black,bg=structure!10}
   \begin{block}{{\bf Proposition }#1}
}{%
   \end{block}%
}

\newenvironment{corollaire}[1][]{%
   \setbeamercolor{block title}{fg=structure,bg=structure!40}
   \setbeamercolor{block body}{fg=black,bg=structure!10}
   \begin{block}{{\bf Corollaire }#1}
}{%
   \end{block}%
}

\newenvironment{mydefinition}[1][]{%
   \setbeamercolor{block title}{fg=structure,bg=structure!40}
   \setbeamercolor{block body}{fg=black,bg=structure!10}
   \begin{block}{{\bf Définition} #1}
}{%
   \end{block}%
}

\newenvironment{lemme}[0]{%
   \setbeamercolor{block title}{fg=structure,bg=structure!40}
   \setbeamercolor{block body}{fg=black,bg=structure!10}
   \begin{block}{\bf Lemme}
}{%
   \end{block}%
}

\newenvironment{remarque}[1][]{%
   \setbeamercolor{block title}{fg=black,bg=structure!20}
   \setbeamercolor{block body}{fg=black,bg=structure!5}
   \begin{block}{Remarque #1}
}{%
   \end{block}%
}


\newenvironment{exemple}[1][]{%
   \setbeamercolor{block title}{fg=black,bg=structure!20}
   \setbeamercolor{block body}{fg=black,bg=structure!5}
   \begin{block}{{\bf Exemple }#1}
}{%
   \end{block}%
}


\newenvironment{miniexercice}[0]{%
   \setbeamercolor{block title}{fg=structure,bg=structure!20}
   \setbeamercolor{block body}{fg=black,bg=structure!5}
   \begin{block}{Mini-exercices}
}{%
   \end{block}%
}


\newenvironment{tp}[0]{%
   \setbeamercolor{block title}{fg=structure,bg=structure!40}
   \setbeamercolor{block body}{fg=black,bg=structure!10}
   \begin{block}{\bf Travaux pratiques}
}{%
   \end{block}%
}
\newenvironment{exercicecours}[1][]{%
   \setbeamercolor{block title}{fg=structure,bg=structure!40}
   \setbeamercolor{block body}{fg=black,bg=structure!10}
   \begin{block}{{\bf Exercice }#1}
}{%
   \end{block}%
}
\newenvironment{algo}[1][]{%
   \setbeamercolor{block title}{fg=structure,bg=structure!40}
   \setbeamercolor{block body}{fg=black,bg=structure!10}
   \begin{block}{{\bf Algorithme}\hfill{\color{gray}\texttt{#1}}}
}{%
   \end{block}%
}


\setbeamertemplate{proof begin}{
   \setbeamercolor{block title}{fg=black,bg=structure!20}
   \setbeamercolor{block body}{fg=black,bg=structure!5}
   \begin{block}{{\footnotesize Démonstration}}
   \footnotesize
   \smallskip}
\setbeamertemplate{proof end}{%
   \end{block}}
\setbeamertemplate{qed symbol}{\openbox}


\makeatother
% Couleur à définir
   
%%%%%%%%%%%%%%%%%%%%%%%%%%%%%%%%%%%%%%%%%%%%%%%%%%%%%%%%%%%%%
%%%%%%%%%%%%%%%%%%%%%%%%%%%%%%%%%%%%%%%%%%%%%%%%%%%%%%%%%%%%%


\begin{document}


\title{{\bf Séries}}
\subtitle{Comparaison série/intégrale}

\begin{frame}
  
  \debutmontitre

  \pause

{\footnotesize
\hfill
\setbeamercovered{transparent=50}
\begin{minipage}{0.6\textwidth}
  \begin{itemize}
    \item<3-> Théorème de comparaison série/intégrale
    \item<4-> Démonstration
    \item<5-> Séries de Riemann
    \item<6-> Séries de Bertrand
    \item<7-> Applications
  \end{itemize}
\end{minipage}
}

\end{frame}

\setcounter{framenumber}{0}



%%%%%%%%%%%%%%%%%%%%%%%%%%%%%%%%%%%%%%%%%%%%%%%%%%%%%%%%%%%%%%%%
\section{Théorème de comparaison série/intégrale}

\begin{frame}
\`A connaître : les intégrales impropres $\int_0^{+\infty} f(t) \;\dd t$

\pause
\medskip

\begin{theoreme}
Soit $f : [0,+\infty[ \to [0,+\infty[$ une fonction décroissante

\pause
Alors la série $\displaystyle\sum_{k \ge 0} f(k)$ et l'intégrale impropre $\displaystyle\int_0^{+\infty} f(t) \; \dd t$ sont de même nature
\end{theoreme}

\pause
\medskip
<<~De même nature~>> signifie que la série et l'intégrale sont toutes deux convergentes ou toutes deux divergentes

\pause
\medskip
\textbf{Attention !} Il est important que $f$ soit positive et décroissante
\end{frame}



%%%%%%%%%%%%%%%%%%%%%%%%%%%%%%%%%%%%%%%%%%%%%%%%%%%%%%%%%%%%%%%%
\section{Démonstration}

\begin{frame}
\begin{proof}
Soit $k\in\Nn$

\begin{itemize}
\item\pause Comme $f$ est décroissante, pour $k\le t \le k+1$, on a $f(k+1)\le f(t)\le f(k)$
\onslide<1->
\myfigure{1.1}{
\tikzinput{fig_series02} 
}  
\item\pause En intégrant sur $[k,k+1]$ on obtient \ \pause $f(k+1)\le \displaystyle\int_k^{k+1} f(t) \; \dd t \le f(k)$
\noqed\qedhere
\end{itemize}
\end{proof}
\end{frame}


\begin{frame}
\begin{proof}
$$f(k+1)\le \int_k^{k+1} f(t) \; \dd t \le f(k)$$
\begin{itemize}
\item\pause On somme ces inégalités pour $k$ variant de $0$ à $n-1$
\pause
$$\sum_{k=0}^{n-1} f(k+1) \le \sum_{k=0}^{n-1} \int_{k}^{k+1} f(t) \; \dd t
\le \sum_{k=0}^{n-1} f(k)
$$
\pause
Soit
$$u_1+\cdots+u_{n} \le \int_0^{n} f(t)\; \dd t \le u_0+\cdots+u_{n-1}$$

\item\pause Si $\sum u_k$ converge et a pour somme $S$\pause , alors $\int_0^{n} f(t)\; \dd t$ est
majorée par $S$

\pause
Comme $\int_0^x f(t)\; \dd t$ est une fonction croissante de $x$, l'intégrale converge 

\item\pause Réciproquement, si l'intégrale
converge, alors $\int_0^{n} f(t)\;\dd t$ est majorée\pause , la suite des
sommes partielles aussi : la série converge \qedhere
\end{itemize}
\end{proof}
\end{frame}


%%%%%%%%%%%%%%%%%%%%%%%%%%%%%%%%%%%%%%%%%%%%%%%%%%%%%%%%%%%%%%%%
\section{Séries de Riemann}

\begin{frame}


$$\sum_{k = 1}^{+\infty} \frac{1}{k^\alpha}$$

\begin{proposition}[Séries de Riemann]
\pause
\mybox{Si \quad $\alpha >1$ \quad alors \quad $\displaystyle  \sum_{k = 1}^{+\infty} \frac{1}{k^\alpha}$ \quad converge}


\pause
\mybox{Si \quad $0 <\alpha \le1$ \quad alors \quad $\displaystyle  \sum_{k \ge 1} \frac{1}{k^\alpha}$ \quad diverge}
\end{proposition}

\end{frame}


\begin{frame}

\begin{proposition}[Séries de Riemann]
Si \quad $\alpha >1$ \quad alors \quad $\displaystyle  \sum_{k = 1}^{+\infty} \frac{1}{k^\alpha}$ \quad converge

Si \quad $0 <\alpha \le1$ \quad alors \quad $\displaystyle  \sum_{k \ge 1} \frac{1}{k^\alpha}$ \quad diverge
\end{proposition}

\pause
\begin{proof}
\pause
Soit $f : [1,+\infty[ \to [0,+\infty[$, $f(t)=\frac{1}{t^\alpha}$. \pause Pour $\alpha >0$, c'est une fonction décroissante et positive\pause : on peut appliquer le théorème précédent.\pause Or
$$
\int_1^{x} \frac{1}{t^\alpha}\;\dd t =
\left\{\begin{array}{ll}
\displaystyle{\frac{1}{1-\alpha}(x^{1-\alpha}-1)}&\mbox{si }\alpha\neq
    1\\[1.5ex]
\pause \ln(x) &\mbox{si }\alpha=1
\end{array}\right. 
$$

\begin{itemize}
\item\pause Pour $\alpha > 1$, $\int_1^{+\infty} \frac{1}{t^\alpha} \; \dd t $ 
est convergente, donc la série $\sum_{k = 1}^{+\infty} \frac{1}{k^\alpha}$ converge

\item\pause Pour $0<\alpha\le 1$, $\int_1^{+\infty} \frac{1}{t^\alpha} \; \dd t $ 
est divergente, donc la série $\sum_{k \ge 1} \frac{1}{k^\alpha}$ diverge\qedhere
\end{itemize}
\end{proof}
\end{frame}

%%%%%%%%%%%%%%%%%%%%%%%%%%%%%%%%%%%%%%%%%%%%%%%%%%%%%%%%%%%%%%%%
\section{Séries de Bertrand}

\begin{frame}
\defi{Séries de Bertrand} : 
$\displaystyle\sum_{k\ge2} \frac{1}{k^\alpha(\ln k)^\beta}$ \ où $ \alpha > 0$ et $\beta \in \Rr$

\pause
\begin{proposition}
\mybox{Si \ $\alpha>1$ \ alors la série converge. \ \ Si \ $0<\alpha<1$ \ alors elle diverge}

\pause
\mybox{Si \quad $\alpha=1$ \quad et \quad $\begin{cases} \beta>1 & \text{alors elle converge}\\   
\beta\le 1 & \text{alors elle diverge}
\end{cases}$}
\end{proposition}

\pause
\begin{proof}
\pause
Cas $\alpha=1$ :  \quad \pause
$
\displaystyle\int_2^{x} \frac{1}{t(\ln t)^{\beta}}\;\dd t =
\left\{\begin{array}{ll}
{\dfrac{1}{1-\beta}\left((\ln x)^{1-\beta}-(\ln 2)^{1-\beta}\right)}&\text{si }\beta\neq
    1\\[1.5ex]
\ln(\ln x)-\ln(\ln 2) &\text{si }\beta=1 \qedhere
\end{array}\right. 
$
\end{proof}
\end{frame}


%%%%%%%%%%%%%%%%%%%%%%%%%%%%%%%%%%%%%%%%%%%%%%%%%%%%%%%%%%%%%%%%
\section{Applications}

\begin{frame}
Nous retrouvons en particulier
\begin{enumerate}
  \item $\displaystyle\sum \frac{1}{k^2}$ converge ($\alpha=2$)
  
  \medskip
  
  \item\pause $\displaystyle\sum \frac{1}{k}$ diverge ($\alpha=1$)
\end{enumerate}

\pause
\begin{exemple}
La série  $\displaystyle\sum_{k \ge 1} \ln\left(1+\frac{1}{\sqrt{k^3}}\right)$ est-elle convergente ? 
 
 \begin{itemize}
 \item\pause $\ln\left(1+\frac{1}{\sqrt{k^3}}\right) \quad \sim \quad \frac{1}{\sqrt{k^3}}$
\item\pause La série de Riemann $\sum \frac{1}{\sqrt{k^3}}=\sum \frac{1}{k^{\frac32}}$ converge (car $\frac32>1$) 
\item\pause Alors par le théorème des équivalents la série $\displaystyle\sum_{k=1}^{+\infty} \ln\left(1+\frac{1}{\sqrt{k^3}}\right)$ converge également
\end{itemize}
\end{exemple}

 \end{frame}


\begin{frame}
\begin{exemple}

$$\text{La série}\qquad\sum_{k \ge 1} 
\frac{1-\cos\left(\frac{1}{k\sqrt{\ln k}}\right)}{\sin\left(\frac{1}{k}\right)}
\qquad\text{est-elle convergente ?}$$
  
 \begin{itemize}
 \item\pause On cherche un équivalent du terme général (qui est positif) :  
$$\frac{1-\cos\left(\frac{1}{k\sqrt{\ln k}}\right)}
{\sin\left(\frac{1}{k}\right)} \quad \sim \quad \frac{1}{2k\ln k}$$
\item\pause Or la série de Bertrand $\sum \frac{1}{k\ln k}$ diverge
\item\pause Donc notre série diverge aussi
\end{itemize}
\end{exemple}
\end{frame}


%%%%%%%%%%%%%%%%%%%%%%%%%%%%%%%%%%%%%%%%%%%%%%%%%%%%%%%%%%%%%%%%
\section{Mini-exercices}

\begin{frame}
\begin{miniexercice}
\begin{enumerate}
  \item Notons $H_n = \sum_{k=1}^{n} \frac 1k$ la somme partielle de la série harmonique
  
  Soit $f : [1,+\infty[\to[0,+\infty[$ définie par $f(t)=\frac1t$
  \begin{enumerate}
    \item Donner un encadrement simple de $\int_k^{k+1} f(t) \;\dd t$
    \item Faire la somme de ces inégalités pour $k$ variant de $1$ à $n-1$, puis $k$ variant de $1$
  à $n$, pour obtenir : $\ln(n+1) \le H_n \le1 + \ln n$
    \item En déduire $H_n \sim \ln n$
    \item La série harmonique converge-t-elle ?
   \end{enumerate}
  
  \item Reprendre le schéma d'étude précédent pour montrer que, pour la série de Riemann 
  et $0\le \alpha <1$, on a
  $\sum_{k=1}^{n} \frac{1}{k^\alpha} \sim \frac{n^{1-\alpha}}{1-\alpha}$
  
  \item Reprendre le schéma d'étude précédent, mais cette fois pour le reste
  $R_n = \sum_{k=n+1}^{\infty} \frac{1}{k^2}$, afin de montrer que
  $R_n \sim \frac{1}{n}.$
  Calculer $R_{100}$. Quelle approximation cela fournit-il de la somme de la série ?
  
  \item \'Etudier la convergence des séries suivantes en fonction des paramètres $\alpha>0$ et $\beta \in \Rr$ :
  $\sum \sqrt{k^\alpha+1}-\sqrt{k^\alpha} \qquad
  \sum \sin\left(\frac{k^\alpha}{\ln k}\right) \qquad
  \sum \ln \left(1+\frac{1}{k(\ln k)^\beta}\right)$
  
  
\end{enumerate}
\end{miniexercice}
\end{frame}

\end{document}