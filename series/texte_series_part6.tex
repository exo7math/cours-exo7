
%%%%%%%%%%%%%%%%%% PREAMBULE %%%%%%%%%%%%%%%%%%


\documentclass[12pt]{article}

\usepackage{amsfonts,amsmath,amssymb,amsthm}
\usepackage[utf8]{inputenc}
\usepackage[T1]{fontenc}
\usepackage[francais]{babel}


% packages
\usepackage{amsfonts,amsmath,amssymb,amsthm}
\usepackage[utf8]{inputenc}
\usepackage[T1]{fontenc}
%\usepackage{lmodern}

\usepackage[francais]{babel}
\usepackage{fancybox}
\usepackage{graphicx}

\usepackage{float}

%\usepackage[usenames, x11names]{xcolor}
\usepackage{tikz}
\usepackage{datetime}

\usepackage{mathptmx}
%\usepackage{fouriernc}
%\usepackage{newcent}
\usepackage[mathcal,mathbf]{euler}

%\usepackage{palatino}
%\usepackage{newcent}


% Commande spéciale prompteur

%\usepackage{mathptmx}
%\usepackage[mathcal,mathbf]{euler}
%\usepackage{mathpple,multido}

\usepackage[a4paper]{geometry}
\geometry{top=2cm, bottom=2cm, left=1cm, right=1cm, marginparsep=1cm}

\newcommand{\change}{{\color{red}\rule{\textwidth}{1mm}\\}}

\newcounter{mydiapo}

\newcommand{\diapo}{\newpage
\hfill {\normalsize  Diapo \themydiapo \quad \texttt{[\jobname]}} \\
\stepcounter{mydiapo}}


%%%%%%% COULEURS %%%%%%%%%%

% Pour blanc sur noir :
%\pagecolor[rgb]{0.5,0.5,0.5}
% \pagecolor[rgb]{0,0,0}
% \color[rgb]{1,1,1}



%\DeclareFixedFont{\myfont}{U}{cmss}{bx}{n}{18pt}
\newcommand{\debuttexte}{
%%%%%%%%%%%%% FONTES %%%%%%%%%%%%%
\renewcommand{\baselinestretch}{1.5}
\usefont{U}{cmss}{bx}{n}
\bfseries

% Taille normale : commenter le reste !
%Taille Arnaud
%\fontsize{19}{19}\selectfont

% Taille Barbara
%\fontsize{21}{22}\selectfont

%Taille François
\fontsize{25}{30}\selectfont

%Taille Pascal
%\fontsize{25}{30}\selectfont

%Taille Laura
%\fontsize{30}{35}\selectfont


%\myfont
%\usefont{U}{cmss}{bx}{n}

%\Huge
%\addtolength{\parskip}{\baselineskip}
}


% \usepackage{hyperref}
% \hypersetup{colorlinks=true, linkcolor=blue, urlcolor=blue,
% pdftitle={Exo7 - Exercices de mathématiques}, pdfauthor={Exo7}}


%section
% \usepackage{sectsty}
% \allsectionsfont{\bf}
%\sectionfont{\color{Tomato3}\upshape\selectfont}
%\subsectionfont{\color{Tomato4}\upshape\selectfont}

%----- Ensembles : entiers, reels, complexes -----
\newcommand{\Nn}{\mathbb{N}} \newcommand{\N}{\mathbb{N}}
\newcommand{\Zz}{\mathbb{Z}} \newcommand{\Z}{\mathbb{Z}}
\newcommand{\Qq}{\mathbb{Q}} \newcommand{\Q}{\mathbb{Q}}
\newcommand{\Rr}{\mathbb{R}} \newcommand{\R}{\mathbb{R}}
\newcommand{\Cc}{\mathbb{C}} 
\newcommand{\Kk}{\mathbb{K}} \newcommand{\K}{\mathbb{K}}

%----- Modifications de symboles -----
\renewcommand{\epsilon}{\varepsilon}
\renewcommand{\Re}{\mathop{\text{Re}}\nolimits}
\renewcommand{\Im}{\mathop{\text{Im}}\nolimits}
%\newcommand{\llbracket}{\left[\kern-0.15em\left[}
%\newcommand{\rrbracket}{\right]\kern-0.15em\right]}

\renewcommand{\ge}{\geqslant}
\renewcommand{\geq}{\geqslant}
\renewcommand{\le}{\leqslant}
\renewcommand{\leq}{\leqslant}

%----- Fonctions usuelles -----
\newcommand{\ch}{\mathop{\mathrm{ch}}\nolimits}
\newcommand{\sh}{\mathop{\mathrm{sh}}\nolimits}
\renewcommand{\tanh}{\mathop{\mathrm{th}}\nolimits}
\newcommand{\cotan}{\mathop{\mathrm{cotan}}\nolimits}
\newcommand{\Arcsin}{\mathop{\mathrm{Arcsin}}\nolimits}
\newcommand{\Arccos}{\mathop{\mathrm{Arccos}}\nolimits}
\newcommand{\Arctan}{\mathop{\mathrm{Arctan}}\nolimits}
\newcommand{\Argsh}{\mathop{\mathrm{Argsh}}\nolimits}
\newcommand{\Argch}{\mathop{\mathrm{Argch}}\nolimits}
\newcommand{\Argth}{\mathop{\mathrm{Argth}}\nolimits}
\newcommand{\pgcd}{\mathop{\mathrm{pgcd}}\nolimits} 

\newcommand{\Card}{\mathop{\text{Card}}\nolimits}
\newcommand{\Ker}{\mathop{\text{Ker}}\nolimits}
\newcommand{\id}{\mathop{\text{id}}\nolimits}
\newcommand{\ii}{\mathrm{i}}
\newcommand{\dd}{\mathrm{d}}
\newcommand{\Vect}{\mathop{\text{Vect}}\nolimits}
\newcommand{\Mat}{\mathop{\mathrm{Mat}}\nolimits}
\newcommand{\rg}{\mathop{\text{rg}}\nolimits}
\newcommand{\tr}{\mathop{\text{tr}}\nolimits}
\newcommand{\ppcm}{\mathop{\text{ppcm}}\nolimits}

%----- Structure des exercices ------

\newtheoremstyle{styleexo}% name
{2ex}% Space above
{3ex}% Space below
{}% Body font
{}% Indent amount 1
{\bfseries} % Theorem head font
{}% Punctuation after theorem head
{\newline}% Space after theorem head 2
{}% Theorem head spec (can be left empty, meaning ‘normal’)

%\theoremstyle{styleexo}
\newtheorem{exo}{Exercice}
\newtheorem{ind}{Indications}
\newtheorem{cor}{Correction}


\newcommand{\exercice}[1]{} \newcommand{\finexercice}{}
%\newcommand{\exercice}[1]{{\tiny\texttt{#1}}\vspace{-2ex}} % pour afficher le numero absolu, l'auteur...
\newcommand{\enonce}{\begin{exo}} \newcommand{\finenonce}{\end{exo}}
\newcommand{\indication}{\begin{ind}} \newcommand{\finindication}{\end{ind}}
\newcommand{\correction}{\begin{cor}} \newcommand{\fincorrection}{\end{cor}}

\newcommand{\noindication}{\stepcounter{ind}}
\newcommand{\nocorrection}{\stepcounter{cor}}

\newcommand{\fiche}[1]{} \newcommand{\finfiche}{}
\newcommand{\titre}[1]{\centerline{\large \bf #1}}
\newcommand{\addcommand}[1]{}
\newcommand{\video}[1]{}

% Marge
\newcommand{\mymargin}[1]{\marginpar{{\small #1}}}



%----- Presentation ------
\setlength{\parindent}{0cm}

%\newcommand{\ExoSept}{\href{http://exo7.emath.fr}{\textbf{\textsf{Exo7}}}}

\definecolor{myred}{rgb}{0.93,0.26,0}
\definecolor{myorange}{rgb}{0.97,0.58,0}
\definecolor{myyellow}{rgb}{1,0.86,0}

\newcommand{\LogoExoSept}[1]{  % input : echelle
{\usefont{U}{cmss}{bx}{n}
\begin{tikzpicture}[scale=0.1*#1,transform shape]
  \fill[color=myorange] (0,0)--(4,0)--(4,-4)--(0,-4)--cycle;
  \fill[color=myred] (0,0)--(0,3)--(-3,3)--(-3,0)--cycle;
  \fill[color=myyellow] (4,0)--(7,4)--(3,7)--(0,3)--cycle;
  \node[scale=5] at (3.5,3.5) {Exo7};
\end{tikzpicture}}
}



\theoremstyle{definition}
%\newtheorem{proposition}{Proposition}
%\newtheorem{exemple}{Exemple}
%\newtheorem{theoreme}{Théorème}
\newtheorem{lemme}{Lemme}
\newtheorem{corollaire}{Corollaire}
%\newtheorem*{remarque*}{Remarque}
%\newtheorem*{miniexercice}{Mini-exercices}
%\newtheorem{definition}{Définition}




%definition d'un terme
\newcommand{\defi}[1]{{\color{myorange}\textbf{\emph{#1}}}}
\newcommand{\evidence}[1]{{\color{blue}\textbf{\emph{#1}}}}



 %----- Commandes divers ------

\newcommand{\codeinline}[1]{\texttt{#1}}

%%%%%%%%%%%%%%%%%%%%%%%%%%%%%%%%%%%%%%%%%%%%%%%%%%%%%%%%%%%%%
%%%%%%%%%%%%%%%%%%%%%%%%%%%%%%%%%%%%%%%%%%%%%%%%%%%%%%%%%%%%%


\begin{document}

\debuttexte


%%%%%%%%%%%%%%%%%%%%%%%%%%%%%%%%%%%%%%%%%%%%%%%%%%%%%%%%%%%
\diapo

Voici une leçon consacrée aux produits de deux séries.

\change
\change
Nous commencerons par motiver la nécessité de définir un tel produit.

\change
Puis nous définirons le produit de Cauchy,

\change
et nous en donnerons un exemple

\change
ainsi qu'un contre-exemple.

Cette leçon consacrée au produit de deux séries peut être éludée si c'est votre 
première approche du chapitre sur les séries.

%%%%%%%%%%%%%%%%%%%%%%%%%%%%%%%%%%%%%%%%%%%%%%%%%%%%%%%%%%%
\diapo

Pour un produit de sommes, il y a plusieurs façons d'ordonner les termes 
une fois le produit développé. 

\change
Ici par exemple, dans le produit $\big(a_0+a_1\big)\big(b_0+b_1\big)$, il y a plusieurs façons d'ordonner les termes $a_0b_0, a_0b_1, a_1b_0$ et $a_1b_1$.

Dans le cas d'une somme finie, comme dans ces deux exemples, l'ordre des termes n'a pas d'importance, mais dans le cas d'une série c'est essentiel.

\change
On choisit de regrouper les termes en fonction des indices, 
et plus précisemment de la somme des deux indices.

\change
Cela se passe de la façon suivante :

pour ce terme, la somme des indices est $0+0=0$, 
pour ces termes, 
la sommes des indices est $0+1=1+0=1$, et enfin pour celui-là, la somme est $2$.

\change
De même pour ce produit : par exemple ces trois termes sont ceux dans la somme des indices est $0+2=1+1=2+0=2$.

\change
Plus généralement, voici différentes façons d'écrire un produit de deux sommes :
$$\left(\sum_{i=0}^n a_i\right)\; \left(\sum_{j=0}^n b_j\right)$$

\change
est égal à
$$
\sum_{i=0}^n \sum_{j=0}^n a_ib_j$$

\change
$$= \sum_{0 \le k \le 2n} \sum_{i+j=k} a_ib_j$$

\change
$$= \sum_{0 \le k \le 2n} \sum_{0 \le i \le k} a_ib_{k-i}.$$

Les deux dernières formes correspondent exactement à notre décomposition 
en fonction de la somme des indices.


%%%%%%%%%%%%%%%%%%%%%%%%%%%%%%%%%%%%%%%%%%%%%%%%%%%%%%%%%%%
\diapo

Soient $\sum_{i \ge 0} a_i$ et $\sum_{j \ge 0} b_j$ deux séries.

\change
On appelle \defi{série produit} ou \defi{produit de Cauchy}
la série $\sum_{k \ge 0} c_k$
o\`u le terme de la série produit est lui-même un somme :

$$\displaystyle c_k=\sum_{i=0}^k a_i b_{k-i}$$ 

\change
Une autre façon d'écrire le coefficient $c_k$ est :
$$\displaystyle c_k=\sum_{i+j=k} a_i b_j$$

%%%%%%%%%%%%%%%%%%%%%%%%%%%%%%%%%%%%%%%%%%%%%%%%%%%%%%%%%%%
\diapo

Théorème.

Si les séries $\displaystyle\sum_{i=0}^{+\infty} a_i$ et $\displaystyle\sum_{j=0}^{+\infty} b_j$ de nombres réels (ou complexes)
sont absolument convergentes, 

\change
alors la série produit 
$$\sum_{k=0}^{+\infty} c_k = \sum_{k=0}^{+\infty} \left(\sum_{i=0}^k a_ib_{k-i}\right)$$
est absolument convergente 

\change
et l'on a surtout que la série produit est le produit des sommes :
$$\sum_{k=0}^{+\infty} c_k = \left(\sum_{i=0}^{+\infty} a_i\right)\ \times\ \left(\sum_{j=0}^{+\infty} b_j\right).$$

\change
La démonstration du théorème est assez technique.
On commence par introduire les notations suivantes.

\change
Soit $S_n=a_0+\dots+a_n$ la somme partielle de la première série. Alors $S_n$ tend vers un nombre $S$.

\change
De même on note  
$T_n=b_0+\dots +b_n$ la somme partielle de la seconde série, et $T_n\to T$,
  
\change
et enfin $P_n= c_0+\dots+c_n$ la somme partielle de la produit.

\change
On doit donc montrer que $P_n$ tend vers le produit $S\cdot T$.

%%%%%%%%%%%%%%%%%%%%%%%%%%%%%%%%%%%%%%%%%%%%%%%%%%%%%%%%%%%
\diapo

On va distinguer deux cas.

Premier cas : les termes généraux $a_k$ et $b_k $ sont tous positifs ou nuls. 

\change
Dans ce cas $c_k\ge 0$ et on a 
$$P_n \le S_n \cdot T_n\le S \cdot T.$$ 


L'inégalité de droite vient du fait que $S_n$ et $T_n$ sont des suites croissantes convergentes.

L'inégalité de gauche s'explique en développant le produit $S_n \times T_n$ qui contient, entre autres,
tous les termes $c_0$ jusqu'à $c_n$.

\change
La suite $(P_n)$ est croissante et majorée, elle converge donc vers un réel $P$.

\change
Or on a aussi 
$$P_n \le S_n \cdot T_n \le P_{2n}.$$

\change
Cette inégalité se retrouve facilement sur le dessin suivant.

Le dessin représente le point correspondant aux indices $(i,j)$.

Le triangle rouge représente  $P_n$ (avec le regroupement des termes correspondant aux diagonales qui ont pour équations $i+j=k$ avec $k$ variant de $0$ à $n$). 

Le carré vert correspond aux termes du produit $S_n \cdot T_n$, 
le triangle bleu représente  $P_{2n}$. 

Le fait que le carré soit compris entre les deux triangles traduit la double inégalité $P_n \le S_n \cdot T_n \le P_{2n}$.

\change
Donc en faisant $n\to+\infty$, on a: $P\le S \cdot T\le P$. 

\change
Donc $P_n\to S \cdot T$.

%%%%%%%%%%%%%%%%%%%%%%%%%%%%%%%%%%%%%%%%%%%%%%%%%%%%%%%%%%%
\diapo

Considérons à présent le cas général, où les termes généraux $a_k$ et $b_k$ sont des nombres complexes.

\change
On pose alors :
$S_n'=|a_0|+\dots+|a_n|$,. Comme la série $\sum  a_i$ est absolument convergente, alors $S_n'\to S'$.

\change  
De même on pose $T_n'=|b_0|+\dots+|b_n|$, et $T_n'\to T'$.

\change  
Enfin $P_n'= c_0'+\dots+ c_n'$ o\`u $c_k'=\sum_{i=0}^k|a_ib_{k-i}|$.

\change
Ces séries étant à termes généraux positifs, d'après le premier cas, on a $P_n'\to P'$ avec $P'=S' \cdot T'$. 

\change
Ainsi $|S_n \cdot T_n - P_n|$
qui vaut 
$$\bigl|\sum_{\stackrel{0\le i,j\le n}{i+j>n}} a_ib_j\bigr|$$

\change
est inférieur ou égal à la somme sur les mêmes indices des $|a_ib_j| $

\change
qui est égal à
$
 S_n' \cdot T_n'-P_n' 
$

\change
qui tend vers $S' \cdot T'-P'$, c'est-à-dire vers $0$.

\change
Ainsi $P_n$ qui est égal à $S_n \cdot T_n-(S_n \cdot T_n-P_n)$ tend vers $ S \cdot T-0=S \cdot T$.

\change
Donc la série $\sum c_k$ est convergente et sa somme est $S \cdot T$. 

\change
De plus, $|c_k|\le c_k'$. 

\change
La convergence de $\sum c_k'$ implique 
donc la convergence absolue de $\sum c_k$.



%%%%%%%%%%%%%%%%%%%%%%%%%%%%%%%%%%%%%%%%%%%%%%%%%%%%%%%%%%%
\diapo

Soit $\displaystyle\sum_{i=0}^{+\infty} a_i$ une série absolument convergente

\change
Soit $\displaystyle\sum_{j=0}^{+\infty} b_j$ la série définie par $b_j = \frac{1}{2^j}$.

C'est une série géométrique absolument convergente.

\change
Notons
$$c_k = \sum_{i=0}^k a_ib_{k-i} = \sum_{i=0}^k a_i \times \frac{1}{2^{k-i}}.$$

\change
Alors le produit de Cauchy $\sum c_k$ converge absolument et
$$\sum_{k=0}^{+\infty} c_k 
= \left(\sum_{i=0}^{+\infty} a_i\right)\ \times\ \left(\sum_{j=0}^{+\infty} b_j\right)$$

\change
ce qui donne $ 2 \displaystyle\sum_{i=0}^{+\infty} a_i.$


%%%%%%%%%%%%%%%%%%%%%%%%%%%%%%%%%%%%%%%%%%%%%%%%%%%%%%%%%%%
\diapo

Si les séries $\sum a_i$ et $\sum b_j$ ne sont pas absolument 
convergentes, mais seulement convergentes,
alors la série de Cauchy peut être divergente.

\change
Soient $a_i=b_i=\frac{(-1)^{i}}{\sqrt {i+1}}, i\ge 0$.

\change
Alors $\sum a_i$ et $\sum b_j$ sont convergentes par le critère de Leibniz, 
mais ne sont pas absolument convergentes. 

\change
Le terme général de leur produit de Cauchy est
$$c_k = \sum_{i=0}^k a_i b_{k-i} $$

\change
c'est-à-dire
$$
 \sum_{i=0}^k \frac{(-1)^{i}}{\sqrt {i+1}} \frac{(-1)^{k-i}}{\sqrt{k-i+1}} 
 $$
 
 \change
ce qui donne
$$  (-1)^{k} \sum_{i=0}^k \frac{1}{\sqrt{(i+1)(k-i+1)}}.$$

\change
Or, pour $x\in\Rr$, $(x+1)(k-x+1)=-x^2+kx+(k+1) \le \frac{(k+2)^2}{4}$ (valeur au sommet de la parabole). 

\change
On obtient $\sqrt{(i+1)(k-i+1)} \le \frac{(k+2)}{2}$.`

\change
Ainsi $|c_k|$ qui est égal à $\displaystyle\sum_{i=0}^k\dfrac{1}{\sqrt{(i+1)(k-i+1)}} $

\change
est supérieure ou égale à 
$\displaystyle\sum_{i=0}^k \dfrac{2}{k+2}$

\change
qui vaut
$ \dfrac{2(k+1)}{k+2} \to 2.$

\change
Donc le terme général $c_k$ ne peut pas tendre $0$, 

\change
donc la série $\sum c_k$ diverge.

%%%%%%%%%%%%%%%%%%%%%%%%%%%%%%%%%%%%%%%%%%%%%%%%%%%%%%%%%%%
\diapo
Entrainez-vous avec ces exercices pour vérifier que vous avez bien compris le cours.


\end{document}
