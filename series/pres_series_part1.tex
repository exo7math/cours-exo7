
%%%%%%%%%%%%%%%%%% PREAMBULE %%%%%%%%%%%%%%%%%%

\documentclass[aspectratio=169,utf8]{beamer}
%\documentclass[aspectratio=169,handout]{beamer}

\usetheme{Boadilla}
%\usecolortheme{seahorse}
\usecolortheme[RGB={245,66,24}]{structure}
\useoutertheme{infolines}

% packages
\usepackage{amsfonts,amsmath,amssymb,amsthm}
\usepackage[utf8]{inputenc}
\usepackage[T1]{fontenc}
\usepackage{lmodern}

\usepackage[francais]{babel}
\usepackage{fancybox}
\usepackage{graphicx}

\usepackage{float}
\usepackage{xfrac}

%\usepackage[usenames, x11names]{xcolor}
\usepackage{tikz}
\usepackage{pgfplots}
\usepackage{datetime}



%-----  Package unités -----
\usepackage{siunitx}
\sisetup{locale = FR,detect-all,per-mode = symbol}

%\usepackage{mathptmx}
%\usepackage{fouriernc}
%\usepackage{newcent}
%\usepackage[mathcal,mathbf]{euler}

%\usepackage{palatino}
%\usepackage{newcent}
% \usepackage[mathcal,mathbf]{euler}



% \usepackage{hyperref}
% \hypersetup{colorlinks=true, linkcolor=blue, urlcolor=blue,
% pdftitle={Exo7 - Exercices de mathématiques}, pdfauthor={Exo7}}


%section
% \usepackage{sectsty}
% \allsectionsfont{\bf}
%\sectionfont{\color{Tomato3}\upshape\selectfont}
%\subsectionfont{\color{Tomato4}\upshape\selectfont}

%----- Ensembles : entiers, reels, complexes -----
\newcommand{\Nn}{\mathbb{N}} \newcommand{\N}{\mathbb{N}}
\newcommand{\Zz}{\mathbb{Z}} \newcommand{\Z}{\mathbb{Z}}
\newcommand{\Qq}{\mathbb{Q}} \newcommand{\Q}{\mathbb{Q}}
\newcommand{\Rr}{\mathbb{R}} \newcommand{\R}{\mathbb{R}}
\newcommand{\Cc}{\mathbb{C}} 
\newcommand{\Kk}{\mathbb{K}} \newcommand{\K}{\mathbb{K}}

%----- Modifications de symboles -----
\renewcommand{\epsilon}{\varepsilon}
\renewcommand{\Re}{\mathop{\text{Re}}\nolimits}
\renewcommand{\Im}{\mathop{\text{Im}}\nolimits}
%\newcommand{\llbracket}{\left[\kern-0.15em\left[}
%\newcommand{\rrbracket}{\right]\kern-0.15em\right]}

\renewcommand{\ge}{\geqslant}
\renewcommand{\geq}{\geqslant}
\renewcommand{\le}{\leqslant}
\renewcommand{\leq}{\leqslant}
\renewcommand{\epsilon}{\varepsilon}

%----- Fonctions usuelles -----
\newcommand{\ch}{\mathop{\text{ch}}\nolimits}
\newcommand{\sh}{\mathop{\text{sh}}\nolimits}
\renewcommand{\tanh}{\mathop{\text{th}}\nolimits}
\newcommand{\cotan}{\mathop{\text{cotan}}\nolimits}
\newcommand{\Arcsin}{\mathop{\text{arcsin}}\nolimits}
\newcommand{\Arccos}{\mathop{\text{arccos}}\nolimits}
\newcommand{\Arctan}{\mathop{\text{arctan}}\nolimits}
\newcommand{\Argsh}{\mathop{\text{argsh}}\nolimits}
\newcommand{\Argch}{\mathop{\text{argch}}\nolimits}
\newcommand{\Argth}{\mathop{\text{argth}}\nolimits}
\newcommand{\pgcd}{\mathop{\text{pgcd}}\nolimits} 


%----- Commandes divers ------
\newcommand{\ii}{\mathrm{i}}
\newcommand{\dd}{\text{d}}
\newcommand{\id}{\mathop{\text{id}}\nolimits}
\newcommand{\Ker}{\mathop{\text{Ker}}\nolimits}
\newcommand{\Card}{\mathop{\text{Card}}\nolimits}
\newcommand{\Vect}{\mathop{\text{Vect}}\nolimits}
\newcommand{\Mat}{\mathop{\text{Mat}}\nolimits}
\newcommand{\rg}{\mathop{\text{rg}}\nolimits}
\newcommand{\tr}{\mathop{\text{tr}}\nolimits}


%----- Structure des exercices ------

\newtheoremstyle{styleexo}% name
{2ex}% Space above
{3ex}% Space below
{}% Body font
{}% Indent amount 1
{\bfseries} % Theorem head font
{}% Punctuation after theorem head
{\newline}% Space after theorem head 2
{}% Theorem head spec (can be left empty, meaning ‘normal’)

%\theoremstyle{styleexo}
\newtheorem{exo}{Exercice}
\newtheorem{ind}{Indications}
\newtheorem{cor}{Correction}


\newcommand{\exercice}[1]{} \newcommand{\finexercice}{}
%\newcommand{\exercice}[1]{{\tiny\texttt{#1}}\vspace{-2ex}} % pour afficher le numero absolu, l'auteur...
\newcommand{\enonce}{\begin{exo}} \newcommand{\finenonce}{\end{exo}}
\newcommand{\indication}{\begin{ind}} \newcommand{\finindication}{\end{ind}}
\newcommand{\correction}{\begin{cor}} \newcommand{\fincorrection}{\end{cor}}

\newcommand{\noindication}{\stepcounter{ind}}
\newcommand{\nocorrection}{\stepcounter{cor}}

\newcommand{\fiche}[1]{} \newcommand{\finfiche}{}
\newcommand{\titre}[1]{\centerline{\large \bf #1}}
\newcommand{\addcommand}[1]{}
\newcommand{\video}[1]{}

% Marge
\newcommand{\mymargin}[1]{\marginpar{{\small #1}}}

\def\noqed{\renewcommand{\qedsymbol}{}}


%----- Presentation ------
\setlength{\parindent}{0cm}

%\newcommand{\ExoSept}{\href{http://exo7.emath.fr}{\textbf{\textsf{Exo7}}}}

\definecolor{myred}{rgb}{0.93,0.26,0}
\definecolor{myorange}{rgb}{0.97,0.58,0}
\definecolor{myyellow}{rgb}{1,0.86,0}

\newcommand{\LogoExoSept}[1]{  % input : echelle
{\usefont{U}{cmss}{bx}{n}
\begin{tikzpicture}[scale=0.1*#1,transform shape]
  \fill[color=myorange] (0,0)--(4,0)--(4,-4)--(0,-4)--cycle;
  \fill[color=myred] (0,0)--(0,3)--(-3,3)--(-3,0)--cycle;
  \fill[color=myyellow] (4,0)--(7,4)--(3,7)--(0,3)--cycle;
  \node[scale=5] at (3.5,3.5) {Exo7};
\end{tikzpicture}}
}


\newcommand{\debutmontitre}{
  \author{} \date{} 
  \thispagestyle{empty}
  \hspace*{-10ex}
  \begin{minipage}{\textwidth}
    \titlepage  
  \vspace*{-2.5cm}
  \begin{center}
    \LogoExoSept{2.5}
  \end{center}
  \end{minipage}

  \vspace*{-0cm}
  
  % Astuce pour que le background ne soit pas discrétisé lors de la conversion pdf -> png
\begin{tikzpicture}
        \fill[opacity=0,green!60!black] (0,0)--++(0,0)--++(0,0)--++(0,0)--cycle; 
\end{tikzpicture}

% toc S'affiche trop tot :
% \tableofcontents[hideallsubsections, pausesections]
}

\newcommand{\finmontitre}{
  \end{frame}
  \setcounter{framenumber}{0}
} % ne marche pas pour une raison obscure

%----- Commandes supplementaires ------

% \usepackage[landscape]{geometry}
% \geometry{top=1cm, bottom=3cm, left=2cm, right=10cm, marginparsep=1cm
% }
% \usepackage[a4paper]{geometry}
% \geometry{top=2cm, bottom=2cm, left=2cm, right=2cm, marginparsep=1cm
% }

%\usepackage{standalone}


% New command Arnaud -- november 2011
\setbeamersize{text margin left=24ex}
% si vous modifier cette valeur il faut aussi
% modifier le decalage du titre pour compenser
% (ex : ici =+10ex, titre =-5ex

\theoremstyle{definition}
%\newtheorem{proposition}{Proposition}
%\newtheorem{exemple}{Exemple}
%\newtheorem{theoreme}{Théorème}
%\newtheorem{lemme}{Lemme}
%\newtheorem{corollaire}{Corollaire}
%\newtheorem*{remarque*}{Remarque}
%\newtheorem*{miniexercice}{Mini-exercices}
%\newtheorem{definition}{Définition}

% Commande tikz
\usetikzlibrary{calc}
\usetikzlibrary{patterns,arrows}
\usetikzlibrary{matrix}
\usetikzlibrary{fadings} 

%definition d'un terme
\newcommand{\defi}[1]{{\color{myorange}\textbf{\emph{#1}}}}
\newcommand{\evidence}[1]{{\color{blue}\textbf{\emph{#1}}}}
\newcommand{\assertion}[1]{\emph{\og#1\fg}}  % pour chapitre logique
%\renewcommand{\contentsname}{Sommaire}
\renewcommand{\contentsname}{}
\setcounter{tocdepth}{2}



%------ Figures ------

\def\myscale{1} % par défaut 
\newcommand{\myfigure}[2]{  % entrée : echelle, fichier figure
\def\myscale{#1}
\begin{center}
\footnotesize
{#2}
\end{center}}


%------ Encadrement ------

\usepackage{fancybox}


\newcommand{\mybox}[1]{
\setlength{\fboxsep}{7pt}
\begin{center}
\shadowbox{#1}
\end{center}}

\newcommand{\myboxinline}[1]{
\setlength{\fboxsep}{5pt}
\raisebox{-10pt}{
\shadowbox{#1}
}
}

%--------------- Commande beamer---------------
\newcommand{\beameronly}[1]{#1} % permet de mettre des pause dans beamer pas dans poly


\setbeamertemplate{navigation symbols}{}
\setbeamertemplate{footline}  % tiré du fichier beamerouterinfolines.sty
{
  \leavevmode%
  \hbox{%
  \begin{beamercolorbox}[wd=.333333\paperwidth,ht=2.25ex,dp=1ex,center]{author in head/foot}%
    % \usebeamerfont{author in head/foot}\insertshortauthor%~~(\insertshortinstitute)
    \usebeamerfont{section in head/foot}{\bf\insertshorttitle}
  \end{beamercolorbox}%
  \begin{beamercolorbox}[wd=.333333\paperwidth,ht=2.25ex,dp=1ex,center]{title in head/foot}%
    \usebeamerfont{section in head/foot}{\bf\insertsectionhead}
  \end{beamercolorbox}%
  \begin{beamercolorbox}[wd=.333333\paperwidth,ht=2.25ex,dp=1ex,right]{date in head/foot}%
    % \usebeamerfont{date in head/foot}\insertshortdate{}\hspace*{2em}
    \insertframenumber{} / \inserttotalframenumber\hspace*{2ex} 
  \end{beamercolorbox}}%
  \vskip0pt%
}


\definecolor{mygrey}{rgb}{0.5,0.5,0.5}
\setlength{\parindent}{0cm}
%\DeclareTextFontCommand{\helvetica}{\fontfamily{phv}\selectfont}

% background beamer
\definecolor{couleurhaut}{rgb}{0.85,0.9,1}  % creme
\definecolor{couleurmilieu}{rgb}{1,1,1}  % vert pale
\definecolor{couleurbas}{rgb}{0.85,0.9,1}  % blanc
\setbeamertemplate{background canvas}[vertical shading]%
[top=couleurhaut,middle=couleurmilieu,midpoint=0.4,bottom=couleurbas] 
%[top=fondtitre!05,bottom=fondtitre!60]



\makeatletter
\setbeamertemplate{theorem begin}
{%
  \begin{\inserttheoremblockenv}
  {%
    \inserttheoremheadfont
    \inserttheoremname
    \inserttheoremnumber
    \ifx\inserttheoremaddition\@empty\else\ (\inserttheoremaddition)\fi%
    \inserttheorempunctuation
  }%
}
\setbeamertemplate{theorem end}{\end{\inserttheoremblockenv}}

\newenvironment{theoreme}[1][]{%
   \setbeamercolor{block title}{fg=structure,bg=structure!40}
   \setbeamercolor{block body}{fg=black,bg=structure!10}
   \begin{block}{{\bf Th\'eor\`eme }#1}
}{%
   \end{block}%
}


\newenvironment{proposition}[1][]{%
   \setbeamercolor{block title}{fg=structure,bg=structure!40}
   \setbeamercolor{block body}{fg=black,bg=structure!10}
   \begin{block}{{\bf Proposition }#1}
}{%
   \end{block}%
}

\newenvironment{corollaire}[1][]{%
   \setbeamercolor{block title}{fg=structure,bg=structure!40}
   \setbeamercolor{block body}{fg=black,bg=structure!10}
   \begin{block}{{\bf Corollaire }#1}
}{%
   \end{block}%
}

\newenvironment{mydefinition}[1][]{%
   \setbeamercolor{block title}{fg=structure,bg=structure!40}
   \setbeamercolor{block body}{fg=black,bg=structure!10}
   \begin{block}{{\bf Définition} #1}
}{%
   \end{block}%
}

\newenvironment{lemme}[0]{%
   \setbeamercolor{block title}{fg=structure,bg=structure!40}
   \setbeamercolor{block body}{fg=black,bg=structure!10}
   \begin{block}{\bf Lemme}
}{%
   \end{block}%
}

\newenvironment{remarque}[1][]{%
   \setbeamercolor{block title}{fg=black,bg=structure!20}
   \setbeamercolor{block body}{fg=black,bg=structure!5}
   \begin{block}{Remarque #1}
}{%
   \end{block}%
}


\newenvironment{exemple}[1][]{%
   \setbeamercolor{block title}{fg=black,bg=structure!20}
   \setbeamercolor{block body}{fg=black,bg=structure!5}
   \begin{block}{{\bf Exemple }#1}
}{%
   \end{block}%
}


\newenvironment{miniexercice}[0]{%
   \setbeamercolor{block title}{fg=structure,bg=structure!20}
   \setbeamercolor{block body}{fg=black,bg=structure!5}
   \begin{block}{Mini-exercices}
}{%
   \end{block}%
}


\newenvironment{tp}[0]{%
   \setbeamercolor{block title}{fg=structure,bg=structure!40}
   \setbeamercolor{block body}{fg=black,bg=structure!10}
   \begin{block}{\bf Travaux pratiques}
}{%
   \end{block}%
}
\newenvironment{exercicecours}[1][]{%
   \setbeamercolor{block title}{fg=structure,bg=structure!40}
   \setbeamercolor{block body}{fg=black,bg=structure!10}
   \begin{block}{{\bf Exercice }#1}
}{%
   \end{block}%
}
\newenvironment{algo}[1][]{%
   \setbeamercolor{block title}{fg=structure,bg=structure!40}
   \setbeamercolor{block body}{fg=black,bg=structure!10}
   \begin{block}{{\bf Algorithme}\hfill{\color{gray}\texttt{#1}}}
}{%
   \end{block}%
}


\setbeamertemplate{proof begin}{
   \setbeamercolor{block title}{fg=black,bg=structure!20}
   \setbeamercolor{block body}{fg=black,bg=structure!5}
   \begin{block}{{\footnotesize Démonstration}}
   \footnotesize
   \smallskip}
\setbeamertemplate{proof end}{%
   \end{block}}
\setbeamertemplate{qed symbol}{\openbox}


\makeatother
% Couleur à définir

   
%%%%%%%%%%%%%%%%%%%%%%%%%%%%%%%%%%%%%%%%%%%%%%%%%%%%%%%%%%%%%
%%%%%%%%%%%%%%%%%%%%%%%%%%%%%%%%%%%%%%%%%%%%%%%%%%%%%%%%%%%%%


\begin{document}


\title{{\bf Séries}}
\subtitle{Définitions -- Série géométrique}

\begin{frame}
  
  \debutmontitre

  \pause

{\footnotesize
\hfill
\setbeamercovered{transparent=50}
\begin{minipage}{0.6\textwidth}
  \begin{itemize}
    \item<3-> Définitions
    \item<4-> Série géométrique
    \item<5-> Séries convergentes
    \item<6-> Suites et séries
    \item<7-> Le terme d'une série convergente tend vers $0$
  \end{itemize}
\end{minipage}
}

\end{frame}

\setcounter{framenumber}{0}



%%%%%%%%%%%%%%%%%%%%%%%%%%%%%%%%%%%%%%%%%%%%%%%%%%%%%%%%%%%%%%%%
\section{Introduction}

\begin{frame}

\begin{itemize}
\item Nous nous intéressons à des sommes ayant une infinité de termes
\item\pause Par exemple :
\[1+\frac{1}{2} +\frac{1}{4} +\frac{1}{8}+\frac{1}{16} +\cdots \ \ = \ \ ? \]

\item\pause C'est le \evidence{paradoxe de Zénon}
\pause
\myfigure{1.3}{
\tikzinput{fig_series01} 
}

\item\pause On ajoute ainsi une infinité de durées non nulles : la flèche n'atteindrait jamais sa cible ?!
\item\pause La somme d'une infinité de termes peut être une valeur finie
\end{itemize}

\end{frame}


%%%%%%%%%%%%%%%%%%%%%%%%%%%%%%%%%%%%%%%%%%%%%%%%%%%%%%%%%%%%%%%%
\section{Définitions}

\begin{frame}

\begin{mydefinition}
Soit $(u_k)_{k \ge 0}$ une suite de nombres réels (ou complexes). \pause
On pose
$$S_n=u_0+u_1+u_2+\cdots+ u_n\pause =\sum_{k=0}^n u_k$$

\pause
La suite $(S_n)_{n \ge 0}$ s'appelle la \defi{série} de terme général 
$u_k$\pause, notée  $\displaystyle \sum_{k \ge 0} u_k $

\pause
On l'appelle aussi \defi{suite des sommes partielles}
\end{mydefinition}

\pause
\begin{exemple}
Pour $q\in \Cc$, considérons la suite géométrique $(u_k)$ définie par
$u_k = q^k$

\pause
La \defi{série géométrique} $\displaystyle \sum_{k \ge 0} q^k$ est la suite des sommes partielles :
\pause\vspace{-.2cm}
\[
S_0 = 1, \pause \quad S_1 = 1 + q, \pause \quad S_2 = 1+q+q^2, 
\pause \ \ldots \ S_n = 1+q+q^2+\cdots+q^n, \ \ldots
\]
\vspace{-.4cm}
\end{exemple}

\end{frame}

%%%%%%%%%%%%%%

\begin{frame}

\begin{mydefinition}
\begin{itemize}
  \item Si la suite $(S_n)_{n \ge 0}$ admet une limite finie dans $\Rr$ (ou $\Cc$),
on note
\vspace*{-2ex}
\[
S = \sum_{k=0}^{+\infty} u_k =\lim_{n\to+\infty} S_n
\]
\vspace*{-2ex}
\pause
  
  \item On appelle alors $\displaystyle S= \sum_{k=0}^{+\infty} u_k$ la \defi{somme} de la série $\displaystyle \sum_{k \ge 0} u_k$ \pause 
et on dit que la série est \defi{convergente}

\pause
  
  \item Sinon on dit qu'elle est \defi{divergente}
\end{itemize}



\end{mydefinition}

\pause
\textbf{Notations}
\[
{\color<6>{red}
\sum_{k = 0}^{+\infty} u_k \qquad\sum_{i = 0}^{+\infty} u_i
} \qquad \qquad 
{\color<6>{blue}
\sum_{n \in \Nn} u_n \qquad  \sum_{k \ge 0} u_k \qquad \sum u_k
}
\]
\end{frame}


%%%%%%%%%%%%%%%%%%%%%%%%%%%%%%%%%%%%%%%%%%%%%%%%%%%%%%%%%%%%%%%%
\section{Série géométrique}

\begin{frame}
Soit $q\in \Cc$ \pause
\begin{proposition}
La série géométrique $\displaystyle\sum_{k \ge 0} q^k$ est convergente 
si et seulement si $|q|<1$.

\pause
Alors
\mybox{$\displaystyle \sum_{k=0}^{+\infty} q^k =1+q+q^2+q^3+\cdots= \frac{1}{1-q}$}
\end{proposition}

\end{frame}

%%%%%%%%%%%%%%

\begin{frame}

\begin{proof}
$S_n=1+q+q^2+q^3+\cdots+q^n$

\begin{itemize}
  \item \pause si $q=1$, $S_n = n+1$ \pause $\to +\infty$  \pause : la série $\sum_{k \ge 0} q^k$ diverge
  
  \item \pause si $q \neq 1$, multiplions $S_n$ par $1-q$ :  
  \pause
$$(1-q)S_n=(1+q+q^2+q^3+\cdots+q^n)-(q+q^2+q^3+\cdots+q^{n+1}) \pause =1-q^{n+1}$$
\pause
Donc
\mybox{$\displaystyle S_n  = \frac{1-q^{n+1}}{1-q}$}

\begin{itemize}
  \item \pause si $|q|<1$ \pause alors $q^n \to 0$ donc $q^{n+1} \to 0$ \pause et ainsi $S_n \to \frac{1}{1-q}$

\pause  
  ainsi la série converge
\item \pause si $|q| \ge 1$ \pause alors $(q^n)$ n'a pas de limite finie
(elle peut tendre vers $+\infty$ ou être divergente) \pause donc $(S_n)$ n'a pas de limite finie

\pause
ainsi la série diverge \qedhere
\end{itemize}
\end{itemize}
\end{proof}
\end{frame}

%%%%%%%%%%%%%%

\begin{frame}

\begin{exemple}
\begin{enumerate}
  \item\pause $q=\frac 12$ : \quad \pause
  $\displaystyle \sum_{k=0}^{+\infty} \frac{1}{2^k} = \pause \frac{1}{1-\frac12} \pause = 2$
  
  \pause
  Cela résout le paradoxe de Zénon 
\vspace{.2cm}
  
  \item\pause $q=\frac 13$ avec premier terme $\frac{1}{3^3}$ : \pause
  $\displaystyle \sum_{k=3}^{+\infty} \frac{1}{3^k} \pause = \sum_{k=0}^{+\infty} \frac{1}{3^k} \  -  1 - \frac 13-\frac 1{3^2}  \pause
  = \frac{1}{1-\frac13} - \frac{13}{9} \pause = \frac32-\frac{13}{9} = \frac{1}{18}$
\vspace{.2cm}
  
  \item\pause  \emph{Autre méthode :} \pause changement d'indice $n=k-3$

$\displaystyle\sum_{k=3}^{+\infty} \frac{1}{3^k} 
\pause= \sum_{n=0}^{+\infty} \frac{1}{3^{n+3}}
\pause= \sum_{n=0}^{+\infty} \frac{1}{3^3}\frac{1}{3^n}
\pause= \frac{1}{3^3}\sum_{n=0}^{+\infty} \frac{1}{3^n}
\pause= \frac{1}{27}\frac{1}{1-\frac13}
= \frac{1}{18}$
\vspace{.2cm}

  \item\pause $\displaystyle \sum_{k=0} ^{+\infty} (-1)^k \left(\frac{1}{2}\right)^{2k}
  \pause =\sum_{k=0} ^{+\infty} \left(-\frac{1}{4}\right)^k \pause=\frac{1}{1-\frac{-1}{4}} \pause=\frac{4}{5}$
 
\end{enumerate}
\end{exemple}

\end{frame}



%%%%%%%%%%%%%%%%%%%%%%%%%%%%%%%%%%%%%%%%%%%%%%%%%%%%%%%%%%%%%%%%
\section{Séries convergentes}


\begin{frame}

Le \defi{reste d'ordre~$n$} d'une série convergente $\displaystyle\sum_{k = 0}^{+\infty} u_k$ \pause est
$$R_n = u_{n+1}+u_{n+2}+\cdots = \sum_{k=n+1}^{+\infty} u_k$$

\pause
\begin{proposition}
Si une série est convergente, alors 
$S=S_n+R_n$ (pour tout $n\ge0$) \pause et 
$$\lim_{n\to+\infty} R_n=0$$
\end{proposition}

%\pause
%\begin{proof}
%\begin{itemize}
%  \item $S = \displaystyle\sum_{k=0}^{+\infty} u_k \pause=\sum_{k=0}^{n} u_k+ \sum_{k=n+1}^{+\infty} u_k=S_n+R_n$
%  \item\pause Donc $R_n = S-S_n \pause \to S-S =0$ lorsque $n\to+\infty$
%\end{itemize} 
%\end{proof}

\end{frame}


%%%%%%%%%%%%%%%%%%%%%%%%%%%%%%%%%%%%%%%%%%%%%%%%%%%%%%%%%%%%%%%%
\section{Suites et séries}

\begin{frame}

\begin{itemize}
\item Une série $\sum_{k \ge 0} u_k$ converge si la suite 
$(S_n)_{n\ge0}$ des sommes partielles converge, où $S_n=\sum_{k=0}^n u_k$

\item\pause Réciproquement considérons une suite $(a_k)_{k\ge0}$
\end{itemize}

\pause
\begin{proposition}
Une \defi{somme télescopique} est une série de la forme
$$\sum_{k\ge0} (a_{k+1}-a_k)$$

\pause
Cette série est convergente si et seulement si $(a_k)_{k\ge0}$ est convergente

\pause
Dans ce cas on a 
\vspace{-.2cm}
$$\sum_{k=0}^{+\infty} (a_{k+1}-a_k) = \lim_{k\to+\infty} a_k - a_0$$
\end{proposition}

\pause
\begin{proof}
$
S_n
= \sum_{k=0}^n (a_{k+1}-a_k)
= -a_0 +a_1-a_1+a_2-a_2+ \cdots +a_n-a_n+a_{n+1}
= a_{n+1}-a_0 $
\end{proof}
\end{frame}

%%%%%%%%%%%%%%

\begin{frame}
\begin{exemple}
\begin{itemize}
\item La série $$\sum_{k=0}^{+\infty} \frac{1}{(k+1)(k+2)}=\frac{1}{1\cdot 2}+\frac{1}{2\cdot 3}+
 \frac{1}{3\cdot 4}+\cdots$$
est convergente et sa somme vaut $1$

\item\pause En effet,
$$S_n=\sum_{k=0}^n \frac{1}{(k+1)(k+2)} = \sum_{k=0}^n \left(\frac{1}{k+1}-\frac{1}{k+2}\right)
\pause = 1 - \frac{1}{n+2} \to 1 $$

\item\pause $\displaystyle\sum_{k=1}^{+\infty} \frac{1}{k(k+1)}$ et 
$\displaystyle\sum_{k=2}^{+\infty} \frac{1}{k(k-1)}$ sont aussi convergentes de somme $1$ (changement d'indice)
\end{itemize}
\end{exemple}
\end{frame}

%%%%%%%%%%%%%%%%%%%%%%%%%%%%%%%%%%%%%%%%%%%%%%%%%%%%%%%%%%%%%%%%
\section{Le terme d'une série convergente tend vers $0$}

\begin{frame}
\begin{theoreme}
Si la série $\sum_{k\ge0} u_k$ converge, 
alors la suite $(u_k)_{k \ge 0}$ tend vers $0$
\end{theoreme}

\pause
Point clé : on retrouve le terme général $u_n$ à partir des sommes partielles $S_n=\sum_{k=0}^{n} u_k$ par
\vspace{-.2cm}
$$u_n = S_n - S_{n-1}$$

\pause
La contraposée de ce résultat est souvent utilisée : 
\mybox{Une série dont le terme général ne tend pas vers $0$ est divergente}

\begin{itemize}
\item\pause les séries $\sum_{k \ge 1} (1+\frac{1}{k})$ et $\sum_{k \ge 1} k^2$ sont divergentes
\item\pause Attention à l'énoncé ! Réciproque fausse

% la série $\sum u_k$ de terme général 
% $$
% u_k = \left\{
% \begin{array}{ll}
% 1&\text{ si } k=2^\ell \quad \text{ pour un certain } \ell \ge 0 \\
% 0&\text{ sinon }
% \end{array}
% \right.
% $$
% diverge
\end{itemize}
\end{frame}

%%%%%%%%%%%%%%%%%%%%%%%%%%%%%%%%%%%%%%%%%%%%%%%%%%%%%%%%%%%%%%%%
\section{Linéarité}

\begin{frame}
\begin{proposition}
$\sum a_k$ et $\sum b_k$ deux séries convergentes de sommes $A$ et $B$, et $\lambda, \mu \in \Rr$

\pause
Alors la série $\sum (\lambda a_k+\mu b_k)$ est convergente et de somme
 $\lambda A+\mu B$ :
 \pause \vspace{-.2cm}
 $$\sum_{k=0}^{+\infty} (\lambda a_k+\mu b_k) = 
 \lambda \sum_{k=0}^{+\infty} a_k+ \mu \sum_{k=0}^{+\infty} b_k$$
\end{proposition}

\pause
% Par exemple :
 \vspace{-.2cm}
$$
\sum_{k=0}^{+\infty} \left(\frac{1}{2^k}+\frac{5}{3^k}\right)
 \pause= 
\sum_{k=0}^{+\infty} \frac{1}{2^k}+
5\sum_{k=0}^{+\infty} \frac{1}{3^k}
\pause=
\frac{1}{1-\frac{1}{2}}+5\frac{1}{1-\frac{1}{3}} 
\pause= 2+5\frac{3}{2}=\frac{19}{2}
$$

% \pause
% \begin{proposition}
% Soit $(u_k)_{k\ge0}$, $u_k = a_k + \ii b_k$, avec $a_k, b_k\in\Rr$. \pause La série $\sum u_k$ converge si et seulement si $\sum a_k$ et $\sum b_k$ convergent. \pause On a alors :
%  \vspace{-.2cm}
% $$
% \sum_{k=0}^{+\infty} u_k = 
% \sum_{k=0}^{+\infty} a_k + \ii \sum_{k=0}^{+\infty} b_k
% $$
% \end{proposition}
\end{frame}

%%%%%%%%%%%%%%
%
%\begin{frame}
%\begin{exemple}
%Considérons la série géométrique $\sum_{k\ge0} r^k$ avec $r = \rho e^{\ii\theta}$ et $\rho<1$
%
%\begin{itemize}
%\item\pause Comme $\vert r \vert<1$, alors la série converge et 
%$\sum_{k=0}^{+\infty} r^k = \frac{1}{1-r}$
%
%\item\pause $r^k = \rho^k e^{\ii k\theta}$ \pause $\implies$ les parties réelle et imaginaire de $r^k$ sont
%$$
%a_k=\rho^k\cos(k\theta)
%\quad\text{ et }\quad 
%b_k=\rho^k\sin(k\theta)
%$$
%\item\pause par la proposition précédente
%\vspace{-.1cm}
%$$
%\sum_{k=0}^{+\infty} a_k  = \Re \left(\frac{1}{1-r}\right) 
%\pause\quad\text{ et }\quad 
%\sum_{k=0}^{+\infty} b_k  = \Im \left(\frac{1}{1-r}\right)
%$$
%\vspace{-.1cm}
%\item\pause Le calcul donne : 
%\end{itemize}
%$$
%\sum_{k=0}^{+\infty} \rho^k\cos(k\theta) = 
%\tfrac{1-\rho\cos\theta}{1+\rho^2-2\rho\cos\theta} 
%\pause\quad \text{ et }\quad
%\sum_{k=0}^{+\infty} \rho^k\sin(k\theta) = 
%\tfrac{\rho\sin\theta}{1+\rho^2-2\rho\cos\theta}
%$$ 
%\end{exemple}
%\end{frame}


%%%%%%%%%%%%%%%%%%%%%%%%%%%%%%%%%%%%%%%%%%%%%%%%%%%%%%%%%%%%%%%%
\section{Critère de Cauchy}

\begin{frame}
\textbf{Rappel.}
Une suite numérique $(s_n)$ converge 
si et seulement si elle est une suite de Cauchy :
\pause
$$\forall \epsilon>0\quad \exists n_0\in\Nn \quad \forall m,n \ge n_0 \qquad |s_n-s_m|<\epsilon$$

\pause
\begin{theoreme}[Critère de Cauchy]
Une série  $\displaystyle\sum_{k\geq0} u_k$ converge si et seulement si  
$$\forall \epsilon>0 \quad \exists n_0\in\Nn \quad \forall m,n\ge n_0 \qquad \big|u_{n}+\cdots +u_m\big| < \epsilon $$
\end{theoreme}

\pause
On le formule aussi de la façon suivante :
$$ \forall \epsilon>0 \quad \exists n_0\in\Nn \quad
\forall m,n \ge n_0 \qquad  \left| \sum_{k=n}^m u_k\right| <\epsilon$$
\pause
ou encore
$$ \forall \epsilon>0 \quad \exists n_0\in\Nn \quad
\forall n\ge n_0 \quad \forall p\in\Nn 
\qquad  \big|u_{n}+\cdots +u_{n+p}\big|<\epsilon$$

\end{frame}

%%%%%%%%%%%%%%

\begin{frame}

\mybox{La \defi{série harmonique} \quad $\displaystyle \sum_{k \ge 1} \frac{1}{k} = 1 + \frac12+\frac13+\frac14+\cdots$ \quad diverge}

\pause
\evidence{Attention !} La série $\sum u_k$ diverge, bien que $u_k = \frac{1}{k} \to 0$

\pause
\vspace{.2cm}
En effet, en notant la somme partielle $S_n = \sum_{k=1}^{n} \frac{1}{k}$
on a
$$
S_{2n}-S_{n} \pause= \frac{1}{n+1}+\cdots+\frac{1}{2n}\ge
\frac{n}{2n}=\frac{1}{2} 
$$
\pause
Donc la suite $\left(S_n\right)$ n'est pas de Cauchy : la série diverge


\end{frame}

%%%%%%%%%%%%%%
%
%\begin{frame}
%
%\begin{proposition}
%$$\text{ Pour } \displaystyle S_n = \sum_{k = 1}^n \frac{1}{k}  \;  ,  \quad  \text{ on a }\lim_{n\to+\infty} S_n=+\infty$$
%\end{proposition}
%
%\pause
%\begin{proof}
%\begin{itemize}
%\item Soit $M>0$ et $m\in \Nn$ tel que $m\ge 2M$. \pause
%Alors pour $n\ge 2^m$ on a:
%\end{itemize}
%\vspace{-.2cm}
%\begin{eqnarray*}
%S_n 
%  & = & 1+\frac{1}{2}+\frac{1}{3}+\cdots+\frac{1}{2^m}+\cdots+ \frac{1}{n}  
%\pause
%\quad  \ge \quad 1+\frac{1}{2}+\frac{1}{3}+\cdots+ \frac{1}{2^m}\\
%\pause
%  & \ge & 1 + \frac{1}{2}+ \left(\frac{1}{3}+\frac{1}{4}\right) 
%  +\left(\frac{1}{5}+\frac{1}{6}+\frac{1}{7}+ \frac{1}{8}\right) +  \cdots
%+\left(\frac{1}{2^{m-1}+1}+\cdots + \frac{1}{2^m}\right) \\
%\pause \pause
% & \ge & 1+\frac{1}{2}+2 \frac{1}{4}+ 4\;\frac{1}{8}+ \cdots +
%2^{m-1} \frac{1}{2^m} 
%\pause
%\quad = \quad 1+m \frac{1}{2} \quad \ge \quad M 
%\end{eqnarray*}
%\vspace{-.2cm}
%\begin{itemize}
%\item\vspace{-.2cm}\uncover<6->{Entre les parenthèses il 
%y a $2,4, \cdots$, $2^m-(2^{m-1}+1)+1= 
%2^m-2^{m-1}=2^{m-1} $ termes}
%\item\pause Ainsi $\forall M>0\quad \exists n_0 \ge 0 \quad \forall n \ge n_0 \quad S_n \ge M$
%
%Donc$(S_n)$ tend vers $+\infty$ \qedhere
%\end{itemize}
%\end{proof}
%
%\end{frame}
%
%%%%%
\begin{frame}

\begin{proposition}
$$\text{ Pour } \displaystyle S_n = \sum_{k = 1}^n \frac{1}{k}  \;  ,  \quad  \text{ on a }\lim_{n\to+\infty} S_n=+\infty$$
\end{proposition}

\pause
\begin{proof}
\begin{itemize}
\item Soit $M>0$ et $m\in \Nn$ tel que $m\ge 2M$. \pause
Alors pour $n\ge 2^m$ on a:
\end{itemize}
\vspace{-.2cm}
\begin{eqnarray*}
S_n 
  & = & 1+\frac{1}{2}+\frac{1}{3}+\cdots+\frac{1}{2^m}+\cdots+ \frac{1}{n}  
\uncover<4->{
\quad  \ge \quad 1+\frac{1}{2}+\frac{1}{3}+\cdots+ \frac{1}{2^m}}\\
\uncover<5->{
  & \ge & 1 + \frac{1}{2}+ \left(\frac{1}{3}+\frac{1}{4}\right) 
  +\left(\frac{1}{5}+\frac{1}{6}+\frac{1}{7}+ \frac{1}{8}\right) +  \cdots
+\left(\frac{1}{2^{m-1}+1}+\cdots + \frac{1}{2^m}\right) \\
}
\uncover<9->{
 & \ge & 1+\frac{1}{2}+2 \frac{1}{4}+ 4\;\frac{1}{8}+ \cdots +
2^{m-1} \frac{1}{2^m} 
}
\uncover<10->{
\quad = \quad 1+m \frac{1}{2} \quad \ge \quad M 
}
\end{eqnarray*}
\vspace{-.2cm}
\begin{itemize}
\item\vspace{-.2cm}\uncover<6->{Entre les parenthèses il 
y a $2$,} \uncover<7->{$4$,} \uncover<8->{$\cdots$, $2^m-(2^{m-1}+1)+1= 
2^m-2^{m-1}=2^{m-1} $ termes}
\item\uncover<11->{Ainsi $\forall M>0\quad \exists n_0 \ge 0 \quad \forall n \ge n_0 \quad S_n \ge M$

Donc $(S_n)$ tend vers $+\infty$} \qedhere
\end{itemize}
\end{proof}

\end{frame}

%%%%%%%%%%%%%%%%%%%%%%%%%%%%%%%%%%%%%%%%%%%%%%%%%%%%%%%%%%%%%%%%
\section{Mini-exercices}

\begin{frame}
\begin{miniexercice}
\begin{enumerate}
  
  \item Calculer les sommes partielles $S_n$ de la série de terme général 
  $\frac{1}{4^k}$, commençant à $k=1$. Cette série est-elle convergente ? 
  Si c'est possible, calculer la somme $S$ et les restes $R_n$
  
  \item Mêmes questions avec :
  
   $ \displaystyle\sum_{k\ge0} (-1)^k$ ; $ \displaystyle\sum_{k\ge0} 3^k$ ; $ \displaystyle\sum_{k\ge1} \frac{1}{10^k}$ ; $ \displaystyle\sum_{k\ge2} \exp(-k)$
  
 \item Pourquoi les séries suivantes sont-elles divergentes ?
  $ \displaystyle\sum_{k\ge1} \left( \frac1k + (-1)^k \right)$ ; $ \displaystyle\sum_{k\ge 0} \frac{k}{k+1}$ ; 
  $ \displaystyle\sum_{k\ge1} \frac{1}{2k}$ ;
  $ \displaystyle\sum _{k\ge1} k \cos(k)$ ; $ \displaystyle\sum_{k\ge1} \exp(\frac1k)$
    
   
  \item Calculer les sommes partielles de la série 
  $ \sum_{k\ge1} \ln\left( 1-\frac{1}{k+1} \right)$
  
  Cette série est-elle convergente ?
  
  \item Montrer que $\displaystyle \sum_{k=0}^{+\infty} k^2q^k = \frac{q^2+q}{(1-q)^3}$

\end{enumerate}
\end{miniexercice}
\end{frame}

\end{document}