\documentclass[class=report,crop=false]{standalone}
\usepackage[screen]{../exo7book}

\begin{document}

\newcommand{\GL}{\mathcal{G}\!\ell}
\newcounter{saveenumi}

%====================================================================
\chapitre{Groupes}
%====================================================================

\insertvideo{p8gSeXA-Pls}{partie 1. Définition}

\insertvideo{TKfhcZ7l9Qo}{partie 2. Sous-groupes}

\insertvideo{Y7bvfX10DPs}{partie 3. Morphismes de groupes}

\insertvideo{6wqgx1r1bJA}{partie 4. Le groupe $\Zz/n\Zz$}

\insertvideo{aX-ARKd1aaQ}{partie 5. Le groupe des permutations}



%%%%%%%%%%%%%%%%%%%%%%%%%%%%%%%%%%%%%%%%%%%%%%%%%%%%%%%%%%%%%%%%
\section*{Motivation}


\'Evariste Galois a tout juste vingt ans lorsqu'il meurt dans un duel.
Il restera pourtant comme l'un des plus grands mathématiciens de son temps
pour avoir introduit la notion de groupe, alors qu'il avait à peine dix-sept
ans.

Vous savez résoudre les équations de degré $2$ du type
$ax^2+bx+c=0$. Les solutions s'expriment en fonction de $a,b,c$
et de la fonction racine carrée $\sqrt{\ \vphantom{2}}$.
Pour les équations de degré $3$, $ax^3+bx^2+cx+d=0$, il existe
aussi des formules. Par exemple une solution de $x^3+3x+1=0$
est $x_0 = \sqrt[3]{\frac{\sqrt 5 - 1}{2}} -  \sqrt[3]{\frac{\sqrt 5 + 1}{2}}$.
De telles formules existent aussi
pour les équations de degré $4$.

Un préoccupation majeure au début du
\textsc{\romannumeral 19}\textsuperscript{e} siècle
était de savoir s'il existait des formules similaires pour les équations de degré $5$ ou plus.
La réponse fut apportée par Galois et Abel : non il n'existe pas en général une telle
 formule. Galois parvient même à dire pour quels polynômes c'est possible
et pour lesquels ça ne l'est pas.
Il introduit pour sa démonstration la notion de groupe.

\medskip

Les groupes sont à la base d'autres notions mathématiques comme
 les anneaux, les corps, les matrices, les espaces vectoriels,...
Mais vous les retrouvez aussi en arithmétique, en géométrie, en cryptographie !


\bigskip

Nous allons introduire dans ce chapitre la notion de groupe,
puis celle de sous-groupe. On étudiera ensuite les applications entre
deux groupes : les morphismes de groupes. Finalement nous détaillerons
deux groupes importants : le groupe $\Zz/n\Zz$ et le groupe des permutations
$\mathcal{S}_n$.


%%%%%%%%%%%%%%%%%%%%%%%%%%%%%%%%%%%%%%%%%%%%%%%%%%%%%%%%%%%%%%%%
\section{Groupe}
%---------------------------------------------------------------
\subsection{Définition}

\begin{definition}
Un \defi{groupe}\index{groupe} $(G,\star)$ est un ensemble $G$ auquel est associé une opération $\star$
(la \defi{loi de composition}\index{loi de composition}) vérifiant les quatre propriétés suivantes :
\begin{enumerate}
  \item \label{it:groupei} pour tout $x,y \in G$, \quad $x \star y \in G$
\quad  ($\star$ est une \defi{loi de composition interne})

  \item \label{it:groupeii} pour tout $x,y,z \in G$, \quad $(x \star y) \star z = x \star (y \star z)$
\quad (la loi est \defi{associative}\index{associativite@associativité})

  \item \label{it:groupeiii} il existe $e \in G$ tel que \quad $\forall x \in G$, $x \star e = x$ \  et \  $e \star x = x$
\quad ($e$ est l'\defi{élément neutre}\index{element neutre@élément neutre})

  \item \label{it:groupeiv} pour tout $x \in G$ il existe $x' \in G$ tel que \quad  $x \star x' = x' \star x = e$
\quad ($x'$ est l'\defi{inverse} de $x$ et est noté $x^{-1}$)
\end{enumerate}
\end{definition}

\bigskip

Si de plus l'opération vérifie
$$ \text{pour tout } x,y \in G,\qquad x \star y = y \star x,$$
on dit que $G$ est un groupe \defi{commutatif}\index{commutatif} (ou \defi{abélien}).

\bigskip

\begin{remarque*}
\sauteligne
\begin{itemize}
  \item L'élément neutre $e$ est unique. En effet si $e'$ vérifie aussi le point (\ref{it:groupeiii}), alors
on a $e' \star e = e$ (car $e$ est élément neutre) et $e' \star e = e'$ (car $e'$ aussi). Donc $e=e'$.
Remarquez aussi que l'inverse de l'élément neutre est lui-même.
S'il y a plusieurs groupes, on pourra noter $e_G$ pour l'élément neutre du groupe $G$.

  \item Un élément $x \in G$ ne possède qu'un seul inverse.
En effet si $x'$ et $x''$ vérifient tous les deux le point (\ref{it:groupeiv})
alors on a $x\star x'' = e$ donc $x' \star (x \star x'') = x' \star e$. Par l'associativité (\ref{it:groupeii})
et la propriété de l'élément neutre  (\ref{it:groupeiii}) alors $(x' \star x) \star x'' = x'$.
Mais $x'\star x = e$ donc $e \star x''= x'$ et ainsi $x''=x'$.
\end{itemize}
\end{remarque*}




%---------------------------------------------------------------
\subsection{Exemples}

%Voici des ensembles et des opérations bien connus qui ont une structure de groupe.

Voici des ensembles bien connus pour lesquels l'opération donnée définit une structure de groupe.

\begin{itemize}
  \item $(\Rr^*,\times)$ est un groupe commutatif, $\times$ est la multiplication habituelle.
Vérifions chacune des propriétés :
   \begin{enumerate}
     \item Si $x,y \in \Rr^*$ alors $x \times y \in \Rr^*$.
     \item Pour tout $x,y,z \in \Rr^*$ alors $x\times (y \times z) = (x \times y) \times z$, c'est l'associativité
de la multiplication des nombres réels.
     \item $1$ est l'élément neutre pour la multiplication, en effet $1 \times x = x$ et $x \times 1=x$, ceci
quelque soit $x \in \Rr^*$.
     \item L'inverse d'un élément $x \in \Rr^*$ est $x' = \frac{1}{x}$
(car $x \times \frac{1}{x}$ est bien égal à l'élément neutre $1$). L'inverse de $x$ est donc $x^{-1} = \frac{1}{x}$.
Notons au passage que nous avions exclu $0$ de notre groupe, car il n'a pas d'inverse.\\
Ces propriétés font de $(\Rr^*,\times)$ un groupe.
     \item Enfin $x \times y = y \times x$, c'est la commutativité de la multiplication des réels.
   \end{enumerate}


  \item $(\Qq^*,\times)$, $(\Cc^*, \times)$ sont des groupes commutatifs.

  \item $(\Zz,+)$ est un groupe commutatif. Ici $+$ est l'addition habituelle.
   \begin{enumerate}
     \item Si $x,y \in \Zz$ alors $x + y \in \Zz$.
     \item Pour tout $x,y,z \in \Zz$ alors $x +  (y + z) = (x + y) + z$.
     \item $0$ est l'élément neutre pour l'addition, en effet $0 + x = x$ et $x + 0=x$, ceci
quelque soit $x \in \Zz$.
     \item L'inverse d'un élément $x \in \Zz$ est $x' = -x$
car $x + (-x) = 0$ est bien l'élément neutre $0$. Quand la loi de groupe est $+$ l'inverse s'appelle
plus couramment l'\defi{opposé}.
     \item Enfin $x + y = y + x$, et donc $(\Zz,+)$ est un groupe commutatif.
   \end{enumerate}

  \item $(\Qq,+)$, $(\Rr,+)$, $(\Cc,+)$ sont des groupes commutatifs.

  \item Soit $\mathcal{R}$ l'ensemble des rotations du plan dont le centre est à l'origine $O$.

\myfigure{1.5}{
\tikzinput{fig_groupes02}
}


Alors pour deux rotations $R_\theta$ et $R_{\theta'}$ la composée $R_\theta \circ R_{\theta'}$
est encore une rotation de centre l'origine et d'angle $\theta+\theta'$. Ici $\circ$ est la composition. Ainsi $(\mathcal{R},\circ)$
forme un groupe (qui est même commutatif). Pour cette loi l'élément neutre est la rotation d'angle $0$ :
c'est l'identité du plan. L'inverse d'une rotation d'angle $\theta$ est la rotation d'angle $-\theta$.

  \item Si $\mathcal{I}$ désigne l'ensemble des isométries du plan (ce sont les translations, rotations,
réflexions et leurs composées) alors $(\mathcal{I},\circ)$ est un groupe. Ce groupe n'est pas un groupe commutatif.
En effet, identifions le plan à $\Rr^2$ et soit par exemple $R$ la rotation de centre $O=(0,0)$ et d'angle $\frac \pi 2$
et $T$ la translation de vecteur $(1,0)$. Alors les isométries $T \circ R$ et $R \circ T$ sont des applications
distinctes. Par exemple les images du point $A=(1,1)$ par ces applications sont distinctes :  $T \circ R (1,1) = T (-1,1) = (0,1)$ alors que $R\circ T(1,1) = R(2,1) = (-1,2)$.


\myfigure{1}{
\tikzinput{fig_groupes04}
\tikzinput{fig_groupes03}
}

\end{itemize}

\bigskip

Voici deux exemples qui \textbf{ne sont pas} des groupes :
\begin{itemize}
  \item $(\Zz^*,\times)$ n'est pas un groupe. Car si $2$ avait un inverse (pour la multiplication $\times$) ce serait
$\frac 12$ qui n'est pas un entier.
  \item $(\Nn,+)$ n'est pas un groupe. En effet l'inverse de $3$ (pour l'addition $+$) devrait être $-3$ mais $-3 \notin \Nn$.
\end{itemize}

\bigskip

Nous étudierons dans les sections \ref{sec:zsurnz} et \ref{sec:sn} deux autres groupes très
importants : les groupes cycliques $(\Zz/n\Zz,+)$ et les groupes de permutations $(\mathcal{S}_n,\circ)$.


%---------------------------------------------------------------
\subsection{Puissance}

Revenons à un groupe $(G,\star)$. Pour $x \in G$ nous noterons $x \star x$ par $x^2$
et $x\star x \star x$ par $x^3$. Plus généralement nous noterons :
\begin{itemize}
  \item $x^{n} = \underbrace{x\star x \star \cdots \star x}_{n \text{ fois}}$,
  \item $x^0 = e$,
  \item $x^{-n} = \underbrace{x^{-1}\star \cdots \star x^{-1}}_{n \text{ fois}}$.
\end{itemize}
Rappelez-vous que $x^{-1}$ désigne l'inverse de $x$ dans le groupe.

\bigskip

Les règles de calcul sont les mêmes que pour les puissances des nombres réels.
Pour $x,y \in G$ et $m,n \in\Zz$ nous avons :
\begin{itemize}
  \item $x^m \star x^n = x^{m+n}$,
  \item $(x^m)^n = x^{mn}$,
  \item $(x \star y)^{-1} = y^{-1} \star x^{-1}$, \quad attention à l'ordre !
  \item \textbf{Si} $(G,\star)$ est \textbf{commutatif} alors $(x\star y)^n = x^n \star y^n$.
\end{itemize}

%---------------------------------------------------------------
\subsection{Exemple des matrices $2\times 2$}

Une \defi{matrice}\index{matrice} $2\times 2$ est un tableau de $4$ nombres (pour nous des réels)
noté ainsi :
$$\begin{pmatrix} a & b \\ c & d \\ \end{pmatrix}.$$
Nous allons définir l'opération \defi{produit} noté $\times$ de deux matrices
$M=\left(\begin{smallmatrix} a & b \\ c & d \\ \end{smallmatrix}\right)$ et
$M'=\left(\begin{smallmatrix} a' & b' \\ c' & d' \\ \end{smallmatrix}\right)$ :
$$M\times M' = \left(\begin{matrix} a & b \\ c & d \\ \end{matrix}\right) \times
\left(\begin{matrix} a' & b' \\ c' & d' \\ \end{matrix}\right)
= \left(\begin{matrix} aa'+bc' & ab' + bd' \\ ca'+dc' & cb'+dd' \\ \end{matrix}\right).$$



Voici comment présenter les calculs, on place $M$ à gauche,
$M'$ au dessus de ce qui va être le résultat.
On calcule un par un, chacun des termes de $M \times M'$.

Pour le premier terme on prend la colonne située au dessus et la ligne
située à gauche :
on effectue les produits $a\times a'$ et $b\times c'$ qu'on additionne
pour obtenir le premier terme du résultat.
Même chose avec le second terme : on prend la colonne située au dessus, la ligne
située à gauche, on fait les produit, on additionne : $ab' + bd'$.
Idem pour les deux autres termes.

  $$
    \begin{tikzpicture}[node distance=4em, thick]
       \matrix(M)[matrix of math nodes, left delimiter=(,right delimiter=), font=\small]{
           aa'+bc' & ab' + bd' \\
         ca'+dc' &  cb'+dd'  \\
       };
       \matrix(A)[left of=M,xshift=-5em,matrix of math nodes, left delimiter=(,right delimiter=)]{
          a & b\\
        c & d\\
       };
       \matrix(B)[above of=M,matrix of math nodes, left delimiter=(,right delimiter=)]{
          a' & \qquad & b'\\
        c' &        & d'\\
       };
       \draw (A-1-1.north) edge[myred,bend left=35,<->] node[fill=white,draw,circle,inner sep=0pt](X){$\times$} (B-1-1.west);
       \draw[myred,dashed,->] (X) -- (M-1-1.155);
       \draw (A-1-2.north) edge[myred,bend left,<->] node[fill=white,draw,circle,inner sep=0pt]{$\times$} (B-2-1.west);
    \end{tikzpicture}
  $$

\bigskip

Par exemple si $M=\left(\begin{smallmatrix} 1 & 1 \\ 0 & -1 \\ \end{smallmatrix}\right)$
et $M' = \left(\begin{smallmatrix} 1 & 0 \\ 2 & 1 \\ \end{smallmatrix}\right)$
alors voici comment poser les calculs ($M\times M'$ à gauche, $M'\times M$ à droite)
$$
\begin{array}{cc}
  & \left(\begin{matrix} \ms 1 & \ms 0 \;\\ \ms 2 & \ms 1 \;\\ \end{matrix}\right)\\
  \left(\begin{matrix} \ms 1 & \ms 1 \;\\ \ms 0 & -1 \;\\ \end{matrix}\right)  &
  \pause
  \left(\begin{matrix} \ms 3 & \ms 1 \;\\ -2 & -1 \;\\ \end{matrix}\right)
\end{array}
\qquad \qquad \qquad
\begin{array}{cc}
  & \left(\begin{matrix} \ms 1 & \ms 1 \;\\ \ms 0 & -1 \;\\ \end{matrix}\right) \\
  \left(\begin{matrix} \ms 1 & \ms 0 \;\\ \ms 2 & \ms 1 \;\\ \end{matrix}\right) &
  \pause
  \left(\begin{matrix} \ms 1 & \ms 1 \;\\ \ms 2 & \ms 1 \;\\ \end{matrix}\right) \\
\end{array}
$$
alors $M\times M' = \left(\begin{smallmatrix} 3 & 1 \\ -2 & -1 \\ \end{smallmatrix}\right)$
et $M'\times M=  \left(\begin{smallmatrix} 1 & 1 \\ 2 & 1 \\ \end{smallmatrix}\right)$.
Remarquez qu'en général $M\times M' \neq M'\times M$.

\bigskip

Le \defi{déterminant}\index{determinant@déterminant} d'une matrice $M=\left(\begin{smallmatrix} a & b \\ c & d \\ \end{smallmatrix}\right)$
est par définition le nombre réel
$$\det M = ad-bc.$$

\begin{proposition}
L'ensemble  des matrices $2\times 2$ ayant un déterminant non nul,
muni de la multiplication des matrices $\times$, forme un groupe non-commutatif.
\end{proposition}

Ce groupe est noté $(\GL_2,\times)$.


\bigskip

Nous aurons besoin d'un résultat préliminaire :
\begin{lemme}
\label{lem:det}
$\det (M\times M') = \det M \cdot \det M'$.
\end{lemme}

Pour la preuve, il suffit de vérifier le calcul :
$\big(aa'+bc'  \big) \big( cb'+dd' \big) - \big( ab' + bd' \big) \big( ca'+dc' \big) = (ad-bc)(a'd'-b'c')$.

Revenons à la preuve de la proposition.
\begin{proof}
~
\begin{enumerate}
  \item Vérifions la loi de composition interne.
Si $M,M'$ sont des matrices $2\times 2$ alors $M\times M'$ aussi.
Maintenant si $M$ et $M'$ sont de déterminants non nuls alors $\det (M\times M') = \det M \cdot \det M'$
est aussi non nul. Donc si $M,M' \in \GL_2$ alors $M\times M' \in \GL_2$.

  \item Pour vérifier que la loi est associative, c'est un peu fastidieux. Pour trois matrices
$M, M', M''$ quelconques il faut montrer $(M\times M')\times M''=M \times (M'\times M'')$.
Faites-le pour vérifier que vous maîtrisez le produit de matrices.

  \item Existence de l'élément neutre. La \defi{matrice identité}
$I = \left(\begin{smallmatrix} 1 & 0 \\ 0 & 1 \\ \end{smallmatrix}\right)$
est l’élément neutre pour la multiplication des matrices : en effet
$\left(\begin{smallmatrix} a & b \\ c & d \\ \end{smallmatrix}\right) \times
\left(\begin{smallmatrix} 1 & 0 \\ 0 & 1 \\ \end{smallmatrix}\right) =
\left(\begin{smallmatrix} a & b \\ c & d \\ \end{smallmatrix}\right)$
et
$\left(\begin{smallmatrix} 1 & 0 \\ 0 & 1 \\ \end{smallmatrix}\right) \times
\left(\begin{smallmatrix} a & b \\ c & d \\ \end{smallmatrix}\right) =
\left(\begin{smallmatrix} a & b \\ c & d \\ \end{smallmatrix}\right)$.

  \item Existence de l'inverse. Soit $M=\left(\begin{smallmatrix} a & b \\ c & d \\ \end{smallmatrix}\right)$
une matrice de déterminant non nul alors
$M^{-1} = \frac{1}{ad-bc} \left(\begin{smallmatrix} d & -b \\ -c & a\\ \end{smallmatrix}\right)$
 est l'inverse de $M$ : vérifiez que $M \times M^{-1} = I$ et que $M^{-1} \times M = I$.

  \item Enfin nous avons déjà vu que cette multiplication n'est pas commutative.
\end{enumerate}


\end{proof}

%---------------------------------------------------------------
%\subsection{Mini-exercices}

\begin{miniexercices}
\sauteligne
\begin{enumerate}

 \item Montrer que $(\Rr_+^*,\times)$ est un groupe commutatif.

 \item Soit $f_{a,b} : \Rr \to \Rr$ la fonction définie par $x \mapsto ax+b$.
Montrer que l'ensemble $\mathcal{F}=\{ f_{a,b} \mid a \in \Rr^*, b\in\Rr\}$ muni
de la composition \og $\circ$ \fg{}  est un groupe non commutatif.

 \item (Plus dur) Soit $G=]-1,1[$. Pour $x,y\in G$ on définit $x\star y = \frac{x+y}{1+xy}$.
Montrer que $(G,\star)$ forme un groupe en (a) montrant que $\star$ est une loi de composition interne :
$x\star y \in G$ ; (b) montrant que la loi est associative ; (c) montrant que $0$ est élément neutre ;
(d) trouvant l'inverse de $x$.

 \setcounter{saveenumi}{\theenumi}
\end{enumerate}

\bigskip

Soit $(G,\star)$ un groupe quelconque ; $x,y,z$ sont des éléments de $G$.
\begin{enumerate}
  \setcounter{enumi}{\thesaveenumi}
  \item Montrer que si $x\star y = x \star z$ alors $y = z$.
  \item Que vaut $\big(x^{-1}\big)^{-1}$ ?
  \item Si $x^n=e$, quel est l'inverse de $x$ ?
  \setcounter{saveenumi}{\theenumi}
\end{enumerate}

\bigskip

Matrices :
\begin{enumerate}
  \setcounter{enumi}{\thesaveenumi}

  \item Soient
$M_1= \left(\begin{smallmatrix} 0 & -1 \\ 1 & 0 \\ \end{smallmatrix}\right)$,
$M_2= \left(\begin{smallmatrix} 1 & 2 \\ 1 & 0 \\ \end{smallmatrix}\right)$,
$M_3= \left(\begin{smallmatrix} 1 & 2 \\ 3 & 4 \\ \end{smallmatrix}\right)$.
Vérifier que $M_1 \times (M_2 \times M_3)= (M_1 \times M_2) \times M_3$.

  \item Calculer $(M_1\times M_2)^2$ et $M_1^2 \times M_2^2$. (Rappel : $M^2=M\times M$)

  \item Calculer le déterminant des $M_i$ ainsi que leur inverse.

  \item Montrer que l'ensemble des matrices $2\times 2$ muni de l'addition $+$ définie par
$\left(\begin{smallmatrix} a & b \\ c & d \\ \end{smallmatrix}\right)
+ \left(\begin{smallmatrix} a' & b' \\ c' & d' \\ \end{smallmatrix}\right)
= \left(\begin{smallmatrix} a+a' & b+b' \\ c+c' & d+d' \\ \end{smallmatrix}\right)$ forme
un groupe commutatif.


\end{enumerate}
\end{miniexercices}


%%%%%%%%%%%%%%%%%%%%%%%%%%%%%%%%%%%%%%%%%%%%%%%%%%%%%%%%%%%%%%%%
\section{Sous-groupes}

Montrer qu'un ensemble est un groupe à partir de la définition peut être
assez long. Il existe une autre technique, c'est de montrer qu'un sous-ensemble
 d'un groupe est lui-même un groupe : c'est la notion de sous-groupe.

%---------------------------------------------------------------
\subsection{Définition}


Soit $(G,\star)$ un groupe.
\begin{definition}
Une partie $H \subset G$ est un  \defi{sous-groupe}\index{sous-groupe} de $G$ si :
\begin{itemize}
  \item $e \in H$,
  \item pour tout $x, y \in H$, on a $x \star y \in H$,
  \item pour tout $x \in H$, on a $x^{-1} \in H$.
\end{itemize}
\end{definition}

Notez qu'un sous-groupe $H$ est aussi un groupe $(H,\star)$
avec la loi induite par celle de $G$.

Par exemple si $x\in H$ alors, pour tout $n\in \Zz$, nous avons
$x^n \in H$.

\begin{remarque*}
Un critère pratique et plus rapide pour prouver que $H$ est un sous-groupe de $G$ est:
\begin{itemize}
  \item $H$ contient au moins un élément
  \item pour tout $x,y \in H$, $x \star y^{-1} \in H$.
\end{itemize}
\end{remarque*}


%---------------------------------------------------------------
\subsection{Exemples}


\begin{itemize}
  \item $(\Rr^*_+,\times)$ est un sous-groupe de $(\Rr^*,\times)$.
En effet :
  \begin{itemize}
     \item $1 \in \Rr^*_+$,
     \item si $x,y \in \Rr^*_+$ alors $x\times y \in \Rr^*_+$,
     \item si $x \in \Rr^*_+$ alors $x^{-1} = \frac 1 x \in \Rr^*_+$.
   \end{itemize}
  \item $(\mathbb{U},\times)$ est un sous-groupe de $(\Cc^*,\times)$, où
$\mathbb{U} = \{ z \in \Cc \mid |z|=1 \}$.
  \item $(\Zz,+)$ est un sous-groupe de $(\Rr,+)$.
  \item $\{e\}$ et $G$ sont les \defi{sous-groupes triviaux}\index{sous-groupe!trivial} du groupe $G$.
  \item L'ensemble $\mathcal{R}$ des rotations du plan dont le centre est à l'origine
est un sous-groupe du groupe des isométries $\mathcal{I}$.
  \item L'ensemble des matrices diagonales $\left(\begin{smallmatrix} a & 0 \\ 0 & d \\ \end{smallmatrix}\right)$
avec $a\neq 0$ et $d\neq 0$ est un sous-groupe de $(\GL_2,\times)$.
\end{itemize}


%---------------------------------------------------------------
\subsection{Sous-groupes de $\Zz$}

\begin{proposition}
Les sous-groupes de $(\Zz,+)$ sont les $n\Zz$, pour $n\in \Zz$.
\end{proposition}

L'ensemble $n\Zz$ désigne l'ensemble des multiples de $n$ :
$$n\Zz = \bigg\{ k\cdot n \mid  k \in \Zz \bigg\}.$$
Par exemple:
\begin{itemize}
  \item $2\Zz = \{\ldots, -4,-2,0,+2,+4,+6,\ldots\}$ est l'ensemble des entiers pairs,
  \item $7\Zz =  \{\ldots, -14,-7,0,+7,+14,+21,\ldots\}$ est l'ensemble des multiples de $7$.
\end{itemize}

\begin{proof}

Fixons $n \in \Zz$. L'ensemble $n\Zz$ est un sous-groupe de $(\Zz,+)$, en effet :
\begin{itemize}
 \item $n\Zz \subset \Zz$,
 \item l'élément neutre $0$ appartient à $n\Zz$,
 \item pour $x = kn$ et $y=k'n$ des éléments de $n\Zz$ alors $x+y = (k+k')n$ est aussi un
élément de $n\Zz$,
 \item enfin si $x=kn$ est un élément de $n\Zz$ alors $-x=(-k)n$ est aussi un élément de $n\Zz$.
\end{itemize}


\bigskip

Réciproquement soit $H$ un sous-groupe de $(\Zz,+)$.
Si $H=\{0\}$ alors $H = 0\Zz$ et c'est fini. Sinon $H$ contient au moins un élément non-nul et positif (puisque tout élément est accompagné de son opposé) et
notons
$$n=\min \big\{ h > 0 \mid h \in H \big\}.$$
Alors $n>0$.
Comme $n\in H$ alors $-n \in H$, $2n=n+n \in H$, et plus généralement pour $k \in \Zz$ alors $kn \in H$.
Ainsi $n\Zz \subset H$. Nous allons maintenant montrer l'inclusion inverse.
Soit $h \in H$. \'Ecrivons la division euclidienne :
$$h = kn + r,\qquad \text{avec } k,r \in \Zz  \ \text{ et } \  0 \le r < n.$$
Mais $h\in H$ et $kn \in H$ donc
$r= h - kn \in H$. Nous avons un entier $r \ge 0$ qui est un élément de
$H$ et strictement plus petit que $n$. Par la définition de $n$, nécessairement
$r=0$. Autrement dit $h=kn$ et donc $h\in n\Zz$.
Conclusion $H=n\Zz$.
\end{proof}



%---------------------------------------------------------------
\subsection{Sous-groupes engendrés}


Soit $(G,\star)$ un groupe et $E \subset G$ un sous-ensemble de $G$.
Le \defi{sous-groupe engendré}\index{sous-groupe!engendré} par $E$ est le plus petit sous-groupe de
$G$ contenant $E$.

Par exemple si $E = \{2\}$ et le groupe est $(\Rr^*,\times)$,
le sous-groupe engendré par $E$ est $H = \{ 2^n \mid n \in \Zz\}$.
Pour le prouver : il faut montrer que $H$ est un sous-groupe, que
$2 \in H$, et que si $H'$ est un autre sous-groupe contenant $2$ alors $H \subset H'$.

Autre exemple avec le groupe  $(\Zz,+)$ : si $E_1= \{2\}$ alors le sous-groupe
engendré par $E_1$ est $H_1 = 2\Zz$.
Si $E_2 = \{8,12\}$ alors $H_2 = 4\Zz$ et plus généralement
si $E = \{a,b\}$ alors $H = n\Zz$ où $n=\pgcd(a,b)$.


%---------------------------------------------------------------
%\subsection{Mini-exercices}

\begin{miniexercices}
\sauteligne
\begin{enumerate}
  \item Montrer que $\{ 2^{n} \mid n \in \Zz \}$ est un sous-groupe de $(\Rr^*,\times)$.
  \item Montrer que si $H$ et $H'$ sont deux sous-groupes de $(G,\star)$
alors $H\cap H'$ est aussi un sous-groupe.
  \item Montrer que $5\Zz \cup 8\Zz$ n'est \emph{pas} un sous-groupe de $(\Zz,+)$.
  \item Montrer que l'ensemble des matrices $2\times 2$ de déterminant $1$ ayant leurs coefficients dans $\Zz$
est un sous-groupe de $(\GL_2,\times)$.
  \item Trouver le sous-groupe de $(\Zz,+)$ engendré par $\{ -12, 8, 20 \}$.
\end{enumerate}
\end{miniexercices}


%%%%%%%%%%%%%%%%%%%%%%%%%%%%%%%%%%%%%%%%%%%%%%%%%%%%%%%%%%%%%%%%

\section{Morphismes de groupes}

%---------------------------------------------------------------
\subsection{Définition}

\begin{definition}
Soient $(G,\star)$ et $(G',\diamond)$ deux groupes. Une application
$f : G \longrightarrow G'$ est un \defi{morphisme de groupes}\index{morphisme} si :
\mybox{$\text{pour tout } x,x' \in G  \qquad f(x \star x') = f(x) \diamond f(x')$}
\end{definition}



L'exemple que vous connaissez déjà est le suivant :
soit $G$ le groupe $(\Rr,+)$ et $G'$ le groupe $(\Rr_+^*,\times)$.
Soit $f : \Rr \longrightarrow \Rr_+^*$ l'application exponentielle définie par
$f(x)= \exp(x)$. Nous avons bien
$$f(x+x') =\exp(x+x')= \exp(x) \times \exp(x') = f(x) \times f(x').$$
Et donc $f$ est bien un morphisme de groupes.


%---------------------------------------------------------------
\subsection{Propriétés}


\begin{proposition}
Soit $f : G \longrightarrow G'$ un morphisme de groupes alors :
\begin{itemize}
  \item $f(e_G) = e_{G'}$,
  \item pour tout $x \in G$, $f(x^{-1}) = \big(f(x)\big)^{-1}$.
\end{itemize}
\end{proposition}


Il faut faire attention à l'ensemble auquel appartiennent les éléments considérés :
%où \og habitent \fg{} les objets :
$e_G$ est l'élément neutre de $G$, $e_{G'}$ celui de $G'$.
Il n'y a pas de raison qu'ils soient égaux (ils ne sont même pas dans le même
ensemble).
Aussi $x^{-1}$ est l'inverse de $x$ dans $G$, alors
que $\big(f(x)\big)^{-1}$ est l'inverse de $f(x)$ mais dans $G'$.

Reprenons l'exemple de la fonction $f : \Rr \longrightarrow \Rr_+^*$ définie par
$f(x)= \exp(x)$. Nous avons bien $f(0)=1$ : l'élément neutre de $(\Rr,+)$
a pour image l'élément neutre de $(\Rr_+^*,\times)$.
Pour $x\in \Rr$ son inverse dans $(\Rr,+)$ est ici son opposé $-x$,
alors $f(-x)= \exp(-x)= \frac{1}{\exp(x)}=\frac{1}{f(x)}$ est bien l'inverse (dans $(\Rr_+^*,\times)$)
de $f(x)$.

\begin{proof}
~
\begin{itemize}
  \item $f(e_G)=f(e_G\star e_G)= f(e_G)\diamond f(e_G)$, en multipliant (à droite par exemple) par
$f(e_G)^{-1}$ on obtient $e_{G'} = f(e_G)$.
  \item Soit $x\in G$ alors $x\star x^{-1} = e_G$ donc
$f(x \star x^{-1}) = f(e_G)$. Cela entraîne $f(x) \diamond f(x^{-1}) = e_{G'}$,
en composant à gauche par  $\big(f(x)\big)^{-1}$, nous obtenons
$f(x^{-1}) = \big(f(x)\big)^{-1}$.
\end{itemize}
\end{proof}



\begin{proposition}
\label{prop:morbij}
\sauteligne
\begin{itemize}
  \item Soient deux morphismes de groupes $f : G \longrightarrow G'$ et $g : G'  \longrightarrow G''$.
Alors $g \circ f : G \longrightarrow G''$ est un morphisme de groupes.
  \item Si $f : G \longrightarrow G'$ est un morphisme bijectif alors $f^{-1} : G' \longrightarrow G$
est aussi un morphisme de groupes.
\end{itemize}
\end{proposition}

\begin{proof}
La première partie est facile. Montrons la deuxième :
Soit $y,y'\in G'$. Comme $f$ est bijective, il existe $x,x'\in G$ tels que $f(x)=y$ et $f(x')=y'$.
Alors $f^{-1}(y\diamond y') = f^{-1}\big(f(x)\diamond f(x') \big)
= f^{-1}\big(f(x \star x') \big) = x\star x' = f^{-1}(y)\star f^{-1}(y')$. Et donc $f^{-1}$
est un morphisme de $G'$ vers $G$.
\end{proof}

\begin{definition}
Un morphisme bijectif est un \defi{isomorphisme}\index{isomorphisme}. Deux groupes $G, G'$ sont \defi{isomorphes}
s'il existe un morphisme bijectif $f : G \longrightarrow G'$.
\end{definition}

Continuons notre exemple $f(x) = \exp(x)$, $f : \Rr \longrightarrow \Rr_+^*$ est une application bijective.
Sa bijection réciproque $f^{-1} :  \Rr_+^* \longrightarrow \Rr$ est définie par $f^{-1}(x) = \ln(x)$.
Par la proposition \ref{prop:morbij} nous savons que $f^{-1}$ est aussi un morphisme
(de $(\Rr_+^*,\times)$ vers $(\Rr,+)$) donc $f^{-1}(x\times x')=f^{-1}(x) + f^{-1}(x')$.
Ce qui s'exprime ici par la formule bien connue :
$$\ln(x \times x') = \ln(x) + \ln(x').$$
Ainsi $f$ est un isomorphisme et les groupes $(\Rr,+)$ et $(\Rr_+^*,\times)$ sont
isomorphes.

%---------------------------------------------------------------
\subsection{Noyau et image}

Soit $f: G \longrightarrow G'$ un morphisme de groupes.
Nous définissons deux sous-ensembles importants qui vont être des sous-groupes.



\begin{definition}
Le \defi{noyau}\index{noyau} de $f$ est
\mybox{$\Ker f = \big\{ x \in G \mid f(x) = e_{G'} \big\}$}
\end{definition}

C'est donc un sous-ensemble de $G$.
En terme d'image réciproque nous avons par définition $\Ker f = f^{-1}\big(\{ e_{G'} \}\big)$.
(Attention, la notation $f^{-1}$ ici désigne l'image réciproque, et ne signifie pas que $f$ est bijective.)
Le noyau est donc l'ensemble des éléments de $G$ qui s'envoient
par $f$ sur l'élément neutre de $G'$.

\begin{definition}
L'\defi{image}\index{image} de $f$ est
\mybox{$\Im f = \big\{ f(x)  \mid x \in G \big\}$}
\end{definition}

C'est donc un sous-ensemble de $G'$ et en terme d'image directe
nous avons $\Im f = f(G)$.
Ce sont les éléments de $G'$ qui ont (au moins) un antécédent par $f$.


\begin{proposition}
Soit $f: G \longrightarrow G'$ un morphisme de groupes.
\begin{enumerate}
  \item $\Ker f$ est un sous-groupe de $G$.
  \item $\Im f$ est un sous-groupe de $G'$.
  \item $f$ est injectif si et seulement si $\Ker f = \{ e_G \}$.
  \item $f$ est surjectif si et seulement si $\Im f = G'$.
\end{enumerate}
\end{proposition}

\begin{proof}
~
\begin{enumerate}
  \item Montrons que le noyau est un sous-groupe de $G$.
   \begin{enumerate}
     \item $f(e_G)=e_{G'}$ donc $e_G \in \Ker f$.
     \item Soient $x,x' \in \Ker f$. Alors $f(x\star x') = f(x) \diamond f(x')
= e_{G'} \diamond e_{G'} = e_{G'}$ et donc $x \star x' \in \Ker f$.
     \item Soit $x \in \Ker f$. Alors $f(x^{-1}) = f(x)^{-1}=e_{G'}^{-1} = e_{G'}$. Et donc $x^{-1} \in \Ker f$.
    \end{enumerate}

  \item Montrons que l'image est un sous-groupe de $G'$.
    \begin{enumerate}
     \item $f(e_G)=e_{G'}$ donc $e_{G'} \in \Im f$.
     \item Soient $y,y' \in \Im f$. Il existe alors $x,x' \in G$ tels que $f(x)=y$, $f(x')=y'$.
Alors $y \diamond y' = f(x)\diamond f(x')=f(x\star x') \in \Im f$.
     \item Soit $y \in \Im f$ et $x \in G$ tel que $y=f(x)$.
Alors $y^{-1} = f(x)^{-1} = f(x^{-1}) \in \Im f$.
    \end{enumerate}

  \item Supposons $f$ injective. Soit $x\in \Ker f$, alors
$f(x)=e_{G'}$ donc $f(x)=f(e_G)$ et comme $f$ est injective alors $x=e_G$. Donc $\Ker f = \{ e_G \}$.
Réciproquement supposons $\Ker f = \{ e_G \}$. Soient $x,x' \in G$ tels que $f(x)=f(x')$
donc $f(x)\diamond \big( f(x') \big)^{-1} = e_{G'}$, d'où $f(x)\diamond f(x'^{-1}) = e_{G'}$ et
donc $f(x \star x'^{-1}) = e_{G'}$. Ceci implique que $x \star x'^{-1} \in \Ker f$.
Comme $\Ker f = \{ e_G \}$ alors $x \star x'^{-1} = e_G$ et donc $x=x'$. Ainsi $f$ est injective.

  \item C'est clair !
\end{enumerate}
\end{proof}


%---------------------------------------------------------------
\subsection{Exemples}


\begin{exemple}
\sauteligne
\begin{enumerate}
  \item Soit $f : \Zz \longrightarrow \Zz$ définie par $f(k)=3k$.
$(\Zz,+)$ est considéré comme ensemble de départ et d'arrivée de l'application.
 Alors $f$ est un morphisme du groupe $(\Zz,+)$ dans lui-même car
$f(k+k')= 3(k+k')= 3k + 3k'=f(k)+f(k')$. Calculons le noyau : $\Ker f = \{ k \in \Zz \mid f(k)=0\}$.
Mais si $f(k)=0$ alors $3k=0$ donc $k=0$. Ainsi $\Ker f = \{ 0 \}$ est réduit à l'élément neutre et donc $f$ est injective.
Calculons maintenant l'image $\Im f = \{ f(k) \mid k \in \Zz \} = \{ 3k \mid k \in\Zz\} = 3\Zz$.
Nous retrouvons que $3\Zz$ est un sous-groupe de $(\Zz,+)$.

Plus généralement si l'on fixe $n \in \Zz$, $n\neq 0$, et que $f$ est définie par $f(k)=k \cdot n$ alors
$\Ker f = \{0\}$ et $\Im f = n\Zz$.

  \item Soient les groupes $(\Rr,+)$ et $(\mathbb{U},\times)$ (où $\mathbb{U} = \{ z \in \Cc \mid |z|=1 \}$)
et $f$ l'application $f : \Rr \longrightarrow \mathbb{U}$ définie par $f(t)=e^{\ii t}$.
Montrons que $f$ est un morphisme : $f(t+t')=e^{\ii (t+t')}= e^{\ii t} \times e^{\ii t'} = f(t)\times f(t')$.
Calculons le noyau $\Ker f = \{ t \in \Rr \mid f(t)=1 \}$. Mais si $f(t)=1$ alors $e^{\ii t}= 1$ donc  $t= 0
\pmod{2\pi}$. D'où $\Ker f = \{ 2k\pi \mid k \in \Zz\} = 2\pi \Zz$. Ainsi $f$ n'est pas injective.
L'image de $f$ est $\mathbb{U}$ car tout nombre complexe de module $1$ s'écrit sous la forme $f(t)=e^{\ii t}$.

  \item Soient les groupes $(\GL_2,\times)$ et $(\Rr^*,\times)$ et $f : \GL_2 \longrightarrow \Rr^*$ définie par
$f(M)=\det M$. Alors la formule vue plus haut (lemme \ref{lem:det}) $\det(M\times M')=\det M \times \det M'$ implique que $f$ est un morphisme de groupes.
Ce morphisme est surjectif, car si $t \in \Rr^*$ alors $\det\left(\begin{smallmatrix} 1 & 0 \\ 0 & t \\ \end{smallmatrix}\right)
= t$. Ce morphisme n'est pas injectif car par exemple $\det\left(\begin{smallmatrix} 1 & 0 \\ 0 & t \\ \end{smallmatrix}\right)=
\det\left(\begin{smallmatrix} t & 0 \\ 0 & 1 \\ \end{smallmatrix}\right)$.


\end{enumerate}
\end{exemple}


Attention : ne pas confondre les différentes notations avec des puissances $-1$ :
$x^{-1}$, $f^{-1}$, $f^{-1}\big(\{ e_{G'} \}\big)$ :
\begin{itemize}
  \item $x^{-1}$ désigne l'inverse de $x$ dans un groupe $(G,\star)$. Cette notation est cohérente avec la notation usuelle
si le groupe est $(\Rr^*,\times)$ alors $x^{-1}=\frac 1 x$.
  \item Pour une application bijective $f^{-1}$ désigne la bijection réciproque.
  \item Pour une application quelconque $f : E \longrightarrow F$, l'image réciproque d'une partie $B\subset F$ est
$f^{-1}(B) = \big\{ x \in E \mid f(x) \in B \big\}$, c'est une partie de $E$.
Pour un morphisme $f$, $\Ker f = f^{-1}\big(\{ e_{G'} \}\big)$ est donc l'ensemble des $x\in G$ tels que leur image
par $f$ soit $e_{G'}$. Le noyau est défini même si $f$ n'est pas bijective.
\end{itemize}

%---------------------------------------------------------------
%\subsection{Mini-exercices}

\begin{miniexercices}
\sauteligne
\begin{enumerate}
  \item Soit $f : (\Zz,+) \longrightarrow (\Qq^*,\times)$ défini par $f(n)=2^n$. Montrer que
$f$ est un morphisme de groupes. Déterminer le noyau de $f$. $f$ est-elle injective ? surjective ?
  \item Mêmes questions pour $f : (\Rr,+) \longrightarrow (\mathcal{R},\circ)$, qui à un réel
$\theta$ associe la rotation d'angle $\theta$ de centre l'origine.
  \item Soit $(G,\star)$ un groupe et $f : G \longrightarrow G$ l'application définie par
$f(x)=x^2$. (Rappel : $x^2=x\star x$.) Montrer que si $(G,\star)$ est commutatif alors $f$
est un morphisme. Montrer ensuite la réciproque.
  \item Montrer qu'il n'existe pas de morphisme $f : (\Zz,+) \to (\Zz,+)$ tel que $f(2)=3$.
  \item Montrer que $f, g : (\Rr^*,\times) \to (\Rr^*,\times)$ définis par $f(x)=x^2$,
$g(x)=x^3$ sont des morphismes de groupes. Calculer leurs images et leurs noyaux respectives.
\end{enumerate}
\end{miniexercices}


%%%%%%%%%%%%%%%%%%%%%%%%%%%%%%%%%%%%%%%%%%%%%%%%%%%%%%%%%%%%%%%%
\section{Le groupe $\Zz/n\Zz$}
\label{sec:zsurnz}

%---------------------------------------------------------------
\subsection{L'ensemble et le groupe $\Zz/n\Zz$}

Fixons $n \ge 1$. Rappelons que $\Zz/n\Zz$
est l'ensemble
$$\Zz/n\Zz =\left\{ \overline{0}, \overline{1}, \overline{2},\ldots, \overline{n-1} \right\}$$
où $\overline p$ désigne la classe d'équivalence de $p$ modulo $n$.

Autrement dit
\mybox{$\overline p = \overline q \Longleftrightarrow p \equiv q \pmod n$}
ou encore
$\overline p = \overline q \Longleftrightarrow \exists k \in \Zz \quad p = q + kn$.


On définit une \defi{addition} sur $\Zz/n\Zz$ par :
\mybox{$\overline p + \overline q = \overline{p+q}$}

Par exemple dans $\Zz/60\Zz$, on a $\overline{31} + \overline{46}  = \overline{31 + 46} =\overline{77} = \overline{17}$.

\bigskip

Nous devons montrer que cette addition est bien définie :
si $\overline{p'}= \overline p$ et $\overline{q'} = \overline q$ alors
$p' \equiv p \pmod n$, $q' \equiv q \pmod n$ et donc
$p'+q' \equiv p+q \pmod n$. Donc $\overline{p'+q'} = \overline{p+q}$.
Donc on a aussi $\overline{p'}+\overline{q'} = \overline p + \overline q$.
Nous avons montré que l'addition est indépendante du choix des représentants.

\bigskip

%L'exemple de la vie courante est le suivant 
Voici un exemple de la vie courante : considérons seulement les minutes
d'une montre ; ces minutes varient de $0$ à $59$. Lorsque l'aiguille passe à $60$,
elle désigne aussi $0$ (on ne s'occupe pas des heures).
Ainsi de suite : $61$ s'écrit aussi $1$, $62$ s'écrit aussi $2$,\ldots
Cela correspond donc à l'ensemble $\Zz/60\Zz$.
On peut aussi additionner des minutes : $50$ minutes plus $15$ minutes font $65$ minutes qui s'écrivent
aussi $5$ minutes. Continuons avec l'écriture dans $\Zz/60\Zz$ par exemple :
$\overline{135}+\overline{50} = \overline{185} = \overline{5}$.
Remarquez que si l'on écrit d'abord $\overline{135} = \overline{15}$ alors $\overline{135}+\overline{50}
=\overline{15} +\overline{50} = \overline{65} = \overline{5}$.
On pourrait même écrire $\overline{50} = -\overline{10}$ et donc
$\overline{135}+\overline{50} = \overline{15} - \overline{10} = \overline{5}$.
C'est le fait que l'addition soit bien définie qui justifie que l'on trouve toujours le même résultat.

\begin{proposition}
$(\Zz/n\Zz,+)$ est un groupe commutatif.
\end{proposition}

C'est facile. L'élément neutre est $\overline{0}$.
L'opposé de $\overline k$ est $-\overline{k}=\overline{-k}=\overline{n-k}$.
L'associativité et la commutativité découlent de celles de $(\Zz,+)$.

%---------------------------------------------------------------
\subsection{Groupes cycliques de cardinal fini}

\begin{definition}
Un groupe $(G,\star)$ est un groupe \defi{cyclique}\index{groupe!cyclique} s'il existe un
élément $a \in G$ tel que :
$$\text{pour tout } x \in G, \text{ il existe } k \in \Zz \text{ tel que } x = a^k$$
\end{definition}

Autrement dit le groupe $G$ est engendré par un seul élément $a$.

Le groupe $(\Zz/n\Zz,+)$ est un groupe cyclique. En effet il est engendré par $a=\overline 1$,
car tout élément $\overline k$ s'écrit $\overline k =
\underbrace{\overline 1 + \overline 1 + \cdots \overline 1}_{k \text{ fois}} = k\cdot \overline 1$.

Voici un résultat intéressant : il n'existe, à isomorphisme près, qu'un seul groupe
cyclique à $n$ éléments, c'est $\Zz/n\Zz$ :
\begin{theoreme}
\label{prop:cyclique}
Si $(G,\star)$ un groupe cyclique de cardinal $n$, alors
$(G,\star)$ est isomorphe à $(\Zz/n\Zz,+)$.
\end{theoreme}

\begin{proof}
Comme $G$ est cyclique alors
$G = \big\{ \ldots, a^{-2},a^{-1},e,a,a^2,a^3,\ldots \big\}$. Dans cette écriture
il y a de nombreuses redondances (car de toute façon  $G$ n'a que $n$ éléments).
Nous allons montrer qu'en fait
$$G = \big\{e, a, a^2, \ldots, a^{n-1}\big\} \quad  \text{ et que } \quad a^n = e.$$

Tout d'abord l'ensemble $\big\{e, a, a^2, \ldots, a^{n-1}\big\}$ est inclus dans $G$. En plus il
a exactement $n$ éléments. En effet si $a^p=a^q$ avec $0\le q < p \le n-1$ alors
$a^{p-q}=e$ (avec $p-q>0$) et ainsi $a^{p-q+1}=a^{p-q}\star a = a$, $a^{p-q+2}=a^2$ et alors le groupe $G$
serait égal à  $\big\{e, a, a^2, \ldots, a^{p-q-1}\big\}$ et n'aurait pas $n$ éléments.
Ainsi $\big\{e, a, a^2, \ldots, a^{n-1}\big\} \subset G$ et les deux ensembles ont le même
nombre $n$ d'éléments, donc ils sont égaux.

Montrons maintenant que $a^n=e$. Comme $a^n \in G$ et que $G=\big\{e, a, a^2, \ldots, a^{n-1}\big\}$
alors il existe $0 \le p \le n-1$ tel que $a^n=a^p$. Encore une fois si $p>0$ cela entraîne  $a^{n-p}=e$
et donc une contradiction. Ainsi $p=0$ donc $a^n=a^0=e$.

\bigskip


Nous pouvons maintenant construire l'isomorphisme entre $(\Zz/n\Zz,+)$ et $(G,\star)$.
Soit $f : \Zz/n\Zz \longrightarrow G$ l'application définie par
$f(\overline k) = a^k$.
\begin{itemize}
 \item Il faut tout d'abord montrer que $f$ est bien définie car notre définition de $f$ dépend du représentant $k$
et pas de la classe $\overline k$ :
 si $\overline{k}=\overline{k'}$ (une même classe définie
par deux représentants distincts) alors $k \equiv k' \pmod n$ et donc il existe $\ell \in \Zz$
tel que $k = k' + \ell n$. Ainsi $f(\overline{k}) = a^k = a^{k'+\ell n} = a^{k'} \star a^{\ell n}=
a^{k'} \star (a^n)^\ell= a^{k'} \star e^\ell = a^{k'} = f(\overline{k'})$. Ainsi $f$ est bien définie.

  \item $f$ est un morphisme de groupes car $f(\overline{k}+\overline{k'})
= f(\overline{k+k'}) = a^{k+k'}=a^{k} \star a^{k'}= f(\overline k) \star f(\overline{k'})$ (pour tout $\overline{k},
\overline{k'}$).

  \item Il est clair que $f$ est surjective car tout élément de $G$ s'écrit $a^k$.

  \item Comme l'ensemble de départ et celui d'arrivée ont le même nombre d'éléments
et que $f$ est surjective alors $f$ est bijective.
\end{itemize}
Conclusion : $f$ est un isomorphisme entre $(\Zz/n\Zz,+)$ et $(G,\star)$.
\end{proof}


%---------------------------------------------------------------
%\subsection{Mini-exercices}

\begin{miniexercices}
\sauteligne
\begin{enumerate}
  \item Trouver tous les sous-groupes de $(\Zz/12\Zz,+)$.

  \item Montrer que le produit défini par $\overline p \times \overline q = \overline{p\times q}$
est bien défini sur l'ensemble $\Zz/n\Zz$.

  \item Dans la preuve du théorème \ref{prop:cyclique}, montrer directement que l'application
$f$ est injective.

  \item Montrer que l'ensemble $\mathbb{U}_n = \big\{ z \in \Cc \mid z^n = 1 \big\}$ est un sous-groupe
de $(\Cc^*,\times)$. Montrer que $\mathbb{U}_n$ est isomorphe à $\Zz/n\Zz$. Expliciter l'isomorphisme.


  \item Montrer que l'ensemble
$H = \big\{
\left(\begin{smallmatrix} 1 & 0 \\ 0 & 1 \\ \end{smallmatrix}\right),
\left(\begin{smallmatrix} 1 & 0 \\ 0 & -1 \\ \end{smallmatrix}\right),
\left(\begin{smallmatrix} -1 & 0 \\ 0 & 1 \\ \end{smallmatrix}\right),
\left(\begin{smallmatrix} -1 & 0 \\ 0 & -1 \\ \end{smallmatrix}\right)
 \big\}$
est un sous-groupe de $(\GL_2,\times)$
ayant $4$ éléments. Montrer que $H$ \emph{n'est pas} isomorphe à $\Zz/4\Zz$.
\end{enumerate}
\end{miniexercices}


%%%%%%%%%%%%%%%%%%%%%%%%%%%%%%%%%%%%%%%%%%%%%%%%%%%%%%%%%%%%%%%%
\section{Le groupe des permutations $\mathcal{S}_n$}
\label{sec:sn}

Fixons un entier $n\ge 2$.

%---------------------------------------------------------------
\subsection{Groupe des permutations}

\begin{proposition}
L'ensemble des bijections de $\{1,2,\ldots,n\}$ dans lui-même, muni de la composition
des fonctions est un groupe, noté $(\mathcal{S}_n,\circ)$.
\end{proposition}

Une bijection de $\{1,2,\ldots,n\}$ (dans lui-même) s'appelle une \defi{permutation}\index{permutation}.
Le groupe $(\mathcal{S}_n,\circ)$ s'appelle le \defi{groupe des permutations}
(ou le \defi{groupe symétrique}).

\begin{proof}
~
\begin{enumerate}
  \item La composition de deux bijections de $\{1,2,\ldots,n\}$ est une bijection de $\{1,2,\ldots,n\}$.
  \item La loi est associative (par l'associativité de la composition des fonctions).
  \item L'élément neutre est l'identité.
  \item L'inverse d'une bijection $f$ est sa bijection réciproque $f^{-1}$.
\end{enumerate}
\end{proof}



Il s'agit d'un autre exemple de groupe ayant un nombre fini d'éléments :
\begin{lemme}
Le cardinal de $\mathcal{S}_n$ est $n!$~.
\end{lemme}

\begin{proof}
La preuve est simple. Pour l'élément $1$, son image appartient à $\{1,2,\ldots,n\}$ donc nous avons $n$ choix.
Pour l'image de $2$, il ne reste plus que $n-1$ choix ($1$ et $2$ ne doivent pas avoir la même image car
notre application est une bijection).
Ainsi de suite... Pour l'image du dernier élément $n$ il ne reste qu'une possibilité.
Au final
il y a $n\times(n-1)\times \cdots \times 2 \times 1 = n!$ façon de construire des bijections
de  $\{1,2,\ldots,n\}$.
\end{proof}
%---------------------------------------------------------------
\subsection{Notation et exemples}

Décrire une permutation $f : \{1,2,\ldots,n\} \longrightarrow \{1,2,\ldots,n\}$ équivaut à donner
les images de chaque $i$ allant de $1$ à $n$.
Nous notons donc $f$ par
$$\left[\begin{matrix}
 1    & 2    & \cdots & n \\
 f(1) & f(2) & \cdots & f(n) \\
        \end{matrix}
 \right]
$$

Par exemple la permutation de $\mathcal{S}_7$ notée
  $$
    \begin{tikzpicture}[thick]
      \matrix(M)[inner sep=0pt, nodes={inner sep=3pt},matrix of math nodes, left delimiter=[,right delimiter={]}, font=\small]{
        1 & 2 & 3 & 4 & 5 & 6 & 7 \\
        3 & 7 & 5 & 4 & 6 & 1 & 2 \\
      };
      \draw ([shift={(.8em,.8em)}]M.east) edge[myred,bend left=45,{[scale=.7]|}->] node[right=-2pt,font=\small]{$f$}([shift={(.8em,-.8em)}]M.east);
    \end{tikzpicture}
  $$
est la bijection $f : \{1,2,\ldots,7\} \longrightarrow \{1,2,\ldots,7\}$
définie par $f(1)= 3$, $f(2)=7$, $f(3)=5$, $f(4)=4$, $f(5)=6$, $f(6)=1$, $f(7)=2$.
C'est bien une bijection car chaque nombre de $1$ à $7$ apparaît une fois et une seule
sur la deuxième ligne.

\bigskip

L'élément neutre du groupe est l'identité $\id$ ; pour $\mathcal{S}_7$ c'est donc
$\left[\begin{smallmatrix}
 1 & 2 & 3 & 4 & 5 & 6 & 7 \\
 1 & 2 & 3 & 4 & 5 & 6 & 7 \\
        \end{smallmatrix} \right]
$.

Il est facile de calculer la composition de deux permutations $f$ et $g$ avec cette notation.
Si
$f =\left[\begin{smallmatrix}
 1 & 2 & 3 & 4 & 5 & 6 & 7 \\
 3 & 7 & 5 & 4 & 6 & 1 & 2 \\
        \end{smallmatrix} \right]
$
et
$g = \left[\begin{smallmatrix}
 1 & 2 & 3 & 4 & 5 & 6 & 7 \\
 4 & 3 & 2 & 1 & 7 & 5 & 6 \\
        \end{smallmatrix} \right]
$
alors $g\circ f$ s'obtient en superposant la permutation $f$ puis $g$
$$g\circ f
= \begin{tikzpicture}[thick,baseline={([yshift=-.3em]M.center)}]
    \matrix(M)[inner sep=0pt, nodes={inner sep=3pt},matrix of math nodes, left delimiter=[,right delimiter={]}, font=\small]{
      1 & 2 & 3 & 4 & 5 & 6 & 7 \\
      3 & 7 & 5 & 4 & 6 & 1 & 2 \\
      2 & 6 & 7 & 1 & 5 & 4 & 3 \\
    };
    \draw[myred] (M.east) -- (M.west);
    \draw ([shift={(1em,1.6em)}]M.east) edge[myred,{[scale=.7]|}->] node[right=-2pt,font=\small]{$f$}([shift={(1em,0em)}]M.east);
    \draw ([shift={(1em,0em)}]M.east) edge[myred,{[scale=.7]|}->] node[right=-2pt,font=\small]{$g$}([shift={(1em,-1.6em)}]M.east);
    \draw ([shift={(1.7em,1.6em)}]M.east) edge[myred,bend left=45,{[scale=.7]|}->] node[right=-2pt,font=\small]{$f\circ g$}([shift={(1.7em,-1.6em)}]M.east);
  \end{tikzpicture}
 = \left[\begin{matrix}
 1 & 2 & 3 & 4 & 5 & 6 & 7 \\
 2 & 6 & 7 & 1 & 5 & 4 & 3
        \end{matrix} \right]
$$
ensuite on élimine la ligne intermédiaire du milieu et donc $g\circ f$ se note
$\left[\begin{smallmatrix}
 1 & 2 & 3 & 4 & 5 & 6 & 7 \\
 2 & 6 & 7 & 1 & 5 & 4 & 3
        \end{smallmatrix} \right]
$.

Il est tout aussi facile de calculer l'inverse d'une permutation :
il suffit d'échanger les lignes du haut et du bas et de réordonner le tableau.
Par exemple l'inverse de
$$
  f=
    \begin{tikzpicture}[thick,baseline=(M.center)]
      \matrix(M)[inner sep=0pt, nodes={inner sep=3pt},matrix of math nodes, left delimiter=[,right delimiter={]}, font=\small]{
        1 & 2 & 3 & 4 & 5 & 6 & 7 \\
        3 & 7 & 5 & 4 & 6 & 1 & 2 \\
      };
      \draw ([shift={(.8em,.8em)}]M.east) edge[myred,bend left=45,<-{[scale=.7]|}] node[right=-2pt,font=\small]{$f^{-1}$}([shift={(.8em,-.8em)}]M.east);
    \end{tikzpicture}
$$
se note
 $f^{-1} = \left[ \begin{smallmatrix}
 3 & 7 & 5 & 4 & 6 & 1 & 2 \\
 1 & 2 & 3 & 4 & 5 & 6 & 7 \\
        \end{smallmatrix} \right]
$
ou plutôt après réordonnement
 $\left[\begin{smallmatrix}
 1 & 2 & 3 & 4 & 5 & 6 & 7 \\
 6 & 7 & 1 & 4 & 3 & 5 & 2 \\
        \end{smallmatrix} \right]
$.


%---------------------------------------------------------------
\subsection{Le groupe $\mathcal{S}_3$}

Nous allons étudier en détails le groupe $\mathcal{S}_3$ des permutations de $\{1,2,3\}$.
Nous savons que $\mathcal{S}_3$ possède $3!=6$ éléments que nous énumérons :
\begin{itemize}
 \item $\id = \left[\begin{smallmatrix}
 1 & 2 & 3 \\
 1 & 2 & 3 \\
        \end{smallmatrix} \right]
$ l'identité,

 \item $\tau_1 = \left[\begin{smallmatrix}
 1 & 2 & 3 \\
 1 & 3 & 2 \\
        \end{smallmatrix} \right]
$ une transposition,

 \item $\tau_2 = \left[\begin{smallmatrix}
 1 & 2 & 3 \\
 3 & 2 & 1 \\
        \end{smallmatrix} \right]
$ une deuxième transposition,

 \item $\tau_3 = \left[\begin{smallmatrix}
 1 & 2 & 3 \\
 2 & 1 & 3 \\
        \end{smallmatrix} \right]
$ une troisième transposition,

 \item $\sigma = \left[\begin{smallmatrix}
 1 & 2 & 3 \\
 2 & 3 & 1 \\
        \end{smallmatrix} \right]
$ un cycle,


 \item $\sigma^{-1} = \left[\begin{smallmatrix}
 1 & 2 & 3 \\
 3 & 1 & 2 \\
        \end{smallmatrix} \right]
$ l'inverse du cycle précédent.
\end{itemize}

Donc $\mathcal{S}_3 = \big\{\id,\tau_1,\tau_2,\tau_3,\sigma,\sigma^{-1}\big\}$.


\bigskip

Calculons $\tau_1 \circ \sigma$ et $\sigma \circ \tau_1$ :
$$\tau_1 \circ \sigma
=\left[\begin{smallmatrix}
 1 & 2 & 3 \\
 2 & 3 & 1 \\
 3 & 2 & 1 \\
        \end{smallmatrix} \right]
= \left[\begin{smallmatrix}
 1 & 2 & 3 \\
 3 & 2 & 1 \\
        \end{smallmatrix} \right]
= \tau_2
\quad \text{et} \quad
\sigma \circ \tau_1
= \left[\begin{smallmatrix}
 1 & 2 & 3 \\
 1 & 3 & 2 \\
 2 & 1 & 3 \\
        \end{smallmatrix} \right]
= \left[\begin{smallmatrix}
 1 & 2 & 3 \\
 2 & 1 & 3 \\
        \end{smallmatrix} \right]
= \tau_3.
 $$

Ainsi $\tau_1 \circ \sigma = \tau_2$ est différent de $\sigma \circ \tau_1=\tau_3$,
ainsi le groupe $\mathcal{S}_3$ n'est pas commutatif. Et plus généralement :
\begin{lemme}
Pour $n\ge 3$, le groupe $\mathcal{S}_n$ n'est pas commutatif.
\end{lemme}



\bigskip

Nous pouvons calculer la table du groupe $\mathcal{S}_3$
\begin{figure}[H]
\centering
\begin{tabular}{c||c|c|c|c|c|c}
\textcolor{green}{$g$} $\circ$ \textcolor{blue}{$f$}  & \textcolor{blue}{$\id$} & \textcolor{blue}{$\tau_1$}
 & \textcolor{blue}{$\tau_2$} & \textcolor{blue}{$\tau_3$}
&  \textcolor{blue}{$\sigma$} & \textcolor{blue}{$\sigma^{-1}$} \\ \hline\hline

\textcolor{green}{$\id$} & $\id$ & $\tau_1$ & $\tau_2$ & $\tau_3$ & $\sigma$ & $\sigma^{-1}$ \\ \hline

\textcolor{green}{$\tau_1$} & $\tau_1$ & $\id$ & $\sigma$ & $\sigma^{-1}$ & \textcolor{red}{$\tau_1 \circ \sigma = \tau_2$} & $\tau_3$ \\ \hline

\textcolor{green}{$\tau_2$} & $\tau_2$  & $\sigma^{-1}$ & $\id$ & $\sigma$ & $\tau_3$ & $\tau_1$ \\ \hline

\textcolor{green}{$\tau_3$} & $\tau_3$ & $\sigma$ & $\sigma^{-1}$ & $\id$ & $\tau_1$ & $\tau_2$ \\ \hline

\textcolor{green}{$\sigma$} & $\sigma$ & \textcolor{orange}{$\sigma \circ \tau_1=\tau_3$} & $\tau_1$ & $\tau_2$ & $\sigma^{-1}$ & $\id$ \\ \hline

\textcolor{green}{$\sigma^{-1}$} & $\sigma^{-1}$ & $\tau_2$ & $\tau_3$ & $\tau_1$ & id & $\sigma$ \\
\end{tabular}
\caption{Table du groupe $\mathcal{S}_3$}
\end{figure}

Comment avons-nous rempli cette table ? Nous avons déjà calculé
\textcolor{red}{$\tau_1 \circ \sigma = \tau_2$} et
\textcolor{orange}{$\sigma \circ \tau_1=\tau_3$}.
Comme $f\circ \id = f$ et $\id \circ f = f$ il est facile de remplir la première colonne noire
ainsi que la première ligne noire. Ensuite il faut faire les calculs !

On retrouve ainsi que $\mathcal{S}_3 = \big\{\id,\tau_1,\tau_2,\tau_3,\sigma,\sigma^{-1}\big\}$ est un groupe :
en particulier la composition de deux permutations de la liste reste une permutation de la liste.
On lit aussi sur la table l'inverse de chaque élément, par exemple sur la ligne de $\tau_2$
on cherche à quelle colonne on trouve l'identité, c'est la colonne de $\tau_2$. Donc l'inverse de $\tau_2$
est lui-même.


%---------------------------------------------------------------
\subsection{Groupe des isométries du triangle}

Soit $(ABC)$ un triangle équilatéral.
Considérons l'ensemble des isométries du plan qui préservent le triangle, c'est-à-dire
que l'on cherche toutes les isométries $f$ telles que $f(A) \in \{A,B,C\}$,
$f(B) \in \{A,B,C\}$, $f(C) \in \{A,B,C\}$.
On trouve les isométries suivantes :
l'identité $\id$, les réflexions $t_1, t_2, t_3$ d'axes $\mathcal{D}_1, \mathcal{D}_2,\mathcal{D}_3$,
la rotation $s$ d'angle $\frac{2\pi}{3}$ et la rotation $s^{-1}$ d'angle $-\frac{2\pi}{3}$
(de centre $O$).


\myfigure{1.5}{
\tikzinput{fig_groupes01}
}

\begin{proposition}
L'ensemble des  isométries d'un triangle équilatéral, muni de la composition, forme un groupe.
Ce groupe est isomorphe à $(\mathcal{S}_3,\circ)$.
\end{proposition}

L'isomorphisme est juste l'application qui à $t_i$ associe $\tau_i$,
à $s$ associe $\sigma$ et à $s^{-1}$ associe $\sigma^{-1}$.


%---------------------------------------------------------------
\subsection{Décomposition en cycles}

\begin{itemize}
  \item Nous allons définir ce qu'est un \defi{cycle}\index{cycle} :
c'est une permutation $\sigma$ qui fixe un certain nombre d'éléments ($\sigma(i)=i$)
et dont les éléments non fixés sont obtenus par itération : $j, \sigma(j),\sigma^2(j),\ldots$
C'est plus facile à comprendre sur un exemple :
$$
\sigma = \left[\begin{matrix}
 1 & \mathbf{2} & 3 & \mathbf{4} & \mathbf{5} & 6 & 7 & \mathbf{8} \\
 1 & \mathbf{8} & 3 & \mathbf{5} & \mathbf{2} & 6 & 7 & \mathbf{4} \\
 \end{matrix} \right]
$$
est un cycle : les éléments $1, 3, 6, 7$ sont fixes,
les autres s'obtiennent comme itération de $2$ :
$2 \mapsto \sigma(2)=8 \mapsto \sigma(8)=\sigma^2(2)=4 \mapsto \sigma(4)=\sigma^3(2) =  5$,
ensuite on retrouve $\sigma^4(2)=\sigma(5)=2$.


  \item Nous noterons ce cycle par
    $$
      \begin{tikzpicture}
        \matrix(M)[matrix of math nodes, left delimiter=(,right delimiter=), inner sep=1pt, column sep=1em]
          {2 & 8 & 4 & 5\\};
        \path[thick,myred]
          foreach \i [evaluate={\j=int(\i+1)}] in {1,2,3}{
            (M-1-\i.280) edge[bend right,->] (M-1-\j.255)}
          (M-1-4.north west) edge[bend right=20,->] (M-1-1.north east);
      \end{tikzpicture}
    $$

Il faut comprendre cette notation ainsi : l'image de $2$ est $8$,
l'image de $8$ est $4$, l'image de $4$ est $5$, l'image de $5$ est $2$.
Les éléments qui n'apparaissent pas (ici $1, 3, 6, 7$) sont fixes.
On aurait pu aussi noter ce même cycle par : $(8\ 4\ 5\ 2)$, $(4\ 5\ 2\ 8)$ ou $(5\ 2\ 8\ 4)$.

  \item Pour calculer l'inverse on renverse les nombres : l'inverse de $\sigma = (2\ 8\ 4\ 5)$ est
$\sigma^{-1}=(5\ 4\ 8\ 2)$.

  \item Le \defi{support}\index{support} d'un cycle sont les éléments qui ne sont pas fixes : le support de $\sigma$ est $\{2, 4, 5, 8\}$.
La \defi{longueur} (ou l'\defi{ordre}) d'un cycle est le nombre d'éléments qui ne sont pas fixes
(c'est donc le cardinal du support).
Par exemple $(2\ 8\ 4\ 5)$ est un cycle de longueur $4$.

  \item Autres exemples :
$
\sigma = \left[\begin{smallmatrix}
 1 & 2 & 3 \\
 2 & 3 & 1 \\
        \end{smallmatrix} \right]
= (1\ 2\ 3)$ est un cycle de longueur $3$ ;
$
\tau = \left[\begin{smallmatrix}
 1 & 2 & 3 & 4 \\
 1 & 4 & 3 & 2 \\
        \end{smallmatrix} \right]
= (2\ 4)$ est un cycle de longueur $2$, aussi appelé une \defi{transposition}\index{transposition}.

  \item Par contre
$f =\left[\begin{smallmatrix}
 1 & 2 & 3 & 4 & 5 & 6 & 7 \\
 7 & 2 & 5 & 4 & 6 & 3 & 1 \\
        \end{smallmatrix} \right]
$ n'est pas un cycle ; il s'écrit comme la composition de deux cycles
$f = (1\ 7) \circ (3\ 5\ 6)$. Comme les supports de $(1\ 7)$ et $(3\ 5\ 6)$ sont disjoints
alors on a aussi $f = (3\ 5\ 6) \circ (1\ 7)$.

\end{itemize}


\medskip

Ce dernier point fait partie d'un résultat plus général que nous admettons :
\begin{theoreme}
Toute permutation de $\mathcal{S}_n$ se décompose en composition
de cycles à supports disjoints.
De plus cette décomposition est unique.
\end{theoreme}

Pour l'unicité il faut comprendre : unique à l'écriture de chaque cycle près
(exemple : $(3\ 5\ 6)$ et $(5\ 6\ 3)$ sont le même cycle)
et à l'ordre près (exemple : $(1\ 7) \circ (3\ 5\ 6)= (3\ 5\ 6) \circ (1\ 7)$).


Exemple :
la décomposition de $
f = \left[\begin{smallmatrix}
 1 & 2 & 3 & 4 & 5 & 6 & 7 & 8 \\
 5 & 2 & 1 & 8 & 3 & 7 & 6 & 4 \\
        \end{smallmatrix} \right]
$
en composition de cycle à supports
disjoints est $(1\ 5\ 3) \circ (4\ 8) \circ (6\ 7)$.

\bigskip

Attention, si les supports ne sont pas disjoints alors cela ne commute plus :
par exemple $g = (1\ 2)\circ(2\ 3\ 4)$ n'est pas égale à $h=(2\ 3\ 4)\circ(1\ 2)$.
En effet l'écriture de $g$ en produit de cycle à support disjoint est
$g = (1\ 2)\circ(2\ 3\ 4) =
\left[\begin{smallmatrix}
 1 & 2 & 3 & 4 \\
 1 & 3 & 4 & 2 \\
 2 & 3 & 4 & 1 \\
\end{smallmatrix} \right] =
\left[\begin{smallmatrix}
 1 & 2 & 3 & 4 \\
 2 & 3 & 4 & 1 \\
\end{smallmatrix} \right] =
(1\ 2\ 3\ 4)$
alors que celle de $h$ est
$h = (2\ 3\ 4)\circ(1\ 2) =
\left[\begin{smallmatrix}
 1 & 2 & 3 & 4 \\
 3 & 1 & 4 & 2 \\
\end{smallmatrix} \right]
= (1\ 3\ 4\ 2)$.





%---------------------------------------------------------------
%\subsection{Mini-exercices}

\begin{miniexercices}
\sauteligne
\begin{enumerate}
  \item Soient $f$ définie par $f(1)=2$, $f(2)=3$, $f(3)=4$, $f(4)=5$, $f(5)=1$
et $g$ définie par $g(1)=2$, $g(2)=1$, $g(3)=4$, $g(4)=3$, $g(5)=5$. \'Ecrire les permutations
$f$, $g$, $f^{-1}$, $g^{-1}$, $g\circ f$, $f\circ g$, $f^2$, $g^2$, $(g\circ f)^2$.
  \item \'Enumérer toutes les permutations de $\mathcal{S}_4$ qui n'ont pas d'éléments fixes.
Les écrire ensuite sous forme de compositions de cycles à supports disjoints.
  \item Trouver les isométries directes préservant un carré. Dresser la table des compositions
et montrer qu'elles forment un groupe. Montrer que ce groupe est isomorphe à $\Zz/4\Zz$.
  \item Montrer qu'il existe un sous-groupe de $\mathcal{S}_3$ isomorphe à $\Zz/2\Zz$.
Même question avec $\Zz/3\Zz$. Est-ce que $\mathcal{S}_3$ et $\Zz/6\Zz$ sont isomorphes ?
  \item Décomposer la permutation suivante en produit de cycles à supports disjoints :
$f=
\left[\begin{smallmatrix}
 1 & 2 & 3 & 4 & 5 & 6 & 7 \\
 5 & 7 & 2 & 6 & 1 & 4 & 3 \\
\end{smallmatrix} \right]
$.
Calculer $f^2$, $f^3$, $f^4$ puis $f^{20xx}$ où $20xx$ est l'année en cours. Mêmes questions avec
$g=
\left[\begin{smallmatrix}
 1 & 2 & 3 & 4 & 5 & 6 & 7 & 8 & 9 \\
 3 & 8 & 9 & 6 & 5 & 2 & 4 & 7 & 1\\
\end{smallmatrix} \right]$ et
$h = (2 5)(1 2 4 3)(1 2)$.
\end{enumerate}
\end{miniexercices}


\auteurs{
Arnaud Bodin,
Benjamin Boutin,
Pascal Romon
}


\finchapitre
\end{document}


