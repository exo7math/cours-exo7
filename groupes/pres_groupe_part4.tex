
%%%%%%%%%%%%%%%%%% PREAMBULE %%%%%%%%%%%%%%%%%%

\documentclass[aspectratio=169,utf8]{beamer}
%\documentclass[aspectratio=169,handout]{beamer}

\usetheme{Boadilla}
%\usecolortheme{seahorse}
\usecolortheme[RGB={245,66,24}]{structure}
\useoutertheme{infolines}

% packages
\usepackage{amsfonts,amsmath,amssymb,amsthm}
\usepackage[utf8]{inputenc}
\usepackage[T1]{fontenc}
\usepackage{lmodern}

\usepackage[francais]{babel}
\usepackage{fancybox}
\usepackage{graphicx}

\usepackage{float}
\usepackage{xfrac}

%\usepackage[usenames, x11names]{xcolor}
\usepackage{tikz}
\usepackage{pgfplots}
\usepackage{datetime}



%-----  Package unités -----
\usepackage{siunitx}
\sisetup{locale = FR,detect-all,per-mode = symbol}

%\usepackage{mathptmx}
%\usepackage{fouriernc}
%\usepackage{newcent}
%\usepackage[mathcal,mathbf]{euler}

%\usepackage{palatino}
%\usepackage{newcent}
% \usepackage[mathcal,mathbf]{euler}



% \usepackage{hyperref}
% \hypersetup{colorlinks=true, linkcolor=blue, urlcolor=blue,
% pdftitle={Exo7 - Exercices de mathématiques}, pdfauthor={Exo7}}


%section
% \usepackage{sectsty}
% \allsectionsfont{\bf}
%\sectionfont{\color{Tomato3}\upshape\selectfont}
%\subsectionfont{\color{Tomato4}\upshape\selectfont}

%----- Ensembles : entiers, reels, complexes -----
\newcommand{\Nn}{\mathbb{N}} \newcommand{\N}{\mathbb{N}}
\newcommand{\Zz}{\mathbb{Z}} \newcommand{\Z}{\mathbb{Z}}
\newcommand{\Qq}{\mathbb{Q}} \newcommand{\Q}{\mathbb{Q}}
\newcommand{\Rr}{\mathbb{R}} \newcommand{\R}{\mathbb{R}}
\newcommand{\Cc}{\mathbb{C}} 
\newcommand{\Kk}{\mathbb{K}} \newcommand{\K}{\mathbb{K}}

%----- Modifications de symboles -----
\renewcommand{\epsilon}{\varepsilon}
\renewcommand{\Re}{\mathop{\text{Re}}\nolimits}
\renewcommand{\Im}{\mathop{\text{Im}}\nolimits}
%\newcommand{\llbracket}{\left[\kern-0.15em\left[}
%\newcommand{\rrbracket}{\right]\kern-0.15em\right]}

\renewcommand{\ge}{\geqslant}
\renewcommand{\geq}{\geqslant}
\renewcommand{\le}{\leqslant}
\renewcommand{\leq}{\leqslant}
\renewcommand{\epsilon}{\varepsilon}

%----- Fonctions usuelles -----
\newcommand{\ch}{\mathop{\text{ch}}\nolimits}
\newcommand{\sh}{\mathop{\text{sh}}\nolimits}
\renewcommand{\tanh}{\mathop{\text{th}}\nolimits}
\newcommand{\cotan}{\mathop{\text{cotan}}\nolimits}
\newcommand{\Arcsin}{\mathop{\text{arcsin}}\nolimits}
\newcommand{\Arccos}{\mathop{\text{arccos}}\nolimits}
\newcommand{\Arctan}{\mathop{\text{arctan}}\nolimits}
\newcommand{\Argsh}{\mathop{\text{argsh}}\nolimits}
\newcommand{\Argch}{\mathop{\text{argch}}\nolimits}
\newcommand{\Argth}{\mathop{\text{argth}}\nolimits}
\newcommand{\pgcd}{\mathop{\text{pgcd}}\nolimits} 


%----- Commandes divers ------
\newcommand{\ii}{\mathrm{i}}
\newcommand{\dd}{\text{d}}
\newcommand{\id}{\mathop{\text{id}}\nolimits}
\newcommand{\Ker}{\mathop{\text{Ker}}\nolimits}
\newcommand{\Card}{\mathop{\text{Card}}\nolimits}
\newcommand{\Vect}{\mathop{\text{Vect}}\nolimits}
\newcommand{\Mat}{\mathop{\text{Mat}}\nolimits}
\newcommand{\rg}{\mathop{\text{rg}}\nolimits}
\newcommand{\tr}{\mathop{\text{tr}}\nolimits}


%----- Structure des exercices ------

\newtheoremstyle{styleexo}% name
{2ex}% Space above
{3ex}% Space below
{}% Body font
{}% Indent amount 1
{\bfseries} % Theorem head font
{}% Punctuation after theorem head
{\newline}% Space after theorem head 2
{}% Theorem head spec (can be left empty, meaning ‘normal’)

%\theoremstyle{styleexo}
\newtheorem{exo}{Exercice}
\newtheorem{ind}{Indications}
\newtheorem{cor}{Correction}


\newcommand{\exercice}[1]{} \newcommand{\finexercice}{}
%\newcommand{\exercice}[1]{{\tiny\texttt{#1}}\vspace{-2ex}} % pour afficher le numero absolu, l'auteur...
\newcommand{\enonce}{\begin{exo}} \newcommand{\finenonce}{\end{exo}}
\newcommand{\indication}{\begin{ind}} \newcommand{\finindication}{\end{ind}}
\newcommand{\correction}{\begin{cor}} \newcommand{\fincorrection}{\end{cor}}

\newcommand{\noindication}{\stepcounter{ind}}
\newcommand{\nocorrection}{\stepcounter{cor}}

\newcommand{\fiche}[1]{} \newcommand{\finfiche}{}
\newcommand{\titre}[1]{\centerline{\large \bf #1}}
\newcommand{\addcommand}[1]{}
\newcommand{\video}[1]{}

% Marge
\newcommand{\mymargin}[1]{\marginpar{{\small #1}}}

\def\noqed{\renewcommand{\qedsymbol}{}}


%----- Presentation ------
\setlength{\parindent}{0cm}

%\newcommand{\ExoSept}{\href{http://exo7.emath.fr}{\textbf{\textsf{Exo7}}}}

\definecolor{myred}{rgb}{0.93,0.26,0}
\definecolor{myorange}{rgb}{0.97,0.58,0}
\definecolor{myyellow}{rgb}{1,0.86,0}

\newcommand{\LogoExoSept}[1]{  % input : echelle
{\usefont{U}{cmss}{bx}{n}
\begin{tikzpicture}[scale=0.1*#1,transform shape]
  \fill[color=myorange] (0,0)--(4,0)--(4,-4)--(0,-4)--cycle;
  \fill[color=myred] (0,0)--(0,3)--(-3,3)--(-3,0)--cycle;
  \fill[color=myyellow] (4,0)--(7,4)--(3,7)--(0,3)--cycle;
  \node[scale=5] at (3.5,3.5) {Exo7};
\end{tikzpicture}}
}


\newcommand{\debutmontitre}{
  \author{} \date{} 
  \thispagestyle{empty}
  \hspace*{-10ex}
  \begin{minipage}{\textwidth}
    \titlepage  
  \vspace*{-2.5cm}
  \begin{center}
    \LogoExoSept{2.5}
  \end{center}
  \end{minipage}

  \vspace*{-0cm}
  
  % Astuce pour que le background ne soit pas discrétisé lors de la conversion pdf -> png
\begin{tikzpicture}
        \fill[opacity=0,green!60!black] (0,0)--++(0,0)--++(0,0)--++(0,0)--cycle; 
\end{tikzpicture}

% toc S'affiche trop tot :
% \tableofcontents[hideallsubsections, pausesections]
}

\newcommand{\finmontitre}{
  \end{frame}
  \setcounter{framenumber}{0}
} % ne marche pas pour une raison obscure

%----- Commandes supplementaires ------

% \usepackage[landscape]{geometry}
% \geometry{top=1cm, bottom=3cm, left=2cm, right=10cm, marginparsep=1cm
% }
% \usepackage[a4paper]{geometry}
% \geometry{top=2cm, bottom=2cm, left=2cm, right=2cm, marginparsep=1cm
% }

%\usepackage{standalone}


% New command Arnaud -- november 2011
\setbeamersize{text margin left=24ex}
% si vous modifier cette valeur il faut aussi
% modifier le decalage du titre pour compenser
% (ex : ici =+10ex, titre =-5ex

\theoremstyle{definition}
%\newtheorem{proposition}{Proposition}
%\newtheorem{exemple}{Exemple}
%\newtheorem{theoreme}{Théorème}
%\newtheorem{lemme}{Lemme}
%\newtheorem{corollaire}{Corollaire}
%\newtheorem*{remarque*}{Remarque}
%\newtheorem*{miniexercice}{Mini-exercices}
%\newtheorem{definition}{Définition}

% Commande tikz
\usetikzlibrary{calc}
\usetikzlibrary{patterns,arrows}
\usetikzlibrary{matrix}
\usetikzlibrary{fadings} 

%definition d'un terme
\newcommand{\defi}[1]{{\color{myorange}\textbf{\emph{#1}}}}
\newcommand{\evidence}[1]{{\color{blue}\textbf{\emph{#1}}}}
\newcommand{\assertion}[1]{\emph{\og#1\fg}}  % pour chapitre logique
%\renewcommand{\contentsname}{Sommaire}
\renewcommand{\contentsname}{}
\setcounter{tocdepth}{2}



%------ Figures ------

\def\myscale{1} % par défaut 
\newcommand{\myfigure}[2]{  % entrée : echelle, fichier figure
\def\myscale{#1}
\begin{center}
\footnotesize
{#2}
\end{center}}


%------ Encadrement ------

\usepackage{fancybox}


\newcommand{\mybox}[1]{
\setlength{\fboxsep}{7pt}
\begin{center}
\shadowbox{#1}
\end{center}}

\newcommand{\myboxinline}[1]{
\setlength{\fboxsep}{5pt}
\raisebox{-10pt}{
\shadowbox{#1}
}
}

%--------------- Commande beamer---------------
\newcommand{\beameronly}[1]{#1} % permet de mettre des pause dans beamer pas dans poly


\setbeamertemplate{navigation symbols}{}
\setbeamertemplate{footline}  % tiré du fichier beamerouterinfolines.sty
{
  \leavevmode%
  \hbox{%
  \begin{beamercolorbox}[wd=.333333\paperwidth,ht=2.25ex,dp=1ex,center]{author in head/foot}%
    % \usebeamerfont{author in head/foot}\insertshortauthor%~~(\insertshortinstitute)
    \usebeamerfont{section in head/foot}{\bf\insertshorttitle}
  \end{beamercolorbox}%
  \begin{beamercolorbox}[wd=.333333\paperwidth,ht=2.25ex,dp=1ex,center]{title in head/foot}%
    \usebeamerfont{section in head/foot}{\bf\insertsectionhead}
  \end{beamercolorbox}%
  \begin{beamercolorbox}[wd=.333333\paperwidth,ht=2.25ex,dp=1ex,right]{date in head/foot}%
    % \usebeamerfont{date in head/foot}\insertshortdate{}\hspace*{2em}
    \insertframenumber{} / \inserttotalframenumber\hspace*{2ex} 
  \end{beamercolorbox}}%
  \vskip0pt%
}


\definecolor{mygrey}{rgb}{0.5,0.5,0.5}
\setlength{\parindent}{0cm}
%\DeclareTextFontCommand{\helvetica}{\fontfamily{phv}\selectfont}

% background beamer
\definecolor{couleurhaut}{rgb}{0.85,0.9,1}  % creme
\definecolor{couleurmilieu}{rgb}{1,1,1}  % vert pale
\definecolor{couleurbas}{rgb}{0.85,0.9,1}  % blanc
\setbeamertemplate{background canvas}[vertical shading]%
[top=couleurhaut,middle=couleurmilieu,midpoint=0.4,bottom=couleurbas] 
%[top=fondtitre!05,bottom=fondtitre!60]



\makeatletter
\setbeamertemplate{theorem begin}
{%
  \begin{\inserttheoremblockenv}
  {%
    \inserttheoremheadfont
    \inserttheoremname
    \inserttheoremnumber
    \ifx\inserttheoremaddition\@empty\else\ (\inserttheoremaddition)\fi%
    \inserttheorempunctuation
  }%
}
\setbeamertemplate{theorem end}{\end{\inserttheoremblockenv}}

\newenvironment{theoreme}[1][]{%
   \setbeamercolor{block title}{fg=structure,bg=structure!40}
   \setbeamercolor{block body}{fg=black,bg=structure!10}
   \begin{block}{{\bf Th\'eor\`eme }#1}
}{%
   \end{block}%
}


\newenvironment{proposition}[1][]{%
   \setbeamercolor{block title}{fg=structure,bg=structure!40}
   \setbeamercolor{block body}{fg=black,bg=structure!10}
   \begin{block}{{\bf Proposition }#1}
}{%
   \end{block}%
}

\newenvironment{corollaire}[1][]{%
   \setbeamercolor{block title}{fg=structure,bg=structure!40}
   \setbeamercolor{block body}{fg=black,bg=structure!10}
   \begin{block}{{\bf Corollaire }#1}
}{%
   \end{block}%
}

\newenvironment{mydefinition}[1][]{%
   \setbeamercolor{block title}{fg=structure,bg=structure!40}
   \setbeamercolor{block body}{fg=black,bg=structure!10}
   \begin{block}{{\bf Définition} #1}
}{%
   \end{block}%
}

\newenvironment{lemme}[0]{%
   \setbeamercolor{block title}{fg=structure,bg=structure!40}
   \setbeamercolor{block body}{fg=black,bg=structure!10}
   \begin{block}{\bf Lemme}
}{%
   \end{block}%
}

\newenvironment{remarque}[1][]{%
   \setbeamercolor{block title}{fg=black,bg=structure!20}
   \setbeamercolor{block body}{fg=black,bg=structure!5}
   \begin{block}{Remarque #1}
}{%
   \end{block}%
}


\newenvironment{exemple}[1][]{%
   \setbeamercolor{block title}{fg=black,bg=structure!20}
   \setbeamercolor{block body}{fg=black,bg=structure!5}
   \begin{block}{{\bf Exemple }#1}
}{%
   \end{block}%
}


\newenvironment{miniexercice}[0]{%
   \setbeamercolor{block title}{fg=structure,bg=structure!20}
   \setbeamercolor{block body}{fg=black,bg=structure!5}
   \begin{block}{Mini-exercices}
}{%
   \end{block}%
}


\newenvironment{tp}[0]{%
   \setbeamercolor{block title}{fg=structure,bg=structure!40}
   \setbeamercolor{block body}{fg=black,bg=structure!10}
   \begin{block}{\bf Travaux pratiques}
}{%
   \end{block}%
}
\newenvironment{exercicecours}[1][]{%
   \setbeamercolor{block title}{fg=structure,bg=structure!40}
   \setbeamercolor{block body}{fg=black,bg=structure!10}
   \begin{block}{{\bf Exercice }#1}
}{%
   \end{block}%
}
\newenvironment{algo}[1][]{%
   \setbeamercolor{block title}{fg=structure,bg=structure!40}
   \setbeamercolor{block body}{fg=black,bg=structure!10}
   \begin{block}{{\bf Algorithme}\hfill{\color{gray}\texttt{#1}}}
}{%
   \end{block}%
}


\setbeamertemplate{proof begin}{
   \setbeamercolor{block title}{fg=black,bg=structure!20}
   \setbeamercolor{block body}{fg=black,bg=structure!5}
   \begin{block}{{\footnotesize Démonstration}}
   \footnotesize
   \smallskip}
\setbeamertemplate{proof end}{%
   \end{block}}
\setbeamertemplate{qed symbol}{\openbox}


\makeatother
\usecolortheme[RGB={0,153,0}]{structure}

% Commande spécifique à ce chapitre
\newcommand{\GL}{\mathcal{G}\ell}
\newcounter{saveenumi}


%%%%%%%%%%%%%%%%%%%%%%%%%%%%%%%%%%%%%%%%%%%%%%%%%%%%%%%%%%%%%
%%%%%%%%%%%%%%%%%%%%%%%%%%%%%%%%%%%%%%%%%%%%%%%%%%%%%%%%%%%%%



\begin{document}



\title{{\bf Groupes}}
\subtitle{Le groupe $\Zz/n\Zz$}

\begin{frame}
  
  \debutmontitre

  \pause

{\footnotesize
\hfill
\setbeamercovered{transparent=50}
\begin{minipage}{0.6\textwidth}
  \begin{itemize}
    \item<3-> L'ensemble et le groupe $\Zz/n\Zz$
    \item<4-> Groupes cycliques de cardinal fini
  \end{itemize}
\end{minipage}
}

\end{frame}

\setcounter{framenumber}{0}


%%%%%%%%%%%%%%%%%%%%%%%%%%%%%%%%%%%%%%%%%%%%%%%%%%%%%%%%%%%%%%%%



%---------------------------------------------------------------
\section{L'ensemble et le groupe $\Zz/n\Zz$}

\begin{frame}
Fixons $n \ge 1$

$$\Zz/n\Zz =\left\{ \overline{0}, \overline{1}, \overline{2},\ldots, \overline{n-1} \right\}$$

\pause


$\overline p$ désigne la classe d'équivalence de $p$ modulo $n$

\mybox{$\overline p = \overline q \Longleftrightarrow p \equiv q \pmod n$}

\pause

\centerline{$\overline p = \overline q \Longleftrightarrow \exists k \in \Zz \quad p = q + kn$}


\pause
\bigskip

On définit une \defi{addition} sur $\Zz/n\Zz$
\mybox{$\overline p + \overline q = \overline{p+q}$}



\end{frame}

\begin{frame}

\begin{exemple}
Fixons $n=60$
\begin{itemize}
  \item $\overline{60}=\overline{0}$, $\overline{61}=\overline{1}$,...
\pause
  \item $\overline{135} = \overline{15}$
\pause
  \item $\Zz/60\Zz= \big\{\overline{0},\overline{1},
\overline{2},\ldots,\overline{59}\big\}$
\pause
  \item $\overline{31} + \overline{46}  = \overline{31 + 46} =\overline{77} = \overline{17}$
\pause
  \item $\overline{15} + \overline{50}  = \overline{15 + 50} =\overline{65} = \overline{5}$
\pause
  \item $\overline{135}+\overline{50} = \overline{185} = \overline{5}$
\pause
  \item $\overline{135}+\overline{50} = \overline{15} - \overline{10} = \overline{5}$
\end{itemize}
\end{exemple}

\pause
\medskip

L'addition est bien définie : elle est indépendante du choix des représentants

\pause

$\begin{array}{rcl}
 \overline{p'}= \overline p \ \text{ et } \ \overline{q'} = \overline q 
    & \Rightarrow & p' \equiv p \pmod n \text{ et } q' \equiv q \pmod n \\
    & \Rightarrow & p'+q' \equiv p+q \pmod n \\
    & \Rightarrow & \overline{p'+q'} = \overline{p+q} \\
    & \Rightarrow & \overline{p'}+\overline{q'} = \overline p + \overline q \\
\end{array}
$

\end{frame}



\begin{frame}

\begin{proposition}
$(\Zz/n\Zz,+)$ est un groupe commutatif 
\end{proposition}

\pause

\begin{itemize}
  \item L'élément neutre est $\overline{0}$
\pause
  \item L'opposé de $\overline p$, noté $-\overline{p}$, est $\overline{-p}=\overline{n-p}$
\pause
  \item L'associativité et la commutativité découlent de celles de $(\Zz,+)$
\end{itemize}

\end{frame}






%---------------------------------------------------------------
\section{Groupes cycliques de cardinal fini}

\begin{frame}

\begin{mydefinition}
$(G,\star)$ est un groupe \defi{cyclique} s'il existe un
élément $a \in G$ tel que 

$\text{pour tout } x \in G, \text{ il existe } k \in \Zz \text{ tel que } x = a^k$
\end{mydefinition}


\pause
\bigskip

Autrement dit : $G$ est engendré par un seul élément $a$

\pause
\bigskip

Le groupe $(\Zz/n\Zz,+)$ est un groupe cyclique, il est engendré par $a=\overline 1$

\centerline{$\overline k =  
\underbrace{\overline 1 + \overline 1 + \cdots + \overline 1}_{k \text{ fois}} = k\cdot \overline 1$}


\end{frame}


\begin{frame}

\begin{theoreme}
\label{prop:cyclique}
Si $(G,\star)$ est un groupe cyclique de cardinal $n$, alors  
$(G,\star)$ est isomorphe à $(\Zz/n\Zz,+)$
\end{theoreme}

\pause
\bigskip

Il n'existe, à isomorphisme près, qu'un seul groupe 
cyclique à $n$ éléments, c'est $\Zz/n\Zz$
\end{frame}

\begin{frame}
\begin{proof}
$G$ est cyclique donc $G = \big\{ \ldots, a^{-2},a^{-1},e,a,a^2,a^3,\ldots \big\}$

\pause

Montrons $G = \big\{e, a, a^2, \ldots, a^{n-1}\big\} \quad  \text{ et que } \quad a^n = e$

\pause

\begin{itemize}
  \item $\big\{e, a, a^2, \ldots, a^{n-1}\big\}$ est inclus dans $G$

\pause

  \item si $a^p=a^q$ (avec $0\le q < p \le n-1$) alors $a^{p-q}=e$ 

$G$ serait égal à  $\big\{e, a, a^2, \ldots, a^{p-q-1}\big\}$ et donc n'aurait pas $n$ éléments

\pause

  \item donc $G = \big\{e, a, a^2, \ldots, a^{n-1}\big\}$

\pause

  \item $a^n \in G$ donc il existe $0 \le p \le n-1$ tel que $a^n=a^p$

Encore une fois si $p>0$ alors  $a^{n-p}=e$ et donc contradiction

\pause

  \item  Ainsi $p=0$ donc $a^n=a^0=e$
\end{itemize}
\end{proof}
\end{frame}


\begin{frame}

\begin{proof}
$$G = \big\{e, a, a^2, \ldots, a^{n-1}\big\} \quad  \text{ et } \quad a^n = e$$


Construction de l'isomorphisme entre $(\Zz/n\Zz,+)$ et $(G,\star)$

\pause
\medskip

\hfil 
$
\begin{array}{rccl}
f : & \Zz/n\Zz & \longrightarrow & G \\
    &  \overline k & \longmapsto & a^k  \\
\end{array}
$
\pause

\begin{itemize}
 \item  $f$ est bien définie :

$\begin{array}{rcl}
 \overline{k}=\overline{k'} 
    & \Rightarrow & k \equiv k' \pmod n \\
    & \Rightarrow & k = k' + \ell n \\
    & \Rightarrow & f(\overline{k}) = a^k = a^{k'+\ell n}  \\
    & \Rightarrow & f(\overline{k}) = a^{k'} \star a^{\ell n}=
a^{k'} \star (a^n)^\ell= a^{k'} \star e^\ell = a^{k'} = f(\overline{k'}) \\
\end{array}$

\pause
  \item $f$ est un morphisme de groupe :

 $f(\overline{k}+\overline{k'})
= f(\overline{k+k'}) = a^{k+k'}=a^{k} \star a^{k'}= f(\overline k) \star f(\overline{k'})$ 

\pause
  \item $f$ est surjective car tout élément de $G$ s'écrit $a^k$

\pause
  \item $G$ et $\Zz/n\Zz$ ont le même nombre d'éléments donc $f$ est bijective
\end{itemize}

\medskip
\pause

Conclusion : $f$ est un isomorphisme entre $(\Zz/n\Zz,+)$ et $(G,\star)$
\end{proof}

\end{frame}



%---------------------------------------------------------------
\section{Mini-exercices}

\begin{frame}

\begin{miniexercice}
\begin{enumerate}
  \item Trouver tous les sous-groupes de $(\Zz/12\Zz,+)$.

  \item Montrer que le produit défini par $\overline p \times \overline q = \overline{p\times q}$
est bien défini sur l'ensemble $\Zz/n\Zz$.

  \item Dans la preuve du théorème (page 5), montrer directement que l'application
$f$ est injective.

  \item Montrer que l'ensemble $\mathbb{U}_n = \big\{ z \in \Cc \mid z^n = 1 \big\}$ est un sous-groupe 
  de $(\Cc^*,\times)$. Montrer que $\mathbb{U}_n$ est isomorphe à $\Zz/n\Zz$. Expliciter l'isomorphisme.

  \item Montrer que l'ensemble 
$H = \big\{
\left(\begin{smallmatrix} 1 & 0 \\ 0 & 1 \\ \end{smallmatrix}\right),
\left(\begin{smallmatrix} 1 & 0 \\ 0 & -1 \\ \end{smallmatrix}\right),
\left(\begin{smallmatrix} -1 & 0 \\ 0 & 1 \\ \end{smallmatrix}\right),
\left(\begin{smallmatrix} -1 & 0 \\ 0 & -1 \\ \end{smallmatrix}\right)
 \big\}$ 
est un sous-groupe de $(\GL_2,\times)$
ayant $4$ éléments. Montrer que $H$ \emph{n'est pas} isomorphe à $\Zz/4\Zz$.
\end{enumerate}

\end{miniexercice}
\end{frame}




\end{document}