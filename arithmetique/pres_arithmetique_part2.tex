
%%%%%%%%%%%%%%%%%% PREAMBULE %%%%%%%%%%%%%%%%%%

\documentclass[aspectratio=169,utf8]{beamer}
%\documentclass[aspectratio=169,handout]{beamer}

\usetheme{Boadilla}
%\usecolortheme{seahorse}
\usecolortheme[RGB={245,66,24}]{structure}
\useoutertheme{infolines}

% packages
\usepackage{amsfonts,amsmath,amssymb,amsthm}
\usepackage[utf8]{inputenc}
\usepackage[T1]{fontenc}
\usepackage{lmodern}

\usepackage[francais]{babel}
\usepackage{fancybox}
\usepackage{graphicx}

\usepackage{float}
\usepackage{xfrac}

%\usepackage[usenames, x11names]{xcolor}
\usepackage{tikz}
\usepackage{pgfplots}
\usepackage{datetime}



%-----  Package unités -----
\usepackage{siunitx}
\sisetup{locale = FR,detect-all,per-mode = symbol}

%\usepackage{mathptmx}
%\usepackage{fouriernc}
%\usepackage{newcent}
%\usepackage[mathcal,mathbf]{euler}

%\usepackage{palatino}
%\usepackage{newcent}
% \usepackage[mathcal,mathbf]{euler}



% \usepackage{hyperref}
% \hypersetup{colorlinks=true, linkcolor=blue, urlcolor=blue,
% pdftitle={Exo7 - Exercices de mathématiques}, pdfauthor={Exo7}}


%section
% \usepackage{sectsty}
% \allsectionsfont{\bf}
%\sectionfont{\color{Tomato3}\upshape\selectfont}
%\subsectionfont{\color{Tomato4}\upshape\selectfont}

%----- Ensembles : entiers, reels, complexes -----
\newcommand{\Nn}{\mathbb{N}} \newcommand{\N}{\mathbb{N}}
\newcommand{\Zz}{\mathbb{Z}} \newcommand{\Z}{\mathbb{Z}}
\newcommand{\Qq}{\mathbb{Q}} \newcommand{\Q}{\mathbb{Q}}
\newcommand{\Rr}{\mathbb{R}} \newcommand{\R}{\mathbb{R}}
\newcommand{\Cc}{\mathbb{C}} 
\newcommand{\Kk}{\mathbb{K}} \newcommand{\K}{\mathbb{K}}

%----- Modifications de symboles -----
\renewcommand{\epsilon}{\varepsilon}
\renewcommand{\Re}{\mathop{\text{Re}}\nolimits}
\renewcommand{\Im}{\mathop{\text{Im}}\nolimits}
%\newcommand{\llbracket}{\left[\kern-0.15em\left[}
%\newcommand{\rrbracket}{\right]\kern-0.15em\right]}

\renewcommand{\ge}{\geqslant}
\renewcommand{\geq}{\geqslant}
\renewcommand{\le}{\leqslant}
\renewcommand{\leq}{\leqslant}
\renewcommand{\epsilon}{\varepsilon}

%----- Fonctions usuelles -----
\newcommand{\ch}{\mathop{\text{ch}}\nolimits}
\newcommand{\sh}{\mathop{\text{sh}}\nolimits}
\renewcommand{\tanh}{\mathop{\text{th}}\nolimits}
\newcommand{\cotan}{\mathop{\text{cotan}}\nolimits}
\newcommand{\Arcsin}{\mathop{\text{arcsin}}\nolimits}
\newcommand{\Arccos}{\mathop{\text{arccos}}\nolimits}
\newcommand{\Arctan}{\mathop{\text{arctan}}\nolimits}
\newcommand{\Argsh}{\mathop{\text{argsh}}\nolimits}
\newcommand{\Argch}{\mathop{\text{argch}}\nolimits}
\newcommand{\Argth}{\mathop{\text{argth}}\nolimits}
\newcommand{\pgcd}{\mathop{\text{pgcd}}\nolimits} 


%----- Commandes divers ------
\newcommand{\ii}{\mathrm{i}}
\newcommand{\dd}{\text{d}}
\newcommand{\id}{\mathop{\text{id}}\nolimits}
\newcommand{\Ker}{\mathop{\text{Ker}}\nolimits}
\newcommand{\Card}{\mathop{\text{Card}}\nolimits}
\newcommand{\Vect}{\mathop{\text{Vect}}\nolimits}
\newcommand{\Mat}{\mathop{\text{Mat}}\nolimits}
\newcommand{\rg}{\mathop{\text{rg}}\nolimits}
\newcommand{\tr}{\mathop{\text{tr}}\nolimits}


%----- Structure des exercices ------

\newtheoremstyle{styleexo}% name
{2ex}% Space above
{3ex}% Space below
{}% Body font
{}% Indent amount 1
{\bfseries} % Theorem head font
{}% Punctuation after theorem head
{\newline}% Space after theorem head 2
{}% Theorem head spec (can be left empty, meaning ‘normal’)

%\theoremstyle{styleexo}
\newtheorem{exo}{Exercice}
\newtheorem{ind}{Indications}
\newtheorem{cor}{Correction}


\newcommand{\exercice}[1]{} \newcommand{\finexercice}{}
%\newcommand{\exercice}[1]{{\tiny\texttt{#1}}\vspace{-2ex}} % pour afficher le numero absolu, l'auteur...
\newcommand{\enonce}{\begin{exo}} \newcommand{\finenonce}{\end{exo}}
\newcommand{\indication}{\begin{ind}} \newcommand{\finindication}{\end{ind}}
\newcommand{\correction}{\begin{cor}} \newcommand{\fincorrection}{\end{cor}}

\newcommand{\noindication}{\stepcounter{ind}}
\newcommand{\nocorrection}{\stepcounter{cor}}

\newcommand{\fiche}[1]{} \newcommand{\finfiche}{}
\newcommand{\titre}[1]{\centerline{\large \bf #1}}
\newcommand{\addcommand}[1]{}
\newcommand{\video}[1]{}

% Marge
\newcommand{\mymargin}[1]{\marginpar{{\small #1}}}

\def\noqed{\renewcommand{\qedsymbol}{}}


%----- Presentation ------
\setlength{\parindent}{0cm}

%\newcommand{\ExoSept}{\href{http://exo7.emath.fr}{\textbf{\textsf{Exo7}}}}

\definecolor{myred}{rgb}{0.93,0.26,0}
\definecolor{myorange}{rgb}{0.97,0.58,0}
\definecolor{myyellow}{rgb}{1,0.86,0}

\newcommand{\LogoExoSept}[1]{  % input : echelle
{\usefont{U}{cmss}{bx}{n}
\begin{tikzpicture}[scale=0.1*#1,transform shape]
  \fill[color=myorange] (0,0)--(4,0)--(4,-4)--(0,-4)--cycle;
  \fill[color=myred] (0,0)--(0,3)--(-3,3)--(-3,0)--cycle;
  \fill[color=myyellow] (4,0)--(7,4)--(3,7)--(0,3)--cycle;
  \node[scale=5] at (3.5,3.5) {Exo7};
\end{tikzpicture}}
}


\newcommand{\debutmontitre}{
  \author{} \date{} 
  \thispagestyle{empty}
  \hspace*{-10ex}
  \begin{minipage}{\textwidth}
    \titlepage  
  \vspace*{-2.5cm}
  \begin{center}
    \LogoExoSept{2.5}
  \end{center}
  \end{minipage}

  \vspace*{-0cm}
  
  % Astuce pour que le background ne soit pas discrétisé lors de la conversion pdf -> png
\begin{tikzpicture}
        \fill[opacity=0,green!60!black] (0,0)--++(0,0)--++(0,0)--++(0,0)--cycle; 
\end{tikzpicture}

% toc S'affiche trop tot :
% \tableofcontents[hideallsubsections, pausesections]
}

\newcommand{\finmontitre}{
  \end{frame}
  \setcounter{framenumber}{0}
} % ne marche pas pour une raison obscure

%----- Commandes supplementaires ------

% \usepackage[landscape]{geometry}
% \geometry{top=1cm, bottom=3cm, left=2cm, right=10cm, marginparsep=1cm
% }
% \usepackage[a4paper]{geometry}
% \geometry{top=2cm, bottom=2cm, left=2cm, right=2cm, marginparsep=1cm
% }

%\usepackage{standalone}


% New command Arnaud -- november 2011
\setbeamersize{text margin left=24ex}
% si vous modifier cette valeur il faut aussi
% modifier le decalage du titre pour compenser
% (ex : ici =+10ex, titre =-5ex

\theoremstyle{definition}
%\newtheorem{proposition}{Proposition}
%\newtheorem{exemple}{Exemple}
%\newtheorem{theoreme}{Théorème}
%\newtheorem{lemme}{Lemme}
%\newtheorem{corollaire}{Corollaire}
%\newtheorem*{remarque*}{Remarque}
%\newtheorem*{miniexercice}{Mini-exercices}
%\newtheorem{definition}{Définition}

% Commande tikz
\usetikzlibrary{calc}
\usetikzlibrary{patterns,arrows}
\usetikzlibrary{matrix}
\usetikzlibrary{fadings} 

%definition d'un terme
\newcommand{\defi}[1]{{\color{myorange}\textbf{\emph{#1}}}}
\newcommand{\evidence}[1]{{\color{blue}\textbf{\emph{#1}}}}
\newcommand{\assertion}[1]{\emph{\og#1\fg}}  % pour chapitre logique
%\renewcommand{\contentsname}{Sommaire}
\renewcommand{\contentsname}{}
\setcounter{tocdepth}{2}



%------ Figures ------

\def\myscale{1} % par défaut 
\newcommand{\myfigure}[2]{  % entrée : echelle, fichier figure
\def\myscale{#1}
\begin{center}
\footnotesize
{#2}
\end{center}}


%------ Encadrement ------

\usepackage{fancybox}


\newcommand{\mybox}[1]{
\setlength{\fboxsep}{7pt}
\begin{center}
\shadowbox{#1}
\end{center}}

\newcommand{\myboxinline}[1]{
\setlength{\fboxsep}{5pt}
\raisebox{-10pt}{
\shadowbox{#1}
}
}

%--------------- Commande beamer---------------
\newcommand{\beameronly}[1]{#1} % permet de mettre des pause dans beamer pas dans poly


\setbeamertemplate{navigation symbols}{}
\setbeamertemplate{footline}  % tiré du fichier beamerouterinfolines.sty
{
  \leavevmode%
  \hbox{%
  \begin{beamercolorbox}[wd=.333333\paperwidth,ht=2.25ex,dp=1ex,center]{author in head/foot}%
    % \usebeamerfont{author in head/foot}\insertshortauthor%~~(\insertshortinstitute)
    \usebeamerfont{section in head/foot}{\bf\insertshorttitle}
  \end{beamercolorbox}%
  \begin{beamercolorbox}[wd=.333333\paperwidth,ht=2.25ex,dp=1ex,center]{title in head/foot}%
    \usebeamerfont{section in head/foot}{\bf\insertsectionhead}
  \end{beamercolorbox}%
  \begin{beamercolorbox}[wd=.333333\paperwidth,ht=2.25ex,dp=1ex,right]{date in head/foot}%
    % \usebeamerfont{date in head/foot}\insertshortdate{}\hspace*{2em}
    \insertframenumber{} / \inserttotalframenumber\hspace*{2ex} 
  \end{beamercolorbox}}%
  \vskip0pt%
}


\definecolor{mygrey}{rgb}{0.5,0.5,0.5}
\setlength{\parindent}{0cm}
%\DeclareTextFontCommand{\helvetica}{\fontfamily{phv}\selectfont}

% background beamer
\definecolor{couleurhaut}{rgb}{0.85,0.9,1}  % creme
\definecolor{couleurmilieu}{rgb}{1,1,1}  % vert pale
\definecolor{couleurbas}{rgb}{0.85,0.9,1}  % blanc
\setbeamertemplate{background canvas}[vertical shading]%
[top=couleurhaut,middle=couleurmilieu,midpoint=0.4,bottom=couleurbas] 
%[top=fondtitre!05,bottom=fondtitre!60]



\makeatletter
\setbeamertemplate{theorem begin}
{%
  \begin{\inserttheoremblockenv}
  {%
    \inserttheoremheadfont
    \inserttheoremname
    \inserttheoremnumber
    \ifx\inserttheoremaddition\@empty\else\ (\inserttheoremaddition)\fi%
    \inserttheorempunctuation
  }%
}
\setbeamertemplate{theorem end}{\end{\inserttheoremblockenv}}

\newenvironment{theoreme}[1][]{%
   \setbeamercolor{block title}{fg=structure,bg=structure!40}
   \setbeamercolor{block body}{fg=black,bg=structure!10}
   \begin{block}{{\bf Th\'eor\`eme }#1}
}{%
   \end{block}%
}


\newenvironment{proposition}[1][]{%
   \setbeamercolor{block title}{fg=structure,bg=structure!40}
   \setbeamercolor{block body}{fg=black,bg=structure!10}
   \begin{block}{{\bf Proposition }#1}
}{%
   \end{block}%
}

\newenvironment{corollaire}[1][]{%
   \setbeamercolor{block title}{fg=structure,bg=structure!40}
   \setbeamercolor{block body}{fg=black,bg=structure!10}
   \begin{block}{{\bf Corollaire }#1}
}{%
   \end{block}%
}

\newenvironment{mydefinition}[1][]{%
   \setbeamercolor{block title}{fg=structure,bg=structure!40}
   \setbeamercolor{block body}{fg=black,bg=structure!10}
   \begin{block}{{\bf Définition} #1}
}{%
   \end{block}%
}

\newenvironment{lemme}[0]{%
   \setbeamercolor{block title}{fg=structure,bg=structure!40}
   \setbeamercolor{block body}{fg=black,bg=structure!10}
   \begin{block}{\bf Lemme}
}{%
   \end{block}%
}

\newenvironment{remarque}[1][]{%
   \setbeamercolor{block title}{fg=black,bg=structure!20}
   \setbeamercolor{block body}{fg=black,bg=structure!5}
   \begin{block}{Remarque #1}
}{%
   \end{block}%
}


\newenvironment{exemple}[1][]{%
   \setbeamercolor{block title}{fg=black,bg=structure!20}
   \setbeamercolor{block body}{fg=black,bg=structure!5}
   \begin{block}{{\bf Exemple }#1}
}{%
   \end{block}%
}


\newenvironment{miniexercice}[0]{%
   \setbeamercolor{block title}{fg=structure,bg=structure!20}
   \setbeamercolor{block body}{fg=black,bg=structure!5}
   \begin{block}{Mini-exercices}
}{%
   \end{block}%
}


\newenvironment{tp}[0]{%
   \setbeamercolor{block title}{fg=structure,bg=structure!40}
   \setbeamercolor{block body}{fg=black,bg=structure!10}
   \begin{block}{\bf Travaux pratiques}
}{%
   \end{block}%
}
\newenvironment{exercicecours}[1][]{%
   \setbeamercolor{block title}{fg=structure,bg=structure!40}
   \setbeamercolor{block body}{fg=black,bg=structure!10}
   \begin{block}{{\bf Exercice }#1}
}{%
   \end{block}%
}
\newenvironment{algo}[1][]{%
   \setbeamercolor{block title}{fg=structure,bg=structure!40}
   \setbeamercolor{block body}{fg=black,bg=structure!10}
   \begin{block}{{\bf Algorithme}\hfill{\color{gray}\texttt{#1}}}
}{%
   \end{block}%
}


\setbeamertemplate{proof begin}{
   \setbeamercolor{block title}{fg=black,bg=structure!20}
   \setbeamercolor{block body}{fg=black,bg=structure!5}
   \begin{block}{{\footnotesize Démonstration}}
   \footnotesize
   \smallskip}
\setbeamertemplate{proof end}{%
   \end{block}}
\setbeamertemplate{qed symbol}{\openbox}


\makeatother
\usecolortheme[RGB={153, 204, 0}]{structure}

% Commande spécifique à ce chapitre
\newcounter{saveenumi}

%%%%%%%%%%%%%%%%%%%%%%%%%%%%%%%%%%%%%%%%%%%%%%%%%%%%%%%%%%%%%
%%%%%%%%%%%%%%%%%%%%%%%%%%%%%%%%%%%%%%%%%%%%%%%%%%%%%%%%%%%%%



\begin{document}



\title{{\bf Arithmétique}}
\subtitle{Théorème de Bézout}

\begin{frame}
  
  \debutmontitre

  \pause

{\footnotesize
\hfill
\setbeamercovered{transparent=50}
\begin{minipage}{0.6\textwidth}
  \begin{itemize}
    \item<3-> Théorème de Bézout
    \item<4-> Corollaires du théorème de Bézout
    \item<5-> \'Equations $ax+by=c$
    \item<6-> ppcm
  \end{itemize}
\end{minipage}
}

\end{frame}

\setcounter{framenumber}{0}


%%%%%%%%%%%%%%%%%%%%%%%%%%%%%%%%%%%%%%%%%%%%%%%%%%%%%%%%%%%%%%%%


%%%%%%%%%%%%%%%%%%%%%%%%%%%%%%%%%%%%%%%%%%%%%%%%%%%%%%%%%%%%%%%%
\section{Théorème de Bézout}

\begin{frame}
  
\begin{theoreme}[de Bézout]
Soient $a,b$ des entiers. Il existe
des entiers $u,v \in \Zz$ tels que 
\mybox{$au+bv=\pgcd(a,b)$}
\end{theoreme}

\pause
\bigskip

\begin{itemize}
  \item Les entiers $u,v$ sont des \defi{coefficients de Bézout}
  \item Ils s'obtiennent en <<remontant>> l'algorithme d'Euclide
  \item Les entiers $u,v$ ne sont pas uniques
\end{itemize}

\end{frame}


\begin{frame}
\begin{exemple}
Calcul des coefficients de Bézout pour $a=600$ et $b=124$
\pause\smallskip
{
$$\footnotesize
\begin{array}{rclclcl}
&&&&& \\
600 & = & 124 & \times & 4 & + & 104\tikz[remember picture]\coordinate(bezun); \\ \\
124 & = & 104 & \times & 1 & + & 20 \\ \\
104 & = & 20  & \times & 5 & + & \textcolor{red}{4}\tikz[remember picture]\coordinate(bezdeux); \\ \\
20  & = & 4   & \times & 5 & + & 0 \\
\end{array}
\qquad
\pause
\begin{array}{rcl}
\uncover<8->{&=& 600 \times \textcolor{blue}{6} + 124 \times (\textcolor{blue}{-29})} \\
\uncover<7->{\tikz[remember picture]\coordinate(bezquatre);\textcolor{red}{4} &=& 124 \times (-5) + (600-124\times4) \times 6}   \\      
\uncover<6->{&=& 124 \times (-5) + 104 \times 6}\\
\uncover<5->{\textcolor{red}{4} &=& 104 - (124-104\times 1) \times 5}  \\ 
\\
\uncover<4->{\tikz[remember picture]\coordinate(beztrois);\textcolor{red}{4} &=& 104 - 20 \times 5}  \\ \\ \\ 
\end{array}
$$
\begin{tikzpicture}[x=1mm,y=1mm, remember picture, overlay]
   \coordinate (mybezun) at ($(bezun)+(+3,+2)$);
   \coordinate (mybezdeux) at ($(bezdeux)+(+6,-5)$);
   \coordinate (mybeztrois) at ($(beztrois)+(-3,-5)$);
   \coordinate (mybezquatre) at ($(bezquatre)+(-3,+2)$);
   \draw[->, myred, very thick] (mybezun)..controls(mybezdeux) and (mybeztrois)..(mybezquatre);
\end{tikzpicture}
}
\pause \pause \pause \pause \pause \pause
\smallskip

Pour $u=6$ et $v=-29$ alors $600 \times 6 + 124 \times (-29)=4$. Vérifiez !
\end{exemple}
\end{frame}

\begin{frame}

\begin{exemple}
Calcul des coefficients de Bézout pour $9945u+3003v=39$
\pause
$$\small
\begin{array}{rclclcl}
9945 & = & 3003 & \times & 3 & + & 936\tikz[remember picture]\coordinate(bbezun); \\ 
3003 & = & 936  & \times & 3 & + & 195 \\
936  & = & 195  & \times & 4 & + & 156 \\
195  & = & 156  & \times & 1 & + & \textcolor{red}{39}\tikz[remember picture]\coordinate(bbezdeux); \\
156  & = & 39   & \times & 4 & + & 0 \\
\end{array}
\pause\qquad
\uncover<4->{
\begin{array}{rclclcl}
\uncover<6->{\tikz[remember picture]\coordinate(bbezquatre);\textcolor{red}{39}  & = & 9945 \times (\textcolor{blue}{-16}) +  3003\times \textcolor{blue}{53}}  \\ 
\uncover<6->{\textcolor{red}{39}  & = & \cdots} \\
\uncover<5->{\textcolor{red}{39}  & = & \cdots} \\
\tikz[remember picture]\coordinate(bbeztrois);\textcolor{red}{39}  & = & 195- 156  \times  1 \\
&& \\
\end{array}
}
$$
\begin{tikzpicture}[x=1mm,y=1mm, remember picture, overlay]
   \coordinate (mybbezun) at ($(bbezun)+(+3,+2)$);
   \coordinate (mybbezdeux) at ($(bbezdeux)+(+6,-5)$);
   \coordinate (mybbeztrois) at ($(bbeztrois)+(-3,-5)$);
   \coordinate (mybbezquatre) at ($(bbezquatre)+(-3,+2)$);

   \draw[->, myred, very thick] (mybbezun)..controls(mybbezdeux) and (mybbeztrois)..(mybbezquatre);
\end{tikzpicture}
\pause
\pause
\pause
\pause
D'où $9945 \times (-16) +  3003\times 53 = 39$
\end{exemple}
  
\end{frame}


%---------------------------------------------------------------
\section{Corollaires du théorème de Bézout}

\begin{frame}

\begin{corollaire}
$a,b$ sont premiers entre eux 
\textbf{si et seulement si} il existe $u,v \in \Zz$ tels que
\vspace*{-3ex}
\mybox{$au+bv=1$}
\end{corollaire}

\pause
\medskip

\begin{proof}
$\Rightarrow$ Théorème de Bézout

\pause
\smallskip

$\Leftarrow$ On suppose qu'il existe $u,v$ tels que $au+bv=1$

Comme $\pgcd(a,b)|a$ alors $\pgcd(a,b)|au$

De même $\pgcd(a,b)|bv$

Donc $\pgcd(a,b)|au+bv=1$. Donc $\pgcd(a,b)=1$
\end{proof}

\pause
\medskip


Si on trouve $u',v'$ tels que $au'+bv'=d$, cela implique seulement que $\pgcd(a,b)|d$

\pause

Exemple : $a=12$, $b=8$ ; $12 \times 1 + 8 \times 3 = 36$ et $\pgcd(a,b)=4$


\end{frame}



\begin{frame}

\begin{corollaire}
Si $d|a$ et $d|b$ alors $d | \pgcd(a,b)$
\end{corollaire}

\pause
\medskip

Exemple : $4|16$ et $4|24$ donc $4$ divise $\pgcd(16,24)$


\pause
\bigskip
\bigskip

\begin{corollaire}[Lemme de Gauss]
\mybox{Si \ \ $a | bc$ \ \  et \ \  $\pgcd(a,b)=1$ \ \  alors \ \  $a|c$}
\end{corollaire}

\pause
\medskip
Exemple : si $4 \, | \, 7c$ alors $4|c$


\end{frame}

%---------------------------------------------------------------
\section{\'Equations $ax+by=c$}


\label{ssec:dioph}

\begin{frame}

\begin{proposition}
\label{prop:dioph}
Soient $a,b,c \in \Zz$. L'équation
\begin{equation}
\label{eq:bezout}
\tag{E}
\myboxinline{ax+by=c}
\end{equation}
\pause
\begin{enumerate}
  \item (\ref{eq:bezout}) possède des solutions $(x,y)\in\! \Zz^2$ si et seulement si
$\pgcd(a,b) | c$

\pause

  \item Les solutions sont alors les 
$(x,y) = (x_0+ \alpha k, y_0 + \beta k)$ avec $k$ parcourant $\Zz$ et $x_0,y_0,\alpha,\beta \in \Zz$ fixés
\end{enumerate}
\end{proposition}
\end{frame}


\begin{frame}

\begin{exemple}
Trouver les solutions entières de 
\vspace*{-1ex}
\begin{equation}
\label{eq:exbezout}
\tag{E}
\myboxinline{161x+368y=115}
\end{equation}
\end{exemple}

\pause
\medskip

\textbf{Première étape.} \evidence{Y a-t'il des solutions ? L'algorithme d'Euclide}

\pause

Calcul du pgcd de $a=161$ et $b=368$

\pause

$$
\begin{array}{rclclcl}
368 & = & 161 & \times & 2 & + & 46 \\ 
\pause
161 & = & 46 & \times & 3 & + & \alert<7>{23} \\
\pause
46  & = & 23   & \times & 2 & + & 0 \\
\end{array}
$$

\pause
Donc $\pgcd(368,161)=23$

\pause
\medskip

Comme $115=5\times23$ alors $\pgcd(368,161) | 115$

\pause

Par le théorème de Bézout, l'équation (\ref{eq:exbezout}) admet des solutions entières


\end{frame}



\begin{frame}
\begin{equation}
\tag{E}
161x+368y=115 
\end{equation}


\textbf{Deuxième étape.} \evidence{Trouver une solution particulière : la remontée de l'algorithme d'Euclide}
\pause
$$
\hspace*{-1em}\begin{array}{rclclcl}
\\
368 & = & 161 & \times & 2 & + & 46 \\ 
161 & = & 46 & \times & 3 & + & \textcolor{red}{23} \\
46  & = & 23   & \times & 2 & + & 0 \\
\end{array}
\quad 
\pause
\begin{array}{rcl}
\uncover<5->{&=&\small 161 \times \textcolor{blue}{7} + 368 \times (\textcolor{blue}{-3})}\\
\uncover<4->{\textcolor{red}{23} &=& 161 + (368-2\times 161)\times (-3)}\\        
\uncover<3->{\textcolor{red}{23} &=& 161 + 46\times (-3)}  \\
&& \\
\end{array}
$$
\pause\pause\pause
$$161\times\textcolor{blue}{7} + 368\times(\textcolor{blue}{-3})=23$$

\pause

Comme $115=5\times23$ on multiplie par $5$ 
\pause
$$161\times\textcolor{blue}{35} + 368\times(\textcolor{blue}{-15})=115$$

\pause

$(x_0,y_0) = (35,-15)$ est une \defi{solution particulière} de (\ref{eq:exbezout})

\end{frame}


\begin{frame}

\textbf{Troisième étape.} \evidence{Recherche de toutes les solutions}

\pause

Soit $(x,y) \in \Zz^2$ une solution de (\ref{eq:exbezout})
\uncover<4->{\quad $(x_0,y_0)$ est aussi solution}
$$\uncover<3->{161x+368y=115} \quad \pause \uncover<5->{\text{ et } \quad 161x_0+368y_0=115}$$
\pause\pause\pause
\vspace*{-2ex}
$$\begin{array}{ll}
& 161 \times (x-x_0) + 368 \times (y-y_0) = 0 \\
\pause
\implies \quad & 23 \times 7 \times (x-x_0) + 23 \times 16 \times  (y-y_0) = 0 \\
\pause
\implies \quad & 7 (x-x_0) = -16 (y-y_0) \qquad  (*)\\
\end{array}$$
\vspace*{-2ex}
\pause
\begin{itemize}
  \item Ainsi $7 | 16(y-y_0)$,
\pause
or $\pgcd(7,16)=1$ donc par le lemme de Gauss $7|y-y_0$.
\pause
Il existe donc $k\in \Zz$ tel que $y-y_0 = 7 \times k$
\pause
  \item Repartons de l'équation $(*)$ : $7 (x-x_0) = -16 (y-y_0)$
\pause
$\implies$ $7(x-x_0)=-16\times 7 \times k$
\pause
$\implies$ $x-x_0 = -16k$
\pause
  \item Donc $(x,y) = (x_0-16k,y_0+7k)$ \pause\quad  avec \quad  $(x_0,y_0)=(35,-15)$
\end{itemize}

\pause

\mybox{
\begin{minipage}{0.8\textwidth}
\center
Les solutions entières de $161x+368y=115$
sont les $(x,y) = (35-16k,-15+7k)$, $k$ parcourant $\Zz$  
\end{minipage}
}

  
\end{frame}

%---------------------------------------------------------------
\section{ppcm}

\begin{frame}


\begin{mydefinition}
Le $\text{ppcm}(a,b)$ (\defi{plus petit multiple commun}) 
est le plus petit entier positif divisible par $a$ et par $b$
\end{mydefinition}

\pause

Exemple : $\text{ppcm}(12,9)=36$

\pause

\begin{proposition}
\mybox{$\pgcd(a,b) \times \text{ppcm}(a,b) = |ab|$} 
\end{proposition}

\pause

\begin{proposition}
Si $a|c$ et $b|c$ alors $\text{ppcm}(a,b)|c$
\end{proposition}

\pause

Exemple : $6|36$, $9|36$ alors $\text{ppcm}(6,9)=18$ divise bien $36$

\pause

Il serait faux de penser que $ab|c$ : ici $6\times 9$ ne divise pas $36$


\end{frame}




%---------------------------------------------------------------
\section*{Mini-exercices}

\begin{frame}
\begin{miniexercice}
\begin{enumerate}
  \item Calculer les coefficients de Bézout correspondant à $\pgcd(560,133)$, $\pgcd(12\,121,789)$.
  \item Montrer à l'aide d'un corollaire du théorème de Bézout que $\pgcd(a,a+1)=1$.
  \item Résoudre les équations : $407x+129y=1$ ; $720x+54y=6$ ; $216x+92y=8$.
  \item Trouver les couples $(a,b)$ vérifiant $\pgcd(a,b)=12$ et $\text{ppcm}(a,b)=360$.
\end{enumerate}  
\end{miniexercice}
\end{frame}


\end{document}