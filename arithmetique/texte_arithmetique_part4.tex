
%%%%%%%%%%%%%%%%%% PREAMBULE %%%%%%%%%%%%%%%%%%


\documentclass[12pt]{article}

\usepackage{amsfonts,amsmath,amssymb,amsthm}
\usepackage[utf8]{inputenc}
\usepackage[T1]{fontenc}
\usepackage[francais]{babel}


% packages
\usepackage{amsfonts,amsmath,amssymb,amsthm}
\usepackage[utf8]{inputenc}
\usepackage[T1]{fontenc}
%\usepackage{lmodern}

\usepackage[francais]{babel}
\usepackage{fancybox}
\usepackage{graphicx}

\usepackage{float}

%\usepackage[usenames, x11names]{xcolor}
\usepackage{tikz}
\usepackage{datetime}

\usepackage{mathptmx}
%\usepackage{fouriernc}
%\usepackage{newcent}
\usepackage[mathcal,mathbf]{euler}

%\usepackage{palatino}
%\usepackage{newcent}


% Commande spéciale prompteur

%\usepackage{mathptmx}
%\usepackage[mathcal,mathbf]{euler}
%\usepackage{mathpple,multido}

\usepackage[a4paper]{geometry}
\geometry{top=2cm, bottom=2cm, left=1cm, right=1cm, marginparsep=1cm}

\newcommand{\change}{{\color{red}\rule{\textwidth}{1mm}\\}}

\newcounter{mydiapo}

\newcommand{\diapo}{\newpage
\hfill {\normalsize  Diapo \themydiapo \quad \texttt{[\jobname]}} \\
\stepcounter{mydiapo}}


%%%%%%% COULEURS %%%%%%%%%%

% Pour blanc sur noir :
%\pagecolor[rgb]{0.5,0.5,0.5}
% \pagecolor[rgb]{0,0,0}
% \color[rgb]{1,1,1}



%\DeclareFixedFont{\myfont}{U}{cmss}{bx}{n}{18pt}
\newcommand{\debuttexte}{
%%%%%%%%%%%%% FONTES %%%%%%%%%%%%%
\renewcommand{\baselinestretch}{1.5}
\usefont{U}{cmss}{bx}{n}
\bfseries

% Taille normale : commenter le reste !
%Taille Arnaud
%\fontsize{19}{19}\selectfont

% Taille Barbara
%\fontsize{21}{22}\selectfont

%Taille François
\fontsize{25}{30}\selectfont

%Taille Pascal
%\fontsize{25}{30}\selectfont

%Taille Laura
%\fontsize{30}{35}\selectfont


%\myfont
%\usefont{U}{cmss}{bx}{n}

%\Huge
%\addtolength{\parskip}{\baselineskip}
}


% \usepackage{hyperref}
% \hypersetup{colorlinks=true, linkcolor=blue, urlcolor=blue,
% pdftitle={Exo7 - Exercices de mathématiques}, pdfauthor={Exo7}}


%section
% \usepackage{sectsty}
% \allsectionsfont{\bf}
%\sectionfont{\color{Tomato3}\upshape\selectfont}
%\subsectionfont{\color{Tomato4}\upshape\selectfont}

%----- Ensembles : entiers, reels, complexes -----
\newcommand{\Nn}{\mathbb{N}} \newcommand{\N}{\mathbb{N}}
\newcommand{\Zz}{\mathbb{Z}} \newcommand{\Z}{\mathbb{Z}}
\newcommand{\Qq}{\mathbb{Q}} \newcommand{\Q}{\mathbb{Q}}
\newcommand{\Rr}{\mathbb{R}} \newcommand{\R}{\mathbb{R}}
\newcommand{\Cc}{\mathbb{C}} 
\newcommand{\Kk}{\mathbb{K}} \newcommand{\K}{\mathbb{K}}

%----- Modifications de symboles -----
\renewcommand{\epsilon}{\varepsilon}
\renewcommand{\Re}{\mathop{\text{Re}}\nolimits}
\renewcommand{\Im}{\mathop{\text{Im}}\nolimits}
%\newcommand{\llbracket}{\left[\kern-0.15em\left[}
%\newcommand{\rrbracket}{\right]\kern-0.15em\right]}

\renewcommand{\ge}{\geqslant}
\renewcommand{\geq}{\geqslant}
\renewcommand{\le}{\leqslant}
\renewcommand{\leq}{\leqslant}

%----- Fonctions usuelles -----
\newcommand{\ch}{\mathop{\mathrm{ch}}\nolimits}
\newcommand{\sh}{\mathop{\mathrm{sh}}\nolimits}
\renewcommand{\tanh}{\mathop{\mathrm{th}}\nolimits}
\newcommand{\cotan}{\mathop{\mathrm{cotan}}\nolimits}
\newcommand{\Arcsin}{\mathop{\mathrm{Arcsin}}\nolimits}
\newcommand{\Arccos}{\mathop{\mathrm{Arccos}}\nolimits}
\newcommand{\Arctan}{\mathop{\mathrm{Arctan}}\nolimits}
\newcommand{\Argsh}{\mathop{\mathrm{Argsh}}\nolimits}
\newcommand{\Argch}{\mathop{\mathrm{Argch}}\nolimits}
\newcommand{\Argth}{\mathop{\mathrm{Argth}}\nolimits}
\newcommand{\pgcd}{\mathop{\mathrm{pgcd}}\nolimits} 

\newcommand{\Card}{\mathop{\text{Card}}\nolimits}
\newcommand{\Ker}{\mathop{\text{Ker}}\nolimits}
\newcommand{\id}{\mathop{\text{id}}\nolimits}
\newcommand{\ii}{\mathrm{i}}
\newcommand{\dd}{\mathrm{d}}
\newcommand{\Vect}{\mathop{\text{Vect}}\nolimits}
\newcommand{\Mat}{\mathop{\mathrm{Mat}}\nolimits}
\newcommand{\rg}{\mathop{\text{rg}}\nolimits}
\newcommand{\tr}{\mathop{\text{tr}}\nolimits}
\newcommand{\ppcm}{\mathop{\text{ppcm}}\nolimits}

%----- Structure des exercices ------

\newtheoremstyle{styleexo}% name
{2ex}% Space above
{3ex}% Space below
{}% Body font
{}% Indent amount 1
{\bfseries} % Theorem head font
{}% Punctuation after theorem head
{\newline}% Space after theorem head 2
{}% Theorem head spec (can be left empty, meaning ‘normal’)

%\theoremstyle{styleexo}
\newtheorem{exo}{Exercice}
\newtheorem{ind}{Indications}
\newtheorem{cor}{Correction}


\newcommand{\exercice}[1]{} \newcommand{\finexercice}{}
%\newcommand{\exercice}[1]{{\tiny\texttt{#1}}\vspace{-2ex}} % pour afficher le numero absolu, l'auteur...
\newcommand{\enonce}{\begin{exo}} \newcommand{\finenonce}{\end{exo}}
\newcommand{\indication}{\begin{ind}} \newcommand{\finindication}{\end{ind}}
\newcommand{\correction}{\begin{cor}} \newcommand{\fincorrection}{\end{cor}}

\newcommand{\noindication}{\stepcounter{ind}}
\newcommand{\nocorrection}{\stepcounter{cor}}

\newcommand{\fiche}[1]{} \newcommand{\finfiche}{}
\newcommand{\titre}[1]{\centerline{\large \bf #1}}
\newcommand{\addcommand}[1]{}
\newcommand{\video}[1]{}

% Marge
\newcommand{\mymargin}[1]{\marginpar{{\small #1}}}



%----- Presentation ------
\setlength{\parindent}{0cm}

%\newcommand{\ExoSept}{\href{http://exo7.emath.fr}{\textbf{\textsf{Exo7}}}}

\definecolor{myred}{rgb}{0.93,0.26,0}
\definecolor{myorange}{rgb}{0.97,0.58,0}
\definecolor{myyellow}{rgb}{1,0.86,0}

\newcommand{\LogoExoSept}[1]{  % input : echelle
{\usefont{U}{cmss}{bx}{n}
\begin{tikzpicture}[scale=0.1*#1,transform shape]
  \fill[color=myorange] (0,0)--(4,0)--(4,-4)--(0,-4)--cycle;
  \fill[color=myred] (0,0)--(0,3)--(-3,3)--(-3,0)--cycle;
  \fill[color=myyellow] (4,0)--(7,4)--(3,7)--(0,3)--cycle;
  \node[scale=5] at (3.5,3.5) {Exo7};
\end{tikzpicture}}
}



\theoremstyle{definition}
%\newtheorem{proposition}{Proposition}
%\newtheorem{exemple}{Exemple}
%\newtheorem{theoreme}{Théorème}
\newtheorem{lemme}{Lemme}
\newtheorem{corollaire}{Corollaire}
%\newtheorem*{remarque*}{Remarque}
%\newtheorem*{miniexercice}{Mini-exercices}
%\newtheorem{definition}{Définition}




%definition d'un terme
\newcommand{\defi}[1]{{\color{myorange}\textbf{\emph{#1}}}}
\newcommand{\evidence}[1]{{\color{blue}\textbf{\emph{#1}}}}



 %----- Commandes divers ------

\newcommand{\codeinline}[1]{\texttt{#1}}

%%%%%%%%%%%%%%%%%%%%%%%%%%%%%%%%%%%%%%%%%%%%%%%%%%%%%%%%%%%%%
%%%%%%%%%%%%%%%%%%%%%%%%%%%%%%%%%%%%%%%%%%%%%%%%%%%%%%%%%%%%%

\begin{document}

\debuttexte


%%%%%%%%%%%%%%%%%%%%%%%%%%%%%%%%%%%%%%%%%%%%%%%%%%%%%%%%%%%
\diapo

\change

\change

Dans cette leçon nous allons définir la notion de congruence
et apprendre à calculer modulo $n$.

\change

Nous résoudrons ensuite les équations du type $ax \equiv b \pmod n$

\change

Nous terminerons par le petit théorème de Fermat et ses applications.


%%%%%%%%%%%%%%%%%%%%%%%%%%%%%%%%%%%%%%%%%%%%%%%%%%%%%%%%%%%
\diapo



Fixons un entier $n\ge 2$ et deux entiers $a,b$.


On dit que $a$ est congru à $b$ modulo $n$ si $n$ divise $b-a$

\change

On note alors 
$$a \equiv b \pmod n$$ 

\change

Une autre formulation est de dire que 

$a \equiv b \pmod n \quad \iff \quad \exists k \in \Zz \quad a = b +kn$

\change
Quelque exemples avec $n=7$.

$15 \equiv 1 \pmod 7$, car $15 -1 = 14$ est un multiple de $7$. 

\change

$72 \equiv 2 \pmod {7}$, car $72-2 = 70$ qui est un multiple de $7$.


\change

Enfin

$3 \equiv -11 \pmod{7}$

Car $3 - (-11) = 14$.

\change

Voici aussi quelques remarques :

tout d'abord 

$a\equiv 0 \pmod n$

équivaut à $a=0+kn$ qui équivaut à $n \text{ divise } a$

\change

D'autres notations sont utilisées

$a$ égal $b \pmod n$ \quad  ou bien  \quad  $a \equiv b$ crochets $[n]$

\change 

La division euclidienne permet de trouver un ``bon'' représentant 
de $a$ modulo $n$.

En effet si $a=bn+r$ alors $a\equiv r \pmod n$,
et l'on sait que le reste est un entier positif strictement inférieur à $n$.

Tout entier $a$ est donc congru à un entier compris entre $0$ et $n-1$.

\change

Enfin la relation congru modulo $n$ vérifie les trois points suivants qui en font une relation d'équivalence :

\change

1. $a \equiv a \pmod n$

\change

2. si $a \equiv b \pmod n$ alors $b \equiv a \pmod n$

\change

3. si $a \equiv b \pmod n$ et $b \equiv c \pmod n$ alors $a \equiv c \pmod n$


%%%%%%%%%%%%%%%%%%%%%%%%%%%%%%%%%%%%%%%%%%%%%%%%%%%%%%%%%%%
\diapo

Continuons avec des propriétés qui faciliteront les calculs.


On part de deux entiers $a,b$ avec  $a \equiv b \pmod n$ et 
deux entiers $c,d$ avec $c \equiv d \pmod n$


\change

 $a+c \equiv b+d \pmod n$

\change

 $a\times c \equiv b \times d \pmod n$

\change

et aussi 

 $a^k \equiv b^k \pmod n$ (pour tout entier $k$)

\change

Cela nous permet par exemple d'affirmer que 

$5x+8 \equiv 3 \pmod 5$ \quad quelque soit l'entier $x$.

En effet 

$8$ est congru à $3$ modulo $5$

et $5$ est congru à $0$ modulo $5$
donc $5 \times x$ vaut aussi $0$ modulo $5$.

Par addition $5x+8$ est congru à $3$ !


\change

  $11$ est congru à $1$ modulo $10$ donc par le troisième
point pour n'importe quel exposant $k$ $11^k$ est congru 
à $1^k$ donc à $1$ modulo $10$.

\change

Montrons le point 2. de la proposition.

c'est-à-dire que si $a \equiv b \pmod n$ et $c \equiv d \pmod n$ alors $a\times c \equiv b \times d \pmod n$

\change

Tout d'abord  $a \equiv b \pmod n$ signifie qu'il existe $k \in \Zz$ tel que $a=b + kn$ 

\change

et de même $c \equiv d \pmod n$ donc il existe $\ell \in \Zz$ tel que $c \equiv d + \ell n$

\change

Calculons le produit $a\times c$ :

$a\times c = (b+kn)\times(d+\ell n)$

$ = bd + (b \ell + d k + k\ell n)n = bd + mn$

$ac$ est donc égal à $bd$ plus un multiple de $n$

\change

C'est exactement dire que $ac \equiv bd \pmod n$


%%%%%%%%%%%%%%%%%%%%%%%%%%%%%%%%%%%%%%%%%%%%%%%%%%%%%%%%%%%
\diapo

Nous pouvons montrer le critère de divisibilité par $9$ que vous 
connaissez depuis longtemps.

\change

Je rappelle qu'un nombre est divisible par $9$

si et seulement si la somme de ses chiffres est divisible par $9$.

\change

La divisibilité par $9$ équivaut à ce que $N$ soit congru à $0$ modulo $9$.

\change

Calculons les puissances de $10$ :

 $10 \equiv 1 \pmod 9$, car $10-1$ est un mulitple de $9$,

\change

$100$ est aussi congru à $1$ modulo $9$, car
par exemple $100-1=99$ est un muliple de $9$,

un autre argument est que $100=10^2$ et que l'on vient de calculer $10$ modulo $9$ donc $100 \equiv 1^2 \pmod 9$.

\change

Ainsi de suite, $10^k \equiv 1 \pmod 9$.

\change

Revenons à notre entier $N$ et écrivons-le en base $10$ 

$N = \underline{a_k \cdots a_2 a_1 a_0}$

$a_0$ est le chiffre des unités, $a_1$ celui des dizaines, $a_2$ celui des centaines, etc.

\change

Cela se traduit par l'égalité 
$N =  a_0 + 10 a_1 + 10^2 a_2 + \cdots 10^k a_k$.

\change

Passons aux calculs modulo $9$.


$N = 10^k a_k + \cdots + 10^2 a_2 + 10^1 a_1 + a_0$

Donc 
$ N \equiv  a_k + \cdots + a_2 + a_1 + a_0 \pmod 9 $

Car chacun des $10^k \equiv 1 \pmod 9$.


\change

Maintenant on a bien $N \equiv 0 \pmod 9$ $\iff$ la somme des chiffres vaut $0$ modulo $9$

\change

(pause)

Pour illustrer, montrons que 

$N = 488\, 889$ est divisible par $9$.

\change

On décompose $N$ avec des puissances de $10$

\change

On réduit modulo $9$, seul les chiffres restes

\change

La somme vaut $45$

\change

On recalcule la somme des chiffres de $45$

qui vaut $9$

\change

qui vaut aussi $0$ modulo $9$.

Ainsi ce $N$ est bien un multiple de $9$.


%%%%%%%%%%%%%%%%%%%%%%%%%%%%%%%%%%%%%%%%%%%%%%%%%%%%%%%%%%%
\diapo

[2 prises]

Les calculs bien menés avec les congruences peuvent être très rapides.

Voyons différentes façons de calculer $2^{21} \pmod {37}$

Plus exactement on souhaite trouver un nombre inférieur à $37$ qui est congru à $2^{21}$ modulo ${37}$. 

\change

Première méthode : On calcule $2^{21}$, puis on fait la division euclidienne de $2^{21}$ par $37$ le reste est notre résultat.
C'est laborieux !

\change

Deuxième  méthode : On calcule successivement les $2^k$ modulo $37$ 

\change

$2^1 \equiv 2 \pmod {37}$,

\change

$2^2 \equiv 4 \pmod {37}$, 

\change

$2^3 \equiv 8 \pmod {37}$,


\change

$2^4 \equiv 16 \pmod {37}$,

\change

$2^5 \equiv 32 \pmod {37}$. 

\change


$2^6 \equiv 64$

Ensuite on n'oublie pas d'utiliser les congruences :

\change

$64 \equiv 27 \pmod {37}$.

\change 

Puis pour $2^7$ on écrit

$2^7 = 2 \cdot 2^6$

et on utilise le résultat précédent :
donc

\change

\change

$2^7  \equiv 2 \cdot 2^6  \equiv 2 \cdot 27 \equiv 54 \equiv 17 \pmod {37}$ 


et ainsi de suite en utilisant le calcul précédent à chaque étape.

\change

C'est assez efficace et on peut raffiner : par exemple on trouve $2^{8} \equiv 34 \pmod {37}$
mais donc aussi $2^{8} \equiv -3 \pmod {37}$ et donc 

\change

\change


$2^9 \equiv 2 \cdot 2^8 \equiv 2 \cdot(-3) \equiv -6 \equiv 31 \pmod{37}$...

C'est une méthode qui conduit assez rapidement au résultat.

\change


Il existe une méthode encore plus efficace.

\change

on écrit l'exposant $21$ en base $2$ :
$21 = 2^{4} + 2^{2} + 2^{0} = 16 + 4 + 1$.

\change

 Alors
$2^{21} =  2^{16} \cdot 2^{4} \cdot 2^{1}$. 

Et il est facile de calculer successivement chacun des ces termes 
car les exposants sont des puissances de $2$. 


\change

En effet $2^8 \equiv  (2^4)^2 \equiv 16^2 \equiv 256 \equiv 34 \equiv -3 \pmod {37}$

\change

Ce qui permet de calculer $2^{16}$ égal $2^8$ au carré.

\change

Donc 

 $2^{16} \equiv  \left(2^{8}\right)^2 
\equiv (-3)^2 \equiv 9 \pmod{37}$


\change


Ainsi  $2^{21} \equiv  2^{16} \cdot 2^{4} \cdot 2^{1} $


\change

$ \equiv 9 \times 16 \times 2 $

\change


$\equiv 288$

\change

$\equiv 29 \pmod {37}$

Conclusion 

$2^{21}$ est congru à $29$ modulo $37$.


%%%%%%%%%%%%%%%%%%%%%%%%%%%%%%%%%%%%%%%%%%%%%%%%%%%%%%%%%%%
\diapo

Etant donné trois entier $a,b$ et $n$
Considérons l'équation 
$ax \equiv b \pmod n$ d'inconnue $x$, $x$ étant aussi un entier.

\change

Voyons comment résoudre ces équation avec l'exemple

$9x \equiv 6 \pmod{24}$

\change

Trouver $x$ tel que $9x \equiv 6 \pmod{24}$ est équivalent à trouver $x$ et $k$ tels que
$9x = 6 + 24k$. 

\change

Mis sous la forme $9x-24k=6$ il s'agit alors d'une équation que nous avons étudié en détails
dans une leçon précédente.

Il y a bien des solutions car $\pgcd(9,24)=3$ divise $6$.

\change

En divisant par le pgcd on obtient l'équation équivalente :
 $$3x-8k=2.$$

\change


Pour le calcul du pgcd et d'une solution particulière nous utilisons normalement l'algorithme d'Euclide et sa remontée.
Ici il est facile de trouver une solution particulière $(x_0=6,k_0=2)$ à la main.

\change

On termine comme d'habitude, une fois que l'on a une solution particulière on sait
trouver toutes les solutions. Faites les calculs
pour trouver que les $x$ solutions sont les  $x=6+ 8\ell$. $\ell$ parcourant $\Zz$.

(ici la forme des entiers $k$ ne nous intéresse pas)

\change

On préfère les regrouper en $3$ classes modulo $24$:

$x_1 = 6 + 24 m$, 

$x_2 = 14 + 24 m$,

$x_3=22+24m$

 avec $m$ parcourant $\Zz$



%%%%%%%%%%%%%%%%%%%%%%%%%%%%%%%%%%%%%%%%%%%%%%%%%%%%%%%%%%%
\diapo

Nous énonçons le petit théorème de Fermat.

Si $p$ est un nombre premier et $a \in \Zz$ alors
$a^p \equiv a \pmod p$

Attention aux hypothèses 

1. $p$ est un nombre premier, 

2. l'exposant est $p$ et la congruence est modulo $p$
pour le même $p$.


\change

Une autre formulation -toute aussi utile-
est la suivante :

Si $p$ ne divise pas $a$ on a :
$a^{p-1} \equiv 1 \pmod p$.

Encore une fois $p$ est un nombre premier 

en plus ici $p$ ne doit pas diviser $a$.

On obtient le corollaire en divisant par $a$ dans l'égalité du petit théorème de Fermat.


%%%%%%%%%%%%%%%%%%%%%%%%%%%%%%%%%%%%%%%%%%%%%%%%%%%%%%%%%%%
\diapo

Comme applications nous allons calculer $14^{3141} \pmod {17}$.

\change


Le nombre $17$ étant premier on sait par le petit théorème de Fermat que
$14^{16} \equiv 1 \pmod {17}$.

\change

Essayons d'utiliser cette propriété.

\'Ecrivons la division euclidienne de $3141$ par $16$ :
$$3141 = 16\times 196 + 5.$$

\change

Alors
$14^{3141} = 14^{16 \times 196 + 5}$

\change

qui vaut $14^{16\times 196}\times 14^5$

\change

ou encore 
$ \left(14^{16}\right)^{196}\times 14^5$

\change

Mais par le petit théorème de Fermat 
$14^{16} \equiv 1$  modulo $17$ donc notre nombre est congru à

$ \equiv 1^{196} \times 14^5$

et ainsi 

$14^{3141} \equiv  14^5\pmod{17}$.

\change

Il ne reste plus qu'à calculer $14^5$ modulo $17$.

\change


Cela peut se faire rapidement :
$14 \equiv -3 \pmod {17}$ 


\change

donc $14^2\equiv (-3)^2 \equiv 9 \pmod {17}$,

\change

pour $14^3$ on obtient $14^3 \equiv 7 \pmod{17}$,

\change

et par exemple $14^5$ s'obtient comme $14^5 = 14^2 \times 14^3$ donc

$14^5 \equiv 9 \times 7 \equiv 63 \equiv 12 \pmod {17}$.

\change

Conclusion : $14^{3141}$ qui était congru à $14^5$ est congru à $12 \pmod {17}$.




%%%%%%%%%%%%%%%%%%%%%%%%%%%%%%%%%%%%%%%%%%%%%%%%%%%%%%%%%%%
\diapo


Il vous reste maintenant à répondre aux mini-exercices.


\end{document}