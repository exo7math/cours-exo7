
%%%%%%%%%%%%%%%%%% PREAMBULE %%%%%%%%%%%%%%%%%%


\documentclass[12pt]{article}

\usepackage{amsfonts,amsmath,amssymb,amsthm}
\usepackage[utf8]{inputenc}
\usepackage[T1]{fontenc}
\usepackage[francais]{babel}


% packages
\usepackage{amsfonts,amsmath,amssymb,amsthm}
\usepackage[utf8]{inputenc}
\usepackage[T1]{fontenc}
%\usepackage{lmodern}

\usepackage[francais]{babel}
\usepackage{fancybox}
\usepackage{graphicx}

\usepackage{float}

%\usepackage[usenames, x11names]{xcolor}
\usepackage{tikz}
\usepackage{datetime}

\usepackage{mathptmx}
%\usepackage{fouriernc}
%\usepackage{newcent}
\usepackage[mathcal,mathbf]{euler}

%\usepackage{palatino}
%\usepackage{newcent}


% Commande spéciale prompteur

%\usepackage{mathptmx}
%\usepackage[mathcal,mathbf]{euler}
%\usepackage{mathpple,multido}

\usepackage[a4paper]{geometry}
\geometry{top=2cm, bottom=2cm, left=1cm, right=1cm, marginparsep=1cm}

\newcommand{\change}{{\color{red}\rule{\textwidth}{1mm}\\}}

\newcounter{mydiapo}

\newcommand{\diapo}{\newpage
\hfill {\normalsize  Diapo \themydiapo \quad \texttt{[\jobname]}} \\
\stepcounter{mydiapo}}


%%%%%%% COULEURS %%%%%%%%%%

% Pour blanc sur noir :
%\pagecolor[rgb]{0.5,0.5,0.5}
% \pagecolor[rgb]{0,0,0}
% \color[rgb]{1,1,1}



%\DeclareFixedFont{\myfont}{U}{cmss}{bx}{n}{18pt}
\newcommand{\debuttexte}{
%%%%%%%%%%%%% FONTES %%%%%%%%%%%%%
\renewcommand{\baselinestretch}{1.5}
\usefont{U}{cmss}{bx}{n}
\bfseries

% Taille normale : commenter le reste !
%Taille Arnaud
%\fontsize{19}{19}\selectfont

% Taille Barbara
%\fontsize{21}{22}\selectfont

%Taille François
\fontsize{25}{30}\selectfont

%Taille Pascal
%\fontsize{25}{30}\selectfont

%Taille Laura
%\fontsize{30}{35}\selectfont


%\myfont
%\usefont{U}{cmss}{bx}{n}

%\Huge
%\addtolength{\parskip}{\baselineskip}
}


% \usepackage{hyperref}
% \hypersetup{colorlinks=true, linkcolor=blue, urlcolor=blue,
% pdftitle={Exo7 - Exercices de mathématiques}, pdfauthor={Exo7}}


%section
% \usepackage{sectsty}
% \allsectionsfont{\bf}
%\sectionfont{\color{Tomato3}\upshape\selectfont}
%\subsectionfont{\color{Tomato4}\upshape\selectfont}

%----- Ensembles : entiers, reels, complexes -----
\newcommand{\Nn}{\mathbb{N}} \newcommand{\N}{\mathbb{N}}
\newcommand{\Zz}{\mathbb{Z}} \newcommand{\Z}{\mathbb{Z}}
\newcommand{\Qq}{\mathbb{Q}} \newcommand{\Q}{\mathbb{Q}}
\newcommand{\Rr}{\mathbb{R}} \newcommand{\R}{\mathbb{R}}
\newcommand{\Cc}{\mathbb{C}} 
\newcommand{\Kk}{\mathbb{K}} \newcommand{\K}{\mathbb{K}}

%----- Modifications de symboles -----
\renewcommand{\epsilon}{\varepsilon}
\renewcommand{\Re}{\mathop{\text{Re}}\nolimits}
\renewcommand{\Im}{\mathop{\text{Im}}\nolimits}
%\newcommand{\llbracket}{\left[\kern-0.15em\left[}
%\newcommand{\rrbracket}{\right]\kern-0.15em\right]}

\renewcommand{\ge}{\geqslant}
\renewcommand{\geq}{\geqslant}
\renewcommand{\le}{\leqslant}
\renewcommand{\leq}{\leqslant}

%----- Fonctions usuelles -----
\newcommand{\ch}{\mathop{\mathrm{ch}}\nolimits}
\newcommand{\sh}{\mathop{\mathrm{sh}}\nolimits}
\renewcommand{\tanh}{\mathop{\mathrm{th}}\nolimits}
\newcommand{\cotan}{\mathop{\mathrm{cotan}}\nolimits}
\newcommand{\Arcsin}{\mathop{\mathrm{Arcsin}}\nolimits}
\newcommand{\Arccos}{\mathop{\mathrm{Arccos}}\nolimits}
\newcommand{\Arctan}{\mathop{\mathrm{Arctan}}\nolimits}
\newcommand{\Argsh}{\mathop{\mathrm{Argsh}}\nolimits}
\newcommand{\Argch}{\mathop{\mathrm{Argch}}\nolimits}
\newcommand{\Argth}{\mathop{\mathrm{Argth}}\nolimits}
\newcommand{\pgcd}{\mathop{\mathrm{pgcd}}\nolimits} 

\newcommand{\Card}{\mathop{\text{Card}}\nolimits}
\newcommand{\Ker}{\mathop{\text{Ker}}\nolimits}
\newcommand{\id}{\mathop{\text{id}}\nolimits}
\newcommand{\ii}{\mathrm{i}}
\newcommand{\dd}{\mathrm{d}}
\newcommand{\Vect}{\mathop{\text{Vect}}\nolimits}
\newcommand{\Mat}{\mathop{\mathrm{Mat}}\nolimits}
\newcommand{\rg}{\mathop{\text{rg}}\nolimits}
\newcommand{\tr}{\mathop{\text{tr}}\nolimits}
\newcommand{\ppcm}{\mathop{\text{ppcm}}\nolimits}

%----- Structure des exercices ------

\newtheoremstyle{styleexo}% name
{2ex}% Space above
{3ex}% Space below
{}% Body font
{}% Indent amount 1
{\bfseries} % Theorem head font
{}% Punctuation after theorem head
{\newline}% Space after theorem head 2
{}% Theorem head spec (can be left empty, meaning ‘normal’)

%\theoremstyle{styleexo}
\newtheorem{exo}{Exercice}
\newtheorem{ind}{Indications}
\newtheorem{cor}{Correction}


\newcommand{\exercice}[1]{} \newcommand{\finexercice}{}
%\newcommand{\exercice}[1]{{\tiny\texttt{#1}}\vspace{-2ex}} % pour afficher le numero absolu, l'auteur...
\newcommand{\enonce}{\begin{exo}} \newcommand{\finenonce}{\end{exo}}
\newcommand{\indication}{\begin{ind}} \newcommand{\finindication}{\end{ind}}
\newcommand{\correction}{\begin{cor}} \newcommand{\fincorrection}{\end{cor}}

\newcommand{\noindication}{\stepcounter{ind}}
\newcommand{\nocorrection}{\stepcounter{cor}}

\newcommand{\fiche}[1]{} \newcommand{\finfiche}{}
\newcommand{\titre}[1]{\centerline{\large \bf #1}}
\newcommand{\addcommand}[1]{}
\newcommand{\video}[1]{}

% Marge
\newcommand{\mymargin}[1]{\marginpar{{\small #1}}}



%----- Presentation ------
\setlength{\parindent}{0cm}

%\newcommand{\ExoSept}{\href{http://exo7.emath.fr}{\textbf{\textsf{Exo7}}}}

\definecolor{myred}{rgb}{0.93,0.26,0}
\definecolor{myorange}{rgb}{0.97,0.58,0}
\definecolor{myyellow}{rgb}{1,0.86,0}

\newcommand{\LogoExoSept}[1]{  % input : echelle
{\usefont{U}{cmss}{bx}{n}
\begin{tikzpicture}[scale=0.1*#1,transform shape]
  \fill[color=myorange] (0,0)--(4,0)--(4,-4)--(0,-4)--cycle;
  \fill[color=myred] (0,0)--(0,3)--(-3,3)--(-3,0)--cycle;
  \fill[color=myyellow] (4,0)--(7,4)--(3,7)--(0,3)--cycle;
  \node[scale=5] at (3.5,3.5) {Exo7};
\end{tikzpicture}}
}



\theoremstyle{definition}
%\newtheorem{proposition}{Proposition}
%\newtheorem{exemple}{Exemple}
%\newtheorem{theoreme}{Théorème}
\newtheorem{lemme}{Lemme}
\newtheorem{corollaire}{Corollaire}
%\newtheorem*{remarque*}{Remarque}
%\newtheorem*{miniexercice}{Mini-exercices}
%\newtheorem{definition}{Définition}




%definition d'un terme
\newcommand{\defi}[1]{{\color{myorange}\textbf{\emph{#1}}}}
\newcommand{\evidence}[1]{{\color{blue}\textbf{\emph{#1}}}}



 %----- Commandes divers ------

\newcommand{\codeinline}[1]{\texttt{#1}}

\renewcommand{\implies}{\; \Rightarrow\; } %\implies apparait mal (de meme que Longrightarrow)

%%%%%%%%%%%%%%%%%%%%%%%%%%%%%%%%%%%%%%%%%%%%%%%%%%%%%%%%%%%%%
%%%%%%%%%%%%%%%%%%%%%%%%%%%%%%%%%%%%%%%%%%%%%%%%%%%%%%%%%%%%%



\begin{document}

\debuttexte

%%%%%%%%%%%%%%%%%%%%%%%%%%%%%%%%%%%%%%%%%%%%%%%%%%%%%%%%%%%
\diapo

\change

Ce chapitre est consacré aux rudiments de la logique

\change

Nous allons définir ce que sont des <<assertions>>
et voir quelles sont les relations entre plusieurs assertions.

\change

Nous construirons ensuite des assertions plus sophistiquées à l'aide
des quantificateurs <<pour tout>> et <<il existe>>.


%%%%%%%%%%%%%%%%%%%%%%%%%%%%%%%%%%%%%%%%%%%%%%%%%%%%%%%%%%%
\diapo

La langue française n'est pas toujours suffisamment claire :
par exemple au restaurant dans la phrase <<fromage ou dessert>>,
il s'agit soit du fromage, soit du dessert mais pas des deux !

Par contre si dans un jeu de carte on cherche <<un as ou un coeur>>
alors il ne faut pas exclure l'as de coeur.

\change

Nous avons donc besoin d'un langage plus rigoureux.

\change

On dit souvent  qu'une fonction est continue si on peut tracer
son graphe sans lever le crayon.

Il est clair que ce n'est pas satisfaisant.

\change

Voici comment on définit qu'une fonction $f$ est continue en un point
$x_0$.

C'est beaucoup plus aride de prime abord...

\change 

... mais c'est le but de chapitre 
de rendre cette ligne plus claire
en introduisant la langage de la logique.


\change

Un autre objectif est de distinguer précisément ce qui est vrai de ce qui est faux.
\og Est-ce qu'une augmentation
de $20\%$, puis de $30\%$ est plus intéressante qu'une augmentation de $50\%$ ?\fg 

\change

Il ne s'agit pas seulement d'avoir une opinion
mais d'être capable d'effectuer une démarche logique,
ceci afin d'en être convaincu et de pouvoir convaincre d'autres personnes.

\change 

Cette démarche s'appelle le raisonnement et sera étudiée
dans la leçon suivante.

%%%%%%%%%%%%%%%%%%%%%%%%%%%%%%%%%%%%%%%%%%%%%%%%%%%%%%%%%%%
\diapo

Par définition une \defi{assertion} est une phrase soit vraie, 
soit fausse, pas les deux en même temps

\change

Voici quelques exemples :

\og Il pleut \fg

\og Je suis plus grand que toi \fg

qui peuvent être vraie ou fausse selon la situation.

\change

Mais aussi $2+2=4$ qui est une assertion vraie.

Alors que \og$2\times 3 = 7$\fg{} est une assertion fausse.


\change

On complique les choses avec 

Pour tout $x \in \Rr$, on a $x^2 \ge 0$

qui est une phrase vraie.

Alors que 

Pour tout $z\in \Cc$, on a $|z| = 1$

est une assertion fausse, car ce n'est pas vrai pour tous les complexes.



%%%%%%%%%%%%%%%%%%%%%%%%%%%%%%%%%%%%%%%%%%%%%%%%%%%%%%%%%%%
\diapo

A partir d'une assertion $P$ et d'une assertion $Q$
on définit maintenant l'assertion <<$P$ et $Q$>>.

L'assertion <<$P$ et $Q$>> est vraie si $P$ est vraie et $Q$ est vraie.

L'assertion <<$P$ et $Q$>> est fausse sinon.


\change

Une autre façon de définir l'assertion <<$P$ et $Q$>> est de dire
si cette assertion est vraie ou fausse en fonction de $P$ et de $Q$ :

c'est la table de vérité de <<$P$ et $Q$>> :

\change

- si $P$ est vraie et $Q$ est vraie alors <<$P$ et $Q$>> est vraie ;

\change

- si $P$ est vraie et $Q$ est fausse alors <<$P$ et $Q$>> est fausse ;

\change

De même 

- si $P$ est fausse et $Q$ est vraie alors <<$P$ et $Q$>> est fausse ;

\change

Enfin 

- si $P$ est fausse et $Q$ est fausse alors <<$P$ et $Q$>> est fausse.

\change

Par exemple si $P$ est l'assertion \og Cette carte est un as\fg

et $Q$ l'assertion \og Cette carte est c\oe ur\fg

Alors si la carte est l'as de c\oe ur \og $P$ et $Q$\fg\ est vraie 

Pour toutes les autres cartes l'assertion \og $P$ et $Q$\fg\ est fausse.


%%%%%%%%%%%%%%%%%%%%%%%%%%%%%%%%%%%%%%%%%%%%%%%%%%%%%%%%%%%
\diapo

On fait la même chose avec le <<ou>> logique.

L'assertion \og $P$ \defi{ou} $Q$\fg\ est vraie si l'une au moins des deux assertions $P$ ou $Q$ est vraie

L'assertion \og $P$ ou $Q$\fg\ est fausse si les deux assertions $P$ et $Q$ sont fausses


\change

La table de vérité de cette assertion est donc la suivante :

\change

- si $P$ est vraie et $Q$ est vraie alors <<$P$ ou $Q$>> est vraie ;

\change

- si $P$ est vraie et $Q$ est fausse alors <<$P$ ou $Q$>> est vraie ;

\change

De même 

- si $P$ est fausse et $Q$ est vraie alors <<$P$ ou $Q$>> est vraie ;

\change

Enfin 

- si $P$ est fausse et $Q$ est fausse alors <<$P$ ou $Q$>> est fausse.




\change

Pour notre exemple, l'assertion  \og $P$ ou $Q$\fg\ est vraie si la carte est un as ou bien un c\oe ur

pour toutes les autres cartes l'assertion \og $P$ ou $Q$\fg\ est fausse.


%%%%%%%%%%%%%%%%%%%%%%%%%%%%%%%%%%%%%%%%%%%%%%%%%%%%%%%%%%%
\diapo

La négation d'une assertion $P$ se définit ainsi :

\og \defi{non} $P$\fg\ est vraie si $P$ est fausse, 

\og \defi{non} $P$\fg\ fausse si $P$ est vraie

La table de vérité s'écrit tout simplement comme cela.


%%%%%%%%%%%%%%%%%%%%%%%%%%%%%%%%%%%%%%%%%%%%%%%%%%%%%%%%%%%
\diapo

Voici une notion plus difficile : l'implication.

Si $P$ et $Q$ sont deux assertions alors l'assertion

\og (non $P$) ou $Q$\fg\ s'appelle par définition \og $P \implies Q$\fg


Cette assertion <<$P \implies Q$>>  se lit aussi 
<<si $P$ est vraie alors $Q$ est vraie>>

\change

A partir de cette définition nous pouvons remplir la table de vérité
de \og $P \implies Q$\fg


\change

- si $P$ est vraie et $Q$ est vraie alors \og (non $P$) ou $Q$\fg \ est vraie donc <<$P \implies Q$>> est vraie ;

\change

Par contre si
  $P$ est vraie et $Q$ est fausse alors \og (non $P$) ou $Q$\fg \ est fausse 
donc <<$P \implies Q$>> est fausse.

\change

- si $P$ est fausse alors (non $P$) est vraie donc <<$P \implies Q$>> est encore vraie ;



\change


Par exemple 
 L'assertion  \og $0 \le x \le 25 \implies \sqrt x \le 5$\fg\  est vraie 
En effet si $x$ est compris entre $0$ et $25$ alors l'assertion $P$ est vraie
mais on sait aussi alors que $\sqrt x$ est plus petit que $5$ donc l'assertion $Q$ est vraie.
<<$P \implies Q$>>  est donc vraie dans ce cas

Si maintenant $x$ n'est pas compris entre $0$ et $25$ l'assertion $P$ est donc fausse
et alors l'assertion <<$P \implies Q$>> est encore vraie (sans même avoir besoin de 
discuter de l'assertion $Q$).

On résume ainsi : si $x$ est compris entre $0$ et $25$ alors $\sqrt x$ est plus petit que $5$.

\change

De même l'assertion \og $x \in ]-\infty, -4[ \implies x^2+3x-4 > 0$\fg est vraie 

car si on prend un $x$ dans cet intervalle alors par une étude de fonction
nous avons  bien $x^2+3x-4 > 0$.

\change

Par contre 
l'assertion \og $\sin(\theta)=0 \implies \theta = 0$\fg est fausse 

en effet prenons $\theta=\pi$ alors 
comme $\sin \pi = 0$ l'assertion $P$ est vraie.

Par contre l'assertion $Q$ est fausse car $\theta \neq 0$.

$P$ est vraie, $Q$ est fausse l'implication est donc fausse.

Les implications fausses ne nous intéressent pas.


En dehors de ce chapitre on n'écrira plus que des implications vraies.


\change 

Il y a une autre situation peu intéressante :

 si $P$ est fausse alors l'assertion \og $P \implies Q$\fg est toujours vraie

Cela découle de la définition comme nous l'avons déjà remarqué




%%%%%%%%%%%%%%%%%%%%%%%%%%%%%%%%%%%%%%%%%%%%%%%%%%%%%%%%%%%
\diapo

L'équivalence << $P$ équivaut à $Q$ >>
est l'assertion \og ($P \implies Q$) \  et \  ($Q \implies P$)\fg

On dira \og $P$ est équivalent à $Q$\fg  ou \og $P$ équivaut à $Q$\fg 
ou \og $P$ si et seulement si $Q$\fg.


\change

Cette assertion est vraie
lorsque $P$ et $Q$ sont vraies en même temps ou lorsque $P$ et $Q$ sont fausses en même temps.

\change

Par exemple

\og $x\cdot x'=0 \iff (x=0 \text{ ou } x'=0)$\fg est une assertion vraie.

On dira qu'un produit est nul si et seulement si l'un des facteurs est nul.



Dans toute la suite lorsque l'on écrira une équivalence
\og $P$ équivaut à $Q$\fg cela signifiera en fait :

l'assertion \og $P$ équivaut à $Q$\fg est vraie.


%%%%%%%%%%%%%%%%%%%%%%%%%%%%%%%%%%%%%%%%%%%%%%%%%%%%%%%%%%%
\diapo

Nous allons voir plusieurs assertions équivalentes.

Tout d'abord les assertions $P$ et $\text{ non}(\text{non}(P))$
sont équivalentes.

Autrement dit la négation d'une négation de $P$ est $P$ elle même !


\change

L'assertion $(P \text{ et } Q)$ est équivalente à l'assertion $(Q \text{ et } P)$

\change

De même pour l'assertion $(P \text{ ou } Q)$ avec $(Q \text{ ou } P)$


\change

La négation de l'assertion <<$P$ et $Q$>> est l'assertion 
$(\text{non } P)  \text{ ou } (\text{non }Q)$

\change

De façon symétrique la négation de <<$P$ ou $Q$>>
est l'assertion $(\text{non } P)  \text{ et } (\text{non }Q)$

\change

 $\big(P \text{ et } (Q \text{ ou } R)  \big)$

équivaut à  
$(P \text{ et } Q) \text{ ou } (P \text{ et }  R)$

\change

Et  $\big(P \text{ ou } (Q \text{ et } R)  \big)$ équivaut à
$(P \text{ ou } Q) \text{ et } (P \text{ ou }  R)$

\change

Et enfin retenez que
l'assertion
 \og $P \implies Q$\fg 
équivaut à l'assertion 
 \og $\text{non}(Q) \implies \text{non}(P)$\fg


On appelle cette dernière assertion la contraposée de \og $P \implies Q$\fg.

Une implication et sa contraposée sont des assertions équivalentes.


%%%%%%%%%%%%%%%%%%%%%%%%%%%%%%%%%%%%%%%%%%%%%%%%%%%%%%%%%%%
\diapo


Voyons comment se prouvent ces équivalences.

Par exemple remplissons la table de vérité de \og $\text{non}(P \text{ et } Q)$\fg 

\change

- si $P$ est vrai et $Q$ est vrai alors <<$P$ et $Q$>> est vrai
donc \og $\text{non}(P \text{ et } Q)$\fg  est faux.

\change


 - si $P$ est vrai et $Q$ est faux alors <<$P$ et $Q$>> est faux
donc \og $\text{non}(P \text{ et } Q)$\fg  est vrai.

\change

c'est la même chose pour les deux dernières cases.

\change



Maintenant dressons la table de vérité de l'assertion \og $(\text{non } P)  \text{ ou } (\text{non }Q)$\fg 

\change

- si $P$ est vrai et $Q$ est vrai alors $(\text{non } P)$ est faux, $(\text{non } Q)$ est faux
donc \og $(\text{non } P)  \text{ ou } (\text{non }Q)$\fg  est faux.

\change

 - si $P$ est vrai et $Q$ est faux alors $(\text{non } P)$ est faux, mais $(\text{non } Q)$ est vrai
donc  \og $(\text{non } P)  \text{ ou } (\text{non }Q)$\fg  est vrai

\change

on fait pareil pour les autres cases.



Les tables de vérités des deux assertions sont identiques.

Les assertions sont donc vraies en même temps et fausses
en mêmes temps, elles sont donc équivalentes.


%%%%%%%%%%%%%%%%%%%%%%%%%%%%%%%%%%%%%%%%%%%%%%%%%%%%%%%%%%%
\diapo

Une assertion peut dépendre d'une variable $x$


Par exemple si on note $P(x)$ l'assertion \og $x^2 \ge 1$\fg

Alors selon la valeur de $x$, $P(x)$ est soit vraie soit fausse.

\change

L'assertion 
$$\forall x \in E \quad P(x)$$
est vraie lorsque les assertions $P(x)$ sont vraies pour tous les $x$
appartenant à l'ensemble $E$

\change

Voici quelques exemples 
$\forall x \in [1,+\infty[ \quad (x^2\ge 1)$\fg\ est une assertion vraie.

En effet pour tous les $x\ge1$ on bien $x^2 \ge 1$.


\change

Par contre l'assertion
 \og $\forall x \in \Rr \quad (x^2\ge 1)$\fg\
n'est pas vraie. 

En effet $x^2\ge 1$ n'est pas vraie pour tous les réels $x$.

Par exemple pour $x=0$, $0$ au carré n'est pas plus grand que $1$.



\change

Dernier exemple :
il est vrai que pour tout entier 
$n$ alors $n(n+1)$ est divisible par $2$.


%%%%%%%%%%%%%%%%%%%%%%%%%%%%%%%%%%%%%%%%%%%%%%%%%%%%%%%%%%%
\diapo

La phrase logique 
$$\exists x \in E \quad P(x)$$
est une assertion vraie si l'on peut trouver un $x$ de l'ensemble $E$ pour lequel $P(x)$ est vraie


\change

Voici des exemples 

\begin{itemize}
  \item \og $\exists x \in \Rr \quad (x(x-1)<0)$\fg\ est vraie
par exemple $x=\frac 12$ vérifie bien la propriété.


  \item \og $\exists n \in \Nn \quad n^2-n > n$\fg\ est vraie

Il y a plein de choix, par exemple $n=3$ convient, mais aussi $n=10$ ou même
$n=100$, en trouver un seul suffit pour dire que l'assertion est vraie.

  \item Par contre \og $\exists x \in \Rr \quad (x^2=-1)$\fg\ est fausse

En effet aucun réel au carré ne donnera un nombre négatif.
\end{itemize} 



%%%%%%%%%%%%%%%%%%%%%%%%%%%%%%%%%%%%%%%%%%%%%%%%%%%%%%%%%%%
\diapo

Nous allons nier les phrases contenant des quantificateur <<pour tout>>
ou <<il existe>>.

La négation de la phrase logique \og $\forall x \in E \quad P(x)$\fg  

est \ \og $\exists x \in E \quad \text{non } P(x)$\fg

\change

Par exemple : la négation de \og $\forall x \in [1,+\infty[ \quad (x^2\ge 1)$\fg

est  \og $\exists x \in [1,+\infty[ \quad (x^2 < 1)$\fg

\change


La négation de \og $\exists x \in E \quad P(x)$\fg \ est 

  \og $\forall x \in E \quad \text{non } P(x)$\fg


\change

Voici des exemples :


La négation de 
$\exists z \in \Cc \quad (z^2+z+1 = 0)$

est
$\forall z \in \Cc \quad (z^2+z+1 \neq 0)$

\change

La négation de 
$\forall x \in \Rr \quad  (x+1 \in \Zz)$


est
$\exists x \in \Rr \quad (x+1 \notin \Zz)$


Remarquez que l'on n'a pas besoin de savoir si une phrase logique est vraie ou fausse
pour écrire sa négation.

\change

Enfin la négation de 
$\forall x \in \Rr \quad \exists y >0 \quad (x+y > 10)$ 

est
$\exists x \in \Rr \quad \forall y > 0 \quad (x+y \le 10)$

En effet la négation de $x+y$ strictement plus grand que $10$
c'est $x+y$ n'est pas strictement plus grand que $10$.
Autrement dit $x+y$ est inférieur ou égal à $10$.

%%%%%%%%%%%%%%%%%%%%%%%%%%%%%%%%%%%%%%%%%%%%%%%%%%%%%%%%%%%
\diapo

Terminons avec quelques remarques.

Tout d'abord une remarque essentielle :

les phrases logique se lisent de gauche à droite

et donc l'ordre des quantificateurs est très important.



Par exemple la phrase

$\forall x \in \Rr \quad  \exists y \in \Rr \quad (x+y>0)$ \quad est vraie

alors que si l'on change l'ordre de $\forall$ et $\exists$ 
la phrase 

$\exists y \in \Rr \quad \forall x \in \Rr \quad (x+y>0)$ devient fausse.

Réfléchissez bien à la différence entre ces deux phrases.

Dans la première phrase, le $y$ qui existe dépend du $x$ 

Alors que dans la seconde phrase le $y$ devrait être le même pour tous les $x$,
ce qui n'est pas possible.

\change


Quand on écrit \og $\exists x \in \Rr \quad (f(x)=0)$\fg\  cela 
signifie juste qu'il existe un réel pour lequel $f$ s'annule. Il peut y avoir 
plusieurs $x$ qui conviennent.

Si vous voulez préciser que $f$ s'annule en une unique valeur, on rajoute un point d'exclamation 

$\exists \mathbf{!}\, x \in \Rr \quad (f(x)=0)$


\change

Il faut être précis : la négation de l'inégalité <<strictement inférieur>>
est <<supérieur ou égal>>.

Inversement la négation d'une inégalité large est une inégalité stricte.


\change

Rédiger vos assertions soit par une phrase en français, soit entièrement
avec des symboles.

Par exemple la phrase 

\og Pour tout réel $x$, si $f(x)=1$ alors $x\ge0$.\fg

est correcte.

Elle s'écrit aussi convenablement à l'aide de quantificateurs  :\\
$\forall x \in \Rr \quad (f(x)=1 \implies x \ge 0).$

Mais surtout ne mélanger pas les deux !

Les quantificateurs ne sont pas des abréviations.


\change

Dans le même genre d'idées n'écrivez pas $\not\!\exists$, $\not\!\!\!\implies$
qui sont des symboles qui n'existent pas !



%%%%%%%%%%%%%%%%%%%%%%%%%%%%%%%%%%%%%%%%%%%%%%%%%%%%%%%%%%%
\diapo


Pour conclure entraînez-vous avec cette série d'exercices.



\end{document}