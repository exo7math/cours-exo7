\documentclass[11pt,class=report,crop=false]{standalone}
\usepackage[screen]{../exo7book}

\begin{document}

%====================================================================
\chapitre[Logique et raisonnements]{Logique et\\ raisonnements}
%====================================================================

\insertvideo{aWSe1fjJHEM}{partie 1. Logique}

\insertvideo{B-I5yZd0Wbk}{partie 2. Raisonnements}

\insertfiche{fic00002.pdf}{Logique, ensembles, raisonnements}

%%%%%%%%%%%%%%%%%%%%%%%%%%%%%%%%%%%%%%%%%%%%%%%%%%%%%%%%%%%%%%%%
\section*{Quelques motivations}

\begin{itemize}
  \item Il est important d'avoir un \evidence{langage rigoureux}. La langue française
est souvent ambigüe. Prenons l'exemple de la conjonction \assertion{ou} ; au restaurant
\assertion{fromage ou dessert} signifie l'un ou l'autre mais pas les deux. Par contre si dans un jeu de
carte on cherche \assertion{les as ou les c\oe urs} alors il ne faut pas exclure l'as de c\oe ur.
Autre exemple : que répondre à la question \assertion{As-tu $10$ euros en poche ?}  si l'on dispose de $15$ euros ?

  \item Il y a des notions difficiles à expliquer avec des mots : par exemple la continuité d'une fonction est
souvent expliquée par \assertion{on trace le graphe sans lever le crayon}. Il est clair que c'est une définition peu satisfaisante.
Voici la définition mathématique de la continuité d'une fonction $f : I \to \Rr$ en un point $x_0\in I$ :
$$\forall \epsilon > 0 \quad \exists \delta >0 \quad \forall x \in I \quad
  (|x-x_0|<\delta \implies |f(x)-f(x_0)|< \epsilon).$$
C'est le but de ce chapitre de rendre cette ligne plus claire ! C'est la \evidence{logique}.

  \item Enfin les mathématiques tentent de \evidence{distinguer le vrai du faux}.
  Par exemple \assertion{Est-ce qu'une augmentation
de $20\%$, puis de $30\%$ est plus intéressante qu'une augmentation de $50\%$ ?}.
Vous pouvez penser \assertion{oui}  ou \assertion{non},
mais pour en être sûr il faut suivre une démarche logique qui mène à la conclusion. Cette démarche doit être convaincante
pour vous mais aussi pour les autres. On parle de \evidence{raisonnement}.
\end{itemize}

\bigskip

Les mathématiques sont un langage pour s'exprimer rigoureusement, adapté aux phénomènes complexes,
qui rend les calculs exacts et vérifiables. Le raisonnement est le moyen de valider --- ou d'infirmer ---
une hypothèse et de l'expliquer à autrui.

\bigskip

%%%%%%%%%%%%%%%%%%%%%%%%%%%%%%%%%%%%%%%%%%%%%%%%%%%%%%%%%%%%%%%%
\section{Logique}
\index{logique}

%---------------------------------------------------------------
\subsection{Assertions}

Une \defi{assertion}\index{assertion} est une phrase soit vraie, soit fausse, pas les deux en même temps.

Exemples :
\begin{itemize}
  \item \assertion{Il pleut.}
  \item \assertion{Je suis plus grand que toi.}
  \item \assertion{$2+2=4$}
  \item \assertion{$2\times 3 = 7$}
  \item \assertion{Pour tout $x \in \Rr$, on a $x^2 \ge 0$.}
  \item \assertion{Pour tout $z\in \Cc$, on a $|z| = 1$.}
\end{itemize}


\bigskip

Si $P$ est une assertion et $Q$ est une autre assertion, nous allons définir de nouvelles assertions construites à partir de $P$ et de $Q$.

%--------------
\subsubsection*{L'opérateur logique \assertion{et}}


L'assertion \assertion{$P$ \defi{et} $Q$}\index{et logique@\og et \fg{} logique} est vraie si $P$ est vraie et $Q$ est vraie.
L'assertion \assertion{$P$ et $Q$} est fausse sinon.

On résume ceci en une \defi{table de vérité}\index{table de verite@table de vérité} :
\begin{figure}[H]
\centering
\begin{tabular}{c|c|c}
\textcolor{olive}{$P$} $\backslash$ \textcolor{blue}{$Q$} & \textcolor{blue}{V} & \textcolor{blue}{F} \\ \hline
\textcolor{olive}{V} & \textcolor{red}{V} & \textcolor{red}{F} \\ \hline
\textcolor{olive}{F} & \textcolor{red}{F} & \textcolor{red}{F} \\
\end{tabular}
\caption{Table de vérité de \assertion{$P$ et $Q$}}
\end{figure}


Par exemple si $P$ est l'assertion \assertion{Cette carte est un as} et $Q$ l'assertion
\assertion{Cette carte est c\oe ur} alors l'assertion \assertion{$P$ et $Q$} est vraie si la carte
est l'as de c\oe ur et est fausse pour toute autre carte.



%--------------
\subsubsection*{L'opérateur logique \assertion{ou}}

L'assertion \assertion{$P$ \defi{ou} $Q$}\index{ou logique@\og ou \fg{} logique} est vraie si l'une (au moins) des deux assertions $P$ ou $Q$ est vraie.
L'assertion \assertion{$P$ ou $Q$} est fausse si les deux assertions $P$ et $Q$ sont fausses.

On reprend ceci dans la table de vérité :
\begin{figure}[H]
\centering
\begin{tabular}{c|c|c}
\textcolor{olive}{$P$} $\backslash$ \textcolor{blue}{$Q$}  & \textcolor{blue}{V} & \textcolor{blue}{F} \\ \hline
\textcolor{olive}{V} & \textcolor{red}{V} & \textcolor{red}{V} \\ \hline
\textcolor{olive}{F} & \textcolor{red}{V} & \textcolor{red}{F} \\
\end{tabular}
\caption{Table de vérité de \assertion{$P$ ou $Q$}}
\end{figure}


Si $P$ est l'assertion \assertion{Cette carte est un as} et $Q$ l'assertion
\assertion{Cette carte est c\oe ur} alors l'assertion \assertion{$P$ ou $Q$} est vraie si la carte
est un as ou bien un c\oe ur (en particulier elle est vraie pour l'as de c\oe ur).


\begin{remarque*}
Pour définir les opérateurs  \assertion{ou}, \assertion{et} on fait appel à une phrase en français utilisant
les mots \emph{ou}, \emph{et} ! Les tables de vérités permettent d'éviter ce problème.
\end{remarque*}



%--------------
\subsubsection*{La négation \assertion{non}}


L'assertion \assertion{\defi{non} $P$}\index{négation} est vraie si $P$ est fausse, et fausse si $P$ est vraie.

\begin{figure}[H]
\centering
\begin{tabular}{c|c|c}
 \textcolor{blue}{$P$}  &  \textcolor{blue}{V} &  \textcolor{blue}{F} \\ \hline
 \textcolor{red}{non $P$}    & \textcolor{red}{F} & \textcolor{red}{V} \\
\end{tabular}
\caption{Table de vérité de \assertion{non $P$}}
\end{figure}


%--------------
\subsubsection*{L'implication $\implies$}


La définition mathématique est la suivante :
\mybox{
L'assertion \assertion{(non $P$) ou $Q$} est notée \assertion{$P \implies Q$}\index{implication}\index{$\implies$}.
}


Sa table de vérité est donc la suivante:
\begin{figure}[H]
\centering
\begin{tabular}{c|c|c}
\textcolor{olive}{$P$} $\backslash$ \textcolor{blue}{$Q$}  & \textcolor{blue}{V} & \textcolor{blue}{F} \\ \hline
\textcolor{olive}{V} & \textcolor{red}{V} & \textcolor{red}{F} \\ \hline
\textcolor{olive}{F} & \textcolor{red}{V} & \textcolor{red}{V} \\
\end{tabular}
\caption{Table de vérité de \assertion{$P \implies Q$}}
\end{figure}
L'assertion \assertion{$P \implies Q$} se lit en français \assertion{$P$ implique $Q$}.

Elle se lit souvent aussi \assertion{si $P$ est vraie alors $Q$ est vraie}
ou \assertion{si $P$ alors $Q$}.

Par exemple :
\begin{itemize}
  \item \assertion{$0 \le x \le 25 \implies \sqrt x \le 5$}  est vraie (prendre la racine carrée).
  \item \assertion{$x \in ]-\infty, -4[ \implies x^2+3x-4 > 0$}  est vraie (étudier le binôme).
  \item \assertion{$\sin(\theta)=0 \implies \theta = 0$}  est fausse (regarder pour $\theta = 2\pi$ par exemple).
  \item \assertion{$2+2=5 \implies \sqrt 2 = 2$}  est vraie ! Eh oui, si
$P$ est fausse alors l'assertion \assertion{$P \implies Q$} est toujours vraie.
\end{itemize}



%--------------
\subsubsection*{L'équivalence $\iff$}


L'\defi{équivalence} est définie par :
\mybox{
\assertion{$P \iff Q$}\index{equivalence@équivalence}\index{$\iff$}  est l'assertion \assertion{($P \implies Q$) \  et \  ($Q \implies P$)}.
}

On dira \assertion{$P$ est équivalent à $Q$} ou \assertion{$P$ équivaut à $Q$}
ou \assertion{$P$ si et seulement si $Q$}. Cette assertion est vraie
lorsque $P$ et $Q$ sont vraies ou lorsque $P$ et $Q$ sont fausses.
La table de vérité est :
\begin{figure}[H]
\centering
\begin{tabular}{c|c|c}
\textcolor{olive}{$P$} $\backslash$ \textcolor{blue}{$Q$}  & \textcolor{blue}{V} & \textcolor{blue}{F} \\ \hline
\textcolor{olive}{V} & \textcolor{red}{V} & \textcolor{red}{F} \\ \hline
\textcolor{olive}{F} & \textcolor{red}{F} & \textcolor{red}{V} \\
\end{tabular}
\caption{Table de vérité de \assertion{$P \iff Q$}}
\end{figure}

Exemples :
\begin{itemize}
  \item Pour $x,x' \in \Rr$, l'équivalence \assertion{$x\cdot x'=0 \iff (x=0 \text{ ou } x'=0)$} est vraie.
  \item Voici une équivalence \emph{toujours fausse} (quelle que soit l'assertion $P$) :
\assertion{$P \iff \text{non}(P)$}.
\end{itemize}

On s'intéresse davantage aux assertions vraies qu'aux fausses, aussi dans la pratique
et en dehors de ce chapitre on écrira \assertion{$P \iff Q$}  ou \assertion{$P \implies Q$}
uniquement lorsque ce sont des assertions vraies. Par exemple si l'on écrit
\assertion{$P \iff Q$}  cela sous-entend \assertion{$P \iff Q$  est vraie}.
Attention rien ne dit que $P$ et $Q$ soient vraies. Cela signifie que $P$ et $Q$ sont vraies en même
temps ou fausses en même temps.

\begin{proposition}
\label{prop:log}
Soient $P, Q, R$ trois assertions.
Nous avons les équivalences (vraies) suivantes :
\begin{enumerate}
  \item \emph{$P \iff \text{ non}(\text{non}(P))$}
  \item \emph{$(P \text{ et } Q) \iff (Q \text{ et } P)$}
  \item \emph{$(P \text{ ou } Q) \iff (Q \text{ ou } P)$}
  \item \emph{$\text{non}(P \text{ et } Q)  \iff  (\text{non } P)  \text{ ou } (\text{non }Q)$}
  \item \emph{$\text{non}(P \text{ ou } Q)  \iff  (\text{non } P)  \text{ et } (\text{non }Q)$}
  \item \emph{$\big(P \text{ et } (Q \text{ ou } R)  \big)   \iff
(P \text{ et } Q) \text{ ou } (P \text{ et }  R)$}
  \item \emph{$\big(P \text{ ou } (Q \text{ et } R)  \big)   \iff
(P \text{ ou } Q) \text{ et } (P \text{ ou }  R)$}
  \item  \assertion{$P \implies Q$}  $\iff$ \assertion{$\text{non}(Q) \implies \text{non}(P)$}
\end{enumerate}
\end{proposition}

\begin{proof}
Voici des exemples de démonstrations :


\begin{itemize}
  \item[4.]
Il suffit de comparer les deux assertions  \assertion{$\text{non}(P \text{ et } Q)$}  et
\assertion{$(\text{non } P)  \text{ ou } (\text{non }Q)$}  pour toutes les valeurs possibles de $P$ et $Q$.
Par exemple si $P$ est vrai et $Q$ est vrai alors \assertion{$P$ et $Q$}  est vrai donc
 \assertion{$\text{non}(P \text{ et } Q)$}  est faux ;
d'autre part (non $P$) est faux,  (non $Q$) est faux donc
 \assertion{$(\text{non } P)  \text{ ou } (\text{non }Q)$}  est faux.
Ainsi dans ce premier cas les assertions sont toutes les deux fausses.
On dresse ainsi les deux tables de vérités et comme elles sont égales les deux assertions sont équivalentes.
\begin{figure}[H]
\centering
\begin{tabular}{c|c|c}
\textcolor{olive}{$P$} $\backslash$ \textcolor{blue}{$Q$}  & \textcolor{blue}{V} & \textcolor{blue}{F} \\ \hline
\textcolor{olive}{V} & \textcolor{red}{F} & \textcolor{red}{V} \\ \hline
\textcolor{olive}{F} & \textcolor{red}{V} & \textcolor{red}{V} \\
\end{tabular}
\caption{Tables de vérité de \assertion{$\text{non}(P \text{ et } Q)$}  et de \assertion{$(\text{non } P)  \text{ ou } (\text{non }Q)$}}
\end{figure}

  \item[6.] On fait la même chose mais il y a trois variables : $P$, $Q$, $R$.
On compare donc les tables de vérité d'abord dans le cas où $P$ est vrai (à gauche), puis dans le cas où
$P$ est faux (à droite). Dans les deux cas les deux assertions \assertion{$\big(P \text{ et } (Q \text{ ou } R)  \big)$}   et
\assertion{$(P \text{ et } Q) \text{ ou } (P \text{ et }  R)$}  ont la même table de vérité donc les assertions sont équivalentes.
\begin{figure}[H]
\centering
\begin{tabular}{c|c|c}
\textcolor{olive}{$Q$} $\backslash$ \textcolor{blue}{$R$}  & \textcolor{blue}{V} & \textcolor{blue}{F} \\ \hline
\textcolor{olive}{V} & \textcolor{red}{V} & \textcolor{red}{V} \\ \hline
\textcolor{olive}{F} & \textcolor{red}{V} & \textcolor{red}{F} \\
\end{tabular}\qquad
\begin{tabular}{c|c|c}
\textcolor{olive}{$Q$} $\backslash$ \textcolor{blue}{$R$}  & \textcolor{blue}{V} & \textcolor{blue}{F} \\ \hline
\textcolor{olive}{V} & \textcolor{red}{F} & \textcolor{red}{F} \\ \hline
\textcolor{olive}{F} & \textcolor{red}{F} & \textcolor{red}{F} \\
\end{tabular}
\end{figure}

  \item[8.] Par définition, l'implication \assertion{$P \implies Q$}  est l'assertion
\assertion{(non $P$) ou $Q$}.
Donc l'implication
\assertion{$\text{non}(Q) \implies \text{non}(P)$}  est équivalente à
\assertion{$\text{non}(\text{non}(Q)) \text{ ou } \text{non}(P)$}
qui équivaut encore à
\assertion{$Q \text{ ou } \text{non}(P)$}  et donc est équivalente à \assertion{$P \implies Q$}.
On aurait aussi pu encore une fois dresser les deux tables de vérité et voir qu'elles sont égales.
\end{itemize}
\end{proof}


%---------------------------------------------------------------
\subsection{Quantificateurs}
\index{quantificateur}

%--------------
\subsubsection*{Le quantificateur $\forall$ : \assertion{pour tout}}
\index{$\forall$}

Une assertion $P$ peut dépendre d'un paramètre $x$, par exemple
 \assertion{$x^2 \ge 1$}, l'assertion $P(x)$ est vraie ou fausse selon la valeur de $x$.

L'assertion
$$\forall x \in E \quad P(x)$$
est une assertion vraie lorsque les assertions $P(x)$ sont vraies pour tous les éléments $x$
de l'ensemble $E$.

On lit \assertion{Pour tout $x$ appartenant à $E$, $P(x)$}, sous-entendu
\assertion{Pour tout $x$ appartenant à $E$, $P(x)$ est vraie}.

Par exemple :
\begin{itemize}
  \item \assertion{$\forall x \in [1,+\infty[ \quad (x^2\ge 1)$}  est une assertion vraie.
  \item \assertion{$\forall x \in \Rr \quad (x^2\ge 1)$}  est une assertion fausse.
  \item \assertion{$\forall n \in \Nn \quad n(n+1) \text{ est divisible par } 2$}  est vraie.
\end{itemize}


%--------------
\subsubsection*{Le quantificateur $\exists$ : \assertion{il existe}}
\index{$\exists$}

L'assertion
$$\exists x \in E \quad P(x)$$
est une assertion vraie lorsque l'on peut trouver au moins un $x$ de $E$ pour lequel $P(x)$ est vraie.
On lit \assertion{il existe $x$ appartenant à $E$ tel que $P(x)$ (soit vraie)}.

Par exemple :
\begin{itemize}
  \item \assertion{$\exists x \in \Rr \quad (x(x-1)<0)$}  est vraie (par exemple $x=\frac 12$ vérifie bien la propriété).
  \item \assertion{$\exists n \in \Nn \quad n^2-n > n$}  est vraie (il y a plein de choix, par exemple $n=3$ convient, mais aussi $n=10$ ou même
$n=100$, un seul suffit pour dire que l'assertion est vraie).
  \item \assertion{$\exists x \in \Rr \quad (x^2=-1)$}  est fausse (aucun réel au carré ne donnera un nombre négatif).
\end{itemize}


%--------------
\subsubsection*{La négation des quantificateurs}

\mybox{
La négation de \assertion{$\forall x \in E \quad P(x)$} \ \ est \ \
\assertion{$\exists x \in E \quad \text{non } P(x)$} .
}

Par exemple la négation de \assertion{$\forall x \in [1,+\infty[ \quad (x^2\ge 1)$}
est l'assertion \assertion{$\exists x \in [1,+\infty[ \quad (x^2 < 1)$}.
En effet la négation de $x^2\ge 1$ est $\text{non}(x^2 \ge 1)$ mais s'écrit plus simplement $x^2 < 1$.


\mybox{
La négation de \assertion{$\exists x \in E \quad P(x)$} \ \ est \ \ \assertion{$\forall x \in E \quad \text{non } P(x)$}.
}

Voici des exemples :
\begin{itemize}
  \item La négation de \assertion{$\exists z \in \Cc \quad (z^2+z+1 = 0)$}  est
\assertion{$\forall z \in \Cc \quad (z^2+z+1 \neq 0)$}.

  \item La négation de \assertion{$\forall x \in \Rr \quad  (x+1 \in \Zz)$}  est
\assertion{$\exists x \in \Rr \quad (x+1 \notin \Zz)$}.

  \item Ce n'est pas plus difficile d'écrire la négation de phrases complexes.
Pour l'assertion :
$$\forall x \in \Rr \quad \exists y >0 \quad (x+y > 10)$$
sa négation est
$$\exists x \in \Rr \quad \forall y > 0 \quad (x+y \le 10).$$
\end{itemize}


%--------------
\subsubsection*{Remarques}

L'ordre des quantificateurs est très important.
Par exemple les deux phrases logiques
 $$\forall x \in \Rr \quad  \exists y \in \Rr \quad (x+y>0) \qquad \text{et}\qquad
\exists y \in \Rr \quad \forall x \in \Rr \quad (x+y>0).$$
sont différentes. La première est vraie, la seconde est fausse.
En effet une phrase logique se lit de gauche à droite, ainsi la première phrase affirme
\assertion{Pour tout réel $x$, il existe un réel $y$ (qui peut donc dépendre de $x$) tel que $x+y>0$.}
(par exemple on peut prendre $y=|x|+1$).
C'est donc une phrase vraie. Par contre la deuxième se lit :
\assertion{Il existe un réel $y$, tel que pour tout réel $x$, $x+y>0$.} Cette phrase est fausse,
cela ne peut pas être le même $y$ qui convient pour tous les $x$ !


On retrouve la même différence dans les phrases en français suivantes.
Voici une phrase vraie \assertion{Pour toute personne, il existe un numéro de téléphone},
bien sûr le numéro dépend de la personne. Par contre cette phrase est fausse :
\assertion{Il existe un numéro, pour toutes les personnes}. Ce serait le même numéro pour tout le monde !

\bigskip

Terminons avec d'autres remarques.
\begin{itemize}

  \item Quand on écrit \assertion{$\exists x \in \Rr \quad (f(x)=0)$} cela signifie juste qu'il existe un réel
pour lequel $f$ s'annule. Rien ne dit que ce $x$ est unique. Dans un premier temps vous pouvez
lire la phrase ainsi : \assertion{il existe \emph{au moins} un réel $x$ tel que $f(x)=0$}.
Afin de préciser que $f$ s'annule en une unique valeur, on rajoute un point d'exclamation\index{$\exists"!"$} :
$$\exists \mathbf{!}\, x \in \Rr \quad (f(x)=0).$$

  \item Pour la négation d'une phrase logique, il n'est pas nécessaire de savoir si la phrase est fausse
ou vraie. Le procédé est algorithmique : on change le \assertion{pour tout}  en
\assertion{il existe}  et inversement, puis on prend la négation de l'assertion $P$.

  \item Pour la négation d'une proposition, il faut être précis : la négation de l'inégalité stricte \assertion{$<$}
est l'inégalité large \assertion{$\ge$}, et inversement.

  \item Les quantificateurs ne sont pas des abréviations. Soit vous écrivez une phrase en français :
\assertion{Pour tout réel $x$, si $f(x)=1$ alors $x\ge0$.} , soit vous écrivez la phrase logique :
$$\forall x \in \Rr \quad (f(x)=1 \implies x \ge 0).$$
Mais surtout n'écrivez pas \assertion{$\forall x$ réel, si $f(x)=1 \implies x$ positif ou nul}. \
Enfin, pour passer d'une ligne à l'autre d'un raisonnement, préférez plutôt \assertion{donc} à \assertion{$\implies$}.

  \item Il est défendu d'écrire $\not\!\exists$, $\not\!\!\!\implies$. Ces symboles n'existent pas !
\end{itemize}



%---------------------------------------------------------------
%\subsection{Mini-exercices}

\begin{miniexercices}
\sauteligne
\begin{enumerate}
  \item \'Ecrire la table de vérité du \assertion{ou exclusif}. (C'est le \emph{ou} dans la phrase
\assertion{fromage ou dessert}, l'un ou l'autre mais pas les deux.)
  \item \'Ecrire la table de vérité de \assertion{non ($P$ et $Q$)}. Que remarquez vous ?
  \item \'Ecrire la négation de \assertion{$P \implies Q$}.
  \item Démontrer les assertions restantes de la proposition \ref{prop:log}.
  \item \'Ecrire la négation de \assertion{$\big(P \text{ et } (Q \text{ ou } R)  \big)$}.
  \item \'Ecrire à l'aide des quantificateurs la phrase suivante :
\assertion{Pour tout nombre réel, son carré est positif}.  Puis écrire la négation.
  \item Mêmes questions avec les phrases : \assertion{Pour chaque réel, je peux trouver un entier
relatif tel que leur produit soit strictement plus grand que $1$}. Puis
\assertion{Pour tout entier $n$, il existe un unique réel $x$ tel que $\exp(x)$ égale $n$}.
\end{enumerate}
\end{miniexercices}




%%%%%%%%%%%%%%%%%%%%%%%%%%%%%%%%%%%%%%%%%%%%%%%%%%%%%%%%%%%%%%%%
\section{Raisonnements}
\index{raisonnement}

Voici des méthodes classiques de raisonnements.

%---------------------------------------------------------------
\subsection{Raisonnement direct}
On veut montrer que l'assertion \assertion{$P \implies Q$} est vraie.
On suppose que $P$ est vraie et on montre qu'alors $Q$ est vraie.
C'est la méthode à laquelle vous êtes le plus habitué.

\begin{exemple}
Montrer que si $a,b \in \Qq$ alors $a+b\in \Qq$.

\begin{proof}
Prenons $a\in \Qq$, $b\in \Qq$.
Rappelons que les rationnels $\Qq$ sont
l’ensemble des réels s'écrivant $\frac{p}{q}$ avec $p\in\Zz$ et $q\in \Nn^*$.

Alors $a = \frac{p}{q}$
pour un certain $p \in \Zz$ et un certain $q\in \Nn^*$. De même $b = \frac{p'}{q'}$
avec $p' \in \Zz$ et $q'\in \Nn^*$.
Maintenant
$$a+b= \frac{p}{q} + \frac{p'}{q'} = \frac{pq'+qp'}{qq'}.$$
Or le numérateur $pq'+qp'$ est bien un élément de $\Zz$ ;
le dénominateur $qq'$ est lui un élément de $\Nn^*$. Donc
$a+b$ s'écrit bien de la forme $a+b=\frac{p''}{q''}$ avec $p''\in \Zz$,
$q''\in \Nn^*$. Ainsi $a+b \in \Qq$.
\end{proof}
\end{exemple}

%---------------------------------------------------------------
\subsection{Cas par cas}

Si l'on souhaite vérifier une assertion $P(x)$ pour tous les $x$ dans un ensemble $E$, on
montre l'assertion pour les $x$ dans une partie $A$ de $E$, puis pour les $x$
n'appartenant pas à $A$. C'est la méthode de \defi{disjonction}\index{disjonction} ou du \defi{cas par cas}.

\begin{exemple}
Montrer que pour tout $x\in \Rr$, $|x-1| \le x^2-x+1$.

\begin{proof}
Soit $x \in \Rr$. Nous distinguons deux cas.

\textbf{Premier cas : $x \ge 1$.}
Alors $|x-1|=x-1$. Calculons alors $x^2-x+1 - |x-1|$.
\begin{align*}
 x^2-x+1 - |x-1|
   &= x^2-x+1 - (x-1) \\
   &= x^2 -2x + 2 \\
   &= (x-1)^2 + 1 \ge 0. \\
\end{align*}
Ainsi $x^2-x+1 - |x-1| \ge 0$ et donc $x^2-x+1 \ge |x-1|$.

\textbf{Deuxième cas : $x < 1$.}
Alors $|x-1| = -(x-1)$.
Nous obtenons $x^2-x+1 - |x-1|
= x^2-x+1 + (x-1) = x^2 \ge 0$. Et donc $x^2-x+1 \ge |x-1|$.


\textbf{Conclusion.} Dans tous les cas $|x-1| \le x^2-x+1$.
\end{proof}
\end{exemple}

%---------------------------------------------------------------
\subsection{Contraposée}

Le raisonnement par \defi{contraposition}\index{contraposition} est basé sur l'équivalence suivante (voir la proposition \ref{prop:log}) :
\mybox{
L'assertion \assertion{$P \implies Q$}  \  est équivalente à \  \assertion{$\text{non}(Q) \implies \text{non}(P)$}.
}

Donc si l'on souhaite montrer l'assertion \assertion{$P \implies Q$}, on montre en fait
que si $\text{non}(Q)$ est vraie alors $\text{non}(P)$ est vraie.

\begin{exemple}
Soit $n \in \Nn$. Montrer que si $n^2$ est pair alors $n$ est pair.

\begin{proof}
Nous supposons que $n$ n'est pas pair. Nous voulons montrer qu'alors $n^2$ n'est pas pair.
Comme $n$ n'est pas pair, il est impair et donc il existe $k\in \Nn$ tel que
$n=2k+1$. Alors
$n^2 = (2k+1)^2 = 4k^2 + 4k + 1 = 2\ell+1$ avec $\ell = 2k^2+2k \in \Nn$.
Et donc $n^2$ est impair.

Conclusion : nous avons montré que si $n$ est impair alors $n^2$ est impair. Par contraposition
ceci est équivalent à : si $n^2$ est pair alors $n$ est pair.
\end{proof}
\end{exemple}



%---------------------------------------------------------------
\subsection{Absurde}

Le \defi{raisonnement par l'absurde}\index{absurde} pour montrer \assertion{$P \implies Q$}  repose sur le principe suivant :
on suppose à la fois que $P$ est vraie et que $Q$ est fausse et on cherche une contradiction.
Ainsi si $P$ est vraie alors $Q$ doit être vraie et donc \assertion{$P \implies Q$} est vraie.

\begin{exemple}
Soient $a,b \ge 0$. Montrer que si $\frac{a}{1+b}=\frac{b}{1+a}$
alors $a=b$.

\begin{proof}
Nous raisonnons par l'absurde en supposant que  $\frac{a}{1+b}=\frac{b}{1+a}$ \textbf{et}
$a \neq b$.
Comme $\frac{a}{1+b}=\frac{b}{1+a}$ alors $a(1+a)=b(1+b)$ donc
$a+a^2=b+b^2$ d'où $a^2-b^2 = b-a$.
Cela conduit à $(a-b)(a+b)= -(a-b)$. Comme $a\neq b$ alors $a-b\neq 0$ et donc en divisant
par $a-b$ on obtient $a+b=-1$. La somme des deux nombres positifs $a$ et $b$ ne peut être négative.
Nous obtenons une contradiction.

Conclusion : si $\frac{a}{1+b}=\frac{b}{1+a}$ alors $a=b$.
\end{proof}
\end{exemple}

Dans la pratique, on peut choisir indifféremment entre un raisonnement par contraposition ou par l'absurde.
Attention cependant de bien préciser quel type de raisonnement vous choisissez et surtout de ne pas changer
en cours de rédaction !

%---------------------------------------------------------------
\subsection{Contre-exemple}

Si l'on veut montrer qu'une assertion du type \assertion{$\forall x \in E \quad P(x)$} est vraie
alors pour chaque $x$ de $E$ il faut montrer que $P(x)$ est vraie. Par contre
pour montrer que cette assertion est fausse alors il suffit de trouver $x \in E$ tel que $P(x)$
 soit fausse. (Rappelez-vous la négation de \assertion{$\forall x \in E \quad P(x)$}
est \assertion{$\exists x \in E \quad \text{non}\ P(x)$}.)
Trouver un tel $x$ c'est trouver un \defi{contre-exemple}\index{contre-exemple} à l'assertion \assertion{$\forall x \in E \quad P(x)$}.

\begin{exemple}
Montrer que l'assertion suivante est fausse \assertion{Tout entier positif est somme de trois carrés}.

(Les carrés sont les $0^2$, $1^2$, $2^2$, $3^2$,... Par exemple
$6 = 2^2+1^2+1^2$.)

\begin{proof}
Un contre-exemple est $7$ : les carrés inférieurs à $7$ sont $0$, $1$, $4$
mais avec trois de ces nombres on ne peut faire $7$.
\end{proof}
\end{exemple}



%---------------------------------------------------------------
\subsection{Récurrence}

Le \defi{principe de récurrence}\index{recurrence@récurrence} permet de montrer qu'une assertion $P(n)$, dépendant de $n$,
est vraie pour tout $n \in \Nn$.
La démonstration par récurrence se déroule en trois étapes : lors de l'\evidence{initialisation} on prouve $P(0)$.
Pour l'étape d'\evidence{hérédité}\index{heredite@hérédité}, on suppose $n\ge 0$ donné avec $P(n)$ vraie,
et on démontre alors que l'assertion  $P(n+1)$ au rang suivant est vraie.
Enfin dans la \evidence{conclusion}, on rappelle que par le principe de récurrence $P(n)$ est vraie pour tout $n\in\Nn$.

\begin{exemple}
Montrer que pour tout $n\in \Nn$, $2^n > n$.

\begin{proof}
Pour $n\ge 0$, notons $P(n)$ l'assertion suivante :
$$2^n > n.$$
Nous allons démontrer par récurrence que $P(n)$ est vraie pour tout $n\ge 0$.


\textbf{Initialisation.} Pour $n=0$ nous avons $2^0=1>0$. Donc $P(0)$ est vraie.

\textbf{Hérédité.}
Fixons $n\ge 0$. Supposons que $P(n)$ soit vraie. Nous allons montrer que $P(n+1)$ est vraie.
\begin{align*}
 2^{n+1} &= 2^n + 2^n 
         > n + 2^n \qquad \text{ car par } P(n) \text{ nous savons } 2^n > n, \\
         &> n + 1 \qquad \text{ car } 2^n \ge 1. \\
\end{align*}
Donc $P(n+1)$ est vraie.



\textbf{Conclusion.} Par le principe de récurrence $P(n)$ est vraie pour tout $n\ge 0$, c'est-à-dire
$2^n > n$ pour tout $n\ge 0$.
\end{proof}
\end{exemple}

Remarques :
\begin{itemize}
  \item La rédaction d'une récurrence est assez rigide. Respectez scrupuleusement la rédaction
proposée : donnez un nom à l'assertion que vous souhaitez montrer (ici $P(n)$), respectez les trois étapes
(même si souvent l'étape d'initialisation est très facile).
En particulier méditez et conservez la première ligne de l'hérédité \og Fixons $n\ge 0$.
Supposons que $P(n)$ soit vraie. Nous allons montrer que $P(n+1)$ est vraie.\fg

  \item Si on doit démontrer qu'une propriété est vraie pour tout $n\ge n_0$, alors on commence l'initialisation au rang $n_0$.

  \item Le principe de récurrence est basé sur la construction de l'ensemble $\Nn$. En effet un des axiomes
pour définir $\Nn$ est le suivant :
\assertion{Soit $A$ une partie de $\Nn$ qui contient $0$ et telle que si $n \in A$ alors $n+1 \in A$.
Alors $A=\Nn$}.
\end{itemize}


%---------------------------------------------------------------
%\subsection{Mini-exercices}

\begin{miniexercices}
\sauteligne
\begin{enumerate}
  \item (Raisonnement direct) Soient $a,b \in \Rr_+$. Montrer que si $a \le b$ alors $a \le \frac{a+b}{2} \le b$ et
$a \le \sqrt{ab} \le b$.
  \item (Cas par cas) Montrer que pour tout $n\in \Nn$, $n(n+1)$ est divisible par $2$ (distinguer
les $n$ pairs des $n$ impairs).
  \item (Contraposée ou absurde) Soient $a,b \in \Zz$. Montrer que si $b\neq 0$ alors
$a+b\sqrt{2} \notin \Qq$. (On utilisera que $\sqrt 2 \notin \Qq$.)
  \item (Absurde) Soit $n \in\Nn^*$. Montrer que $\sqrt{n^2+1}$ n'est pas un entier.
  \item (Contre-exemple) Est-ce que pour tout $x\in \Rr$ on a $x<2 \implies x^2<4$ ?
  \item (Récurrence) Montrer que pour tout $n \ge 1$, $1+2+ \cdots +n = \frac{n(n+1)}{2}$.
  \item (Récurrence) Fixons un réel $x\ge0$. Montrer que pour tout entier $n \ge 1$,\ \  $(1+x)^n \ge 1+nx$.
\end{enumerate}
\end{miniexercices}





\auteurs{
Arnaud Bodin,
Benjamin Boutin,
Pascal Romon
}

\finchapitre
\end{document}


