
%%%%%%%%%%%%%%%%%% PREAMBULE %%%%%%%%%%%%%%%%%%


\documentclass[12pt]{article}

\usepackage{amsfonts,amsmath,amssymb,amsthm}
\usepackage[utf8]{inputenc}
\usepackage[T1]{fontenc}
\usepackage[francais]{babel}


% packages
\usepackage{amsfonts,amsmath,amssymb,amsthm}
\usepackage[utf8]{inputenc}
\usepackage[T1]{fontenc}
%\usepackage{lmodern}

\usepackage[francais]{babel}
\usepackage{fancybox}
\usepackage{graphicx}

\usepackage{float}

%\usepackage[usenames, x11names]{xcolor}
\usepackage{tikz}
\usepackage{datetime}

\usepackage{mathptmx}
%\usepackage{fouriernc}
%\usepackage{newcent}
\usepackage[mathcal,mathbf]{euler}

%\usepackage{palatino}
%\usepackage{newcent}


% Commande spéciale prompteur

%\usepackage{mathptmx}
%\usepackage[mathcal,mathbf]{euler}
%\usepackage{mathpple,multido}

\usepackage[a4paper]{geometry}
\geometry{top=2cm, bottom=2cm, left=1cm, right=1cm, marginparsep=1cm}

\newcommand{\change}{{\color{red}\rule{\textwidth}{1mm}\\}}

\newcounter{mydiapo}

\newcommand{\diapo}{\newpage
\hfill {\normalsize  Diapo \themydiapo \quad \texttt{[\jobname]}} \\
\stepcounter{mydiapo}}


%%%%%%% COULEURS %%%%%%%%%%

% Pour blanc sur noir :
%\pagecolor[rgb]{0.5,0.5,0.5}
% \pagecolor[rgb]{0,0,0}
% \color[rgb]{1,1,1}



%\DeclareFixedFont{\myfont}{U}{cmss}{bx}{n}{18pt}
\newcommand{\debuttexte}{
%%%%%%%%%%%%% FONTES %%%%%%%%%%%%%
\renewcommand{\baselinestretch}{1.5}
\usefont{U}{cmss}{bx}{n}
\bfseries

% Taille normale : commenter le reste !
%Taille Arnaud
%\fontsize{19}{19}\selectfont

% Taille Barbara
%\fontsize{21}{22}\selectfont

%Taille François
\fontsize{25}{30}\selectfont

%Taille Pascal
%\fontsize{25}{30}\selectfont

%Taille Laura
%\fontsize{30}{35}\selectfont


%\myfont
%\usefont{U}{cmss}{bx}{n}

%\Huge
%\addtolength{\parskip}{\baselineskip}
}


% \usepackage{hyperref}
% \hypersetup{colorlinks=true, linkcolor=blue, urlcolor=blue,
% pdftitle={Exo7 - Exercices de mathématiques}, pdfauthor={Exo7}}


%section
% \usepackage{sectsty}
% \allsectionsfont{\bf}
%\sectionfont{\color{Tomato3}\upshape\selectfont}
%\subsectionfont{\color{Tomato4}\upshape\selectfont}

%----- Ensembles : entiers, reels, complexes -----
\newcommand{\Nn}{\mathbb{N}} \newcommand{\N}{\mathbb{N}}
\newcommand{\Zz}{\mathbb{Z}} \newcommand{\Z}{\mathbb{Z}}
\newcommand{\Qq}{\mathbb{Q}} \newcommand{\Q}{\mathbb{Q}}
\newcommand{\Rr}{\mathbb{R}} \newcommand{\R}{\mathbb{R}}
\newcommand{\Cc}{\mathbb{C}} 
\newcommand{\Kk}{\mathbb{K}} \newcommand{\K}{\mathbb{K}}

%----- Modifications de symboles -----
\renewcommand{\epsilon}{\varepsilon}
\renewcommand{\Re}{\mathop{\text{Re}}\nolimits}
\renewcommand{\Im}{\mathop{\text{Im}}\nolimits}
%\newcommand{\llbracket}{\left[\kern-0.15em\left[}
%\newcommand{\rrbracket}{\right]\kern-0.15em\right]}

\renewcommand{\ge}{\geqslant}
\renewcommand{\geq}{\geqslant}
\renewcommand{\le}{\leqslant}
\renewcommand{\leq}{\leqslant}

%----- Fonctions usuelles -----
\newcommand{\ch}{\mathop{\mathrm{ch}}\nolimits}
\newcommand{\sh}{\mathop{\mathrm{sh}}\nolimits}
\renewcommand{\tanh}{\mathop{\mathrm{th}}\nolimits}
\newcommand{\cotan}{\mathop{\mathrm{cotan}}\nolimits}
\newcommand{\Arcsin}{\mathop{\mathrm{Arcsin}}\nolimits}
\newcommand{\Arccos}{\mathop{\mathrm{Arccos}}\nolimits}
\newcommand{\Arctan}{\mathop{\mathrm{Arctan}}\nolimits}
\newcommand{\Argsh}{\mathop{\mathrm{Argsh}}\nolimits}
\newcommand{\Argch}{\mathop{\mathrm{Argch}}\nolimits}
\newcommand{\Argth}{\mathop{\mathrm{Argth}}\nolimits}
\newcommand{\pgcd}{\mathop{\mathrm{pgcd}}\nolimits} 

\newcommand{\Card}{\mathop{\text{Card}}\nolimits}
\newcommand{\Ker}{\mathop{\text{Ker}}\nolimits}
\newcommand{\id}{\mathop{\text{id}}\nolimits}
\newcommand{\ii}{\mathrm{i}}
\newcommand{\dd}{\mathrm{d}}
\newcommand{\Vect}{\mathop{\text{Vect}}\nolimits}
\newcommand{\Mat}{\mathop{\mathrm{Mat}}\nolimits}
\newcommand{\rg}{\mathop{\text{rg}}\nolimits}
\newcommand{\tr}{\mathop{\text{tr}}\nolimits}
\newcommand{\ppcm}{\mathop{\text{ppcm}}\nolimits}

%----- Structure des exercices ------

\newtheoremstyle{styleexo}% name
{2ex}% Space above
{3ex}% Space below
{}% Body font
{}% Indent amount 1
{\bfseries} % Theorem head font
{}% Punctuation after theorem head
{\newline}% Space after theorem head 2
{}% Theorem head spec (can be left empty, meaning ‘normal’)

%\theoremstyle{styleexo}
\newtheorem{exo}{Exercice}
\newtheorem{ind}{Indications}
\newtheorem{cor}{Correction}


\newcommand{\exercice}[1]{} \newcommand{\finexercice}{}
%\newcommand{\exercice}[1]{{\tiny\texttt{#1}}\vspace{-2ex}} % pour afficher le numero absolu, l'auteur...
\newcommand{\enonce}{\begin{exo}} \newcommand{\finenonce}{\end{exo}}
\newcommand{\indication}{\begin{ind}} \newcommand{\finindication}{\end{ind}}
\newcommand{\correction}{\begin{cor}} \newcommand{\fincorrection}{\end{cor}}

\newcommand{\noindication}{\stepcounter{ind}}
\newcommand{\nocorrection}{\stepcounter{cor}}

\newcommand{\fiche}[1]{} \newcommand{\finfiche}{}
\newcommand{\titre}[1]{\centerline{\large \bf #1}}
\newcommand{\addcommand}[1]{}
\newcommand{\video}[1]{}

% Marge
\newcommand{\mymargin}[1]{\marginpar{{\small #1}}}



%----- Presentation ------
\setlength{\parindent}{0cm}

%\newcommand{\ExoSept}{\href{http://exo7.emath.fr}{\textbf{\textsf{Exo7}}}}

\definecolor{myred}{rgb}{0.93,0.26,0}
\definecolor{myorange}{rgb}{0.97,0.58,0}
\definecolor{myyellow}{rgb}{1,0.86,0}

\newcommand{\LogoExoSept}[1]{  % input : echelle
{\usefont{U}{cmss}{bx}{n}
\begin{tikzpicture}[scale=0.1*#1,transform shape]
  \fill[color=myorange] (0,0)--(4,0)--(4,-4)--(0,-4)--cycle;
  \fill[color=myred] (0,0)--(0,3)--(-3,3)--(-3,0)--cycle;
  \fill[color=myyellow] (4,0)--(7,4)--(3,7)--(0,3)--cycle;
  \node[scale=5] at (3.5,3.5) {Exo7};
\end{tikzpicture}}
}



\theoremstyle{definition}
%\newtheorem{proposition}{Proposition}
%\newtheorem{exemple}{Exemple}
%\newtheorem{theoreme}{Théorème}
\newtheorem{lemme}{Lemme}
\newtheorem{corollaire}{Corollaire}
%\newtheorem*{remarque*}{Remarque}
%\newtheorem*{miniexercice}{Mini-exercices}
%\newtheorem{definition}{Définition}




%definition d'un terme
\newcommand{\defi}[1]{{\color{myorange}\textbf{\emph{#1}}}}
\newcommand{\evidence}[1]{{\color{blue}\textbf{\emph{#1}}}}



 %----- Commandes divers ------

\newcommand{\codeinline}[1]{\texttt{#1}}

\newcommand{\codeinline}[1]{\texttt{#1}}

%%%%%%%%%%%%%%%%%%%%%%%%%%%%%%%%%%%%%%%%%%%%%%%%%%%%%%%%%%%%%
%%%%%%%%%%%%%%%%%%%%%%%%%%%%%%%%%%%%%%%%%%%%%%%%%%%%%%%%%%%%%


\begin{document}

\debuttexte

%%%%%%%%%%%%%%%%%%%%%%%%%%%%%%%%%%%%%%%%%%%%%%%%%%%%%%%%%%%
\diapo

\change

Dans cette leçon on s'interesse aux entiers.

\change

Nous commençons par un peu d'arithmétique avec la division euclidienne et les calculs avec les modulos.

\change

Nous verrons l'écriture décimales des entiers 

\change

Pour cela nous utiliserons la notion de listes et le module \codeinline{math}.

\change

Le but de ce chapitre étant d'obtenir l'écriture binaire des entiers.

%%%%%%%%%%%%%%%%%%%%%%%%%%%%%%%%%%%%%%%%%%%%%%%%%%%%%%%%%%%
\diapo

La division euclidienne de $a$ par $b$ s'écrit :
$$a = bq+r \quad \text{ et } \quad 0 \le r < b$$

\change

où $q \in \Zz$ est le \defi{quotient} 

\change

En Python le quotient se calcule par : \codeinline{a // b}. Notez la double barre de division !

\change

Et $r \in \Nn$ est le \defi{reste}.

\change

Qui se calcule par \codeinline{a \% b}.

\change

Par exemple : \codeinline{14 // 3} retourne $4$

\change

Alors que \codeinline{14 \% 3}  retourne $2$ (il faut lire $14$ modulo $3$ égale $2$).

\change

On a bien $14 = 3 \times 4 + 2$ qui est l'écriture de la division euclidienne.

\change

Les calculs avec les modulos sont très pratiques.
Par exemple pour tester si un entier est pair, ou impair cela revient à un test modulo $2$.
Le code est \codeinline{if (n\%2 == 0): ... else: ...}.

\change

Si on besoin de calculer $\cos( n\frac\pi2)$ alors il faut discuter
suivant les valeurs de \codeinline{n\%4}.

%%%%%%%%%%%%%%%%%%%%%%%%%%%%%%%%%%%%%%%%%%%%%%%%%%%%%%%%%%%
\diapo

Pour s'échauffer voici le petit exercice suivant !

Il s'agit de compter le nombre de fois où le chiffre $1$ apparaît dans tous les nombres
de $1$ à $999$.

Ne continuer la vidéo qu'une fois que vous avez terminé la programmation.


\change

On écrit une boucle "pour" qui fait varier notre compteur $N$ de $1$ à $999$.

Comment obtient le chiffre des unités d'un nombre ? C'est tout simplement le reste
de $N$ modulo $10$.

Par exemple $237$ modulo $10$ donne $7$.

\change

Pour le chiffre des dizaines c'est un peu plus compliqué.

On commence par diviser $N$ par $10$. C'est ici la division d'entiers.

Par exemple $237$ diviser par $10$ c'est $23$. 
Ensuite on prend le reste modulo $10$.

Donc pour $237$ on divise par $10$ pour obtenir $23$ puis modulo $10$ on obtient $3$.
Qui est bien le chiffre des dizaines de $237$.

\change

Pour obtenir le chiffre des centaines on commence par diviser par $100$.


\change

On a défini un compteur NbdeUn qui compte le nombre de fois où le $1$ apparaît.

\change

Compteur que l'on incrémente à chaque fois que l'on a le chiffre $1$.

\change

\change

Il ne reste plus qu'à afficher le résultat.


%%%%%%%%%%%%%%%%%%%%%%%%%%%%%%%%%%%%%%%%%%%%%%%%%%%%%%%%%%%
\diapo

L'écriture décimale d'un nombre, c'est associer à un entier $N$
la suite de ses chiffres $[a_0,a_1,\ldots,a_n]$ de sorte que $a_i$ soit le $i$-ème chiffre
de $N$.

\change

 C'est-à-dire 
$N= a_n 10^n+ a_{n-1}10^{n-1}+\cdots + a_2 10^2 + a_1 10 + a_0$
[partir de la fin]

et ce qui est important $a_i \in \{0,1,\ldots,9\}$

\change

$a_0$ est le chiffre des unités, $a_1$ celui des dizaines, $a_2$ celui des centaines,...

\change

Voici votre travail :

tout d'abord à partir de la liste des chiffres, calculer l'entier.

Puis l'opération inverse : partant d'un entier $N$ retrouver la liste de ses chiffres.



%%%%%%%%%%%%%%%%%%%%%%%%%%%%%%%%%%%%%%%%%%%%%%%%%%%%%%%%%%%
\diapo

Voici le premier algorithme de la liste vers l'entier :

\change

C'est juste la mise en oeuvre de la formule mathématique 
$N= a_n 10^n+ a_{n-1}10^{n-1}+\cdots + a_2 10^2 + a_1 10 + a_0$.

\change

Par exemple \codeinline{chiffres-vers-entier([4,3,2,1])} renvoie l'entier $1234$.

$4$ est le chiffre des unités, $3$ celui des dizaines,...

Nous avons un indice $i$ qui correspond à l'exposant $10^i$,

$10^0$ pour les unités, $10^1$ pour les dizaines, $10^2$ pour les centaines.

Il reste à multiplier $10^i$ par le chiffre $a_i$ qui est pour nous stocké dans $tab[i]$.



%%%%%%%%%%%%%%%%%%%%%%%%%%%%%%%%%%%%%%%%%%%%%%%%%%%%%%%%%%%
\diapo

Expliquons les bases sur les \evidence{listes} (qui s'appelle aussi des \evidence{tableaux})

 En Python une liste est présentée entre des crochets.
 
 
\change

Voici un exemple de liste \codeinline{tab = [4,3,2,1]}

\change

On accède aux valeurs par \codeinline{tab[i]} :

\change

$i$ étant un indice qui varie ici de $0$ à $3$.

 \codeinline{tab[0]} vaut $4$,  \codeinline{tab[1]} vaut $3$, 
 
 \codeinline{tab[2]} vaut $2$,  \codeinline{tab[3]} vaut $1$.


\change

Une façon simple de parcourir les éléments d'un tableau est \codeinline{for x in tab}, 
  $x$ vaut alors successivement $4$, $3$, $2$, $1$.
  
\change

La longueur du tableau s'obtient par \codeinline{len(tab)}. Pour notre exemple
  \codeinline{len([4,3,2,1])} vaut $4$.
  
  \change
  
  Une autre façon de parcourir toutes les valeurs d'un tableau est  donc d'écrire
  
  \codeinline{for i in range(len(tab))}, puis utiliser \codeinline{tab[i]}, 
  
  ici $i$ variant ici de $0$ à $3$.
  
   
  \change 
  
  La liste vide est seulement notée avec deux crochets : \codeinline{[]}.
  
  Elle est utile pour initialiser une liste.
  
  \change
  
  Pour ajouter un élément à une liste \codeinline{tab} existante on utilise
  la fonction \codeinline{append}. Par exemple définissons la liste vide \codeinline{tab=[]},
  
  \change
  
   pour ajouter une valeur à la fin de la liste on saisit : \codeinline{tab.append(4)}. Maintenant notre liste 
  est $[4]$, elle contient un seul élément. 
  
   \change 
   
 Si on continue avec \codeinline{tab.append(3)}. Alors maintenant 
  notre liste a deux éléments : $[4,3]$.  etc.
  


%%%%%%%%%%%%%%%%%%%%%%%%%%%%%%%%%%%%%%%%%%%%%%%%%%%%%%%%%%%
\diapo

Voici l'écriture d'un entier $N$ en base $10$.

\change

Par exemple \codeinline{entier-vers-chiffres(1234)} renvoie le tableau $[4,3,2,1]$.

Nous avons expliqué tout ce dont nous avions besoin sur les listes au-dessus, 
expliquons les mathématiques.

\change

Décomposons les entiers naturels non nuls sous la forme 

  $[1,10[  \ \cup  \ [10,100[ \ \cup$
  $\  [100,1000[ \  \cup \  [1\,000,10\, 000[ \ \cup \cdots$ 

  Ce qui correspond aux entiers à $1$ chiffre, à $2$ chiffres, à $3$ chiffres,...
  
\change

  Chaque intervalle est du type $[10^n,10^{n+1}[$.
  
\change

[[grand N, petit n]]

  Pour un entier $N$ non nul il appartient à un de ces intervalles, il existe donc $n\in\Nn$ 
  tel que $10^n \le N < 10^{n+1}$. 
    
\change 

  Ce qui indique que le nombre de chiffres
  de $N$ est $n+1$.
  
\change 

[[grand N, petit n]]

  Par exemple si $N=1234$ alors $1 \, 000 = 10^3 \le N < 10^4 = 10\, 000$, 
  ainsi $n=3$ et le nombre de chiffres est $3+1=4$.
    
\change 

  [[grand N, petit n]]
  
Comment calcule-t'on $n$ à partir de $N$ ?

\change

Nous allons utiliser le logarithme décimal $\log_{10}$ qui vérifie
  $\log_{10}(10) = 1$ et $\log_{10}(10^i) = i$. 
  
  
\change

Le logarithme est une fonction croissante, donc l'inégalité 
  $10^n \le N < 10^{n+1}$ devient $\log_{10}(10^n) \le \log_{10}(N) < \log_{10}(10^{n+1})$. 
  
\change
  
  
  Ainsi $n \le \log_{10}(N) < n+1$.
  
  Ce qui indique donc que $n = E(\log_{10}(N))$ où $E(x)$ désigne la partie entière d'un réel $x$.
  
  [pause]
  
Pour l'algorithme, on part de la liste vide, 
on calcule $n$ comme la partie entière du log décimale de $N$
et on calcule les chiffres $1$ par $1$,
la $i$-ème chiffre étant obtenu comme précédemment
puis ajouté à la liste.


%%%%%%%%%%%%%%%%%%%%%%%%%%%%%%%%%%%%%%%%%%%%%%%%%%%%%%%%%%%
\diapo

Quelques commentaires informatiques sur un module important pour nous.
Les fonctions mathématiques ne sont pas définies par défaut dans Python

il faut faire appel à une librairie spéciale : le module \codeinline{math} 
contient les fonctions mathématiques principales.

La première façon d'utiliser ce module est d'écrire la ligne \codeinline{import math}  en début de fichier

puis pour utiliser la fonction cosinus par exemple on l'appelle par \codeinline{math.cos()}


\change

Comme on aura souvent besoin de ce module on préfère mettre en tête de fichier la ligne 
\codeinline{from math import *}.
  Cela signifie que l'on importe toutes les fonctions de ce module 
  et qu'en plus on n'a pas besoin de préciser que la fonction
  vient du module \codeinline{math}. 
  
  On peut écrire directement \codeinline{cos()} au lieu \codeinline{math.cos()}.
  
\change


Voici maintenant une liste de fonction mathématiques usuelles .

   \codeinline{abs(x)}     pour la valeur absolue   $|x|$  
   
   \codeinline{x ** n}     pour la puissance   $x^n$  
   
Ces fonctions sont toujours accessibles dans Python alors que les suivantes sont issues du module math.

La racine carrée,

L'exponentielle,

Le logarithme néperien ou logarithme naturel qui vérifie $ln(e)=1$

Le logarithme décimal avec $log(10)=1$.

Et on peut aussi spécifier une base différente de $10$.

Les fonctions cosinus, sinus, tangente : mais attention les angles sont exprimés en radians.
   
Les fonctions trigonométriques inverses : qui renvoient des angles en radians. 
   
La fonction partie entière, qui renvoie le plus grand entier en-dessous de $x$ (\emph{floor} signifie plancher)

Et \emph{ceil} (qui signifie plafond) renvoie le plus petit entier au-dessus de $x$.



%%%%%%%%%%%%%%%%%%%%%%%%%%%%%%%%%%%%%%%%%%%%%%%%%%%%%%%%%%%
\diapo

On dispose d'une rampe de lumière, chacune des $8$ lampes pouvant être 
allumée (rouge) ou éteinte (gris).


On numérote les lampes de $0$ à $7$.
On souhaite contrôler cette rampe : afficher toutes les combinaisons possibles, faire défiler 
une combinaison de la gauche à droite (la ``chenille'')

[7 clics]

\change

\change

\change

\change

\change

\change

\change

, inverser l'état de toutes les lampes,...

[3 clics]

\change

\change

\change


Voyons comment l'écriture binaire des nombres peut nous aider.
L'\defi{écriture binaire} d'un nombre c'est son écriture en base $2$.

\change

Comment calculer un nombre qui est écrit en binaire ?
Le chiffre des ``dizaines'' correspond à $2$ (au lieu de $10$ en écriture décimale),
le chiffre des ``centaines'' à $4=2^2$ (au lieu de $100=10^2$), 
le chiffres des ``milliers'' à $8=2^3$ (au lieu de $1000=10^3$),...
Pour le chiffre des unités cela correspond à $2^0 = 1$ (de même que $10^0=1$).

\change

Par exemple quel est l'entier dont l'écriture est $10011_b$.

\change


${\color{blue}10011}_b = {\color{blue}1} \cdot 2^4 + {\color{blue}0} \cdot 2^3 + 
{\color{blue}0} \cdot 2^2 + {\color{blue}1}\cdot 2^1 + {\color{blue}1}\cdot 2^0$ 

[[lire ordre inverse]]

\change

$= 16+2+1=19.$

\change

C'est donc le nombre $19$

\change

De façon générale tout entier $N\in \Nn$ s'écrit de manière unique sous la forme
$$N= a_n 2^n+ a_{n-1}2^{n-1}+\cdots + a_2 2^2 + a_1 2 + a_0$$

et cette fois $a_i$ ne peut être que $0$ ou $1$.

\change



On note alors $N= a_n a_{n-1}\ldots a_1 a_0 \ _b$ 

(avec un indice $b$ pour indiquer que c'est son écriture binaire).


 

%%%%%%%%%%%%%%%%%%%%%%%%%%%%%%%%%%%%%%%%%%%%%%%%%%%%%%%%%%%
\diapo

Voici votre premier travail avec les nombres binaires :

A partir d'une écriture binaire calculer l'entier correspondant.

Et réciproquement à partir d'un entier calculer son écriture binaire.

Programmer ces fonctions avant de continuer.

\change

Voici maintenant les algorithmes 
ils sont en tout point similaire à ceux pour l'écriture en base $10$.

Sauf que bien sûr ici on calcule avec des puissances de $2$.


\change

Idem pour le sens inverse où l'on a besoin du logarithme en base $2$, qui vérifie
$\log_2(2)=1$ et $\log_2(2^i)=i$.

On obtient ainsi le nombre de chiffre de l'écriture en base $2$.

Et on calcule chacun des chiffres par des divisions euclidiennes par $2^i$ 
puis le reste modulo $2$ : qui est donc soit $0$ soit $1$.


%%%%%%%%%%%%%%%%%%%%%%%%%%%%%%%%%%%%%%%%%%%%%%%%%%%%%%%%%%%
\diapo

Maintenant appliquons ceci à notre problème de lampes.
Si une lampe est allumée on lui attribut $1$, et si elle est éteinte $0$.
Pour une rampe de $8$ lampes on code $[a_0,a_1,\ldots,a_7]$ l'état des lampes.

\change

Par exemple la configuration suivante :
est codé $[1,0,0,1,0,1,1,1]$ 

\change

ce qui correspond au nombre binaire
$11101001_b = 233$.

(attention à l'inversion de l'ordre).

\change

Voici votre travail :

vous devez en premier afficher toutes les  configurations.

Avec les trois questions suivantes vous allez programmer la "chenille" : une configuration 
se décale cycliquement vers la droite.

Enfin la dernière question est de voir mathématiquement à quoi correspond l'inversion d'une configuration :
les lampes allumées s'éteignent et les lampes éteintes s'allument.


%%%%%%%%%%%%%%%%%%%%%%%%%%%%%%%%%%%%%%%%%%%%%%%%%%%%%%%%%%%
\diapo

Il s'agit d'abord d'afficher les configurations. Par exemple si l'on a $4$ lampes alors
les configurations sont $[0,0,0,0]$, $[1,0,0,0]$, $[0,1,0,0]$, $[1,1,0,0]$,\ldots, $[1,1,1,1]$.

\change

Pour chaque lampe nous avons deux choix (allumé ou éteint),

\change

 il y a en tout $n+1$ lampes donc un total de $2^{n+1}$ configurations.

\change

 Si l'on considère ces configurations comme des nombres écrits en binaire alors 
l'énumération ci-dessus correspond à compter $0,1,2,3,\ldots, 2^{n+1}-1$.

\change

D'où l'algorithme suivant.

C'est juste l'énumération des entiers de $0$ à $2^{n+1}-1$
et leur conversion en écriture binaire.


\change

Où \codeinline{entier-vers-binaire-bis(N,n)} est similaire à \codeinline{entier-vers-binaire(N)},
mais en affichant aussi les zéros non significatifs,

\change

par exemple $7$ en binaire s'écrit simplement $111_b$,
mais codé sur $8$ chiffres on ajoute devant des $0$ non significatifs : $00000111_b$.




%%%%%%%%%%%%%%%%%%%%%%%%%%%%%%%%%%%%%%%%%%%%%%%%%%%%%%%%%%%
\diapo

Maintenant nous allons décaler cycliquement la configuration d'une rampe.


Commençons par une analogie : 
En écriture décimale, multiplier par $10$ revient à décaler le nombre initial et rajouter un zéro.
Par exemple $10\times{\color{blue}19} = {\color{blue}19}{\color{red}0}$.

\change

C'est la même chose en binaire !
Multiplier un nombre par $2$ revient sur l'écriture à un décalage 
vers la gauche et ajout d'un zéro sur le chiffre des unité.

\change 

Exemple : $19 = 10011_b$,

$2 \times{\color{blue}10011}_b  = {\color{blue}10011}{\color{red}0}_b$

c'est l'écriture de $2 \times 19 = 38$.

\change

 Partant de $N=a_n  a_{n-1}\ldots a_1 a_0 \ _b$. 
  
\change

Notons $N'=2N$, son écriture binaire est 
$N' = a_n  a_{n-1}\ldots a_1 a_0 0 \ _b$.

avec un $0$ à la fin.

\change

L'entier $N'$ est $a_n \times 2^{n+1} + a_{n-1} \times 2^n + ...$

donc $N' \pmod {2^{n+1}}$ fait disparaitre $a_n$ et s'écrit 
$a_{n-1} a_{n-2} \ldots a_1 a_0 0 \ _b$ 
  
\change

Pour faire permuter notre écriture il faut remplacer le chiffre des unités  par $a_n$ :
il suffit d'additionner $a_n$.

Comment s'obtient $a_n$ : c'est le quotient de $N'$ par $2^{n+1}$.

\change

En résumé $N' \pmod{2^{n+1}} + a_n$ vaut exactement ceci,

\change


ce qui signifie que son écriture binaire est $a_{n-1} a_{n-2} \ldots a_1 a_0 a_n \ _b$.

Exactement la permutation que nous souhaitions !


\change

Une fois que l'on compris cela, le code est très simple :

Partant de notre configuration, on transforme cette configuration en un entier.

On fait notre transformation de $N$ en $N' \pmod{2^{n+1}} + a_n$

et on retransforme ce nombre en écriture binaire.


%%%%%%%%%%%%%%%%%%%%%%%%%%%%%%%%%%%%%%%%%%%%%%%%%%%%%%%%%%%
\diapo

On remarque que si l'on a deux configurations opposées alors leur somme vaut $2^{n+1}-1$

\change

par exemple avec $[1,0,0,1,0,1,1,1]$ et $[0,1,1,0,1,0,0,0]$,

\change

les deux nombres associés sont
$N$ et $N'$ (il s'agit juste de les réécrire de droite à gauche)

\change

La somme $N + N' = 11101001_b + 00010110_b = 11111111_b $

L'addition en écriture binaire se fait de la même façon qu'en écriture décimale et 
ici il n'y a pas de retenue.

\change

cela vaut $2^{8}-1$. 

\change

C'est donc la formule $N' = 2^{n+1}-1 - N$  que l'on va appliquer.

\change

Voici l'algorithme : 

on part d'une liste que l'on transforme en un entier $N$

On calcule son opposé $N'$ par la formule  $N' = 2^{n+1}-1 - N$ 

On converti $N'$ en écriture binaire.

%%%%%%%%%%%%%%%%%%%%%%%%%%%%%%%%%%%%%%%%%%%%%%%%%%%%%%%%%%%
\diapo

Voici une liste assez longue d'exercices pour bien assimiler l'écriture des nombres. 


\end{document}