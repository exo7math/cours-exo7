
%%%%%%%%%%%%%%%%%% PREAMBULE %%%%%%%%%%%%%%%%%%

\documentclass[aspectratio=169,utf8]{beamer}
%\documentclass[aspectratio=169,handout]{beamer}

\usetheme{Boadilla}
%\usecolortheme{seahorse}
\usecolortheme[RGB={245,66,24}]{structure}
\useoutertheme{infolines}

% packages
\usepackage{amsfonts,amsmath,amssymb,amsthm}
\usepackage[utf8]{inputenc}
\usepackage[T1]{fontenc}
\usepackage{lmodern}

\usepackage[francais]{babel}
\usepackage{fancybox}
\usepackage{graphicx}

\usepackage{float}
\usepackage{xfrac}

%\usepackage[usenames, x11names]{xcolor}
\usepackage{tikz}
\usepackage{pgfplots}
\usepackage{datetime}



%-----  Package unités -----
\usepackage{siunitx}
\sisetup{locale = FR,detect-all,per-mode = symbol}

%\usepackage{mathptmx}
%\usepackage{fouriernc}
%\usepackage{newcent}
%\usepackage[mathcal,mathbf]{euler}

%\usepackage{palatino}
%\usepackage{newcent}
% \usepackage[mathcal,mathbf]{euler}



% \usepackage{hyperref}
% \hypersetup{colorlinks=true, linkcolor=blue, urlcolor=blue,
% pdftitle={Exo7 - Exercices de mathématiques}, pdfauthor={Exo7}}


%section
% \usepackage{sectsty}
% \allsectionsfont{\bf}
%\sectionfont{\color{Tomato3}\upshape\selectfont}
%\subsectionfont{\color{Tomato4}\upshape\selectfont}

%----- Ensembles : entiers, reels, complexes -----
\newcommand{\Nn}{\mathbb{N}} \newcommand{\N}{\mathbb{N}}
\newcommand{\Zz}{\mathbb{Z}} \newcommand{\Z}{\mathbb{Z}}
\newcommand{\Qq}{\mathbb{Q}} \newcommand{\Q}{\mathbb{Q}}
\newcommand{\Rr}{\mathbb{R}} \newcommand{\R}{\mathbb{R}}
\newcommand{\Cc}{\mathbb{C}} 
\newcommand{\Kk}{\mathbb{K}} \newcommand{\K}{\mathbb{K}}

%----- Modifications de symboles -----
\renewcommand{\epsilon}{\varepsilon}
\renewcommand{\Re}{\mathop{\text{Re}}\nolimits}
\renewcommand{\Im}{\mathop{\text{Im}}\nolimits}
%\newcommand{\llbracket}{\left[\kern-0.15em\left[}
%\newcommand{\rrbracket}{\right]\kern-0.15em\right]}

\renewcommand{\ge}{\geqslant}
\renewcommand{\geq}{\geqslant}
\renewcommand{\le}{\leqslant}
\renewcommand{\leq}{\leqslant}
\renewcommand{\epsilon}{\varepsilon}

%----- Fonctions usuelles -----
\newcommand{\ch}{\mathop{\text{ch}}\nolimits}
\newcommand{\sh}{\mathop{\text{sh}}\nolimits}
\renewcommand{\tanh}{\mathop{\text{th}}\nolimits}
\newcommand{\cotan}{\mathop{\text{cotan}}\nolimits}
\newcommand{\Arcsin}{\mathop{\text{arcsin}}\nolimits}
\newcommand{\Arccos}{\mathop{\text{arccos}}\nolimits}
\newcommand{\Arctan}{\mathop{\text{arctan}}\nolimits}
\newcommand{\Argsh}{\mathop{\text{argsh}}\nolimits}
\newcommand{\Argch}{\mathop{\text{argch}}\nolimits}
\newcommand{\Argth}{\mathop{\text{argth}}\nolimits}
\newcommand{\pgcd}{\mathop{\text{pgcd}}\nolimits} 


%----- Commandes divers ------
\newcommand{\ii}{\mathrm{i}}
\newcommand{\dd}{\text{d}}
\newcommand{\id}{\mathop{\text{id}}\nolimits}
\newcommand{\Ker}{\mathop{\text{Ker}}\nolimits}
\newcommand{\Card}{\mathop{\text{Card}}\nolimits}
\newcommand{\Vect}{\mathop{\text{Vect}}\nolimits}
\newcommand{\Mat}{\mathop{\text{Mat}}\nolimits}
\newcommand{\rg}{\mathop{\text{rg}}\nolimits}
\newcommand{\tr}{\mathop{\text{tr}}\nolimits}


%----- Structure des exercices ------

\newtheoremstyle{styleexo}% name
{2ex}% Space above
{3ex}% Space below
{}% Body font
{}% Indent amount 1
{\bfseries} % Theorem head font
{}% Punctuation after theorem head
{\newline}% Space after theorem head 2
{}% Theorem head spec (can be left empty, meaning ‘normal’)

%\theoremstyle{styleexo}
\newtheorem{exo}{Exercice}
\newtheorem{ind}{Indications}
\newtheorem{cor}{Correction}


\newcommand{\exercice}[1]{} \newcommand{\finexercice}{}
%\newcommand{\exercice}[1]{{\tiny\texttt{#1}}\vspace{-2ex}} % pour afficher le numero absolu, l'auteur...
\newcommand{\enonce}{\begin{exo}} \newcommand{\finenonce}{\end{exo}}
\newcommand{\indication}{\begin{ind}} \newcommand{\finindication}{\end{ind}}
\newcommand{\correction}{\begin{cor}} \newcommand{\fincorrection}{\end{cor}}

\newcommand{\noindication}{\stepcounter{ind}}
\newcommand{\nocorrection}{\stepcounter{cor}}

\newcommand{\fiche}[1]{} \newcommand{\finfiche}{}
\newcommand{\titre}[1]{\centerline{\large \bf #1}}
\newcommand{\addcommand}[1]{}
\newcommand{\video}[1]{}

% Marge
\newcommand{\mymargin}[1]{\marginpar{{\small #1}}}

\def\noqed{\renewcommand{\qedsymbol}{}}


%----- Presentation ------
\setlength{\parindent}{0cm}

%\newcommand{\ExoSept}{\href{http://exo7.emath.fr}{\textbf{\textsf{Exo7}}}}

\definecolor{myred}{rgb}{0.93,0.26,0}
\definecolor{myorange}{rgb}{0.97,0.58,0}
\definecolor{myyellow}{rgb}{1,0.86,0}

\newcommand{\LogoExoSept}[1]{  % input : echelle
{\usefont{U}{cmss}{bx}{n}
\begin{tikzpicture}[scale=0.1*#1,transform shape]
  \fill[color=myorange] (0,0)--(4,0)--(4,-4)--(0,-4)--cycle;
  \fill[color=myred] (0,0)--(0,3)--(-3,3)--(-3,0)--cycle;
  \fill[color=myyellow] (4,0)--(7,4)--(3,7)--(0,3)--cycle;
  \node[scale=5] at (3.5,3.5) {Exo7};
\end{tikzpicture}}
}


\newcommand{\debutmontitre}{
  \author{} \date{} 
  \thispagestyle{empty}
  \hspace*{-10ex}
  \begin{minipage}{\textwidth}
    \titlepage  
  \vspace*{-2.5cm}
  \begin{center}
    \LogoExoSept{2.5}
  \end{center}
  \end{minipage}

  \vspace*{-0cm}
  
  % Astuce pour que le background ne soit pas discrétisé lors de la conversion pdf -> png
\begin{tikzpicture}
        \fill[opacity=0,green!60!black] (0,0)--++(0,0)--++(0,0)--++(0,0)--cycle; 
\end{tikzpicture}

% toc S'affiche trop tot :
% \tableofcontents[hideallsubsections, pausesections]
}

\newcommand{\finmontitre}{
  \end{frame}
  \setcounter{framenumber}{0}
} % ne marche pas pour une raison obscure

%----- Commandes supplementaires ------

% \usepackage[landscape]{geometry}
% \geometry{top=1cm, bottom=3cm, left=2cm, right=10cm, marginparsep=1cm
% }
% \usepackage[a4paper]{geometry}
% \geometry{top=2cm, bottom=2cm, left=2cm, right=2cm, marginparsep=1cm
% }

%\usepackage{standalone}


% New command Arnaud -- november 2011
\setbeamersize{text margin left=24ex}
% si vous modifier cette valeur il faut aussi
% modifier le decalage du titre pour compenser
% (ex : ici =+10ex, titre =-5ex

\theoremstyle{definition}
%\newtheorem{proposition}{Proposition}
%\newtheorem{exemple}{Exemple}
%\newtheorem{theoreme}{Théorème}
%\newtheorem{lemme}{Lemme}
%\newtheorem{corollaire}{Corollaire}
%\newtheorem*{remarque*}{Remarque}
%\newtheorem*{miniexercice}{Mini-exercices}
%\newtheorem{definition}{Définition}

% Commande tikz
\usetikzlibrary{calc}
\usetikzlibrary{patterns,arrows}
\usetikzlibrary{matrix}
\usetikzlibrary{fadings} 

%definition d'un terme
\newcommand{\defi}[1]{{\color{myorange}\textbf{\emph{#1}}}}
\newcommand{\evidence}[1]{{\color{blue}\textbf{\emph{#1}}}}
\newcommand{\assertion}[1]{\emph{\og#1\fg}}  % pour chapitre logique
%\renewcommand{\contentsname}{Sommaire}
\renewcommand{\contentsname}{}
\setcounter{tocdepth}{2}



%------ Figures ------

\def\myscale{1} % par défaut 
\newcommand{\myfigure}[2]{  % entrée : echelle, fichier figure
\def\myscale{#1}
\begin{center}
\footnotesize
{#2}
\end{center}}


%------ Encadrement ------

\usepackage{fancybox}


\newcommand{\mybox}[1]{
\setlength{\fboxsep}{7pt}
\begin{center}
\shadowbox{#1}
\end{center}}

\newcommand{\myboxinline}[1]{
\setlength{\fboxsep}{5pt}
\raisebox{-10pt}{
\shadowbox{#1}
}
}

%--------------- Commande beamer---------------
\newcommand{\beameronly}[1]{#1} % permet de mettre des pause dans beamer pas dans poly


\setbeamertemplate{navigation symbols}{}
\setbeamertemplate{footline}  % tiré du fichier beamerouterinfolines.sty
{
  \leavevmode%
  \hbox{%
  \begin{beamercolorbox}[wd=.333333\paperwidth,ht=2.25ex,dp=1ex,center]{author in head/foot}%
    % \usebeamerfont{author in head/foot}\insertshortauthor%~~(\insertshortinstitute)
    \usebeamerfont{section in head/foot}{\bf\insertshorttitle}
  \end{beamercolorbox}%
  \begin{beamercolorbox}[wd=.333333\paperwidth,ht=2.25ex,dp=1ex,center]{title in head/foot}%
    \usebeamerfont{section in head/foot}{\bf\insertsectionhead}
  \end{beamercolorbox}%
  \begin{beamercolorbox}[wd=.333333\paperwidth,ht=2.25ex,dp=1ex,right]{date in head/foot}%
    % \usebeamerfont{date in head/foot}\insertshortdate{}\hspace*{2em}
    \insertframenumber{} / \inserttotalframenumber\hspace*{2ex} 
  \end{beamercolorbox}}%
  \vskip0pt%
}


\definecolor{mygrey}{rgb}{0.5,0.5,0.5}
\setlength{\parindent}{0cm}
%\DeclareTextFontCommand{\helvetica}{\fontfamily{phv}\selectfont}

% background beamer
\definecolor{couleurhaut}{rgb}{0.85,0.9,1}  % creme
\definecolor{couleurmilieu}{rgb}{1,1,1}  % vert pale
\definecolor{couleurbas}{rgb}{0.85,0.9,1}  % blanc
\setbeamertemplate{background canvas}[vertical shading]%
[top=couleurhaut,middle=couleurmilieu,midpoint=0.4,bottom=couleurbas] 
%[top=fondtitre!05,bottom=fondtitre!60]



\makeatletter
\setbeamertemplate{theorem begin}
{%
  \begin{\inserttheoremblockenv}
  {%
    \inserttheoremheadfont
    \inserttheoremname
    \inserttheoremnumber
    \ifx\inserttheoremaddition\@empty\else\ (\inserttheoremaddition)\fi%
    \inserttheorempunctuation
  }%
}
\setbeamertemplate{theorem end}{\end{\inserttheoremblockenv}}

\newenvironment{theoreme}[1][]{%
   \setbeamercolor{block title}{fg=structure,bg=structure!40}
   \setbeamercolor{block body}{fg=black,bg=structure!10}
   \begin{block}{{\bf Th\'eor\`eme }#1}
}{%
   \end{block}%
}


\newenvironment{proposition}[1][]{%
   \setbeamercolor{block title}{fg=structure,bg=structure!40}
   \setbeamercolor{block body}{fg=black,bg=structure!10}
   \begin{block}{{\bf Proposition }#1}
}{%
   \end{block}%
}

\newenvironment{corollaire}[1][]{%
   \setbeamercolor{block title}{fg=structure,bg=structure!40}
   \setbeamercolor{block body}{fg=black,bg=structure!10}
   \begin{block}{{\bf Corollaire }#1}
}{%
   \end{block}%
}

\newenvironment{mydefinition}[1][]{%
   \setbeamercolor{block title}{fg=structure,bg=structure!40}
   \setbeamercolor{block body}{fg=black,bg=structure!10}
   \begin{block}{{\bf Définition} #1}
}{%
   \end{block}%
}

\newenvironment{lemme}[0]{%
   \setbeamercolor{block title}{fg=structure,bg=structure!40}
   \setbeamercolor{block body}{fg=black,bg=structure!10}
   \begin{block}{\bf Lemme}
}{%
   \end{block}%
}

\newenvironment{remarque}[1][]{%
   \setbeamercolor{block title}{fg=black,bg=structure!20}
   \setbeamercolor{block body}{fg=black,bg=structure!5}
   \begin{block}{Remarque #1}
}{%
   \end{block}%
}


\newenvironment{exemple}[1][]{%
   \setbeamercolor{block title}{fg=black,bg=structure!20}
   \setbeamercolor{block body}{fg=black,bg=structure!5}
   \begin{block}{{\bf Exemple }#1}
}{%
   \end{block}%
}


\newenvironment{miniexercice}[0]{%
   \setbeamercolor{block title}{fg=structure,bg=structure!20}
   \setbeamercolor{block body}{fg=black,bg=structure!5}
   \begin{block}{Mini-exercices}
}{%
   \end{block}%
}


\newenvironment{tp}[0]{%
   \setbeamercolor{block title}{fg=structure,bg=structure!40}
   \setbeamercolor{block body}{fg=black,bg=structure!10}
   \begin{block}{\bf Travaux pratiques}
}{%
   \end{block}%
}
\newenvironment{exercicecours}[1][]{%
   \setbeamercolor{block title}{fg=structure,bg=structure!40}
   \setbeamercolor{block body}{fg=black,bg=structure!10}
   \begin{block}{{\bf Exercice }#1}
}{%
   \end{block}%
}
\newenvironment{algo}[1][]{%
   \setbeamercolor{block title}{fg=structure,bg=structure!40}
   \setbeamercolor{block body}{fg=black,bg=structure!10}
   \begin{block}{{\bf Algorithme}\hfill{\color{gray}\texttt{#1}}}
}{%
   \end{block}%
}


\setbeamertemplate{proof begin}{
   \setbeamercolor{block title}{fg=black,bg=structure!20}
   \setbeamercolor{block body}{fg=black,bg=structure!5}
   \begin{block}{{\footnotesize Démonstration}}
   \footnotesize
   \smallskip}
\setbeamertemplate{proof end}{%
   \end{block}}
\setbeamertemplate{qed symbol}{\openbox}


\makeatother
\usecolortheme[RGB={192,41,0}]{structure}

% Commande spécifique à ce chapitre

\newcommand{\Python}{\texttt{Python}}
\renewcommand{\evidence}[1]{{\color{blue}\textbf{#1}}}

\usepackage{textcomp}

\usepackage{listings}
\lstset{
  upquote=true,
  columns=flexible,
  keepspaces=true,
  basicstyle=\ttfamily,
  commentstyle=\color{gray},
  language=Python,
  showstringspaces=false,
  aboveskip=0em,  
  belowskip=0em,
  escapeinside=||
}

\lstset{
  literate={é}{{\'e}}1
           {è}{{\`e}}1
           {à}{{\`a}}1
}


\newcommand{\codeinline}[1]{\lstinline!#1!}



%%%%%%%%%%%%%%%%%%%%%%%%%%%%%%%%%%%%%%%%%%%%%%%%%%%%%%%%%%%%%
%%%%%%%%%%%%%%%%%%%%%%%%%%%%%%%%%%%%%%%%%%%%%%%%%%%%%%%%%%%%%


\begin{document}


\title{{\bf Algorithmes et mathématiques}}
\subtitle{Calculs de sinus, cosinus, tangente}

\begin{frame}
  
  \debutmontitre

  \pause

{\footnotesize
\hfill
\setbeamercovered{transparent=50}
\begin{minipage}{0.6\textwidth}
  \begin{itemize}
    \item<3-> Calcul de $\Arctan x$
    \item<4-> Calcul de $\tan x$
    \item<5-> Calcul de $\sin x$ et $\cos x$
  \end{itemize}
\end{minipage}
}

\end{frame}

\setcounter{framenumber}{0}


%%%%%%%%%%%%%%%%%%%%%%%%%%%%%%%%%%%%%%%%%%%%%%%%%%%%%%%%%%%%%%%%
\section{Calcul de $\Arctan x$}

\begin{frame}


\begin{itemize}
  \item $\Arctan (10^{-i})$, pour $i=0,\ldots,8$
\pause
  \item Angle \ $\theta_i \in \ ]-\frac\pi2,\frac\pi2[$ \ tels que $\tan \theta_i = 10^{-i}$
\end{itemize}

\pause

$$\Arctan x = \sum_{k=0}^{+\infty} (-1)^k\frac{x^{2k+1}}{2k+1} = 
x - \frac{x^3}{3}+ \frac{x^5}{5}-\frac{x^7}{7}+\cdots$$

\pause

\begin{tp}
\begin{enumerate}
  \item Calculer $\Arctan 1$.
  \item Calculer $\theta_i = \Arctan 10^{-i}$ (avec $8$ chiffres après la virgule) pour $i =1,\ldots,8$.
  \item Pour quelles valeurs de $i$, l'approximation $\Arctan x \simeq x$ était-elle suffisante ?
\end{enumerate}  
\end{tp}
\end{frame}


\begin{frame}[fragile]

\begin{algo}[tangente.py (1)]
\begin{lstlisting}
def mon_arctan(x,n):
    somme = 0
    for k in range(0,n+1):
        if (k%2 == 0):  # si k est pair signe +
            somme = somme + 1/(2*k+1) * (x ** (2*k+1))  
        else:           # si k est impair signe -
            somme = somme - 1/(2*k+1) * (x ** (2*k+1))
    return somme
\end{lstlisting}  
\end{algo}

\pause

\begin{itemize}
  \item 
  \begin{itemize}
    \item $\Arctan x= \sum_{k=0}^{+\infty} (-1)^k\frac{x^{2k+1}}{2k+1}= x - \frac{x^3}{3}+ \frac{x^5}{5}-\frac{x^7}{7}+\cdots$
\pause
    \item  $\tan \pi/4 = 1 \implies \Arctan 1 = \pi/4$   
\pause
    \item si $0 \le x \le \frac 1{10}$ alors $x^{2k+1} \le \frac{1}{10^{2k+1}}$
\pause
    \item pour $k \ge 4$,  $\left| (-1)^k\frac{x^{2k+1}}{2k+1} \right| < 10^{-9}$
  \end{itemize}

\pause
  \item $\Arctan x \simeq  x - \frac{x^3}{3}+ \frac{x^5}{5}-\frac{x^7}{7}$

\pause 
  \item Pour $i \ge 4$, $\Arctan x \simeq x$ donne déjà $8$ chiffres après la virgule !

\pause  
  \item Liste \codeinline{theta} contenant les angles $\theta_i$
\end{itemize}

\end{frame}




%%%%%%%%%%%%%%%%%%%%%%%%%%%%%%%%%%%%%%%%%%%%%%%%%%%%%%%%%%%%%%%%
\section{Calcul de $\tan x$}

\begin{frame}

\myfigure{1.5}{
\tikzinput{fig_algo04} 
}

\end{frame}

\begin{frame}
\myfigure{1}{
\tikzinput{fig_algo03} 
} 
\vspace*{-4ex}
\pause
\begin{itemize}
  \item $M(x,y)$ un point
\pause
  \item $N(x',y')$ image de $M$ par la rotation centrée à l'origine et d'angle $\theta$
\pause
 $$\begin{pmatrix} x' \\ y' \end{pmatrix}
= \begin{pmatrix} \cos \theta & - \sin \theta \\ \sin \theta & \cos \theta \end{pmatrix}
\begin{pmatrix} x \\y \end{pmatrix}
\qquad
\pause
\left\{ \begin{array}{rcl} 
        x' = x \cos \theta - y \sin \theta \\
        y' = x \sin \theta + y \cos \theta \\
        \end{array}
\right. $$
\pause
  \item $M'$ le point de la demi-droite $[ON)$ tel que  $(OM)\perp(MM')$ 
\end{itemize}
\end{frame}


\begin{frame}

\myfigure{0.6}{
\tikzinput{fig_algo03} \qquad \tikzinput{fig_algo04} \qquad \tikzinput{fig_algo04bis} 
}
\vspace*{-2ex}
\begin{tp}
\begin{enumerate}
  \item
  \begin{itemize}
    \item Calculer la longueur $OM'$. 
    \item En déduire les coordonnées de $M'$. 
    \item Exprimez-les uniquement en fonction de $x,y$ et $\tan \theta$.
  \end{itemize}
  
  \item Faire une boucle qui décompose l'angle $a$ en somme d'angles $\theta_i$ (à une précision de $10^{-8}$ ; 
  avec un minimum d'angles, les angles pouvant se répéter).
  
  \item Partant de $M_0 = (1,0)$ calculer les coordonnées des différents $M_k$, 
  jusqu'au point $M_n(x_n,y_n)$ correspondant à l'approximation de l'angle $a$. 
  Renvoyer la valeur $\frac{y_n}{x_n}$ comme approximation de $\tan a$.
\end{enumerate}  
\end{tp}
\end{frame}


\begin{frame}[fragile]

\begin{algo}[tangente.py (2)]
\begin{lstlisting}
i = 0
while (a > precision):   # boucle precision pas atteinte
    while (a < theta[i]):   # choix du bon angle theta_i
        i = i+1
    a = a - theta[i]         # on retire l'angle theta_i
\end{lstlisting}  
\end{algo}

\end{frame}


\begin{frame}

\myfigure{0.7}{
\tikzinput{fig_algo03} 
}
\vspace*{-4ex}
\pause
  \begin{itemize}
    \item $\cos \theta = \frac{OM}{OM'}$ \pause \quad donc $OM' = \frac{OM}{\cos \theta}$  \pause \quad ; \quad $OM=ON$
\pause    
    \item $M'(x'',y'')$ 
\end{itemize}
\pause
$$ \left\{ \begin{array}{l} 
        \vphantom{\displaystyle\int}x'' = \frac{1}{\cos \theta} x' 
        \uncover<7->{= \frac{1}{\cos \theta} \big( x \cos \theta - y \sin \theta \big)} 
        \uncover<9->{= x - y \tan \theta} \\
         y'' = \frac{1}{\cos \theta} y' 
         \uncover<8->{= \frac{1}{\cos \theta} \big(x \sin \theta + y \cos \theta \big)} 
         \uncover<10->{= x \tan \theta + y} \\
        \end{array}        
\right. $$ 
\pause\pause\pause\pause\pause
$$\begin{pmatrix} x'' \\ y'' \end{pmatrix}
= \begin{pmatrix} 1 & - \tan \theta \\ \tan \theta & 1 \end{pmatrix}
\begin{pmatrix} x \\y \end{pmatrix}$$
 
\end{frame}




\begin{frame}
\myfigure{1}{
\hfill\tikzinput{fig_algo04} 
}
\vspace*{-15mm}\pause
\begin{itemize}
  \item $x_0 = 1$, $y_0=0$ et $M_0=\begin{pmatrix}x_0\\y_0\end{pmatrix}$
\pause  
  \item $M_{k+1} = P(\theta_i) \cdot M_k$ \pause
   \quad où \quad $P(\theta) = \begin{pmatrix} 1 & - \tan \theta \\ \tan \theta & 1 \end{pmatrix}$
\pause   
  \item $\tan \theta_i = 10^{-i}$
\pause 
        $$ \left\{ \begin{array}{l} 
        x_{k+1} = x_k - y_k \cdot 10^{-i} \\
        y_{k+1} = x_k \cdot 10^{-i}+ y_k \\
        \end{array}
\right. $$ 
\pause
  \item $\frac{y_n}{x_n}$  est la tangente de la somme des $\theta_i$ 
  \pause : approximation de $\tan a$
\end{itemize}

\end{frame}


\begin{frame}[fragile]

\begin{algo}[tangente.py (3)]
\begin{lstlisting}
def ma_tan(a):
    precision = 10**(-9)
    i = 0 ; x = 1 ; y = 0   |\pause|
    while (a > precision):      
        while (a < theta[i]):   
            i = i+1               
        newa = a - theta[i]  # on retire l'angle theta_i     |\pause|
        newx = x - (10**(-i))*y          # nouveau point
        newy = (10**(-i))*x + y                                   |\pause|
        x = newx                              
        y = newy
        a = newa                                                  |\pause|
    return y/x                  # on renvoie la tangente
\end{lstlisting}  
\end{algo}

\end{frame}



%%%%%%%%%%%%%%%%%%%%%%%%%%%%%%%%%%%%%%%%%%%%%%%%%%%%%%%%%%%%%%%%
\section{Calcul de $\sin x$ et $\cos x$}


\begin{frame}


\begin{tp}
Pour $0 \le x \le \frac \pi2$, calculer $\sin x$ et $\cos x$ en fonction de $\tan x$.
En déduire comment calculer les sinus et cosinus de $x$.
\end{tp}

\pause
\bigskip

$$\cos x = \frac{1}{\sqrt{1+\tan^2 x}} \qquad
\pause \sin x = \frac{\tan x}{\sqrt{1+\tan^2 x}} \qquad
\pause 0 \le x \le \frac \pi2$$
\pause
$$\cos^2+\sin^2 x = 1 
\pause \implies 1+\tan^2 x = \frac{1}{\cos^2 x}
\pause \implies \cos x = \frac{1}{\sqrt{1+\tan^2 x}} $$

\end{frame}




%%%%%%%%%%%%%%%%%%%%%%%%%%%%%%%%%%%%%%%%%%%%%%%%%%%%%%%%%%%%%%%%
\section{Mini-exercices}

\begin{frame}

\begin{miniexercice}
\begin{enumerate}
 
  \item On dispose de billets de $1$, $5$, $20$ et $100$ euros. Trouvez la façon de payer une somme de
  $n$ euros avec le minimum de billets.
 
  \item Faire un programme qui pour \emph{n'importe quel} $x \in \Rr$, calcule $\sin x$, $\cos x$, $\tan x$. 
  
  \item Pour $t = \tan \frac x2$ montrer que $\tan x = \frac{2t}{1-t^2}$. 
  En déduire une fonction qui calcule $\tan x$. (Utiliser que pour $x$ assez petit $\tan x \simeq x$).
  
  \item Modifier l'algorithme de la tangente pour qu'il calcule aussi directement
  le sinus et le cosinus.

\end{enumerate}

\end{miniexercice}

\end{frame}

\end{document}