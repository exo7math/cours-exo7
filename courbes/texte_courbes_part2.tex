
%%%%%%%%%%%%%%%%%% PREAMBULE %%%%%%%%%%%%%%%%%%


\documentclass[12pt]{article}

\usepackage{amsfonts,amsmath,amssymb,amsthm}
\usepackage[utf8]{inputenc}
\usepackage[T1]{fontenc}
\usepackage[francais]{babel}


% packages
\usepackage{amsfonts,amsmath,amssymb,amsthm}
\usepackage[utf8]{inputenc}
\usepackage[T1]{fontenc}
%\usepackage{lmodern}

\usepackage[francais]{babel}
\usepackage{fancybox}
\usepackage{graphicx}

\usepackage{float}

%\usepackage[usenames, x11names]{xcolor}
\usepackage{tikz}
\usepackage{datetime}

\usepackage{mathptmx}
%\usepackage{fouriernc}
%\usepackage{newcent}
\usepackage[mathcal,mathbf]{euler}

%\usepackage{palatino}
%\usepackage{newcent}


% Commande spéciale prompteur

%\usepackage{mathptmx}
%\usepackage[mathcal,mathbf]{euler}
%\usepackage{mathpple,multido}

\usepackage[a4paper]{geometry}
\geometry{top=2cm, bottom=2cm, left=1cm, right=1cm, marginparsep=1cm}

\newcommand{\change}{{\color{red}\rule{\textwidth}{1mm}\\}}

\newcounter{mydiapo}

\newcommand{\diapo}{\newpage
\hfill {\normalsize  Diapo \themydiapo \quad \texttt{[\jobname]}} \\
\stepcounter{mydiapo}}


%%%%%%% COULEURS %%%%%%%%%%

% Pour blanc sur noir :
%\pagecolor[rgb]{0.5,0.5,0.5}
% \pagecolor[rgb]{0,0,0}
% \color[rgb]{1,1,1}



%\DeclareFixedFont{\myfont}{U}{cmss}{bx}{n}{18pt}
\newcommand{\debuttexte}{
%%%%%%%%%%%%% FONTES %%%%%%%%%%%%%
\renewcommand{\baselinestretch}{1.5}
\usefont{U}{cmss}{bx}{n}
\bfseries

% Taille normale : commenter le reste !
%Taille Arnaud
%\fontsize{19}{19}\selectfont

% Taille Barbara
%\fontsize{21}{22}\selectfont

%Taille François
\fontsize{25}{30}\selectfont

%Taille Pascal
%\fontsize{25}{30}\selectfont

%Taille Laura
%\fontsize{30}{35}\selectfont


%\myfont
%\usefont{U}{cmss}{bx}{n}

%\Huge
%\addtolength{\parskip}{\baselineskip}
}


% \usepackage{hyperref}
% \hypersetup{colorlinks=true, linkcolor=blue, urlcolor=blue,
% pdftitle={Exo7 - Exercices de mathématiques}, pdfauthor={Exo7}}


%section
% \usepackage{sectsty}
% \allsectionsfont{\bf}
%\sectionfont{\color{Tomato3}\upshape\selectfont}
%\subsectionfont{\color{Tomato4}\upshape\selectfont}

%----- Ensembles : entiers, reels, complexes -----
\newcommand{\Nn}{\mathbb{N}} \newcommand{\N}{\mathbb{N}}
\newcommand{\Zz}{\mathbb{Z}} \newcommand{\Z}{\mathbb{Z}}
\newcommand{\Qq}{\mathbb{Q}} \newcommand{\Q}{\mathbb{Q}}
\newcommand{\Rr}{\mathbb{R}} \newcommand{\R}{\mathbb{R}}
\newcommand{\Cc}{\mathbb{C}} 
\newcommand{\Kk}{\mathbb{K}} \newcommand{\K}{\mathbb{K}}

%----- Modifications de symboles -----
\renewcommand{\epsilon}{\varepsilon}
\renewcommand{\Re}{\mathop{\text{Re}}\nolimits}
\renewcommand{\Im}{\mathop{\text{Im}}\nolimits}
%\newcommand{\llbracket}{\left[\kern-0.15em\left[}
%\newcommand{\rrbracket}{\right]\kern-0.15em\right]}

\renewcommand{\ge}{\geqslant}
\renewcommand{\geq}{\geqslant}
\renewcommand{\le}{\leqslant}
\renewcommand{\leq}{\leqslant}

%----- Fonctions usuelles -----
\newcommand{\ch}{\mathop{\mathrm{ch}}\nolimits}
\newcommand{\sh}{\mathop{\mathrm{sh}}\nolimits}
\renewcommand{\tanh}{\mathop{\mathrm{th}}\nolimits}
\newcommand{\cotan}{\mathop{\mathrm{cotan}}\nolimits}
\newcommand{\Arcsin}{\mathop{\mathrm{Arcsin}}\nolimits}
\newcommand{\Arccos}{\mathop{\mathrm{Arccos}}\nolimits}
\newcommand{\Arctan}{\mathop{\mathrm{Arctan}}\nolimits}
\newcommand{\Argsh}{\mathop{\mathrm{Argsh}}\nolimits}
\newcommand{\Argch}{\mathop{\mathrm{Argch}}\nolimits}
\newcommand{\Argth}{\mathop{\mathrm{Argth}}\nolimits}
\newcommand{\pgcd}{\mathop{\mathrm{pgcd}}\nolimits} 

\newcommand{\Card}{\mathop{\text{Card}}\nolimits}
\newcommand{\Ker}{\mathop{\text{Ker}}\nolimits}
\newcommand{\id}{\mathop{\text{id}}\nolimits}
\newcommand{\ii}{\mathrm{i}}
\newcommand{\dd}{\mathrm{d}}
\newcommand{\Vect}{\mathop{\text{Vect}}\nolimits}
\newcommand{\Mat}{\mathop{\mathrm{Mat}}\nolimits}
\newcommand{\rg}{\mathop{\text{rg}}\nolimits}
\newcommand{\tr}{\mathop{\text{tr}}\nolimits}
\newcommand{\ppcm}{\mathop{\text{ppcm}}\nolimits}

%----- Structure des exercices ------

\newtheoremstyle{styleexo}% name
{2ex}% Space above
{3ex}% Space below
{}% Body font
{}% Indent amount 1
{\bfseries} % Theorem head font
{}% Punctuation after theorem head
{\newline}% Space after theorem head 2
{}% Theorem head spec (can be left empty, meaning ‘normal’)

%\theoremstyle{styleexo}
\newtheorem{exo}{Exercice}
\newtheorem{ind}{Indications}
\newtheorem{cor}{Correction}


\newcommand{\exercice}[1]{} \newcommand{\finexercice}{}
%\newcommand{\exercice}[1]{{\tiny\texttt{#1}}\vspace{-2ex}} % pour afficher le numero absolu, l'auteur...
\newcommand{\enonce}{\begin{exo}} \newcommand{\finenonce}{\end{exo}}
\newcommand{\indication}{\begin{ind}} \newcommand{\finindication}{\end{ind}}
\newcommand{\correction}{\begin{cor}} \newcommand{\fincorrection}{\end{cor}}

\newcommand{\noindication}{\stepcounter{ind}}
\newcommand{\nocorrection}{\stepcounter{cor}}

\newcommand{\fiche}[1]{} \newcommand{\finfiche}{}
\newcommand{\titre}[1]{\centerline{\large \bf #1}}
\newcommand{\addcommand}[1]{}
\newcommand{\video}[1]{}

% Marge
\newcommand{\mymargin}[1]{\marginpar{{\small #1}}}



%----- Presentation ------
\setlength{\parindent}{0cm}

%\newcommand{\ExoSept}{\href{http://exo7.emath.fr}{\textbf{\textsf{Exo7}}}}

\definecolor{myred}{rgb}{0.93,0.26,0}
\definecolor{myorange}{rgb}{0.97,0.58,0}
\definecolor{myyellow}{rgb}{1,0.86,0}

\newcommand{\LogoExoSept}[1]{  % input : echelle
{\usefont{U}{cmss}{bx}{n}
\begin{tikzpicture}[scale=0.1*#1,transform shape]
  \fill[color=myorange] (0,0)--(4,0)--(4,-4)--(0,-4)--cycle;
  \fill[color=myred] (0,0)--(0,3)--(-3,3)--(-3,0)--cycle;
  \fill[color=myyellow] (4,0)--(7,4)--(3,7)--(0,3)--cycle;
  \node[scale=5] at (3.5,3.5) {Exo7};
\end{tikzpicture}}
}



\theoremstyle{definition}
%\newtheorem{proposition}{Proposition}
%\newtheorem{exemple}{Exemple}
%\newtheorem{theoreme}{Théorème}
\newtheorem{lemme}{Lemme}
\newtheorem{corollaire}{Corollaire}
%\newtheorem*{remarque*}{Remarque}
%\newtheorem*{miniexercice}{Mini-exercices}
%\newtheorem{definition}{Définition}




%definition d'un terme
\newcommand{\defi}[1]{{\color{myorange}\textbf{\emph{#1}}}}
\newcommand{\evidence}[1]{{\color{blue}\textbf{\emph{#1}}}}



 %----- Commandes divers ------

\newcommand{\codeinline}[1]{\texttt{#1}}

%%%%%%%%%%%%%%%%%%%%%%%%%%%%%%%%%%%%%%%%%%%%%%%%%%%%%%%%%%%%%
%%%%%%%%%%%%%%%%%%%%%%%%%%%%%%%%%%%%%%%%%%%%%%%%%%%%%%%%%%%%%


\begin{document}

\debuttexte


%%%%%%%%%%%%%%%%%%%%%%%%%%%%%%%%%%%%%%%%%%%%%%%%%%%%%%%%%%%
\diapo

Dans cette deuxième partie sur les courbes paramétrées, 
nous allons étudier la notion de tangente.

\change

Voici les points que nous allons aborder.

\change

Nous définirons la tangente,

\change

puis le vecteur dérivé,

\change 

ce qui nous permettra de montrer l'existence d'une tangente en un point régulier.

\change

Nous conclurons cette partie par des formules 
de dérivation d'expressions vectorielles faisant intervenir le vecteur dérivé.


%%%%%%%%%%%%%%%%%%%%%%%%%%%%%%%%%%%%%%%%%%%%%%%%%%%%%%%%%%%
\diapo

On se donne une courbe paramétrée et un point $t_0$ du domaine de définition $D$. 

\change
On cherche à définir la tangente à un point de paramètre $t_0$.

\change

On doit déjà prendre garde au fait que lorsque ce point $M(t_0)$ 
est un point multiple de la courbe, alors la courbe peut 
tout à fait avoir plusieurs tangentes en ce point.

\change

Pour éviter cela, on supposera que la courbe est  \defi{localement simple en $t_0$}, 

\change
c'est-à-dire qu'il existe un intervalle 
 $I$ tel que l'équation $M(t)=M(t_0)$ 
admette une seule solution dans $D\cap I$, comme sur cette figure [gauche].  

Alors qu'ici [droite], il y aurait deux solutions distinctes.


Dans toute la suite, nous supposerons 
systématiquement que cette condition "localement simple" est réalisée.



%%%%%%%%%%%%%%%%%%%%%%%%%%%%%%%%%%%%%%%%%%%%%%%%%%%%%%%%%%%
\diapo

Voici la définition de tangente : une droite est tangente en $M(t_0)$ 
à la courbe paramétrée si c'est la limite des droites sécantes 
$(M(t_0)M(t))$ à la courbe, lorsque quand $t$ tend vers $t_0$. 

Géométriquement pour définir cette tangente, on prend une succession 
de sécantes passant par $M(t_0)$ et un autre point $M(t)$ que l'on fait tendre vers $M(t_0)$.

\change
\change
\change
\change
\change

%%%%%%%%%%%%%%%%%%%%%%%%%%%%%%%%%%%%%%%%%%%%%%%%%%%%%%%%%%%
\diapo


Pour déterminer la tangente à une courbe paramétrée en un point, 
il est naturel de regarder le \og{}vecteur vitesse\fg{} en ce point, 
qui d'un point de vue mathématique est le \emph{vecteur dérivé}. 

\change
La courbe est dite \defi{dérivable en $t_0$} si les fonctions 
$x$ et $y$ le sont.

\change

Le vecteur dérivé s'obtient simplement en dérivant individuellement les 
fonctions coordonnées $x(t)$ et $y(t)$ définissant la courbe paramétrée. 


\change

On le notera par la suite 
$\overrightarrow{\dfrac{\dd M}{\dd t}}(t_0)$.

\change

On obtient ainsi un vecteur qui, s'il est non nul, 
est tangent à la courbe en $M(t_0)$.


%%%%%%%%%%%%%%%%%%%%%%%%%%%%%%%%%%%%%%%%%%%%%%%%%%%%%%%%%%%
\diapo

Il y donc lieu de distinguer les points de la courbe où le 
vecteur dérivé s'annule des autres. C'est le sens de la définition suivante : 
on appelle point \emph{régulier} un point de la courbe paramétrée 
où le vecteur dérivé n'est pas le vecteur nul.

\change
On appelle point \emph{singulier} un point où le vecteur 
dérivé s'annule.  


\change
La courbe est dite \emph{régulière} si tous ses points le sont.


L'interprétation cinématique est la suivante : si $t$ est le temps,  
le vecteur dérivé $\overrightarrow{\frac{\dd M}{\dd t}}(t_0)$ est  
le \emph{vecteur vitesse} au point $M(t_0)$.  

Un point singulier, c'est-à-dire un point en lequel 
la vitesse est nulle, s’appellera alors aussi un \emph{point stationnaire}.


%%%%%%%%%%%%%%%%%%%%%%%%%%%%%%%%%%%%%%%%%%%%%%%%%%%%%%%%%%%
\diapo

D'un point de vue cinématique, il est intuitif de penser que 
le vecteur vitesse en un point, quand il est non nul, 
dirige la tangente à la trajectoire en ce point.

C'est ce qu'exprime le théorème suivant : 
en un point régulier, une courbe paramétrée admet une tangente, 

et bien sûr, un vecteur directeur de la tangente est donné 
par le vecteur dérivé en ce point.


\change

Comme on connaît un point $M(t_0)$ et un vecteur directeur de la tangente, 
on obtient facilement une équation cartésienne de la tangente en écrivant 
que pour un point $M$ de coordonnées $(x,y)$ de la tangente, 
le vecteur $\overrightarrow{M(t_0)M}$ est parallèle au vecteur dérivé, ce qui s'écrit ainsi :
le déterminant du vecteur $\overrightarrow{M(t_0)M}$ et du 
vecteur vitesse doit être nul. 


%%%%%%%%%%%%%%%%%%%%%%%%%%%%%%%%%%%%%%%%%%%%%%%%%%%%%%%%%%%
\diapo

On considère l'exemple de la courbe de Lissajous, 
dont on a déjà vu l'équation paramétrique, et dont on va 
chercher les points où la tangente est verticale ou horizontale.

\change

Tout d'abord, par symétrie, on limite notre étude sur $t\in[0,\frac\pi2]$.
[montrer bleu foncé]

\change

Or au point $M(t)$, le vecteur dérivé est
$\overrightarrow{\frac{\dd M}{\dd t}}
= \left(\begin{matrix} x'(t) \\ y'(t) \end{matrix}\right)
= \left(\begin{matrix} 2\cos(2t) \\ 3\cos(3t)\end{matrix}\right).$


\change

et on trouve immédiatement que 

$x'(t) = 0 \iff t = \frac\pi4$ et

\change
Donc il y a un tangente verticale en $M(\frac\pi4)$ et c'est l'unique 
tangente verticale sur $[0,\frac\pi2]$

\change
Comme
$y'(t) = 0 \iff t \in \left\{ \frac\pi6  ,\frac\pi2\right\} $

\change
il y une tangente horizontale en $M(\frac\pi6)$ 
et en $M(\frac\pi2)$.

%\change
On obtiendrait les autres tangentes horizontales et verticales par symétrie.


%%%%%%%%%%%%%%%%%%%%%%%%%%%%%%%%%%%%%%%%%%%%%%%%%%%%%%%%%%%
\diapo

Passons aux expressions vectorielles.

Voyons comment dériver le produit scalaire de deux fonctions vectorielles ainsi
que la norme d'une fonction vectorielle.

On fixe donc deux fonctions vectorielles $f$ et $g$, c'est-à-dire des fonctions définie sur $\Rr$
mais à valeurs dans $\Rr^2$.

On suppose qu'elles sont dérivables en un point $t_0$.

\change

Le produit scalaire de ces deux fonctions est donc une fonction à valeurs 
réelles.

Le théorème affirme qu'on dérive un produit scalaire 
comme on dérive un produit usuel de deux fonctions : 
la dérivée du produit scalaire de $f$ et $g$ est le 
produit scalaire de la dérivée de $f$ avec  $g$ *plus* le produit 
scalaire de $f$ avec la dérivée de $g$. 

\change

Voici la formule donnant la dérivée de la norme d'une fonction vectorielle, 
en un point où la fonction est non nulle.

La dérivée de la norme de $f$ est le produit scalaire de $f$ avec 
sa dérivée, divisé par la norme de $f$.


La première formule, se démontre aisément 
en utilisant l'expression analytique du produit scalaire.

La seconde  est une conséquence directe de la première, 
car la norme est la racine carrée du produit scalaire de $f$ avec 
elle même.

%%%%%%%%%%%%%%%%%%%%%%%%%%%%%%%%%%%%%%%%%%%%%%%%%%%%%%%%%%%
\diapo

Voici un exemple : on considère la paramétrisation usuelle 
du cercle trigonométrique donnée par $t\mapsto M(t)=(\cos t,\sin t)$. 

\change
Un point du cercle vérifie que la distance du centre au point est égale au rayon,
c-à-d $\|\overrightarrow{OM(t)}\|=1$.

\change

En dérivant l'expression $\|\overrightarrow{OM(t)}\|=1$, on obtient

$$\big\langle \overrightarrow{OM(t)} \mid \overrightarrow{\frac{\dd M}{\dd t}}(t) \big\rangle =0$$


\change

Cette égalité a une interprétation géométrique évidente : 
la tangente au cercle est en chaque point orthogonale au rayon.

%%%%%%%%%%%%%%%%%%%%%%%%%%%%%%%%%%%%%%%%%%%%%%%%%%%%%%%%%%%
\diapo

Voici quelques exercices pour vous entrainer sur les notions que nous avons abordées dans cette deuxième partie.


\end{document}
