
%%%%%%%%%%%%%%%%%% PREAMBULE %%%%%%%%%%%%%%%%%%

\documentclass[aspectratio=169,utf8]{beamer}
%\documentclass[aspectratio=169,handout]{beamer}

\usetheme{Boadilla}
%\usecolortheme{seahorse}
\usecolortheme[RGB={245,66,24}]{structure}
\useoutertheme{infolines}

% packages
\usepackage{amsfonts,amsmath,amssymb,amsthm}
\usepackage[utf8]{inputenc}
\usepackage[T1]{fontenc}
\usepackage{lmodern}

\usepackage[francais]{babel}
\usepackage{fancybox}
\usepackage{graphicx}

\usepackage{float}
\usepackage{xfrac}

%\usepackage[usenames, x11names]{xcolor}
\usepackage{tikz}
\usepackage{pgfplots}
\usepackage{datetime}



%-----  Package unités -----
\usepackage{siunitx}
\sisetup{locale = FR,detect-all,per-mode = symbol}

%\usepackage{mathptmx}
%\usepackage{fouriernc}
%\usepackage{newcent}
%\usepackage[mathcal,mathbf]{euler}

%\usepackage{palatino}
%\usepackage{newcent}
% \usepackage[mathcal,mathbf]{euler}



% \usepackage{hyperref}
% \hypersetup{colorlinks=true, linkcolor=blue, urlcolor=blue,
% pdftitle={Exo7 - Exercices de mathématiques}, pdfauthor={Exo7}}


%section
% \usepackage{sectsty}
% \allsectionsfont{\bf}
%\sectionfont{\color{Tomato3}\upshape\selectfont}
%\subsectionfont{\color{Tomato4}\upshape\selectfont}

%----- Ensembles : entiers, reels, complexes -----
\newcommand{\Nn}{\mathbb{N}} \newcommand{\N}{\mathbb{N}}
\newcommand{\Zz}{\mathbb{Z}} \newcommand{\Z}{\mathbb{Z}}
\newcommand{\Qq}{\mathbb{Q}} \newcommand{\Q}{\mathbb{Q}}
\newcommand{\Rr}{\mathbb{R}} \newcommand{\R}{\mathbb{R}}
\newcommand{\Cc}{\mathbb{C}} 
\newcommand{\Kk}{\mathbb{K}} \newcommand{\K}{\mathbb{K}}

%----- Modifications de symboles -----
\renewcommand{\epsilon}{\varepsilon}
\renewcommand{\Re}{\mathop{\text{Re}}\nolimits}
\renewcommand{\Im}{\mathop{\text{Im}}\nolimits}
%\newcommand{\llbracket}{\left[\kern-0.15em\left[}
%\newcommand{\rrbracket}{\right]\kern-0.15em\right]}

\renewcommand{\ge}{\geqslant}
\renewcommand{\geq}{\geqslant}
\renewcommand{\le}{\leqslant}
\renewcommand{\leq}{\leqslant}
\renewcommand{\epsilon}{\varepsilon}

%----- Fonctions usuelles -----
\newcommand{\ch}{\mathop{\text{ch}}\nolimits}
\newcommand{\sh}{\mathop{\text{sh}}\nolimits}
\renewcommand{\tanh}{\mathop{\text{th}}\nolimits}
\newcommand{\cotan}{\mathop{\text{cotan}}\nolimits}
\newcommand{\Arcsin}{\mathop{\text{arcsin}}\nolimits}
\newcommand{\Arccos}{\mathop{\text{arccos}}\nolimits}
\newcommand{\Arctan}{\mathop{\text{arctan}}\nolimits}
\newcommand{\Argsh}{\mathop{\text{argsh}}\nolimits}
\newcommand{\Argch}{\mathop{\text{argch}}\nolimits}
\newcommand{\Argth}{\mathop{\text{argth}}\nolimits}
\newcommand{\pgcd}{\mathop{\text{pgcd}}\nolimits} 


%----- Commandes divers ------
\newcommand{\ii}{\mathrm{i}}
\newcommand{\dd}{\text{d}}
\newcommand{\id}{\mathop{\text{id}}\nolimits}
\newcommand{\Ker}{\mathop{\text{Ker}}\nolimits}
\newcommand{\Card}{\mathop{\text{Card}}\nolimits}
\newcommand{\Vect}{\mathop{\text{Vect}}\nolimits}
\newcommand{\Mat}{\mathop{\text{Mat}}\nolimits}
\newcommand{\rg}{\mathop{\text{rg}}\nolimits}
\newcommand{\tr}{\mathop{\text{tr}}\nolimits}


%----- Structure des exercices ------

\newtheoremstyle{styleexo}% name
{2ex}% Space above
{3ex}% Space below
{}% Body font
{}% Indent amount 1
{\bfseries} % Theorem head font
{}% Punctuation after theorem head
{\newline}% Space after theorem head 2
{}% Theorem head spec (can be left empty, meaning ‘normal’)

%\theoremstyle{styleexo}
\newtheorem{exo}{Exercice}
\newtheorem{ind}{Indications}
\newtheorem{cor}{Correction}


\newcommand{\exercice}[1]{} \newcommand{\finexercice}{}
%\newcommand{\exercice}[1]{{\tiny\texttt{#1}}\vspace{-2ex}} % pour afficher le numero absolu, l'auteur...
\newcommand{\enonce}{\begin{exo}} \newcommand{\finenonce}{\end{exo}}
\newcommand{\indication}{\begin{ind}} \newcommand{\finindication}{\end{ind}}
\newcommand{\correction}{\begin{cor}} \newcommand{\fincorrection}{\end{cor}}

\newcommand{\noindication}{\stepcounter{ind}}
\newcommand{\nocorrection}{\stepcounter{cor}}

\newcommand{\fiche}[1]{} \newcommand{\finfiche}{}
\newcommand{\titre}[1]{\centerline{\large \bf #1}}
\newcommand{\addcommand}[1]{}
\newcommand{\video}[1]{}

% Marge
\newcommand{\mymargin}[1]{\marginpar{{\small #1}}}

\def\noqed{\renewcommand{\qedsymbol}{}}


%----- Presentation ------
\setlength{\parindent}{0cm}

%\newcommand{\ExoSept}{\href{http://exo7.emath.fr}{\textbf{\textsf{Exo7}}}}

\definecolor{myred}{rgb}{0.93,0.26,0}
\definecolor{myorange}{rgb}{0.97,0.58,0}
\definecolor{myyellow}{rgb}{1,0.86,0}

\newcommand{\LogoExoSept}[1]{  % input : echelle
{\usefont{U}{cmss}{bx}{n}
\begin{tikzpicture}[scale=0.1*#1,transform shape]
  \fill[color=myorange] (0,0)--(4,0)--(4,-4)--(0,-4)--cycle;
  \fill[color=myred] (0,0)--(0,3)--(-3,3)--(-3,0)--cycle;
  \fill[color=myyellow] (4,0)--(7,4)--(3,7)--(0,3)--cycle;
  \node[scale=5] at (3.5,3.5) {Exo7};
\end{tikzpicture}}
}


\newcommand{\debutmontitre}{
  \author{} \date{} 
  \thispagestyle{empty}
  \hspace*{-10ex}
  \begin{minipage}{\textwidth}
    \titlepage  
  \vspace*{-2.5cm}
  \begin{center}
    \LogoExoSept{2.5}
  \end{center}
  \end{minipage}

  \vspace*{-0cm}
  
  % Astuce pour que le background ne soit pas discrétisé lors de la conversion pdf -> png
\begin{tikzpicture}
        \fill[opacity=0,green!60!black] (0,0)--++(0,0)--++(0,0)--++(0,0)--cycle; 
\end{tikzpicture}

% toc S'affiche trop tot :
% \tableofcontents[hideallsubsections, pausesections]
}

\newcommand{\finmontitre}{
  \end{frame}
  \setcounter{framenumber}{0}
} % ne marche pas pour une raison obscure

%----- Commandes supplementaires ------

% \usepackage[landscape]{geometry}
% \geometry{top=1cm, bottom=3cm, left=2cm, right=10cm, marginparsep=1cm
% }
% \usepackage[a4paper]{geometry}
% \geometry{top=2cm, bottom=2cm, left=2cm, right=2cm, marginparsep=1cm
% }

%\usepackage{standalone}


% New command Arnaud -- november 2011
\setbeamersize{text margin left=24ex}
% si vous modifier cette valeur il faut aussi
% modifier le decalage du titre pour compenser
% (ex : ici =+10ex, titre =-5ex

\theoremstyle{definition}
%\newtheorem{proposition}{Proposition}
%\newtheorem{exemple}{Exemple}
%\newtheorem{theoreme}{Théorème}
%\newtheorem{lemme}{Lemme}
%\newtheorem{corollaire}{Corollaire}
%\newtheorem*{remarque*}{Remarque}
%\newtheorem*{miniexercice}{Mini-exercices}
%\newtheorem{definition}{Définition}

% Commande tikz
\usetikzlibrary{calc}
\usetikzlibrary{patterns,arrows}
\usetikzlibrary{matrix}
\usetikzlibrary{fadings} 

%definition d'un terme
\newcommand{\defi}[1]{{\color{myorange}\textbf{\emph{#1}}}}
\newcommand{\evidence}[1]{{\color{blue}\textbf{\emph{#1}}}}
\newcommand{\assertion}[1]{\emph{\og#1\fg}}  % pour chapitre logique
%\renewcommand{\contentsname}{Sommaire}
\renewcommand{\contentsname}{}
\setcounter{tocdepth}{2}



%------ Figures ------

\def\myscale{1} % par défaut 
\newcommand{\myfigure}[2]{  % entrée : echelle, fichier figure
\def\myscale{#1}
\begin{center}
\footnotesize
{#2}
\end{center}}


%------ Encadrement ------

\usepackage{fancybox}


\newcommand{\mybox}[1]{
\setlength{\fboxsep}{7pt}
\begin{center}
\shadowbox{#1}
\end{center}}

\newcommand{\myboxinline}[1]{
\setlength{\fboxsep}{5pt}
\raisebox{-10pt}{
\shadowbox{#1}
}
}

%--------------- Commande beamer---------------
\newcommand{\beameronly}[1]{#1} % permet de mettre des pause dans beamer pas dans poly


\setbeamertemplate{navigation symbols}{}
\setbeamertemplate{footline}  % tiré du fichier beamerouterinfolines.sty
{
  \leavevmode%
  \hbox{%
  \begin{beamercolorbox}[wd=.333333\paperwidth,ht=2.25ex,dp=1ex,center]{author in head/foot}%
    % \usebeamerfont{author in head/foot}\insertshortauthor%~~(\insertshortinstitute)
    \usebeamerfont{section in head/foot}{\bf\insertshorttitle}
  \end{beamercolorbox}%
  \begin{beamercolorbox}[wd=.333333\paperwidth,ht=2.25ex,dp=1ex,center]{title in head/foot}%
    \usebeamerfont{section in head/foot}{\bf\insertsectionhead}
  \end{beamercolorbox}%
  \begin{beamercolorbox}[wd=.333333\paperwidth,ht=2.25ex,dp=1ex,right]{date in head/foot}%
    % \usebeamerfont{date in head/foot}\insertshortdate{}\hspace*{2em}
    \insertframenumber{} / \inserttotalframenumber\hspace*{2ex} 
  \end{beamercolorbox}}%
  \vskip0pt%
}


\definecolor{mygrey}{rgb}{0.5,0.5,0.5}
\setlength{\parindent}{0cm}
%\DeclareTextFontCommand{\helvetica}{\fontfamily{phv}\selectfont}

% background beamer
\definecolor{couleurhaut}{rgb}{0.85,0.9,1}  % creme
\definecolor{couleurmilieu}{rgb}{1,1,1}  % vert pale
\definecolor{couleurbas}{rgb}{0.85,0.9,1}  % blanc
\setbeamertemplate{background canvas}[vertical shading]%
[top=couleurhaut,middle=couleurmilieu,midpoint=0.4,bottom=couleurbas] 
%[top=fondtitre!05,bottom=fondtitre!60]



\makeatletter
\setbeamertemplate{theorem begin}
{%
  \begin{\inserttheoremblockenv}
  {%
    \inserttheoremheadfont
    \inserttheoremname
    \inserttheoremnumber
    \ifx\inserttheoremaddition\@empty\else\ (\inserttheoremaddition)\fi%
    \inserttheorempunctuation
  }%
}
\setbeamertemplate{theorem end}{\end{\inserttheoremblockenv}}

\newenvironment{theoreme}[1][]{%
   \setbeamercolor{block title}{fg=structure,bg=structure!40}
   \setbeamercolor{block body}{fg=black,bg=structure!10}
   \begin{block}{{\bf Th\'eor\`eme }#1}
}{%
   \end{block}%
}


\newenvironment{proposition}[1][]{%
   \setbeamercolor{block title}{fg=structure,bg=structure!40}
   \setbeamercolor{block body}{fg=black,bg=structure!10}
   \begin{block}{{\bf Proposition }#1}
}{%
   \end{block}%
}

\newenvironment{corollaire}[1][]{%
   \setbeamercolor{block title}{fg=structure,bg=structure!40}
   \setbeamercolor{block body}{fg=black,bg=structure!10}
   \begin{block}{{\bf Corollaire }#1}
}{%
   \end{block}%
}

\newenvironment{mydefinition}[1][]{%
   \setbeamercolor{block title}{fg=structure,bg=structure!40}
   \setbeamercolor{block body}{fg=black,bg=structure!10}
   \begin{block}{{\bf Définition} #1}
}{%
   \end{block}%
}

\newenvironment{lemme}[0]{%
   \setbeamercolor{block title}{fg=structure,bg=structure!40}
   \setbeamercolor{block body}{fg=black,bg=structure!10}
   \begin{block}{\bf Lemme}
}{%
   \end{block}%
}

\newenvironment{remarque}[1][]{%
   \setbeamercolor{block title}{fg=black,bg=structure!20}
   \setbeamercolor{block body}{fg=black,bg=structure!5}
   \begin{block}{Remarque #1}
}{%
   \end{block}%
}


\newenvironment{exemple}[1][]{%
   \setbeamercolor{block title}{fg=black,bg=structure!20}
   \setbeamercolor{block body}{fg=black,bg=structure!5}
   \begin{block}{{\bf Exemple }#1}
}{%
   \end{block}%
}


\newenvironment{miniexercice}[0]{%
   \setbeamercolor{block title}{fg=structure,bg=structure!20}
   \setbeamercolor{block body}{fg=black,bg=structure!5}
   \begin{block}{Mini-exercices}
}{%
   \end{block}%
}


\newenvironment{tp}[0]{%
   \setbeamercolor{block title}{fg=structure,bg=structure!40}
   \setbeamercolor{block body}{fg=black,bg=structure!10}
   \begin{block}{\bf Travaux pratiques}
}{%
   \end{block}%
}
\newenvironment{exercicecours}[1][]{%
   \setbeamercolor{block title}{fg=structure,bg=structure!40}
   \setbeamercolor{block body}{fg=black,bg=structure!10}
   \begin{block}{{\bf Exercice }#1}
}{%
   \end{block}%
}
\newenvironment{algo}[1][]{%
   \setbeamercolor{block title}{fg=structure,bg=structure!40}
   \setbeamercolor{block body}{fg=black,bg=structure!10}
   \begin{block}{{\bf Algorithme}\hfill{\color{gray}\texttt{#1}}}
}{%
   \end{block}%
}


\setbeamertemplate{proof begin}{
   \setbeamercolor{block title}{fg=black,bg=structure!20}
   \setbeamercolor{block body}{fg=black,bg=structure!5}
   \begin{block}{{\footnotesize Démonstration}}
   \footnotesize
   \smallskip}
\setbeamertemplate{proof end}{%
   \end{block}}
\setbeamertemplate{qed symbol}{\openbox}


\makeatother
\usecolortheme[RGB={102,102,0}]{structure}
   
%%%%%%%%%%%%%%%%%%%%%%%%%%%%%%%%%%%%%%%%%%%%%%%%%%%%%%%%%%%%%
%%%%%%%%%%%%%%%%%%%%%%%%%%%%%%%%%%%%%%%%%%%%%%%%%%%%%%%%%%%%%


\begin{document}


\title{{\bf Courbes paramétrées}}
\subtitle{Tangente}

\begin{frame}
  
  \debutmontitre

  \pause

{\footnotesize
\hfill
\setbeamercovered{transparent=50}
\begin{minipage}{0.6\textwidth}
  \begin{itemize}
    \item<3-> Tangente à une courbe
    \item<4-> Vecteur dérivé
    \item<5-> Tangente en un point régulier
    \item<6-> Dérivation d'expressions usuelles   
  \end{itemize}
\end{minipage}
}

\end{frame}

\setcounter{framenumber}{0}



%%%%%%%%%%%%%%%%%%%%%%%%%%%%%%%%%%%%%%%%%%%%%%%%%%%%%%%%%%%%%%%%
\section{Tangente à une courbe}



\begin{frame}

\begin{itemize}
  \item Soit $f : t\mapsto M(t)$ une courbe
  \item Soit $t_0\in D$. On veut définir la tangente en $M(t_0)$
\myfigure{0.8}{
\tikzinput{fig_courbes_part2_01} 
\tikzinput{fig_courbes_part2_02}
}

  \pause\pause\pause
  \item On supposera que la courbe est 
\defi{localement simple en $t_0$}
\pause
  \item C'est-à-dire qu'il existe un intervalle $I$
(ouvert non vide de centre $t_0$) tel que l'équation $M(t)=M(t_0)$ 
admette une et une seule solution dans $D\cap I$, à savoir $t=t_0$
\end{itemize}



\end{frame}


\begin{frame}
\begin{mydefinition}
On dit que la courbe admet une tangente en $M(t_0)$ si la droite $(M(t_0)M(t))$ 
admet une position limite quand $t$ tend vers $t_0$. Dans ce cas, 
la droite limite est la \defi{tangente} en $M(t_0)$ 
\end{mydefinition}

\myfigure{1.2}{
\tikzinput{fig_courbes_part2_03}
}

\end{frame}



%%%%%%%%%%%%%%%%%%%%%%%%%%%%%%%%%%%%%%%%%%%%%%%%%%%%%%%%%%%%%%%%
\section{Vecteur dérivé}

\begin{frame}
Soient $t\mapsto M(t)=(x(t),y(t))$, $t\in D\subset\Rr$, une courbe paramétrée 
\pause
\begin{mydefinition}
\begin{itemize}
  \item La courbe est \defi{dérivable en $t_0$} si les fonctions 
$x$ et $y$ le sont
  \pause
  \item Dans ce cas, le \defi{vecteur dérivé} de la courbe en $t_0$ est 
$\left(
\begin{array}{c}
x'(t_0)\\
y'(t_0)
\end{array}
\right)$
  \pause
  \item Ce vecteur se note $\overrightarrow{\dfrac{\dd M}{\dd t}}(t_0)$
\end{itemize}

\end{mydefinition}

\pause

\myfigure{0.9}{
\tikzinput{fig_courbes_part2_04}
}

\end{frame}


%%%%%%%%%%%%%%%%%%%%%%%%%%%%%%%%%%%%%%%%%%%%%%%%%%%%%%%%%%%%%%%%
\section{Tangente en un point régulier}

\begin{frame}
Soit $t\mapsto M(t)$, $t\in D\subset\Rr$, une courbe dérivable sur 
$D$ et soit $t_0 \in D$
\begin{mydefinition}

\begin{itemize}
  \item Si $\overrightarrow{\frac{\dd M}{\dd t}}(t_0)\neq\vec{0}$, 
le point $M(t_0)$ est dit \defi{régulier}
 
 \medskip
 \pause
 
  \item Si $\overrightarrow{\frac{\dd M}{\dd t}}(t_0)=\vec{0}$, 
  le point $M(t_0)$ est dit \defi{singulier}
  
  \medskip
 \pause 
  
  \item Une courbe dont tous les points sont réguliers est 
appelée \defi{courbe régulière}  


\end{itemize} 
\end{mydefinition}

\end{frame}


\begin{frame}
\begin{theoreme}
\begin{itemize}
  \item En tout point régulier la courbe admet une tangente
  
  \item La tangente en un point régulier est dirigée par le vecteur dérivé
\end{itemize}
\end{theoreme}

\myfigure{0.7}{
\tikzinput{fig_courbes_part2_05}
}

\pause

\mybox{
$M(x,y)\in T_0 \iff \left|
\begin{array}{cc}
x-x(t_0)&x'(t_0)\\
y-y(t_0)&y'(t_0)
\end{array}
\right|= 0$
}

\end{frame}


\begin{frame}
\begin{exemple}
Trouver les points où la tangente à la courbe de Lissajous 
$\left\{
\begin{array}{l}
x(t)=\sin(2t)\\
y(t)=\sin(3t)
\end{array}
\right.$, $t\in [-\pi,\pi]$,
est verticale, puis horizontale

\uncover<2->{

\bigskip
\textbf{Solution}

\vspace*{-6ex}
}
\begin{minipage}{0.55\textwidth}
\begin{itemize}
\uncover<3->{
  \item $\overrightarrow{\frac{\dd M}{\dd t}}
= \left(\begin{matrix} x'(t) \\ y'(t) \end{matrix}\right)
= \left(\begin{matrix} 2\cos(2t) \\ 3\cos(3t)\end{matrix}\right)$
\medskip
}
\uncover<4->{
  \item $x'(t) = 0 \iff t = \frac\pi4$
\medskip 
}
\uncover<6->{
  \item $y'(t) = 0 \iff t \in \left\{ \frac\pi6  ,\frac\pi2\right\} $
}
\end{itemize}  
\end{minipage}
\begin{minipage}{0.39\textwidth}
\vspace*{4ex}
\myfigure{0.9}{
\tikzinput{fig_courbes_part2_06}
}  
\end{minipage}

\vspace*{-2ex}

\end{exemple}
	
\end{frame}

%%%%%%%%%%%%%%%%%%%%%%%%%%%%%%%%%%%%%%%%%%%%%%%%%%%%%%%%%%%%%%%%
\section{Dérivation d'expressions usuelles}

\begin{frame}
Soient $f$ et $g$ deux applications à valeurs dans $\Rr^2$ dérivables en $t_0$

\begin{theoreme}

\pause
\begin{enumerate}
\item L'application 
$t\mapsto \big\langle \overrightarrow{f(t)} \mid \overrightarrow{g(t)} \big\rangle$ 
est dérivable en $t_0$ et 
$$\frac{\dd \big\langle \overrightarrow{f} \mid \overrightarrow{g} \big\rangle}{\dd t}(t_0)
=\big\langle \overrightarrow{\frac{\dd f}{\dd t}}(t_0) \mid \overrightarrow{g(t_0)}\big\rangle+
\big\langle \overrightarrow{f(t_0)} \mid \overrightarrow{\frac{\dd g}{\dd t}}(t_0)\big\rangle$$ 
\pause

\item Si $\overrightarrow{f(t_0)}\neq\vec{0}$, l'application 
$t\mapsto\|\overrightarrow{f(t)}\|$ est dérivable en $t_0$ et
$$\frac{\dd\|\overrightarrow{f}\|}{\dd t}(t_0)
=\frac{\big\langle \overrightarrow{f(t_0)} \mid 
\overrightarrow{\frac{\dd f}{\dd t}}(t_0)\big\rangle}{\|\overrightarrow{f(t_0)}\|}$$
\end{enumerate}
\end{theoreme}

\end{frame}


\begin{frame}
\begin{exemple}
\begin{itemize}
  \item Soit  $t\mapsto M(t)=(\cos t,\sin t)$ une paramétrisation du cercle 
  % de centre $O$ et de rayon $1$
  \uncover<2->{
  \smallskip
  \item Pour tout 
réel $t$, on a $OM(t)=1$ ou encore $\|\overrightarrow{OM(t)}\|=1$
  }
  \uncover<3->{
  \smallskip
  \item En dérivant :  
$\big\langle \overrightarrow{OM(t)} \mid \overrightarrow{\frac{\dd M}{\dd t}}(t) \big\rangle =0$ 
  }
  \uncover<4->{  
  \smallskip
  \item La tangente au point $M(t)$ est orthogonale au rayon $\overrightarrow{OM(t)}$
  }
\end{itemize}

\vspace*{-3ex}
\myfigure{0.8}{
\hspace*{7ex}\tikzinput{fig_courbes_part2_07}
}
\end{exemple}
\end{frame}




%%%%%%%%%%%%%%%%%%%%%%%%%%%%%%%%%%%%%%%%%%%%%%%%%%%%%%%%%%%%%%%%
\section{Mini-exercices}

\begin{frame}
\begin{miniexercice}
\begin{enumerate}
  \item Soit la courbe définie par $x(t)= t^5-4t^3$, $y(t)=t^2$. 
  Calculer le vecteur dérivé en chaque point. Déterminer le point singulier.
  Calculer une équation de la tangente au point $(3,1)$.
  Calculer les équations de deux tangentes au point $(0,4)$.
  
  \item Soit $f$ une fonction dérivable de $D \subset \Rr$ dans $\Rr^2$.
  Calculer la dérivée de l'application $t \mapsto \| f(t) \|^2$.
  
  \item Calculer le vecteur dérivé en tout point de l'astroïde définie par $x(t) = \cos^3 t$,
  $y(t) = \sin^3 t$. Quels sont les points singuliers ? Trouver une expression simple pour 
  la pente de tangente en un point régulier.
  
  \item Calculer le vecteur dérivé en tout point de la cycloïde définie par 
  $x(t) = r(t-\sin t)$,  $y(t) = r(1-\cos t)$. Quels sont les points singuliers ?
  En quels points la tangente est-elle horizontale ? En quels points la tangente est-elle
  parallèle à la bissectrice d'équation $(y=x)$ ? 
  
\end{enumerate}
\end{miniexercice}
\end{frame}

\end{document}
