
%%%%%%%%%%%%%%%%%% PREAMBULE %%%%%%%%%%%%%%%%%%


\documentclass[12pt]{article}

\usepackage{amsfonts,amsmath,amssymb,amsthm}
\usepackage[utf8]{inputenc}
\usepackage[T1]{fontenc}
\usepackage[francais]{babel}


% packages
\usepackage{amsfonts,amsmath,amssymb,amsthm}
\usepackage[utf8]{inputenc}
\usepackage[T1]{fontenc}
%\usepackage{lmodern}

\usepackage[francais]{babel}
\usepackage{fancybox}
\usepackage{graphicx}

\usepackage{float}

%\usepackage[usenames, x11names]{xcolor}
\usepackage{tikz}
\usepackage{datetime}

\usepackage{mathptmx}
%\usepackage{fouriernc}
%\usepackage{newcent}
\usepackage[mathcal,mathbf]{euler}

%\usepackage{palatino}
%\usepackage{newcent}


% Commande spéciale prompteur

%\usepackage{mathptmx}
%\usepackage[mathcal,mathbf]{euler}
%\usepackage{mathpple,multido}

\usepackage[a4paper]{geometry}
\geometry{top=2cm, bottom=2cm, left=1cm, right=1cm, marginparsep=1cm}

\newcommand{\change}{{\color{red}\rule{\textwidth}{1mm}\\}}

\newcounter{mydiapo}

\newcommand{\diapo}{\newpage
\hfill {\normalsize  Diapo \themydiapo \quad \texttt{[\jobname]}} \\
\stepcounter{mydiapo}}


%%%%%%% COULEURS %%%%%%%%%%

% Pour blanc sur noir :
%\pagecolor[rgb]{0.5,0.5,0.5}
% \pagecolor[rgb]{0,0,0}
% \color[rgb]{1,1,1}



%\DeclareFixedFont{\myfont}{U}{cmss}{bx}{n}{18pt}
\newcommand{\debuttexte}{
%%%%%%%%%%%%% FONTES %%%%%%%%%%%%%
\renewcommand{\baselinestretch}{1.5}
\usefont{U}{cmss}{bx}{n}
\bfseries

% Taille normale : commenter le reste !
%Taille Arnaud
%\fontsize{19}{19}\selectfont

% Taille Barbara
%\fontsize{21}{22}\selectfont

%Taille François
\fontsize{25}{30}\selectfont

%Taille Pascal
%\fontsize{25}{30}\selectfont

%Taille Laura
%\fontsize{30}{35}\selectfont


%\myfont
%\usefont{U}{cmss}{bx}{n}

%\Huge
%\addtolength{\parskip}{\baselineskip}
}


% \usepackage{hyperref}
% \hypersetup{colorlinks=true, linkcolor=blue, urlcolor=blue,
% pdftitle={Exo7 - Exercices de mathématiques}, pdfauthor={Exo7}}


%section
% \usepackage{sectsty}
% \allsectionsfont{\bf}
%\sectionfont{\color{Tomato3}\upshape\selectfont}
%\subsectionfont{\color{Tomato4}\upshape\selectfont}

%----- Ensembles : entiers, reels, complexes -----
\newcommand{\Nn}{\mathbb{N}} \newcommand{\N}{\mathbb{N}}
\newcommand{\Zz}{\mathbb{Z}} \newcommand{\Z}{\mathbb{Z}}
\newcommand{\Qq}{\mathbb{Q}} \newcommand{\Q}{\mathbb{Q}}
\newcommand{\Rr}{\mathbb{R}} \newcommand{\R}{\mathbb{R}}
\newcommand{\Cc}{\mathbb{C}} 
\newcommand{\Kk}{\mathbb{K}} \newcommand{\K}{\mathbb{K}}

%----- Modifications de symboles -----
\renewcommand{\epsilon}{\varepsilon}
\renewcommand{\Re}{\mathop{\text{Re}}\nolimits}
\renewcommand{\Im}{\mathop{\text{Im}}\nolimits}
%\newcommand{\llbracket}{\left[\kern-0.15em\left[}
%\newcommand{\rrbracket}{\right]\kern-0.15em\right]}

\renewcommand{\ge}{\geqslant}
\renewcommand{\geq}{\geqslant}
\renewcommand{\le}{\leqslant}
\renewcommand{\leq}{\leqslant}

%----- Fonctions usuelles -----
\newcommand{\ch}{\mathop{\mathrm{ch}}\nolimits}
\newcommand{\sh}{\mathop{\mathrm{sh}}\nolimits}
\renewcommand{\tanh}{\mathop{\mathrm{th}}\nolimits}
\newcommand{\cotan}{\mathop{\mathrm{cotan}}\nolimits}
\newcommand{\Arcsin}{\mathop{\mathrm{Arcsin}}\nolimits}
\newcommand{\Arccos}{\mathop{\mathrm{Arccos}}\nolimits}
\newcommand{\Arctan}{\mathop{\mathrm{Arctan}}\nolimits}
\newcommand{\Argsh}{\mathop{\mathrm{Argsh}}\nolimits}
\newcommand{\Argch}{\mathop{\mathrm{Argch}}\nolimits}
\newcommand{\Argth}{\mathop{\mathrm{Argth}}\nolimits}
\newcommand{\pgcd}{\mathop{\mathrm{pgcd}}\nolimits} 

\newcommand{\Card}{\mathop{\text{Card}}\nolimits}
\newcommand{\Ker}{\mathop{\text{Ker}}\nolimits}
\newcommand{\id}{\mathop{\text{id}}\nolimits}
\newcommand{\ii}{\mathrm{i}}
\newcommand{\dd}{\mathrm{d}}
\newcommand{\Vect}{\mathop{\text{Vect}}\nolimits}
\newcommand{\Mat}{\mathop{\mathrm{Mat}}\nolimits}
\newcommand{\rg}{\mathop{\text{rg}}\nolimits}
\newcommand{\tr}{\mathop{\text{tr}}\nolimits}
\newcommand{\ppcm}{\mathop{\text{ppcm}}\nolimits}

%----- Structure des exercices ------

\newtheoremstyle{styleexo}% name
{2ex}% Space above
{3ex}% Space below
{}% Body font
{}% Indent amount 1
{\bfseries} % Theorem head font
{}% Punctuation after theorem head
{\newline}% Space after theorem head 2
{}% Theorem head spec (can be left empty, meaning ‘normal’)

%\theoremstyle{styleexo}
\newtheorem{exo}{Exercice}
\newtheorem{ind}{Indications}
\newtheorem{cor}{Correction}


\newcommand{\exercice}[1]{} \newcommand{\finexercice}{}
%\newcommand{\exercice}[1]{{\tiny\texttt{#1}}\vspace{-2ex}} % pour afficher le numero absolu, l'auteur...
\newcommand{\enonce}{\begin{exo}} \newcommand{\finenonce}{\end{exo}}
\newcommand{\indication}{\begin{ind}} \newcommand{\finindication}{\end{ind}}
\newcommand{\correction}{\begin{cor}} \newcommand{\fincorrection}{\end{cor}}

\newcommand{\noindication}{\stepcounter{ind}}
\newcommand{\nocorrection}{\stepcounter{cor}}

\newcommand{\fiche}[1]{} \newcommand{\finfiche}{}
\newcommand{\titre}[1]{\centerline{\large \bf #1}}
\newcommand{\addcommand}[1]{}
\newcommand{\video}[1]{}

% Marge
\newcommand{\mymargin}[1]{\marginpar{{\small #1}}}



%----- Presentation ------
\setlength{\parindent}{0cm}

%\newcommand{\ExoSept}{\href{http://exo7.emath.fr}{\textbf{\textsf{Exo7}}}}

\definecolor{myred}{rgb}{0.93,0.26,0}
\definecolor{myorange}{rgb}{0.97,0.58,0}
\definecolor{myyellow}{rgb}{1,0.86,0}

\newcommand{\LogoExoSept}[1]{  % input : echelle
{\usefont{U}{cmss}{bx}{n}
\begin{tikzpicture}[scale=0.1*#1,transform shape]
  \fill[color=myorange] (0,0)--(4,0)--(4,-4)--(0,-4)--cycle;
  \fill[color=myred] (0,0)--(0,3)--(-3,3)--(-3,0)--cycle;
  \fill[color=myyellow] (4,0)--(7,4)--(3,7)--(0,3)--cycle;
  \node[scale=5] at (3.5,3.5) {Exo7};
\end{tikzpicture}}
}



\theoremstyle{definition}
%\newtheorem{proposition}{Proposition}
%\newtheorem{exemple}{Exemple}
%\newtheorem{theoreme}{Théorème}
\newtheorem{lemme}{Lemme}
\newtheorem{corollaire}{Corollaire}
%\newtheorem*{remarque*}{Remarque}
%\newtheorem*{miniexercice}{Mini-exercices}
%\newtheorem{definition}{Définition}




%definition d'un terme
\newcommand{\defi}[1]{{\color{myorange}\textbf{\emph{#1}}}}
\newcommand{\evidence}[1]{{\color{blue}\textbf{\emph{#1}}}}



 %----- Commandes divers ------

\newcommand{\codeinline}[1]{\texttt{#1}}

%%%%%%%%%%%%%%%%%%%%%%%%%%%%%%%%%%%%%%%%%%%%%%%%%%%%%%%%%%%%%
%%%%%%%%%%%%%%%%%%%%%%%%%%%%%%%%%%%%%%%%%%%%%%%%%%%%%%%%%%%%%


\begin{document}

\debuttexte


%%%%%%%%%%%%%%%%%%%%%%%%%%%%%%%%%%%%%%%%%%%%%%%%%%%%%%%%%%%
\diapo

\change
Dans cette partie, nous abordons,  
l'étude des courbes en coordonnées polaires.


\change

Nous commençons par rappeler la définition 
des coordonnées polaires,

\change
puis nous définissons 
une courbe en coordonnées polaires,

\change
nous terminons par l'étude des tangentes, 
d'abord en dehors de l'origine,

\change
puis à l'origine.

%%%%%%%%%%%%%%%%%%%%%%%%%%%%%%%%%%%%%%%%%%%%%%%%%%%%%%%%%%%
\diapo

On commence par rappeler 
la définition des coordonnées polaires. 

Le plan est rapporté à un repère orthonormé 
$(O,\overrightarrow{i},\overrightarrow{j})$. 

Pour $\theta$ réel, on définit un nouveau repère orthonormé en posant
$$\overrightarrow{u_\theta}=\cos\theta\overrightarrow{i}+\sin\theta\overrightarrow{j}$$

Ce vecteur est l'image du vecteur 
$\overrightarrow{i}$ par une rotation 
de centre $O$ et d'angle $\theta$.

\change
On pose aussi :
$$\overrightarrow{v_\theta}=-\sin\theta\overrightarrow{i}
+\cos\theta\overrightarrow{j}.$$ 

Ce vecteur est l'image du vecteur 
$\overrightarrow{u_\theta}$ par une rotation de centre 
$O$ et d'angle $\frac{\pi}{2}$.

\change


Si $M$ est un point du plan, on dit que le couple $[r:\theta]$ est un 
couple de \defi{coordonnées polaires} du point $M$ si et seulement 
si le vecteur $\overrightarrow{OM}$ est égal à $r$ fois le vecteur $\overrightarrow{u_\theta}$. 



%%%%%%%%%%%%%%%%%%%%%%%%%%%%%%%%%%%%%%%%%%%%%%%%%%%%%%%%%%%
\diapo

On définit la courbe d'équation polaire $r=r(\theta)$, 
comme  l'application $F$ qui à un réel $\theta$ associe 
le point de coordonnées polaires $ \big[r(\theta):\theta\big]$,
où $r(\theta)$ est une fonction donnée de l'angle $\theta$.



\change

On confond fréquemment, et abusivement, 
cette application $F$ avec son image, qui est l'ensemble $\mathcal C$ 
des points du plan de coordonnées polaires $ \big[r(\theta):\theta\big] $.

Etre un point de la courbe signifie exactement qu'il existe  de coordonnées \\
polaires $[r:\theta]$ de ce point telles que $r=r(\theta)$

\change

On peut aussi voir la courbe donnée par une équation polaire comme une 
courbe paramétrique dépendant du paramètre $\theta$. On convertit 
les coordonnées polaires en coordonnées cartésiennes, grâce aux formules usuelles :

$$\begin{array}{l}
x=r(\theta)\cos(\theta)\\
y=r(\theta)\sin(\theta)
\end{array}
.$$



%%%%%%%%%%%%%%%%%%%%%%%%%%%%%%%%%%%%%%%%%%%%%%%%%%%%%%%%%%%
\diapo

Voici l'exemple de la spirale donnée par l'équation polaire 

$$r = \sqrt{\theta}$$

pour $\theta \in [0,+\infty[$.



Par exemple pour $\theta = 0$, $r(\theta)= 0$,
donc l'origine appartient à la courbe $\mathcal{C}$.

Pour $\theta = \pi/2$ le point est donc situé sur la droite faisant un angle $\pi/2$ avec l'horizontale,
donc sur l'axe vertical. Et le point est situé à une distance de $\sqrt(\pi/2)$. C'est ce point.

Voici le point obtenu pour $\theta = \pi$, le rayon est $\sqrt{\pi}$,

celui pour $\theta=2\pi$, 

celui pour $\theta = 5\pi/2$, etc.

à chaque fois 
le point $M(\theta)$ est situé sur la droite passant par l'origine 
et faisant un angle $\theta$ avec l'axe des abscisses.


Notez bien que $5\pi/2$ et $pi/2$ c'est le même angle modulo $2\pi$, mais
les rayons sont différents !

Sur cet exemple le rayon est toujours positif, mais en général 
rien n'interdit que la fonction $r$ soit négative. Cela signifie que pour un angle $\theta$
le point est situé sur l'autre demi-droite par rapport à l'origine [gestes mains].

%%%%%%%%%%%%%%%%%%%%%%%%%%%%%%%%%%%%%%%%%%%%%%%%%%%%%%%%%%%
\diapo

Pour pouvoir dériver un arc en coordonnées polaires, 
il faut d'abord savoir 
dériver ces vecteurs par rapport à $\theta$.

On rappelle :
$$\overrightarrow{u_\theta}=\cos\theta\overrightarrow{i}+\sin\theta\overrightarrow{j}
$$

\change
$$\quad \text{ et  }\quad 
\overrightarrow{v_\theta}=-\sin\theta\overrightarrow{i}+\cos\theta\overrightarrow{j}
.$$ 



\change

On dérive ces vecteurs composante par composante et 
vu les dérivées des fonctions $\cos$ et $\sin$, on obtient immédiatement que 

$$\frac{\dd\overrightarrow{u_\theta}}{\dd\theta}
=\overrightarrow{v_\theta}$$

et que 


$$\frac{\dd\overrightarrow{v_\theta}}{\dd\theta}=-\overrightarrow{u_\theta}.$$

\change

On retient que l'effet de la dérivation est pour ces deux vecteurs, 
le même que celui d'une rotation de centre l'origine et d'angle $+\frac{\pi}{2}$.






%%%%%%%%%%%%%%%%%%%%%%%%%%%%%%%%%%%%%%%%%%%%%%%%%%%%%%%%%%%
\diapo


Dans ce théorème, on suppose que la courbe en coordonnées polaires 
est définie par une fonction $r$ de la variable $\theta$ 
dérivable sur son domaine de définition.

\change
On souhaite obtenir la tangente et pour cela on va calculer le vecteur dérivé.

\change
Le premier point du théorème affirme que tout point de la 
courbe distinct de l'origine est régulier, et qu'un vecteur directeur de la tangente est donné par 
le vecteur vitesse 
$\overrightarrow{\dfrac{\dd M}{\dd \theta}}$
qui vaut $=r'(\theta)\overrightarrow{u_\theta}+r(\theta)\overrightarrow{v_\theta}$.

C'est une conséquence immédiate des formules de dérivation
du vecteur $OM$ qui est $r(\theta) \overrightarrow{u_\theta}$.

\change
Voici géométriquement les coordonnées du vecteur vitesse : 

c'est $r'(theta)$ sur l'axe porté par l'angle $\theta$

c'est $r(theta)$ sur l'axe orthogonal.

Le repère 
$$(M(\theta),\overrightarrow{u_\theta},\overrightarrow{v_\theta})$$
est le \defi{repère polaire} 
en $M(\theta)$. Attention il change avec $\theta$.

Dans ce repère, les coordonnées du vecteur directeur à la tangente sont donc $(r',r)$.

\change

Le deuxième point identifie l'angle $\beta$ que voici.

C'est l'angle entre le vecteur $\overrightarrow{u_\theta}$ et la tangente.

On trouve $\beta$ simplement par la relation
$\tan \beta = \frac{r}{r'}$  si $r'\neq 0$, 

et si $r'=0$, l'angle $\beta = \frac{\pi}{2}$.



%%%%%%%%%%%%%%%%%%%%%%%%%%%%%%%%%%%%%%%%%%%%%%%%%%%%%%%%%%%
\diapo

Voici un exemple d'application ; 
voici la la courbe polaire d'équation $$r=1-2\cos\theta$$

\change
On va déterminer 
la tangente au point $M$ de paramètre $\frac{\pi}{2}$.

\change
On va donc calculer le vecteur dérivé.

\change
Pour ce faire, on calcule $r'(\theta)=2\sin\theta$, ce qui donne en évaluant en 
$\frac{\pi}{2}$ que $r'(\frac{\pi}{2})=2$.

\change
De plus il est immédiat que $r(\frac{\pi}{2})=1$, 

\change
on a donc les deux coordonnées du vecteur directeur à la tangente 
recherchée dans le repère polaire.

\change
Mais ici en $\frac\pi2$ le vecteur $u_\theta$ est le vecteur unité vertial $j$,
alors que le vecteur $v_\theta$ est ici le vecteur $-i$.

Ainsi la tangente en ce point est dirigé par le vecteur $(-1,2)$, vecteur écrit dans le repère
cartésien usuel.


%%%%%%%%%%%%%%%%%%%%%%%%%%%%%%%%%%%%%%%%%%%%%%%%%%%%%%%%%%%
\diapo

Le théorème précédent ne s'appliquait pas en l'origine.  C'est en fait beaucoup plus simple.

Le théorème dit que si la courbe passe par l'origine en $\theta_0$, 
alors la tangente en l'origine (pour cette valeur du paramètre) est la droite 
faisant un angle $\theta_0$ avec l'axe des abscisses.

\change

La preuve est très simple, on obtient la tangente comme limite des sécantes. 
Chaque sécante admet pour vecteur directeur 

$$\frac{1}{r(\theta)}\overrightarrow{M(\theta_0)M(\theta)}=
\frac{1}{r(\theta)}\overrightarrow{OM(\theta)}
=\overrightarrow{u_\theta}$$

\change
Or, quand $\theta$ 
tend vers $\theta_0$, $\overrightarrow{u_\theta}$ tend vers 
$\overrightarrow{u_{\theta_0}}$. Ainsi $\overrightarrow{u_{\theta_0}}$
est un vecteur directeur de la tangente, comme on le souhaitait.


%%%%%%%%%%%%%%%%%%%%%%%%%%%%%%%%%%%%%%%%%%%%%%%%%%%%%%%%%%%
\diapo

Voici deux exemples : on étudie la tangente 
au point de paramètre $\frac{\pi}{2}$ pour les deux courbes 
données par les équations polaires respectives :
$$r=(\theta+1)\cos \theta \qquad \text{ et } \qquad r=\cos^2(\theta).$$

\change

Pour les deux courbes, on a que $M(\frac{\pi}{2})=O$ 

\change
et  donc par le théorème, la tangente en $M(\frac{\pi}{2})$ est la droite passant par l'origine, 
faisant un angle $\frac{\pi}{2}$ avec l'axe des abscisses, c'est-à-dire l'axe des ordonnées.


L'allure des deux courbes au voisinage de l'origine est cependant 
bien différente. Cette allure dépend du signe de la fonction $r$ 
au voisinage de $\frac{\pi}{2}$, point en lequel elle s'annule.

\begin{itemize}
  \item Dans le premier cas, $r$ change de signe en franchissant 
$\frac{\pi}{2}$, de positif à négatif. Ainsi, en tournant toujours dans le même
sens, on se rapproche de l'origine, on la franchit et on 
s'en écarte : c'est un point d'allure ordinaire.
  
  \item Dans le deuxième cas, $r$ ne change pas de signe. 
On ne franchit pas l'origine. On rebrousse chemin tout en tournant
toujours dans le même sens : c'est un point de rebroussement de première espèce.

Ce type de point ne peut apparaître qu'à l'origine, car on a vu avant, qu'en dehors 
de l'origine les points de la courbes sont réguliers.
\end{itemize}


%%%%%%%%%%%%%%%%%%%%%%%%%%%%%%%%%%%%%%%%%%%%%%%%%%%%%%%%%%%
\diapo

Voici quelques exercices pour vous entrainer sur les courbes en coordonnées polaires.


\end{document}
