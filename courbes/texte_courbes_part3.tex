
%%%%%%%%%%%%%%%%%% PREAMBULE %%%%%%%%%%%%%%%%%%


\documentclass[12pt]{article}

\usepackage{amsfonts,amsmath,amssymb,amsthm}
\usepackage[utf8]{inputenc}
\usepackage[T1]{fontenc}
\usepackage[francais]{babel}


% packages
\usepackage{amsfonts,amsmath,amssymb,amsthm}
\usepackage[utf8]{inputenc}
\usepackage[T1]{fontenc}
%\usepackage{lmodern}

\usepackage[francais]{babel}
\usepackage{fancybox}
\usepackage{graphicx}

\usepackage{float}

%\usepackage[usenames, x11names]{xcolor}
\usepackage{tikz}
\usepackage{datetime}

\usepackage{mathptmx}
%\usepackage{fouriernc}
%\usepackage{newcent}
\usepackage[mathcal,mathbf]{euler}

%\usepackage{palatino}
%\usepackage{newcent}


% Commande spéciale prompteur

%\usepackage{mathptmx}
%\usepackage[mathcal,mathbf]{euler}
%\usepackage{mathpple,multido}

\usepackage[a4paper]{geometry}
\geometry{top=2cm, bottom=2cm, left=1cm, right=1cm, marginparsep=1cm}

\newcommand{\change}{{\color{red}\rule{\textwidth}{1mm}\\}}

\newcounter{mydiapo}

\newcommand{\diapo}{\newpage
\hfill {\normalsize  Diapo \themydiapo \quad \texttt{[\jobname]}} \\
\stepcounter{mydiapo}}


%%%%%%% COULEURS %%%%%%%%%%

% Pour blanc sur noir :
%\pagecolor[rgb]{0.5,0.5,0.5}
% \pagecolor[rgb]{0,0,0}
% \color[rgb]{1,1,1}



%\DeclareFixedFont{\myfont}{U}{cmss}{bx}{n}{18pt}
\newcommand{\debuttexte}{
%%%%%%%%%%%%% FONTES %%%%%%%%%%%%%
\renewcommand{\baselinestretch}{1.5}
\usefont{U}{cmss}{bx}{n}
\bfseries

% Taille normale : commenter le reste !
%Taille Arnaud
%\fontsize{19}{19}\selectfont

% Taille Barbara
%\fontsize{21}{22}\selectfont

%Taille François
\fontsize{25}{30}\selectfont

%Taille Pascal
%\fontsize{25}{30}\selectfont

%Taille Laura
%\fontsize{30}{35}\selectfont


%\myfont
%\usefont{U}{cmss}{bx}{n}

%\Huge
%\addtolength{\parskip}{\baselineskip}
}


% \usepackage{hyperref}
% \hypersetup{colorlinks=true, linkcolor=blue, urlcolor=blue,
% pdftitle={Exo7 - Exercices de mathématiques}, pdfauthor={Exo7}}


%section
% \usepackage{sectsty}
% \allsectionsfont{\bf}
%\sectionfont{\color{Tomato3}\upshape\selectfont}
%\subsectionfont{\color{Tomato4}\upshape\selectfont}

%----- Ensembles : entiers, reels, complexes -----
\newcommand{\Nn}{\mathbb{N}} \newcommand{\N}{\mathbb{N}}
\newcommand{\Zz}{\mathbb{Z}} \newcommand{\Z}{\mathbb{Z}}
\newcommand{\Qq}{\mathbb{Q}} \newcommand{\Q}{\mathbb{Q}}
\newcommand{\Rr}{\mathbb{R}} \newcommand{\R}{\mathbb{R}}
\newcommand{\Cc}{\mathbb{C}} 
\newcommand{\Kk}{\mathbb{K}} \newcommand{\K}{\mathbb{K}}

%----- Modifications de symboles -----
\renewcommand{\epsilon}{\varepsilon}
\renewcommand{\Re}{\mathop{\text{Re}}\nolimits}
\renewcommand{\Im}{\mathop{\text{Im}}\nolimits}
%\newcommand{\llbracket}{\left[\kern-0.15em\left[}
%\newcommand{\rrbracket}{\right]\kern-0.15em\right]}

\renewcommand{\ge}{\geqslant}
\renewcommand{\geq}{\geqslant}
\renewcommand{\le}{\leqslant}
\renewcommand{\leq}{\leqslant}

%----- Fonctions usuelles -----
\newcommand{\ch}{\mathop{\mathrm{ch}}\nolimits}
\newcommand{\sh}{\mathop{\mathrm{sh}}\nolimits}
\renewcommand{\tanh}{\mathop{\mathrm{th}}\nolimits}
\newcommand{\cotan}{\mathop{\mathrm{cotan}}\nolimits}
\newcommand{\Arcsin}{\mathop{\mathrm{Arcsin}}\nolimits}
\newcommand{\Arccos}{\mathop{\mathrm{Arccos}}\nolimits}
\newcommand{\Arctan}{\mathop{\mathrm{Arctan}}\nolimits}
\newcommand{\Argsh}{\mathop{\mathrm{Argsh}}\nolimits}
\newcommand{\Argch}{\mathop{\mathrm{Argch}}\nolimits}
\newcommand{\Argth}{\mathop{\mathrm{Argth}}\nolimits}
\newcommand{\pgcd}{\mathop{\mathrm{pgcd}}\nolimits} 

\newcommand{\Card}{\mathop{\text{Card}}\nolimits}
\newcommand{\Ker}{\mathop{\text{Ker}}\nolimits}
\newcommand{\id}{\mathop{\text{id}}\nolimits}
\newcommand{\ii}{\mathrm{i}}
\newcommand{\dd}{\mathrm{d}}
\newcommand{\Vect}{\mathop{\text{Vect}}\nolimits}
\newcommand{\Mat}{\mathop{\mathrm{Mat}}\nolimits}
\newcommand{\rg}{\mathop{\text{rg}}\nolimits}
\newcommand{\tr}{\mathop{\text{tr}}\nolimits}
\newcommand{\ppcm}{\mathop{\text{ppcm}}\nolimits}

%----- Structure des exercices ------

\newtheoremstyle{styleexo}% name
{2ex}% Space above
{3ex}% Space below
{}% Body font
{}% Indent amount 1
{\bfseries} % Theorem head font
{}% Punctuation after theorem head
{\newline}% Space after theorem head 2
{}% Theorem head spec (can be left empty, meaning ‘normal’)

%\theoremstyle{styleexo}
\newtheorem{exo}{Exercice}
\newtheorem{ind}{Indications}
\newtheorem{cor}{Correction}


\newcommand{\exercice}[1]{} \newcommand{\finexercice}{}
%\newcommand{\exercice}[1]{{\tiny\texttt{#1}}\vspace{-2ex}} % pour afficher le numero absolu, l'auteur...
\newcommand{\enonce}{\begin{exo}} \newcommand{\finenonce}{\end{exo}}
\newcommand{\indication}{\begin{ind}} \newcommand{\finindication}{\end{ind}}
\newcommand{\correction}{\begin{cor}} \newcommand{\fincorrection}{\end{cor}}

\newcommand{\noindication}{\stepcounter{ind}}
\newcommand{\nocorrection}{\stepcounter{cor}}

\newcommand{\fiche}[1]{} \newcommand{\finfiche}{}
\newcommand{\titre}[1]{\centerline{\large \bf #1}}
\newcommand{\addcommand}[1]{}
\newcommand{\video}[1]{}

% Marge
\newcommand{\mymargin}[1]{\marginpar{{\small #1}}}



%----- Presentation ------
\setlength{\parindent}{0cm}

%\newcommand{\ExoSept}{\href{http://exo7.emath.fr}{\textbf{\textsf{Exo7}}}}

\definecolor{myred}{rgb}{0.93,0.26,0}
\definecolor{myorange}{rgb}{0.97,0.58,0}
\definecolor{myyellow}{rgb}{1,0.86,0}

\newcommand{\LogoExoSept}[1]{  % input : echelle
{\usefont{U}{cmss}{bx}{n}
\begin{tikzpicture}[scale=0.1*#1,transform shape]
  \fill[color=myorange] (0,0)--(4,0)--(4,-4)--(0,-4)--cycle;
  \fill[color=myred] (0,0)--(0,3)--(-3,3)--(-3,0)--cycle;
  \fill[color=myyellow] (4,0)--(7,4)--(3,7)--(0,3)--cycle;
  \node[scale=5] at (3.5,3.5) {Exo7};
\end{tikzpicture}}
}



\theoremstyle{definition}
%\newtheorem{proposition}{Proposition}
%\newtheorem{exemple}{Exemple}
%\newtheorem{theoreme}{Théorème}
\newtheorem{lemme}{Lemme}
\newtheorem{corollaire}{Corollaire}
%\newtheorem*{remarque*}{Remarque}
%\newtheorem*{miniexercice}{Mini-exercices}
%\newtheorem{definition}{Définition}




%definition d'un terme
\newcommand{\defi}[1]{{\color{myorange}\textbf{\emph{#1}}}}
\newcommand{\evidence}[1]{{\color{blue}\textbf{\emph{#1}}}}



 %----- Commandes divers ------

\newcommand{\codeinline}[1]{\texttt{#1}}

%%%%%%%%%%%%%%%%%%%%%%%%%%%%%%%%%%%%%%%%%%%%%%%%%%%%%%%%%%%%%
%%%%%%%%%%%%%%%%%%%%%%%%%%%%%%%%%%%%%%%%%%%%%%%%%%%%%%%%%%%%%


\begin{document}

\debuttexte


%%%%%%%%%%%%%%%%%%%%%%%%%%%%%%%%%%%%%%%%%%%%%%%%%%%%%%%%%%%
\diapo

Dans cette partie du chapitre sur les courbes paramétrées, 
nous allons nous intéresser au comportement d'une courbe 
en des points particuliers, où à l'infini.

\change

Voici le plan de cette partie.

\change

Nous étudions d'abord la tangente d'une courbe en un point singulier.

\change

Puis nous déterminons la position d'une courbe par rapport à 
sa tangente au voisinage d'un point.

\change

Enfin nous concluons par l'étude des branches infinies.


%%%%%%%%%%%%%%%%%%%%%%%%%%%%%%%%%%%%%%%%%%%%%%%%%%%%%%%%%%%
\diapo



On rappelle qu'un point d'une courbe paramétrée est dit 
\defi{point singulier} si le vecteur dérivé en ce point est nul.

\change
c'est donc dire que le vecteur $\overrightarrow{\frac{\dd M}{\dd t}}(t_0)$ est le vecteur nul.

\change
Ce qui en coordonnées s'écrit $\big( x'(t_0), y'(t_0)\big) = (0,0)$

\change
Pour obtenir une éventuelle tangente en un point singulier, 
on ne peut pas utiliser le vecteur dérivé qui est le vecteur nul.

Le plus immédiat est de revenir à la définition en étudiant 
la direction limite de la droite $(M(t_0)M(t))$, c-à-d
en étudiant la limite du coefficient directeur de cette droite, on étudie donc 

$$\displaystyle \lim_{t\rightarrow t_0}\frac{y(t)-y(t_0)}{x(t)-x(t_0)}$$

\change
Si cette limite existe, la tangente en $M(t_0)$ existe et a pour pente 
la limite de ces pentes 

\change
\change
la limite peut éventuellement être infinie auquel cas la tangente est verticale.


%%%%%%%%%%%%%%%%%%%%%%%%%%%%%%%%%%%%%%%%%%%%%%%%%%%%%%%%%%%
\diapo

Voici un exemple : la courbe paramétrée donnée par les équations 

$$\left\{
\begin{array}{l}
x(t)=3t^2\\
y(t)=2t^3\\
\end{array}\right.$$

dont on va déterminer les tangentes en tous points.

\change

On calcule  le vecteur dérivé immédiatement 

\change

$$\overrightarrow{\frac{\dd M}{\dd t}}(t)=\left(\begin{smallmatrix}
6t\\6t^2\end{smallmatrix}\right)$$ 

Ce vecteur est nul si et seulement si $6t=6t^2=0$ ou encore $t=0$. Tous
les points de la courbe sont réguliers, à l'exception de $M(0)$. 

\change
\change

Comme en dehors de l'origine le vecteur dérivée est non nul, il
dirige la tangente.

\change
On peut même diviser par $6t$ qui est non nul.

\change
Une équation de la tangente au point $(3t^2,2t^3)$ est 
donc $t(x-3t^2)-(y-2t^3)=0$

\change
ou encore $y=tx-t^3$


\change
Mais à l'origine le point est singulier donc les formules
précédentes ne sont plus valides, en effet le vecteur dérivé 
est alors le vecteur nul.

\change
On calcule donc la pente de la droite $\left( M(0)M(t) \right)$ :

$\frac{y(t)-y(0)}{x(t)-x(0)}=\frac{2t^3}{3t^2}=\frac{2t}{3}$

la pente tend vers $0$ lorsque $t$ tend vers $0$.

\change
La tangente en $M(0)$ est donc la droite passant par 
$M(0)$ et de pente $0$, c'est-à-dire l'axe des abscisses.




%%%%%%%%%%%%%%%%%%%%%%%%%%%%%%%%%%%%%%%%%%%%%%%%%%%%%%%%%%%
\diapo

Quand la courbe arrive en un point singulier $M(t_0)$, le long de sa tangente, on a 
quatre possibilités :
\begin{enumerate}
  \item la courbe continue dans le même sens, 
  sans traverser la tangente :
  c'est un \defi{point d'allure ordinaire},

  \change
  
  \item la courbe continue dans le même sens, en traversant la tangente :
  c'est un \defi{point d'inflexion},

  \change
  
  \item la courbe rebrousse chemin le long de 
cette tangente en la traversant, c'est un
\defi{point de rebroussement de première espèce},

  \change
  
  \item la courbe rebrousse chemin le long de 
cette tangente sans la traverser, c'est un
\defi{point de rebroussement de seconde espèce}.
  
\end{enumerate}

Intuitivement, on ne peut rencontrer des points de rebroussement qu'en 
un point singulier, car en un point où la vitesse est non nulle, 
on continue son chemin dans le même sens.


%%%%%%%%%%%%%%%%%%%%%%%%%%%%%%%%%%%%%%%%%%%%%%%%%%%%%%%%%%%
\diapo
Pour déterminer de façon systématique la position 
de la courbe par rapport à sa tangente
en un point singulier $M(t_0)$, 
on effectue un développement limité des coordonnées 
de $M(t) = \big(x(t),y(t)\big)$ 
au voisinage de $t=t_0$.
Pour simplifier l'expression on suppose $t_0=0$.
On écrit : 
$$M(t) = M(0) + t^p \overrightarrow{v} + t^q \overrightarrow{w} +t^q \overrightarrow{\epsilon}(t)$$

\change
où :
\begin{itemize}
  \item $p$ et $q$ sont des entiers, avec $p<q$,
  
\change
  
  \item $\overrightarrow{v}$ et $\overrightarrow{w}$ sont deux vecteurs formant une base, il suffit donc qu'ils soient non colinéaires.
  
  \change
  \item $\overrightarrow{\epsilon}(t)$ est le reste, c'est ici une fonction 
  vectorielle, qui tend vers $0$ lorsque $t$ tend vers $0$
\end{itemize}

\change
En un tel point $M(0)$, la courbe $\mathcal{C}$ admet une tangente, dont un vecteur directeur est 
$\overrightarrow{v}$. 

\change
La position de la courbe $\mathcal{C}$ par rapport à cette tangente
est donnée par la parité de $p$ et $q$. 

On retrouve :


\change
un \defi{point d'allure ordinaire}, (pour $p$ impair, $q$ pair)

\change
un \defi{point d'inflexion}, (pour $p$ impair, $q$ impair)

\change
un \defi{point de rebroussement de première espèce}, (pour $p$ pair, $q$ impair)

\change 
un \defi{point de rebroussement de seconde espèce} (pour $p$ pair, $q$ pair).


Ce n'est pas indispensable d'apprendre par cœur ces quatre situations. 
On retrouve le cas dans lequel on est en réfléchissant 
au signe des coordonnées dans la base 
$(\overrightarrow{v},\overrightarrow{w})$. 
Par exemple pour $p$ pair et $q$ impair, 
la coordonnée en $\overrightarrow{v}$ 
reste positive alors que celle en 
$\overrightarrow{w}$ change de signe. 
Comme la courbe est tangente à $\overrightarrow{v}$, 
on en déduit qu'on a affaire à un point de 
rebroussement de première espèce.



%%%%%%%%%%%%%%%%%%%%%%%%%%%%%%%%%%%%%%%%%%%%%%%%%%%%%%%%%%%
\diapo

Voici un premier exemple : la courbe donnée par 

$$\left\{\begin{array}{l} x(t) = t^5\\ y(t) = t^3\end{array}\right..$$

\change
Voilà le dessin

\change
En $M(0)=(0,0)$, il y a bien un point singulier.


On écrit
$$M(t) = t^3 \begin{pmatrix}0\\1\end{pmatrix} + t^5 
\begin{pmatrix}1\\0\end{pmatrix}.$$

\change
Ce qui veut dire $p=3$, $q=5$, 

\change
et 
$\overrightarrow{v}=\left(\begin{smallmatrix}0\\1\end{smallmatrix}\right)$,
$\overrightarrow{w}=\left(\begin{smallmatrix}1\\0\end{smallmatrix}\right)$.

La tangente, dirigée par $\overrightarrow{v}$, est verticale à l'origine.
Comme $p=3$ est impair, $t^3$ change de signe en $0$,
donc la courbe continue le long de la tangente, et comme $q=5$ est aussi
impair, la courbe franchit la tangente au point singulier.

\change
C'est un point d'inflexion.


%%%%%%%%%%%%%%%%%%%%%%%%%%%%%%%%%%%%%%%%%%%%%%%%%%%%%%%%%%%
\diapo

Voici un deuxième exemple : la courbe donnée par 

$$\left\{\begin{array}{l} x(t) = 2t^2\\ y(t) = t^2 - t^3\end{array}\right..$$

\change

\change
En $M(0)=(0,0)$, il y a bien un point singulier.

\change
On écrit
$$M(t) = t^2 \begin{pmatrix}2\\1\end{pmatrix} + t^3 
\begin{pmatrix}0\\-1\end{pmatrix}.$$

\change
Ce qui donne $p=2$, $q=3$, 

\change
$\overrightarrow{v}=\left(\begin{smallmatrix}2\\1\end{smallmatrix}\right)$,
$\overrightarrow{w}=\left(\begin{smallmatrix}0\\-1\end{smallmatrix}\right)$.

\change
C'est cette fois un point de rebroussement de première espèce. 
En effet comme $p=2$ est pair la courbe reste orientée vers 
son vecteur tangent $\overrightarrow{v}$, et comme $q=3$ 
est impair elle traverse la tangente.

%%%%%%%%%%%%%%%%%%%%%%%%%%%%%%%%%%%%%%%%%%%%%%%%%%%%%%%%%%%
\diapo

Nous allons maintenant nous intéresser aux branches infinies. 
Précisons cette notion : on dit que la courbe paramétrée
par $(x(t),y(t))$
admet une branche infinie en une borne $t_0$ de l'intervalle 
de définition si $x(t)$ ou $y(t)$ tend vers $+\infty$ en valeur 
absolue lorsque $t$ tend vers $t_0$. 

Il revient au même de dire que la norme du point $M(t)$ 
tend vers $+\infty$  lorsque $t$ tend vers $t_0$.

\change
Commençons par décrire les deux situations les plus simples :

si  $x(t)$ tend vers 
$+ \infty$, et $y(t)$ tend vers une limite finie $l$, (lorsque  $t$ tend vers $t_0$,)
alors la droite d'équation $y=l$ est dite asymptote horizontale à 
la courbe paramétrée.

\change
L'asymptote est ici dessinée en rouge, et la courbe "tend" vers l'asymptote.

\change
Si c'est $x(t)$ tend vers une limite finie $l$ et $y(t)$ tend vers $+\infty$,   
(toujours lorsque  $t$ tend vers $t_0$) 

alors la droite d'équation $x=l$ est 
asymptote verticale.

\change

On a bien sûr des définitions similaires si $x(t)$ ou $y(t)$ 
tend vers $-\infty$.


%%%%%%%%%%%%%%%%%%%%%%%%%%%%%%%%%%%%%%%%%%%%%%%%%%%%%%%%%%%
\diapo

Passons à la situation plus complexe où  $x(t)$ et $y(t)$ 
tendent simultanément vers $+\infty$ en valeur absolue 
lorsque $t$ tend vers $t_0$. 

\change
On cherche à savoir quand il y a une droite asymptote.
Comme ici.

\change
Pour qu'il y ait une droite asymptote oblique il faut :

(1) $\frac{y(t)}{x(t)}$ tend vers $a$ lorsque  $t$ tend vers $t_0$.
Attention ici $a$ est non nul.

\change
(2) puis on étudie la différence $y(t)-ax(t)$, si cette différence
tend vers une limite finie $b$,

alors on dit que la droite d'équation $y=ax+b$ est asymptote oblique 
à la courbe. 

C'est-à-dire, la courbe s'approche infiniment près de la courbe 
lorsque $t$ tend vers $t_0$.

\change

En résumé pour montrer qu'il existe une asymptote oblique il faut trouver
un réel $a$ non nul et un réel $b$, tel que 
$y(t)-(ax(t)+b) \to 0$ quand $t \to t_0$




%%%%%%%%%%%%%%%%%%%%%%%%%%%%%%%%%%%%%%%%%%%%%%%%%%%%%%%%%%%
\diapo

\'Etudions les branches infinies de la courbe donnée par les équations
$$\left\{\begin{array}{l} x(t) = \frac{t}{t-1} \\ y(t) = \frac{3t}{t^2-1} \end{array}\right..$$

La courbe est définie sur $\Rr$ privé de $\pm 1$.
  
  
$|x(t)| \to +\infty$ uniquement lorsque $t$ tend vers  $1$, 
par valeurs supérieures ou inférieures.

$|y(t)| \to +\infty$ lorsque $t$ tend vers  $\pm 1$, 
par valeurs supérieures ou inférieures.

\change
Il y a donc $4$ branches infinies correspondant à 
$t$ tend vers $\pm 1$, par valeurs supérieures ou inférieures 
à chaque fois.

\change
Lorsque $t\to-1^-$, $x(t) \to \frac12$ et $y(t) \to -\infty$

\change
  la droite verticale $(x=\frac12)$ est donc asymptote 
  pour cette branche infinie (qui part vers le bas).

\change

Lorsque $t\to-1^+$, $x(t) \to \frac12$ et $y(t) \to +\infty$

\change
  la même droite verticale d'équation $(x=\frac12)$ est  asymptote 
  pour cette branche infinie (qui part cette fois vers le haut).  

\change

Lorsque $t\to +1^-$, $x(t) \to -\infty$ et $y(t) \to -\infty$.
  

On cherche s'il y a une asymptote oblique... 

\change
en calculant la limite de  $\frac{y(t)}{x(t)}$, et 
un calcul immédiat montre qu'il y a une limite et que 
c'est $a=\frac32$.

\change
 On cherche ensuite si $y(t)-\frac32 x(t)$ admet une limite finie, 
 et un calcul un peu plus compliqué montre qu'effectivement il y a une limite
 qui vaut $b=-\frac34$.

\change
Ainsi la droite d'équation $y=ax+b$, c-à-d $y = \frac32 x -\frac34$ est 
asymptote à cette branche infinie.

   \change

 Lorsque $t\to +1^+$, les calculs sont similaires
  et la même droite d'équation $y = \frac32 x -\frac34$ est asymptote 
  à cette autre branche infinie.

%%%%%%%%%%%%%%%%%%%%%%%%%%%%%%%%%%%%%%%%%%%%%%%%%%%%%%%%%%%
\diapo

Voici l'allure de cette courbe paramétrée. 

\change
On part de $-\infty$ en ce point limite, puis lorsque $t$ tend vers $-1^-$ on a l'asymptote
verticale. 

\change
On repart de $-1^+$ ici et toujours avec la même asymptote verticale.

Et lorsque $t$ tend vers $+1$ par la gauche la courbe se rapproche de son asymptote
oblique. 

\change
On repart en $+1$ par la droite, toujours le long de l'asymptote oblique
pour tendre vers ce point limite lorsque $t$ tend vers $+\infty$.


Pour aller plus loin il est toujours possible de préciser la position de 
la courbe par rapport à ses asymptotes en faisant une étude de signe.


%%%%%%%%%%%%%%%%%%%%%%%%%%%%%%%%%%%%%%%%%%%%%%%%%%%%%%%%%%%
\diapo

Voici des exercices pour vous exercer à l'étude des tangentes 
et des branches infinies.

\end{document}
