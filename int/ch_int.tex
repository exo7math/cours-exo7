\documentclass[class=report,crop=false]{standalone}
\usepackage[screen]{../exo7book}

\begin{document}

%====================================================================
\chapitre{Intégrales}
%====================================================================

\insertvideo{qXDhD4y_GtQ}{partie 1. L'intégrale de Riemann}

\insertvideo{0Na28lA-2aA}{partie 2. Propriétés}

\insertvideo{y8-0OvFmAf8}{partie 3. Primitive}

\insertvideo{Nltru5TPFCI}{partie 4. Intégration par parties - Changement de variable}

\insertvideo{Yd74zJSYTPU}{partie 5. Intégration des fractions rationnelles}

\insertfiche{fic00015.pdf}{Calculs d'intégrales}


%%%%%%%%%%%%%%%%%%%%%%%%%%%%%%%%%%%%%%%%%%%%%%%%%%%%%%%%%%%%%%%%
\section*{Motivation}


Nous allons introduire l'intégrale à l'aide d'un exemple.
Considérons la fonction exponentielle $f(x)=e^x$. On souhaite calculer
l'aire $\mathcal{A}$ en-dessous du graphe de $f$ et entre les droites d'équation $(x=0)$, $(x=1)$ et l'axe $(O x)$.

\myfigure{1}{
\tikzinput{fig_int01a}
}

Nous approchons cette aire par des sommes d'aires des rectangles situés sous la courbe.
Plus précisément, soit $n \ge 1$ un entier; découpons notre intervalle $[0,1]$
à l'aide de la subdivision $(0,\frac{1}{n},\frac{2}{n}, \ldots, \frac{i}{n}, \cdots,\frac{n-1}{n},1)$.

On considère les \og rectangles inférieurs \fg{} $\mathcal{R}_i^-$,
chacun ayant pour base l'intervalle $\big[\frac{i-1}{n},\frac{i}{n}\big]$
et pour hauteur $f\big(\frac{i-1}{n}\big)=e^{(i-1)/n}$. L'entier $i$ varie de $1$ à $n$.
L'aire de $\mathcal{R}_i^-$ est \og base $\times$ hauteur \fg{} : $\big(\frac{i}{n}-\frac{i-1}{n}\big) \times e^{(i-1)/n}
= \frac{1}{n} e^{\frac{i-1}{n}}$.

\myfigure{1}{
\tikzinput{fig_int01b1}
\qquad
\tikzinput{fig_int01b2}
}

La somme des aires des $\mathcal{R}_i^-$ se calcule alors comme somme d'une suite géométrique:
\[
\sum_{i=1}^{n} \frac{e^{\frac{i-1}{n}}}{n}
= \frac{1}{n} \sum_{i=1}^{n} \big(e^{\frac 1n}\big)^{i-1}
= \frac{1}{n} \frac{1-\big(e^{\frac 1n}\big)^{n}}{1-e^{\frac 1n}}
= \frac{\frac{1}{n}}{e^{\frac 1n}-1}\big(e-1\big)
\xrightarrow[n\to+\infty]{} e-1.
\]
Pour la limite on a reconnu l'expression du type $\frac{e^x-1}{x} \xrightarrow[x\to 0]{} 1$
(avec ici $x=\frac1n$).

Soit maintenant les \og rectangles supérieurs \fg{} $\mathcal{R}_i^+$,
ayant la même base $\big[\frac{i-1}{n},\frac{i}{n}\big]$ mais la hauteur
$f\big(\frac{i}{n}\big)=e^{i/n}$. Un calcul similaire montre que
$\sum_{i=1}^{n} \frac{e^{\frac{i}{n}}}{n} \to e-1$ lorsque $n\to +\infty$.

L'aire $\mathcal{A}$ de notre région est supérieure à la somme des aires des rectangles inférieurs  ;
et elle est inférieure
 à la somme des aires des rectangles supérieurs. Lorsque l'on considère des subdivisions de plus en plus petites
(c'est-à-dire lorsque l'on fait tendre $n$ vers $+\infty$) alors on obtient à la limite que l'aire $\mathcal{A}$
de notre région est encadrée par deux aires qui tendent vers $e-1$. Donc l'aire de notre région est
$\mathcal{A} = e-1$.


\myfigure{1}{
\tikzinput{fig_int01c}
}

Voici le plan de lecture conseillé pour ce chapitre :
il est tout d'abord nécessaire de bien comprendre comment est définie l'intégrale
et quelles sont ses principales propriétés (parties \ref{sec:int1} et \ref{sec:int2}).
Mais il est important d'arriver rapidement à savoir calculer des intégrales :
à l'aide de primitives ou par les deux outils efficaces que sont l'intégration par parties
et le changement de variable.

Dans un premier temps on peut lire les sections \ref{ssec:int11}, \ref{ssec:int12}
puis \ref{ssec:int21}, \ref{ssec:int22}, \ref{ssec:int23},
avant de s'attarder longuement sur les parties \ref{sec:int3}, \ref{sec:int4}.
Lors d'une seconde lecture, revenez sur la construction de l'intégrale et les preuves.


Dans ce chapitre on s'autorisera (abusivement) une confusion
entre une fonction $f$ et son expression $f(x)$. Par exemple on écrira
\assertion{une primitive de la fonction $\sin x$ est $-\cos x$} au lieu
\assertion{une primitive de la fonction $x\mapsto \sin x$ est $x\mapsto -\cos x$}.




%%%%%%%%%%%%%%%%%%%%%%%%%%%%%%%%%%%%%%%%%%%%%%%%%%%%%%%%%%%%%%%%
\section{L'intégrale de Riemann}
\label{sec:int1}

Nous allons reprendre la construction faite dans l'introduction pour une fonction $f$ quelconque.
Ce qui va remplacer les rectangles seront des \evidence{fonctions en escalier}.
Si la limite des aires en-dessous égale la limite des aires au-dessus on appelle cette limite commune
\evidence{l'intégrale} de $f$ que l'on note $\int_a^b f(x) \; dx$.
Cependant il n'est pas toujours vrai que ces limites soient égales, l'intégrale n'est donc
définie que pour les fonctions \evidence{intégrables}. Heureusement nous verrons
que si la fonction $f$ est continue alors elle est intégrable.

\myfigure{1}{
\tikzinput{fig_int02}
}


%---------------------------------------------------------------
\subsection{Intégrale d'une fonction en escalier}
\label{ssec:escalier}
\label{ssec:int11}

\begin{definition}
Soit $[a,b]$ un intervalle fermé borné de $\Rr$
($-\infty<a<b<+\infty$). On appelle une \defi{subdivision}\index{subdivision} de $[a,b]$ une suite finie,
strictement croissante, de nombres $\mathcal{S}=(x_0,x_1,\ldots,x_n)$ telle que $x_0=a$ et
$x_n=b$. Autrement dit $a=x_0< x_1<\cdots< x_n=b$.
\end{definition}

\myfigure{1}{
\tikzinput{fig_int03}
}

\begin{definition}
Une fonction $f : [a,b] \to \Rr$ est une \defi{fonction en escalier}\index{fonction!en escalier}
s'il existe  une subdivision $(x_0,x_1,\ldots,x_n)$ et des nombres réels
$c_1,\ldots,c_n$ tels que pour tout $i\in \{1,\ldots,n\}$ on ait
$$\forall x \in ]x_{i-1},x_i[ \quad f(x)=c_i$$
\end{definition}

Autrement dit $f$ est une fonction constante sur chacun des sous-intervalles de la subdivision.


\begin{remarque*}
La valeur de $f$ aux points $x_i$ de la subdivision n'est pas imposée. Elle peut être égale à
celle de l'intervalle qui précède ou de celui qui suit, ou encore une autre valeur arbitraire.
Cela n'a pas d'importance car l'aire ne changera pas.
\end{remarque*}
\myfigure{1}{
\tikzinput{fig_int04}
}

\begin{definition}
Pour une fonction en escalier comme ci-dessus, son \defi{intégrale}\index{integrale@intégrale}
est le réel $\int_a^b f(x) \; dx$ défini par
\mybox{$\displaystyle \int_a^b f(x) \; dx = \sum_{i=1}^n c_i(x_i-x_{i-1})$}
\end{definition}



\begin{remarque*}
Notez que chaque terme $c_i(x_i-x_{i-1})$ est l'aire du rectangle compris entre les abscisses
$x_{i-1}$ et $x_i$ et de hauteur $c_i$. Il faut juste prendre garde que l'on compte l'aire
avec un signe \og $+$ \fg{} si $c_i>0$ et un signe \og $-$ \fg{} si $c_i<0$.


L'intégrale d'une fonction en escalier est l'aire de la partie % [[en rouge ??]]
située au-dessus
de l'axe des abscisses \couleurnb{(ici en rouge)}{} moins l'aire de la partie %[[en bleu]]
située en-dessous  \couleurnb{(en bleu)}{}.
L'intégrale d'une fonction en escalier est bien un nombre réel qui mesure l'aire algébrique
(c'est-à-dire avec signe) entre la courbe de $f$ et l'axe des abscisses.
\end{remarque*}

%---------------------------------------------------------------
\subsection{Fonction intégrable}
\label{ssec:int12}

Rappelons qu'une fonction $f : [a,b] \to \Rr$ est \defi{bornée} s'il existe $M\ge0$ tel que :
$$\forall x \in [a,b] \quad -M \le f(x) \le M.$$

Rappelons aussi que si l'on a deux fonctions $f,g : [a,b] \to \Rr$, alors on note
$$f \le g  \qquad \iff \qquad \forall x \in [a,b] \quad f(x) \le g(x).$$

\bigskip

On suppose à présent que $f : [a,b] \to \Rr$ est une fonction bornée quelconque. On définit deux nombres réels :
$$I^-(f) = \sup \left\{ \int_a^b \phi(x) \; dx \mid \phi \text{ en escalier et } \phi \le f \right\}$$
$$I^+(f) = \inf \left\{ \int_a^b \phi(x) \; dx \mid \phi \text{ en escalier et } \phi \ge f \right\}$$


\myfigure{1}{
\tikzinput{fig_int05}
}

Pour $I^-(f)$ on prend toutes les fonctions en escalier (avec toutes les subdivisions possibles)
qui restent inférieures à $f$. On prend l'aire la plus grande parmi toutes ces fonctions en escalier,
comme on n'est pas sûr que ce maximum existe on prend la borne supérieure.
Pour $I^+(f)$ c'est le même principe mais les fonctions en escalier sont supérieures à $f$
et on cherche l'aire la plus petite possible.

Il est intuitif que l'on a :
\begin{proposition}
\label{prop:ImoinsIplus}
$I^-(f) \le I^+(f)$.
\end{proposition}
Les preuves sont reportées en fin de section.

\begin{definition}
Une fonction bornée $f :[a,b] \to \Rr$ est dite \defi{intégrable}\index{fonction!integrable@intégrable} (\defi{au sens de Riemann})
si $I^-(f) = I^+(f)$. On appelle alors ce nombre 
\defi{l'intégrale de Riemann}\index{integrale@intégrale!de Riemann} de $f$ sur $[a,b]$ et on le note
$\int_a^b f(x)\; dx$.
\end{definition}

\begin{exemple}
\sauteligne
\begin{itemize}
  \item Les fonctions en escalier sont intégrables ! En effet si $f$ est une fonction en escalier
alors la borne inférieure $I^-(f)$ et supérieure $I^+(f)$ sont atteintes avec la fonction
$\phi=f$. Bien sûr l'intégrale $\int_a^b f(x)\; dx$ coïncide avec l'intégrale de la fonction
en escalier définie lors du paragraphe \ref{ssec:escalier}.


  \item Nous verrons dans la section suivante que les fonctions continues
et les fonctions monotones sont intégrables.

  \item Cependant toutes les fonctions ne sont pas intégrables. La fonction $f : [0,1] \to \Rr$ définie par
$f(x)=1$ si $x$ est rationnel et $f(x)=0$ sinon, n'est pas intégrable sur $[0,1]$.
Convainquez-vous que si $\phi$ est une fonction en escalier avec $\phi \le f$ alors $\phi \le 0$
et que si $\phi \ge f$ alors $\phi \ge 1$. On en déduit que $I^-(f)=0$ et $I^+(f)=1$.
Les bornes inférieure et supérieure ne coïncident pas, donc $f$ n'est pas intégrable.


\myfigure{1}{
\tikzinput{fig_int06}
}
\end{itemize}
\end{exemple}


Il n'est pas si facile de calculer des exemples avec la définition.
Nous avons vu l'exemple de la fonction exponentielle dans l'introduction
où nous avions en fait montré que
$\int_0^1 e^x \; dx = e-1$. Nous allons voir maintenant
l'exemple de la fonction $f(x)=x^2$.
Plus tard nous verrons que les primitives permettent de calculer simplement
beaucoup d'intégrales.


\begin{exemple}
Soit $f:[0,1] \to \Rr$, $f(x)=x^2$. Montrons qu'elle est intégrable et calculons
$\int_0^1f(x) \; dx$.

\myfigure{1}{
\tikzinput{fig_int11}
}

Soit $n\ge 1$ et considérons la subdivision régulière de $[0,1]$ suivante
$\mathcal{S}=\big(0,\frac1n,\frac2n,\ldots,\frac in,\ldots, \frac{n-1}{n},1\big)$.

Sur l'intervalle $\big[\frac{i-1}{n},\frac in\big]$ nous avons
\[
\forall x \in \big[\tfrac{i-1}{n},\tfrac in\big] \quad  \big(\tfrac{i-1}{n}\big)^2 \le x^2 \le \big(\tfrac in\big)^2 \; .
\]

Nous construisons une fonction en escalier $\phi^-$ en-dessous de $f$  par
$\phi^-(x) = \frac{(i-1)^2}{n^2}$ si $x \in \big[\frac{i-1}{n},\frac in\big[$
(pour chaque $i=1,\ldots,n$) et $\phi^-(1)=1$.
De même nous construisons une fonction en escalier $\phi^+$ au-dessus de $f$ définie
par $\phi^+(x) = \frac{i^2}{n^2}$ si $x \in \big[\frac{i-1}{n},\frac in\big[$
(pour chaque $i=1,\ldots,n$) et $\phi^+(1)=1$.
$\phi^-$ et $\phi^+$ sont des fonctions en escalier et l'on a $\phi^- \le f \le \phi^+$.

L'intégrale de la fonction en escalier $\phi^+$ est par définition
\[
\int_0^1 \phi^+(x)\; dx = \sum_{i=1}^n  \frac{i^2}{n^2} \left(\frac in - \frac{i-1}{n}\right)
= \sum_{i=1}^n  \frac{i^2}{n^2} \frac 1n  = \frac{1}{n^3} \sum_{i=1}^n i^2.
\]

On se souvient de la formule $\sum_{i=1}^n i^2=\tfrac{n(n + 1)(2n + 1)}{6}$, et donc
\[
\int_0^1 \phi^+(x)\; dx = \frac{n(n + 1)(2n + 1)}{6n^3} =  \frac{(n + 1)(2n + 1)}{6n^2} \; \cdotp
\]

De même pour la fonction $\phi^-$ :
\[
\int_0^1 \phi^-(x)\; dx = \sum_{i=1}^n  \frac{(i-1)^2}{n^2} \frac 1n
= \frac{1}{n^3} \sum_{j=1}^{n-1} j^2 =\frac{(n-1)n(2n - 1)}{6n^3} =  \frac{(n - 1)(2n - 1)}{6n^2}\; \cdotp
\]

Maintenant $I^-(f)$ est la borne supérieure sur toutes les fonctions en escalier inférieures à $f$
donc en particulier $I^-(f) \ge \int_0^1 \phi^-(x)\; dx$. De même $I^+(f) \le \int_0^1 \phi^+(x)\; dx$.
En résumé :
$$\tfrac{(n - 1)(2n - 1)}{6n^2}=\int_0^1 \phi^-(x)\; dx \le I^-(f) \le I^+(f) \le \int_0^1 \phi^+(x)\; dx =\tfrac{(n + 1)(2n + 1)}{6n^2}.$$

Lorsque l'on fait tendre $n$ vers $+\infty$ alors les deux extrémités tendent vers $\frac13$.
On en déduit que $I^-(f)= I^+(f)=\frac13$. Ainsi $f$ est intégrable
et $\int_0^1 x^2\; dx = \frac13$.
\end{exemple}


%---------------------------------------------------------------
\subsection{Premières propriétés}

\begin{proposition}
\label{prop:intprop}
\sauteligne
\begin{enumerate}
  \item Si $f : [a,b] \to \Rr$ est intégrable et si l'on change les valeurs de $f$
en un nombre fini de points de $[a,b]$ alors la fonction $f$ est
toujours intégrable et la valeur de l'intégrale $\int_a^b f(x)\; dx$ ne change pas.

  \item Si $f  : [a,b] \to \Rr$ est intégrable alors la restriction de $f$
à tout intervalle $[a',b'] \subset [a,b]$ est encore intégrable.
\end{enumerate}
\end{proposition}



%---------------------------------------------------------------
\subsection{Les fonctions continues sont intégrables}

Voici le résultat théorique le plus important de ce chapitre.
\begin{theoreme}
\label{th:continueintegrable}
Si $f : [a,b] \to \Rr$ est continue alors $f$ est intégrable.
\end{theoreme}

La preuve sera vue plus loin mais l'idée est que les fonctions continues peuvent être
approchées d'aussi près que l'on veut par des fonctions en escalier, tout en gardant
un contrôle d'erreur uniforme sur l'intervalle.

\medskip

Une fonction $f : [a,b] \to \Rr$ est dite \defi{continue par morceaux}\index{fonction!continue par morceaux} s'il existe un entier $n$ et
une subdivision $(x_0,\ldots,x_n)$ telle que $f_{|]x_{i-1},x_i[}$ soit continue, admette une limite finie à droite en $x_{i-1}$ et une limite à gauche en $x_{i}$ pour tout $i \in \{ 1,\ldots,n \}$.

\myfigure{0.8}{
\tikzinput{fig_int12}
}

\begin{corollaire}
Les fonctions continues par morceaux sont intégrables.
\end{corollaire}

\medskip


Voici un résultat qui prouve que l'on peut aussi intégrer des fonctions qui ne sont pas continues
à condition que la fonction soit croissante (ou décroissante).
\begin{theoreme}
\label{th:monotoneintegrable}
Si $f : [a,b] \to \Rr$ est monotone alors $f$ est intégrable.
\end{theoreme}
%
% %---------------------------------------------------------------
% \subsection{Subdivision}
%

%
% Nous aurons besoin d'un peu plus de connaissance sur les subdivisions.
% Soit $[a,b]$ un intervalle fermé borné de $\Rr$
% ($-\infty<a<b<+\infty$).
% \begin{definition}
% \begin{enumerate}
%   \item (Rappel)  Une \defi{subdivision} de $[a,b]$ une suite finie
% croissante de nombres $\mathcal{S}=(x_0,x_1,\cdots,x_n)$ telle que $x_0=a$ et
% $x_n=b$.
%   \item Le \defi{pas} de la subdivision est le réel $\delta(\mathcal{S})=\sup_{i=1}^k(x_i-x_{i-1})$.
%   \item Une subdivision est dite \defi{réguli\`ere} ou à \defi{pas constant} si $x_i-x_{i-1}$
% est constant, donc pour tout $i$, $x_i-x_{i-1}=\frac{b-a}{n}$.
%   \item On dit qu'une subdivision $(y_0,y_1,\ldots,y_m)$ est \defi{plus fine} que $(x_0,x_1,\cdots,x_n)$
% si  $\{x_0,x_1,\cdots,x_n\} \subset \{y_0,y_1\cdots,y_m\}$.
% \end{enumerate}
% \end{definition}
%
% [[dessins]]

%---------------------------------------------------------------
\subsection{Les preuves}

Les preuves peuvent être sautées lors d'une première lecture.
Les démonstrations demandent une bonne maîtrise des bornes sup et inf
et donc des \og epsilons \fg{}.
La proposition \ref{prop:ImoinsIplus} se prouve en manipulant les \og epsilons \fg{}.
Pour la preuve de la proposition \ref{prop:intprop} : on prouve d'abord les propriétés
pour les fonctions en escalier et on en déduit qu'elles restent vraies pour les fonctions intégrables
(cette technique sera développée en détails dans la partie suivante).


Le théorème \ref{th:continueintegrable} établit que les fonctions continues sont intégrables.
Nous allons démontrer une version affaiblie de ce résultat. Rappelons que $f$ est dite de
\defi{classe $\mathcal{C}^1$}\index{fonction!de classe@de classe $\mathcal{C}^1$} si $f$ est continue, dérivable et $f'$ est aussi continue.
\begin{theoreme}[Théorème \ref{th:continueintegrable} faible]
\label{th:c1integrable}
Si $f : [a,b] \to \Rr$ est de classe $\mathcal{C}^1$ alors $f$ est intégrable.
\end{theoreme}

\begin{proof}
Comme $f$ est de classe  $\mathcal{C}^1$ alors $f'$ est une fonction continue sur l'intervalle fermé et borné $[a,b]$ ;
$f'$ est donc une fonction bornée : il existe $M\ge 0$ tel que pour tout $x \in [a,b]$ on ait $|f'(x)|\le M$.

Nous allons utiliser l'inégalité des accroissements finis :
\begin{equation}
\label{eq:ineqacct}
\tag{$\star$}
\forall x,y \in [a,b] \quad |f(x)-f(y)| \le M |x-y|.
\end{equation}

\medskip

Soit $\epsilon>0$ et soit $(x_0,x_1,\ldots,x_n)$ une subdivision de $[a,b]$ vérifiant pour tout $i=1,\ldots,n$ :
\begin{equation}
\label{eq:epsilon}
\tag{$\star\star$}
0 < x_i-x_{i-1} \le \epsilon.
\end{equation}

 Nous allons construire deux fonctions
en escalier $\phi^-, \phi^+ : [a,b] \to \Rr$ définies de la façon suivante :
pour chaque $i=1,\ldots,n$ et chaque $x\in[x_{i-1},x_i[$ on pose
$$c_i = \phi^-(x)= \inf_{t\in [x_{i-1},x_i[} f(t) \quad \text{ et } \quad
d_i = \phi^+(x)= \sup_{t\in [x_{i-1},x_i[} f(t)$$
et aussi $\phi^-(b)= \phi^+(b)=f(b)$.
$\phi^-$ et $\phi^+$ sont bien deux fonctions en escalier (elles sont constantes sur chaque intervalle
$[x_{i-1},x_i[$).

\myfigure{1.1}{
\tikzinput{fig_int13}
}

\medskip

De plus par construction on a bien $\phi^-\le f \le \phi^+$
et donc
\[
\int_a^b \phi^-(x)\;dx \le I^-(f) \le I^+(f) \le \int_a^b \phi^+(x)\;dx \; .
\]
En utilisant la continuité de $f$ sur l'intervalle $[x_{i - 1}, x_i]$, on déduit l'existence
de $a_i,b_i \in [x_{i -1},x_i]$ tels que $f(a_i)=c_i$ et $f(b_i)=d_i$.
Avec (\ref{eq:ineqacct}) et (\ref{eq:epsilon}) on sait que
$d_i-c_i = f(b_i) - f(a_i) \le M |b_i - a_i| \le M (x_i-x_{i-1}) \le M\epsilon$
(pour tout $i=1,\ldots,n$). Alors
$$\int_a^b \phi^+(x)\;dx - \int_a^b \phi^-(x)\;dx
\le \sum_{i=1}^n M\epsilon(x_i-x_{i-1})=M\epsilon(b-a)$$

Ainsi $0 \le I^+(f) - I^-(f) \le M\epsilon(b-a)$ et lorsque l'on fait tendre $\epsilon \to 0$
on trouve $I^+(f) = I^-(f)$, ce qui prouve que $f$ est intégrable.
\end{proof}

La preuve du théorème \ref{th:monotoneintegrable} est du même style
et nous l'omettons.


%---------------------------------------------------------------
%\subsection{Mini-exercices}

\begin{miniexercices}
\sauteligne
\begin{enumerate}
  \item Soit $f : [1,4] \to \Rr$ définie par $f(x)=1$ si $x\in[1,2[$, $f(x)=3$ si $x\in [2,3[$
et $f(x)=-1$ si $x \in [3,4]$. Calculer $\int_1^2 f(x) \;dx$, $\int_1^3 f(x)\;dx$, $\int_1^4 f(x)\;dx$,
$\int_1^{\frac32} f(x) \;dx$, $\int_{\frac32}^{\frac72} f(x) \;dx$.
  \item Montrer que $\int_0^1 x \; dx =\frac12$ (prendre une subdivision régulière
et utiliser $\sum_{i=1}^n i = \frac{n(n+1)}{2}$).
  \item Montrer que si $f$ est une fonction intégrable et \emph{paire} sur l'intervalle $[-a,a]$ alors
$\int_{-a}^a f(x)\;dx = 2 \int_0^a f(x)\; dx$ (on prendra une subdivision symétrique par rapport à l'origine).
  \item Montrer que si $f$ est une fonction intégrable et \emph{impaire} sur l'intervalle $[-a,a]$ alors
  $\int_{-a}^a f(x)\;dx = 0$.
  \item Montrer que toute fonction monotone est intégrable en s'inspirant de la preuve du théorème
\ref{th:c1integrable}.
\end{enumerate}
\end{miniexercices}

%%%%%%%%%%%%%%%%%%%%%%%%%%%%%%%%%%%%%%%%%%%%%%%%%%%%%%%%%%%%%%%%
\section{Propriétés de l'intégrale}
\label{sec:int2}

Les trois principales propriétés de l'intégrale sont
la relation de Chasles, la positivité et la linéarité.

%---------------------------------------------------------------
\subsection{Relation de Chasles}
\label{ssec:int21}

\begin{proposition}[Relation de Chasles]
\index{relation de Chasles}
Soient $a<c<b$. Si $f$ est intégrable sur $[a,c]$ et $[c,b]$, alors
$f$ est intégrable sur $[a,b]$.
Et on a $$\int_a^b f(x)\;dx = \int_a^c f(x) \; dx + \int_c^b f(x)\;dx$$
\end{proposition}



Pour s'autoriser des bornes sans se préoccuper de l'ordre on définit :
$$\int_a^a f(x) \;dx=0 \qquad \text{ et pour } a<b  \quad \int_b^a f(x) \;dx= -\int_a^b f(x) \; dx.$$

Pour $a,b,c$ quelconques la \evidence{relation de Chasles} devient alors
\mybox{$\displaystyle\int_a^b f(x)\;dx = \int_a^c f(x)\;dx + \int_c^b f(x)\;dx$}


%---------------------------------------------------------------

\subsection{Positivité de l'intégrale}
\label{ssec:int22}

\begin{proposition}[Positivité de l'intégrale]
\index{integrale@intégrale!positivite@positivité}
Soit $a \le b$ deux réels et $f$ et $g$ deux fonctions intégrables sur $[a,b]$.
\mybox{Si $f\le g$ \quad alors \quad  $\displaystyle \int_a^b f(x)\;dx \le\int_a^b g(x)\;dx$}
\end{proposition}

En particulier l'intégrale d'une fonction positive est positive :
\mybox{Si \quad $f\ge 0$ \quad alors \quad $\displaystyle \int_a^bf(x)\;dx \ge 0$}

%---------------------------------------------------------------

\subsection{Linéarité de l'intégrale}
\label{ssec:int23}

\begin{proposition}
\index{integrale@intégrale!linearite@linearité}
Soient $f,g$ deux fonctions intégrables sur $[a,b]$.
\begin{enumerate}
  \item $f+g$ est une fonction intégrable et
$\int_a^b (f+g)(x) \; dx= \int_a^b f(x) \; dx+ \int_a^b g(x) \; dx$.

  \item Pour tout réel $\lambda$,  $\lambda f$ est
intégrable et on a $\int_a^b \lambda f(x) \; dx= \lambda \int_a^b f(x) \; dx$.

Par ces deux premiers points
nous avons la \evidence{linéarité de l'intégrale} :
pour tous réels $\lambda,\mu$
\mybox{$\displaystyle
\int_a^b \big(\lambda f(x)+\mu g(x)\big)\;dx= \lambda\int_a^b f(x)\;dx+\mu\int_a^b g(x)\;dx$}


  \item $f \times g$ est une fonction intégrable sur $[a,b]$
mais en général $\int_a^b (f g)(x)\;dx \neq
\big(\int_a^b f(x)\;dx\big)\big(\int_a^b g(x)\;dx\big)$.

  \item $|f|$ est une fonction intégrable sur $[a,b]$ et
\mybox{$\displaystyle\left\vert\int_a^b f(x) \;dx\right\vert\le\int_a^b\big\vert f(x)\big\vert \;dx$}
\end{enumerate}
\end{proposition}


\begin{exemple}
$$\int_0^1 \big(7x^2-e^x\big) \; dx  = 7 \int_0^1 x^2\; dx \ \ - \  \int_0^1 e^x \; dx = 7 \frac 13 \ - \ (e-1) = \frac{10}{3}-e$$
Nous avons utilisé les calculs déjà vus : $\int_0^1 x^2\; dx = \frac13$ et $\int_0^1 e^x \; dx = e-1$.
\end{exemple}

\begin{exemple}
Soit $I_n = \int_1^n \frac{\sin(n x)}{1+x^n} \; dx$. Montrons que $I_n \to 0$ lorsque $n\to +\infty$.

$$|I_n| = \left| \int_1^n \frac{\sin(n x)}{1+x^n} \; dx \right| \le  \int_1^n  \frac{|\sin(n x)|}{1+x^n} \; dx
\le \int_1^n  \frac{1}{1+x^n} \; dx \le  \int_1^n  \frac{1}{x^n} \; dx$$

Il ne reste plus qu'à calculer cette dernière intégrale (en anticipant un peu sur la suite du chapitre) :
$$\int_1^n  \frac{1}{x^n} \; dx = \int_1^n x^{-n} \; dx = \left[\frac{x^{-n+1}}{-n+1}\right]_1^n
= \frac{n^{-n+1}}{-n+1} - \frac{1}{-n+1}  \xrightarrow[n\to+\infty]{} 0$$
(car $n^{-n+1} \to 0$ et $\frac{1}{-n+1} \to 0$).
\end{exemple}


\begin{remarque*}
Notez que même si $f\times g$ est intégrable on a en général $\int_a^b (f g)(x)\;dx \neq
\big(\int_a^b f(x)\;dx\big)\big(\int_a^b g(x)\;dx\big)$.
Par exemple, soit $f : [0,1] \to \Rr$ la fonction définie par $f(x)=1$ si $x\in[0,\frac12[$ et $f(x)=0$ sinon.
Soit $g : [0,1] \to \Rr$ la fonction définie par $g(x)=1$ si $x\in[\frac12,1[$ et $g(x)=0$ sinon.
Alors $f(x)\times g(x)=0$ pour tout $x\in[0,1]$ et donc $\int_0^1 f(x)g(x) \; dx=0$ alors
que $\int_0^1 f(x) \; dx= \frac12$ et $\int_0^1 g(x) \; dx=\frac12$.
\end{remarque*}




%---------------------------------------------------------------
\subsection{Une preuve}

Nous allons prouver la linéarité de l'intégrale:
$\int \lambda f= \lambda \int f$ et $\int f+g = \int f + \int g$.
L'idée est la suivante : il est facile de voir que pour des fonctions en escalier
l'intégrale (qui est alors une somme finie) est linéaire. Comme les fonctions en escalier
approchent autant qu'on le souhaite les fonctions intégrables alors cela implique la linéarité
de l'intégrale.

\begin{proof}[Preuve de $\int \lambda f= \lambda \int f$]
Soit $f : [a,b] \to \Rr$ une fonction intégrable et $\lambda \in \Rr$. Soit $\epsilon >0 $.
Il existe $\phi^-$ et $\phi^+$ deux fonctions en escalier approchant suffisamment $f$,
avec $\phi^- \le f \le \phi^+$ :
\begin{equation}
\label{eq:ineqint}
\tag{$\dag$}
\int_a^b f(x)\; dx  \ - \epsilon\  \le \int_a^b \phi^-(x)\;dx \quad \text{ et } \quad \int_a^b \phi^+(x)\;dx \  \le \int_a^b f(x)\; dx  \ + \epsilon\
\end{equation}


Quitte à rajouter des points, on peut supposer que la subdivision $(x_0,x_1,\ldots,x_n)$ de $[a,b]$
est suffisamment fine pour que $\phi^-$ et $\phi^+$ soient toutes les deux constantes
sur les intervalles $]x_{i-1},x_i[$ ; on note $c_i^-$ et $c_i^+$ leurs valeurs respectives.

Dans un premier temps on suppose $\lambda \ge 0$. Alors $\lambda \phi^-$ et $\lambda \phi^+$
sont encore des fonctions en escalier
vérifiant $\lambda \phi^- \le \lambda f \le \lambda\phi^+$. De plus
$$\int_a^b \lambda \phi^-(x)\;dx =  \sum_{i=1}^n \lambda c_i^-(x_{i}-x_{i-1})
=  \lambda \sum_{i=1}^n c_i^-(x_{i}-x_{i-1}) = \lambda\int_a^b  \phi^-(x)\;dx$$
De même pour $\phi^+$.
Ainsi
$$\lambda\int_a^b \phi^-(x)\;dx \le I^-(\lambda f) \le I^+(\lambda f)
\le \lambda\int_a^b \phi^+(x)\;dx$$
En utilisant les deux inégalités (\ref{eq:ineqint}) on obtient
$$\lambda \int_a^b f(x)\; dx \  -\lambda \epsilon \ \le I^-(\lambda f) \le I^+(\lambda f) \le  \lambda \int_a^b f(x)\; dx \ +\lambda \epsilon$$
Lorsque l'on fait tendre $\epsilon \to 0$ cela prouve que $I^-(\lambda f) = I^+(\lambda f)$, donc $\lambda f$ est intégrable
et $\int_a^b \lambda f(x)\; dx = \lambda \int_a^b f(x)\; dx$.
Si $\lambda \le 0$ on a $\lambda \phi^+ \le \lambda f \le \lambda\phi^-$ et le raisonnement est similaire.
\end{proof}

\begin{proof}[Preuve de $\int f+g = \int f + \int g$]
Soit $\epsilon >0 $. Soient $f,g : [a,b] \to \Rr$ deux fonctions intégrables.
On définit deux fonctions en escalier $\phi^+,\phi^-$ pour $f$ et deux fonctions en escalier
$\psi^+,\psi^-$ pour $g$ vérifiant des inégalités  similaires à (\ref{eq:ineqint}) de la preuve au-dessus.
On fixe une subdivision suffisamment fine pour toutes les fonctions
$\phi^\pm, \psi^\pm$. On note $c_i^\pm, d_i^\pm$ les constantes respectives sur l'intervalle $]x_{i-1},x_i[$.
Les fonctions $\phi^- + \psi^-$ et $\phi^+ + \psi^+$ sont en escalier et vérifient  $\phi^- + \psi^- \le f+g \le \phi^+ + \psi^+$.
Nous avons aussi que
$$\int_a^b (\phi^-+\psi^-)(x)\;dx = \sum_{i=1}^n (c_i^-+d_i^-)(x_{i}-x_{i-1})
= \int_a^b \phi^-(x)\;dx+\int_a^b \psi^-(x)\;dx$$
De même pour $\phi^+ + \psi^+$.
Ainsi
{\small
$$\int_a^b  \phi^-(x)\;dx + \int_a^b \psi^-(x)\;dx \le I^-(f+g) \le I^+(f+g)
\le \int_a^b  \phi^+(x)\;dx + \int_a^b \psi^+(x)\;dx $$
}
Les conditions du type (\ref{eq:ineqint}) impliquent alors
{\small$$
\int_a^b f(x)\; dx+\int_a^b g(x)\; dx \  -2\epsilon \ \le I^-(f+g) 
\le I^+(f+g) \le \int_a^b f(x)\; dx+\int_a^b g(x)\; dx\  +2\epsilon
$$}
Lorsque $\epsilon\to 0$ on déduit $I^-(f+g) = I^+(f+g)$, donc $f+g$ est intégrable et
$\int_a^b \big(f(x)+ g(x)\big)\; dx = \int_a^b f(x)\; dx+\int_a^b g(x)\; dx$.
\end{proof}



%---------------------------------------------------------------
%\subsection{Mini-exercices}

\begin{miniexercices}
\sauteligne
\begin{enumerate}
  \item En admettant que $\int_0^1 x^n \; dx = \frac1{n+1}$. Calculer l'intégrale
$\int_0^1 P(x)\; dx$ où $P(x)=a_n x^n+\cdots+a_1x+a_0$. Trouver un polynôme $P(x)$
non nul de degré $2$ dont l'intégrale est nulle : $\int_0^1 P(x) \; dx=0$.

  \item A-t-on $\int_a^b f(x)^2 \; dx = \left( \int_a^b f(x) \; dx \right)^2$ ;
$\int_a^b \sqrt{f(x)} \; dx = \sqrt{\int_a^b f(x) \; dx}$ ;
$\int_a^b |f(x)| \; dx = \left| \int_a^b f(x) \; dx \right|$ ;
$\int |f(x)+g(x)| \; dx = \left| \int_a^b f(x) \; dx \right| + \left| \int_a^b g(x) \; dx \right|$ ?

  \item Peut-on trouver $a<b$ tels que $\int_a^b x\; dx = -1$ ;
$\int_a^b x\; dx = 0$ ; $\int_a^b x\; dx = +1$ ?
Mêmes questions avec $\int_a^b x^2 \; dx$.

  \item Montrer que $0 \le \int_1^2  \sin^2 x \; dx \le 1$ et $\left|\int_a^b \cos^3 x \; dx \right| \le |b-a|$.
\end{enumerate}
\end{miniexercices}


%%%%%%%%%%%%%%%%%%%%%%%%%%%%%%%%%%%%%%%%%%%%%%%%%%%%%%%%%%%%%%%%
\section{Primitive d'une fonction}
\label{sec:int3}

%---------------------------------------------------------------
\subsection{Définition}

\begin{definition}
Soit $f:I \to \Rr$ une fonction définie sur un intervalle $I$ quelconque.
On dit que $F : I \to \Rr$ est une \defi{primitive}\index{primitive} de $f$ sur $I$ si
$F$ est une fonction dérivable sur $I$ vérifiant $F'(x)=f(x)$ pour tout $x \in I$.
\end{definition}
Trouver une primitive est donc l'opération inverse de calculer la fonction dérivée.

\begin{exemple}
\sauteligne
\begin{enumerate}
\item Soit $I=\Rr$ et $f: \Rr \to \Rr$ définie par $f(x) = x^2$.
Alors $F: \Rr \to \Rr$ définie par $F(x) = \frac{x^3}{3}$ est une primitive de $f$.
La fonction définie par $F(x)= \frac{x^3}{3}+1$ est aussi une primitive de $f$.

\item Soit $I=[0,+\infty[$ et $g : I\to\Rr$ définie par $g(x)=\sqrt x$.
Alors $G : I\to\Rr$ définie par $G(x)=\frac{2}{3} x^{\frac{3}{2}}$
est une primitive de $g$ sur $I$. Pour tout $c\in \Rr$, la fonction $G+c$
est aussi une primitive de $g$.
\end{enumerate}
\end{exemple}


Nous allons voir que trouver une primitive permet de les trouver toutes.
\begin{proposition}
\label{prop:primitunic}
Soit $f : I \to \Rr$ une fonction et soit $F : I \to \Rr$ une primitive de $f$.
Toute primitive de $f$ s'écrit $G=F+c$ où $c\in \Rr$.
\end{proposition}

\begin{proof}
Notons tout d'abord que si l'on note $G$ la fonction définie par $G(x)=F(x)+c$ alors
$G'(x)=F'(x)$ mais comme $F'(x)=f(x)$ alors $G'(x)=f(x)$ et $G$ est bien une primitive de $f$.

Pour la réciproque supposons que $G$ soit une primitive quelconque de $f$.
Alors $(G-F)'(x)=G'(x)-F'(x)=f(x)-f(x)=0$, ainsi
la fonction $G-F$ a une dérivée nulle sur un intervalle, c'est donc une fonction constante!
Il existe donc $c\in \Rr$ tel que $(G-F)(x)=c$. Autrement dit
$G(x)=F(x)+c$ (pour tout $x\in I$).
\end{proof}


\textbf{Notations.}
On notera une primitive de $f$ par
$\int f(t) \; dt$  ou $\int f(x) \; dx$ ou $\int f(u) \; du$
(les lettres $t, x, u, ...$ sont  des  lettres dites \emph{muettes},
c'est-à-dire interchangeables).
On peut même noter une primitive simplement par~$\int f$.

La proposition \ref{prop:primitunic}
nous dit que si $F$ est une primitive de $f$
alors il existe un réel $c$, tel que $F=\int f(t) \; dt + c$.

Attention: $\int f(t)\;dt$ désigne une fonction de $I$ dans $\Rr$
alors que l'intégrale $\int_a^b f(t) \; dt$ désigne un nombre réel.
Plus précisément nous verrons que si $F$ est une primitive de $f$ alors
$\int_a^b f(t) \; dt = F(b)-F(a)$.

\medskip

Par dérivation on prouve facilement le résultat suivant :
\begin{proposition}
Soient $F$ une primitive de $f$ et $G$ une primitive de $g$. Alors $F+G$ est une
primitive de $f+g$.
Et si $\lambda \in \Rr$ alors $\lambda F$ est une primitive de $\lambda f$.
\end{proposition}

Une autre formulation est de dire que pour tous réels $\lambda,\mu$ on a
\mybox{
$\displaystyle\int\big(\lambda f(t)+ \mu g(t)\big) \; dt=\lambda \int f(t) \; dt+\mu \int g(t)\; dt$}


%---------------------------------------------------------------
\subsection{Primitives des fonctions usuelles}


\index{primitive}
\begin{center}
%\noindent
\setlength{\arrayrulewidth}{0.05mm}
%\begin{tabular}{|l|l|l|} \hline
%\begin{tabular}[t]{|c@{\vrule depth 1.2ex height 5ex width 0mm \ }|}
\begin{tabular}{|c@{\vrule depth 3ex height 4ex width 0mm \ }|}
\hline
   $\int e^x \; dx  = e^x + c$  \quad sur $\Rr$ \\ \hline
   $\int \cos x \; dx  = \sin x  + c$  \quad sur $\Rr$ \\ \hline
   $\int \sin x \; dx  = -\cos x  + c$  \quad sur $\Rr$ \\ \hline
   $\int x^n \; dx = \frac{x^{n+1}}{n+1} + c$  \quad ($n \in \Nn$)  \quad sur $\Rr$ \\ \hline
   $\int x^\alpha \; dx = \frac{x^{\alpha+1}}{\alpha+1} + c$  \quad ($\alpha \in \Rr\setminus\{-1\}$)  sur $]0,+\infty[$\\ \hline
   $\int \frac 1x \; dx  = \ln |x|  + c$  \quad sur $]0,+\infty[$ ou $]-\infty,0[$ \\ \hline
\end{tabular}
\begin{tabular}{|c@{\vrule depth 3ex height 4ex width 0mm \ }|}
\hline
   $\int\sh x \; dx=\ch x+c$, $\int \ch x \; dx=\sh x+c$ \quad sur $\Rr$ \\ \hline
   $\int \frac{dx}{1+x^2}= \arctan x+c$ \quad sur $\Rr$ \\ \hline
   $\int\frac{dx}{\sqrt{1-x^2}} = \left\{ \begin{array}{l}
   \arcsin x + c \\ \frac\pi2-\arccos x +c \end{array} \right.$ \quad  sur $]-1,1[$ \\ \hline
   $\int \frac{dx}{\sqrt {x^2+1}}=  \left\{ \begin{array}{l} \mbox{Argsh} x+c \\
   \ln\big(x+\sqrt{x^2+1}\big)+c  \end{array} \right.$ \quad sur $\Rr$ \\ \hline
   $\int \frac{dx}{\sqrt {x^2-1}} = \left\{ \begin{array}{l} \mbox{Argch} x+c \\
   \ln\big(x+\sqrt{x^2-1}\big)+c \end{array} \right.$ \quad sur $x\in ]1,+ \infty[$\\ \hline
\end{tabular}
\end{center}


\begin{remarque*}
Ces primitives sont à connaître par c\oe ur.
\begin{enumerate}
  \item  Voici comment lire ce tableau. Si $f$ est la fonction définie sur $\Rr$ par $f(x)=x^n$
alors la fonction : $x \mapsto \frac{x^{n+1}}{n+1}$ est une primitive de $f$ sur $\Rr$. Les primitives
de $f$ sont les fonctions définies par $x \mapsto \frac{x^{n+1}}{n+1}+c$ (pour $c$ une constante réelle quelconque).
Et on écrit $\int x^n \; dx=\frac{x^{n+1}}{n+1}+c$, où $c\in \Rr$.

  \item Souvenez vous que la variable sous le symbole intégrale est une variable muette. On peut aussi bien écrire
$\int t^n \; dt = \frac{x^{n+1}}{n+1}+c$.

  \item La constante est définie pour un intervalle. Si l'on a deux intervalles, il y a
deux constantes qui peuvent être différentes. Par exemple pour $\int \frac 1x \; dx$ nous avons deux domaines de validité :
$I_1=]0,+\infty[$ et $I_2=]-\infty,0[$. Donc
$\int \frac{1}{x} \; dx=\ln x +c_1$ si $x>0$ et $\int \frac{1}{x} \; dx=\ln |x| +c_2 =\ln (-x)+c_2$ si $x<0$.

  \item On peut trouver des primitives aux allures très différentes par exemple $x\mapsto \arcsin x$ et
$x\mapsto \frac{\pi}{2}-\arccos x$ sont deux primitives de la même fonction $x\mapsto \frac{1}{\sqrt{1-x^2}}$.
Mais bien sûr on sait que $\arcsin x + \arccos x = \frac\pi2$, donc les primitives diffèrent bien d'une constante !
\end{enumerate}
\end{remarque*}

%---------------------------------------------------------------
\subsection{Relation primitive-intégrale}

\begin{theoreme}
Soit $f : [a,b] \to \Rr$ une fonction continue.
La fonction $F:I \to \Rr$ définie par
\mybox{$\displaystyle F(x)=\int_a^x f(t) \; dt$}
est une primitive de $f$, c'est-à-dire $F$ est dérivable et $F'(x)=f(x)$.

Par conséquent pour une primitive $F$ quelconque de $f$:
\mybox{$\displaystyle \int_a^b f(t) \; dt = F(b)-F(a)$}
\end{theoreme}

\textbf{Notation.} On note $\big[F(x)\big]_a^b=F(b)-F(a)$.


\begin{exemple}
Nous allons pouvoir calculer plein d'intégrales. Recalculons d'abord
les intégrales déjà rencontrées.
\begin{enumerate}
  \item  Pour $f(x)=e^x$ une primitive est $F(x)=e^x$ donc
$$\int_0^1 e^x \; dx =  \big[e^x\big]_0^1=e^1-e^0=e-1.$$

  \item Pour $g(x)=x^2$ une primitive est $G(x)=\frac{x^3}{3}$ donc
$$\int_0^1 x^2 \; dx =  \big[\tfrac{x^3}{3}\big]_0^1=\tfrac{1}{3}.$$

  \item $\int_a^x \cos t \; dt = \big[ \sin t\big]_{t=a}^{t=x} = \sin x - \sin a$
est une primitive de $\cos x$.

  \item Si $f$ est impaire alors ses primitives sont paires (le montrer).
  En déduire que $\int_{-a}^a f(t)  \; dt = 0$.
\end{enumerate}
\end{exemple}


\begin{remarque*}
\sauteligne
\begin{enumerate}

  \item $F(x)=\int_a^x f(t) \; dt$ est même \evidence{l'unique primitive de $f$ qui s'annule en $a$}.

  \item En particulier si $F$ est une fonction de classe $\mathcal{C}^1$ alors
\myboxinline{$\int_a^b F'(t) \; dt = F(b)-F(a)$}.

  \item On évitera la notation $\int_a^x f(x) \; dx$ où la variable $x$ est présente à la fois aux bornes et à l'intérieur de l'intégrale. Mieux vaut utiliser la
  notation $\int_a^x f(t) \; dt$ ou $\int_a^x f(u) \;du$ pour éviter toute
  confusion.

  \item Une fonction intégrable n'admet pas forcément une primitive.
Considérer par exemple $f : [0,1] \to \Rr$ définie par $f(x)=0$
si $x \in [0,\frac{1}{2}[$ et $f(x)=1$ si $x \in [\frac{1}{2},1]$.
$f$ est intégrable sur $[0,1]$ mais elle n'admet pas de primitive sur $[0,1]$.
En effet par l'absurde si $F$ était une primitive de $f$, par exemple la primitive qui vérifie $F(0)=0$.
Alors $F(x)=0$ pour $x\in[0,\frac12[$ et $F(x)= x -\frac{1}{2}$ pour $x\in [\frac{1}{2},1]$.
Mais alors nous obtenons une contradiction car $F$ n'est pas dérivable en $\frac12$ alors que
par définition une primitive doit être dérivable.
\end{enumerate}
\end{remarque*}

\begin{proof}
Essayons de visualiser tout d'abord pourquoi la fonction $F$ est dérivable et $F'(x)=f(x)$.


\myfigure{1}{
\tikzinput{fig_int14}
}

Fixons $x_0\in[a,b]$.
Par la relation de Chasles nous savons :
$$F(x)-F(x_0) = \int_a^x f(t) \; dt - \int_a^{x_0} f(t) \; dt =
\int_{x_0}^a f(t) \; dt + \int_a^x f(t) \; dt  = \int_{x_0}^x f(t) \; dt$$
Donc le taux d'accroissement
$$\frac{F(x)-F(x_0)}{x - x_0} = \frac{1}{x - x_0} \int_{x_0}^x f(t) \; dt = \frac{\mathcal{A}}{x - x_0} $$
où $\mathcal{A}$ est l'aire hachurée \couleurnb{(en rouge)}{}. Mais cette aire est presque un rectangle, si $x$ est suffisamment
proche de $x_0$, donc l'aire $\mathcal{A}$ vaut environ $(x - x_0)\times f(x_0)$ ; 
lorsque $x\to x_0$ le taux d'accroissement tend donc vers $f(x_0)$. Autrement dit $F'(x_0)=f(x_0)$.

\bigskip

Passons à la preuve rigoureuse.
Comme $f(x_0)$ est une constante alors $\int_{x_0}^x f(x_0) \; dt=(x - x_0)f(x_0)$, donc
\begin{eqnarray*}
\frac{F(x)-F(x_0)}{x - x_0} -f(x_0)
  &=& \frac{1}{x - x_0} \int_{x_0}^x f(t) \; dt - \frac{1}{x - x_0} \int_{x_0}^x f(x_0) \; dt \\
  &=& \frac{1}{x - x_0} \int_{x_0}^x \big(f(t)-f(x_0)\big) \; dt  
\end{eqnarray*}
Fixons $\epsilon >0$. Puisque $f$ est continue en $x_0$, il existe $\delta > 0$
tel que $(|t-x_0| < \delta \implies |f(t)-f(x_0)| < \epsilon)$.
Donc :
\begin{eqnarray*}
\left| \frac{F(x)-F(x_0)}{x - x_0} -f(x_0)\right|
  &=& \left| \frac{1}{x - x_0} \int_{x_0}^x \big(f(t)-f(x_0)\big) \; dt\right| \\
  &\le& \frac{1}{|x - x_0|}\left| \int_{x_0}^x \big| f(t)-f(x_0)\big| \; dt\right| \\
  &\le&  \frac{1}{|x - x_0|} \left|\int_{x_0}^x \epsilon \; dt \right|= \epsilon
\end{eqnarray*}
Ce qui prouve que $F$ est dérivable en $x_0$ et $F'(x_0)=f(x_0)$.

\bigskip

Maintenant on sait que $F$ est une primitive de $f$, $F$ est même la primitive qui s'annule en $a$ car
$F(a) =\int_a^a f(t) \; dt=0$.  Si $G$ est une autre primitive on sait $F=G+c$. Ainsi
\[
G(b)-G(a) = F(b)+c - \big( F(a)+c \big) = F(b)-F(a) = F(b) = \int_a^b f(t) \; dt .
\]
\end{proof}



%---------------------------------------------------------------
\subsection{Sommes de Riemann}

L'intégrale est définie à partir de limites de sommes.
Mais maintenant que nous savons calculer des intégrales sans utiliser ces sommes
on peut faire le cheminement inverse : calculer des limites de sommes à partir d'intégrales.


\begin{theoreme}
Soit $f : [a,b] \to \Rr$ une fonction intégrable, alors 
\mybox{$\displaystyle S_n = \tfrac{b-a}{n} \sum_{k=1}^{n} f\big(a+k\tfrac{b-a}{n} \big)
\qquad \xrightarrow[n\to+\infty]{} \qquad \int_a^b f(x) \; dx$}
\end{theoreme}

La somme $S_n$ s'appelle la \defi{somme de Riemann}\index{somme de Riemann} associée à l'intégrale et correspond à une subdivision régulière
de l'intervalle $[a,b]$ en $n$ petits intervalles. La hauteur de chaque rectangle
étant évaluée à son extrémité droite.

Le cas le plus utile est le cas où $a=0$, $b=1$ alors $\frac{b-a}{n}=\frac1n$ et
$f\big(a+k\frac{b-a}{n}\big) = f\big(\frac kn\big)$ et ainsi
$$S_n = \tfrac{1}{n} \sum_{k=1}^{n} f\big(\tfrac kn \big)
\qquad \xrightarrow[n\to+\infty]{} \qquad \int_0^1 f(x) \; dx$$

\myfigure{1}{
\tikzinput{fig_int10}
}

\begin{exemple}
Calculer la limite de la somme $S_n= \sum_{k=1}^{n} \frac1{n+k}$.

On a $S_1=\frac12$, $S_2=\frac13+\frac14$, $S_3=\frac14+\frac15+\frac16$,
$S_4=\frac15+\frac16+\frac17+\frac18$,\ldots

La somme $S_n$ s'écrit aussi $S_n = \frac{1}{n}  \sum_{k=1}^{n} \frac1{1+\frac kn}$.
En posant $f(x)=\frac{1}{1+x}$, $a=0$ et $b=1$,
on reconnaît que $S_n$ est une somme de Riemann.
Donc
\begin{multline*}
S_n=\frac{1}{n} \sum_{k=1}^{n} \frac1{1+\frac kn}=\frac{1}{n} \sum_{k=1}^{n} f\big(\tfrac kn\big)\\
\xrightarrow[n\to+\infty]{} \int_a^b f(x) \; dx = \int_0^1 \frac{1}{1+x} \; dx
=\big[\ln|1+x|\big]_0^1 = \ln 2-\ln 1 = \ln 2.
\end{multline*}

Ainsi $S_n \to \ln 2$ (lorsque $n\to +\infty$).
\end{exemple}

% \bigskip
%
% [[paragraphe suivant : à virer ??]]
%
% Il existe en fait plusieurs sommes de Riemann, qui convergent toutes vers $\int_a^b f(x) \; dx$.
% De façon générale soit $\mathcal{S}_n=(x_0,x_1,\ldots,x_n)$ une subdivision de l'intervalle $[a,b]$
% pour chaque $i=1,\ldots,n$ soit $t_i \in [x_{i-1},x_i]$.
% Le \defi{pas} de la subdivision $\mathcal{S}_n$ est par définition
% $\delta(\mathcal{S}_n) :=\max_{i=,\ldots,n} (x_i-x_{i-1})$.
%
% \begin{theoreme}
% Soit $f$ une fonction intégrable et supposons que $\delta(\mathcal{S}_n) \to 0$ quand $n\to+\infty$.
% Alors
% $$\sum_{i=1}^n f(t_i)(x_i-x_{i-1})  \xrightarrow[n\to+\infty]{} \int_a^b f(x) \; dx$$
% \end{theoreme}





%---------------------------------------------------------------
%\subsection{Mini-exercices}

\begin{miniexercices}
\sauteligne
\begin{enumerate}
  \item Trouver les primitives des fonctions : $x^3-x^7$, $\cos x+\exp x$, $\sin(2x)$, $1+\sqrt{x}+x$,
$\frac{1}{\sqrt x}$, $\sqrt[3]{x}$, $\frac{1}{x+1}$.
  \item Trouver les primitives des fonctions : $\ch (x)-\sh (\frac{x}{2})$, $\frac{1}{1+4x^2}$,
$\frac{1}{\sqrt {1+x^2}} - \frac{1}{\sqrt {1-x^2}}$.
  \item Trouver une primitive de $x^2e^x$ sous la forme $(a x^2+b x+c)e^x$.
  \item Trouver toutes les primitives de $x\mapsto \frac{1}{x^2}$ (préciser les intervalles et les constantes).
  \item Calculer les intégrales $\int_0^1 x^n \; dx$, $\int_0^{\frac\pi4} \frac{dx}{1+x^2}$,
$\int_1^e \frac{1-x}{x^2}\; dx$, $\int_0^{\frac12} \frac{dx}{x^2-1}$.
  \item Calculer la limite (lorsque $n\to+\infty$) de la somme
$S_n = \sum_{k=1}^n \frac{e^{k/n}}{n}$. 
Idem avec $S'_n = \sum_{k=1}^n \frac{n}{(n+k)^2}$.
\end{enumerate}
\end{miniexercices}


%%%%%%%%%%%%%%%%%%%%%%%%%%%%%%%%%%%%%%%%%%%%%%%%%%%%%%%%%%%%%%%%
\section{Intégration par parties -- Changement de variable}
\label{sec:int4}

Pour trouver une primitive d'une fonction $f$ on peut avoir la chance de reconnaître
que $f$ est la dérivée d'une fonction bien connue. C'est malheureusement très rarement le cas,
et on ne connaît pas les primitives de la plupart des fonctions. Cependant nous allons
voir deux techniques qui permettent des calculer des intégrales et des primitives :
l'intégration par parties et le changement de variable.


%---------------------------------------------------------------
\subsection{Intégration par parties}
\index{integration par parties@intégration par parties}
\begin{theoreme}
Soient $u$ et $v$ deux fonctions de classe $\mathcal{C}^1$ sur un intervalle $[a,b]$.
\mybox{$\displaystyle\int_a^b u(x) \, v'(x)\;dx= \big[uv\big]_a^b - \int_a^b u'(x) \, v(x)\;dx$}
\end{theoreme}

\textbf{Notation.} Le crochet $\big[F\big]_a^b$ est par définition $\big[F\big]_a^b=F(b)-F(a)$.
Donc $\big[uv\big]_a^b = u(b)v(b)-u(a)v(a)$.
Si l'on omet les bornes alors $\big[F\big]$ désigne la fonction $F+c$ où $c$ est une constante quelconque.

La formule d'intégration par parties pour les primitives est la même mais sans les bornes :
$$\int u(x)v'(x)\;dx= \big[uv\big] - \int u'(x)v(x)\;dx.$$


La preuve est très simple :
\begin{proof}
On a $(uv)'=u'v+uv'$. Donc $\int_a^b (u'v+uv')=\int_a^b (uv)'=\big[uv\big]_a^b$.
D'où $\int_a^b uv'= \big[uv\big]_a^b - \int_a^b u'v$.
\end{proof}

L'utilisation de l'intégration par parties repose sur l'idée suivante :
on ne sait pas calculer directement l'intégrale d'une fonction $f$ s'écrivant
comme un produit $f(x)=u(x)v'(x)$ mais si l'on sait calculer l'intégrale
de $g(x)=u'(x)v(x)$ (que l'on espère plus simple) alors par la formule d'intégration par parties
on aura l'intégrale de $f$.

\begin{exemple}
\sauteligne
\begin{enumerate}

  \item Calcul de $\int_0^1 x e^x \; dx$.
On pose $u(x)=x$ et $v'(x)=e^x$.
Nous aurons besoin de savoir que $u'(x)=1$ et qu'une primitive de $v'$ est simplement $v(x)=e^x$.
La formule d'intégration par parties donne :
$$
\begin{array}{rcl}
\int_0^1 x e^x \; dx
  & = & \int_0^1 u(x)v'(x)\;dx \\
  & = & \big[u(x)v(x)\big]_0^1 - \int_0^1 u'(x)v(x)\;dx \\
  & = & \big[x e^x\big]_0^1 \ \  - \  \int_0^1 1\cdot e^x \;dx \\
  & = & \big(1\cdot e^1-0\cdot e^0\big) - \big[e^x\big]_0^1 \\
  & = & e -(e^1-e^0) \\
  & = & 1
\end{array}
$$

  \item Calcul de $\int_1^e x\ln x \; dx$.

On pose cette fois $u=\ln x$ et $v'=x$. Ainsi
$u'=\frac1x$ et $v=\frac{x^2}{2}$.
Alors
\begin{align*}
\int_1^e \ln x \cdot  x\; dx
= \int_1^e  uv' = \big[uv\big]_1^e - \int_1^e u'v
= \big[\ln x \cdot\tfrac{x^2}{2}\big]_1^e - \int_1^e \tfrac 1x \tfrac{x^2}{2} \; dx \\
= \big(\ln e \tfrac{e^2}{2} - \ln 1 \tfrac{1^2}{2} \big) - \tfrac12 \int_1^e x  \; dx
= \tfrac{e^2}{2} -\frac12 \left[ \tfrac{x^2}{2} \right]_1^e
= \tfrac{e^2}{2} - \tfrac{e^2}{4} + \tfrac{1}{4} = \tfrac{e^2+1}{4}\\
\end{align*}



  \item Calcul de $\int \arcsin x \; dx$.

Pour déterminer une primitive de $\arcsin x$,
nous faisons artificiellement apparaître un produit en écrivant $\arcsin x = 1 \cdot \arcsin x$
pour appliquer la formule d'intégration par parties.
On pose $u=\arcsin x$, $v'=1$ (et donc $u'=\frac{1}{\sqrt{1-x^2}}$ et $v=x$) alors
\begin{eqnarray*}
\int 1\cdot \arcsin x \; dx 
  &=& \big[x\arcsin x\big] - \int \frac{x}{\sqrt{1-x^2}} \; dx \\
  &=& \big[x\arcsin x\big] - \big[-\sqrt {1-x^2}\big] \\
  &=& x\arcsin x+ \sqrt {1-x^2}+c
\end{eqnarray*}

  \item Calcul de $\int x^2e^x \; dx$.
On pose $u=x^2$ et $v'=e^x$ pour obtenir :
$$\int x^2e^x \; dx = \big[ x^2e^x \big] - 2\int x e^x \; dx$$
On refait une deuxième intégration par parties pour calculer
$$\int x e^x \; dx = \big[x e^x\big] - \int e^x \; dx = (x-1)e^x+c$$
D'où
$$\int x^2e^x \; dx = (x^2-2x+2) e^x + c.$$
\end{enumerate}
\end{exemple}

\begin{exemple}
Nous allons étudier les intégrales définies par
$I_n=\displaystyle \int_0^1 \frac{\sin(\pi x)}{x+n} \;dx$, pour tout entier $n>0$.
\begin{enumerate}
\item Montrer que $0\le I_{n+1}\le I_n$.

Pour $0\le x\le 1$, on a $0<x+n\le x+n+1$ et $\sin(\pi x)\ge 0$,
donc $0\le \frac{\sin(\pi x)}{x+n+1}\le \frac{\sin(\pi x)}{x+n}$.
D'où $0\le I_{n+1}\le I_n$ par la positivité de l'intégrale.

\item Montrer que $I_n \le \ln\frac{n+1}{n}$. En déduire $\lim_{n\to+\infty}I_n$.

De $0\le \sin(\pi x)\le 1$, on a $\frac{\sin(\pi x)}{x+n}\le
\frac{1}{x+n}$. D'où
$0 \le I_n\le \int_0^1 \frac{1}{x+n} \; dx=\big[\ln(x+n)\big]_0^1=\ln\frac{n+1}{n}\to 0$.

\item Calculer $\lim_{n\to+\infty} n I_n$.

Nous allons faire une intégration par parties avec $u=\frac{1}{x+n}$ et $v'=\sin(\pi x)$
(et donc $u' = - \frac{1}{(x+n)^2}$ et $v = - \frac 1\pi \cos(\pi x)$) :
\begin{eqnarray*}
n I_n 
  &=& n\int_0^1\frac1{x+n} \sin(\pi x) \; dx \\
  &=& - \frac n\pi\left[\frac{1}{x+n} \cos(\pi x)\right]_0^1
      - \frac n\pi \int_0^1 \frac{1}{(x+n)^2}\cos(\pi x) \; dx \\
  &=& \frac{n}{\pi(n+1)}+\frac{1}{\pi} - \frac{n}{\pi}J_n
\end{eqnarray*}
Il nous reste à évaluer $J_n = \int_0^1 \frac{\cos(\pi x)}{(x+n)^2} \; dx$.
\begin{multline*}
\left|\frac{n}{\pi}J_n\right| \le \frac{n}{\pi}\int_0^1 \frac{|\cos(\pi x)|}{(x+n)^2} \; dx
\le\frac{n}{\pi}\int_0^1\frac{1}{(x+n)^2} \; dx \\
=\frac{n}{\pi}\left[-\frac{1}{x+n}\right]_0^1
=\frac{n}{\pi}\left(-\frac{1}{1+n}+\frac{1}{n}\right) =\frac{1}{\pi}\frac{1}{n+1} \to0.
\end{multline*}

Donc $\lim_{n\to+\infty} n I_n =\lim_{n\to+\infty} \frac{n}{\pi(n+1)}+\frac{1}{\pi} - \frac{n}{\pi}J_n =\frac{2}{\pi}.$
\end{enumerate}
\end{exemple}


%---------------------------------------------------------------

\subsection{Changement de variable}
\index{integrale@intégrale!changement de variable}

\begin{theoreme}
Soit $f$ une fonction définie sur un intervalle $I$ et $\varphi : J \to I$ une bijection de classe $\mathcal{C}^1$.
Pour tout $a,b\in J$
\mybox{$\displaystyle\int_{\varphi(a)}^{\varphi(b)} f(x) \; dx = \int_a^b f\big(\varphi(t)\big)\cdot\varphi'(t) \; dt$}

Si $F$ est une primitive de $f$ alors $F\circ \varphi$ est une primitive de
$\big(f \circ \varphi\big)\cdot\varphi'$.
%Autrement dit: \mybox{$\displaystyle \left( \int f(x) \; dx \right) \circ \varphi
%= \int f\big(\varphi(t)\big)\varphi'(t) \; dt$}
\end{theoreme}

%C'est-à-dire qu'une primitive de $f(\varphi(t))\varphi'(t)$ s'obtient à partir
%de celle de $f$ en composant avec $\varphi$.

%La formule $\int f(x) \; dx = \int f\big(\varphi(t)\big)\varphi'(t) \; dt$ est bien un changement de variable.
Voici un moyen simple de s'en souvenir. En effet si l'on note
$x=\varphi(t)$ alors par dérivation on obtient $\frac{dx}{dt} = \varphi'(t)$ donc
$dx = \varphi'(t) \; dt$. D'où la substitution
$\int_{\varphi(a)}^{\varphi(b)} f(x) \; dx = \int_a^b f(\varphi(t)) \; \varphi'(t) \; dt$.

\begin{proof}
Comme $F$ est une primitive de $f$ alors $F'(x)=f(x)$ et
par la formule de la dérivation de la composition $F\circ \varphi$ on a
$$(F\circ\varphi)'(t)=F'(\varphi(t))\varphi'(t)=f(\varphi(t))\varphi'(t).$$
Donc $F\circ \varphi$ est une primitive de $f(\varphi(t))\varphi'(t)$.
% c'est-à-dire $\displaystyle\int f(\varphi(t))\varphi'(t) \; dt =\int f(x) \; dx$.

Pour les intégrales :
 $\displaystyle\int_a^b f(\varphi(t))\varphi'(t) \; dt = \big[ F\circ \varphi\big]_a^b
= F\big(\varphi(b)\big)-F\big(\varphi(a)\big) = \big[ F\big]_{\varphi(a)}^{\varphi(b)}
= \int_{\varphi(a)}^{\varphi(b)} f(x) \; dx$.
\end{proof}


\begin{remarque*}
Comme $\varphi$ est une bijection de $J$ sur $\varphi(J)$, sa réciproque $\varphi^{-1}$ existe et
est dérivable sauf quand $\varphi$ s'annule. Si $\varphi$ ne s'annule pas, on peut écrire $t=\varphi^{-1}(x)$
et faire un changement de variable en sens inverse.
\end{remarque*}


\begin{exemple}
Calculons la primitive $F = \int \tan t \; dt$.
\[
F=\int \tan t \; dt = \int \frac{\sin t}{\cos t} \; dt \; .
\]
On reconnaît ici une forme $\frac{u'}{u}$ (avec $u=\cos t$ et $u'=-\sin t$) dont une primitive est $\ln|u|$.
Donc $F = \int -\frac{u'}{u} = -\big[\ln |u| \big] = -\ln|u|+c = -\ln|\cos t|+c$.

\bigskip

Nous allons reformuler tout cela en terme de changement de variable.
Notons $\varphi(t)= \cos t$ alors $\varphi'(t) = -\sin t$, donc
\[
F = \int -\frac{\varphi'(t)}{\varphi(t)} \; dt
\]
Si $f$ désigne la fonction définie par $f(x)=\frac1x$, qui est bijective tant que $x \neq 0$;
alors $F = - \int \varphi'(t) f(\varphi(t))\; dt$.
En posant $x = \varphi(t)$ et donc $dx = \varphi'(t) dt$, on reconnaît la formule du changement de variable,
par conséquent
\[
F \circ \varphi^{-1} = -\int f(x) \; dx=-\int \frac1x \;dx
= -\ln|x|+c \; .
\]
Comme $x = \varphi(t)=\cos t$, on retrouve bien $F(t) = -\ln|\cos t| + c$.

Remarque : pour que l'intégrale soit bien définie il faut que $\tan t$ soit définie,
donc $t \not\equiv \frac{\pi}{2} \bmod \pi$. La restriction d'une primitive à un intervalle
$]-\frac{\pi}{2} + k \pi, \frac{\pi}{2} + k \pi[$ est donc de la forme $-\ln|\cos t| + c$.
Mais la constante $c$ peut être différente sur un intervalle différent.
\end{exemple}

\begin{exemple}
Calcul de $\int_0^{1/2}\frac{x}{(1-x^2)^{3/2}} \;dx$.

Soit le changement de variable $u=\varphi(x) = 1-x^2$. Alors $du = \varphi'(x) \; dx = -2x \; dx$.
Pour $x=0$ on a $u=\varphi(0)=1$ et pour $x=\frac12$ on a $u=\varphi(\frac{1}{2})=\frac34$.
Comme $\varphi'(x)=-2x$, $\varphi$ est une bijection de $[0,\frac{1}{2}]$ sur $[1,\frac{3}{4}]$. Alors
\begin{eqnarray*}
\int_0^{1/2}\frac{x \; dx}{(1-x^2)^{3/2}} 
  &=& \int_1^{3/4} \frac{\frac{-du}{2}}{u^{3/2}}
  = -\frac12\int_1^{3/4} u^{-3/2}\;du \\
  &=& -\frac12\big[-2u^{-1/2}\big]_1^{3/4}
  =\big[\frac1{\sqrt{u}}\big]_1^{3/4} 
  = \frac1{\sqrt{\frac34}}-1= \frac{2}{\sqrt3}-1.
\end{eqnarray*}
\end{exemple}

\begin{exemple}
Calcul de $\int_0^{1/2}\frac{1}{(1-x^2)^{3/2}} \;dx$.

On effectue le changement de variable $x=\varphi(t) = \sin t$ et $dx = \cos t \; dt$.
De plus $t=\arcsin x$ donc pour $x=0$ on a $t=\arcsin(0)=0$ et pour $x=\frac12$ on a $t=\arcsin(\frac12)=\frac\pi6$. Comme $\varphi$ est une bijection de $[0,\frac\pi6]$ sur $[0,\frac12]$,
\begin{eqnarray*}
\int_0^{1/2} \frac{dx}{(1-x^2)^{3/2}}
&=& \int_0^{\pi/6}\frac{\cos t \; dt}{(1-\sin^2 t)^{3/2}}
= \int_0^{\pi/6}\frac{\cos t \; dt}{(\cos^2 t)^{3/2}} \\
&=& \int_0^{\pi/6}\frac{\cos t}{\cos^3 t} \;dt
=  \int_0^{\pi/6}\frac{1}{\cos^2 t} \;dt
= \big[\tan t\big]_0^{\pi/6}
=\frac{1}{\sqrt{3}} \; .
\end{eqnarray*}
\end{exemple}


\begin{exemple}
Calcul de $\int \frac{1}{(1+x^2)^{3/2}} \; dx$.

Soit le changement de variable $x=\tan t$ donc $t = \arctan x$ et $dx = \frac{dt}{\cos^2 t}$
(la fonction tangente établit une bijection de $]-\frac\pi2,+\frac\pi2[$ sur $\Rr$).
Donc
\begin{align*}
F & = \int\frac{1}{(1+x^2)^{3/2}} \; dx  = \int \frac{1}{(1+\tan^2 t)^{3/2}}\; \frac{dt}{\cos^2 t} \\
  & = \int (\cos^2 t)^{3/2}\; \frac{dt}{\cos^2 t} \qquad \text{ car } 1+\tan^2t=\frac{1}{\cos^2t} \\
  & = \int \cos t\; dt
    = \big[ \sin t\big]
    = \sin t + c
    = \sin(\arctan x)+c \\
\end{align*}


Donc
$$\int \frac{1}{(1+x^2)^{3/2}}\;dx=\sin(\arctan x)+c.$$

En manipulant un peu les fonctions on trouverait que la primitive s'écrit aussi
$F(x)=\frac{x}{\sqrt{1+x^2}}+c$.
\end{exemple}


%---------------------------------------------------------------
%\subsection{Mini-exercices}

\begin{miniexercices}
\sauteligne
\begin{enumerate}
  \item Calculer les intégrales à l'aide d'intégrations par parties:
  $\int_0^{\pi/2} t \sin t \; dt$, $\int_0^{\pi/2} t^2 \sin t \; dt$, puis par récurrence
  $\int_0^{\pi/2} t^n \sin t \; dt$.

  \item Déterminer les primitives à l'aide d'intégrations par parties:
  $\int t \sh t \; dt$, $\int t^2 \sh t \; dt$, puis par récurrence
  $\int t^n \sh t \; dt$.

  \item Calculer les intégrales à l'aide de changements de variable:
  $\int_0^a \sqrt{a^2 - t^2} \; dt$ ;
  $\int_{-\pi}^\pi \sqrt{1+\cos t} \; dt$ (pour ce dernier poser deux changements de variables :
  $u = \cos t$, puis $v=1-u$).

  \item Déterminer les primitives suivantes à l'aide de changements de variable :
  $\int \tanh t \; dt$ où $\tanh t = \frac{\sh t}{\ch t}$,
  $\int e^{\sqrt{t}} \; dt$.
\end{enumerate}
\end{miniexercices}


%%%%%%%%%%%%%%%%%%%%%%%%%%%%%%%%%%%%%%%%%%%%%%%%%%%%%%%%%%%%%%%%
\section{Intégration des fractions rationnelles}


Nous savons intégrer beaucoup de fonctions simples. Par exemple
toutes les fonctions polynomiales: si $f(x)=a_0+a_1x+a_2x^2+\cdots+ a_n x^n$
alors $\int f(x)\; dx = a_0x+a_1\frac{x^2}{2}+a_2\frac{x^3}{3}+\cdots+a_n\frac{x^{n+1}}{n+1}+c$.

\bigskip

Il faut être conscient cependant que beaucoup de fonctions ne s'intègrent
pas à l'aide de fonctions simples.
Par exemple si $f(t)=\sqrt{a^2\cos^2 t+ b^2 \sin^2 t}$ alors
l'intégrale $\int_0^{2\pi} f(t) \; dt$ ne peut pas s'exprimer
comme somme, produit, inverse ou composition de fonctions que vous connaissez.
En fait cette intégrale vaut la longueur d'une ellipse
d'équation paramétrique $(a\cos t, b\sin t)$ ; il n'y a donc pas de formule
pour le périmètre d'une ellipse (sauf si $a=b$ auquel cas l'ellipse est un cercle !).

%[[dessin ellipse avec demi-grand axe]]
\myfigure{1.2}{
\tikzinput{fig_int09}
}

\bigskip

Mais de façon remarquable, il y a toute une famille de fonctions que l'on saura
intégrer : les fractions rationnelles.


%---------------------------------------------------------------
\subsection{Trois situations de base}

On souhaite d'abord intégrer les fractions rationnelles $f(x)=\frac{\alpha x + \beta}{a x^2+b x+c}$
avec $\alpha, \beta, a, b, c \in \Rr$, $a\neq 0$ et $(\alpha,\beta)\neq (0,0)$.

\textbf{Premier cas.} Le dénominateur $a x^2+b x+c$ possède deux racines réelles distinctes $x_1,x_ 2\in \Rr$.


Alors $f(x)$ s'écrit aussi $f(x)=\frac{\alpha x + \beta}{a(x - x_1)(x - x_2)}$ et il existe des nombres $A,B \in \Rr$
tels que $f(x)=\frac{A}{x - x_1}+\frac{B}{x -x_2}$. On a donc
$$\int f(x)\;dx = A \ln|x - x_1|+B\ln|x -x_2|+c$$
sur chacun des intervalles
$]-\infty,x_1[$, $]x_1,x_2[$, $]x_2,+\infty[$ (si $x_1<x_2$).


\bigskip

\textbf{Deuxième cas.} Le dénominateur $a x^2+b x+c$ possède une racine double $x_0 \in \Rr$.


Alors $f(x)=\frac{\alpha x + \beta}{a(x -x_0)^2}$ et il existe des nombres $A,B \in \Rr$
tels que $f(x)=\frac{A}{(x - x_0)^2}+\frac{B}{x - x_0}$. On a alors
$$\int f(x)\;dx = -\frac{A}{x - x_0} + B\ln|x - x_0|+c$$
sur chacun des intervalles
$]-\infty,x_0[$, $]x_0,+\infty[$.

\bigskip

\textbf{Troisième cas.}   Le dénominateur $a x^2+b x+c$ ne possède pas de racine réelle.
Voyons comment faire sur un exemple.

\begin{exemple}
Soit $f(x)=\frac{x+1}{2x^2+x+1}$.
Dans un premier temps on fait apparaître une fraction du type $\frac{u'}{u}$
(que l'on sait intégrer en $\ln|u|$).
$$f(x) = \frac{(4x+1)\frac14-\frac14+1}{2x^2+x+1} = \frac14 \cdot \frac{4x+1}{2x^2+x+1} + \frac34 \cdot \frac{1}{2x^2+x+1}$$

On peut intégrer la fraction $\frac{4x+1}{2x^2+x+1}$ :
$$\int \frac{4x+1}{2x^2+x+1} \; dx = \int \frac{u'(x)}{u(x)} \; dx = \ln \big| 2x^2+x+1 \big|+c$$

\bigskip

Occupons nous de l'autre partie $\frac{1}{2x^2+x+1}$, nous allons l'écrire sous la forme
$\frac{1}{u^2+1}$ (dont une primitive est $\arctan u$).

\begin{eqnarray*}
\frac{1}{2x^2+x+1} 
&=&  \frac{1}{2(x+\frac 14)^2-\frac18+1}
=\frac{1}{2(x+\frac 14)^2+\frac78}\\
&=&\frac87 \cdot \frac{1}{\frac87 2(x+\frac 14)^2+1}
= \frac87 \cdot \frac{1}{\big(\frac{4}{\sqrt7}(x+\frac 14)\big)^2+1}
\end{eqnarray*}
On pose le changement de variable $u= \frac{4}{\sqrt7}(x+\frac 14)$
(et donc $du = \frac{4}{\sqrt7} dx$) pour trouver
\begin{eqnarray*}
\int \frac{dx}{2x^2+x+1}
&=& \int \frac87 \frac{dx}{\big(\frac{4}{\sqrt7}(x+\frac 14)\big)^2+1}
= \frac 87 \int \frac{du}{u^2+1} \cdot \frac{\sqrt7}{4} \\
&=& \frac{2}{\sqrt7}\arctan u+ c
= \frac{2}{\sqrt7}\arctan \left(\frac{4}{\sqrt7}\big(x+\frac 14\big)\right) + c \; .
\end{eqnarray*}

\bigskip

Finalement :
$$\int f(x)\; dx = \frac14\ln \big(2x^2+x+1\big) + \frac{3}{2\sqrt7}\arctan \left(\frac{4}{\sqrt7}\big(x+\frac 14\big)\right)+c$$
\end{exemple}


%---------------------------------------------------------------
\subsection{Intégration des éléments simples}

Soit $\frac{P(x)}{Q(x)}$ une fraction rationnelle, où $P(x),Q(x)$ sont des polynômes à coefficients réels.
Alors la fraction $\frac{P(x)}{Q(x)}$ s'écrit comme somme d'un polynôme $E(x) \in \Rr[x]$ (la partie entière)
et d'éléments simples d'une des formes suivantes :
$$\frac{\gamma}{(x - x_0)^k} \quad \text{ ou } \quad \frac{\alpha x+\beta}{(a x^2+b x+c)^k} \text{ avec } b^2-4ac < 0$$
où $\alpha,\beta,\gamma,a,b,c \in \Rr$ et $k \in \Nn\setminus\{0\}$.

\begin{enumerate}
  \item On sait intégrer le polynôme $E(x)$.

  \item Intégration de l'élément simple $\frac{\gamma}{(x - x_0)^k}$.
  \begin{enumerate}
     \item Si $k=1$ alors $\int \frac{\gamma \; dx}{x - x_0} = \gamma \ln|x - x_0|+c_0$
(sur $]-\infty,x_0[$ ou $]x_0,+\infty[$).
     \item Si $k\ge 2$ alors  $\int \frac{\gamma \; dx}{(x - x_0)^k} = \gamma \int (x - x_0)^{-k} \; dx
= \frac{\gamma}{-k+1}(x - x_0)^{-k+1}+c_0$ (sur $]-\infty,x_0[$ ou $]x_0,+\infty[$).
  \end{enumerate}

  \item Intégration de l'élément simple $\frac{\alpha x+\beta}{(a x^2+b x+c)^k}$.
On écrit cette fraction sous la forme
$$\frac{\alpha x+\beta}{(a x^2+b x+c)^k} = \gamma \frac{2a x+b}{(a x^2+b x+c)^k} + \delta \frac{1}{(a x^2+b x+c)^k}$$
  \begin{enumerate}
     \item Si $k=1$, calcul de $\int \frac{2a x+b}{a x^2+b x+c} \; dx = \int \frac{u'(x)}{u(x)} \; dx = \ln |u(x)| + c_0
= \ln |a x^2+b x+c|+c_0$.
     
     \item Si $k\ge 2$, calcul de $\int \frac{2a x+b}{(a x^2+b x+c)^k} \; dx = \int \frac{u'(x)}{u(x)^k} \; dx = \frac{1}{-k+1}u(x)^{-k+1}+c_0
= \frac{1}{-k+1}(a x^2+b x+c)^{-k+1}+c_0$.

     \item Si $k=1$, calcul de $\int \frac{1}{a x^2+b x+c}\; dx$. Par un changement de variable $u= px+q$ on se ramène à
calculer une primitive du type
$\int \frac{du}{u^2+1}=\arctan u + c_0$.

     \item Si $k\ge 2$, calcul de $\int \frac{1}{(a x^2+b x+c)^k} \; dx$. On effectue le changement de variable $u=px+q$
pour se ramener au calcul de $I_k =\int \frac{du}{(u^2+1)^k}$.
Une intégration par parties permet de passer de $I_k$ à $I_{k-1}$.

Par exemple calculons $I_2$. Partant de $I_1=\int \frac{du}{u^2+1}$ on pose
$f=\frac{1}{u^2+1}$ et $g'=1$. La formule d'intégration par parties $\int f g'=[f g]-\int f' g$ donne
(avec $f'=-\frac{2u}{(u^2+1)^2}$ et $g=u$)
\begin{eqnarray*}
I_1
  & = & \int \frac{du}{u^2+1} = \left[ \frac{u}{u^2+1} \right] + \int \frac{2u^2 \; du}{(u^2+1)^2}
        = \left[ \frac{u}{u^2+1} \right] + 2 \int  \frac{u^2+1 - 1}{(u^2+1)^2}du \\
 & = & \left[ \frac{u}{u^2+1} \right] + 2 \int \frac{du}{u^2+1} - 2 \int \frac{du}{(u^2+1)^2}
    = \left[ \frac{u}{u^2+1} \right] + 2I_1 - 2I_2
\end{eqnarray*}
On en déduit
$I_2 = \frac12 I_1 + \frac12\frac{u}{u^2+1} + c_0$.
Mais comme $I_1=\arctan u$ alors
$$I_2 = \int \frac{du}{(u^2+1)^2} =  \frac12 \arctan u + \frac12\frac{u}{u^2+1} + c_0.$$
  \end{enumerate}
\end{enumerate}


%---------------------------------------------------------------
\subsection{Intégration des fonctions trigonométriques}

On peut aussi calculer les primitives de la forme $\int P(\cos x,\sin x)\;dx$ ou de la forme
$\int \frac{P(\cos x,\sin x)}{Q(\cos x, \sin x)}\;dx$
quand $P$ et $Q$ sont des polynômes, en se ramenant à intégrer une fraction rationnelle.

Il existe deux méthodes :
\begin{itemize}
  \item les règles de Bioche sont assez efficaces mais ne fonctionnent pas toujours ;
  \item le changement de variable $t = \tan \frac x2$ fonctionne tout le temps mais conduit à davantage de calculs.
\end{itemize}

\bigskip

\textbf{Les règles de Bioche.}\index{regle@règle!de Bioche}
On note $\omega(x) = f(x)\;dx$.
On a alors $\omega(-x)= f(-x)\;d(-x)=-f(-x)\;dx$ et
$\omega (\pi-x)= f(\pi-x)\;d(\pi-x)=-f(\pi-x)\;dx$.



\begin{itemize}
  \item Si $\omega(-x)=\omega(x)$ alors on effectue le changement de variable $u=\cos x$.

  \item Si $\omega(\pi-x)=\omega(x)$ alors on effectue le changement de variable $u=\sin x$.

  \item Si $\omega(\pi + x)=\omega(x)$ alors on effectue le changement de variable $u=\tan x$.
\end{itemize}

\begin{exemple}
Calcul de la primitive $\int \frac{\cos x \; dx}{2-\cos^2 x}$

On note $\omega(x)= \frac{\cos x \; dx}{2-\cos^2 x}$.
Comme $\omega(\pi-x)=\frac{\cos(\pi-x) \; d(\pi-x)}{2-\cos^2 (\pi-x)} = \frac{(-\cos x) \; (-dx)}{2-\cos^2 x}
= \omega(x)$ alors le changement de variable qui convient est $u = \sin x$
pour lequel $du= \cos x \; dx$. Ainsi:
\[
\int \frac{\cos x \; dx}{2-\cos^2 x}
= \int \frac{\cos x \; dx}{2-(1-\sin^2 x)}
= \int \frac{du}{1+u^2} = \big[ \arctan u \big]
= \arctan (\sin x) + c \; .
\]
\end{exemple}


\bigskip

\textbf{Le changement de variable $t=\tan \frac x2$.}

Les formules de la \og tangente de l'arc moitié \fg{} permettent d'exprimer sinus, cosinus et tangente
en fonction de $\tan \frac x2$.
\mybox{
\begin{tabular}{c}
Avec \quad  $\displaystyle t=\tan \frac{x}{2}$ \quad on a \\[2ex]
$\displaystyle\cos x = \frac {1-t^2}{1+t^2} 
\qquad  \sin x = \frac{2t}{1+t^2} 
\qquad \tan x = \frac{2t}{1-t^2}$ \\[2ex]
et \qquad  $\displaystyle dx=\dfrac{2\;dt}{1+t^2}.$
\end{tabular}
}

\begin{exemple}
Calcul de l'intégrale $\int_{-\pi/2}^0 \frac{dx}{1-\sin x}$.

Le changement de variable $t=\tan \frac{x}{2}$ définit une bijection de $[-\frac\pi2,0]$ vers $[-1,0]$
(pour $x=-\frac\pi2$, $t=-1$ et pour $x=0$, $t=0$). De plus on a $\sin x = \frac{2t}{1+t^2}$
et $dx=\frac{2\;dt}{1+t^2}$.
\begin{eqnarray*}
\int_{-\frac\pi2}^0 \frac{dx}{1-\sin x}
&=& \int_{-1}^0 \frac{\frac{2\;dt}{1+t^2}}{1-\frac{2t}{1+t^2}}
= 2\int_{-1}^0 \frac{dt}{1+t^2-2t}\\
&=& 2\int_{-1}^0 \frac{dt}{(1-t)^2}
= 2 \left[\frac{1}{1-t}\right]_{-1}^0
= 2\big(1-\frac12\big)
= 1  
\end{eqnarray*}


\end{exemple}




%---------------------------------------------------------------
%\subsection{Mini-exercices}

\begin{miniexercices}
\sauteligne
\begin{enumerate}
  \item Calculer les primitives $\int \frac{4x+5}{x^2+x-2}\; dx$,
$\int \frac{6-x}{x^2-4x+4}\; dx$, $\int \frac{2x-4}{(x-2)^2+1}\; dx$, $\int \frac{1}{(x-2)^2+1}\; dx$.

  \item Calculer les primitives $I_k = \int \frac{dx}{(x-1)^k}$ pour tout $k \ge 1$.
Idem avec $J_k = \int \frac{x\; dx}{(x^2+1)^k}$.

  \item Calculer les intégrales suivantes :
$\int_0^1 \frac{dx}{x^2+x+1}$, $\int_0^1 \frac{x \; dx}{x^2+x+1}$, $\int_0^1\frac{x\; dx}{(x^2+x+1)^2}$,
$\int_0^1\frac{dx}{(x^2+x+1)^2}$.

  \item Calculer les intégrales suivantes :
$\int_{-\frac\pi2}^{\frac\pi2} \sin^2 x \cos^3 x \; dx$,
$\int_{0}^{\frac\pi2} \cos^4 x \; dx$,
$\int_0^{2\pi} \frac{dx}{2+\sin x}$.
\end{enumerate}
\end{miniexercices}


\auteurs{

Rédaction : Arnaud Bodin

Basé sur des cours de Guoting Chen et Marc Bourdon

Relecture : Pascal Romon

Dessins : Benjamin Boutin
}

\finchapitre
\end{document}
