
%%%%%%%%%%%%%%%%%% PREAMBULE %%%%%%%%%%%%%%%%%%


\documentclass[12pt]{article}

\usepackage{amsfonts,amsmath,amssymb,amsthm}
\usepackage[utf8]{inputenc}
\usepackage[T1]{fontenc}
\usepackage[francais]{babel}


% packages
\usepackage{amsfonts,amsmath,amssymb,amsthm}
\usepackage[utf8]{inputenc}
\usepackage[T1]{fontenc}
%\usepackage{lmodern}

\usepackage[francais]{babel}
\usepackage{fancybox}
\usepackage{graphicx}

\usepackage{float}

%\usepackage[usenames, x11names]{xcolor}
\usepackage{tikz}
\usepackage{datetime}

\usepackage{mathptmx}
%\usepackage{fouriernc}
%\usepackage{newcent}
\usepackage[mathcal,mathbf]{euler}

%\usepackage{palatino}
%\usepackage{newcent}


% Commande spéciale prompteur

%\usepackage{mathptmx}
%\usepackage[mathcal,mathbf]{euler}
%\usepackage{mathpple,multido}

\usepackage[a4paper]{geometry}
\geometry{top=2cm, bottom=2cm, left=1cm, right=1cm, marginparsep=1cm}

\newcommand{\change}{{\color{red}\rule{\textwidth}{1mm}\\}}

\newcounter{mydiapo}

\newcommand{\diapo}{\newpage
\hfill {\normalsize  Diapo \themydiapo \quad \texttt{[\jobname]}} \\
\stepcounter{mydiapo}}


%%%%%%% COULEURS %%%%%%%%%%

% Pour blanc sur noir :
%\pagecolor[rgb]{0.5,0.5,0.5}
% \pagecolor[rgb]{0,0,0}
% \color[rgb]{1,1,1}



%\DeclareFixedFont{\myfont}{U}{cmss}{bx}{n}{18pt}
\newcommand{\debuttexte}{
%%%%%%%%%%%%% FONTES %%%%%%%%%%%%%
\renewcommand{\baselinestretch}{1.5}
\usefont{U}{cmss}{bx}{n}
\bfseries

% Taille normale : commenter le reste !
%Taille Arnaud
%\fontsize{19}{19}\selectfont

% Taille Barbara
%\fontsize{21}{22}\selectfont

%Taille François
\fontsize{25}{30}\selectfont

%Taille Pascal
%\fontsize{25}{30}\selectfont

%Taille Laura
%\fontsize{30}{35}\selectfont


%\myfont
%\usefont{U}{cmss}{bx}{n}

%\Huge
%\addtolength{\parskip}{\baselineskip}
}


% \usepackage{hyperref}
% \hypersetup{colorlinks=true, linkcolor=blue, urlcolor=blue,
% pdftitle={Exo7 - Exercices de mathématiques}, pdfauthor={Exo7}}


%section
% \usepackage{sectsty}
% \allsectionsfont{\bf}
%\sectionfont{\color{Tomato3}\upshape\selectfont}
%\subsectionfont{\color{Tomato4}\upshape\selectfont}

%----- Ensembles : entiers, reels, complexes -----
\newcommand{\Nn}{\mathbb{N}} \newcommand{\N}{\mathbb{N}}
\newcommand{\Zz}{\mathbb{Z}} \newcommand{\Z}{\mathbb{Z}}
\newcommand{\Qq}{\mathbb{Q}} \newcommand{\Q}{\mathbb{Q}}
\newcommand{\Rr}{\mathbb{R}} \newcommand{\R}{\mathbb{R}}
\newcommand{\Cc}{\mathbb{C}} 
\newcommand{\Kk}{\mathbb{K}} \newcommand{\K}{\mathbb{K}}

%----- Modifications de symboles -----
\renewcommand{\epsilon}{\varepsilon}
\renewcommand{\Re}{\mathop{\text{Re}}\nolimits}
\renewcommand{\Im}{\mathop{\text{Im}}\nolimits}
%\newcommand{\llbracket}{\left[\kern-0.15em\left[}
%\newcommand{\rrbracket}{\right]\kern-0.15em\right]}

\renewcommand{\ge}{\geqslant}
\renewcommand{\geq}{\geqslant}
\renewcommand{\le}{\leqslant}
\renewcommand{\leq}{\leqslant}

%----- Fonctions usuelles -----
\newcommand{\ch}{\mathop{\mathrm{ch}}\nolimits}
\newcommand{\sh}{\mathop{\mathrm{sh}}\nolimits}
\renewcommand{\tanh}{\mathop{\mathrm{th}}\nolimits}
\newcommand{\cotan}{\mathop{\mathrm{cotan}}\nolimits}
\newcommand{\Arcsin}{\mathop{\mathrm{Arcsin}}\nolimits}
\newcommand{\Arccos}{\mathop{\mathrm{Arccos}}\nolimits}
\newcommand{\Arctan}{\mathop{\mathrm{Arctan}}\nolimits}
\newcommand{\Argsh}{\mathop{\mathrm{Argsh}}\nolimits}
\newcommand{\Argch}{\mathop{\mathrm{Argch}}\nolimits}
\newcommand{\Argth}{\mathop{\mathrm{Argth}}\nolimits}
\newcommand{\pgcd}{\mathop{\mathrm{pgcd}}\nolimits} 

\newcommand{\Card}{\mathop{\text{Card}}\nolimits}
\newcommand{\Ker}{\mathop{\text{Ker}}\nolimits}
\newcommand{\id}{\mathop{\text{id}}\nolimits}
\newcommand{\ii}{\mathrm{i}}
\newcommand{\dd}{\mathrm{d}}
\newcommand{\Vect}{\mathop{\text{Vect}}\nolimits}
\newcommand{\Mat}{\mathop{\mathrm{Mat}}\nolimits}
\newcommand{\rg}{\mathop{\text{rg}}\nolimits}
\newcommand{\tr}{\mathop{\text{tr}}\nolimits}
\newcommand{\ppcm}{\mathop{\text{ppcm}}\nolimits}

%----- Structure des exercices ------

\newtheoremstyle{styleexo}% name
{2ex}% Space above
{3ex}% Space below
{}% Body font
{}% Indent amount 1
{\bfseries} % Theorem head font
{}% Punctuation after theorem head
{\newline}% Space after theorem head 2
{}% Theorem head spec (can be left empty, meaning ‘normal’)

%\theoremstyle{styleexo}
\newtheorem{exo}{Exercice}
\newtheorem{ind}{Indications}
\newtheorem{cor}{Correction}


\newcommand{\exercice}[1]{} \newcommand{\finexercice}{}
%\newcommand{\exercice}[1]{{\tiny\texttt{#1}}\vspace{-2ex}} % pour afficher le numero absolu, l'auteur...
\newcommand{\enonce}{\begin{exo}} \newcommand{\finenonce}{\end{exo}}
\newcommand{\indication}{\begin{ind}} \newcommand{\finindication}{\end{ind}}
\newcommand{\correction}{\begin{cor}} \newcommand{\fincorrection}{\end{cor}}

\newcommand{\noindication}{\stepcounter{ind}}
\newcommand{\nocorrection}{\stepcounter{cor}}

\newcommand{\fiche}[1]{} \newcommand{\finfiche}{}
\newcommand{\titre}[1]{\centerline{\large \bf #1}}
\newcommand{\addcommand}[1]{}
\newcommand{\video}[1]{}

% Marge
\newcommand{\mymargin}[1]{\marginpar{{\small #1}}}



%----- Presentation ------
\setlength{\parindent}{0cm}

%\newcommand{\ExoSept}{\href{http://exo7.emath.fr}{\textbf{\textsf{Exo7}}}}

\definecolor{myred}{rgb}{0.93,0.26,0}
\definecolor{myorange}{rgb}{0.97,0.58,0}
\definecolor{myyellow}{rgb}{1,0.86,0}

\newcommand{\LogoExoSept}[1]{  % input : echelle
{\usefont{U}{cmss}{bx}{n}
\begin{tikzpicture}[scale=0.1*#1,transform shape]
  \fill[color=myorange] (0,0)--(4,0)--(4,-4)--(0,-4)--cycle;
  \fill[color=myred] (0,0)--(0,3)--(-3,3)--(-3,0)--cycle;
  \fill[color=myyellow] (4,0)--(7,4)--(3,7)--(0,3)--cycle;
  \node[scale=5] at (3.5,3.5) {Exo7};
\end{tikzpicture}}
}



\theoremstyle{definition}
%\newtheorem{proposition}{Proposition}
%\newtheorem{exemple}{Exemple}
%\newtheorem{theoreme}{Théorème}
\newtheorem{lemme}{Lemme}
\newtheorem{corollaire}{Corollaire}
%\newtheorem*{remarque*}{Remarque}
%\newtheorem*{miniexercice}{Mini-exercices}
%\newtheorem{definition}{Définition}




%definition d'un terme
\newcommand{\defi}[1]{{\color{myorange}\textbf{\emph{#1}}}}
\newcommand{\evidence}[1]{{\color{blue}\textbf{\emph{#1}}}}



 %----- Commandes divers ------

\newcommand{\codeinline}[1]{\texttt{#1}}

%%%%%%%%%%%%%%%%%%%%%%%%%%%%%%%%%%%%%%%%%%%%%%%%%%%%%%%%%%%%%
%%%%%%%%%%%%%%%%%%%%%%%%%%%%%%%%%%%%%%%%%%%%%%%%%%%%%%%%%%%%%



\begin{document}

\debuttexte


%%%%%%%%%%%%%%%%%%%%%%%%%%%%%%%%%%%%%%%%%%%%%%%%%%%%%%%%%%%
\diapo

Nous allons étudier les équations différentielles linéaires 
du second ordre, à coefficients constants.

\change

\change
On commence par la définition,

\change
le cas des équations homogènes,

\change
puis le cas avec avec second membre,

\change
qui implique la recherche d'une solution particulière.


%%%%%%%%%%%%%%%%%%%%%%%%%%%%%%%%%%%%%%%%%%%%%%%%%%%%%%%%%%%
\diapo



Une équation différentielle linéaire du second ordre, à
 coefficients constants, est une équation de la forme :
\begin{equation}
ay''+by'+cy=g(x) 
\label{eq:linscd}
\tag{$E$}
\end{equation}
où $a,b,c$ sont des constantes réelles, d'où le nom,

par contre $g$ est une fonction continue définie sur un intervalle ouvert $I$.

\change

A une telle équation on associe comme d'habitude 
une autre équation différentielle :
\begin{equation}
ay''+by'+cy=0 
\label{eq:linscdhom}
\tag{$E_0$}
\end{equation}
appelée l'équation homogène associée à la précédente.


\change

La structure des solutions de l'équation est très simple :

Théorème : 
L'ensemble des solutions de l'équation homogène est 
un $\Rr$-espace vectoriel de dimension~$2$.


Nous admettons ce résultat. Nous avions vu que pour les équations différentielles
linéaires d'ordre 1, les solutions formaient un ev de dimension 1, ici
pour l'ordre 2 c'est un ev de dimension 2. 

%%%%%%%%%%%%%%%%%%%%%%%%%%%%%%%%%%%%%%%%%%%%%%%%%%%%%%%%%%%
\diapo

Voyons comment résoudre les équations différentielles linéaires
du second ordre à  coefficients constants sans second membre.

\change
A l'équation différentielle $ay''+by'+cy=0$, on associe
\defi{l'équation caractéristique} $ar^2+br+c=0$.

\change
On note comme d'habitude $\Delta= b^2-4ac$, le discriminant 
de l'équation caractéristique. 


\change
Nous allons expliciter les solutions de l'équations différentielles
à l'aide des solutions de l'équations caractéristique et donc en fonction du discriminant.

Tout d'abord si $\Delta >0$, alors on sait que 
l'équation caractéristique possède deux racines réelles distinctes 
notons-les $r_1$ et $r_2$ 

\change
Dans ce cas les solutions de l'équation différentielles sont exactement
les
$$y(x) = \lambda e^{r_1x}+ \mu e^{r_2x}$$
avec $\lambda, \mu$ deux constantes réelles quelconques.

\change
Dans le cas $\Delta=0$, 
l'équation caractéristique possède une racine double $r_0$ 

\change
et les solutions de l'équation différentielles sont les
$$y(x) = (\lambda+\mu x)e^{r_0 x}$$
avec $\lambda, \mu$ parcourant $\Rr$.


\change
Enfin si $\Delta<0$, l'équation caractéristique 
possède deux racines complexes et conjuguées que j'écris 
$r_1=\alpha+\ii \beta$ et $r_2=\alpha-\ii \beta$ 

\change
Ici les solutions de l'équation différentielles  sont les
fonctions 
$$y(x) = e^{\alpha x}\big(\lambda\cos (\beta x)+\mu\sin (\beta x)\big)$$
où $\lambda, \mu \in \Rr.$

Reprenons  :
 si $\Delta >0$ les solutions sont les 
 $y(x) = \lambda e^{r_1x}+ \mu e^{r_2x}$ avec $r_1$, $r_2$ les deux racines réelles ;
 
 si $\Delta=0$,les solutions sont les 
 $y(x) = (\lambda+\mu x)e^{r_0 x}$ ; 

 si $\Delta<0$ alors $y(x) = e^{\alpha x}\big(\lambda\cos (\beta x)+\mu\sin (\beta x)\big)$
 où $\alpha$ est la partie réelle d'une racine et $\beta$
 la partie imaginaire.
 
 Dans tous les cas c'est une famille à deux paramètres : $\lambda$ et $\mu$ sont des réels
 qcq.
 
 
%%%%%%%%%%%%%%%%%%%%%%%%%%%%%%%%%%%%%%%%%%%%%%%%%%%%%%%%%%%
\diapo

Illustrons chacune des situations du théorème par un exemple.

Commençons par résoudre l'équation $y'' - y' - 2y = 0$.

\change
L'équation caractéristique est $r^2 - r - 2 = 0$,  

\change
Le discriminant est ici strictement positif et on a deux racines
$r_1 = -1$, $r_2 = 2$.

\change
Alors par le théorème les solutions 
sont l'ensemble de fonctions
 $y(x) = \lambda e^{-x} + \mu e^{2x}$,
 pour $\lambda,\mu$ des réels qcq.
 
 On retrouve ici la racine $-1$ et ici la racine $2$.
 
\change
Pour $y'' - 4y' + 4y = 0$.

\change
son équation caractéristique est $r^2 - 4r + 4 = 0$,

\change
pour lequel $\Delta=0$, et la racine double est $r_0=2$.

\change
Les solutions sont $y(x) = (\lambda x + \mu) e^{2x}$
on retrouve la racine double ici, 
notez bien le facteur $\lambda x + \mu$ devant l'exponentielle.

\change
Enfin pour $y'' - 2y' + 5y = 0$,

\change
et son équation caractéristique $r^2-2r+5 = 0$

\change
$\Delta<0$ et les deux solutions complexes conjuguées sont :
$r_1 = 1 + 2\ii$ et $r_2 = 1 - 2\ii$.

\change 
Les solutions sont les 
$y(x) = e^x (\lambda \cos(2x) + \mu \sin(2x))$.

On retrouve ici la partie réelle et là la partie imaginaire.




%%%%%%%%%%%%%%%%%%%%%%%%%%%%%%%%%%%%%%%%%%%%%%%%%%%%%%%%%%%
\diapo

Avant de prouver une partie du théorème, justifions le lien entre l'équation différentielle
et l'équation caractéristique.

\change
On va chercher une solution de cette équation sous la forme 
$y(x)=e^{rx}$

\change où $r \in \Cc$ est une constante à déterminer. 

\change
Que $y$ soit solution
c'est $ay''+by'+cy=0$ 

\change
ce qui donne pour notre forme cherchée
$(ar^2+br+c)e^{rx}=0$

\change
et comme l'exponentielle ne s'annule jamais.
$ar^2+br+c=0$.

Ainsi $r$ est solution de l'équation caractéristique si et seulement si
$y=e^{rx}$ est solution de l'équation différentielle.

\change
Cela justifie l'utilisation de l'équation caractéristique.
Pour prouver le théorème il nous faut prouver qu'il n'y a pas d'autres solutions possibles.

\change
On va le faire pour le cas $\Delta>0$,

\change
l'équation caractéristique a donc deux racines réelles
distinctes $r_1, r_2$,

\change
On obtient ainsi deux solutions $y_1=e^{r_1x},
y_2=e^{r_2x}$ de l'équation différentielle.

\change
ces deux fonctions sont linéairement indépendantes 
dans l'espace vectoriel des fonctions de $\Rr$ dans $\Rr$,
car $r_1 \neq r_2$.


\change
L'espace des solutions est un espace vectoriel de dimension $2$,
c'est le théorème du début de cette vidéo, théorème que l'on a admis

\change
alors ces deux solutions indépendantes forme une base de
l'espace de solutions,


\change

c-à-d que toute solution est combinaison linéaire des
$e^{r_1x}$, et $e^{r_2x}$, elle s'écrit donc 

$\lambda e^{r_1x} + \mu e^{r_2x},$ où $\lambda,
\mu$ sont des constantes réelles.

%%%%%%%%%%%%%%%%%%%%%%%%%%%%%%%%%%%%%%%%%%%%%%%%%%%%%%%%%%%
\diapo

Nous passons au cas général d'une équation différentielle linéaire d'ordre $2$, à coefficients constants,
mais avec un second membre $g$ qui est une fonction continue sur
un intervalle ouvert $I \subset \Rr$.
\begin{equation}
ay''+by'+cy=g(x) 
%\label{eq:linscd}
\tag{$E$}
\end{equation}

\change
Pour ce type d'équation nous admettons le théorème de Cauchy-Lipschitz 
qui s'énonce ainsi : 

Pour chaque $x_0\in I$ et pour chaque couple $(y_0,y_1) \in \Rr^2$,  
l'équation [[ici]] admet une *unique*
solution $y(x)$ satisfaisant aux deux conditions initiales : 
$y(x_0) = y_0$ \quad et \quad $y'(x_0) = y_1$.

Notez bien que c'est le même $x_0$ dans les deux conditions.

\change

Dans la pratique, pour résoudre une équation différentielle linéaire avec second membre
(avec ou sans conditions initiales), on cherche d'abord les solutions de l'équation homogène,
puis une solution particulière de l'équation avec second membre et on applique
le principe de superposition :

Proposition : 
Les solutions générales de l'équation (\ref{eq:linscd}) s'obtiennent en
ajoutant les solutions générales de l'équation homogène (\ref{eq:linscdhom}) 
à une solution particulière de (\ref{eq:linscd}). 

Il reste donc à déterminer une solution particulière.


%%%%%%%%%%%%%%%%%%%%%%%%%%%%%%%%%%%%%%%%%%%%%%%%%%%%%%%%%%%
\diapo

On donne deux cas particuliers importants et une méthode générale.

\change
Premier cas particulier le second membre est du type 
exponentielle fois polynôme.

\change
Dans ce cas on cherche une solution particulière aussi sous la forme 
exponentielle fois polynôme.

\change
Plus précisemment si le second membre est $e^{\alpha x}P(x)$
alors il va y avoir une solution particulière :

\change
de la forme $e^{\alpha x}Q(x)$,  si $\alpha$ n'est pas 
une racine de l'équation caractéristique,

\change
de la forme $xe^{\alpha x}Q(x)$, ($m=1$), 
si $\alpha$ est une racine simple de l'équation caractéristique,

\change
et enfin de la forme 
$x^2e^{\alpha x}Q(x)$, si $\alpha$ est une 
racine double de l'équation caractéristique.

Dans tous les cas $Q$ est un polynôme de même degré que $P$.


\change
Second cas particulier : le second membre du type 

$e^{\alpha x}\big(P_1(x)\cos (\beta x)+P_2(x)\sin (\beta x)\big)$.

\change
encore une fois on cherche une solution particulière du même type

soit 
$e^{\alpha x} \big( Q_1(x)\cos (\beta x)+Q_2(x)\sin (\beta x) \big)$, 

\change
\change

ou $xe^{\alpha x}  \big( Q_1(x)\cos (\beta x)+Q_2(x)\sin (\beta x) \big)$, 
selon que 
$\alpha +\ii \beta$ n'est pas ou est une racine 
de l'équation caractéristique.


%%%%%%%%%%%%%%%%%%%%%%%%%%%%%%%%%%%%%%%%%%%%%%%%%%%%%%%%%%%
\diapo

Voyons un exemple,

commençons par résoudre l'éuation homgène 
$(E_0) \quad y''-5y'+6y=0$

\change
L'équation caractéristique est $r^2-5r+6$,

\change
$\Delta$ est positif et les deux solutions sont $r_1=2, r_2=3$,

\change
donc l'ensemble de solutions de $(E_0)$ est $\big\{\lambda e^{2x}+ \mu e^{3x} \mid 
\lambda, \mu \in \Rr\big\}$. 

\change

Voic maintenant une équation avec second membre :
$(E) \quad y''-5y'+6y=4xe^x$


Le second membre est du type polynôme fois
exponentielle, donc va chercher une solution particulière aussi sous cette forme là.

\change
Plus précisemment c'est polynôme de degré $1$ fois exponentielle de 
1 fois $x$ et $1$ n'est pas racine de l'équation caractéristique, donc 
on cherche 
 une solution particulière à $(E)$ sous la forme 
 polynôme de degré $1$ fois exponentielle $x$.
 
\change

Lorsque l'on injecte $y_0$ dans l'équation $(E_1)$, on obtient 
une succession d'équivalence,

que je vous laisse vérifier en détail.

\change

\change

\change
ce qui conduit à déterminer les coefficients $a$ et $b$ qui conviennet.

Ici on trouve $y_0=(2x+3)e^x$. 

\change
On ajoute à la solution particulière les solutions de l'équation homogène.
pour obtenir toutes les solutions.

\change
Enfin on nous demande de trouver la solution vérifiant les conditions initiales 
$y(0)=1, y'(0)=0$. Par le théorème de Cauchy-Lipschitz on sait qu'une 
telle solution existe et est unique.
Calculons-là !

\change
On part d'une solution générale de cette forme là 
[[$y(x) = (2x+3)e^x+\lambda e^{2x} + \mu e^{3x}$]]


$y(0)=1$ équivaut à  $3+\lambda+\mu=1$

\change
$y'(0)=0$ équivaut à $5+2\lambda+3\mu=0$

\change
Il n'y a qu'une possibilité $\lambda=-1, \mu=-1$, 

\change
c'est-à-dire que la fonction cherchée est 
$y(x)=(2x+3)e^{x}-e^{2x}-e^{3x}$. 


%%%%%%%%%%%%%%%%%%%%%%%%%%%%%%%%%%%%%%%%%%%%%%%%%%%%%%%%%%%
\diapo

Une méthode générale pour trouver une solution particulière est
la méthode de variation des constantes.

\change

Notons $\{y_1,y_2\}$ est une base de solution de l'équation homogène (\ref{eq:linscdhom}),

\change
La méthode de variation des constantes consiste à chercher 
une solution particulière de l'équation avec second membre sous la forme 
$y_0= \lambda y_1 + \mu y_2$, mais cette fois $\lambda$ et $\mu$ sont deux fonctions 
vérifiant :

($S$) \qquad $
\left\{\begin{array}{ccl}  
\lambda'y_1+\mu'y_2&=&0\\ 
\lambda'y'_1+\mu'y'_2&=& \frac{g(x)}{a}.
\end{array}\right. 
$


\change

Pourquoi cela ?
Si $y_0$ est une telle fonction alors

sa dérivée se simplifie en 
$y_0' = \lambda y_1'+ \mu y_2'$

et sa dérivée seconde devient 
$y_0'' = \frac{g(x)}{a} + \lambda y_1''+ \mu y_2''$.

\change
Et maintenant, avec ces conditions l'équation (\ref{eq:linscd}) est vérifiée par $y_0$,

c'est un simple calculs que je vous laisse lire et qu'il faut refaire à chaque fois.


On utilise le fait que $y_1$ et $y_2$ sont solutions de l'équation homogène.

\change
Le système ($S$) se résout facilement :
c'est un système linéaire à deux équations et deux inconnus qui sont les fonctions $\lambda'$ et $\mu'$.

Cela donne l'expression de $\lambda'$ et $\mu'$, 
puis on trouve les fonction $\lambda$ et $\mu$ par intégration.


%%%%%%%%%%%%%%%%%%%%%%%%%%%%%%%%%%%%%%%%%%%%%%%%%%%%%%%%%%%
\diapo

On termine par la résolution de l'équation 
$$y'' + y = \frac{1}{\cos x}$$
sur l'intervalle $]-\frac\pi2,+\frac\pi2[$.


\change
Les solutions de l'équation homogène $y'' + y =0$ sont
les 
$\lambda \cos x  + \mu \sin x$ où $\lambda,\mu$ sont des constantes qcq.

\change
Par la méthode de variation des constantes, on cherche une solution 
particulière de l'équation avec second membre sous la forme
$$y_0(x) =\lambda(x) \cos x  + \mu(x) \sin x$$

\change
où cette fois $\lambda(x),\mu(x)$ sont des fonctions à déterminer.

\change

On sait qu'elles doivent vérifier le système suivant :
$$
\left\{\begin{array}{ccl}  
\lambda'y_1+\mu'y_2&=&0\\ 
\lambda'y'_1+\mu'y'_2&=& \frac{g(x)}{a}
\end{array}\right. $$

\change
qui s'écrit ici 
$$\left\{\begin{array}{ccl}  
\lambda' \cos x + \mu' \sin x &=&0\\ 
-\lambda' \sin x + \mu' \cos x &=& \frac{1}{\cos x}.
\end{array}\right. 
$$

\change
En multipliant la première ligne par $\sin x$ et la seconde par $\cos x$, on obtient
ceci,

\change
donc par somme des deux lignes $\mu'=1$.

\change
Ainsi, par exemple, $\mu(x) = x$ et la première ligne des équations devient 
$\lambda' = -\frac{\sin x}{\cos x}$ 
donc, par exemple, $\lambda(x) = \ln(\cos x)$.

\change
On vérifie pour se rassurer que $y_0(x) = \ln(\cos x) \cos x + x\sin x$ est une solution
particulière de l'équation 

\change
ainsi les fonctions solutions sont les :
$$\lambda \cos x  + \mu \sin x + \ln(\cos x) \cos x + x\sin x$$
quels que soient $\lambda,\mu\in \Rr$.


%%%%%%%%%%%%%%%%%%%%%%%%%%%%%%%%%%%%%%%%%%%%%%%%%%%%%%%%%%%
\diapo

\change



\end{document}
