
%%%%%%%%%%%%%%%%%% PREAMBULE %%%%%%%%%%%%%%%%%%


\documentclass[12pt]{article}

\usepackage{amsfonts,amsmath,amssymb,amsthm}
\usepackage[utf8]{inputenc}
\usepackage[T1]{fontenc}
\usepackage[francais]{babel}


% packages
\usepackage{amsfonts,amsmath,amssymb,amsthm}
\usepackage[utf8]{inputenc}
\usepackage[T1]{fontenc}
%\usepackage{lmodern}

\usepackage[francais]{babel}
\usepackage{fancybox}
\usepackage{graphicx}

\usepackage{float}

%\usepackage[usenames, x11names]{xcolor}
\usepackage{tikz}
\usepackage{datetime}

\usepackage{mathptmx}
%\usepackage{fouriernc}
%\usepackage{newcent}
\usepackage[mathcal,mathbf]{euler}

%\usepackage{palatino}
%\usepackage{newcent}


% Commande spéciale prompteur

%\usepackage{mathptmx}
%\usepackage[mathcal,mathbf]{euler}
%\usepackage{mathpple,multido}

\usepackage[a4paper]{geometry}
\geometry{top=2cm, bottom=2cm, left=1cm, right=1cm, marginparsep=1cm}

\newcommand{\change}{{\color{red}\rule{\textwidth}{1mm}\\}}

\newcounter{mydiapo}

\newcommand{\diapo}{\newpage
\hfill {\normalsize  Diapo \themydiapo \quad \texttt{[\jobname]}} \\
\stepcounter{mydiapo}}


%%%%%%% COULEURS %%%%%%%%%%

% Pour blanc sur noir :
%\pagecolor[rgb]{0.5,0.5,0.5}
% \pagecolor[rgb]{0,0,0}
% \color[rgb]{1,1,1}



%\DeclareFixedFont{\myfont}{U}{cmss}{bx}{n}{18pt}
\newcommand{\debuttexte}{
%%%%%%%%%%%%% FONTES %%%%%%%%%%%%%
\renewcommand{\baselinestretch}{1.5}
\usefont{U}{cmss}{bx}{n}
\bfseries

% Taille normale : commenter le reste !
%Taille Arnaud
%\fontsize{19}{19}\selectfont

% Taille Barbara
%\fontsize{21}{22}\selectfont

%Taille François
\fontsize{25}{30}\selectfont

%Taille Pascal
%\fontsize{25}{30}\selectfont

%Taille Laura
%\fontsize{30}{35}\selectfont


%\myfont
%\usefont{U}{cmss}{bx}{n}

%\Huge
%\addtolength{\parskip}{\baselineskip}
}


% \usepackage{hyperref}
% \hypersetup{colorlinks=true, linkcolor=blue, urlcolor=blue,
% pdftitle={Exo7 - Exercices de mathématiques}, pdfauthor={Exo7}}


%section
% \usepackage{sectsty}
% \allsectionsfont{\bf}
%\sectionfont{\color{Tomato3}\upshape\selectfont}
%\subsectionfont{\color{Tomato4}\upshape\selectfont}

%----- Ensembles : entiers, reels, complexes -----
\newcommand{\Nn}{\mathbb{N}} \newcommand{\N}{\mathbb{N}}
\newcommand{\Zz}{\mathbb{Z}} \newcommand{\Z}{\mathbb{Z}}
\newcommand{\Qq}{\mathbb{Q}} \newcommand{\Q}{\mathbb{Q}}
\newcommand{\Rr}{\mathbb{R}} \newcommand{\R}{\mathbb{R}}
\newcommand{\Cc}{\mathbb{C}} 
\newcommand{\Kk}{\mathbb{K}} \newcommand{\K}{\mathbb{K}}

%----- Modifications de symboles -----
\renewcommand{\epsilon}{\varepsilon}
\renewcommand{\Re}{\mathop{\text{Re}}\nolimits}
\renewcommand{\Im}{\mathop{\text{Im}}\nolimits}
%\newcommand{\llbracket}{\left[\kern-0.15em\left[}
%\newcommand{\rrbracket}{\right]\kern-0.15em\right]}

\renewcommand{\ge}{\geqslant}
\renewcommand{\geq}{\geqslant}
\renewcommand{\le}{\leqslant}
\renewcommand{\leq}{\leqslant}

%----- Fonctions usuelles -----
\newcommand{\ch}{\mathop{\mathrm{ch}}\nolimits}
\newcommand{\sh}{\mathop{\mathrm{sh}}\nolimits}
\renewcommand{\tanh}{\mathop{\mathrm{th}}\nolimits}
\newcommand{\cotan}{\mathop{\mathrm{cotan}}\nolimits}
\newcommand{\Arcsin}{\mathop{\mathrm{Arcsin}}\nolimits}
\newcommand{\Arccos}{\mathop{\mathrm{Arccos}}\nolimits}
\newcommand{\Arctan}{\mathop{\mathrm{Arctan}}\nolimits}
\newcommand{\Argsh}{\mathop{\mathrm{Argsh}}\nolimits}
\newcommand{\Argch}{\mathop{\mathrm{Argch}}\nolimits}
\newcommand{\Argth}{\mathop{\mathrm{Argth}}\nolimits}
\newcommand{\pgcd}{\mathop{\mathrm{pgcd}}\nolimits} 

\newcommand{\Card}{\mathop{\text{Card}}\nolimits}
\newcommand{\Ker}{\mathop{\text{Ker}}\nolimits}
\newcommand{\id}{\mathop{\text{id}}\nolimits}
\newcommand{\ii}{\mathrm{i}}
\newcommand{\dd}{\mathrm{d}}
\newcommand{\Vect}{\mathop{\text{Vect}}\nolimits}
\newcommand{\Mat}{\mathop{\mathrm{Mat}}\nolimits}
\newcommand{\rg}{\mathop{\text{rg}}\nolimits}
\newcommand{\tr}{\mathop{\text{tr}}\nolimits}
\newcommand{\ppcm}{\mathop{\text{ppcm}}\nolimits}

%----- Structure des exercices ------

\newtheoremstyle{styleexo}% name
{2ex}% Space above
{3ex}% Space below
{}% Body font
{}% Indent amount 1
{\bfseries} % Theorem head font
{}% Punctuation after theorem head
{\newline}% Space after theorem head 2
{}% Theorem head spec (can be left empty, meaning ‘normal’)

%\theoremstyle{styleexo}
\newtheorem{exo}{Exercice}
\newtheorem{ind}{Indications}
\newtheorem{cor}{Correction}


\newcommand{\exercice}[1]{} \newcommand{\finexercice}{}
%\newcommand{\exercice}[1]{{\tiny\texttt{#1}}\vspace{-2ex}} % pour afficher le numero absolu, l'auteur...
\newcommand{\enonce}{\begin{exo}} \newcommand{\finenonce}{\end{exo}}
\newcommand{\indication}{\begin{ind}} \newcommand{\finindication}{\end{ind}}
\newcommand{\correction}{\begin{cor}} \newcommand{\fincorrection}{\end{cor}}

\newcommand{\noindication}{\stepcounter{ind}}
\newcommand{\nocorrection}{\stepcounter{cor}}

\newcommand{\fiche}[1]{} \newcommand{\finfiche}{}
\newcommand{\titre}[1]{\centerline{\large \bf #1}}
\newcommand{\addcommand}[1]{}
\newcommand{\video}[1]{}

% Marge
\newcommand{\mymargin}[1]{\marginpar{{\small #1}}}



%----- Presentation ------
\setlength{\parindent}{0cm}

%\newcommand{\ExoSept}{\href{http://exo7.emath.fr}{\textbf{\textsf{Exo7}}}}

\definecolor{myred}{rgb}{0.93,0.26,0}
\definecolor{myorange}{rgb}{0.97,0.58,0}
\definecolor{myyellow}{rgb}{1,0.86,0}

\newcommand{\LogoExoSept}[1]{  % input : echelle
{\usefont{U}{cmss}{bx}{n}
\begin{tikzpicture}[scale=0.1*#1,transform shape]
  \fill[color=myorange] (0,0)--(4,0)--(4,-4)--(0,-4)--cycle;
  \fill[color=myred] (0,0)--(0,3)--(-3,3)--(-3,0)--cycle;
  \fill[color=myyellow] (4,0)--(7,4)--(3,7)--(0,3)--cycle;
  \node[scale=5] at (3.5,3.5) {Exo7};
\end{tikzpicture}}
}



\theoremstyle{definition}
%\newtheorem{proposition}{Proposition}
%\newtheorem{exemple}{Exemple}
%\newtheorem{theoreme}{Théorème}
\newtheorem{lemme}{Lemme}
\newtheorem{corollaire}{Corollaire}
%\newtheorem*{remarque*}{Remarque}
%\newtheorem*{miniexercice}{Mini-exercices}
%\newtheorem{definition}{Définition}




%definition d'un terme
\newcommand{\defi}[1]{{\color{myorange}\textbf{\emph{#1}}}}
\newcommand{\evidence}[1]{{\color{blue}\textbf{\emph{#1}}}}



 %----- Commandes divers ------

\newcommand{\codeinline}[1]{\texttt{#1}}

%%%%%%%%%%%%%%%%%%%%%%%%%%%%%%%%%%%%%%%%%%%%%%%%%%%%%%%%%%%%%
%%%%%%%%%%%%%%%%%%%%%%%%%%%%%%%%%%%%%%%%%%%%%%%%%%%%%%%%%%%%%

\begin{document}

\debuttexte

%%%%%%%%%%%%%%%%%%%%%%%%%%%%%%%%%%%%%%%%%%%%%%%%%%%%%%%%%%%
\diapo

\change

Voici le coeur de ce chapitre sur les ensembles et les applications.

\change

Nous allons définir ce que sont des applications injectives, des applications surjectives

\change

et des applications bijectives.


%%%%%%%%%%%%%%%%%%%%%%%%%%%%%%%%%%%%%%%%%%%%%%%%%%%%%%%%%%%
\diapo

Soit $f$ une application de $E$ dans $F$.

Nous dirons que $f$ est \defi{injective} si pour tout $x,x' \in E$ vérifiant $f(x)=f(x')$ alors $x=x'$.


\change

En terme de phrase mathématique : 

$\forall x, x' \in E \quad \big( f(x)=f(x') \implies x=x'\big)$


\change

Voici la représentation d'une application injective,

\change

Même chose sur cette fonction qui est injective.

\change

si $f(x) = f(x')$ alors c'est que dès le départ $x=x'$,

en d'autre terme on a bien que si $x\neq x'$ alors $f(x) \neq f(x')$.




%%%%%%%%%%%%%%%%%%%%%%%%%%%%%%%%%%%%%%%%%%%%%%%%%%%%%%%%%%%
\diapo


Soit toujours $f : E \to F$ 

Nous dirons que $f$ est surjective si pour tout $y$ de l'ensemble d'arrivée il existe $x$ de l'ensemble de départ 
tel que $y=f(x)$

\change

Autrement dit :
$\forall y \in F \quad \exists x \in E \quad \big( y = f(x) \big)$

\change

En terme d'image directe $f$ est surjective équivaut à $f(E)=F$

\change

Voici deux dessins de fonctions surjectives.

\change

Pour chaque $y$ de l'ensemble d'arrivée $F$ on peut trouver au moins un
$x$ de l'ensemble de départ qui s'envoie sur $y$.



%%%%%%%%%%%%%%%%%%%%%%%%%%%%%%%%%%%%%%%%%%%%%%%%%%%%%%%%%%%
\diapo


Regardons un autre point vue,

je vous rappelle qu'on appelle antécédent de $y$ toute valeur $x$ tel que $f(x)=y$.

Ici les antécédents de $y$ sont $x_1, x_2$ et $x_3$.

\change

Avec le langage des antécédents : 

$f$ est \emph{injective} si et seulement si tout élément $y$ de $F$ a \emph{au plus} $1$ antécédent

Eventuellement $y$ ne peut avoir aucun antécédent.

\change

$f$ est \emph{surjective} si et seulement si tout élément $y$ de $F$ a \emph{au moins} $1$ antécédent


%%%%%%%%%%%%%%%%%%%%%%%%%%%%%%%%%%%%%%%%%%%%%%%%%%%%%%%%%%%
\diapo


Voici des exemples de fonctions non injectives.

Cette première fonction n'est pas injective car on peut trouver deux éléments 
$x$ et $x'$ différents qui s'envoie sur la même valeur $y$.

On a donc ici $f(x)=f(x')$ mais pas $x=x'$. $f$ n'est pas injective.

\change

Même chose celle-là, on peut trouver deux éléments $x$ et $x'$ distincts mais qui  
prennent la même valeur par $f$ ; cette fonction n'est pas injective.

%%%%%%%%%%%%%%%%%%%%%%%%%%%%%%%%%%%%%%%%%%%%%%%%%%%%%%%%%%%
\diapo

Cette fonction n'est pas surjective,

en effet on peut trouver un $y$ de l'ensemble d'arrivée 
qui n'a pas d'antécédent.

\change

Idem pour celle-ci : elle n'est pas surjective, car par exemple la valeur $y$
marquée n'est pas atteinte.



%%%%%%%%%%%%%%%%%%%%%%%%%%%%%%%%%%%%%%%%%%%%%%%%%%%%%%%%%%%
\diapo

[2 prises]

Etudions la fonction $f_1$ qui va des entiers naturels vers les rationnels et qui a $x$ associe $1/1+x$.

\change

$f_1$ est une fonction injective 

\change

pour le montrer on part de deux éléments $x$ et $x'$ qui vérifient $f_1(x)=f_1(x')$.

Nous devons montrer qu'en fait $x=x'$.

\change

Comme $f_1(x)=f_1(x')$ alors $\frac{1}{1+x}=\frac{1}{1+x'}$

\change

cela entraîne $1+x=1+x'$ 

\change

et donc $x=x'$ 

C'est vrai pour tout $x$ et $x'$ qui vérifient  $f_1(x)=f_1(x')$ et donc $f_1$ est une fonction injective.

\change

Par contre  $f_1$ n'est pas surjective. Nous devons trouver
un élément $y$ qui n'a pas d'antécédent.

\change

Comme on a toujours  $f_1(x) \le 1$ alors par exemple $y=2$ n'a pas d'antécédent 

\change

Faisons la même étude pour $f_2$ qui va de $\Zz$ dans $\Nn$ et qui a $x$ associe $x^2$.

\change

Cette fois-ci $f_2$ n'est pas injective. 

\change

Pour le prouver on doit trouver 
$x$ et $x'$ distincts mais qui prennent la même valeur.


\change

Il suffit de prendre $x$ et $x'$ opposés : par exemple $x=2$, $x'=-2$ 
qui sont distinct mais qui prennent par $f_2$ la même valeur $+4$.


\change

$f_2$ n'est pas non plus surjective, certaines valeurs $y$ de l'ensemble d'arrivée 
n'ont pas d'antécédents. 

\change

Par exemple $y=3$ n'a pas d'antécédent dans $\Zz$.

Si $y=3$ avait un antécédent $x$ il vérifierait $x^2=3$ donc 
$x$ serait égal à $+\sqrt3$ ou $-\sqrt3$. Mais alors $x$ n'est plus un entier.

Donc $y$ n'a pas d'antécédent dans $\Zz$. Ainsi $f_2$ n'est pas surjective.


%%%%%%%%%%%%%%%%%%%%%%%%%%%%%%%%%%%%%%%%%%%%%%%%%%%%%%%%%%%
\diapo

Nous avons tous les ingrédients pour définir une bijection.

Une application $f$ est bijective si elle est à la fois injective et surjective.

\change

Si l'on combine les définitions d'injectivité et surjectivité alors

$f$ est bijective si et seulement si 
pour tout $y \in F$ il existe un unique $x \in E$ tel que $y=f(x)$


\change

La phrase mathématique est donc la suivante :

$\forall y \in F \quad \exists! x \in E \quad \big( y = f(x) \big)$

\change

ou encore avec la notion d'antécédent

$f$ est bijective est équivalent à ce que 
 tout élément de l'ensemble d'arrivée ait un unique antécédent par l'application $f$

\change

Voici un premier dessin d'une bijection, tout point de l'ensemble d'arrivée a 
bien un unique antécédent.

\change

Même chose ici : toute valeur $y$ de l'ensemble grand $F$ est atteinte
par $f(x)$ en un seul point $x$ de l'ensemble grand $E$. 



%%%%%%%%%%%%%%%%%%%%%%%%%%%%%%%%%%%%%%%%%%%%%%%%%%%%%%%%%%%
\diapo

Nous allons donner une autre caractérisation des applications bijectives.

\change

L'application $f : E \to F$ est bijective si et seulement si il existe $g : F \to E$
telle que $f \circ g = \id_F$ et $g \circ f = \id_E$.


Je vous rappelle que l'identité est simplement l'application qui a $x$ associe $x$.


\change

Si $f$ est bijective alors cette application $g$ est unique 

on appelle $g$ la bijection réciproque (ou l'inverse) de $f$ et on la notera $f^{-1}$

\change

$g$ est elle aussi une application bijective et sa bijection réciproque est $f$

Autrement dit $\left( f^{-1} \right)^{-1} = f$

\change

L'égalité $f \circ g = \id_F$ est une égalité de fonctions et signifie exactement
 $\forall y \in F  \quad f\big(g(y)\big) = y$

\change

De même 
$g \circ f = \id_E$ signifie  $\forall x \in E \quad g\big(f(x)\big) = x$


%%%%%%%%%%%%%%%%%%%%%%%%%%%%%%%%%%%%%%%%%%%%%%%%%%%%%%%%%%%
\diapo

Si $f$ est une bijection alors 

\change

la bijection réciproque $f^{-1}$ correspond à inverser le sens des flèches !


\change

Voyons maintenant l'exemple de la fonction $f$ qui va de l'ensemble de réels $\Rr$
dans l'intervalle $]0,+\infty[$, et qui est définie par la formule $f(x)=\exp(x)$.

Cette fonction $f$ est bijective...

\change

et vous savez que sa bijection réciproque est
le logarithme, plus précisément c'est la fonction $g$
qui va de l'intervalle $]0,+\infty[$ dans $\Rr$ et qui a $y$ associe $\ln(y)$.

\change

On a bien les deux formules 

$f \circ g= \id$ et $g\circ f = \id$,

c'est-à-dire :

$\exp\big(\ln(y) \big) = y$ pour tout $y$ de $]0,+\infty[$


et $\ln\big(\exp(x)\big) = x$ pour tout $x$ de $\Rr$.

\change

si l'on trace le graphe de $f$ dans un repère orthonormé

\change

alors on obtient le graphe de sa bijection réciproque $f^{-1}$
par symétrie par rapport à la diagonale d'équation $(y=x)$.




%%%%%%%%%%%%%%%%%%%%%%%%%%%%%%%%%%%%%%%%%%%%%%%%%%%%%%%%%%%
\diapo


Si l'on a deux applications bijectives $f : E \to F$ et $g : F \to G$ 


Alors d'une part $g \circ f$ est aussi une application bijective

\change

d'autre part on sait calculer la bijection réciproque de $g\circ f$ c'est

$(g\circ f)^{-1} = f^{-1} \circ g^{-1}$

Retenez bien que prendre l'inverse d'une composition renverse l'ordre !


%%%%%%%%%%%%%%%%%%%%%%%%%%%%%%%%%%%%%%%%%%%%%%%%%%%%%%%%%%%
\diapo

Terminons par des démonstrations, sur lesquelles vous pourrez revenir lors d'une seconde lecture.


Nous allons prouver qu'être une fonction bijective est équivalent à l'existence 
d'un inverse $g$

\change

Commençons par le sens direct.

\change

Nous supposons $f$ bijective et construisons l'inverse $g : F \to E$ 

\change

Comme $f$ est bijective elle est surjective : pour chaque $y \in F$, il existe un $x \in E$ tel que $y=f(x)$

\change

On définit alors $g(y)$ comme étant égal à ce $x$ et on a bien $f\big( g(y) \big) = f(x) =y$

\change

Ceci en vrai pour tous les $y\in F$ donc $f \circ g = \id_F$

\change

En composant à droite par $f$ on obtient 
 $f \circ g \circ f = \id_F \circ f$, donc $=f$

\change

Ce qui en terme d'élément donne 
$f\big( g\circ f(x) \big) = f(x)$ 

\change

Mais $f$ est bijective 

donc est aussi injective 

comme $f\big( g\circ f(x) \big) = f(x)$ alors 
par l'injectivité de $f$ 
$g\circ f(x)=x$

\change

Ceci est vrai pour tous les  $x$   donc 
on a l'égalité des applications $g\circ f = \id$


%%%%%%%%%%%%%%%%%%%%%%%%%%%%%%%%%%%%%%%%%%%%%%%%%%%%%%%%%%%
\diapo

Passons à la démonstration de l'implication réciproque.

\change

Nous supposons qu'un tel $g$ existe et nous devons montrer que $f$ est bijective.

\change

Montrons d'abord que $f$ est surjective, soit $y \in F$, 

nous devons lui trouver un antécédent par $f$.

S'il l'on note $x = g(y)$

\change

Alors on a 
$f(x) = f\big( g(y) \big) = f \circ g(y) = \id_F(y)=y$

car par hypothèse $f\circ g = \id$.

\change

$f(x)=y$ donc $x$ est bien un antécédent de $y$, ceci étant vrai pour tout $y$, 
 $f$ est surjective.

\change

Montrons maintenant que $f$ est injective

partons de deux éléments $x,x'$ tels que $f(x)=f(x')$

\change


Alors $g\circ f(x)=g\circ f(x')$, donc en utilisant l'hypothèse
$g\circ f= \id$ on obtient 
 $\id(x)=\id(x')$


\change

Ce qui conduit à $x=x'$, 

et ainsi $f$ est bien injective.

$f$ est surjective et injective elle est ainsi bijective.





%%%%%%%%%%%%%%%%%%%%%%%%%%%%%%%%%%%%%%%%%%%%%%%%%%%%%%%%%%%
\diapo

Passez beaucoup de temps sur ce chapitre, les efforts en valent vraiment la peine.


Pour les mini-exercices, n'oubliez pas qu'être injective et surjective 
dépend bien sûr de la formule définissant $f$ mais aussi de l'ensemble
de départ et de l'ensemble d'arrivée.


\end{document}