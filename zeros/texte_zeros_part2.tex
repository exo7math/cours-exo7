
%%%%%%%%%%%%%%%%%% PREAMBULE %%%%%%%%%%%%%%%%%%


\documentclass[12pt]{article}

\usepackage{amsfonts,amsmath,amssymb,amsthm}
\usepackage[utf8]{inputenc}
\usepackage[T1]{fontenc}
\usepackage[francais]{babel}


% packages
\usepackage{amsfonts,amsmath,amssymb,amsthm}
\usepackage[utf8]{inputenc}
\usepackage[T1]{fontenc}
%\usepackage{lmodern}

\usepackage[francais]{babel}
\usepackage{fancybox}
\usepackage{graphicx}

\usepackage{float}

%\usepackage[usenames, x11names]{xcolor}
\usepackage{tikz}
\usepackage{datetime}

\usepackage{mathptmx}
%\usepackage{fouriernc}
%\usepackage{newcent}
\usepackage[mathcal,mathbf]{euler}

%\usepackage{palatino}
%\usepackage{newcent}


% Commande spéciale prompteur

%\usepackage{mathptmx}
%\usepackage[mathcal,mathbf]{euler}
%\usepackage{mathpple,multido}

\usepackage[a4paper]{geometry}
\geometry{top=2cm, bottom=2cm, left=1cm, right=1cm, marginparsep=1cm}

\newcommand{\change}{{\color{red}\rule{\textwidth}{1mm}\\}}

\newcounter{mydiapo}

\newcommand{\diapo}{\newpage
\hfill {\normalsize  Diapo \themydiapo \quad \texttt{[\jobname]}} \\
\stepcounter{mydiapo}}


%%%%%%% COULEURS %%%%%%%%%%

% Pour blanc sur noir :
%\pagecolor[rgb]{0.5,0.5,0.5}
% \pagecolor[rgb]{0,0,0}
% \color[rgb]{1,1,1}



%\DeclareFixedFont{\myfont}{U}{cmss}{bx}{n}{18pt}
\newcommand{\debuttexte}{
%%%%%%%%%%%%% FONTES %%%%%%%%%%%%%
\renewcommand{\baselinestretch}{1.5}
\usefont{U}{cmss}{bx}{n}
\bfseries

% Taille normale : commenter le reste !
%Taille Arnaud
%\fontsize{19}{19}\selectfont

% Taille Barbara
%\fontsize{21}{22}\selectfont

%Taille François
\fontsize{25}{30}\selectfont

%Taille Pascal
%\fontsize{25}{30}\selectfont

%Taille Laura
%\fontsize{30}{35}\selectfont


%\myfont
%\usefont{U}{cmss}{bx}{n}

%\Huge
%\addtolength{\parskip}{\baselineskip}
}


% \usepackage{hyperref}
% \hypersetup{colorlinks=true, linkcolor=blue, urlcolor=blue,
% pdftitle={Exo7 - Exercices de mathématiques}, pdfauthor={Exo7}}


%section
% \usepackage{sectsty}
% \allsectionsfont{\bf}
%\sectionfont{\color{Tomato3}\upshape\selectfont}
%\subsectionfont{\color{Tomato4}\upshape\selectfont}

%----- Ensembles : entiers, reels, complexes -----
\newcommand{\Nn}{\mathbb{N}} \newcommand{\N}{\mathbb{N}}
\newcommand{\Zz}{\mathbb{Z}} \newcommand{\Z}{\mathbb{Z}}
\newcommand{\Qq}{\mathbb{Q}} \newcommand{\Q}{\mathbb{Q}}
\newcommand{\Rr}{\mathbb{R}} \newcommand{\R}{\mathbb{R}}
\newcommand{\Cc}{\mathbb{C}} 
\newcommand{\Kk}{\mathbb{K}} \newcommand{\K}{\mathbb{K}}

%----- Modifications de symboles -----
\renewcommand{\epsilon}{\varepsilon}
\renewcommand{\Re}{\mathop{\text{Re}}\nolimits}
\renewcommand{\Im}{\mathop{\text{Im}}\nolimits}
%\newcommand{\llbracket}{\left[\kern-0.15em\left[}
%\newcommand{\rrbracket}{\right]\kern-0.15em\right]}

\renewcommand{\ge}{\geqslant}
\renewcommand{\geq}{\geqslant}
\renewcommand{\le}{\leqslant}
\renewcommand{\leq}{\leqslant}

%----- Fonctions usuelles -----
\newcommand{\ch}{\mathop{\mathrm{ch}}\nolimits}
\newcommand{\sh}{\mathop{\mathrm{sh}}\nolimits}
\renewcommand{\tanh}{\mathop{\mathrm{th}}\nolimits}
\newcommand{\cotan}{\mathop{\mathrm{cotan}}\nolimits}
\newcommand{\Arcsin}{\mathop{\mathrm{Arcsin}}\nolimits}
\newcommand{\Arccos}{\mathop{\mathrm{Arccos}}\nolimits}
\newcommand{\Arctan}{\mathop{\mathrm{Arctan}}\nolimits}
\newcommand{\Argsh}{\mathop{\mathrm{Argsh}}\nolimits}
\newcommand{\Argch}{\mathop{\mathrm{Argch}}\nolimits}
\newcommand{\Argth}{\mathop{\mathrm{Argth}}\nolimits}
\newcommand{\pgcd}{\mathop{\mathrm{pgcd}}\nolimits} 

\newcommand{\Card}{\mathop{\text{Card}}\nolimits}
\newcommand{\Ker}{\mathop{\text{Ker}}\nolimits}
\newcommand{\id}{\mathop{\text{id}}\nolimits}
\newcommand{\ii}{\mathrm{i}}
\newcommand{\dd}{\mathrm{d}}
\newcommand{\Vect}{\mathop{\text{Vect}}\nolimits}
\newcommand{\Mat}{\mathop{\mathrm{Mat}}\nolimits}
\newcommand{\rg}{\mathop{\text{rg}}\nolimits}
\newcommand{\tr}{\mathop{\text{tr}}\nolimits}
\newcommand{\ppcm}{\mathop{\text{ppcm}}\nolimits}

%----- Structure des exercices ------

\newtheoremstyle{styleexo}% name
{2ex}% Space above
{3ex}% Space below
{}% Body font
{}% Indent amount 1
{\bfseries} % Theorem head font
{}% Punctuation after theorem head
{\newline}% Space after theorem head 2
{}% Theorem head spec (can be left empty, meaning ‘normal’)

%\theoremstyle{styleexo}
\newtheorem{exo}{Exercice}
\newtheorem{ind}{Indications}
\newtheorem{cor}{Correction}


\newcommand{\exercice}[1]{} \newcommand{\finexercice}{}
%\newcommand{\exercice}[1]{{\tiny\texttt{#1}}\vspace{-2ex}} % pour afficher le numero absolu, l'auteur...
\newcommand{\enonce}{\begin{exo}} \newcommand{\finenonce}{\end{exo}}
\newcommand{\indication}{\begin{ind}} \newcommand{\finindication}{\end{ind}}
\newcommand{\correction}{\begin{cor}} \newcommand{\fincorrection}{\end{cor}}

\newcommand{\noindication}{\stepcounter{ind}}
\newcommand{\nocorrection}{\stepcounter{cor}}

\newcommand{\fiche}[1]{} \newcommand{\finfiche}{}
\newcommand{\titre}[1]{\centerline{\large \bf #1}}
\newcommand{\addcommand}[1]{}
\newcommand{\video}[1]{}

% Marge
\newcommand{\mymargin}[1]{\marginpar{{\small #1}}}



%----- Presentation ------
\setlength{\parindent}{0cm}

%\newcommand{\ExoSept}{\href{http://exo7.emath.fr}{\textbf{\textsf{Exo7}}}}

\definecolor{myred}{rgb}{0.93,0.26,0}
\definecolor{myorange}{rgb}{0.97,0.58,0}
\definecolor{myyellow}{rgb}{1,0.86,0}

\newcommand{\LogoExoSept}[1]{  % input : echelle
{\usefont{U}{cmss}{bx}{n}
\begin{tikzpicture}[scale=0.1*#1,transform shape]
  \fill[color=myorange] (0,0)--(4,0)--(4,-4)--(0,-4)--cycle;
  \fill[color=myred] (0,0)--(0,3)--(-3,3)--(-3,0)--cycle;
  \fill[color=myyellow] (4,0)--(7,4)--(3,7)--(0,3)--cycle;
  \node[scale=5] at (3.5,3.5) {Exo7};
\end{tikzpicture}}
}



\theoremstyle{definition}
%\newtheorem{proposition}{Proposition}
%\newtheorem{exemple}{Exemple}
%\newtheorem{theoreme}{Théorème}
\newtheorem{lemme}{Lemme}
\newtheorem{corollaire}{Corollaire}
%\newtheorem*{remarque*}{Remarque}
%\newtheorem*{miniexercice}{Mini-exercices}
%\newtheorem{definition}{Définition}




%definition d'un terme
\newcommand{\defi}[1]{{\color{myorange}\textbf{\emph{#1}}}}
\newcommand{\evidence}[1]{{\color{blue}\textbf{\emph{#1}}}}



 %----- Commandes divers ------

\newcommand{\codeinline}[1]{\texttt{#1}}

%%%%%%%%%%%%%%%%%%%%%%%%%%%%%%%%%%%%%%%%%%%%%%%%%%%%%%%%%%%%%
%%%%%%%%%%%%%%%%%%%%%%%%%%%%%%%%%%%%%%%%%%%%%%%%%%%%%%%%%%%%%



\begin{document}

\debuttexte


%%%%%%%%%%%%%%%%%%%%%%%%%%%%%%%%%%%%%%%%%%%%%%%%%%%%%%%%%%%
\diapo

\change

Nous allons voir une nouvelle méthode pour obtenir une approximation d'une solution
d'une équation $(f(x)=0)$ :

\change

cette nouvelle méthode est la méthode de la sécante.

\change

Nous l'appliquerons pour trouver une approximation 
de $\sqrt{10}$

\change

et de $(1,10)^{1/12}$

\change

Nous terminons par une majoration de l'erreur 

\change

et un algorithme

%%%%%%%%%%%%%%%%%%%%%%%%%%%%%%%%%%%%%%%%%%%%%%%%%%%%%%%%%%
\diapo

[grand $A$, grand $B$]

L'idée de la méthode de la sécante est très simple : pour une fonction $f$ 
continue sur un intervalle $[a,b]$, et vérifiant $f(a) \le 0$, $f(b) > 0$,

\change

on trace le segment $[AB]$ où $A=(a,f(a))$ 
et $B=(b,f(b))$. 


\change

Si le segment reste au-dessus du graphe de $f$ alors la fonction s'annule sur l'intervalle $[a',b]$ où 
$(a',0)$ est le point d'intersection de 
la droite $(AB)$ avec l'axe des abscisses. 


\change


On recommence en partant maintenant de l'intervalle $[a',b]$,


on trace la sécante entre le point $A'$ et $B$, 

\change

cette sécante recoupe l'axe des abscisses en un point d'abscisse $a''$.

\change

On recommence en partant de $a''$.


%%%%%%%%%%%%%%%%%%%%%%%%%%%%%%%%%%%%%%%%%%%%%%%%%%%%%%%%%%%
\diapo


Voici l'énoncé théorique de cette leçon :


 Soit $f : [a,b] \to \Rr$ une fonction continue, strictement croissante et convexe (nous y reviendrons) telle
 que $f(a) \le 0$, $f(b) > 0$.
 
 On définit une suite par récurrence :
 
 Le terme initial est $a_0=a$
 
 et la formule de récurrence est $a_{n+1} = a_n - \frac{b-a_n}{f(b)-f(a_n)}f(a_n)$.
 
 Alors la suite $(a_n)$  est croissante et converge vers la solution $\ell$ de $(f(x)=0)$.
 
 

%%%%%%%%%%%%%%%%%%%%%%%%%%%%%%%%%%%%%%%%%%%%%%%%%%%%%%%%%%
\diapo


Expliquons géométriquement ce qu'est l'hypothèse $f$ \defi{convexe} :

on fixe deux abscisses $x,x'$, on obtient
deux points du graphe $(x,f(x))$ et $(x',f(x'))$.


$f$ est convexe si le segment reliant ces deux points est entièrement au-dessus
du graphe de la fonction.

Il faut que ceci soit vrai quelques soient les abscisses $x$ et $x'$.


%%%%%%%%%%%%%%%%%%%%%%%%%%%%%%%%%%%%%%%%%%%%%%%%%%%%%%%%%%%
\diapo


Justifions maintenant la formule définissant la suite récurrente. 

\change

Nous devons calculer $a_{n+1}$ en fonction de $a_n$,

ce qui revient à calculer $a'$ en fonction de $a$, puis à itérer la formule.


\change

\change

      L'équation de la droite $(AB)$ passant par les deux points $(a,f(a))$ et $(b,f(b))$ 
est 
$$y=(x-a) \frac{f(b)-f(a)}{b-a} + f(a)$$


On cherche l'intersection de cette sécante $(AB)$ avec l'axe des abscisses.

\change

Le point d'intersection a pour coordonnées $(a',0)$ 

et comme il est sur la sécante il vérifie :

$0=(a'-a) \frac{f(b)-f(a)}{b-a} + f(a)$, 

on obtient $a'=a - \frac{b-a}{f(b)-f(a)}f(a)$.

Ce qui donne la formule de récurrence voulue.
  




%%%%%%%%%%%%%%%%%%%%%%%%%%%%%%%%%%%%%%%%%%%%%%%%%%%%%%%%%%%
\diapo

Montrons maintenant que la suite $(a_n)$ définie par récurrence est une suite croissante.

\change

Tout d'abord comme $f$ est une fonction convexe : la sécante 
est au-dessus du graphe et on montre que ceci implique que $f(a_n) \le 0$

\change

On se rappelle que $a_{n+1}$ s'obtient en fonction de $a_n$ par cette formule de récurrence.

  Comme $f$ est croissante, $\frac{b-a_n}{f(b)-f(a_n)}$ est positif, et
  l'on vient de voir que $f(a_n) \le 0$.
  
  Ainsi toute cette expression, y compris avec le signe moins, est positive.

\change
  
  C'est exactement dire que $a_{n+1} \ge a_n$.
  
\change

Montrons que la suite $(a_n)$ a une limite et calculons-la.

\change

  La suite $(a_n)$, on vient de le voir, est une suite croissante ; elle est en majorée par $b$, 
  l’extrémité de l'intervalle, 
  
  Ainsi $(a_n)$ est une suite croissante et majorée donc elle converge. 
  
\change

  Notons $\ell$ sa limite. 
  

\change

Ainsi $a_n \to \ell$,

mais aussi bien sûr $a_{n+1} \to \ell$

et par continuité de la fonction $f$ : $f(a_n)\to f(\ell)$

\change

  L'égalité $a_{n+1} = a_n - \frac{b-a_n}{f(b)-f(a_n)}f(a_n)$ devient à la limite (lorsque $n\to+\infty$) :
  $\ell = \ell - \frac{b-\ell}{f(b)-f(\ell)} f(\ell)$, 
  
  
\change

Les $\ell$ se simplifie, le quotient est non nul car $\ell < b$

et donc cela implique $f(\ell)=0$.
  
  
  Conclusion : $(a_n)$ converge vers le zéro de $f$.
  
%%%%%%%%%%%%%%%%%%%%%%%%%%%%%%%%%%%%%%%%%%%%%%%%%%%%%%%%%%
\diapo

Appliquons la méthode pour obtenir une approximation de $\sqrt{10}$.

\change

On pose comme fonction  $f(x)=x^2-10$, qui s'annule en $\sqrt{10}$.


$f$ est continue, strictement croissante et est convexe sur l'intervalle $[0,+\infty[$.

\change

On pose $a=3$, $b=4$. Alors $f(3) \le 0$, $f(4) \ge 0$ donc $\sqrt{10}$ est dans l'intervalle $[3,4]$.

\change

Voici les résultats numériques,
pour la suite $a_n$, en partant de $a_0=3$.

Et sur la colonne de droite est indiquée
une majoration de l'erreur $\epsilon_n = \sqrt{10}-a_n$ (nous y reviendrons).


%%%%%%%%%%%%%%%%%%%%%%%%%%%%%%%%%%%%%%%%%%%%%%%%%%%%%%%%%%%
\diapo

Voici les résultats numériques 
pour une approximation de $(1,10)^{1/12}$.

La fonction est cette fois $f(x) = x^{12} - 1,10$ avec $a=1$ et $b=1,1$

%%%%%%%%%%%%%%%%%%%%%%%%%%%%%%%%%%%%%%%%%%%%%%%%%%%%%%%%%%
\diapo


La méthode de la sécante fournit l'encadrement $a_n \le l \le b$.
Mais comme $b$ est fixe cela ne donne pas d'information exploitable pour $|l-a_n|$.
Voici une façon générale d'estimer l'erreur, à l'aide du théorème des accroissements finis.

Proposition :
 Soit $f : I \to \Rr$ une fonction dérivable et $\ell$ tel que $f(\ell)=0$.
 S'il existe une constante $m>0$ tel que pour tout $x\in I$, $|f'(x)| \ge m$ alors
 $$|x-\ell| \le \frac{|f(x)|}{m} \qquad \text{ pour tout } x \in I.$$

 \change
 
  Par l'inégalité des accroissement finis entre $x$ et $\ell$ :
  $|f(x)-f(\ell)| \ge m |x-\ell|$
  
  mais $\ell$ est un zéro de $f$ donc $f(\ell)=0$, d'où la majoration.



%%%%%%%%%%%%%%%%%%%%%%%%%%%%%%%%%%%%%%%%%%%%%%%%%%%%%%%%%%%
\diapo


Voyons comment la proposition permet d'estimer l'erreur commise par la méthode de la sécante
pour le calcul de $\sqrt{10}$

On pose donc  $f(x)=x^2-10$ et on considère que $f$ est définie sur l'intervalle $I=[3,4]$. 

\change


La dérivée vaut $f'(x)=2x$ 

et donc sur l'intervalle $[3,4]$,  $|f'(x)| \ge 6$. 

\change

Un minorant de $|f'|$ est donc $m=6$; 

ici le zéro dont on cherche l'approximation est $\ell=\sqrt{10}$, 

par rapport aux termes de notre suite $a_n$.

\change


Notons $\epsilon_n$ l'erreur commise lorsque l'on approxime $\sqrt {10}$ par $a_n$.

C'est-à-dire on pose $\epsilon_n = |\ell-a_n|$

par la proposition précédente 
$|\ell-a_n| \le \frac{|f(a_n)|}{m}$

ce qui nous donne $|\ell-a_n| \le  \frac{|a_n^2 - 10|}{6}$

\change

Par exemple, pour $a_2$  on a trouvé $3,16...$ donc $a_2 \le 3,17$

Et donc $\sqrt{10}-a_2 \le \frac{|3,17^2-10|}{6} = 0,489$.

\change

Au bout de $8$ étape on a trouvé $a_8=3,1622776543347473\ldots$
et l'erreur  $\sqrt{10}-a_8$ est $ \le \frac{|a_8^2-10|}{6} = 6,14\ldots \cdot 10^{-9}$. 
On a en fait $7$ décimales exactes après la virgule.

\change


Dans la pratique, voici le nombre d'itérations suffisantes pour avoir une 
précision de $10^{-N}$ pour cet exemple :

  $10^{-10}$ ($\sim 10$ décimales) : $10$ itérations 
  $10^{-100}$ ($\sim 100$ décimales) : $107$ itérations 
  $10^{-1000}$ ($\sim 1000$ décimales) : $1073$ itérations 
  
Grosso-modo, une itération 
de plus donne une décimale supplémentaire.



%%%%%%%%%%%%%%%%%%%%%%%%%%%%%%%%%%%%%%%%%%%%%%%%%%%%%%%%%%
\diapo

Voici l'algorithme : c'est tout simplement 
la mise en \oe uvre de la suite récurrente $(a_n)$.

$a$ et $b$ sont les extrémités de l'intervalle initial, $n$ est le nombre d'itérations.

Et à chaque étape de la boucle on calcule $a_{i+1}$ en fonction de $a_i$, $f$, et $b$.




%%%%%%%%%%%%%%%%%%%%%%%%%%%%%%%%%%%%%%%%%%%%%%%%%%%%%%%%%%%
\diapo

N'oubliez pas de passer du temps à travailler par vous même.



\end{document}
