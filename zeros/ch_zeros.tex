\documentclass[class=report,crop=false]{standalone}
\usepackage[screen]{../exo7book}

\begin{document}

%====================================================================
\chapitre{Zéros des fonctions}
%====================================================================

\insertvideo{Ji7EmmTjDU4}{partie 1. La dichotomie}

\insertvideo{74YbUMAUX2k}{partie 2. La méthode de la sécante}

\insertvideo{Z_BDJaveZNs}{partie 3. La méthode de Newton}


\bigskip
\bigskip

%%%%%%%%%%%%%%%%%%%%%%%%%%%%%%%%%%%%%%%%%%%%%%%%%%%%%%%%%%%%%%%%

Dans ce chapitre nous allons appliquer toutes les notions précédentes sur les suites et les fonctions,
à la recherche des zéros des fonctions. Plus précisément, nous allons voir trois méthodes afin
de trouver des approximations des solutions d'une équation du type $(f(x)=0)$.

%%%%%%%%%%%%%%%%%%%%%%%%%%%%%%%%%%%%%%%%%%%%%%%%%%%%%%%%%%%%%%%%
\section{La dichotomie}

%---------------------------------------------------------------
\subsection{Principe de la dichotomie}

Le principe de dichotomie repose sur la version suivante du \evidence{théorème des valeurs intermédiaires} :
\begin{theoreme}
Soit $f:[a,b]\to\Rr$ une fonction continue sur un segment.
\mybox{Si $f(a)\cdot f(b) \le 0$, alors il existe $\ell\in[a,b]$ tel que $f(\ell)=0$.}
\end{theoreme}

La condition $f(a)\cdot f(b)\le0$ signifie que $f(a)$ et $f(b)$
sont de signes opposés (ou que l'un des deux est nul). L'hypothèse de continuité est essentielle !


\myfigure{1}{
\tikzinput{fig_zeros01}
\qquad\qquad
\tikzinput{fig_zeros02}
}

Ce théorème affirme qu'il existe au moins une solution de l'équation $(f(x)=0)$
dans l'intervalle $[a,b]$. Pour le rendre effectif, et trouver une solution (approchée) de l'équation $(f(x)=0)$,
il s'agit maintenant de l'appliquer sur un intervalle suffisamment petit.
On va voir que cela permet d'obtenir un $\ell$ solution de l'équation $(f(x)=0)$ comme la limite d'une suite.

Voici comment construire une suite d'intervalles emboîtés, dont la longueur tend vers $0$,
et contenant chacun une solution de l'équation $(f(x)=0)$.

On part d'une fonction $f : [a,b] \to \Rr$ continue, avec $a < b$, et $f(a)\cdot f(b)\le0$.

Voici la première étape de la construction : on regarde le signe
de la valeur de la fonction $f$ appliquée au point milieu $\frac{a+b}{2}$.
\begin{itemize}
  \item Si $f(a)\cdot f(\frac{a+b}{2})\le0$, alors il existe
  $c \in [a,\frac{a+b}{2}]$ tel que $f(c)=0$.
  \item Si $f(a)\cdot f(\frac{a+b}{2})>0$, cela implique que
  $f(\frac{a+b}{2})\cdot f(b)\le0$, et alors il existe $c \in [\frac{a+b}{2},b]$ tel que $f(c)=0$.
\end{itemize}

\myfigure{1}{
\tikzinput{fig_zeros03}
\qquad\qquad
\tikzinput{fig_zeros04}
}


Nous avons obtenu un intervalle de longueur moitié dans
lequel l'équation $(f(x)=0)$ admet une solution. On itère alors le procédé pour
diviser de nouveau l'intervalle en deux.

Voici le processus complet :
\begin{itemize}
  \item {\bf Au rang 0 :}

  On pose $a_0=a$, $b_0=b$. Il existe une solution $x_0$ de l'équation $(f(x)=0)$ dans l'intervalle $[a_0,b_0]$.
  \item {\bf Au rang 1 :}

  \begin{itemize}
  \item  Si $f(a_0)\cdot f(\frac{a_0+b_0}{2}) \le 0$, alors on pose $a_1=a_0$ et $b_1=\frac{a_0+b_0}{2}$,
    \item sinon on pose $a_1=\frac{a_0+b_0}{2}$ et $b_1=b$.
    \item Dans les deux cas, il existe une solution $x_1$ de l'équation $(f(x)=0)$ dans l'intervalle $[a_1,b_1]$.
  \end{itemize}

  \item ...
  \item {\bf Au rang $n$ :}   supposons construit un intervalle $[a_n,b_n]$, de longueur $\frac{b-a}{2^n}$, et contenant une solution $x_n$ de l'équation $(f(x)=0)$. Alors :

    \begin{itemize}
   \item Si $f(a_n)\cdot f(\frac{a_n+b_n}{2}) \le 0$, alors on
   pose $a_{n+1}=a_n$ et $b_{n+1}=\frac{a_n+b_n}{2}$,
  \item sinon on pose $a_{n+1}=\frac{a_n+b_n}{2}$ et $b_{n+1}=b_n$.
   \item Dans les deux cas, il existe une solution $x_{n+1}$ de l'équation $(f(x)=0)$ dans l'intervalle $[a_{n+1},b_{n+1}]$.
\end{itemize}

\end{itemize}

\`A chaque étape on a
$$a_n \le x_n \le b_n.$$

On arrête le processus dès que $b_n-a_n=\frac{b-a}{2^n}$ est inférieur à la précision souhaitée.


Comme $(a_n)$ est par construction une suite croissante, $(b_n)$ une suite décroissante,
et $(b_n-a_n) \to 0$ lorsque $n\to +\infty$, les suites $(a_n)$ et $(b_n)$ sont
adjacentes et donc elles admettent une même limite. D'après le théorème des gendarmes,
c'est aussi la limite disons $\ell$ de la suite $(x_n)$. La continuité de $f$ montre
que $f(\ell)=\lim_{n\to +\infty} f(x_n)=\lim_{n\to +\infty} 0 =0$.
Donc les suites $(a_n)$ et $(b_n)$ tendent toutes les deux vers $\ell$,
qui est une solution de l'équation $(f(x)=0)$.


%---------------------------------------------------------------
\subsection{Résultats numériques pour $\sqrt{10}$}

Nous allons calculer une approximation de $\sqrt{10}$.
Soit la fonction $f$ définie par $f(x)=x^2 - 10$, c'est une fonction continue sur $\Rr$
qui s'annule en $\pm\sqrt{10}$. De plus $\sqrt{10}$ est l'unique solution positive de l'équation $(f(x)=0)$.
Nous pouvons restreindre la fonction $f$ à l'intervalle $[3,4]$ : en effet $3^2=9\le 10$ donc $3 \le \sqrt{10}$ et $4^2 = 16 \ge 10$ donc $4 \ge \sqrt{10}$. En d'autre termes $f(3) \le 0$ et $f(4) \ge 0$, donc l'équation $(f(x)=0)$ admet une solution dans l'intervalle $[3,4]$ d'après le théorème des valeurs intermédiaires, et par unicité c'est $\sqrt{10}$, donc $\sqrt{10} \in [3,4]$.

Notez que l'on ne choisit pas pour $f$ la fonction $x\mapsto x-\sqrt{10}$ car on ne connaît pas la valeur de $\sqrt{10}$. C'est ce que l'on cherche à calculer !

\myfigure{1.5}{
\tikzinput{fig_zeros05}
}

Voici les toutes premières étapes :
\begin{enumerate}
  \item On pose $a_0=3$ et $b_0=4$, on a bien $f(a_0)\le 0$ et $f(b_0) \ge 0$.
On calcule $\frac{a_0+b_0}{2} = 3,5$ puis $f(\frac{a_0+b_0}{2})$ :
$f(3,5)=3,5^2-10 = 2,25 \ge0$. Donc $\sqrt{10}$ est dans l'intervalle
$[3 ; 3,5]$ et on pose $a_1 = a_0 = 3$ et $b_1 = \frac{a_0+b_0}{2} = 3,5$.

  \item On sait donc que $f(a_1) \le 0$ et $f(b_1) \ge0$. On calcule
  $f(\frac{a_1+b_1}{2})=f(3,25)=0,5625 \ge 0$, on pose $a_2=3$ et $b_2=3,25$.

  \item On calcule $f(\frac{a_2+b_2}{2})=f(3,125)=-0,23\ldots \le 0$.
  Comme $f(b_2) \ge 0$ alors cette fois $f$ s'annule sur le second intervalle
  $[\frac{a_2+b_2}{2},b_2]$ et on pose $a_3=\frac{a_2+b_2}{2}=3,125$ et $b_3=b_2=3,25$.
\end{enumerate}

\`A ce stade, on a prouvé : $3,125 \le \sqrt{10} \le 3,25$.

Voici la suite des étapes :
$$
\begin{array}{ll}
  a_0 = 3      & b_0 = 4 \\
  a_1 = 3      & b_1 = 3,5 \\
  a_2 = 3      & b_2 = 3,25 \\
  a_3 = 3,125  & b_3 = 3,25 \\
  a_4 = 3,125  & b_4 = 3,1875 \\
  a_5 = 3,15625 & b_5 = 3,1875 \\
  a_6 = 3,15625 & b_6 = 3,171875 \\
  a_7 = 3,15625 & b_7 = 3,164062\ldots \\
  a_8 = 3,16015\ldots & b_8 = 3,164062\ldots \\
\end{array}
$$

Donc en $8$ étapes on obtient l'encadrement :
$$3,160  \le \sqrt{10} \le 3,165$$

En particulier, on vient d'obtenir les deux premières décimales :
$\sqrt{10}=3,16\ldots$





%---------------------------------------------------------------
\subsection{Résultats numériques pour $(1,10)^{1/12}$}


Nous cherchons maintenant une approximation de $(1,10)^{1/12}$.
Soit $f(x) = x^{12} - 1,10$.
On pose $a_0=1$ et $b_0=1,1$. Alors $f(a_0)=-0,10 \le 0$ et $f(b_0)=2,038\ldots \ge 0$.

$$
\begin{array}{ll}
  a_0 = 1             & b_0 = 1,10 \\
  a_1 = 1             & b_1 = 1,05 \\
  a_2 = 1             & b_2 = 1,025 \\
  a_3 = 1             & b_3 = 1,0125 \\
  a_4 = 1,00625       & b_4 = 1,0125 \\
  a_5 = 1,00625       & b_5 = 1,00937\ldots\\
  a_6 = 1,00781\ldots & b_6 = 1,00937\ldots \\
  a_7 = 1,00781\ldots & b_7 = 1,00859\ldots \\
  a_8 = 1,00781\ldots & b_8 = 1,00820\ldots \\
\end{array}
$$

Donc en $8$ étapes on obtient l'encadrement :
$$ 1,00781 \le (1,10)^{1/12} \le 1,00821$$



%---------------------------------------------------------------
\subsection{Calcul de l'erreur}

La méthode de dichotomie a l'énorme avantage de fournir un encadrement d'une solution $\ell$ de l'équation $(f(x)=0)$.
Il est donc facile d'avoir une majoration de l'erreur. En effet, à chaque étape,
la taille l'intervalle contenant $\ell$ est divisée par $2$.
Au départ, on sait que $\ell \in [a,b]$ (de longueur $b-a$) ;
puis $\ell  \in [a_1,b_1]$ (de longueur $\frac{b-a}{2}$) ;
puis $\ell \in [a_2,b_2]$ (de longueur $\frac{b-a}{4}$) ; ... ;
$[a_n,b_n]$ étant de longueur $\frac{b-a}{2^n}$.


Si, par exemple, on souhaite obtenir une approximation de $\ell$ à $10^{-N}$ près,
comme on sait que $a_n \le \ell \le b_n$, on obtient $|\ell-a_n| \le |b_n-a_n| = \frac{b-a}{2^n}$.
Donc pour avoir $|\ell-a_n| \le 10^{-N}$, il suffit de choisir $n$ tel que  $\frac{b-a}{2^n} \le 10^{-N}$.

Nous allons utiliser le logarithme décimal :
\begin{align*}
\frac{b-a}{2^n} \le 10^{-N}
 & \iff (b-a)10^N \le 2^n  \\
 & \iff \log(b-a) + \log(10^N)  \le \log(2^n) \\
 & \iff \log(b-a) + N \le n \log 2 \\
 & \iff n \ge \frac{N +  \log(b-a)}{\log 2} \\
\end{align*}

Sachant $\log 2 = 0,301\ldots$, si par exemple $b-a \le 1$, voici le nombre d'itérations suffisantes pour avoir une  précision de $10^{-N}$ (ce qui correspond, à peu près, à $N$ chiffres exacts après la virgule).
\begin{center}
\begin{tabular}{ll}
  $10^{-10}$ ($\sim 10$ décimales) &  $34$ itérations \\
  $10^{-100}$ ($\sim 100$ décimales) &  $333$ itérations \\
  $10^{-1000}$ ($\sim 1000$ décimales) &  $3322$ itérations \\
\end{tabular}
\end{center}

Il faut entre $3$ et $4$ itérations supplémentaires pour obtenir une nouvelle décimale.

\begin{remarque*}
En toute rigueur il ne faut pas confondre précision et nombre de décimales exactes,
par exemple $0,999$ est une approximation de $1,000$ à $10^{-3}$ près, mais aucune
décimale après la virgule n'est exacte.  En pratique, c'est la précision qui est la plus importante,
mais il est plus frappant de parler du nombre de décimales exactes.
\end{remarque*}



%---------------------------------------------------------------
\subsection{Algorithmes}

Voici comment implémenter la dichotomie dans le langage \Python. Tout d'abord
on définit une fonction $f$ (ici par exemple $f(x)=x^2-10$) :

\insertcode{algos/dichotomie-tex1.py}{dichotomie.py (1)}

Puis la dichotomie proprement dite : en entrée de la fonction, on a
pour variables $a,b$ et $n$ le nombre d'étapes voulues.

\insertcode{algos/dichotomie-tex2.py}{dichotomie.py (2)}

Même algorithme, mais avec cette fois en entrée la précision souhaitée :

\insertcode{algos/dichotomie-tex3.py}{dichotomie.py (3)}


Enfin, voici la version récursive de l'algorithme de dichotomie.

\insertcode{algos/dichotomie-tex4.py}{dichotomie.py (4)}


%---------------------------------------------------------------
%\subsection{Mini-exercices}

\begin{miniexercices}
\begin{enumerate}
  \item \`A la main, calculer un encadrement à $0,1$ près de $\sqrt{3}$.
  Idem avec $\sqrt[3]{2}$.

  \item Calculer une approximation des solutions de l'équation $x^3+1=3x$.

  \item Est-il plus efficace de diviser l'intervalle en $4$ au lieu d'en $2$ ?
  (\`A chaque itération, la dichotomie classique nécessite l'évaluation de $f$ en une nouvelle valeur
  $\frac{a+b}{2}$ pour une précision améliorée d'un facteur $2$.)

  \item \'Ecrire un algorithme pour calculer plusieurs solutions de $(f(x)=0)$.

  \item On se donne un tableau trié de taille $N$, rempli de nombres appartenant à $\{1,\ldots,n\}$.
  \'Ecrire un algorithme qui teste si une valeur $k$ apparaît dans le tableau et en quelle position.

\end{enumerate}
\end{miniexercices}


%%%%%%%%%%%%%%%%%%%%%%%%%%%%%%%%%%%%%%%%%%%%%%%%%%%%%%%%%%%%%%%%
\section{La méthode de la sécante}

%---------------------------------------------------------------
\subsection{Principe de la sécante}

L'idée de la méthode de la sécante est très simple : pour une fonction $f$ continue sur un intervalle $[a,b]$, et vérifiant $f(a) \le 0$, $f(b) > 0$, on trace le segment $[AB]$ où $A=(a,f(a))$ et $B=(b,f(b))$. Si le segment reste au-dessus du graphe de $f$ alors la fonction s'annule sur l'intervalle $[a',b]$ où $(a',0)$ est le point d'intersection de
la droite $(AB)$ avec l'axe des abscisses. La droite $(AB)$ s'appelle la \defi{sécante}.
On recommence en partant maintenant de l'intervalle $[a',b]$ pour obtenir une valeur $a''$.


\myfigure{1}{
\tikzinput{fig_zeros06}
}

\begin{proposition}
 Soit $f : [a,b] \to \Rr$ une fonction continue, strictement croissante et convexe telle
 que $f(a) \le 0$, $f(b) > 0$.
 Alors la suite définie par
 $$a_0=a \qquad \text{ et } \qquad a_{n+1} = a_n - \frac{b-a_n}{f(b)-f(a_n)}f(a_n)$$
 est croissante et converge vers la solution $\ell$ de $(f(x)=0)$.
\end{proposition}

L'hypothèse $f$ \defi{convexe} signifie exactement que pour tout $x,x'$
dans $[a,b]$ la sécante (ou corde) entre $(x,f(x))$ et $(x',f(x'))$ est au-dessus du graphe de $f$.

\myfigure{1}{
\tikzinput{fig_zeros07}
}

\begin{proof}
\begin{enumerate}
  \item Justifions d'abord la construction de la suite récurrente.

	  L'équation de la droite passant par les deux points $(a,f(a))$ et $(b,f(b))$
est
$$y=(x-a) \frac{f(b)-f(a)}{b-a} + f(a)$$
Cette droite intersecte l'axe des abscisses en $(a',0)$ qui vérifie donc
$0=(a'-a) \frac{f(b)-f(a)}{b-a} + f(a)$, donc $a'=a - \frac{b-a}{f(b)-f(a)}f(a)$.

  \item Croissance de $(a_n)$.

  Montrons par récurrence que $f(a_n)\le 0$. C'est vrai au rang $0$ car $f(a_0)=f(a)\le 0$ par hypothèse.  Supposons vraie l'hypothèse au rang $n$. Si $a_{n+1} < a_n$ (un cas qui s'avérera \emph{a posteriori} jamais réalisé), alors comme $f$ est strictement croissante, on a $f(a_{n+1})<f(a_n)$, et en particulier $f(a_{n+1})\le 0$. Sinon $a_{n+1}\ge a_n$. Comme $f$ est convexe : la sécante entre $(a_n, f(a_n))$ et $(b, f(b))$ est au-dessus du graphe de $f$. En particulier le point $(a_{n+1},0)$ (qui est sur cette sécante par définition $a_{n+1}$) est au-dessus du point $(a_{n+1},f(a_{n+1}))$, et donc $f(a_{n+1})\le 0$ aussi dans ce cas, ce qui conclut la récurrence.

  Comme $f(a_n) \le 0$ et $f$ est croissante, alors
  par la formule $a_{n+1} = a_n - \frac{b-a_n}{f(b)-f(a_n)}f(a_n)$, on obtient que $a_{n+1} \ge a_n$.


  \item Convergence de $(a_n)$.

  La suite $(a_n)$ est croissante et majorée par $b$, donc elle converge. Notons $\ell$ sa limite. Par continuité $f(a_n)\to f(\ell)$. Comme pour tout $n$, $f(a_n) \le 0$, on en déduit que $f(\ell)\le 0$. En particulier, comme on suppose $f(b)>0$, on a $\ell<b$. Comme $a_n \to \ell$, $a_{n+1} \to \ell$, $f(a_n)\to f(\ell)$, l'égalité $a_{n+1} = a_n - \frac{b-a_n}{f(b)-f(a_n)}f(a_n)$ devient à la limite (lorsque $n\to+\infty$) :
  $\ell = \ell - \frac{b-\ell}{f(b)-f(\ell)} f(\ell)$, ce qui implique $f(\ell)=0$.

  Conclusion : $(a_n)$ converge vers la solution de $(f(x)=0)$.

\end{enumerate}



\bigskip



\end{proof}

%---------------------------------------------------------------
\subsection{Résultats numériques pour $\sqrt{10}$}

Pour $a=3$, $b=4$, $f(x)=x^2-10$ voici les résultats numériques, est aussi indiquée
une majoration de l'erreur $\epsilon_n = \sqrt{10}-a_n$ (voir ci-après).
$$
\begin{array}{l|l}
  a_0 = 3                          & \epsilon_0 \le 0,1666\ldots \\
  a_1 = 3,14285714285\ldots    & \epsilon_1 \le 0,02040\ldots \\
  a_2 = 3,16000000000\ldots    & \epsilon_2 \le 0,00239\ldots \\
  a_3 = 3,16201117318\ldots   & \epsilon_3 \le 0,00028\ldots \\
  a_4 = 3,16224648985\ldots   & \epsilon_4 \le 3,28\ldots \cdot 10^{-5} \\
  a_5 = 3,16227401437\ldots    & \epsilon_5 \le 3,84\ldots \cdot 10^{-6}\\
  a_6 = 3,16227723374\ldots    & \epsilon_6 \le 4,49\ldots \cdot 10^{-7} \\
  a_7 = 3,16227761029\ldots   & \epsilon_7 \le 5,25\ldots \cdot 10^{-8} \\
  a_8 = 3,16227765433\ldots   & \epsilon_8 \le 6,14\ldots \cdot 10^{-9} \\
\end{array}
$$

%---------------------------------------------------------------
\subsection{Résultats numériques pour $(1,10)^{1/12}$}

Voici les résultats numériques avec une majoration de l'erreur $\epsilon_n = (1,10)^{1/12}-a_n$,
avec $f(x) = x^{12} - 1,10$, $a=1$ et $b=1,1$
$$
\begin{array}{l|l}
  a_0 = 1                          & \epsilon_0 \le 0,0083\ldots \\
  a_1 = 1,00467633\ldots   & \epsilon_1 \le 0,0035\ldots \\
  a_2 = 1,00661950\ldots   & \epsilon_2 \le 0,0014\ldots \\
  a_3 = 1,00741927\ldots   & \epsilon_3 \le 0,00060\ldots \\
  a_4 = 1,00774712\ldots    & \epsilon_4 \le 0,00024\ldots \\
  a_5 = 1,00788130\ldots    & \epsilon_5 \le 0,00010\ldots\\
  a_6 = 1,00793618\ldots   & \epsilon_6 \le 4,14\ldots \cdot 10^{-5} \\
  a_7 = 1,00795862\ldots   & \epsilon_7 \le 1,69\ldots \cdot 10^{-5} \\
  a_8 = 1,00796779\ldots   & \epsilon_8 \le 6,92\ldots \cdot 10^{-6} \\
\end{array}
$$

%---------------------------------------------------------------
\subsection{Calcul de l'erreur}

La méthode de la sécante fournit l'encadrement $a_n \le l \le b$.
Mais comme $b$ est fixe cela ne donne pas d'information exploitable pour $|l-a_n|$.
Voici une façon générale d'estimer l'erreur, à l'aide du théorème des accroissements finis.

\begin{proposition}
 Soit $f : I \to \Rr$ une fonction dérivable et $\ell$ tel que $f(\ell)=0$.
 S'il existe une constante $m>0$ telle que pour tout $x\in I$, $|f'(x)| \ge m$ alors
 $$|x-\ell| \le \frac{|f(x)|}{m} \qquad \text{ pour tout } x \in I.$$
\end{proposition}


\begin{proof}
  Par l'inégalité des accroissement finis entre $x$ et $\ell$ :
  $|f(x)-f(\ell)| \ge m |x-\ell|$
  mais $f(\ell)=0$, d'où la majoration.
\end{proof}

\begin{exemple}[Erreur pour $\sqrt{10}$]
Soit $f(x)=x^2-10$ et l'intervalle $I=[3,4]$. Alors
$f'(x)=2x$ donc $|f'(x)| \ge 6$ sur $I$.
On pose donc $m=6$, $\ell=\sqrt{10}$, $x=a_n$.
On obtient l'estimation de l'erreur :
$$\epsilon_n = |\ell-a_n| \le \frac{|f(a_n)|}{m} = \frac{|a_n^2 - 10|}{6}$$


Par exemple
on a trouvé $a_2 = 3,16... \le 3,17$
donc $\sqrt{10}-a_2 \le \frac{|3,17^2-10|}{6} = 0,489$.

Pour $a_8$ on a trouvé $a_8=3,1622776543347473\ldots$
donc $\sqrt{10}-a_8 \le \frac{|a_8^2-10|}{6} = 6,14\ldots \cdot 10^{-9}$. On a en fait $7$ décimales exactes après la virgule.
\end{exemple}

Dans la pratique, voici le nombre d'itérations suffisantes pour avoir une
précision de $10^{-n}$ pour cet exemple. Grosso-modo, une itération
de plus donne une décimale supplémentaire.
\begin{center}
\begin{tabular}{ll}
  $10^{-10}$ ($\sim 10$ décimales) &  $10$ itérations \\
  $10^{-100}$ ($\sim 100$ décimales) &  $107$ itérations \\
  $10^{-1000}$ ($\sim 1000$ décimales) &  $1073$ itérations \\
\end{tabular}
\end{center}



\bigskip


\begin{exemple}[Erreur pour $(1,10)^{1/12}$]
  On pose $f(x)=x^{12}-1,10$, $I=[1;1,10]$ et $\ell=(1,10)^{1/12}$.
  Comme $f'(x)=12x^{11}$, si on pose de plus $m=12$, on a $|f'(x)| \ge m$ pour $x\in I$.
  On obtient
  $$\epsilon_n =|\ell-a_n| \le \frac{|a_n^{12} - 1,10|}{12}.$$

  Par exemple $a_8 = 1.0079677973185432\ldots$ donc
  $$|(1,10)^{1/12}-a_8| \le \frac{|a_8^{12} - 1,10|}{12} = 6,92\ldots \cdot 10^{-6}.$$
\end{exemple}


%---------------------------------------------------------------
\subsection{Algorithme}

Voici l'algorithme : c'est tout simplement
la mise en \oe uvre de la suite récurrente $(a_n)$.

\insertcode{algos/secante-tex.py}{secante.py}


%---------------------------------------------------------------
%\subsection{Mini-exercices}

\begin{miniexercices}
\begin{enumerate}
  \item \`A la main, calculer un encadrement à $0,1$ près de $\sqrt{3}$.
  Idem avec $\sqrt[3]{2}$.

  \item Calculer une approximation des solutions de l'équation $x^3+1=3x$.

  \item Calculer une approximation de la solution de l'équation $(\cos x = 0)$ sur $[0,\pi]$.
  Idem avec $(\cos x = 2\sin x)$.

  \item \'Etudier l'équation $(\exp(-x) = - \ln(x))$. Donner une approximation de la (ou des)
  solution(s) et une majoration de l'erreur correspondante.
\end{enumerate}
\end{miniexercices}


%%%%%%%%%%%%%%%%%%%%%%%%%%%%%%%%%%%%%%%%%%%%%%%%%%%%%%%%%%%%%%%%
\section{La méthode de Newton}

%---------------------------------------------------------------
\subsection{Méthode de Newton}

La méthode de Newton consiste à remplacer la sécante de la méthode précédente par la tangente.
Elle est d'une redoutable efficacité.


Partons d'une fonction dérivable $f:[a,b] \to \Rr$ et d'un point $u_0 \in[a,b]$.
On appelle $(u_1,0)$ l'intersection de
la tangente au graphe de $f$ en $(u_0,f(u_0))$ avec l'axe des abscisses.
Si $u_1 \in[a,b]$ alors on recommence l'opération avec la tangente au point d'abscisse $u_1$.
Ce processus conduit à la définition d'une suite récurrente :
$$u_0 \in [a,b] \qquad \text{ et } \qquad u_{n+1} = u_n - \frac{f(u_n)}{f'(u_n)}.$$

\begin{proof}
En effet la tangente au point d'abscisse $u_n$ a pour équation :
$y = f'(u_n)(x-u_n)+f(u_n)$. Donc le point $(x,0)$ appartenant à la tangente (et à l'axe des abscisses)
vérifie $0=f'(u_n)(x-u_n)+f(u_n)$. D'où $x=u_n - \frac{f(u_n)}{f'(u_n)}.$
\end{proof}

\myfigure{1}{
\tikzinput{fig_zeros09}
}


%---------------------------------------------------------------
\subsection{Résultats pour $\sqrt{10}$}

Pour calculer $\sqrt{a}$, on pose $f(x)=x^2-a$, avec $f'(x)=2x$. La suite issue de la méthode de Newton est déterminée par $u_0>0$ et la relation de récurrence $u_{n+1} = u_n - \frac{u_n^2-a}{2u_n}$.
Suite qui pour cet exemple s'appelle \defi{suite de Héron} et que l'on récrit souvent
\mybox{$\displaystyle u_0>0 \quad \text{ et } \quad u_{n+1} = \frac12 \left(u_n+\frac{a}{u_n}\right).$}

\begin{proposition}
\label{prop:heron}
Cette suite $(u_n)$ converge vers $\sqrt{a}$.
\end{proposition}

Pour le calcul de $\sqrt{10}$, on pose par exemple $u_0=4$, et on peut même commencer les calculs à la main :

$$
\begin{array}{l}
  u_0 = 4     \\
  u_1 = \frac12 \left(u_0+\frac{10}{u_0}\right) = \frac12\left(4+\frac{10}{4}\right) = \frac{13}{4} = 3,25 \\
  u_2 = \frac12 \left(u_1+\frac{10}{u_1}\right) = \frac12\left(\tfrac{13}{4}+\frac{10}{\tfrac{13}{4}}\right)
  = \frac{329}{104} = 3,1634\ldots \\
  u_3 = \frac12 \left(u_2+\frac{10}{u_2}\right) = \frac{216\,401}{68\,432} = 3,16227788 \ldots \\
  u_4 = 3,162277660168387\ldots
\end{array}
$$

Pour $u_4$ on obtient $\sqrt{10} = 3,1622776601683\ldots$
avec déjà $13$ décimales exactes !

Voici la preuve de la convergence de la suite $(u_n)$ vers $\sqrt{a}$.
\begin{proof}
$$ u_0>0 \quad \text{ et } \quad u_{n+1} = \frac12 \left(u_n+\frac{a}{u_n}\right).$$
\begin{enumerate}
  \item Montrons que $u_n \ge \sqrt{a}$ pour $n\ge1$.

  Tout d'abord
  $$u_{n+1}^2-a = \frac14 \left(\frac{u_n^2 + a}{u_n}\right)^2 - a
  = \frac{1}{4u_n^2}(u_n^4-2au_n^2+a^2)=\frac14 \frac{(u_n^2-a)^2}{u_n^2}$$

  Donc $u_{n+1}^2 - a \ge 0$. Comme il est clair que pour tout $n\ge 0$, $u_{n} \ge 0$, on en déduit que pour tout $n\ge 0$, $u_{n+1} \ge \sqrt{a}$. (Notez que $u_0$ lui est quelconque.)


  \item Montrons que $(u_n)_{n\ge1}$ est une suite décroissante qui converge.

  Comme $\frac{u_{n+1}}{u_n} = \frac12 \left(1+\frac{a}{u_n^2}\right)$,
  et que pour $n\ge 1$ on vient de voir que
  $u_n^2 \ge a$ (donc $\frac{a}{u_n^2}\le 1$), alors $\frac{u_{n+1}}{u_n} \le 1$, pour tout $n\le 1$.

  Conséquence : la suite $(u_n)_{n\ge1}$ est décroissante et minorée par $0$ donc elle converge.

  \item $(u_n)$ converge vers $\sqrt{a}$.

  Notons $\ell$ la limite de $(u_n)$. Alors $u_n \to \ell$ et $u_{n+1} \to \ell$.
  Lorsque $n\to +\infty$ dans la relation $u_{n+1} = \frac12 \left(u_n+\frac{a}{u_n}\right)$, on obtient
  $\ell = \frac12 \left(\ell+\frac{a}{\ell}\right)$. Ce qui conduit à la relation $\ell^2=a$ et par positivité de la suite, $\ell = \sqrt{a}$.
\end{enumerate}

\end{proof}

%---------------------------------------------------------------
\subsection{Résultats numériques pour $(1,10)^{1/12}$}

Pour calculer $(1,10)^{1/12}$, on pose $f(x)=x^{12}-a$ avec $a=1,10$.
On a $f'(x)=12x^{11}$.
On obtient $u_{n+1} = u_n - \frac{u_n^{12}-a}{12u_n^{11}}$. Ce que l'on reformule ainsi :
$$u_0>0 \quad \text{ et } \quad u_{n+1} = \frac1{12} \left(11u_n+\frac{a}{u_n^{11}}\right).$$

Voici les résultats numériques pour $(1,10)^{1/12}$ en partant de $u_0=1$.
$$
\begin{array}{l}
  u_0 = 1     \\
  u_1 = 1,0083333333333333 \ldots \\
  u_2 = 1,0079748433368980\ldots \\
  u_3 = 1,0079741404315996 \ldots \\
  u_4 = 1,0079741404289038 \ldots \\
\end{array}
$$

Toutes les décimales affichées pour $u_4$ sont exactes :
$(1,10)^{1/12}=1,0079741404289038\ldots$

%---------------------------------------------------------------
\subsection{Calcul de l'erreur pour $\sqrt{10}$}

\begin{proposition}
\begin{enumerate}
  \item Soit $k$ tel que $u_1-\sqrt a\le k$. Alors pour tout $n\ge 1$ :
  $$u_n - \sqrt{a} \le 2\sqrt{a} \left( \frac{k}{2\sqrt{a}} \right)^{2^{n-1}}$$

  \item Pour $a=10$, $u_0=4$, on a :
  $$u_n - \sqrt{10} \le 8 \left(\frac{1}{24} \right)^{2^{n-1}}$$
%  Donc si $\ell \ge 1+\frac{\log\left(\frac{\ell-\log8}{\log24}\right)}{\log2}$ alors $u_n - \sqrt{10} \le 10^{-\ell}$.
\end{enumerate}
\end{proposition}

Admirez la puissance de la méthode de Newton : $11$ itérations donnent déjà $1000$ décimales
exactes après la virgule. Cette rapidité de convergence se justifie grâce
au calcul de l'erreur : la précision est multipliée par $2$ à chaque étape, donc à chaque
itération le nombre de décimales exactes double !

\begin{center}
\begin{tabular}{ll}
  $10^{-10}$ ($\sim 10$ décimales) &  $4$ itérations \\
  $10^{-100}$ ($\sim 100$ décimales) &  $8$ itérations \\
  $10^{-1000}$ ($\sim 1000$ décimales) &  $11$ itérations \\
\end{tabular}
\end{center}


\begin{proof}
\begin{enumerate}
  \item Dans la preuve de la proposition \ref{prop:heron}, nous avons vu l'égalité :
 $$u_{n+1}^2-a = \frac{(u_n^2-a)^2}{4u_n^2} \text{ donc }
 (u_{n+1}-\sqrt{a})(u_{n+1}+\sqrt{a}) = \frac{(u_{n}-\sqrt{a})^2(u_{n}+\sqrt{a})^2}{4u_n^2}$$

 Ainsi comme $u_n \ge \sqrt{a}$ pour $n\ge 1$ :
 $$u_{n+1}-\sqrt{a} = (u_{n}-\sqrt{a})^2 \times \frac{1}{u_{n+1}+\sqrt{a}} \times \frac14 \left( 1 + \frac{\sqrt{a}}{u_n} \right)^2
 \le (u_{n}-\sqrt{a})^2 \times \frac{1}{2\sqrt{a}}\times \frac14\cdot (1+1)^2  = \frac{1}{2\sqrt{a}}(u_{n}-\sqrt{a})^2 $$

 Si $k$ vérifie $u_1-\sqrt a\le k$, nous allons en déduire par récurrence, pour tout $n\ge 1$, la formule
 $$u_n - \sqrt{a} \le 2\sqrt{a} \left( \frac{k}{2\sqrt{a}} \right)^{2^{n-1}}$$
 C'est vrai pour $n=1$. Supposons la formule vraie au rang $n$, alors :
 $$u_{n+1}-\sqrt{a} \le \frac{1}{2\sqrt{a}}(u_{n}-\sqrt{a})^2
 =  \frac{1}{2\sqrt{a}} (2\sqrt{a})^2 \left(\left( \frac{k}{2\sqrt{a}} \right)^{2^{n-1}}\right)^2
 = 2\sqrt{a} \left( \frac{k}{2\sqrt{a}} \right)^{2^{n}}$$
 La formule est donc vrai au rang suivant.

  \item Pour $a = 10$ avec $u_0=4$ on a $u_1=3,25$. Comme $3 \le \sqrt{10} \le 4$
  alors $u_1-\sqrt{10} \le u_1-3 \le \frac14$. On fixe donc $k = \frac14$.
  Toujours par l'encadrement $3 \le \sqrt{10} \le 4$, la formule obtenue précédemment devient
  $$u_n - \sqrt{a} \le 2 \cdot 4 \left( \frac{\frac14}{2\cdot 3} \right)^{2^{n-1}}=8 \left(\frac{1}{24} \right)^{2^{n-1}}\; .$$
\end{enumerate}

\end{proof}

%---------------------------------------------------------------
\subsection{Algorithme}

Voici l'algorithme pour le calcul de $\sqrt{a}$.
On précise en entrée le réel $a\ge0$ dont on veut calculer
la racine et le nombre $n$ d'itérations.

\insertcode{algos/newton-tex.py}{newton.py}

\bigskip

En utilisant le module \codeinline{decimal} le calcul de $u_n$ pour $n=11$
donne $1000$ décimales de $\sqrt{10}$ :

\begin{center}
{\footnotesize
3, \hfill\hfill\  \\
16227766016837933199889354443271853371955513932521 \ 68268575048527925944386392382213442481083793002951 \\
87347284152840055148548856030453880014690519596700 \ 15390334492165717925994065915015347411333948412408 \\
53169295770904715764610443692578790620378086099418 \ 28371711548406328552999118596824564203326961604691 \\
31433612894979189026652954361267617878135006138818 \ 62785804636831349524780311437693346719738195131856 \\
78403231241795402218308045872844614600253577579702 \ 82864402902440797789603454398916334922265261206779 \\
26516760310484366977937569261557205003698949094694 \ 21850007358348844643882731109289109042348054235653 \\
40390727401978654372593964172600130699000095578446 \ 31096267906944183361301813028945417033158077316263 \\
86395193793704654765220632063686587197822049312426 \ 05345411160935697982813245229700079888352375958532 \\
85792513629646865114976752171234595592380393756251 \ 25369855194955325099947038843990336466165470647234 \\
99979613234340302185705218783667634578951073298287 \ 51579452157716521396263244383990184845609357626020\ \\
  }
\end{center}

%---------------------------------------------------------------
%\subsection{Mini-exercices}

\begin{miniexercices}
\begin{enumerate}
  \item \`A la calculette, calculer les trois premières étapes pour une approximation de
  $\sqrt{3}$, sous forme de nombres rationnels.   Idem avec $\sqrt[3]{2}$.

  \item Implémenter la méthode de Newton, étant données une fonction $f$ et sa dérivée $f'$.

  \item Calculer une approximation des solutions de l'équation $x^3+1=3x$.

  \item Soit $a>0$. Comment calculer $\frac{1}{a}$ par une méthode de Newton ?

  \item Calculer $n$ de sorte que $u_n-\sqrt{10} \le 10^{-\ell}$ (avec $u_0=4$,
  $u_{n+1} =\frac{1}{2} \left(u_n+\frac{a}{u_n}\right)$, $a=10$).

\end{enumerate}
\end{miniexercices}

\auteurs{
Auteurs : Arnaud Bodin, Niels Borne, Laura Desideri

Dessins : Benjamin Boutin
}


\finchapitre
\end{document}


