
%%%%%%%%%%%%%%%%%% PREAMBULE %%%%%%%%%%%%%%%%%%

\documentclass[aspectratio=169,utf8]{beamer}
%\documentclass[aspectratio=169,handout]{beamer}

\usetheme{Boadilla}
%\usecolortheme{seahorse}
\usecolortheme[RGB={245,66,24}]{structure}
\useoutertheme{infolines}

% packages
\usepackage{amsfonts,amsmath,amssymb,amsthm}
\usepackage[utf8]{inputenc}
\usepackage[T1]{fontenc}
\usepackage{lmodern}

\usepackage[francais]{babel}
\usepackage{fancybox}
\usepackage{graphicx}

\usepackage{float}
\usepackage{xfrac}

%\usepackage[usenames, x11names]{xcolor}
\usepackage{tikz}
\usepackage{pgfplots}
\usepackage{datetime}



%-----  Package unités -----
\usepackage{siunitx}
\sisetup{locale = FR,detect-all,per-mode = symbol}

%\usepackage{mathptmx}
%\usepackage{fouriernc}
%\usepackage{newcent}
%\usepackage[mathcal,mathbf]{euler}

%\usepackage{palatino}
%\usepackage{newcent}
% \usepackage[mathcal,mathbf]{euler}



% \usepackage{hyperref}
% \hypersetup{colorlinks=true, linkcolor=blue, urlcolor=blue,
% pdftitle={Exo7 - Exercices de mathématiques}, pdfauthor={Exo7}}


%section
% \usepackage{sectsty}
% \allsectionsfont{\bf}
%\sectionfont{\color{Tomato3}\upshape\selectfont}
%\subsectionfont{\color{Tomato4}\upshape\selectfont}

%----- Ensembles : entiers, reels, complexes -----
\newcommand{\Nn}{\mathbb{N}} \newcommand{\N}{\mathbb{N}}
\newcommand{\Zz}{\mathbb{Z}} \newcommand{\Z}{\mathbb{Z}}
\newcommand{\Qq}{\mathbb{Q}} \newcommand{\Q}{\mathbb{Q}}
\newcommand{\Rr}{\mathbb{R}} \newcommand{\R}{\mathbb{R}}
\newcommand{\Cc}{\mathbb{C}} 
\newcommand{\Kk}{\mathbb{K}} \newcommand{\K}{\mathbb{K}}

%----- Modifications de symboles -----
\renewcommand{\epsilon}{\varepsilon}
\renewcommand{\Re}{\mathop{\text{Re}}\nolimits}
\renewcommand{\Im}{\mathop{\text{Im}}\nolimits}
%\newcommand{\llbracket}{\left[\kern-0.15em\left[}
%\newcommand{\rrbracket}{\right]\kern-0.15em\right]}

\renewcommand{\ge}{\geqslant}
\renewcommand{\geq}{\geqslant}
\renewcommand{\le}{\leqslant}
\renewcommand{\leq}{\leqslant}
\renewcommand{\epsilon}{\varepsilon}

%----- Fonctions usuelles -----
\newcommand{\ch}{\mathop{\text{ch}}\nolimits}
\newcommand{\sh}{\mathop{\text{sh}}\nolimits}
\renewcommand{\tanh}{\mathop{\text{th}}\nolimits}
\newcommand{\cotan}{\mathop{\text{cotan}}\nolimits}
\newcommand{\Arcsin}{\mathop{\text{arcsin}}\nolimits}
\newcommand{\Arccos}{\mathop{\text{arccos}}\nolimits}
\newcommand{\Arctan}{\mathop{\text{arctan}}\nolimits}
\newcommand{\Argsh}{\mathop{\text{argsh}}\nolimits}
\newcommand{\Argch}{\mathop{\text{argch}}\nolimits}
\newcommand{\Argth}{\mathop{\text{argth}}\nolimits}
\newcommand{\pgcd}{\mathop{\text{pgcd}}\nolimits} 


%----- Commandes divers ------
\newcommand{\ii}{\mathrm{i}}
\newcommand{\dd}{\text{d}}
\newcommand{\id}{\mathop{\text{id}}\nolimits}
\newcommand{\Ker}{\mathop{\text{Ker}}\nolimits}
\newcommand{\Card}{\mathop{\text{Card}}\nolimits}
\newcommand{\Vect}{\mathop{\text{Vect}}\nolimits}
\newcommand{\Mat}{\mathop{\text{Mat}}\nolimits}
\newcommand{\rg}{\mathop{\text{rg}}\nolimits}
\newcommand{\tr}{\mathop{\text{tr}}\nolimits}


%----- Structure des exercices ------

\newtheoremstyle{styleexo}% name
{2ex}% Space above
{3ex}% Space below
{}% Body font
{}% Indent amount 1
{\bfseries} % Theorem head font
{}% Punctuation after theorem head
{\newline}% Space after theorem head 2
{}% Theorem head spec (can be left empty, meaning ‘normal’)

%\theoremstyle{styleexo}
\newtheorem{exo}{Exercice}
\newtheorem{ind}{Indications}
\newtheorem{cor}{Correction}


\newcommand{\exercice}[1]{} \newcommand{\finexercice}{}
%\newcommand{\exercice}[1]{{\tiny\texttt{#1}}\vspace{-2ex}} % pour afficher le numero absolu, l'auteur...
\newcommand{\enonce}{\begin{exo}} \newcommand{\finenonce}{\end{exo}}
\newcommand{\indication}{\begin{ind}} \newcommand{\finindication}{\end{ind}}
\newcommand{\correction}{\begin{cor}} \newcommand{\fincorrection}{\end{cor}}

\newcommand{\noindication}{\stepcounter{ind}}
\newcommand{\nocorrection}{\stepcounter{cor}}

\newcommand{\fiche}[1]{} \newcommand{\finfiche}{}
\newcommand{\titre}[1]{\centerline{\large \bf #1}}
\newcommand{\addcommand}[1]{}
\newcommand{\video}[1]{}

% Marge
\newcommand{\mymargin}[1]{\marginpar{{\small #1}}}

\def\noqed{\renewcommand{\qedsymbol}{}}


%----- Presentation ------
\setlength{\parindent}{0cm}

%\newcommand{\ExoSept}{\href{http://exo7.emath.fr}{\textbf{\textsf{Exo7}}}}

\definecolor{myred}{rgb}{0.93,0.26,0}
\definecolor{myorange}{rgb}{0.97,0.58,0}
\definecolor{myyellow}{rgb}{1,0.86,0}

\newcommand{\LogoExoSept}[1]{  % input : echelle
{\usefont{U}{cmss}{bx}{n}
\begin{tikzpicture}[scale=0.1*#1,transform shape]
  \fill[color=myorange] (0,0)--(4,0)--(4,-4)--(0,-4)--cycle;
  \fill[color=myred] (0,0)--(0,3)--(-3,3)--(-3,0)--cycle;
  \fill[color=myyellow] (4,0)--(7,4)--(3,7)--(0,3)--cycle;
  \node[scale=5] at (3.5,3.5) {Exo7};
\end{tikzpicture}}
}


\newcommand{\debutmontitre}{
  \author{} \date{} 
  \thispagestyle{empty}
  \hspace*{-10ex}
  \begin{minipage}{\textwidth}
    \titlepage  
  \vspace*{-2.5cm}
  \begin{center}
    \LogoExoSept{2.5}
  \end{center}
  \end{minipage}

  \vspace*{-0cm}
  
  % Astuce pour que le background ne soit pas discrétisé lors de la conversion pdf -> png
\begin{tikzpicture}
        \fill[opacity=0,green!60!black] (0,0)--++(0,0)--++(0,0)--++(0,0)--cycle; 
\end{tikzpicture}

% toc S'affiche trop tot :
% \tableofcontents[hideallsubsections, pausesections]
}

\newcommand{\finmontitre}{
  \end{frame}
  \setcounter{framenumber}{0}
} % ne marche pas pour une raison obscure

%----- Commandes supplementaires ------

% \usepackage[landscape]{geometry}
% \geometry{top=1cm, bottom=3cm, left=2cm, right=10cm, marginparsep=1cm
% }
% \usepackage[a4paper]{geometry}
% \geometry{top=2cm, bottom=2cm, left=2cm, right=2cm, marginparsep=1cm
% }

%\usepackage{standalone}


% New command Arnaud -- november 2011
\setbeamersize{text margin left=24ex}
% si vous modifier cette valeur il faut aussi
% modifier le decalage du titre pour compenser
% (ex : ici =+10ex, titre =-5ex

\theoremstyle{definition}
%\newtheorem{proposition}{Proposition}
%\newtheorem{exemple}{Exemple}
%\newtheorem{theoreme}{Théorème}
%\newtheorem{lemme}{Lemme}
%\newtheorem{corollaire}{Corollaire}
%\newtheorem*{remarque*}{Remarque}
%\newtheorem*{miniexercice}{Mini-exercices}
%\newtheorem{definition}{Définition}

% Commande tikz
\usetikzlibrary{calc}
\usetikzlibrary{patterns,arrows}
\usetikzlibrary{matrix}
\usetikzlibrary{fadings} 

%definition d'un terme
\newcommand{\defi}[1]{{\color{myorange}\textbf{\emph{#1}}}}
\newcommand{\evidence}[1]{{\color{blue}\textbf{\emph{#1}}}}
\newcommand{\assertion}[1]{\emph{\og#1\fg}}  % pour chapitre logique
%\renewcommand{\contentsname}{Sommaire}
\renewcommand{\contentsname}{}
\setcounter{tocdepth}{2}



%------ Figures ------

\def\myscale{1} % par défaut 
\newcommand{\myfigure}[2]{  % entrée : echelle, fichier figure
\def\myscale{#1}
\begin{center}
\footnotesize
{#2}
\end{center}}


%------ Encadrement ------

\usepackage{fancybox}


\newcommand{\mybox}[1]{
\setlength{\fboxsep}{7pt}
\begin{center}
\shadowbox{#1}
\end{center}}

\newcommand{\myboxinline}[1]{
\setlength{\fboxsep}{5pt}
\raisebox{-10pt}{
\shadowbox{#1}
}
}

%--------------- Commande beamer---------------
\newcommand{\beameronly}[1]{#1} % permet de mettre des pause dans beamer pas dans poly


\setbeamertemplate{navigation symbols}{}
\setbeamertemplate{footline}  % tiré du fichier beamerouterinfolines.sty
{
  \leavevmode%
  \hbox{%
  \begin{beamercolorbox}[wd=.333333\paperwidth,ht=2.25ex,dp=1ex,center]{author in head/foot}%
    % \usebeamerfont{author in head/foot}\insertshortauthor%~~(\insertshortinstitute)
    \usebeamerfont{section in head/foot}{\bf\insertshorttitle}
  \end{beamercolorbox}%
  \begin{beamercolorbox}[wd=.333333\paperwidth,ht=2.25ex,dp=1ex,center]{title in head/foot}%
    \usebeamerfont{section in head/foot}{\bf\insertsectionhead}
  \end{beamercolorbox}%
  \begin{beamercolorbox}[wd=.333333\paperwidth,ht=2.25ex,dp=1ex,right]{date in head/foot}%
    % \usebeamerfont{date in head/foot}\insertshortdate{}\hspace*{2em}
    \insertframenumber{} / \inserttotalframenumber\hspace*{2ex} 
  \end{beamercolorbox}}%
  \vskip0pt%
}


\definecolor{mygrey}{rgb}{0.5,0.5,0.5}
\setlength{\parindent}{0cm}
%\DeclareTextFontCommand{\helvetica}{\fontfamily{phv}\selectfont}

% background beamer
\definecolor{couleurhaut}{rgb}{0.85,0.9,1}  % creme
\definecolor{couleurmilieu}{rgb}{1,1,1}  % vert pale
\definecolor{couleurbas}{rgb}{0.85,0.9,1}  % blanc
\setbeamertemplate{background canvas}[vertical shading]%
[top=couleurhaut,middle=couleurmilieu,midpoint=0.4,bottom=couleurbas] 
%[top=fondtitre!05,bottom=fondtitre!60]



\makeatletter
\setbeamertemplate{theorem begin}
{%
  \begin{\inserttheoremblockenv}
  {%
    \inserttheoremheadfont
    \inserttheoremname
    \inserttheoremnumber
    \ifx\inserttheoremaddition\@empty\else\ (\inserttheoremaddition)\fi%
    \inserttheorempunctuation
  }%
}
\setbeamertemplate{theorem end}{\end{\inserttheoremblockenv}}

\newenvironment{theoreme}[1][]{%
   \setbeamercolor{block title}{fg=structure,bg=structure!40}
   \setbeamercolor{block body}{fg=black,bg=structure!10}
   \begin{block}{{\bf Th\'eor\`eme }#1}
}{%
   \end{block}%
}


\newenvironment{proposition}[1][]{%
   \setbeamercolor{block title}{fg=structure,bg=structure!40}
   \setbeamercolor{block body}{fg=black,bg=structure!10}
   \begin{block}{{\bf Proposition }#1}
}{%
   \end{block}%
}

\newenvironment{corollaire}[1][]{%
   \setbeamercolor{block title}{fg=structure,bg=structure!40}
   \setbeamercolor{block body}{fg=black,bg=structure!10}
   \begin{block}{{\bf Corollaire }#1}
}{%
   \end{block}%
}

\newenvironment{mydefinition}[1][]{%
   \setbeamercolor{block title}{fg=structure,bg=structure!40}
   \setbeamercolor{block body}{fg=black,bg=structure!10}
   \begin{block}{{\bf Définition} #1}
}{%
   \end{block}%
}

\newenvironment{lemme}[0]{%
   \setbeamercolor{block title}{fg=structure,bg=structure!40}
   \setbeamercolor{block body}{fg=black,bg=structure!10}
   \begin{block}{\bf Lemme}
}{%
   \end{block}%
}

\newenvironment{remarque}[1][]{%
   \setbeamercolor{block title}{fg=black,bg=structure!20}
   \setbeamercolor{block body}{fg=black,bg=structure!5}
   \begin{block}{Remarque #1}
}{%
   \end{block}%
}


\newenvironment{exemple}[1][]{%
   \setbeamercolor{block title}{fg=black,bg=structure!20}
   \setbeamercolor{block body}{fg=black,bg=structure!5}
   \begin{block}{{\bf Exemple }#1}
}{%
   \end{block}%
}


\newenvironment{miniexercice}[0]{%
   \setbeamercolor{block title}{fg=structure,bg=structure!20}
   \setbeamercolor{block body}{fg=black,bg=structure!5}
   \begin{block}{Mini-exercices}
}{%
   \end{block}%
}


\newenvironment{tp}[0]{%
   \setbeamercolor{block title}{fg=structure,bg=structure!40}
   \setbeamercolor{block body}{fg=black,bg=structure!10}
   \begin{block}{\bf Travaux pratiques}
}{%
   \end{block}%
}
\newenvironment{exercicecours}[1][]{%
   \setbeamercolor{block title}{fg=structure,bg=structure!40}
   \setbeamercolor{block body}{fg=black,bg=structure!10}
   \begin{block}{{\bf Exercice }#1}
}{%
   \end{block}%
}
\newenvironment{algo}[1][]{%
   \setbeamercolor{block title}{fg=structure,bg=structure!40}
   \setbeamercolor{block body}{fg=black,bg=structure!10}
   \begin{block}{{\bf Algorithme}\hfill{\color{gray}\texttt{#1}}}
}{%
   \end{block}%
}


\setbeamertemplate{proof begin}{
   \setbeamercolor{block title}{fg=black,bg=structure!20}
   \setbeamercolor{block body}{fg=black,bg=structure!5}
   \begin{block}{{\footnotesize Démonstration}}
   \footnotesize
   \smallskip}
\setbeamertemplate{proof end}{%
   \end{block}}
\setbeamertemplate{qed symbol}{\openbox}


\makeatother
\usecolortheme[RGB={0,127,0}]{structure}


% Commande spécifique à ce chapitre

\newcommand{\Python}{\texttt{Python}}
\renewcommand{\evidence}[1]{{\color{blue}\textbf{#1}}}

\usepackage{textcomp}

\usepackage{listings}
\lstset{
  upquote=true,
  columns=flexible,
  keepspaces=true,
  basicstyle=\ttfamily,
  commentstyle=\color{gray},
  language=Python,
  showstringspaces=false,
  aboveskip=0em,  
  belowskip=0em,
  escapeinside=||
}

\lstset{
  literate={é}{{\'e}}1
           {è}{{\`e}}1
           {à}{{\`a}}1
}

\newcommand{\codeinline}[1]{\lstinline!#1!}


%%%%%%%%%%%%%%%%%%%%%%%%%%%%%%%%%%%%%%%%%%%%%%%%%%%%%%%%%%%%%
%%%%%%%%%%%%%%%%%%%%%%%%%%%%%%%%%%%%%%%%%%%%%%%%%%%%%%%%%%%%%


\begin{document}


\title{{\bf Zéros des fonctions}}
\subtitle{La dichotomie}

\begin{frame}
  
  \debutmontitre

  \pause

{\footnotesize
\hfill
\setbeamercovered{transparent=50}
\begin{minipage}{0.6\textwidth}
  \begin{itemize}
    \item<3-> Principe de la dichotomie
    \item<4-> Résultats numériques pour $\sqrt{10}$
    \item<5-> Résultats numériques pour $(1,10)^{1/12}$
    \item<6-> Calcul de l'erreur
    \item<7-> Algorithme   
  \end{itemize}
\end{minipage}
}

\end{frame}

\setcounter{framenumber}{0}




%%%%%%%%%%%%%%%%%%%%%%%%%%%%%%%%%%%%%%%%%%%%%%%%%%%%%%%%%%%%%%%%
\section{Principe de la dichotomie}

\begin{frame}

\begin{theoreme}[des valeurs intermédiaires]
Soit $f:[a,b]\to\Rr$ une fonction continue sur un segment
\mybox{Si $f(a)\cdot f(b) \le 0$, alors il existe $\ell\in[a,b]$ tel que $f(\ell)=0$}
\end{theoreme}

\pause

\myfigure{0.8}{
\tikzinput{fig_zeros01}
\quad\pause
\tikzinput{fig_zeros02}
}

\end{frame}


\begin{frame}

$f : [a,b] \to \Rr$ continue, avec $a < b$, et $f(a)\cdot f(b)\le0$

\pause

Première étape de la construction : on regarde le signe 
de la valeur de la fonction $f$ appliquée au point milieu $\frac{a+b}{2}$

\pause

\begin{itemize}
\uncover<3->{
  \item Si $f(a)\cdot f(\frac{a+b}{2})\le0$, alors il existe 
  $c \in [a,\frac{a+b}{2}]$ tel que $f(c)=0$
  }
\uncover<5->{  
  \item Si $f(a)\cdot f(\frac{a+b}{2})>0$, cela implique que 
  $f(\frac{a+b}{2})\cdot f(b)\le0$, et alors il existe $c \in [\frac{a+b}{2},b]$ tel que $f(c)=0$
  }
\end{itemize}

\myfigure{0.75}{
\uncover<4->{\tikzinput{fig_zeros03}}
\quad\pause
\uncover<6->{\tikzinput{fig_zeros04}}
}

\pause\pause\pause

Nous avons obtenu un intervalle de longueur moitié dans 
lequel l'équation $(f(x)=0)$ admet une solution


\end{frame}




\begin{frame}

\begin{itemize}
  \item {\bf Au rang 0} On pose $a_0=a$, $b_0=b$. Il existe une solution $x_0$ de l'équation $(f(x)=0)$ 
  dans l'intervalle $[a_0,b_0]$

\pause
\medskip

  \item {\bf Au rang 1}

  \begin{itemize}
  \item  Si $f(a_0)\cdot f(\frac{a_0+b_0}{2}) \le 0$, alors on pose $a_1=a_0$ et 
  $b_1=\frac{a_0+b_0}{2}$,
    \item sinon on pose $a_1=\frac{a_0+b_0}{2}$ et $b_1=b$
    \item il existe une solution $x_1$ de $(f(x)=0)$ dans $[a_1,b_1]$ 
  \end{itemize}
  
\pause  
\medskip
  
  \item $ \cdots$
  
\pause  
\medskip

  \item {\bf Au rang $n$}
  \begin{itemize}
    \item Si $f(a_n)\cdot f(\frac{a_n+b_n}{2}) \le 0$, alors on
   pose $a_{n+1}=a_n$ et $b_{n+1}=\frac{a_n+b_n}{2}$
    \item sinon on pose $a_{n+1}=\frac{a_n+b_n}{2}$ et $b_{n+1}=b_n$ 
    \item il existe une solution $x_{n+1}$ de $(f(x)=0)$ dans $[a_{n+1},b_{n+1}]$
  \end{itemize} 
\end{itemize}   
\end{frame}


\begin{frame}

\begin{itemize}[<+->]
  \setlength{\itemsep}{5pt} 
  \item \`A chaque étape on a $a_n \le x_n \le b_n$

  \item $(a_n)$ est une suite croissante, $(b_n)$ une suite décroissante, 
  $(b_n-a_n) \to 0$ lorsque $n\to +\infty$ : les suites $(a_n)$ et $(b_n)$ sont adjacentes
  
  \item $(a_n), (b_n), (x_n)$ tendent vers $\ell$
  
  \item $f$ continue : $f(\ell)=\lim_{n\to +\infty} f(x_n)=\lim_{n\to +\infty} 0 =0$
  
  \item Donc $(a_n)$\! et \!$(b_n)$ tendent vers $\ell$, qui est une solution de $(f(x)=0)$  
  
  \item On arrête le processus dès que $b_n-a_n=\frac{b-a}{2^n}$ est inférieur à la précision souhaitée
  
\end{itemize}

\end{frame}





%%%%%%%%%%%%%%%%%%%%%%%%%%%%%%%%%%%%%%%%%%%%%%%%%%%%%%%%%%%%%%%%
\section{Résultats numériques pour $\sqrt{10}$}

\begin{frame}
\begin{itemize}[<+->]
  \setlength{\itemsep}{7pt}
  \item Approximation de $\sqrt{10}$
  
  \item $f$ définie par $f(x)=x^2 - 10$, $f$ continue sur $\Rr$, 
  unique solution positive $(f(x)=0)$ est $\sqrt{10}$
  
  \item Intervalle $[3,4]$ : $f(3) \le 0$ et $f(4) \ge 0$, donc $\sqrt{10} \in [3,4]$
\end{itemize}
\end{frame}



\begin{frame}


\myfigure{1.6}{\hspace*{-2cm}
\tikzinput{fig_zeros05}
}

  
\end{frame}

\begin{frame}
\begin{enumerate} \setlength{\itemsep}{7pt}
  \item 
  \begin{itemize}\setlength{\itemsep}{4pt}
    \item $a_0=3$ et $b_0=4$ ; $f(a_0)\le 0$ et $f(b_0) \ge 0$
    \item $f(\frac{a_0+b_0}{2})=f(3,5)=3,5^2-10 = 2,25 \ge0$
    \item Donc $\sqrt{10}$ est dans l'intervalle $[3 ; 3,5]$
    \item $a_1 = a_0 = 3$ et $b_1 = \frac{a_0+b_0}{2} = 3,5$    
  \end{itemize}

  \pause
  
  \item
  \begin{itemize}\setlength{\itemsep}{4pt}
    \item $f(a_1) \le 0$ et $f(b_1) \ge0$
    \item $f(\frac{a_1+b_1}{2})=f(3,25)=0,5625 \ge 0$
    \item $a_2=3$ et $b_2=3,25$   
  \end{itemize}  

  \pause
  
  \item
  \begin{itemize}\setlength{\itemsep}{4pt}
    \item $f(\frac{a_2+b_2}{2})=f(3,125)=-0,23\ldots \le 0$
    \item $f(b_2) \ge 0$, $f$ s'annule sur $[\frac{a_2+b_2}{2},b_2]$
    \item $a_3=\frac{a_2+b_2}{2}=3,125$ et $b_3=b_2=3,25$ 
    \item $3,125 \le \sqrt{10} \le 3,25$
  \end{itemize}  
\end{enumerate}

\end{frame}



\begin{frame}
$$
\begin{array}{ll}
  a_0 = 3      & b_0 = 4 \\
  a_1 = 3      & b_1 = 3,5 \\
  a_2 = 3      & b_2 = 3,25 \\
  a_3 = 3,125  & b_3 = 3,25 \\
  a_4 = 3,125  & b_4 = 3,1875 \\
  a_5 = 3,15625 & b_5 = 3,1875 \\
  a_6 = 3,15625 & b_6 = 3,171875 \\
  a_7 = 3,15625 & b_7 = 3,164062\ldots \\
  a_8 = 3,16015\ldots & b_8 = 3,164062\ldots \\
\end{array}
$$


$$3,160  \le \sqrt{10} \le 3,165$$

\end{frame}



%%%%%%%%%%%%%%%%%%%%%%%%%%%%%%%%%%%%%%%%%%%%%%%%%%%%%%%%%%%%%%%%
\section{Résultats numériques pour $(1,10)^{1/12}$}

\begin{frame}
\begin{itemize}
  \item Approximation de $(1,10)^{1/12}$
  
  \item $f(x) = x^{12} - 1,10$
  
  \item $a_0=1$ et $b_0=1,1$
  
  \item $f(a_0)=-0,10 \le 0$ et $f(b_0)=2,038\ldots \ge 0$
 
\end{itemize}

$$
\begin{array}{ll}
  a_0 = 1             & b_0 = 1,10 \\
  a_1 = 1             & b_1 = 1,05 \\
  a_2 = 1             & b_2 = 1,025 \\
  a_3 = 1             & b_3 = 1,0125 \\
  a_4 = 1,00625       & b_4 = 1,0125 \\
  a_5 = 1,00625       & b_5 = 1,00937\ldots\\
  a_6 = 1,00781\ldots & b_6 = 1,00937\ldots \\
  a_7 = 1,00781\ldots & b_7 = 1,00859\ldots \\
  a_8 = 1,00781\ldots & b_8 = 1,00820\ldots \\
\end{array}
$$

$$ 1,00781 \le (1,10)^{1/12} \le 1,00821$$

\end{frame}



%%%%%%%%%%%%%%%%%%%%%%%%%%%%%%%%%%%%%%%%%%%%%%%%%%%%%%%%%%%%%%%%
\section{Calcul de l'erreur}

\begin{frame}

\begin{itemize}\setlength{\itemsep}{6pt}
  \item $\ell \in [a_n,b_n]$ de longueur $\frac{b-a}{2^n}$
  
\pause

  \item $$\begin{array}{rcl}
\frac{b-a}{2^n} \le 10^{-N} 
 & \iff & (b-a)10^N \le 2^n  \\
 & \iff & \log(b-a) + \log(10^N)  \le \log(2^n) \\
 & \iff & \log(b-a) + N \le n \log 2 \\
 & \iff & n \ge \frac{N +  \log(b-a)}{\log 2} \\            
          \end{array}
$$


\pause

  \item Exemple avec $b-a \le 1$

\begin{center}
\begin{tabular}{ll}
  $10^{-10}$ ($\sim 10$ décimales) &  $34$ itérations \\
  $10^{-100}$ ($\sim 100$ décimales) &  $333$ itérations \\ 
  $10^{-1000}$ ($\sim 1000$ décimales) &  $3322$ itérations \\ 
\end{tabular}  
\end{center}
  
\end{itemize}

\end{frame}



%%%%%%%%%%%%%%%%%%%%%%%%%%%%%%%%%%%%%%%%%%%%%%%%%%%%%%%%%%%%%%%%
\section{Algorithme}

\begin{frame}[fragile]

\begin{algo}[dichotomie.py (1)]
\begin{lstlisting}
def f(x):
    return x*x - 10 
\end{lstlisting}  
\end{algo}

\begin{algo}[dichotomie.py (2)]
\begin{lstlisting}
def dicho(a,b,n):    
    for i in range(n):
        c = (a+b)/2
        if f(a)*f(c) <= 0:
            b = c
        else:
            a = c
    return a,b
\end{lstlisting}  
\end{algo}

\end{frame}




%%%%%%%%%%%%%%%%%%%%%%%%%%%%%%%%%%%%%%%%%%%%%%%%%%%%%%%%%%%%%%%%
\section{Mini-exercices}

\begin{frame}

\begin{miniexercice}
\begin{enumerate}
  \item \`A la main, calculer un encadrement à $0,1$ près de $\sqrt{3}$.
  Idem avec $\sqrt[3]{2}$. 
  
  \item Calculer une approximation des solutions de l'équation $x^3+1=3x$.
  
  \item Est-il plus efficace de diviser l'intervalle en $4$ au lieu d'en $2$ ?
  (\`A chaque itération, la dichotomie classique nécessite l'évaluation de $f$ en une nouvelle valeur 
  $\frac{a+b}{2}$ pour une précision améliorée d'un facteur $2$.) 
  
  \item \'Ecrire un algorithme pour calculer plusieurs solutions de $(f(x)=0)$.
  
  \item On se donne un tableau trié de taille $N$, rempli de nombres appartenant à $\{1,\ldots,n\}$.
  \'Ecrire un algorithme qui teste si une valeur $k$ apparaît dans le tableau et en quelle position.

\end{enumerate}
\end{miniexercice}

\end{frame}

\end{document}