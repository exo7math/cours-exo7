
%%%%%%%%%%%%%%%%%% PREAMBULE %%%%%%%%%%%%%%%%%%


\documentclass[12pt]{article}

\usepackage{amsfonts,amsmath,amssymb,amsthm}
\usepackage[utf8]{inputenc}
\usepackage[T1]{fontenc}
\usepackage[francais]{babel}


% packages
\usepackage{amsfonts,amsmath,amssymb,amsthm}
\usepackage[utf8]{inputenc}
\usepackage[T1]{fontenc}
%\usepackage{lmodern}

\usepackage[francais]{babel}
\usepackage{fancybox}
\usepackage{graphicx}

\usepackage{float}

%\usepackage[usenames, x11names]{xcolor}
\usepackage{tikz}
\usepackage{datetime}

\usepackage{mathptmx}
%\usepackage{fouriernc}
%\usepackage{newcent}
\usepackage[mathcal,mathbf]{euler}

%\usepackage{palatino}
%\usepackage{newcent}


% Commande spéciale prompteur

%\usepackage{mathptmx}
%\usepackage[mathcal,mathbf]{euler}
%\usepackage{mathpple,multido}

\usepackage[a4paper]{geometry}
\geometry{top=2cm, bottom=2cm, left=1cm, right=1cm, marginparsep=1cm}

\newcommand{\change}{{\color{red}\rule{\textwidth}{1mm}\\}}

\newcounter{mydiapo}

\newcommand{\diapo}{\newpage
\hfill {\normalsize  Diapo \themydiapo \quad \texttt{[\jobname]}} \\
\stepcounter{mydiapo}}


%%%%%%% COULEURS %%%%%%%%%%

% Pour blanc sur noir :
%\pagecolor[rgb]{0.5,0.5,0.5}
% \pagecolor[rgb]{0,0,0}
% \color[rgb]{1,1,1}



%\DeclareFixedFont{\myfont}{U}{cmss}{bx}{n}{18pt}
\newcommand{\debuttexte}{
%%%%%%%%%%%%% FONTES %%%%%%%%%%%%%
\renewcommand{\baselinestretch}{1.5}
\usefont{U}{cmss}{bx}{n}
\bfseries

% Taille normale : commenter le reste !
%Taille Arnaud
%\fontsize{19}{19}\selectfont

% Taille Barbara
%\fontsize{21}{22}\selectfont

%Taille François
\fontsize{25}{30}\selectfont

%Taille Pascal
%\fontsize{25}{30}\selectfont

%Taille Laura
%\fontsize{30}{35}\selectfont


%\myfont
%\usefont{U}{cmss}{bx}{n}

%\Huge
%\addtolength{\parskip}{\baselineskip}
}


% \usepackage{hyperref}
% \hypersetup{colorlinks=true, linkcolor=blue, urlcolor=blue,
% pdftitle={Exo7 - Exercices de mathématiques}, pdfauthor={Exo7}}


%section
% \usepackage{sectsty}
% \allsectionsfont{\bf}
%\sectionfont{\color{Tomato3}\upshape\selectfont}
%\subsectionfont{\color{Tomato4}\upshape\selectfont}

%----- Ensembles : entiers, reels, complexes -----
\newcommand{\Nn}{\mathbb{N}} \newcommand{\N}{\mathbb{N}}
\newcommand{\Zz}{\mathbb{Z}} \newcommand{\Z}{\mathbb{Z}}
\newcommand{\Qq}{\mathbb{Q}} \newcommand{\Q}{\mathbb{Q}}
\newcommand{\Rr}{\mathbb{R}} \newcommand{\R}{\mathbb{R}}
\newcommand{\Cc}{\mathbb{C}} 
\newcommand{\Kk}{\mathbb{K}} \newcommand{\K}{\mathbb{K}}

%----- Modifications de symboles -----
\renewcommand{\epsilon}{\varepsilon}
\renewcommand{\Re}{\mathop{\text{Re}}\nolimits}
\renewcommand{\Im}{\mathop{\text{Im}}\nolimits}
%\newcommand{\llbracket}{\left[\kern-0.15em\left[}
%\newcommand{\rrbracket}{\right]\kern-0.15em\right]}

\renewcommand{\ge}{\geqslant}
\renewcommand{\geq}{\geqslant}
\renewcommand{\le}{\leqslant}
\renewcommand{\leq}{\leqslant}

%----- Fonctions usuelles -----
\newcommand{\ch}{\mathop{\mathrm{ch}}\nolimits}
\newcommand{\sh}{\mathop{\mathrm{sh}}\nolimits}
\renewcommand{\tanh}{\mathop{\mathrm{th}}\nolimits}
\newcommand{\cotan}{\mathop{\mathrm{cotan}}\nolimits}
\newcommand{\Arcsin}{\mathop{\mathrm{Arcsin}}\nolimits}
\newcommand{\Arccos}{\mathop{\mathrm{Arccos}}\nolimits}
\newcommand{\Arctan}{\mathop{\mathrm{Arctan}}\nolimits}
\newcommand{\Argsh}{\mathop{\mathrm{Argsh}}\nolimits}
\newcommand{\Argch}{\mathop{\mathrm{Argch}}\nolimits}
\newcommand{\Argth}{\mathop{\mathrm{Argth}}\nolimits}
\newcommand{\pgcd}{\mathop{\mathrm{pgcd}}\nolimits} 

\newcommand{\Card}{\mathop{\text{Card}}\nolimits}
\newcommand{\Ker}{\mathop{\text{Ker}}\nolimits}
\newcommand{\id}{\mathop{\text{id}}\nolimits}
\newcommand{\ii}{\mathrm{i}}
\newcommand{\dd}{\mathrm{d}}
\newcommand{\Vect}{\mathop{\text{Vect}}\nolimits}
\newcommand{\Mat}{\mathop{\mathrm{Mat}}\nolimits}
\newcommand{\rg}{\mathop{\text{rg}}\nolimits}
\newcommand{\tr}{\mathop{\text{tr}}\nolimits}
\newcommand{\ppcm}{\mathop{\text{ppcm}}\nolimits}

%----- Structure des exercices ------

\newtheoremstyle{styleexo}% name
{2ex}% Space above
{3ex}% Space below
{}% Body font
{}% Indent amount 1
{\bfseries} % Theorem head font
{}% Punctuation after theorem head
{\newline}% Space after theorem head 2
{}% Theorem head spec (can be left empty, meaning ‘normal’)

%\theoremstyle{styleexo}
\newtheorem{exo}{Exercice}
\newtheorem{ind}{Indications}
\newtheorem{cor}{Correction}


\newcommand{\exercice}[1]{} \newcommand{\finexercice}{}
%\newcommand{\exercice}[1]{{\tiny\texttt{#1}}\vspace{-2ex}} % pour afficher le numero absolu, l'auteur...
\newcommand{\enonce}{\begin{exo}} \newcommand{\finenonce}{\end{exo}}
\newcommand{\indication}{\begin{ind}} \newcommand{\finindication}{\end{ind}}
\newcommand{\correction}{\begin{cor}} \newcommand{\fincorrection}{\end{cor}}

\newcommand{\noindication}{\stepcounter{ind}}
\newcommand{\nocorrection}{\stepcounter{cor}}

\newcommand{\fiche}[1]{} \newcommand{\finfiche}{}
\newcommand{\titre}[1]{\centerline{\large \bf #1}}
\newcommand{\addcommand}[1]{}
\newcommand{\video}[1]{}

% Marge
\newcommand{\mymargin}[1]{\marginpar{{\small #1}}}



%----- Presentation ------
\setlength{\parindent}{0cm}

%\newcommand{\ExoSept}{\href{http://exo7.emath.fr}{\textbf{\textsf{Exo7}}}}

\definecolor{myred}{rgb}{0.93,0.26,0}
\definecolor{myorange}{rgb}{0.97,0.58,0}
\definecolor{myyellow}{rgb}{1,0.86,0}

\newcommand{\LogoExoSept}[1]{  % input : echelle
{\usefont{U}{cmss}{bx}{n}
\begin{tikzpicture}[scale=0.1*#1,transform shape]
  \fill[color=myorange] (0,0)--(4,0)--(4,-4)--(0,-4)--cycle;
  \fill[color=myred] (0,0)--(0,3)--(-3,3)--(-3,0)--cycle;
  \fill[color=myyellow] (4,0)--(7,4)--(3,7)--(0,3)--cycle;
  \node[scale=5] at (3.5,3.5) {Exo7};
\end{tikzpicture}}
}



\theoremstyle{definition}
%\newtheorem{proposition}{Proposition}
%\newtheorem{exemple}{Exemple}
%\newtheorem{theoreme}{Théorème}
\newtheorem{lemme}{Lemme}
\newtheorem{corollaire}{Corollaire}
%\newtheorem*{remarque*}{Remarque}
%\newtheorem*{miniexercice}{Mini-exercices}
%\newtheorem{definition}{Définition}




%definition d'un terme
\newcommand{\defi}[1]{{\color{myorange}\textbf{\emph{#1}}}}
\newcommand{\evidence}[1]{{\color{blue}\textbf{\emph{#1}}}}



 %----- Commandes divers ------

\newcommand{\codeinline}[1]{\texttt{#1}}

%%%%%%%%%%%%%%%%%%%%%%%%%%%%%%%%%%%%%%%%%%%%%%%%%%%%%%%%%%%%%
%%%%%%%%%%%%%%%%%%%%%%%%%%%%%%%%%%%%%%%%%%%%%%%%%%%%%%%%%%%%%



\begin{document}

\debuttexte


%%%%%%%%%%%%%%%%%%%%%%%%%%%%%%%%%%%%%%%%%%%%%%%%%%%%%%%%%%%
\diapo

\change

\change

La méthode de Newton consiste à remplacer 
la droite de la méthode de la sécante par la tangente.
Elle est d'une redoutable efficacité.

\change

Nous l'appliquerons au calcul d'une approximation de $\sqrt{10}$

\change

et de $(1,10)^{1/12}$

\change

Nous énoncerons les résultats de majoration de l'erreur pour $\sqrt{10}$.

\change

Et on termine par un algorithme et les 1000 premières décimales de $\sqrt{10}$.


%%%%%%%%%%%%%%%%%%%%%%%%%%%%%%%%%%%%%%%%%%%%%%%%%%%%%%%%%%
\diapo



Partons d'une fonction dérivable $f:[a,b] \to \Rr$ et d'un point $u_0 \in[a,b]$.

\change
Voici un exemple de fonction 

\change
et une valeur $u_0$.

On souhaite bien sûr trouver une approximation de ce zéro.

\change
On considère le point du graphe dont l'abscisse est $u_0$




\change
et on considère la tangente au graphe de $f$ en ce point.


\change
Voici la tangente.

\change
Cette tangente recoupe l'axe des abscisses en un point que l'on appelle $(u_1,0)$.

\change

$u_1$ est donc cette valeur.

\change

Si $u_1$ est dans l'intervalle de définition $[a,b]$ de notre fonction
alors on recommence l'opération avec cette fois la tangente au point d'abscisse $u_1$.

\change

On prend le point d'abscisse $u_1$,

\change

on trace la tangente au graphe en ce point,

cette tangente recoupe l'axe des abscisses pour donner la valeur $u_2$.
etc.

\change

Ce processus conduit à la définition d'une suite récurrente :
$u_0$ est le terme initial

et pour $n \ge 0$, $u_{n+1} = u_n - \frac{f(u_n)}{f'(u_n)}.$

%%%%%%%%%%%%%%%%%%%%%%%%%%%%%%%%%%%%%%%%%%%%%%%%%%%%%%%%%%%
\diapo

Justifions la formule de récurrence $u_{n+1} = u_n - \frac{f(u_n)}{f'(u_n)}.$

\change

Comment s'effectue une étape de la construction.

On part d'une abscisse $u_n$, qui correspond à un point du graphe d'ordonnée $f(u_n)$.

\change

On trace la tangente au graphe en ce point.

\change

Cette tangente recoupe l'axe des abscisses en un point dont on appelle l'abscisse $u_{n+1}$.

\change

Passons aux calculs.

\change

La tangente au point d'abscisse $u_n$ a pour équation :
$y = f'(u_n)(x-u_n)+f(u_n)$. 

\change

Donc cette tangente recoupe l'axe des abscisses en un point $(x,0)$ 
qui vérifie $0=f'(u_n)(x-u_n)+f(u_n)$. 

\change

Donc $x=u_n - \frac{f(u_n)}{f'(u_n)}.$

Et c'est ce $x$ que l'on appelle $u_{n+1}$.

%%%%%%%%%%%%%%%%%%%%%%%%%%%%%%%%%%%%%%%%%%%%%%%%%%%%%%%%%%%
\diapo

Pour calculer $\sqrt{a}$, on pose $f(x)=x^2-a$, avec $f'(x)=2x$. 

La suite issue de la méthode de Newton est déterminée par $u_0>0$ 
et la relation de récurrence $u_{n+1} = u_n - \frac{u_n^2-a}{2u_n}$.

\change


Suite qui pour cet exemple s'appelle \defi{suite de Héron} et que l'on récrit souvent
$u_{n+1} = \frac12 \left(u_n+\frac{a}{u_n}\right).$


\change

Voici le principal résultat théorique de cette leçon. :

Proposition : 
Cette suite de Héron $(u_n)$ converge vers $\sqrt{a}$. 

\change


Pour le calcul de $\sqrt{10}$, on pose par exemple $u_0=4$, et on peut même commencer les calculs à la main :

\change

$u_1 = \frac12 \left(u_0+\frac{10}{u_0}\right) = \frac12\left(4+\frac{10}{4}\right) = \frac{13}{4} = 3,25 $

\change

$u_2 = \frac12 \left(u_1+\frac{10}{u_1}\right) = \frac12\left(\tfrac{13}{4}+\frac{10}{\tfrac{13}{4}}\right) 
  = \frac{329}{104} = 3,1634\ldots$
 
\change 

$u_3 = \frac12 \left(u_2+\frac{10}{u_2}\right) = \frac{216\,401}{68\,432} = 3,16227788 \ldots$

\change

$u_4 = 3,1622776601683\ldots  $


Pour $u_4$ les $13$ décimales affichées après la virgule  sont déjà exactes !


%%%%%%%%%%%%%%%%%%%%%%%%%%%%%%%%%%%%%%%%%%%%%%%%%%%%%%%%%%%
\diapo


Voici la preuve de la convergence de la suite de Héron $(u_n)$ vers $\sqrt{a}$.

\change

Première étape, montrons que $u_n \ge \sqrt{a}$ ceci à partir des rangs $n\ge1$.
  
  Tout d'abord 
  $u_{n+1}^2-a = \frac14 \left(\frac{u_n^2 + a}{u_n}\right)^2 - a $
  
  on développe, on simplifie et on trouve que 
  $u_{n+1}^2-a=\frac14 \frac{(u_n^2-a)^2}{u_n^2}$.
  
  Donc $u_{n+1}^2 - a \ge 0$. Comme il est clair que la suite est positive 
  on en déduit que $u_{n+1} \ge \sqrt{a}$. 
  %(Notez que $u_0$ lui est quelconque.)
  

\change  

Montrons maintenant que $(u_n)_{n\ge1}$ est une suite décroissante qui converge.

  Pour  montrer que $(u_n)$ est décroissante on calcule le rapport 
$\frac{u_{n+1}}{u_n}$ qui par la formule de récurrence vaut :
$\frac12 \left(1+\frac{a}{u_n^2}\right)$

mais on vient de voir que pour $n\ge 1$,  $u_n^2 \ge a$,
donc $\frac{a}{u_n^2}\le 1$, et alors $\frac{u_{n+1}}{u_n} \le 1$, pour tout $n\ge 1$.
  
  Conséquence : la suite $(u_n)_{n\ge1}$ est une suite décroissante.
  
  Comme la suite est une suite positive alors elle est bien sûr minorée par $0$,
  
  la suite est décroissante et minorée donc elle converge.
  
  \change
  
Il nous reste à montrer que la limite est bien  $\sqrt{a}$.


  
  Notons pour l'instant $\ell$ la limite de $(u_n)$. 
  
  D'une part $u_n \to \ell$, mais aussi $u_{n+1} \to \ell$.
  
  
  Lorsque l'on fait tendre $n\to +\infty$ dans la relation $u_{n+1} = \frac12 \left(u_n+\frac{a}{u_n}\right)$, 
  alors par continuité on obtient à la limite 
  $\ell = \frac12 \left(\ell+\frac{a}{\ell}\right)$. 
  
  Ce qui conduit à la relation $\ell^2=a$ et par positivité de la suite, $\ell = \sqrt{a}$.  


%%%%%%%%%%%%%%%%%%%%%%%%%%%%%%%%%%%%%%%%%%%%%%%%%%%%%%%%%%%
\diapo

Pour calculer $(1,10)^{1/12}$, 

\change

on pose $f(x)=x^{12}-a$ avec $a=1,10$. 


\change

La dérivée est $f'(x)=12x^{11}$. 

\change

Et la formule de récurrence est  $u_{n+1} = u_n - \frac{u_n^{12}-a}{12u_n^{11}}$. 
Ce que l'on reformule ainsi :
$ u_{n+1} = \frac1{12} \left(11u_n+\frac{a}{u_n^{11}}\right)$.

\change



Voici les résultats numériques pour $(1,10)^{1/12}$ en partant de $u_0=1$.


Toutes les décimales affichées pour $u_4$ sont exactes

%%%%%%%%%%%%%%%%%%%%%%%%%%%%%%%%%%%%%%%%%%%%%%%%%%%%%%%%%%%
\diapo

Voici une proposition un peu technique pour le calcul de l'erreur, 
je vous encourage à travailler la démonstration par vous même.

On note $k$ un majorant de la différence : $u_1-\sqrt a\le k$

Alors l'erreur $u_n - \sqrt{a}$ commise en approchant
$\sqrt{a}$ par la suite $(u_n)$ est 

  $\le 2\sqrt{a} \left( \frac{k}{2\sqrt{a}} \right)^{2^{n-1}}$
  
  \change
  
  Pour notre exemple d'approximation de $\sqrt{10}$,
  on part de $u_0=4$.
  
  On calcule $u_1=3,25$, et $u_1-\sqrt{10} \le \frac14$, on pose donc $k=\frac14$.
  
  Alors l'erreur au rang $n$ vérifie 
  $u_n - \sqrt{10} \le 8 \left(\frac{1}{24} \right)^{2^{n-1}}$
  
  Notez l'exposant $2^{n-1}$ qui fait que l'erreur tend doublement exponentiellement vers $0$.
  
\change

Admirez la puissance de la méthode de Newton :

$4$ itérations, donnent $10$ décimales exactes après la virgule;

$8$ itérations, donnent déjà $100$ décimales

et $11$ itérations donnent $1000$ décimales !

Cette rapidité de convergence se justifie grâce
au calcul de l'erreur : *l'exposant* dans la majoration de l'erreur est 
multiplié par $2$ à chaque étape, 

donc à chaque itération le nombre de décimales exactes double !




%%%%%%%%%%%%%%%%%%%%%%%%%%%%%%%%%%%%%%%%%%%%%%%%%%%%%%%%%%%
\diapo

Voici l'algorithme pour le calcul de $\sqrt{a}$. 
On précise en entrée le réel $a\ge0$ dont on veut calculer 
la racine et le nombre $n$ d'itérations.

C'est simplement l'application de la formule :
$u_{n+1} = \frac12 \left(u_n+\frac{a}{u_n}\right).$
,

%%%%%%%%%%%%%%%%%%%%%%%%%%%%%%%%%%%%%%%%%%%%%%%%%%%%%%%%%%%
\diapo

En utilisant le module \texttt{decimal}, et avec seulement $11$
itérations, voici les $1000$ premières décimales exactes de $\sqrt{10}$ :


%%%%%%%%%%%%%%%%%%%%%%%%%%%%%%%%%%%%%%%%%%%%%%%%%%%%%%%%%%%
\diapo

Voici quelques exercices, pour assimiler la méthode de Newton.


\end{document}
