
%%%%%%%%%%%%%%%%%% PREAMBULE %%%%%%%%%%%%%%%%%%

\documentclass[aspectratio=169,utf8]{beamer}
%\documentclass[aspectratio=169,handout]{beamer}

\usetheme{Boadilla}
%\usecolortheme{seahorse}
\usecolortheme[RGB={245,66,24}]{structure}
\useoutertheme{infolines}

% packages
\usepackage{amsfonts,amsmath,amssymb,amsthm}
\usepackage[utf8]{inputenc}
\usepackage[T1]{fontenc}
\usepackage{lmodern}

\usepackage[francais]{babel}
\usepackage{fancybox}
\usepackage{graphicx}

\usepackage{float}
\usepackage{xfrac}

%\usepackage[usenames, x11names]{xcolor}
\usepackage{tikz}
\usepackage{pgfplots}
\usepackage{datetime}



%-----  Package unités -----
\usepackage{siunitx}
\sisetup{locale = FR,detect-all,per-mode = symbol}

%\usepackage{mathptmx}
%\usepackage{fouriernc}
%\usepackage{newcent}
%\usepackage[mathcal,mathbf]{euler}

%\usepackage{palatino}
%\usepackage{newcent}
% \usepackage[mathcal,mathbf]{euler}



% \usepackage{hyperref}
% \hypersetup{colorlinks=true, linkcolor=blue, urlcolor=blue,
% pdftitle={Exo7 - Exercices de mathématiques}, pdfauthor={Exo7}}


%section
% \usepackage{sectsty}
% \allsectionsfont{\bf}
%\sectionfont{\color{Tomato3}\upshape\selectfont}
%\subsectionfont{\color{Tomato4}\upshape\selectfont}

%----- Ensembles : entiers, reels, complexes -----
\newcommand{\Nn}{\mathbb{N}} \newcommand{\N}{\mathbb{N}}
\newcommand{\Zz}{\mathbb{Z}} \newcommand{\Z}{\mathbb{Z}}
\newcommand{\Qq}{\mathbb{Q}} \newcommand{\Q}{\mathbb{Q}}
\newcommand{\Rr}{\mathbb{R}} \newcommand{\R}{\mathbb{R}}
\newcommand{\Cc}{\mathbb{C}} 
\newcommand{\Kk}{\mathbb{K}} \newcommand{\K}{\mathbb{K}}

%----- Modifications de symboles -----
\renewcommand{\epsilon}{\varepsilon}
\renewcommand{\Re}{\mathop{\text{Re}}\nolimits}
\renewcommand{\Im}{\mathop{\text{Im}}\nolimits}
%\newcommand{\llbracket}{\left[\kern-0.15em\left[}
%\newcommand{\rrbracket}{\right]\kern-0.15em\right]}

\renewcommand{\ge}{\geqslant}
\renewcommand{\geq}{\geqslant}
\renewcommand{\le}{\leqslant}
\renewcommand{\leq}{\leqslant}
\renewcommand{\epsilon}{\varepsilon}

%----- Fonctions usuelles -----
\newcommand{\ch}{\mathop{\text{ch}}\nolimits}
\newcommand{\sh}{\mathop{\text{sh}}\nolimits}
\renewcommand{\tanh}{\mathop{\text{th}}\nolimits}
\newcommand{\cotan}{\mathop{\text{cotan}}\nolimits}
\newcommand{\Arcsin}{\mathop{\text{arcsin}}\nolimits}
\newcommand{\Arccos}{\mathop{\text{arccos}}\nolimits}
\newcommand{\Arctan}{\mathop{\text{arctan}}\nolimits}
\newcommand{\Argsh}{\mathop{\text{argsh}}\nolimits}
\newcommand{\Argch}{\mathop{\text{argch}}\nolimits}
\newcommand{\Argth}{\mathop{\text{argth}}\nolimits}
\newcommand{\pgcd}{\mathop{\text{pgcd}}\nolimits} 


%----- Commandes divers ------
\newcommand{\ii}{\mathrm{i}}
\newcommand{\dd}{\text{d}}
\newcommand{\id}{\mathop{\text{id}}\nolimits}
\newcommand{\Ker}{\mathop{\text{Ker}}\nolimits}
\newcommand{\Card}{\mathop{\text{Card}}\nolimits}
\newcommand{\Vect}{\mathop{\text{Vect}}\nolimits}
\newcommand{\Mat}{\mathop{\text{Mat}}\nolimits}
\newcommand{\rg}{\mathop{\text{rg}}\nolimits}
\newcommand{\tr}{\mathop{\text{tr}}\nolimits}


%----- Structure des exercices ------

\newtheoremstyle{styleexo}% name
{2ex}% Space above
{3ex}% Space below
{}% Body font
{}% Indent amount 1
{\bfseries} % Theorem head font
{}% Punctuation after theorem head
{\newline}% Space after theorem head 2
{}% Theorem head spec (can be left empty, meaning ‘normal’)

%\theoremstyle{styleexo}
\newtheorem{exo}{Exercice}
\newtheorem{ind}{Indications}
\newtheorem{cor}{Correction}


\newcommand{\exercice}[1]{} \newcommand{\finexercice}{}
%\newcommand{\exercice}[1]{{\tiny\texttt{#1}}\vspace{-2ex}} % pour afficher le numero absolu, l'auteur...
\newcommand{\enonce}{\begin{exo}} \newcommand{\finenonce}{\end{exo}}
\newcommand{\indication}{\begin{ind}} \newcommand{\finindication}{\end{ind}}
\newcommand{\correction}{\begin{cor}} \newcommand{\fincorrection}{\end{cor}}

\newcommand{\noindication}{\stepcounter{ind}}
\newcommand{\nocorrection}{\stepcounter{cor}}

\newcommand{\fiche}[1]{} \newcommand{\finfiche}{}
\newcommand{\titre}[1]{\centerline{\large \bf #1}}
\newcommand{\addcommand}[1]{}
\newcommand{\video}[1]{}

% Marge
\newcommand{\mymargin}[1]{\marginpar{{\small #1}}}

\def\noqed{\renewcommand{\qedsymbol}{}}


%----- Presentation ------
\setlength{\parindent}{0cm}

%\newcommand{\ExoSept}{\href{http://exo7.emath.fr}{\textbf{\textsf{Exo7}}}}

\definecolor{myred}{rgb}{0.93,0.26,0}
\definecolor{myorange}{rgb}{0.97,0.58,0}
\definecolor{myyellow}{rgb}{1,0.86,0}

\newcommand{\LogoExoSept}[1]{  % input : echelle
{\usefont{U}{cmss}{bx}{n}
\begin{tikzpicture}[scale=0.1*#1,transform shape]
  \fill[color=myorange] (0,0)--(4,0)--(4,-4)--(0,-4)--cycle;
  \fill[color=myred] (0,0)--(0,3)--(-3,3)--(-3,0)--cycle;
  \fill[color=myyellow] (4,0)--(7,4)--(3,7)--(0,3)--cycle;
  \node[scale=5] at (3.5,3.5) {Exo7};
\end{tikzpicture}}
}


\newcommand{\debutmontitre}{
  \author{} \date{} 
  \thispagestyle{empty}
  \hspace*{-10ex}
  \begin{minipage}{\textwidth}
    \titlepage  
  \vspace*{-2.5cm}
  \begin{center}
    \LogoExoSept{2.5}
  \end{center}
  \end{minipage}

  \vspace*{-0cm}
  
  % Astuce pour que le background ne soit pas discrétisé lors de la conversion pdf -> png
\begin{tikzpicture}
        \fill[opacity=0,green!60!black] (0,0)--++(0,0)--++(0,0)--++(0,0)--cycle; 
\end{tikzpicture}

% toc S'affiche trop tot :
% \tableofcontents[hideallsubsections, pausesections]
}

\newcommand{\finmontitre}{
  \end{frame}
  \setcounter{framenumber}{0}
} % ne marche pas pour une raison obscure

%----- Commandes supplementaires ------

% \usepackage[landscape]{geometry}
% \geometry{top=1cm, bottom=3cm, left=2cm, right=10cm, marginparsep=1cm
% }
% \usepackage[a4paper]{geometry}
% \geometry{top=2cm, bottom=2cm, left=2cm, right=2cm, marginparsep=1cm
% }

%\usepackage{standalone}


% New command Arnaud -- november 2011
\setbeamersize{text margin left=24ex}
% si vous modifier cette valeur il faut aussi
% modifier le decalage du titre pour compenser
% (ex : ici =+10ex, titre =-5ex

\theoremstyle{definition}
%\newtheorem{proposition}{Proposition}
%\newtheorem{exemple}{Exemple}
%\newtheorem{theoreme}{Théorème}
%\newtheorem{lemme}{Lemme}
%\newtheorem{corollaire}{Corollaire}
%\newtheorem*{remarque*}{Remarque}
%\newtheorem*{miniexercice}{Mini-exercices}
%\newtheorem{definition}{Définition}

% Commande tikz
\usetikzlibrary{calc}
\usetikzlibrary{patterns,arrows}
\usetikzlibrary{matrix}
\usetikzlibrary{fadings} 

%definition d'un terme
\newcommand{\defi}[1]{{\color{myorange}\textbf{\emph{#1}}}}
\newcommand{\evidence}[1]{{\color{blue}\textbf{\emph{#1}}}}
\newcommand{\assertion}[1]{\emph{\og#1\fg}}  % pour chapitre logique
%\renewcommand{\contentsname}{Sommaire}
\renewcommand{\contentsname}{}
\setcounter{tocdepth}{2}



%------ Figures ------

\def\myscale{1} % par défaut 
\newcommand{\myfigure}[2]{  % entrée : echelle, fichier figure
\def\myscale{#1}
\begin{center}
\footnotesize
{#2}
\end{center}}


%------ Encadrement ------

\usepackage{fancybox}


\newcommand{\mybox}[1]{
\setlength{\fboxsep}{7pt}
\begin{center}
\shadowbox{#1}
\end{center}}

\newcommand{\myboxinline}[1]{
\setlength{\fboxsep}{5pt}
\raisebox{-10pt}{
\shadowbox{#1}
}
}

%--------------- Commande beamer---------------
\newcommand{\beameronly}[1]{#1} % permet de mettre des pause dans beamer pas dans poly


\setbeamertemplate{navigation symbols}{}
\setbeamertemplate{footline}  % tiré du fichier beamerouterinfolines.sty
{
  \leavevmode%
  \hbox{%
  \begin{beamercolorbox}[wd=.333333\paperwidth,ht=2.25ex,dp=1ex,center]{author in head/foot}%
    % \usebeamerfont{author in head/foot}\insertshortauthor%~~(\insertshortinstitute)
    \usebeamerfont{section in head/foot}{\bf\insertshorttitle}
  \end{beamercolorbox}%
  \begin{beamercolorbox}[wd=.333333\paperwidth,ht=2.25ex,dp=1ex,center]{title in head/foot}%
    \usebeamerfont{section in head/foot}{\bf\insertsectionhead}
  \end{beamercolorbox}%
  \begin{beamercolorbox}[wd=.333333\paperwidth,ht=2.25ex,dp=1ex,right]{date in head/foot}%
    % \usebeamerfont{date in head/foot}\insertshortdate{}\hspace*{2em}
    \insertframenumber{} / \inserttotalframenumber\hspace*{2ex} 
  \end{beamercolorbox}}%
  \vskip0pt%
}


\definecolor{mygrey}{rgb}{0.5,0.5,0.5}
\setlength{\parindent}{0cm}
%\DeclareTextFontCommand{\helvetica}{\fontfamily{phv}\selectfont}

% background beamer
\definecolor{couleurhaut}{rgb}{0.85,0.9,1}  % creme
\definecolor{couleurmilieu}{rgb}{1,1,1}  % vert pale
\definecolor{couleurbas}{rgb}{0.85,0.9,1}  % blanc
\setbeamertemplate{background canvas}[vertical shading]%
[top=couleurhaut,middle=couleurmilieu,midpoint=0.4,bottom=couleurbas] 
%[top=fondtitre!05,bottom=fondtitre!60]



\makeatletter
\setbeamertemplate{theorem begin}
{%
  \begin{\inserttheoremblockenv}
  {%
    \inserttheoremheadfont
    \inserttheoremname
    \inserttheoremnumber
    \ifx\inserttheoremaddition\@empty\else\ (\inserttheoremaddition)\fi%
    \inserttheorempunctuation
  }%
}
\setbeamertemplate{theorem end}{\end{\inserttheoremblockenv}}

\newenvironment{theoreme}[1][]{%
   \setbeamercolor{block title}{fg=structure,bg=structure!40}
   \setbeamercolor{block body}{fg=black,bg=structure!10}
   \begin{block}{{\bf Th\'eor\`eme }#1}
}{%
   \end{block}%
}


\newenvironment{proposition}[1][]{%
   \setbeamercolor{block title}{fg=structure,bg=structure!40}
   \setbeamercolor{block body}{fg=black,bg=structure!10}
   \begin{block}{{\bf Proposition }#1}
}{%
   \end{block}%
}

\newenvironment{corollaire}[1][]{%
   \setbeamercolor{block title}{fg=structure,bg=structure!40}
   \setbeamercolor{block body}{fg=black,bg=structure!10}
   \begin{block}{{\bf Corollaire }#1}
}{%
   \end{block}%
}

\newenvironment{mydefinition}[1][]{%
   \setbeamercolor{block title}{fg=structure,bg=structure!40}
   \setbeamercolor{block body}{fg=black,bg=structure!10}
   \begin{block}{{\bf Définition} #1}
}{%
   \end{block}%
}

\newenvironment{lemme}[0]{%
   \setbeamercolor{block title}{fg=structure,bg=structure!40}
   \setbeamercolor{block body}{fg=black,bg=structure!10}
   \begin{block}{\bf Lemme}
}{%
   \end{block}%
}

\newenvironment{remarque}[1][]{%
   \setbeamercolor{block title}{fg=black,bg=structure!20}
   \setbeamercolor{block body}{fg=black,bg=structure!5}
   \begin{block}{Remarque #1}
}{%
   \end{block}%
}


\newenvironment{exemple}[1][]{%
   \setbeamercolor{block title}{fg=black,bg=structure!20}
   \setbeamercolor{block body}{fg=black,bg=structure!5}
   \begin{block}{{\bf Exemple }#1}
}{%
   \end{block}%
}


\newenvironment{miniexercice}[0]{%
   \setbeamercolor{block title}{fg=structure,bg=structure!20}
   \setbeamercolor{block body}{fg=black,bg=structure!5}
   \begin{block}{Mini-exercices}
}{%
   \end{block}%
}


\newenvironment{tp}[0]{%
   \setbeamercolor{block title}{fg=structure,bg=structure!40}
   \setbeamercolor{block body}{fg=black,bg=structure!10}
   \begin{block}{\bf Travaux pratiques}
}{%
   \end{block}%
}
\newenvironment{exercicecours}[1][]{%
   \setbeamercolor{block title}{fg=structure,bg=structure!40}
   \setbeamercolor{block body}{fg=black,bg=structure!10}
   \begin{block}{{\bf Exercice }#1}
}{%
   \end{block}%
}
\newenvironment{algo}[1][]{%
   \setbeamercolor{block title}{fg=structure,bg=structure!40}
   \setbeamercolor{block body}{fg=black,bg=structure!10}
   \begin{block}{{\bf Algorithme}\hfill{\color{gray}\texttt{#1}}}
}{%
   \end{block}%
}


\setbeamertemplate{proof begin}{
   \setbeamercolor{block title}{fg=black,bg=structure!20}
   \setbeamercolor{block body}{fg=black,bg=structure!5}
   \begin{block}{{\footnotesize Démonstration}}
   \footnotesize
   \smallskip}
\setbeamertemplate{proof end}{%
   \end{block}}
\setbeamertemplate{qed symbol}{\openbox}


\makeatother
\usecolortheme[RGB={0,127,0}]{structure}

% Commande spécifique à ce chapitre

\newcommand{\Python}{\texttt{Python}}
\renewcommand{\evidence}[1]{{\color{blue}\textbf{#1}}}

\usepackage{textcomp}

\usepackage{listings}
\lstset{
  upquote=true,
  columns=flexible,
  keepspaces=true,
  basicstyle=\ttfamily,
  commentstyle=\color{gray},
  language=Python,
  showstringspaces=false,
  aboveskip=0em,  
  belowskip=0em,
  escapeinside=||
}

\lstset{
  literate={é}{{\'e}}1
           {è}{{\`e}}1
           {à}{{\`a}}1
}


\newcommand{\codeinline}[1]{\lstinline!#1!}


%%%%%%%%%%%%%%%%%%%%%%%%%%%%%%%%%%%%%%%%%%%%%%%%%%%%%%%%%%%%%
%%%%%%%%%%%%%%%%%%%%%%%%%%%%%%%%%%%%%%%%%%%%%%%%%%%%%%%%%%%%%


\begin{document}


\title{{\bf Zéros des fonctions}}
\subtitle{La méthode de la sécante}

\begin{frame}
  
  \debutmontitre

  \pause

{\footnotesize
\hfill
\setbeamercovered{transparent=50}
\begin{minipage}{0.6\textwidth}
  \begin{itemize}
    \item<3-> Principe de la sécante
    \item<4-> Résultats numériques pour $\sqrt{10}$
    \item<5-> Résultats numériques pour $(1,10)^{1/12}$
    \item<6-> Calcul de l'erreur
    \item<7-> Algorithme    
  \end{itemize}
\end{minipage}
}

\end{frame}

\setcounter{framenumber}{0}


%%%%%%%%%%%%%%%%%%%%%%%%%%%%%%%%%%%%%%%%%%%%%%%%%%%%%%%%%%%%%%%%
\section{Principe de la sécante}

\begin{frame}


\myfigure{1.5}{
\tikzinput{fig_zeros06-pres}
}

\end{frame}

\begin{frame}
\begin{proposition}
 Soit $f : [a,b] \to \Rr$ une fonction continue, strictement croissante et convexe telle
 que $f(a) \le 0$, $f(b) > 0$.
 Alors la suite définie par
 \mybox{$a_0=a \qquad \text{ et } \qquad a_{n+1} = a_n - \frac{b-a_n}{f(b)-f(a_n)}f(a_n)$}
 est croissante et converge vers la solution $\ell$ de $(f(x)=0)$
\end{proposition}  
\end{frame}


\begin{frame}
L'hypothèse $f$ \defi{convexe} signifie exactement que 
pour tout $x,x'$ dans $[a,b]$ la sécante (ou corde) entre 
$(x,f(x))$ et $(x',f(x'))$ est au-dessus du graphe de $f$

\myfigure{1}{
\tikzinput{fig_zeros07}
}

\end{frame}


\begin{frame}
\begin{minipage}{0.62\textwidth}
\mybox{$a_0=a \quad \text{et} \quad a_{n+1} = a_n - \frac{b-a_n}{f(b)-f(a_n)}f(a_n)$}  
\end{minipage}
\pause
\begin{minipage}{0.3\textwidth}
\myfigure{0.7}{
\tikzinput{fig_zeros06-pres-bis}
}  
\end{minipage}
\vspace*{-3ex}
\pause
\begin{proof}
\begin{enumerate}
  \item \textbf{Construction de la suite}

  \pause
  
  \begin{itemize}
    \item L'équation de la droite passant par les deux points $(a,f(a))$ et $(b,f(b))$ est 
$$y=(x-a) \frac{f(b)-f(a)}{b-a} + f(a)$$

\pause

    \item Cette droite intersecte l'axe des abscisses en $(a',0)$ qui vérifie donc 
$0=(a'-a) \frac{f(b)-f(a)}{b-a} + f(a)$, donc $a'=a - \frac{b-a}{f(b)-f(a)}f(a)$ \qedhere
  \end{itemize}
\end{enumerate}
\end{proof}
\end{frame} 

\begin{frame}
\begin{proof}
\begin{enumerate}  
  \setcounter{enumi}{1}
  \item \textbf{Croissance de $(a_n)$}
  
\pause  

 
  \begin{itemize}\setlength{\itemsep}{4pt}
    \item Comme $f$ est convexe : $f(a_n)\le 0$ 
\pause     
    \item Formule de récurrence $a_{n+1} = a_n - \frac{b-a_n}{f(b)-f(a_n)}f(a_n)$
\pause 
    \item Ainsi $a_{n+1} \ge a_n$
  \end{itemize}
  
\pause
\bigskip

  \item \textbf{Convergence de $(a_n)$}
\pause    
  \begin{itemize}\setlength{\itemsep}{4pt}
    \item La suite $(a_n)$ est croissante et majorée par $b$, donc elle converge
\pause    
    \item Notons $\ell<b$ sa limite
\pause    
    \item $a_n \to \ell$, $a_{n+1} \to \ell$, $f(a_n)\to f(\ell)$
\pause    
    \item L'égalité $a_{n+1} = a_n - \frac{b-a_n}{f(b)-f(a_n)}f(a_n)$ devient 
  $\ell = \ell - \frac{b-\ell}{f(b)-f(\ell)} f(\ell)$
\pause  
    \item Ce qui implique $f(\ell)=0$ \qedhere
  \end{itemize}
  
\end{enumerate}

\end{proof}
\end{frame}


%%%%%%%%%%%%%%%%%%%%%%%%%%%%%%%%%%%%%%%%%%%%%%%%%%%%%%%%%%%%%%%%
\section{Résultats numériques pour $\sqrt{10}$}

\begin{frame}
\begin{itemize}
  \item Approximation de $\sqrt{10}$

\pause

  \item $f$ définie par $f(x)=x^2 - 10$, $f$ continue, strictement croissante et convexe sur $[0,+\infty[$

\pause

  \item Intervalle $[3,4]$ : $f(3) \le 0$ et $f(4) \ge 0$, donc $\sqrt{10} \in [3,4]$
\end{itemize}

\pause

$$
\begin{array}{l|l}
  a_0 = 3                          & \epsilon_0 \le 0,1666\ldots \\
  a_1 = 3,14285714285\ldots    & \epsilon_1 \le 0,02040\ldots \\
  a_2 = 3,16000000000\ldots    & \epsilon_2 \le 0,00239\ldots \\
  a_3 = 3,16201117318\ldots   & \epsilon_3 \le 0,00028\ldots \\
  a_4 = 3,16224648985\ldots   & \epsilon_4 \le 3,28\ldots \cdot 10^{-5} \\
  a_5 = 3,16227401437\ldots    & \epsilon_5 \le 3,84\ldots \cdot 10^{-6}\\
  a_6 = 3,16227723374\ldots    & \epsilon_6 \le 4,49\ldots \cdot 10^{-7} \\
  a_7 = 3,16227761029\ldots   & \epsilon_7 \le 5,25\ldots \cdot 10^{-8} \\
  a_8 = 3,16227765433\ldots   & \epsilon_8 \le 6,14\ldots \cdot 10^{-9} \\
\end{array}
$$
\end{frame}



%%%%%%%%%%%%%%%%%%%%%%%%%%%%%%%%%%%%%%%%%%%%%%%%%%%%%%%%%%%%%%%%
\section{Résultats numériques pour $(1,10)^{1/12}$}


 
\begin{frame}
Approximation de $(1,10)^{1/12}$ avec $f(x) = x^{12} - 1,10$, $a=1$ et $b=1,1$

\bigskip

$$
\begin{array}{l|l}
  a_0 = 1                          & \epsilon_0 \le 0,0083\ldots \\
  a_1 = 1,00467633\ldots   & \epsilon_1 \le 0,0035\ldots \\
  a_2 = 1,00661950\ldots   & \epsilon_2 \le 0,0014\ldots \\
  a_3 = 1,00741927\ldots   & \epsilon_3 \le 0,00060\ldots \\
  a_4 = 1,00774712\ldots    & \epsilon_4 \le 0,00024\ldots \\
  a_5 = 1,00788130\ldots    & \epsilon_5 \le 0,00010\ldots\\
  a_6 = 1,00793618\ldots   & \epsilon_6 \le 4,14\ldots \cdot 10^{-5} \\
  a_7 = 1,00795862\ldots   & \epsilon_7 \le 1,69\ldots \cdot 10^{-5} \\
  a_8 = 1,00796779\ldots   & \epsilon_8 \le 6,92\ldots \cdot 10^{-6} \\
\end{array}
$$
\end{frame}




%%%%%%%%%%%%%%%%%%%%%%%%%%%%%%%%%%%%%%%%%%%%%%%%%%%%%%%%%%%%%%%%
\section{Calcul de l'erreur}

\begin{frame}

\begin{proposition}
 Soit $f : I \to \Rr$ une fonction dérivable et $\ell$ tel que $f(\ell)=0$.
 S'il existe une constante $m>0$ telle que pour tout $x\in I$, $|f'(x)| \ge m$ alors
 $$|x-\ell| \le \frac{|f(x)|}{m} \qquad \text{ pour tout } x \in I$$
\end{proposition}

\pause

\begin{proof}
  Par l'inégalité des accroissement finis entre $x$ et $\ell$ :
  $$|f(x)-f(\ell)| \ge m |x-\ell|$$
  mais $f(\ell)=0$, d'où la majoration
\end{proof}



\end{frame}


\begin{frame}
\begin{exemple}[Erreur pour $\sqrt{10}$]
\begin{itemize}\setlength{\itemsep}{4pt}
  \item Soit $f(x)=x^2-10$ et l'intervalle $I=[3,4]$
\pause
  \item Alors $f'(x)=2x$ donc $|f'(x)| \ge 6$ sur $I$
\pause  
  \item On pose donc $m=6$, $\ell=\sqrt{10}$, $x=a_n$
\pause  
  \item Estimation de l'erreur $\epsilon_n = |\ell-a_n| \le \frac{|f(a_n)|}{m} = \frac{|a_n^2 - 10|}{6}$
\pause  
  \item Exemple : $a_2 = 3,16... \le 3,17$ donc $\sqrt{10}-a_2 \le \frac{|3,17^2-10|}{6} = 0,489$
\pause  
  \item Exemple : $a_8=3,1622776543347473\ldots$ donc \\ \hfill $\sqrt{10}-a_8 \le \frac{|a_8^2-10|}{6} = 6,14\ldots \cdot 10^{-9}$
\end{itemize}

\end{exemple}

\medskip
\pause

\begin{center}
\begin{tabular}{ll}
  $10^{-10}$ ($\sim 10$ décimales) &  $10$ itérations \\
  $10^{-100}$ ($\sim 100$ décimales) &  $107$ itérations \\ 
  $10^{-1000}$ ($\sim 1000$ décimales) &  $1073$ itérations \\ 
\end{tabular}  
\end{center}
\end{frame}


%%%%%%%%%%%%%%%%%%%%%%%%%%%%%%%%%%%%%%%%%%%%%%%%%%%%%%%%%%%%%%%%
\section{Algorithme}

\begin{frame}[fragile]

\begin{algo}[secante.py]
\begin{lstlisting}
def secante(a,b,n):  
    for i in range(n):
        a = a-f(a)*(b-a)/(f(b)-f(a))
    return a
\end{lstlisting}  
\end{algo}   
    
\end{frame}


%%%%%%%%%%%%%%%%%%%%%%%%%%%%%%%%%%%%%%%%%%%%%%%%%%%%%%%%%%%%%%%%
\section{Mini-exercices}

\begin{frame}

\begin{miniexercice}
\begin{enumerate}
  \item \`A la main, calculer un encadrement à $0,1$ près de $\sqrt{3}$.
  Idem avec $\sqrt[3]{2}$. 
  
  \item Calculer une approximation des solutions de l'équation $x^3+1=3x$.
  
  \item Calculer une approximation de la solution de l'équation $(\cos x = 0)$ sur $[0,\pi]$.   
  Idem avec $(\cos x = 2\sin x)$.
  
  \item \'Etudier l'équation $(\exp(-x) = - \ln(x))$. Donner une approximation de la (ou des) 
  solution(s) et une majoration de l'erreur correspondante.
\end{enumerate}
\end{miniexercice}

\end{frame}

\end{document}