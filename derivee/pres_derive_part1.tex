
%%%%%%%%%%%%%%%%%% PREAMBULE %%%%%%%%%%%%%%%%%%

\documentclass[aspectratio=169,utf8]{beamer}
%\documentclass[aspectratio=169,handout]{beamer}

\usetheme{Boadilla}
%\usecolortheme{seahorse}
\usecolortheme[RGB={245,66,24}]{structure}
\useoutertheme{infolines}

% packages
\usepackage{amsfonts,amsmath,amssymb,amsthm}
\usepackage[utf8]{inputenc}
\usepackage[T1]{fontenc}
\usepackage{lmodern}

\usepackage[francais]{babel}
\usepackage{fancybox}
\usepackage{graphicx}

\usepackage{float}
\usepackage{xfrac}

%\usepackage[usenames, x11names]{xcolor}
\usepackage{tikz}
\usepackage{pgfplots}
\usepackage{datetime}



%-----  Package unités -----
\usepackage{siunitx}
\sisetup{locale = FR,detect-all,per-mode = symbol}

%\usepackage{mathptmx}
%\usepackage{fouriernc}
%\usepackage{newcent}
%\usepackage[mathcal,mathbf]{euler}

%\usepackage{palatino}
%\usepackage{newcent}
% \usepackage[mathcal,mathbf]{euler}



% \usepackage{hyperref}
% \hypersetup{colorlinks=true, linkcolor=blue, urlcolor=blue,
% pdftitle={Exo7 - Exercices de mathématiques}, pdfauthor={Exo7}}


%section
% \usepackage{sectsty}
% \allsectionsfont{\bf}
%\sectionfont{\color{Tomato3}\upshape\selectfont}
%\subsectionfont{\color{Tomato4}\upshape\selectfont}

%----- Ensembles : entiers, reels, complexes -----
\newcommand{\Nn}{\mathbb{N}} \newcommand{\N}{\mathbb{N}}
\newcommand{\Zz}{\mathbb{Z}} \newcommand{\Z}{\mathbb{Z}}
\newcommand{\Qq}{\mathbb{Q}} \newcommand{\Q}{\mathbb{Q}}
\newcommand{\Rr}{\mathbb{R}} \newcommand{\R}{\mathbb{R}}
\newcommand{\Cc}{\mathbb{C}} 
\newcommand{\Kk}{\mathbb{K}} \newcommand{\K}{\mathbb{K}}

%----- Modifications de symboles -----
\renewcommand{\epsilon}{\varepsilon}
\renewcommand{\Re}{\mathop{\text{Re}}\nolimits}
\renewcommand{\Im}{\mathop{\text{Im}}\nolimits}
%\newcommand{\llbracket}{\left[\kern-0.15em\left[}
%\newcommand{\rrbracket}{\right]\kern-0.15em\right]}

\renewcommand{\ge}{\geqslant}
\renewcommand{\geq}{\geqslant}
\renewcommand{\le}{\leqslant}
\renewcommand{\leq}{\leqslant}
\renewcommand{\epsilon}{\varepsilon}

%----- Fonctions usuelles -----
\newcommand{\ch}{\mathop{\text{ch}}\nolimits}
\newcommand{\sh}{\mathop{\text{sh}}\nolimits}
\renewcommand{\tanh}{\mathop{\text{th}}\nolimits}
\newcommand{\cotan}{\mathop{\text{cotan}}\nolimits}
\newcommand{\Arcsin}{\mathop{\text{arcsin}}\nolimits}
\newcommand{\Arccos}{\mathop{\text{arccos}}\nolimits}
\newcommand{\Arctan}{\mathop{\text{arctan}}\nolimits}
\newcommand{\Argsh}{\mathop{\text{argsh}}\nolimits}
\newcommand{\Argch}{\mathop{\text{argch}}\nolimits}
\newcommand{\Argth}{\mathop{\text{argth}}\nolimits}
\newcommand{\pgcd}{\mathop{\text{pgcd}}\nolimits} 


%----- Commandes divers ------
\newcommand{\ii}{\mathrm{i}}
\newcommand{\dd}{\text{d}}
\newcommand{\id}{\mathop{\text{id}}\nolimits}
\newcommand{\Ker}{\mathop{\text{Ker}}\nolimits}
\newcommand{\Card}{\mathop{\text{Card}}\nolimits}
\newcommand{\Vect}{\mathop{\text{Vect}}\nolimits}
\newcommand{\Mat}{\mathop{\text{Mat}}\nolimits}
\newcommand{\rg}{\mathop{\text{rg}}\nolimits}
\newcommand{\tr}{\mathop{\text{tr}}\nolimits}


%----- Structure des exercices ------

\newtheoremstyle{styleexo}% name
{2ex}% Space above
{3ex}% Space below
{}% Body font
{}% Indent amount 1
{\bfseries} % Theorem head font
{}% Punctuation after theorem head
{\newline}% Space after theorem head 2
{}% Theorem head spec (can be left empty, meaning ‘normal’)

%\theoremstyle{styleexo}
\newtheorem{exo}{Exercice}
\newtheorem{ind}{Indications}
\newtheorem{cor}{Correction}


\newcommand{\exercice}[1]{} \newcommand{\finexercice}{}
%\newcommand{\exercice}[1]{{\tiny\texttt{#1}}\vspace{-2ex}} % pour afficher le numero absolu, l'auteur...
\newcommand{\enonce}{\begin{exo}} \newcommand{\finenonce}{\end{exo}}
\newcommand{\indication}{\begin{ind}} \newcommand{\finindication}{\end{ind}}
\newcommand{\correction}{\begin{cor}} \newcommand{\fincorrection}{\end{cor}}

\newcommand{\noindication}{\stepcounter{ind}}
\newcommand{\nocorrection}{\stepcounter{cor}}

\newcommand{\fiche}[1]{} \newcommand{\finfiche}{}
\newcommand{\titre}[1]{\centerline{\large \bf #1}}
\newcommand{\addcommand}[1]{}
\newcommand{\video}[1]{}

% Marge
\newcommand{\mymargin}[1]{\marginpar{{\small #1}}}

\def\noqed{\renewcommand{\qedsymbol}{}}


%----- Presentation ------
\setlength{\parindent}{0cm}

%\newcommand{\ExoSept}{\href{http://exo7.emath.fr}{\textbf{\textsf{Exo7}}}}

\definecolor{myred}{rgb}{0.93,0.26,0}
\definecolor{myorange}{rgb}{0.97,0.58,0}
\definecolor{myyellow}{rgb}{1,0.86,0}

\newcommand{\LogoExoSept}[1]{  % input : echelle
{\usefont{U}{cmss}{bx}{n}
\begin{tikzpicture}[scale=0.1*#1,transform shape]
  \fill[color=myorange] (0,0)--(4,0)--(4,-4)--(0,-4)--cycle;
  \fill[color=myred] (0,0)--(0,3)--(-3,3)--(-3,0)--cycle;
  \fill[color=myyellow] (4,0)--(7,4)--(3,7)--(0,3)--cycle;
  \node[scale=5] at (3.5,3.5) {Exo7};
\end{tikzpicture}}
}


\newcommand{\debutmontitre}{
  \author{} \date{} 
  \thispagestyle{empty}
  \hspace*{-10ex}
  \begin{minipage}{\textwidth}
    \titlepage  
  \vspace*{-2.5cm}
  \begin{center}
    \LogoExoSept{2.5}
  \end{center}
  \end{minipage}

  \vspace*{-0cm}
  
  % Astuce pour que le background ne soit pas discrétisé lors de la conversion pdf -> png
\begin{tikzpicture}
        \fill[opacity=0,green!60!black] (0,0)--++(0,0)--++(0,0)--++(0,0)--cycle; 
\end{tikzpicture}

% toc S'affiche trop tot :
% \tableofcontents[hideallsubsections, pausesections]
}

\newcommand{\finmontitre}{
  \end{frame}
  \setcounter{framenumber}{0}
} % ne marche pas pour une raison obscure

%----- Commandes supplementaires ------

% \usepackage[landscape]{geometry}
% \geometry{top=1cm, bottom=3cm, left=2cm, right=10cm, marginparsep=1cm
% }
% \usepackage[a4paper]{geometry}
% \geometry{top=2cm, bottom=2cm, left=2cm, right=2cm, marginparsep=1cm
% }

%\usepackage{standalone}


% New command Arnaud -- november 2011
\setbeamersize{text margin left=24ex}
% si vous modifier cette valeur il faut aussi
% modifier le decalage du titre pour compenser
% (ex : ici =+10ex, titre =-5ex

\theoremstyle{definition}
%\newtheorem{proposition}{Proposition}
%\newtheorem{exemple}{Exemple}
%\newtheorem{theoreme}{Théorème}
%\newtheorem{lemme}{Lemme}
%\newtheorem{corollaire}{Corollaire}
%\newtheorem*{remarque*}{Remarque}
%\newtheorem*{miniexercice}{Mini-exercices}
%\newtheorem{definition}{Définition}

% Commande tikz
\usetikzlibrary{calc}
\usetikzlibrary{patterns,arrows}
\usetikzlibrary{matrix}
\usetikzlibrary{fadings} 

%definition d'un terme
\newcommand{\defi}[1]{{\color{myorange}\textbf{\emph{#1}}}}
\newcommand{\evidence}[1]{{\color{blue}\textbf{\emph{#1}}}}
\newcommand{\assertion}[1]{\emph{\og#1\fg}}  % pour chapitre logique
%\renewcommand{\contentsname}{Sommaire}
\renewcommand{\contentsname}{}
\setcounter{tocdepth}{2}



%------ Figures ------

\def\myscale{1} % par défaut 
\newcommand{\myfigure}[2]{  % entrée : echelle, fichier figure
\def\myscale{#1}
\begin{center}
\footnotesize
{#2}
\end{center}}


%------ Encadrement ------

\usepackage{fancybox}


\newcommand{\mybox}[1]{
\setlength{\fboxsep}{7pt}
\begin{center}
\shadowbox{#1}
\end{center}}

\newcommand{\myboxinline}[1]{
\setlength{\fboxsep}{5pt}
\raisebox{-10pt}{
\shadowbox{#1}
}
}

%--------------- Commande beamer---------------
\newcommand{\beameronly}[1]{#1} % permet de mettre des pause dans beamer pas dans poly


\setbeamertemplate{navigation symbols}{}
\setbeamertemplate{footline}  % tiré du fichier beamerouterinfolines.sty
{
  \leavevmode%
  \hbox{%
  \begin{beamercolorbox}[wd=.333333\paperwidth,ht=2.25ex,dp=1ex,center]{author in head/foot}%
    % \usebeamerfont{author in head/foot}\insertshortauthor%~~(\insertshortinstitute)
    \usebeamerfont{section in head/foot}{\bf\insertshorttitle}
  \end{beamercolorbox}%
  \begin{beamercolorbox}[wd=.333333\paperwidth,ht=2.25ex,dp=1ex,center]{title in head/foot}%
    \usebeamerfont{section in head/foot}{\bf\insertsectionhead}
  \end{beamercolorbox}%
  \begin{beamercolorbox}[wd=.333333\paperwidth,ht=2.25ex,dp=1ex,right]{date in head/foot}%
    % \usebeamerfont{date in head/foot}\insertshortdate{}\hspace*{2em}
    \insertframenumber{} / \inserttotalframenumber\hspace*{2ex} 
  \end{beamercolorbox}}%
  \vskip0pt%
}


\definecolor{mygrey}{rgb}{0.5,0.5,0.5}
\setlength{\parindent}{0cm}
%\DeclareTextFontCommand{\helvetica}{\fontfamily{phv}\selectfont}

% background beamer
\definecolor{couleurhaut}{rgb}{0.85,0.9,1}  % creme
\definecolor{couleurmilieu}{rgb}{1,1,1}  % vert pale
\definecolor{couleurbas}{rgb}{0.85,0.9,1}  % blanc
\setbeamertemplate{background canvas}[vertical shading]%
[top=couleurhaut,middle=couleurmilieu,midpoint=0.4,bottom=couleurbas] 
%[top=fondtitre!05,bottom=fondtitre!60]



\makeatletter
\setbeamertemplate{theorem begin}
{%
  \begin{\inserttheoremblockenv}
  {%
    \inserttheoremheadfont
    \inserttheoremname
    \inserttheoremnumber
    \ifx\inserttheoremaddition\@empty\else\ (\inserttheoremaddition)\fi%
    \inserttheorempunctuation
  }%
}
\setbeamertemplate{theorem end}{\end{\inserttheoremblockenv}}

\newenvironment{theoreme}[1][]{%
   \setbeamercolor{block title}{fg=structure,bg=structure!40}
   \setbeamercolor{block body}{fg=black,bg=structure!10}
   \begin{block}{{\bf Th\'eor\`eme }#1}
}{%
   \end{block}%
}


\newenvironment{proposition}[1][]{%
   \setbeamercolor{block title}{fg=structure,bg=structure!40}
   \setbeamercolor{block body}{fg=black,bg=structure!10}
   \begin{block}{{\bf Proposition }#1}
}{%
   \end{block}%
}

\newenvironment{corollaire}[1][]{%
   \setbeamercolor{block title}{fg=structure,bg=structure!40}
   \setbeamercolor{block body}{fg=black,bg=structure!10}
   \begin{block}{{\bf Corollaire }#1}
}{%
   \end{block}%
}

\newenvironment{mydefinition}[1][]{%
   \setbeamercolor{block title}{fg=structure,bg=structure!40}
   \setbeamercolor{block body}{fg=black,bg=structure!10}
   \begin{block}{{\bf Définition} #1}
}{%
   \end{block}%
}

\newenvironment{lemme}[0]{%
   \setbeamercolor{block title}{fg=structure,bg=structure!40}
   \setbeamercolor{block body}{fg=black,bg=structure!10}
   \begin{block}{\bf Lemme}
}{%
   \end{block}%
}

\newenvironment{remarque}[1][]{%
   \setbeamercolor{block title}{fg=black,bg=structure!20}
   \setbeamercolor{block body}{fg=black,bg=structure!5}
   \begin{block}{Remarque #1}
}{%
   \end{block}%
}


\newenvironment{exemple}[1][]{%
   \setbeamercolor{block title}{fg=black,bg=structure!20}
   \setbeamercolor{block body}{fg=black,bg=structure!5}
   \begin{block}{{\bf Exemple }#1}
}{%
   \end{block}%
}


\newenvironment{miniexercice}[0]{%
   \setbeamercolor{block title}{fg=structure,bg=structure!20}
   \setbeamercolor{block body}{fg=black,bg=structure!5}
   \begin{block}{Mini-exercices}
}{%
   \end{block}%
}


\newenvironment{tp}[0]{%
   \setbeamercolor{block title}{fg=structure,bg=structure!40}
   \setbeamercolor{block body}{fg=black,bg=structure!10}
   \begin{block}{\bf Travaux pratiques}
}{%
   \end{block}%
}
\newenvironment{exercicecours}[1][]{%
   \setbeamercolor{block title}{fg=structure,bg=structure!40}
   \setbeamercolor{block body}{fg=black,bg=structure!10}
   \begin{block}{{\bf Exercice }#1}
}{%
   \end{block}%
}
\newenvironment{algo}[1][]{%
   \setbeamercolor{block title}{fg=structure,bg=structure!40}
   \setbeamercolor{block body}{fg=black,bg=structure!10}
   \begin{block}{{\bf Algorithme}\hfill{\color{gray}\texttt{#1}}}
}{%
   \end{block}%
}


\setbeamertemplate{proof begin}{
   \setbeamercolor{block title}{fg=black,bg=structure!20}
   \setbeamercolor{block body}{fg=black,bg=structure!5}
   \begin{block}{{\footnotesize Démonstration}}
   \footnotesize
   \smallskip}
\setbeamertemplate{proof end}{%
   \end{block}}
\setbeamertemplate{qed symbol}{\openbox}


\makeatother
\usecolortheme[RGB={0,199,174}]{structure}

%%%%%%%%%%%%%%%%%%%%%%%%%%%%%%%%%%%%%%%%%%%%%%%%%%%%%%%%%%%%%
%%%%%%%%%%%%%%%%%%%%%%%%%%%%%%%%%%%%%%%%%%%%%%%%%%%%%%%%%%%%%



\begin{document}



\title{{\bf Dérivée d'une fonction}}
\subtitle{Dérivée en un point}

\begin{frame}
  
  \debutmontitre

  \pause

{\footnotesize
\hfill
\setbeamercovered{transparent=50}
\begin{minipage}{0.6\textwidth}
  \begin{itemize}
    \item<3-> Définition
    \item<4-> Tangente
    \item<5-> Autres écritures de la dérivée
  \end{itemize}
\end{minipage}
}

\end{frame}

\setcounter{framenumber}{0}


%%%%%%%%%%%%%%%%%%%%%%%%%%%%%%%%%%%%%%%%%%%%%%%%%%%%%%%%%%%%%%%%


\section*{Motivation}


\begin{frame}

\begin{itemize}
\setlength{\itemsep}{5pt} 
\uncover<1->{  \item Calcul de $\sqrt{1,01}$}
\uncover<2->{    \item $1,01 \approx 1$ donc $\sqrt{1,01} \approx \sqrt 1$ }
\uncover<3->{    \item $f(x)=\sqrt{x}$ continue  }
\uncover<4->{    \item $x$ proche de $x_0$ $\implies$ $f(x)$  proche de $f(x_0)$ }
\uncover<5->{    \item Peut-on être plus précis ? }
\uncover<6->{    \item Une autre droite : la tangente }
\uncover<7->{    \item $y = (x-x_0) f'(x_0) + f(x_0)$ }
\uncover<8->{    \item $f'(x)=\frac{1}{2\sqrt x}$ }
\uncover<9->{    \item $y=(x-1)\frac12+1$ } 
\uncover<10->{    \item $x$ proche de $1$ on a $f(x) \approx (x-1)\frac12+1$ }
\uncover<11->{    \item $x=1,01$ $\implies$ $f(x) \approx 1+\frac12(x-1) = 1 + \frac{0,01}{2}=1,005$ }
\uncover<12->{    \item $\sqrt{1,01}=1,00498\ldots$ }
\end{itemize}

\vspace*{-60mm}\hfill
\begin{minipage}{0.5\textwidth}
\uncover<3->{
\myfigure{1}{
\tikzinput{fig_derive01pres} 
}   
}
\bigskip
\bigskip

\end{minipage}



\end{frame}


%%%%%%%%%%%%%%%%%%%%%%%%%%%%%%%%%%%%%%%%%%%%%%%%%%%%%%%%%%%%%%%%


\section*{Dérivée en un point}


\begin{frame}

Soit $I$ un intervalle ouvert de $\Rr$ et $f : I \to \Rr$ une fonction. Soit $x_0 \in I$

\pause

\begin{mydefinition}
\begin{itemize}
  \item $f$ est \defi{dérivable en $x_0$} si le \evidence{taux d'accroissement} $\frac{f(x)-f(x_0)}{x-x_0}$ 
a une limite finie lorsque $x$ tend vers $x_0$
\pause
  \item La limite s'appelle alors le \defi{nombre dérivé} de $f$ en $x_0$ et est noté $f'(x_0)$

\end{itemize}
\pause
\mybox{$\displaystyle f'(x_0)= \lim_{x\to x_0} \frac{f(x)-f(x_0)}{x-x_0}$} 
\end{mydefinition}

\end{frame}


\begin{frame}
\begin{mydefinition}
\begin{itemize}
  \item $f$ est \defi{dérivable sur $I$} si $f$ est dérivable en tout point $x_0 \in I$
\pause
  \item $x \mapsto f'(x)$ est la \defi{fonction dérivée} de $f$
\pause
  \item Elle se note $f'$ ou $\frac{df}{dx}$
\end{itemize}
\end{mydefinition}

\pause
\bigskip

\begin{exemple}
\begin{itemize}
  \item La fonction définie par $f(x)=x^2$  est dérivable en tout point $x_0 \in \Rr$
\pause
\medskip

  \item $\displaystyle \frac{f(x)-f(x_0)}{x-x_0} \pause= \frac{x^2-x_0^2}{x-x_0} 
  \pause= \frac{(x-x_0)(x+x_0)}{x-x_0}\pause=x+x_0 \pause\xrightarrow[x \to x_0]{} 2x_0$
\pause
\medskip

  \item Le nombre dérivé de $f$ en $x_0$ est $2x_0$, autrement dit : $f'(x)=2x$
\end{itemize}
\end{exemple}

\end{frame}


\begin{frame}
\begin{exemple}
Soit $f(x)=\sin x$. Montrons $f'(x)=\cos x$

\pause

\begin{enumerate}  

  \item En $x_0=0$
\pause
  \begin{itemize} \setlength{\itemsep}{5pt} 
    \item $\frac{\sin x}{x} \xrightarrow[x\to 0]{} 1$
\pause
    \item $\frac{f(x)-f(0)}{x-0} = \frac{\sin x}{x} \to 1$
\pause
    \item $f$ est dérivable en $x_0=0$ et $f'(0)=1$
  \end{itemize}

\pause
  \item En $x_0$ quelconque
\pause
  \begin{itemize} \setlength{\itemsep}{5pt}
    \item $\sin p-\sin q = 2\sin \frac{p-q}{2}\cdot\cos\frac{p+q}{2}$
\pause
    \item $\frac{f(x)-f(x_0)}{x-x_0} = \frac{\sin x - \sin x_0}{x-x_0} 
= \frac{\sin \frac{x-x_0}{2}}{\frac{x-x_0}{2}} \cdot \cos \frac{x+x_0}{2}$
\pause
    \item $\cos \frac{x+x_0}{2}\to \cos x_0$ lorsque $x\to x_0$
\pause
    \item $u=\frac{x-x_0}{2}$ alors  $\frac {\sin u}u \to 1$ lorsque $u\to 0$
\pause
    \item $\frac{f(x)-f(x_0)}{x-x_0} \to \cos x_0$
\pause
    \item $f'(x)=\cos x$
  \end{itemize}
\end{enumerate}
\end{exemple}
\end{frame}


%%%%%%%%%%%%%%%%%%%%%%%%%%%%%%%%%%%%%%%%%%%%%%%%%%%%%%%%%%%%%%%%


\section*{Tangente}


\begin{frame}


La \defi{tangente} au point $(x_0,f(x_0))$ est la droite d'équation 
\mybox{$y = (x-x_0) f'(x_0) + f(x_0)$}

\pause

La droite passant par $(x_0,f(x_0))$ et $(x,f(x))$

a pour coefficient directeur $\frac{f(x)-f(x_0)}{x-x_0}$

\pause

\myfigure{1.6}{
\tikzinput{/fig_derive02pres} 
}  

\end{frame}


%%%%%%%%%%%%%%%%%%%%%%%%%%%%%%%%%%%%%%%%%%%%%%%%%%%%%%%%%%%%%%%%


\section*{Autres écritures de la dérivée}


\begin{frame}
\begin{proposition}
\label{prop:ecrideriv}
\ 
\begin{itemize}
  \item $f$ est dérivable en $x_0$\!
$\iff$ \!$\displaystyle \lim_{h\to 0} \frac{f(x_0+h)-f(x_0)}{h}$ existe et est finie

\pause
\medskip

  \item $f$ est dérivable en $x_0$ 
$\iff$ il existe $\ell \in \Rr$ (qui sera $f'(x_0)$)

et une fonction $\epsilon : I \to \Rr$ telle que $\epsilon(x) \xrightarrow[x\to x_0]{} 0$ avec
$$f(x)=f(x_0)+(x-x_0) \ell + (x-x_0) \epsilon(x)$$
\end{itemize}
\end{proposition}


\pause

\begin{proof}
$$\frac{f(x)-f(x_0)}{x-x_0} = \ell + \epsilon(x)$$
\end{proof}
\end{frame}



\begin{frame}
\begin{proposition}
Soit $I$ un intervalle ouvert, $x_0 \in I$ et soit $f : I \to \Rr$ une fonction
\pause
\begin{itemize}
  \item Si $f$ est dérivable en $x_0$ alors $f$ est continue en $x_0$
\pause
  \item Si $f$ est dérivable sur $I$ alors $f$ est continue sur $I$
\end{itemize}
\end{proposition}

\pause
\bigskip

\begin{proof}
\begin{itemize}
  \item Supposons $f$ dérivable en $x_0$
\pause
  \item $f(x)= f(x_0)+\alt<5>{(x-x_0) \ell\vphantom{\underbrace{(x-x_0) \ell}_{\to 0}}}{\underbrace{(x-x_0) \ell}_{\to 0}} + \alt<5>{(x-x_0) \epsilon(x)}{\underbrace{(x-x_0) \epsilon(x)}_{\to 0}}$
\pause \pause
  \item $f(x) \to f(x_0)$ lorsque $x \to x_0$
\pause
  \item Ainsi $f$ est continue en $x_0$
\end{itemize}
\vspace*{-2ex}
\end{proof}

\end{frame}



\begin{frame}

\centerline{La réciproque est \alert{fausse}}

\pause

\begin{exemple}
\centerline{La fonction valeur absolue est continue en $0$}

\centerline{\textbf{mais n'est pas dérivable en $0$}}
  
\pause

\myfigure{1}{
\tikzinput{fig_derive03} 
} 

\pause

Taux d'accroissement de $f(x)=|x|$ en $x_0=0$ 
$$\frac{f(x)-f(0)}{x-0} \pause= \frac{|x|}{x} \pause
=
\begin{cases}

+1 & \text{ si } x>0 \\
\pause
-1 & \text{ si } x < 0 \\
\end{cases}
$$
\vspace*{-2ex}
\end{exemple}


\end{frame}

%%%%%%%%%%%%%%%%%%%%%%%%%%%%%%%%%%%%%%%%%%%%%%%%%%%%%%%%%%%%%%%%
\section*{Mini-exercices}


\begin{frame}
\begin{miniexercice}
\begin{enumerate}
  \item Montrer que la fonction $f(x)=x^3$ est dérivable en tout point $x_0\in \Rr$ et que $f'(x_0)=3x_0^2$.
  \item Montrer que la fonction $f(x)=\sqrt x$ est dérivable en tout point $x_0 >0$ et que $f'(x_0)=\frac{1}{2\sqrt{x_0}}$.
  \item Montrer que la fonction $f(x)=\sqrt x$ (qui est continue en $x_0=0$) n'est pas dérivable en $x_0=0$.
  \item Calculer l'équation de la tangente $(T_0)$ à la courbe d'équation $y=x^3-x^2-x$ au point d'abscisse $x_0=2$.
Calculer $x_1$ afin que la tangente $(T_1)$ au point d'abscisse $x_1$ soit parallèle à $(T_0)$.
  \item Montrer que si une fonction $f$ est paire et dérivable, alors $f'$ est une fonction impaire.
\end{enumerate}
\end{miniexercice}
\end{frame}


\end{document}