
%%%%%%%%%%%%%%%%%% PREAMBULE %%%%%%%%%%%%%%%%%%

\documentclass[aspectratio=169,utf8]{beamer}
%\documentclass[aspectratio=169,handout]{beamer}

\usetheme{Boadilla}
%\usecolortheme{seahorse}
\usecolortheme[RGB={245,66,24}]{structure}
\useoutertheme{infolines}

% packages
\usepackage{amsfonts,amsmath,amssymb,amsthm}
\usepackage[utf8]{inputenc}
\usepackage[T1]{fontenc}
\usepackage{lmodern}

\usepackage[francais]{babel}
\usepackage{fancybox}
\usepackage{graphicx}

\usepackage{float}
\usepackage{xfrac}

%\usepackage[usenames, x11names]{xcolor}
\usepackage{tikz}
\usepackage{pgfplots}
\usepackage{datetime}



%-----  Package unités -----
\usepackage{siunitx}
\sisetup{locale = FR,detect-all,per-mode = symbol}

%\usepackage{mathptmx}
%\usepackage{fouriernc}
%\usepackage{newcent}
%\usepackage[mathcal,mathbf]{euler}

%\usepackage{palatino}
%\usepackage{newcent}
% \usepackage[mathcal,mathbf]{euler}



% \usepackage{hyperref}
% \hypersetup{colorlinks=true, linkcolor=blue, urlcolor=blue,
% pdftitle={Exo7 - Exercices de mathématiques}, pdfauthor={Exo7}}


%section
% \usepackage{sectsty}
% \allsectionsfont{\bf}
%\sectionfont{\color{Tomato3}\upshape\selectfont}
%\subsectionfont{\color{Tomato4}\upshape\selectfont}

%----- Ensembles : entiers, reels, complexes -----
\newcommand{\Nn}{\mathbb{N}} \newcommand{\N}{\mathbb{N}}
\newcommand{\Zz}{\mathbb{Z}} \newcommand{\Z}{\mathbb{Z}}
\newcommand{\Qq}{\mathbb{Q}} \newcommand{\Q}{\mathbb{Q}}
\newcommand{\Rr}{\mathbb{R}} \newcommand{\R}{\mathbb{R}}
\newcommand{\Cc}{\mathbb{C}} 
\newcommand{\Kk}{\mathbb{K}} \newcommand{\K}{\mathbb{K}}

%----- Modifications de symboles -----
\renewcommand{\epsilon}{\varepsilon}
\renewcommand{\Re}{\mathop{\text{Re}}\nolimits}
\renewcommand{\Im}{\mathop{\text{Im}}\nolimits}
%\newcommand{\llbracket}{\left[\kern-0.15em\left[}
%\newcommand{\rrbracket}{\right]\kern-0.15em\right]}

\renewcommand{\ge}{\geqslant}
\renewcommand{\geq}{\geqslant}
\renewcommand{\le}{\leqslant}
\renewcommand{\leq}{\leqslant}
\renewcommand{\epsilon}{\varepsilon}

%----- Fonctions usuelles -----
\newcommand{\ch}{\mathop{\text{ch}}\nolimits}
\newcommand{\sh}{\mathop{\text{sh}}\nolimits}
\renewcommand{\tanh}{\mathop{\text{th}}\nolimits}
\newcommand{\cotan}{\mathop{\text{cotan}}\nolimits}
\newcommand{\Arcsin}{\mathop{\text{arcsin}}\nolimits}
\newcommand{\Arccos}{\mathop{\text{arccos}}\nolimits}
\newcommand{\Arctan}{\mathop{\text{arctan}}\nolimits}
\newcommand{\Argsh}{\mathop{\text{argsh}}\nolimits}
\newcommand{\Argch}{\mathop{\text{argch}}\nolimits}
\newcommand{\Argth}{\mathop{\text{argth}}\nolimits}
\newcommand{\pgcd}{\mathop{\text{pgcd}}\nolimits} 


%----- Commandes divers ------
\newcommand{\ii}{\mathrm{i}}
\newcommand{\dd}{\text{d}}
\newcommand{\id}{\mathop{\text{id}}\nolimits}
\newcommand{\Ker}{\mathop{\text{Ker}}\nolimits}
\newcommand{\Card}{\mathop{\text{Card}}\nolimits}
\newcommand{\Vect}{\mathop{\text{Vect}}\nolimits}
\newcommand{\Mat}{\mathop{\text{Mat}}\nolimits}
\newcommand{\rg}{\mathop{\text{rg}}\nolimits}
\newcommand{\tr}{\mathop{\text{tr}}\nolimits}


%----- Structure des exercices ------

\newtheoremstyle{styleexo}% name
{2ex}% Space above
{3ex}% Space below
{}% Body font
{}% Indent amount 1
{\bfseries} % Theorem head font
{}% Punctuation after theorem head
{\newline}% Space after theorem head 2
{}% Theorem head spec (can be left empty, meaning ‘normal’)

%\theoremstyle{styleexo}
\newtheorem{exo}{Exercice}
\newtheorem{ind}{Indications}
\newtheorem{cor}{Correction}


\newcommand{\exercice}[1]{} \newcommand{\finexercice}{}
%\newcommand{\exercice}[1]{{\tiny\texttt{#1}}\vspace{-2ex}} % pour afficher le numero absolu, l'auteur...
\newcommand{\enonce}{\begin{exo}} \newcommand{\finenonce}{\end{exo}}
\newcommand{\indication}{\begin{ind}} \newcommand{\finindication}{\end{ind}}
\newcommand{\correction}{\begin{cor}} \newcommand{\fincorrection}{\end{cor}}

\newcommand{\noindication}{\stepcounter{ind}}
\newcommand{\nocorrection}{\stepcounter{cor}}

\newcommand{\fiche}[1]{} \newcommand{\finfiche}{}
\newcommand{\titre}[1]{\centerline{\large \bf #1}}
\newcommand{\addcommand}[1]{}
\newcommand{\video}[1]{}

% Marge
\newcommand{\mymargin}[1]{\marginpar{{\small #1}}}

\def\noqed{\renewcommand{\qedsymbol}{}}


%----- Presentation ------
\setlength{\parindent}{0cm}

%\newcommand{\ExoSept}{\href{http://exo7.emath.fr}{\textbf{\textsf{Exo7}}}}

\definecolor{myred}{rgb}{0.93,0.26,0}
\definecolor{myorange}{rgb}{0.97,0.58,0}
\definecolor{myyellow}{rgb}{1,0.86,0}

\newcommand{\LogoExoSept}[1]{  % input : echelle
{\usefont{U}{cmss}{bx}{n}
\begin{tikzpicture}[scale=0.1*#1,transform shape]
  \fill[color=myorange] (0,0)--(4,0)--(4,-4)--(0,-4)--cycle;
  \fill[color=myred] (0,0)--(0,3)--(-3,3)--(-3,0)--cycle;
  \fill[color=myyellow] (4,0)--(7,4)--(3,7)--(0,3)--cycle;
  \node[scale=5] at (3.5,3.5) {Exo7};
\end{tikzpicture}}
}


\newcommand{\debutmontitre}{
  \author{} \date{} 
  \thispagestyle{empty}
  \hspace*{-10ex}
  \begin{minipage}{\textwidth}
    \titlepage  
  \vspace*{-2.5cm}
  \begin{center}
    \LogoExoSept{2.5}
  \end{center}
  \end{minipage}

  \vspace*{-0cm}
  
  % Astuce pour que le background ne soit pas discrétisé lors de la conversion pdf -> png
\begin{tikzpicture}
        \fill[opacity=0,green!60!black] (0,0)--++(0,0)--++(0,0)--++(0,0)--cycle; 
\end{tikzpicture}

% toc S'affiche trop tot :
% \tableofcontents[hideallsubsections, pausesections]
}

\newcommand{\finmontitre}{
  \end{frame}
  \setcounter{framenumber}{0}
} % ne marche pas pour une raison obscure

%----- Commandes supplementaires ------

% \usepackage[landscape]{geometry}
% \geometry{top=1cm, bottom=3cm, left=2cm, right=10cm, marginparsep=1cm
% }
% \usepackage[a4paper]{geometry}
% \geometry{top=2cm, bottom=2cm, left=2cm, right=2cm, marginparsep=1cm
% }

%\usepackage{standalone}


% New command Arnaud -- november 2011
\setbeamersize{text margin left=24ex}
% si vous modifier cette valeur il faut aussi
% modifier le decalage du titre pour compenser
% (ex : ici =+10ex, titre =-5ex

\theoremstyle{definition}
%\newtheorem{proposition}{Proposition}
%\newtheorem{exemple}{Exemple}
%\newtheorem{theoreme}{Théorème}
%\newtheorem{lemme}{Lemme}
%\newtheorem{corollaire}{Corollaire}
%\newtheorem*{remarque*}{Remarque}
%\newtheorem*{miniexercice}{Mini-exercices}
%\newtheorem{definition}{Définition}

% Commande tikz
\usetikzlibrary{calc}
\usetikzlibrary{patterns,arrows}
\usetikzlibrary{matrix}
\usetikzlibrary{fadings} 

%definition d'un terme
\newcommand{\defi}[1]{{\color{myorange}\textbf{\emph{#1}}}}
\newcommand{\evidence}[1]{{\color{blue}\textbf{\emph{#1}}}}
\newcommand{\assertion}[1]{\emph{\og#1\fg}}  % pour chapitre logique
%\renewcommand{\contentsname}{Sommaire}
\renewcommand{\contentsname}{}
\setcounter{tocdepth}{2}



%------ Figures ------

\def\myscale{1} % par défaut 
\newcommand{\myfigure}[2]{  % entrée : echelle, fichier figure
\def\myscale{#1}
\begin{center}
\footnotesize
{#2}
\end{center}}


%------ Encadrement ------

\usepackage{fancybox}


\newcommand{\mybox}[1]{
\setlength{\fboxsep}{7pt}
\begin{center}
\shadowbox{#1}
\end{center}}

\newcommand{\myboxinline}[1]{
\setlength{\fboxsep}{5pt}
\raisebox{-10pt}{
\shadowbox{#1}
}
}

%--------------- Commande beamer---------------
\newcommand{\beameronly}[1]{#1} % permet de mettre des pause dans beamer pas dans poly


\setbeamertemplate{navigation symbols}{}
\setbeamertemplate{footline}  % tiré du fichier beamerouterinfolines.sty
{
  \leavevmode%
  \hbox{%
  \begin{beamercolorbox}[wd=.333333\paperwidth,ht=2.25ex,dp=1ex,center]{author in head/foot}%
    % \usebeamerfont{author in head/foot}\insertshortauthor%~~(\insertshortinstitute)
    \usebeamerfont{section in head/foot}{\bf\insertshorttitle}
  \end{beamercolorbox}%
  \begin{beamercolorbox}[wd=.333333\paperwidth,ht=2.25ex,dp=1ex,center]{title in head/foot}%
    \usebeamerfont{section in head/foot}{\bf\insertsectionhead}
  \end{beamercolorbox}%
  \begin{beamercolorbox}[wd=.333333\paperwidth,ht=2.25ex,dp=1ex,right]{date in head/foot}%
    % \usebeamerfont{date in head/foot}\insertshortdate{}\hspace*{2em}
    \insertframenumber{} / \inserttotalframenumber\hspace*{2ex} 
  \end{beamercolorbox}}%
  \vskip0pt%
}


\definecolor{mygrey}{rgb}{0.5,0.5,0.5}
\setlength{\parindent}{0cm}
%\DeclareTextFontCommand{\helvetica}{\fontfamily{phv}\selectfont}

% background beamer
\definecolor{couleurhaut}{rgb}{0.85,0.9,1}  % creme
\definecolor{couleurmilieu}{rgb}{1,1,1}  % vert pale
\definecolor{couleurbas}{rgb}{0.85,0.9,1}  % blanc
\setbeamertemplate{background canvas}[vertical shading]%
[top=couleurhaut,middle=couleurmilieu,midpoint=0.4,bottom=couleurbas] 
%[top=fondtitre!05,bottom=fondtitre!60]



\makeatletter
\setbeamertemplate{theorem begin}
{%
  \begin{\inserttheoremblockenv}
  {%
    \inserttheoremheadfont
    \inserttheoremname
    \inserttheoremnumber
    \ifx\inserttheoremaddition\@empty\else\ (\inserttheoremaddition)\fi%
    \inserttheorempunctuation
  }%
}
\setbeamertemplate{theorem end}{\end{\inserttheoremblockenv}}

\newenvironment{theoreme}[1][]{%
   \setbeamercolor{block title}{fg=structure,bg=structure!40}
   \setbeamercolor{block body}{fg=black,bg=structure!10}
   \begin{block}{{\bf Th\'eor\`eme }#1}
}{%
   \end{block}%
}


\newenvironment{proposition}[1][]{%
   \setbeamercolor{block title}{fg=structure,bg=structure!40}
   \setbeamercolor{block body}{fg=black,bg=structure!10}
   \begin{block}{{\bf Proposition }#1}
}{%
   \end{block}%
}

\newenvironment{corollaire}[1][]{%
   \setbeamercolor{block title}{fg=structure,bg=structure!40}
   \setbeamercolor{block body}{fg=black,bg=structure!10}
   \begin{block}{{\bf Corollaire }#1}
}{%
   \end{block}%
}

\newenvironment{mydefinition}[1][]{%
   \setbeamercolor{block title}{fg=structure,bg=structure!40}
   \setbeamercolor{block body}{fg=black,bg=structure!10}
   \begin{block}{{\bf Définition} #1}
}{%
   \end{block}%
}

\newenvironment{lemme}[0]{%
   \setbeamercolor{block title}{fg=structure,bg=structure!40}
   \setbeamercolor{block body}{fg=black,bg=structure!10}
   \begin{block}{\bf Lemme}
}{%
   \end{block}%
}

\newenvironment{remarque}[1][]{%
   \setbeamercolor{block title}{fg=black,bg=structure!20}
   \setbeamercolor{block body}{fg=black,bg=structure!5}
   \begin{block}{Remarque #1}
}{%
   \end{block}%
}


\newenvironment{exemple}[1][]{%
   \setbeamercolor{block title}{fg=black,bg=structure!20}
   \setbeamercolor{block body}{fg=black,bg=structure!5}
   \begin{block}{{\bf Exemple }#1}
}{%
   \end{block}%
}


\newenvironment{miniexercice}[0]{%
   \setbeamercolor{block title}{fg=structure,bg=structure!20}
   \setbeamercolor{block body}{fg=black,bg=structure!5}
   \begin{block}{Mini-exercices}
}{%
   \end{block}%
}


\newenvironment{tp}[0]{%
   \setbeamercolor{block title}{fg=structure,bg=structure!40}
   \setbeamercolor{block body}{fg=black,bg=structure!10}
   \begin{block}{\bf Travaux pratiques}
}{%
   \end{block}%
}
\newenvironment{exercicecours}[1][]{%
   \setbeamercolor{block title}{fg=structure,bg=structure!40}
   \setbeamercolor{block body}{fg=black,bg=structure!10}
   \begin{block}{{\bf Exercice }#1}
}{%
   \end{block}%
}
\newenvironment{algo}[1][]{%
   \setbeamercolor{block title}{fg=structure,bg=structure!40}
   \setbeamercolor{block body}{fg=black,bg=structure!10}
   \begin{block}{{\bf Algorithme}\hfill{\color{gray}\texttt{#1}}}
}{%
   \end{block}%
}


\setbeamertemplate{proof begin}{
   \setbeamercolor{block title}{fg=black,bg=structure!20}
   \setbeamercolor{block body}{fg=black,bg=structure!5}
   \begin{block}{{\footnotesize Démonstration}}
   \footnotesize
   \smallskip}
\setbeamertemplate{proof end}{%
   \end{block}}
\setbeamertemplate{qed symbol}{\openbox}


\makeatother
\usecolortheme[RGB={0,199,174}]{structure}

%%%%%%%%%%%%%%%%%%%%%%%%%%%%%%%%%%%%%%%%%%%%%%%%%%%%%%%%%%%%%
%%%%%%%%%%%%%%%%%%%%%%%%%%%%%%%%%%%%%%%%%%%%%%%%%%%%%%%%%%%%%



\begin{document}



\title{{\bf Dérivée d'une fonction}}
\subtitle{Calcul des dérivées}

\begin{frame}
  
  \debutmontitre

  \pause

{\footnotesize
\hfill
\setbeamercovered{transparent=50}
\begin{minipage}{0.6\textwidth}
  \begin{itemize}
    \item<3-> Somme, produit,...
    \item<4-> Dérivée de fonctions usuelles
    \item<5-> Composition
    \item<6-> Dérivées successives
  \end{itemize}
\end{minipage}
}

\end{frame}

\setcounter{framenumber}{0}


%%%%%%%%%%%%%%%%%%%%%%%%%%%%%%%%%%%%%%%%%%%%%%%%%%%%%%%%%%%%%%%%


\section*{Somme, produit,...}


\begin{frame}

\begin{proposition}
Soient $f,g : I \to \Rr$ deux fonctions dérivables sur $I$. Pour tout $x \in I$ :
\pause
\begin{itemize}[<+->]
\setlength{\itemsep}{7pt} 
  \item $(f+g)'(x) = f'(x)+g'(x)$
  \item $(\lambda f)'(x) = \lambda f'(x)$ où $\lambda$ est un réel fixé
  \item $(f \times g)'(x) = f'(x)g(x)+f(x)g'(x)$
  \item $\left(\frac{1}{f}\right)'(x)=-\frac{f'(x)}{f(x)^2}$ (si $f(x) \neq 0$)
  \item $\left(\frac{f}{g}\right)'(x)=\frac{f'(x)g(x)-f(x)g'(x)}{g(x)^2}$ (si $g(x) \neq 0$)
\end{itemize}  
\end{proposition}

\end{frame}



\begin{frame}

$$(f+g)'=f'+g',\quad  (\lambda f)' = \lambda f' \quad (f \times g)' = f'g+fg' $$
\pause
$$\left(\frac{1}{f}\right)'=-\frac{f'}{f^2} \pause\qquad \qquad \left(\frac{f}{g}\right)'=\frac{f'g-fg'}{g^2}$$
\pause

\begin{proof}
Prouvons par exemple $(f \times g)' = f'g+fg'$

\pause 

Fixons $x_0 \in I$

\pause 

$$\frac{f(x)g(x)-f(x_0)g(x_0)}{x-x_0} 
\pause 
=\frac{f(x)-f(x_0)}{x-x_0} g(x)+\frac{g(x)-g(x_0)}{x-x_0}f(x_0)$$
\pause
$$\hspace*{7em} \xrightarrow[x\to x_0]{} \quad  f'(x_0) \  g(x_0) \quad + \quad g'(x_0) \ f(x_0)$$

\pause 

Ainsi la fonction $f\times g$ est dérivable sur $I$ de dérivée $f'g+fg'$
\end{proof}

\end{frame}


%%%%%%%%%%%%%%%%%%%%%%%%%%%%%%%%%%%%%%%%%%%%%%%%%%%%%%%%%%%%%%%%


\section*{Dérivée de fonctions usuelles}


\begin{frame}
\begin{center}
%\noindent
\setlength{\arrayrulewidth}{0.05mm}
%\begin{tabular}{|l|l|l|} \hline
\begin{tabular}[t]{cc@{\vrule depth 1.2ex height 3ex width 0mm \ }} 
\textbf{Fonction}         & \textbf{Dérivée} \\ \hline
   $x^n$         & $nx^{n-1}$  \quad ($n \in \Zz$)   \\ \hline
   $\frac 1x$    & $-\frac{1}{x^2}$              \\ \hline
   $\sqrt{x}$    & $\frac12 \frac1{\sqrt{x}}$   \\ \hline
   $x^\alpha$   & $\alpha x^{\alpha-1}$  \quad ($\alpha\in\Rr$)  \\ \hline
   $e^x$         & $e^x$                        \\ \hline
   $\ln x$       & $\frac 1x$                   \\ \hline
   $\cos x$      & $-\sin x$                    \\ \hline
   $\sin x$      & $\cos x$                     \\ \hline
   $\tan x$      & $1+\tan^2 x = \frac{1}{\cos^2 x}$        \\ \hline
\end{tabular} 
\end{center}
\end{frame}



%%%%%%%%%%%%%%%%%%%%%%%%%%%%%%%%%%%%%%%%%%%%%%%%%%%%%%%%%%%%%%%%


\section*{Composition}


\begin{frame}

\begin{proposition}
Si $f$ est dérivable en $x$ et $g$ est dérivable en $f(x)$ alors $g\circ f$ est
dérivable en $x$ 
\pause 
\mybox{$\big( g \circ f \big)'(x) = g'\big( f(x) \big) \cdot f'(x)$}
\end{proposition}
\pause 

\begin{exemple}
Calculons la dérivée de $\ln(1+x^2)$
\pause 
\begin{itemize}
  \item $f(x)=1+x^2$ avec $f'(x) = 2x$
 \pause  
  \item $g(x)=\ln(x)$ avec $g'(x) = \frac 1x$
 \pause  
  \item $\ln(1+x^2)=g\circ f(x)$
 \pause  
  \item $\big( g \circ f \big)'(x) = g'\big( f(x) \big) \cdot f'(x) \pause = g'\big( 1+x^2 \big) \cdot 2x \pause = \frac{2x}{1+x^2}$
\end{itemize}
\end{exemple}

\end{frame}


\begin{frame}
\begin{center}
\begin{tabular}[t]{cc@{\vrule depth 1.2ex height 3ex width 0mm \ }} 
\textbf{Fonction}         & \textbf{Dérivée} \\ \hline
   $u^n$         & $nu'u^{n-1}$  \quad  ($n \in \Zz$)   \\ \hline
   $\frac 1u$    & $-\frac{u'}{u^2}$              \\ \hline
   $\sqrt{u}$    & $\frac12 \frac{u'}{\sqrt{u}}$   \\ \hline
   $u^\alpha$   & $\alpha u' u^{\alpha-1}$ \quad ($\alpha\in\Rr$)  \\ \hline
   $e^u$         & $u'e^u$                        \\ \hline
   $\ln u$       & $\frac {u'}{u}$                   \\ \hline
   $\cos u$      & $-u'\sin u$                    \\ \hline
   $\sin u$      & $u'\cos u$                     \\ \hline
   $\tan u$      & $u'(1+\tan^2 u) = \frac{u'}{\cos^2 u}$        \\ \hline
\end{tabular} 
\hfill
\end{center}
\end{frame}



\begin{frame}
\begin{remarque}
\begin{itemize}
\setlength{\itemsep}{7pt} 
  \item Les formules pour $x^n$, $\frac 1x$ $\sqrt x$ et $x^\alpha$ découlent de la dérivée de $e^x$
\pause
  \item Exemple : $x^\alpha \pause= e^{\alpha \ln x}$ 
\pause
\centerline{$\displaystyle \frac{d }{dx}(x^\alpha) \pause= \frac{d}{dx} (e^{\alpha \ln x}) 
\pause= \alpha \frac{1}{x} e^{\alpha \ln x} \pause=\alpha \frac 1x x^{\alpha} \pause=\alpha x^{\alpha-1}$}
\pause
  \item Pour dériver une fonction avec un exposant dépendant de $x$ il faut repasser à la forme exponentielle

\pause
  \item Exemple : $f(x)= 2^x \pause= e^{x\ln 2}$ 
\pause
\centerline{$f'(x)= \ln 2  \cdot e^{x\ln 2} \pause= \ln 2 \cdot 2^x$}
\end{itemize}

\end{remarque}

\end{frame}

\begin{frame}
Soit $I$ un intervalle ouvert. Soit $f : I \to J$ dérivable et bijective

 $f^{-1} : J \to I$ sa bijection réciproque

\pause

\begin{corollaire}
Si $f'$ ne s'annule pas sur $I$ alors $f^{-1}$ est dérivable

\pause

Pour tout $x \in J$ 
\mybox{$\displaystyle \big(f^{-1}\big)'(x)= \frac{1}{f'\big( f^{-1}(x) \big)} $}
\end{corollaire}

\pause

Notons $g=f^{-1}$
\pause
\vspace*{-2ex}
$$f\big( g(x) \big)  = x$$ 
\pause
\vspace*{-2ex}
$$\implies f'\big(g(x)\big) \cdot g'(x) = 1 $$
\pause
\vspace*{-2ex}
$$\implies \big(f^{-1}\big)'(x)= \frac{1}{f'\big( f^{-1}(x) \big)}$$
\end{frame}

\begin{frame}

\begin{exemple}
Soit $f : \Rr \to \Rr$ définie par $f(x)=x+\exp(x)$
\pause
\begin{enumerate}
  \item \'Etude de $f$
\pause
  \begin{itemize}
    \item $f$ est dérivable car $f$ est la somme de deux fonctions dérivables
\pause
    \item $f$ est strictement croissante
\pause    
    \item $f$ est une bijection car $\lim_{x\to\pm\infty} f(x)=\pm\infty$
\pause    
    \item $f'(x) = 1 + \exp(x)$ ne s'annule jamais
  \end{itemize}
\pause  
  \item \'Etude de $g=f^{-1}$
\pause
  \begin{itemize}[<+->]
    \item $f\big( g(x) \big)  = x$ 
    \item En dérivant $f'\big(g(x)\big) \cdot g'(x) = 1$
    \item $g'(x) = \frac{1}{f'\big( g(x) \big)} = \frac{1}{1+\exp\big( g(x) \big)}$
    \item $f\big( g(x)\big) \!=\!x$ alors $g(x)+\exp\big( g(x) \big)\!=\!x$ donc 
$\exp\big( g(x) \big)=x-g(x)$
    \item $g'(x) =  \frac{1}{1+x -g(x)}$
    \item $f(0)=1$ donc $g(1)=0$ et donc $g'(1)=\frac12$
    \item La tangente à  $f^{-1}$ en $x_0=1$ est $y=\frac12 (x-1)$
  \end{itemize}
\end{enumerate}

\end{exemple}

\end{frame}

\begin{frame}
\myfigure{1}{
\tikzinput{fig_derive13} 
}  
\end{frame}

%%%%%%%%%%%%%%%%%%%%%%%%%%%%%%%%%%%%%%%%%%%%%%%%%%%%%%%%%%%%%%%%


\section*{Dérivées successives}


\begin{frame}

\begin{itemize}
  \item Soit $f : I \to \Rr$ une fonction dérivable
\pause
  \item Si $f' : I \to \Rr$ est dérivable on note $f''=(f')'$ la \defi{dérivée seconde}
\pause
  \item $f^{(0)} = f, \quad f^{(1)} = f', \quad f^{(2)} = f'' \quad \text{ et } \quad f^{(n+1)} = \big(f^{(n)}\big)'$
\pause
  \item Si la \defi{dérivée $n$-ième} $f^{(n)}$ existe  on dit que $f$ est \defi{$n$ fois dérivable}
\end{itemize}
\pause
\begin{theoreme}[Formule de Leibniz]
\ 
\vspace*{-4ex}
\mybox{\small $\displaystyle \!\!\!\big( f \cdot g \big)^{(n)} =  f^{(n)} \cdot g + \binom{n}{1}\ f^{(n-1)}\cdot g^{(1)}
+ \cdots  \binom{n}{k} \ f^{(n-k)} \cdot g^{(k)}\cdots + f \cdot g^{(n)}\!\!$} 
\end{theoreme}
\pause
$$\big( f \cdot g \big)^{(n)} = \sum_{k=0}^n \binom{n}{k} \ f^{(n-k)} \cdot g^{(k)}$$

\pause
\begin{itemize}
  \item $n=1$ \quad $(f\cdot g)'= f' g + f g'$
\pause
  \item $n=2$ \quad $(f\cdot g)''= f''g + 2f' g' + fg''$
\end{itemize}  
\end{frame}


\begin{frame}
\begin{exemple}
\begin{itemize}[<+->]
\setlength{\itemsep}{7pt}
  \item Dérivées de $\exp(x) \cdot (x^2+1)$
  \item $f(x)=\exp x$ \quad $f'(x)=\exp x$ \quad  $f''(x)=\exp x$ \quad $f^{(k)}(x)=\exp x$
  \item $g(x)=x^2+1$ \quad $g'(x)=2x$ \quad $g''(x)=2$ \quad$g^{(k)}(x)=0$ ($k\ge 3$)
  \item Formule de Leibniz 
$\big( f \cdot g \big)^{(n)}(x) =  f^{(n)}(x) \cdot g(x) + \binom{n}{1}\ f^{(n-1)}(x)\cdot g^{(1)}(x)
+ \binom{n}{2}\  f^{(n-2)}(x)\cdot g^{(2)}(x) + \binom{n}{3}\ f^{(n-3)}(x)\cdot g^{(3)}(x) + \cdots $
  \item $\big( f \cdot g \big)^{(n)}(x) =  \exp(x) \cdot (x^2+1) + \binom{n}{1}\ \exp(x) \cdot 2x
+ \binom{n}{2}\  \exp(x) \cdot 2$
  \item $\big( f \cdot g \big)^{(n)}(x) =  \exp(x) \cdot \Big(x^2 + 2nx + \frac{n(n-1)}{2}+1  \Big)$
\end{itemize}
\end{exemple}
\end{frame}

%%%%%%%%%%%%%%%%%%%%%%%%%%%%%%%%%%%%%%%%%%%%%%%%%%%%%%%%%%%%%%%%
\section*{Mini-exercices}


\begin{frame}
\begin{miniexercice}
\begin{enumerate}
  \item Calculer les dérivées des fonctions suivantes :
$f_1(x) = x\ln x$, $f_2(x)=\sin \frac 1x$, $f_3(x)=\sqrt{1+\sqrt{1+x^2}}$, $f_4(x)= \big(\ln(\frac{1+x}{1-x})\big)^{\frac13}$,
$f_5(x) = x^x$, $f_6(x) = \arctan x + \arctan \frac 1x$.
  \item On note $\Delta(f)=\frac{f'}{f}$. Calculer $\Delta(f\times g)$.
  \item Soit $f : ]1, +\infty[ \to ]-1, +\infty[$ définie par $f(x)= x\ln (x) - x$.
Montrer que $f$ est une bijection. Notons $g=f^{-1}$. Calculer $g(0)$ et $g'(0)$.
  \item Calculer les dérivées successives de $f(x)=\ln(1+x)$.
  \item Calculer les dérivées successives de $f(x)=\ln(x) \cdot x^3$.
\end{enumerate}
\end{miniexercice}
\end{frame}


\end{document}