
%%%%%%%%%%%%%%%%%% PREAMBULE %%%%%%%%%%%%%%%%%%


\documentclass[12pt]{article}

\usepackage{amsfonts,amsmath,amssymb,amsthm}
\usepackage[utf8]{inputenc}
\usepackage[T1]{fontenc}
\usepackage[francais]{babel}


% packages
\usepackage{amsfonts,amsmath,amssymb,amsthm}
\usepackage[utf8]{inputenc}
\usepackage[T1]{fontenc}
%\usepackage{lmodern}

\usepackage[francais]{babel}
\usepackage{fancybox}
\usepackage{graphicx}

\usepackage{float}

%\usepackage[usenames, x11names]{xcolor}
\usepackage{tikz}
\usepackage{datetime}

\usepackage{mathptmx}
%\usepackage{fouriernc}
%\usepackage{newcent}
\usepackage[mathcal,mathbf]{euler}

%\usepackage{palatino}
%\usepackage{newcent}


% Commande spéciale prompteur

%\usepackage{mathptmx}
%\usepackage[mathcal,mathbf]{euler}
%\usepackage{mathpple,multido}

\usepackage[a4paper]{geometry}
\geometry{top=2cm, bottom=2cm, left=1cm, right=1cm, marginparsep=1cm}

\newcommand{\change}{{\color{red}\rule{\textwidth}{1mm}\\}}

\newcounter{mydiapo}

\newcommand{\diapo}{\newpage
\hfill {\normalsize  Diapo \themydiapo \quad \texttt{[\jobname]}} \\
\stepcounter{mydiapo}}


%%%%%%% COULEURS %%%%%%%%%%

% Pour blanc sur noir :
%\pagecolor[rgb]{0.5,0.5,0.5}
% \pagecolor[rgb]{0,0,0}
% \color[rgb]{1,1,1}



%\DeclareFixedFont{\myfont}{U}{cmss}{bx}{n}{18pt}
\newcommand{\debuttexte}{
%%%%%%%%%%%%% FONTES %%%%%%%%%%%%%
\renewcommand{\baselinestretch}{1.5}
\usefont{U}{cmss}{bx}{n}
\bfseries

% Taille normale : commenter le reste !
%Taille Arnaud
%\fontsize{19}{19}\selectfont

% Taille Barbara
%\fontsize{21}{22}\selectfont

%Taille François
\fontsize{25}{30}\selectfont

%Taille Pascal
%\fontsize{25}{30}\selectfont

%Taille Laura
%\fontsize{30}{35}\selectfont


%\myfont
%\usefont{U}{cmss}{bx}{n}

%\Huge
%\addtolength{\parskip}{\baselineskip}
}


% \usepackage{hyperref}
% \hypersetup{colorlinks=true, linkcolor=blue, urlcolor=blue,
% pdftitle={Exo7 - Exercices de mathématiques}, pdfauthor={Exo7}}


%section
% \usepackage{sectsty}
% \allsectionsfont{\bf}
%\sectionfont{\color{Tomato3}\upshape\selectfont}
%\subsectionfont{\color{Tomato4}\upshape\selectfont}

%----- Ensembles : entiers, reels, complexes -----
\newcommand{\Nn}{\mathbb{N}} \newcommand{\N}{\mathbb{N}}
\newcommand{\Zz}{\mathbb{Z}} \newcommand{\Z}{\mathbb{Z}}
\newcommand{\Qq}{\mathbb{Q}} \newcommand{\Q}{\mathbb{Q}}
\newcommand{\Rr}{\mathbb{R}} \newcommand{\R}{\mathbb{R}}
\newcommand{\Cc}{\mathbb{C}} 
\newcommand{\Kk}{\mathbb{K}} \newcommand{\K}{\mathbb{K}}

%----- Modifications de symboles -----
\renewcommand{\epsilon}{\varepsilon}
\renewcommand{\Re}{\mathop{\text{Re}}\nolimits}
\renewcommand{\Im}{\mathop{\text{Im}}\nolimits}
%\newcommand{\llbracket}{\left[\kern-0.15em\left[}
%\newcommand{\rrbracket}{\right]\kern-0.15em\right]}

\renewcommand{\ge}{\geqslant}
\renewcommand{\geq}{\geqslant}
\renewcommand{\le}{\leqslant}
\renewcommand{\leq}{\leqslant}

%----- Fonctions usuelles -----
\newcommand{\ch}{\mathop{\mathrm{ch}}\nolimits}
\newcommand{\sh}{\mathop{\mathrm{sh}}\nolimits}
\renewcommand{\tanh}{\mathop{\mathrm{th}}\nolimits}
\newcommand{\cotan}{\mathop{\mathrm{cotan}}\nolimits}
\newcommand{\Arcsin}{\mathop{\mathrm{Arcsin}}\nolimits}
\newcommand{\Arccos}{\mathop{\mathrm{Arccos}}\nolimits}
\newcommand{\Arctan}{\mathop{\mathrm{Arctan}}\nolimits}
\newcommand{\Argsh}{\mathop{\mathrm{Argsh}}\nolimits}
\newcommand{\Argch}{\mathop{\mathrm{Argch}}\nolimits}
\newcommand{\Argth}{\mathop{\mathrm{Argth}}\nolimits}
\newcommand{\pgcd}{\mathop{\mathrm{pgcd}}\nolimits} 

\newcommand{\Card}{\mathop{\text{Card}}\nolimits}
\newcommand{\Ker}{\mathop{\text{Ker}}\nolimits}
\newcommand{\id}{\mathop{\text{id}}\nolimits}
\newcommand{\ii}{\mathrm{i}}
\newcommand{\dd}{\mathrm{d}}
\newcommand{\Vect}{\mathop{\text{Vect}}\nolimits}
\newcommand{\Mat}{\mathop{\mathrm{Mat}}\nolimits}
\newcommand{\rg}{\mathop{\text{rg}}\nolimits}
\newcommand{\tr}{\mathop{\text{tr}}\nolimits}
\newcommand{\ppcm}{\mathop{\text{ppcm}}\nolimits}

%----- Structure des exercices ------

\newtheoremstyle{styleexo}% name
{2ex}% Space above
{3ex}% Space below
{}% Body font
{}% Indent amount 1
{\bfseries} % Theorem head font
{}% Punctuation after theorem head
{\newline}% Space after theorem head 2
{}% Theorem head spec (can be left empty, meaning ‘normal’)

%\theoremstyle{styleexo}
\newtheorem{exo}{Exercice}
\newtheorem{ind}{Indications}
\newtheorem{cor}{Correction}


\newcommand{\exercice}[1]{} \newcommand{\finexercice}{}
%\newcommand{\exercice}[1]{{\tiny\texttt{#1}}\vspace{-2ex}} % pour afficher le numero absolu, l'auteur...
\newcommand{\enonce}{\begin{exo}} \newcommand{\finenonce}{\end{exo}}
\newcommand{\indication}{\begin{ind}} \newcommand{\finindication}{\end{ind}}
\newcommand{\correction}{\begin{cor}} \newcommand{\fincorrection}{\end{cor}}

\newcommand{\noindication}{\stepcounter{ind}}
\newcommand{\nocorrection}{\stepcounter{cor}}

\newcommand{\fiche}[1]{} \newcommand{\finfiche}{}
\newcommand{\titre}[1]{\centerline{\large \bf #1}}
\newcommand{\addcommand}[1]{}
\newcommand{\video}[1]{}

% Marge
\newcommand{\mymargin}[1]{\marginpar{{\small #1}}}



%----- Presentation ------
\setlength{\parindent}{0cm}

%\newcommand{\ExoSept}{\href{http://exo7.emath.fr}{\textbf{\textsf{Exo7}}}}

\definecolor{myred}{rgb}{0.93,0.26,0}
\definecolor{myorange}{rgb}{0.97,0.58,0}
\definecolor{myyellow}{rgb}{1,0.86,0}

\newcommand{\LogoExoSept}[1]{  % input : echelle
{\usefont{U}{cmss}{bx}{n}
\begin{tikzpicture}[scale=0.1*#1,transform shape]
  \fill[color=myorange] (0,0)--(4,0)--(4,-4)--(0,-4)--cycle;
  \fill[color=myred] (0,0)--(0,3)--(-3,3)--(-3,0)--cycle;
  \fill[color=myyellow] (4,0)--(7,4)--(3,7)--(0,3)--cycle;
  \node[scale=5] at (3.5,3.5) {Exo7};
\end{tikzpicture}}
}



\theoremstyle{definition}
%\newtheorem{proposition}{Proposition}
%\newtheorem{exemple}{Exemple}
%\newtheorem{theoreme}{Théorème}
\newtheorem{lemme}{Lemme}
\newtheorem{corollaire}{Corollaire}
%\newtheorem*{remarque*}{Remarque}
%\newtheorem*{miniexercice}{Mini-exercices}
%\newtheorem{definition}{Définition}




%definition d'un terme
\newcommand{\defi}[1]{{\color{myorange}\textbf{\emph{#1}}}}
\newcommand{\evidence}[1]{{\color{blue}\textbf{\emph{#1}}}}



 %----- Commandes divers ------

\newcommand{\codeinline}[1]{\texttt{#1}}

%%%%%%%%%%%%%%%%%%%%%%%%%%%%%%%%%%%%%%%%%%%%%%%%%%%%%%%%%%%%%
%%%%%%%%%%%%%%%%%%%%%%%%%%%%%%%%%%%%%%%%%%%%%%%%%%%%%%%%%%%%%



\begin{document}

\debuttexte

%%%%%%%%%%%%%%%%%%%%%%%%%%%%%%%%%%%%%%%%%%%%%%%%%%%%%%%%%%
\diapo

\change

Lorsqu'on a un problème de trigonométrie il est fréquent d'avoir 
à trouver un angle $\theta$ tel que, par exemple, $\cos \theta = \frac12$.

ici $\theta= \frac\pi3$ convient.

Mais si on doit trouver un angle tel que $\cos \theta = \frac13$ alors
l'angle n'est pas une valeur simple.


\change

C'est la fonction $\arccos$ qui va nous permettre de trouver les angles $\theta$ qui conviennent.

\change

On fera de même avec la fonction sinus et sa bijection réciproque arcsinus.

\change

Et aussi la fonction tangente et sa bijection réciproque arctangente.


%%%%%%%%%%%%%%%%%%%%%%%%%%%%%%%%%%%%%%%%%%%%%%%%%%%%%%%%%%
\diapo

La fonction cosinus, qui à $x$ associe $\cos(x)$, 
est définie sur $\Rr$ tout entier et est à valeurs dans $[-1,1]$.

Ce n'est pas une bijection car, par exemple, son
graphe recoupe une infinité de fois l'axe des abscisses.

\change

Pour obtenir une bijection à partir de cette fonction, il faut considérer la restriction
de cosinus à l'intervalle $[0,\pi]$. 


\change

Sur cet intervalle la fonction cosinus, ici en rouge, est continue
et strictement décroissante.

\change

Donc cette restriction
$$\cos_| : [0,\pi] \to [-1,1]$$
est bien une bijection.

\change

Cette restriction admet donc une bijection réciproque 
qui par définition  est la fonction \defi{arccosinus} :

Arccosinus est une fonction qui va de l'intervalle $[-1,1]$ vers l'intervalle $[0,\pi]$.

\change


Voici le graphe de $\arccos$.

Comme $\cos(0)=1$ alors $\arccos 1 = 0$,

comme $\cos(\pi/2)=0$ alors $\arccos 0 = \pi/2$,

comme $\cos(\pi)=-1$ alors $\arccos -1 = \pi$.

Et plus généralement le graphe de $\arccos$ est le symétrique du graphe de notre restriction de cosinus
par rapport à la droite d'équation $(y=x)$.

Arccos est une fonction continue et strictement décroissante.

Noter qu'en $-1$ et $+1$ nous avons deux "tangentes" [[guillemets avec les doigts]] verticales.

%%%%%%%%%%%%%%%%%%%%%%%%%%%%%%%%%%%%%%%%%%%%%%%%%%%%%%%%%%
\diapo


Par définition de ce qu'est une bijection réciproque :

nous avons la première identité : 
$\cos\big(\arccos(x)\big) = x$

ceci $\forall x \in [-1,1]$.


\change

Et aussi l'identité 
$\arccos\big(\cos(x)\big) = x$,

ceci pour $\forall x \in [0,\pi]$.

\change

Ce sont deux identité fondamentales.
Faites bien attention aux intervalles pour lesquelles elles sont valides.
Normalement on ne peut pas se tromper car nous avons ici arccos qui n'est définie
que pour $x \in [-1,1]$.

Et ici nous avons considéré la restriction de cosinus pour les  $x \in [0,\pi]$.

\change

Cela résout bien notre problème de trouver un angle correspondant à une valeur de cosinus donné :

On peux combiner les deux identités en une seule, qui est plutôt pratique

$\cos(x)=y \iff x = \arccos y$



%%%%%%%%%%%%%%%%%%%%%%%%%%%%%%%%%%%%%%%%%%%%%%%%%%%%%%%%%%
\diapo

Terminons l'étude de arccosinus avec le calcul de sa dérivée :


$\arccos'(x) = \frac{-1}{\sqrt{1-x^2}} \qquad \forall x \in ]-1,1[$

Notez que la fonction est dérivable sur l'intervalle *ouvert* $]-1,1[$
mais pas en $-1$ ni en $+1$.

\change

Voici la démonstration : on démarre de l'égalité $\cos(\arccos x) = x$ que l'on dérive :


\change

La formule de la dérivation d'une composition donne
$-\arccos'(x) \times  \sin(\arccos x) = 1$

où $\arccos'(x)$ est ce que l'on souhaite calculer.

\change

On a donc 

$\arccos'(x) = \frac{-1}{\sin(\arccos x)}$

\change


Mais comme nous allons le voir $\sin(\arccos x) = \sqrt{1-\cos^2(\arccos x)}$

\change

Mais $\cos(\arccos x)$ vaut par définition $x$
donc le dénominateur devient $\sqrt{1-x^2}$.

Et l'on obtient que $\arccos'(x) = \frac{-1}{\sqrt{1-x^2}}$



\change

Revenons sur le point crucial $(*)$ ; il se justifie ainsi :
on démarre de l'égalité $\cos^2 y + \sin^2 y = 1$,  

\change

en substituant $y= \arccos x$ on obtient $\cos^2(\arccos x)+ \sin^2(\arccos x) = 1$

\change

Mais encore une fois $\cos(\arccos x)=x$

donc $x^2+ \sin^2(\arccos x) = 1$. 

\change

On en déduit  $\sin(\arccos x) = + \sqrt{1-x^2}$, avec le signe $+$ car $\sin $ est positif sur $[0,\pi]$.


%%%%%%%%%%%%%%%%%%%%%%%%%%%%%%%%%%%%%%%%%%%%%%%%%%%%%%%%%%
\diapo

Je vous conseille d'arrêter la vidéo ici de faire vous même tout le travail que l'on vient
de faire mais cette fois pour la fonction sinus ! Cela fait un excellent exercice et
vous assimilerez plus vite le cours.

[[pause]]

\change

Reprenons pour voir si vous avez obtenu les bonnes formules.

Pour obtenir une bijection il faut restreindre la fonction sinus,
on choisit de la restreindre à l'intervalle $[-\frac\pi2,+\frac\pi2]$.

La restriction -dont le graphe est ici en rouge- 
$\sin_|$
est maintenant une fonction continue et strictement croissante,
c'est donc une bijection.

\change



Sa bijection réciproque est par définition la fonction \defi{arcsinus} :
c'est une fonction qui va de $[-1,1]$ vers $[-\frac\pi2,+\frac\pi2]$

C'est une fonction continue, strictement croissante.

Comme $\sin(-\pi/2)=-1$ alors $\arcsin(-1) = -\pi/2$,

comme $\sin(0)=0$ alors $\arcsin 0 = 0$,

comme $\sin(\pi/2)=+1$ alors $\arcsin (1) = \pi/2$.


%%%%%%%%%%%%%%%%%%%%%%%%%%%%%%%%%%%%%%%%%%%%%%%%%%%%%%%%%%
\diapo


L'arcsinus vérifie par définition l'identité

$\sin\big(\arcsin(x)\big) = x$, ceci $\forall x \in [-1,1]$

\change

et aussi $\arcsin\big(\sin(x)\big) = x$ cette fois $\forall x \in [-\frac\pi2,+\frac\pi2]$.

\change

\change

Autrement dit :


$\sin(x)=y \iff x = \arcsin y$


\change

Enfin la dérivée de arcsin vaut
$\frac{1}{\sqrt{1-x^2}}$,

Notez que c'est exactement la même que pour celle d'arccosinus mais avec ici un signe $+$.



%%%%%%%%%%%%%%%%%%%%%%%%%%%%%%%%%%%%%%%%%%%%%%%%%%%%%%%%%%
\diapo

La fonction tangente est une fonction périodique de période $\pi$.

On se limite à l'intervalle $]-\frac\pi2,+\frac\pi2[$, le graphe est en rouge ici.

Alors la restriction est une fonction continue et 
strictement croissante et définit donc une bijection de $]-\frac\pi2,+\frac\pi2[$ sur $\Rr$.


Sa bijection réciproque s'appelle la fonction \defi{arctangente} :


arctangente est définie sur $\Rr$ tout entier et prend ses valeurs dans l'intervalle $]-\frac\pi2,+\frac\pi2[$
ouvert.

Il y a deux asymptote horizontales : une en $-\infty$ 
où l'asymptote est donnée par $y=-\frac\pi2$ ;

elle correspond à l’asymptote verticale en $x=-\pi/2$.

une asymptote horizontale en $+\infty$ donnée par $y=+\frac\pi2$,
qui correspond  à l’asymptote verticale en $x=+\pi/2$.





%%%%%%%%%%%%%%%%%%%%%%%%%%%%%%%%%%%%%%%%%%%%%%%%%%%%%%%%%%
\diapo


Voici les formules à connaître pour l'arctangente :


$
\begin{array}{cl}
\tan\big(\arctan(x)\big) = x &  \forall x \in \Rr\\
 \arctan\big(\tan(x)\big) = x &  \forall x \in ]-\frac\pi2,+\frac\pi2[ \\
\end{array}
$


Ce qui se reformule en  
$\tan(x)=y \iff x = \arctan y$

Enfin la dérivée de arctan est très simple et apparaît très fréquemment :
$ \arctan'(x) = \frac{1}{1+x^2}$



%%%%%%%%%%%%%%%%%%%%%%%%%%%%%%%%%%%%%%%%%%%%%%%%%%%%%%%%%%
\diapo

Les fonctions trigonométriques inverses sont vraiment importante en mathématiques, 
et je vous encourage vivement à apprendre les formules et à vous entraîner.



\end{document}
