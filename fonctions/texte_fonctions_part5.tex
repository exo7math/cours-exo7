
%%%%%%%%%%%%%%%%%% PREAMBULE %%%%%%%%%%%%%%%%%%


\documentclass[12pt]{article}

\usepackage{amsfonts,amsmath,amssymb,amsthm}
\usepackage[utf8]{inputenc}
\usepackage[T1]{fontenc}
\usepackage[francais]{babel}


% packages
\usepackage{amsfonts,amsmath,amssymb,amsthm}
\usepackage[utf8]{inputenc}
\usepackage[T1]{fontenc}
%\usepackage{lmodern}

\usepackage[francais]{babel}
\usepackage{fancybox}
\usepackage{graphicx}

\usepackage{float}

%\usepackage[usenames, x11names]{xcolor}
\usepackage{tikz}
\usepackage{datetime}

\usepackage{mathptmx}
%\usepackage{fouriernc}
%\usepackage{newcent}
\usepackage[mathcal,mathbf]{euler}

%\usepackage{palatino}
%\usepackage{newcent}


% Commande spéciale prompteur

%\usepackage{mathptmx}
%\usepackage[mathcal,mathbf]{euler}
%\usepackage{mathpple,multido}

\usepackage[a4paper]{geometry}
\geometry{top=2cm, bottom=2cm, left=1cm, right=1cm, marginparsep=1cm}

\newcommand{\change}{{\color{red}\rule{\textwidth}{1mm}\\}}

\newcounter{mydiapo}

\newcommand{\diapo}{\newpage
\hfill {\normalsize  Diapo \themydiapo \quad \texttt{[\jobname]}} \\
\stepcounter{mydiapo}}


%%%%%%% COULEURS %%%%%%%%%%

% Pour blanc sur noir :
%\pagecolor[rgb]{0.5,0.5,0.5}
% \pagecolor[rgb]{0,0,0}
% \color[rgb]{1,1,1}



%\DeclareFixedFont{\myfont}{U}{cmss}{bx}{n}{18pt}
\newcommand{\debuttexte}{
%%%%%%%%%%%%% FONTES %%%%%%%%%%%%%
\renewcommand{\baselinestretch}{1.5}
\usefont{U}{cmss}{bx}{n}
\bfseries

% Taille normale : commenter le reste !
%Taille Arnaud
%\fontsize{19}{19}\selectfont

% Taille Barbara
%\fontsize{21}{22}\selectfont

%Taille François
\fontsize{25}{30}\selectfont

%Taille Pascal
%\fontsize{25}{30}\selectfont

%Taille Laura
%\fontsize{30}{35}\selectfont


%\myfont
%\usefont{U}{cmss}{bx}{n}

%\Huge
%\addtolength{\parskip}{\baselineskip}
}


% \usepackage{hyperref}
% \hypersetup{colorlinks=true, linkcolor=blue, urlcolor=blue,
% pdftitle={Exo7 - Exercices de mathématiques}, pdfauthor={Exo7}}


%section
% \usepackage{sectsty}
% \allsectionsfont{\bf}
%\sectionfont{\color{Tomato3}\upshape\selectfont}
%\subsectionfont{\color{Tomato4}\upshape\selectfont}

%----- Ensembles : entiers, reels, complexes -----
\newcommand{\Nn}{\mathbb{N}} \newcommand{\N}{\mathbb{N}}
\newcommand{\Zz}{\mathbb{Z}} \newcommand{\Z}{\mathbb{Z}}
\newcommand{\Qq}{\mathbb{Q}} \newcommand{\Q}{\mathbb{Q}}
\newcommand{\Rr}{\mathbb{R}} \newcommand{\R}{\mathbb{R}}
\newcommand{\Cc}{\mathbb{C}} 
\newcommand{\Kk}{\mathbb{K}} \newcommand{\K}{\mathbb{K}}

%----- Modifications de symboles -----
\renewcommand{\epsilon}{\varepsilon}
\renewcommand{\Re}{\mathop{\text{Re}}\nolimits}
\renewcommand{\Im}{\mathop{\text{Im}}\nolimits}
%\newcommand{\llbracket}{\left[\kern-0.15em\left[}
%\newcommand{\rrbracket}{\right]\kern-0.15em\right]}

\renewcommand{\ge}{\geqslant}
\renewcommand{\geq}{\geqslant}
\renewcommand{\le}{\leqslant}
\renewcommand{\leq}{\leqslant}

%----- Fonctions usuelles -----
\newcommand{\ch}{\mathop{\mathrm{ch}}\nolimits}
\newcommand{\sh}{\mathop{\mathrm{sh}}\nolimits}
\renewcommand{\tanh}{\mathop{\mathrm{th}}\nolimits}
\newcommand{\cotan}{\mathop{\mathrm{cotan}}\nolimits}
\newcommand{\Arcsin}{\mathop{\mathrm{Arcsin}}\nolimits}
\newcommand{\Arccos}{\mathop{\mathrm{Arccos}}\nolimits}
\newcommand{\Arctan}{\mathop{\mathrm{Arctan}}\nolimits}
\newcommand{\Argsh}{\mathop{\mathrm{Argsh}}\nolimits}
\newcommand{\Argch}{\mathop{\mathrm{Argch}}\nolimits}
\newcommand{\Argth}{\mathop{\mathrm{Argth}}\nolimits}
\newcommand{\pgcd}{\mathop{\mathrm{pgcd}}\nolimits} 

\newcommand{\Card}{\mathop{\text{Card}}\nolimits}
\newcommand{\Ker}{\mathop{\text{Ker}}\nolimits}
\newcommand{\id}{\mathop{\text{id}}\nolimits}
\newcommand{\ii}{\mathrm{i}}
\newcommand{\dd}{\mathrm{d}}
\newcommand{\Vect}{\mathop{\text{Vect}}\nolimits}
\newcommand{\Mat}{\mathop{\mathrm{Mat}}\nolimits}
\newcommand{\rg}{\mathop{\text{rg}}\nolimits}
\newcommand{\tr}{\mathop{\text{tr}}\nolimits}
\newcommand{\ppcm}{\mathop{\text{ppcm}}\nolimits}

%----- Structure des exercices ------

\newtheoremstyle{styleexo}% name
{2ex}% Space above
{3ex}% Space below
{}% Body font
{}% Indent amount 1
{\bfseries} % Theorem head font
{}% Punctuation after theorem head
{\newline}% Space after theorem head 2
{}% Theorem head spec (can be left empty, meaning ‘normal’)

%\theoremstyle{styleexo}
\newtheorem{exo}{Exercice}
\newtheorem{ind}{Indications}
\newtheorem{cor}{Correction}


\newcommand{\exercice}[1]{} \newcommand{\finexercice}{}
%\newcommand{\exercice}[1]{{\tiny\texttt{#1}}\vspace{-2ex}} % pour afficher le numero absolu, l'auteur...
\newcommand{\enonce}{\begin{exo}} \newcommand{\finenonce}{\end{exo}}
\newcommand{\indication}{\begin{ind}} \newcommand{\finindication}{\end{ind}}
\newcommand{\correction}{\begin{cor}} \newcommand{\fincorrection}{\end{cor}}

\newcommand{\noindication}{\stepcounter{ind}}
\newcommand{\nocorrection}{\stepcounter{cor}}

\newcommand{\fiche}[1]{} \newcommand{\finfiche}{}
\newcommand{\titre}[1]{\centerline{\large \bf #1}}
\newcommand{\addcommand}[1]{}
\newcommand{\video}[1]{}

% Marge
\newcommand{\mymargin}[1]{\marginpar{{\small #1}}}



%----- Presentation ------
\setlength{\parindent}{0cm}

%\newcommand{\ExoSept}{\href{http://exo7.emath.fr}{\textbf{\textsf{Exo7}}}}

\definecolor{myred}{rgb}{0.93,0.26,0}
\definecolor{myorange}{rgb}{0.97,0.58,0}
\definecolor{myyellow}{rgb}{1,0.86,0}

\newcommand{\LogoExoSept}[1]{  % input : echelle
{\usefont{U}{cmss}{bx}{n}
\begin{tikzpicture}[scale=0.1*#1,transform shape]
  \fill[color=myorange] (0,0)--(4,0)--(4,-4)--(0,-4)--cycle;
  \fill[color=myred] (0,0)--(0,3)--(-3,3)--(-3,0)--cycle;
  \fill[color=myyellow] (4,0)--(7,4)--(3,7)--(0,3)--cycle;
  \node[scale=5] at (3.5,3.5) {Exo7};
\end{tikzpicture}}
}



\theoremstyle{definition}
%\newtheorem{proposition}{Proposition}
%\newtheorem{exemple}{Exemple}
%\newtheorem{theoreme}{Théorème}
\newtheorem{lemme}{Lemme}
\newtheorem{corollaire}{Corollaire}
%\newtheorem*{remarque*}{Remarque}
%\newtheorem*{miniexercice}{Mini-exercices}
%\newtheorem{definition}{Définition}




%definition d'un terme
\newcommand{\defi}[1]{{\color{myorange}\textbf{\emph{#1}}}}
\newcommand{\evidence}[1]{{\color{blue}\textbf{\emph{#1}}}}



 %----- Commandes divers ------

\newcommand{\codeinline}[1]{\texttt{#1}}

%%%%%%%%%%%%%%%%%%%%%%%%%%%%%%%%%%%%%%%%%%%%%%%%%%%%%%%%%%%%%
%%%%%%%%%%%%%%%%%%%%%%%%%%%%%%%%%%%%%%%%%%%%%%%%%%%%%%%%%%%%%



\begin{document}

\debuttexte

%%%%%%%%%%%%%%%%%%%%%%%%%%%%%%%%%%%%%%%%%%%%%%%%%%%%%%%%%%
\diapo


\change

\change

Dans cette leçon nous rappelons le matériel nécessaire concernant les fonctions bijectives.

\change

Et nous allons étudier en détails le théorème de la bijection pour les fonctions continue 
et strictement monotones.


%%%%%%%%%%%%%%%%%%%%%%%%%%%%%%%%%%%%%%%%%%%%%%%%%%%%%%%%%%
\diapo

Je commence par des rappels nécessaires mais succincts sur les fonctions bijectives.


Soit $f$ une fonction qui va d'une partie de $\Rr$ dans une partie de $\Rr$.


\change

Par définition on dit que cette fonction est injective si
pour tous $x$ et $x'$ si $f(x)=f(x')$ alors $x=x'$.

[[Montrer dessin]]

Autrement dit tout élément de l'ensemble d'arrivée admet $1$ ou $0$ antécédent
comme sur ce dessin.


\change

Par définition $f$ est surjective si pour chaque $y$ de l'ensemble d'arrivée, il existe
$x$ dans l'ensemble de départ tel que $y=f(x)$.

[[Montrer dessin]]

Autrement dit tout élément de l'ensemble d'arrivée a au moins un antécédent, comme 
par exemple cette valeur $y$ qui a trois antécédents.



\change

Maintenant $f$ est dite bijective si $f$ est à la fois injective et surjective, 

ce qui se reformule par la phrase mathématique suivante :

pour tout $y$ de l'ensemble d'arrivée $F$ il existe *un unique*
$x$ de l'ensemble de départ $E$ tel que $y=f(x)$.

Cela signifie exactement que tout élément de l'ensemble d'arrivé admet un, et un seul, antécédent.

Voici le graphe d'une fonction bijective.

%%%%%%%%%%%%%%%%%%%%%%%%%%%%%%%%%%%%%%%%%%%%%%%%%%%%%%%%%%
\diapo

Voici la propriété fondamentale des fonctions bijectives : elles admettent une réciproque.


Si $f :  E \to F$ est une fonction bijective alors il existe une 
unique application $g : F \to E$ telle que $g\circ f = \id_E$ et $f\circ g = \id_F$.

La fonction $g$ s’appelle l'inverse de $f$ ou la bijection réciproque de $f$ 

et on la note $f^{-1}$.

\change


Quelques remarques pour mieux comprendre :

l'identité d'un ensemble dans lui même est simplement l'application qui à $x$ associe $x$.

\change

Donc dire que $g \circ f = \id_E$ signifie que $\forall x \in E\quad  g\big(f(x)\big) = x$.


\change

Et $f \circ g = \id_F$  signifie  $\forall y \in F\quad  f\big(g(y)\big) = y$.

\change

Dans un repère orthonormé les graphes des fonctions $f$ et $f^{-1}$ sont symétriques 
  par rapport à la droite d'équation $(y=x)$.
  
  
  
%%%%%%%%%%%%%%%%%%%%%%%%%%%%%%%%%%%%%%%%%%%%%%%%%%%%%%%%%%
\diapo

Voici le principal résultat de cette leçon : le théorème de la bijection, c'est un théorème 
puissant qui permet de montrer qu'une fonction est bijective.


Partons d'une fonction $f$ définie sur un intervalle $I$, et pour l'instant l'ensemble d'arrivée est $\Rr$ tout entier.

Voici les deux hypothèses :

(1) $f$ est une fonction continue sur l'intervalle $I$,

(2) $f$ est strictement croissante ou bien strictement décroissante sur cet intervalle

Voici le graphe d'une fonction vérifiant ces deux hypothèses.

\change

Passons maintenant aux conclusions.


tout d'abord si on considère comme ensemble d'arrivée, non pas $\Rr$ tout entier 
mais l'intervalle $J$ qui est l'image par $f$ de l'intervalle $I$. Alors la 
fonction $f$ qui va de $I$ dans $J$ est une bijection.


\change 

De plus la fonction réciproque $f^{-1}$ qui va donc de $J$ vers $I$ est elle aussi continue 
et elle aussi strictement monotone.

Enfin si $f$ était strictement croissante alors $f^{-1}$ est aussi strictement croissante.
Si $f$ était strictement décroissante alors $f^{-1}$ est aussi strictement décroissante.


\change

Sur le dessin notre fonction $f : I \to J$ est continue et strictement croissante,
et la fonction réciproque $f^{-1} : J \to I$ est aussi continue et strictement croissante.


%%%%%%%%%%%%%%%%%%%%%%%%%%%%%%%%%%%%%%%%%%%%%%%%%%%%%%%%%%%
\diapo

Passons à un exemple avec la fonction $f$ définie par $f(x)=x^2$.

Vue comme une fonction de $\Rr$ dans $\Rr$, $f$ n'est pas une fonction bijective (elle n'est ni injective, 
ni non plus surjective).


\change

On va donc considérer les deux restrictions suivantes :
\[
f_1 :
\left\{\begin{array}{c}
]-\infty,0] \longrightarrow [0,+\infty[ \\
x \longmapsto x^2
\end{array}\right.
\qquad \text{et } \qquad
f_2 : 
\left\{\begin{array}{c}
[0,+\infty[ \longrightarrow [0,+\infty[ \\
x \longmapsto x^2
\end{array}\right.
\]

\change

Voici le graphe de $f_1$ à gauche en vert  et celui de $f_2$ à droite en bleu.

Sur l'intervalle $]-\infty,0]$ $f_1$ est continue et strictement décroissante,
de plus l'image de l'intervalle $]-\infty,0]$ par  $f_1$ est l'intervalle $[0,+\infty[$.

Par le théorème précédent $f_1$ est une fonction bijective.

De même $f_2$ est continue et strictement croissante, c'est donc aussi une bijection 
de $[0,+\infty[$ dans $[0,+\infty[$.


\change

Voyons quelles sont les bijections réciproques.

Pour $f_1^{-1}$ c'est la fonction qui a $y$ associe $-\sqrt{y}$

et pour $f_2^{-1}$ c'est la fonction qui a $y$ associe $+\sqrt{y}$.

\change

Comme d'habitude le graphe de $f_1^{-1}$, s'obtient par symétrie par rapport à la droite $(y=x)$.
$f_1^{-1}$ est une fonction continue et strictement décroissante, comme $f_1$.

\change

Quant à $f_2^{-1}$ c'est une fonction continue et strictement croissante. On retrouve bien que chacune 
des deux fonctions $f_1$ et $f_2$ a le même sens de variation que sa réciproque. 



On remarque que la courbe totale en pointillée (à la fois la partie bleue et la verte), 
qui est l'image du graphe de $f$ par la symétrie 
par rapport à la première bissectrice, ne peut pas être le graphe d'une fonction : 
c'est une autre manière de voir que $f$ n'est pas bijective.




%%%%%%%%%%%%%%%%%%%%%%%%%%%%%%%%%%%%%%%%%%%%%%%%%%%%%%%%%%
\diapo

Généralisons l'exemple précédent.

Fixons un entier  $n$. 

Et soit $f$ la fonction définie par $f(x)=x^n$. Les intervalles de départ et d'arrivée sont $[0,+\infty[$.


\change

Alors $f$ est continue et est strictement croissante.

\change

Comme 
$\lim_{+\infty} f = +\infty$ et que $f(0)=0$ alors $f$ est une bijection
de $[0,+\infty[$ vers $[0,+\infty[$.

\change

Elle admet donc une bijection réciproque $f^{-1}$,
c'est la fonction racine $n$-ième. 
notée  $x^{\frac{1}{n}}$ ou quelque fois $\sqrt[n]{x}$.

\change

Par le théorème de la bijection la fonction racine $n$-ième est 
continue et strictement croissante.


%%%%%%%%%%%%%%%%%%%%%%%%%%%%%%%%%%%%%%%%%%%%%%%%%%%%%%%%%%
\diapo

Nous allons voir deux ingrédients essentiels de la démontration du théorème de la bijection.

On établit d'abord le lemme suivant :

"Soit $f$ une fonction définie sur un intervalle $I$ de $\Rr$. Si $f$ est 
strictement monotone, alors $f$ est injective."

Notez qu'ici la fonction n'a pas besoin d'être continue.


\change

La preuve n'est pas difficile.

Partons de deux réels $x,x'$ tels que $f(x)=f(x')$. Nous devons montrer que $x=x'$. 

\change
Faisons un petit raisonnement par l'absurde :

Si on avait $x<x'$, 
alors on aurait nécessairement $f(x)<f(x')$ si $f$ était strictement croissante

ou bien $f(x)>f(x')$ si $f$ était strictement décroissante.

Mais comme c'est impossible par cette hypothèse [montrer] 

on en déduit $x$ n'est pas strictement inférieur à $x'$, donc $x\geq x'$.

\change

Si on change $x$ en $x'$ et $x'$ en $x$
on montrerait de même que $x\leq x'$. 


\change

Ainsi  :  $x=x'$.

En conclusion : si $f(x)=f(x')$ alors $x=x'$ ;
nous avons exactement prouver que $f$ est injective.



%%%%%%%%%%%%%%%%%%%%%%%%%%%%%%%%%%%%%%%%%%%%%%%%%%%%%%%%%%
\diapo


Voici un autre résultat intermédiaire : 

"Si $f$ est une fonction continue 
et strictement monotone sur un intervalle $I$ alors
$f$ établit une bijection de $I$ dans l'intervalle image $J=f(I)$"  

\change
D'après le lemme précédent, on sait déjà que $f$ est injective sur $I$.

\change

Mais comme on restreint l'ensemble d'arrivée à son image $J=f(I)$, on obtient que 
$f$ est en plus surjective. 


\change

$f$ est injective et surjective, donc c'est une bijection de $I$ sur $J$.

\change

Mais comme en plus $f$ est continue,
alors par le théorème des valeurs intermédiaires,
l'ensemble d'arrivée $J$ est un intervalle.

[pause]

Pour finir le reste de la preuve du théorème de la bijection, il reste
encore à montrer que $f^{-1}$ est continue et de même sens de variation que $f$.

Je vous encourage à étudier par vous-même le reste de la démonstration.



%%%%%%%%%%%%%%%%%%%%%%%%%%%%%%%%%%%%%%%%%%%%%%%%%%%%%%%%%%
\diapo

Il est temps pour vous de passer à la pratique !

\end{document}
