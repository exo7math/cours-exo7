
%%%%%%%%%%%%%%%%%% PREAMBULE %%%%%%%%%%%%%%%%%%


\documentclass[12pt]{article}

\usepackage{amsfonts,amsmath,amssymb,amsthm}
\usepackage[utf8]{inputenc}
\usepackage[T1]{fontenc}
\usepackage[francais]{babel}


% packages
\usepackage{amsfonts,amsmath,amssymb,amsthm}
\usepackage[utf8]{inputenc}
\usepackage[T1]{fontenc}
%\usepackage{lmodern}

\usepackage[francais]{babel}
\usepackage{fancybox}
\usepackage{graphicx}

\usepackage{float}

%\usepackage[usenames, x11names]{xcolor}
\usepackage{tikz}
\usepackage{datetime}

\usepackage{mathptmx}
%\usepackage{fouriernc}
%\usepackage{newcent}
\usepackage[mathcal,mathbf]{euler}

%\usepackage{palatino}
%\usepackage{newcent}


% Commande spéciale prompteur

%\usepackage{mathptmx}
%\usepackage[mathcal,mathbf]{euler}
%\usepackage{mathpple,multido}

\usepackage[a4paper]{geometry}
\geometry{top=2cm, bottom=2cm, left=1cm, right=1cm, marginparsep=1cm}

\newcommand{\change}{{\color{red}\rule{\textwidth}{1mm}\\}}

\newcounter{mydiapo}

\newcommand{\diapo}{\newpage
\hfill {\normalsize  Diapo \themydiapo \quad \texttt{[\jobname]}} \\
\stepcounter{mydiapo}}


%%%%%%% COULEURS %%%%%%%%%%

% Pour blanc sur noir :
%\pagecolor[rgb]{0.5,0.5,0.5}
% \pagecolor[rgb]{0,0,0}
% \color[rgb]{1,1,1}



%\DeclareFixedFont{\myfont}{U}{cmss}{bx}{n}{18pt}
\newcommand{\debuttexte}{
%%%%%%%%%%%%% FONTES %%%%%%%%%%%%%
\renewcommand{\baselinestretch}{1.5}
\usefont{U}{cmss}{bx}{n}
\bfseries

% Taille normale : commenter le reste !
%Taille Arnaud
%\fontsize{19}{19}\selectfont

% Taille Barbara
%\fontsize{21}{22}\selectfont

%Taille François
\fontsize{25}{30}\selectfont

%Taille Pascal
%\fontsize{25}{30}\selectfont

%Taille Laura
%\fontsize{30}{35}\selectfont


%\myfont
%\usefont{U}{cmss}{bx}{n}

%\Huge
%\addtolength{\parskip}{\baselineskip}
}


% \usepackage{hyperref}
% \hypersetup{colorlinks=true, linkcolor=blue, urlcolor=blue,
% pdftitle={Exo7 - Exercices de mathématiques}, pdfauthor={Exo7}}


%section
% \usepackage{sectsty}
% \allsectionsfont{\bf}
%\sectionfont{\color{Tomato3}\upshape\selectfont}
%\subsectionfont{\color{Tomato4}\upshape\selectfont}

%----- Ensembles : entiers, reels, complexes -----
\newcommand{\Nn}{\mathbb{N}} \newcommand{\N}{\mathbb{N}}
\newcommand{\Zz}{\mathbb{Z}} \newcommand{\Z}{\mathbb{Z}}
\newcommand{\Qq}{\mathbb{Q}} \newcommand{\Q}{\mathbb{Q}}
\newcommand{\Rr}{\mathbb{R}} \newcommand{\R}{\mathbb{R}}
\newcommand{\Cc}{\mathbb{C}} 
\newcommand{\Kk}{\mathbb{K}} \newcommand{\K}{\mathbb{K}}

%----- Modifications de symboles -----
\renewcommand{\epsilon}{\varepsilon}
\renewcommand{\Re}{\mathop{\text{Re}}\nolimits}
\renewcommand{\Im}{\mathop{\text{Im}}\nolimits}
%\newcommand{\llbracket}{\left[\kern-0.15em\left[}
%\newcommand{\rrbracket}{\right]\kern-0.15em\right]}

\renewcommand{\ge}{\geqslant}
\renewcommand{\geq}{\geqslant}
\renewcommand{\le}{\leqslant}
\renewcommand{\leq}{\leqslant}

%----- Fonctions usuelles -----
\newcommand{\ch}{\mathop{\mathrm{ch}}\nolimits}
\newcommand{\sh}{\mathop{\mathrm{sh}}\nolimits}
\renewcommand{\tanh}{\mathop{\mathrm{th}}\nolimits}
\newcommand{\cotan}{\mathop{\mathrm{cotan}}\nolimits}
\newcommand{\Arcsin}{\mathop{\mathrm{Arcsin}}\nolimits}
\newcommand{\Arccos}{\mathop{\mathrm{Arccos}}\nolimits}
\newcommand{\Arctan}{\mathop{\mathrm{Arctan}}\nolimits}
\newcommand{\Argsh}{\mathop{\mathrm{Argsh}}\nolimits}
\newcommand{\Argch}{\mathop{\mathrm{Argch}}\nolimits}
\newcommand{\Argth}{\mathop{\mathrm{Argth}}\nolimits}
\newcommand{\pgcd}{\mathop{\mathrm{pgcd}}\nolimits} 

\newcommand{\Card}{\mathop{\text{Card}}\nolimits}
\newcommand{\Ker}{\mathop{\text{Ker}}\nolimits}
\newcommand{\id}{\mathop{\text{id}}\nolimits}
\newcommand{\ii}{\mathrm{i}}
\newcommand{\dd}{\mathrm{d}}
\newcommand{\Vect}{\mathop{\text{Vect}}\nolimits}
\newcommand{\Mat}{\mathop{\mathrm{Mat}}\nolimits}
\newcommand{\rg}{\mathop{\text{rg}}\nolimits}
\newcommand{\tr}{\mathop{\text{tr}}\nolimits}
\newcommand{\ppcm}{\mathop{\text{ppcm}}\nolimits}

%----- Structure des exercices ------

\newtheoremstyle{styleexo}% name
{2ex}% Space above
{3ex}% Space below
{}% Body font
{}% Indent amount 1
{\bfseries} % Theorem head font
{}% Punctuation after theorem head
{\newline}% Space after theorem head 2
{}% Theorem head spec (can be left empty, meaning ‘normal’)

%\theoremstyle{styleexo}
\newtheorem{exo}{Exercice}
\newtheorem{ind}{Indications}
\newtheorem{cor}{Correction}


\newcommand{\exercice}[1]{} \newcommand{\finexercice}{}
%\newcommand{\exercice}[1]{{\tiny\texttt{#1}}\vspace{-2ex}} % pour afficher le numero absolu, l'auteur...
\newcommand{\enonce}{\begin{exo}} \newcommand{\finenonce}{\end{exo}}
\newcommand{\indication}{\begin{ind}} \newcommand{\finindication}{\end{ind}}
\newcommand{\correction}{\begin{cor}} \newcommand{\fincorrection}{\end{cor}}

\newcommand{\noindication}{\stepcounter{ind}}
\newcommand{\nocorrection}{\stepcounter{cor}}

\newcommand{\fiche}[1]{} \newcommand{\finfiche}{}
\newcommand{\titre}[1]{\centerline{\large \bf #1}}
\newcommand{\addcommand}[1]{}
\newcommand{\video}[1]{}

% Marge
\newcommand{\mymargin}[1]{\marginpar{{\small #1}}}



%----- Presentation ------
\setlength{\parindent}{0cm}

%\newcommand{\ExoSept}{\href{http://exo7.emath.fr}{\textbf{\textsf{Exo7}}}}

\definecolor{myred}{rgb}{0.93,0.26,0}
\definecolor{myorange}{rgb}{0.97,0.58,0}
\definecolor{myyellow}{rgb}{1,0.86,0}

\newcommand{\LogoExoSept}[1]{  % input : echelle
{\usefont{U}{cmss}{bx}{n}
\begin{tikzpicture}[scale=0.1*#1,transform shape]
  \fill[color=myorange] (0,0)--(4,0)--(4,-4)--(0,-4)--cycle;
  \fill[color=myred] (0,0)--(0,3)--(-3,3)--(-3,0)--cycle;
  \fill[color=myyellow] (4,0)--(7,4)--(3,7)--(0,3)--cycle;
  \node[scale=5] at (3.5,3.5) {Exo7};
\end{tikzpicture}}
}



\theoremstyle{definition}
%\newtheorem{proposition}{Proposition}
%\newtheorem{exemple}{Exemple}
%\newtheorem{theoreme}{Théorème}
\newtheorem{lemme}{Lemme}
\newtheorem{corollaire}{Corollaire}
%\newtheorem*{remarque*}{Remarque}
%\newtheorem*{miniexercice}{Mini-exercices}
%\newtheorem{definition}{Définition}




%definition d'un terme
\newcommand{\defi}[1]{{\color{myorange}\textbf{\emph{#1}}}}
\newcommand{\evidence}[1]{{\color{blue}\textbf{\emph{#1}}}}



 %----- Commandes divers ------

\newcommand{\codeinline}[1]{\texttt{#1}}

%%%%%%%%%%%%%%%%%%%%%%%%%%%%%%%%%%%%%%%%%%%%%%%%%%%%%%%%%%%%%
%%%%%%%%%%%%%%%%%%%%%%%%%%%%%%%%%%%%%%%%%%%%%%%%%%%%%%%%%%%%%



\begin{document}

\debuttexte


%%%%%%%%%%%%%%%%%%%%%%%%%%%%%%%%%%%%%%%%%%%%%%%%%%%%%%%%%%%
\diapo

\change
Ce chapitre sur les limites et les 
fonctions continues est le premier d'une série de chapitres consacrés à l'étude des 
fonctions réelles. Nous commençons par une leçon de définitions.

\change
Nous allons tout d'abord définir ce qu'est une telle fonction

\change
puis comment on peut additionner ou multiplier des fonctions entre elles. Nous verrons ensuite ce que signifie qu'une fonction est majorée ou minorée,

\change
qu'elle est croissante ou décroissante,

\change
et enfin, qu'elle est paire, impaire, ou périodique.

%%%%%%%%%%%%%%%%%%%%%%%%%%%%%%%%%%%%%%%%%%%%%%%%%%%%%%%%%%
\diapo

Voici une motivation pour ce chapitre.

Considérons l'équation x + exponentielle x = 0. Elle est très simple, et 
pourtant, il n'existe pas de formule connue qui permettrait d'exprimer 
la solution x à partir de fonctions usuelles, en faisant des sommes, des produits, des compositions...

\change
Dans ce chapitre nous allons voir que grâce à l'étude de la fonction x + exp x,
il est possible d'obtenir beaucoup d'informations sur la solution de cette l'équation,

\change
et même de l'équation plus générale x + exp x= y, où y est un nombre réel fixé.


\change
Nous serons capable de montrer que pour chaque réel y,  l'équation x+exp x = y  admet bien une solution x,

\change
que cette solution est unique,


\change
et nous saurons dire comment varie x en fonction de y. Pour cela, le point clé est l'étude de la fonction f, et en particulier de sa continuité. Même s'il n'est pas possible de trouver l'expression exacte de la solution $x$ en fonction de $y$, nous allons mettre en place les outils théoriques qui permettent d'en trouver une solution approchée.



%%%%%%%%%%%%%%%%%%%%%%%%%%%%%%%%%%%%%%%%%%%%%%%%%%%%%%%%%%
\diapo

Une fonction d'une variable réelle à valeurs réelles est une application $f$ d'une partie $U$ de $\Rr$ dans $\Rr$.
 
\change
On appelle $U$ le domaine de définition de la fonction $f$. En général, $U$ est un intervalle ou une réunion d'intervalles.

\change
Le graphe de la fonction $f$ est la partie $\Gamma_f$ de $\Rr^2$ constituée des couples $(x,f(x)) $ où $x$ décrit le domaine de définition de $f$.

\change
On voit ici le graphe d'une fonction $f$. A tout réel $x$ de l'ensemble de définition, il correspond une unique image par $f$, et donc un unique point $(x,f(x)) $ du graphe.



%%%%%%%%%%%%%%%%%%%%%%%%%%%%%%%%%%%%%%%%%%%%%%%%%%%%%%%%%%
\diapo

Considérons par exemple la fonction inverse, qui à un réel 
non nul $x$ associe le réel $1/x$, cette fonction est définie 
sur la réunion d'intervalles ouverts $ ]-\infty,0[ \,\cup \, ]0,+\infty[$

\change
et voici le graphe.


%%%%%%%%%%%%%%%%%%%%%%%%%%%%%%%%%%%%%%%%%%%%%%%%%%%%%%%%%%
\diapo

Soient deux fonctions $f$ et $g$ définies sur une même partie $U$ de $\Rr$. On va définir à partir de $f$ et $g$ de nouvelles fonctions par les opérations suivantes. Tout d'abord la somme de $f$ et $g$ est la fonction définie sur $U$ par $(f+g)(x) = f(x) + g(x)$.

\change
On voit ici le graphe de $f$ en vert, celui de  $g$ en bleu et leur somme en rouge, 
la fonction $f+g$ est obtenue en sommant point par point les valeurs de $f$ et $g$.

\change
Le produit de $f$ et $g$ est la fonction $f\times g$ définie sur $U$ par $(f\times g)(x) = f(x) \times g(x)$.
  
\change
Enfin, on définit de même la multiplication de $f$ par un scalaire réel $\lambda$ : c'est la fonction $\lambda\cdot f$ définie sur $U$ par $(\lambda\cdot f)(x) =\lambda\cdot f(x)$.
 

%%%%%%%%%%%%%%%%%%%%%%%%%%%%%%%%%%%%%%%%%%%%%%%%%%%%%%%%%%
\diapo

[petit m, grand M]

Une fonction $f$ de $U$ dans $\Rr$ est dite majorée sur $U$ 
s'il existe un réel $M$ tel que pour *tous* les $x$ de $U$ on a $f(x)\leq M$ .
 
\change
$f$ est dite minorée s'il existe un réel $m$ tel que $ \forall x\in U \ f(x)\geq m$.

\change
Enfin, on dira que $f$ est bornée sur $U$ si $f$ est à la fois majorée et 
minorée sur $U$, ce qui est équivalent à dire :
$\exists M\in\Rr \ \forall x\in U \ |f(x)|\leq M$.

\change
Voici un exemple de fonction qui est bornée sur $\Rr$. On a représenté 
un minorant petit $m$ qui est même atteint
et un majorant $M$ qui lui n'est pas atteint.


%%%%%%%%%%%%%%%%%%%%%%%%%%%%%%%%%%%%%%%%%%%%%%%%%%%%%%%%%%
\diapo
Poursuivons cette série de définitions par les notions de fonctions croissantes et décroissantes. Une fonction $f$ est dite croissante si pour tous $x, y$ de son ensemble de définition, $x\leq y \implies f(x)\leq f(y)$.
 
\change
On voit ici le graphe d'une fonction croissante : si $x$ est plus petit que $y$ alors $f(x)$ est plus petit que $f(y)$, $f$ préserve la relation d'ordre.

\change
$f$ sera dite strictement croissante sur $U$ si $x< y \implies f(x)< f(y)$.

[montrer dessin] 
Cette fonction croissante est aussi strictement croissante.


%%%%%%%%%%%%%%%%%%%%%%%%%%%%%%%%%%%%%%%%%%%%%%%%%%%%%%%%%%
\diapo
On définit de même ce qu'est une fonction décroissante et strictement décroissante en renversant dans les définitions précédentes le sens de la deuxième inégalité : une fonction est décroissante si pour tous $x,y$ \ $x\leq y \implies f(x)\geq f(y)$. On obtient une fonction strictement décroissante en remplaçant les inégalités larges par des inégalités strictes.
 
\change
Enfin, $f$ est dite monotone sur $U$ si elle est croissante ou décroissante sur $U$, et elle est dite strictement monotone sur $U$ si elle est strictement croissante ou strictement décroissante.


%%%%%%%%%%%%%%%%%%%%%%%%%%%%%%%%%%%%%%%%%%%%%%%%%%%%%%%%%%
\diapo
Voyons à présent quelques exemples. La fonction racine carrée est strictement croissante sur son ensemble de définition, l'intervalle $[0,+\infty[$.
 
\change
Les fonctions exponentielle et logarithme népérien sont également strictement croissantes.

\change
La fonction valeur absolue n'est pas monotone sur $\Rr$, puisqu'elle n'est ni croissante, ni décroissante. 

\change
Par contre, sa restriction à l'intervalle $[0,+\infty[$ est strictement croissante.

%%%%%%%%%%%%%%%%%%%%%%%%%%%%%%%%%%%%%%%%%%%%%%%%%%%%%%%%%%
\diapo
Considérons une fonction $f$ définie sur un intervalle $I$ de la forme $[-a,a]$ (ouvert ou fermé) ou bien sur $\Rr$ entier. On dit que $f$ est paire si $\forall x\in I \ f(-x)=f(x)$,
 
\change
et que $f$ est impaire si $\forall x\in I \ f(-x)=-f(x)$

\change
Ces propriétés se traduisent graphiquement de la façon suivante : 
le graphe d'une fonction paire est symétrique par rapport à 
l'axe des $y$, alors que le graphe d'une fonction impaire est 
symétrique par rapport à l'origine.

%%%%%%%%%%%%%%%%%%%%%%%%%%%%%%%%%%%%%%%%%%%%%%%%%%%%%%%%%%
\diapo
Voyons quelques exemples. La fonction qui à $x$ réel associe $x^{2n}$ est paire,
 
\change
alors que la fonction définie par $x^{2n+1}$ est impaire. Ces deux exemples fondamentaux sont à l'origine de la dénomination "fonctions paire", "fonction impaire".

\change
Voici les graphes des fonctions $x^2$ en bleu, et $x^3$ en vert. Il faut les connaître. Ils peuvent en particulier vous aider à ne pas confondre fonctions paire et fonctions impaires.

\change
La fonction cosinus est paire, la fonction sinus est impaire.



%%%%%%%%%%%%%%%%%%%%%%%%%%%%%%%%%%%%%%%%%%%%%%%%%%%%%%%%%%%
\diapo
Donnons enfin la dernière définition de cette leçon. Fixons $T>0$. Une fonction $f$ définie sur $\Rr$ est dite périodique de période $T$ si $\forall x\in\Rr \ f(x+T)=f(x)$.
 
\change

[Sur le graphe]
Pour n'importe quel $x$ les valeurs en $x$ et $x+T$ sont égales.

Graphiquement, cette propriété se traduit par le fait que le graphe 
de $f$ est invariant par la translation de vecteur $T \vec{i}$, 
où $\vec{i}$ est le vecteur unité horizontal.


%%%%%%%%%%%%%%%%%%%%%%%%%%%%%%%%%%%%%%%%%%%%%%%%%%%%%%%%%%
\diapo

 Par exemple, les fonctions sinus et cosinus sont $2\pi$-périodiques,
 comme on le voit sur les graphes. 
 La fonction tangente est quant à elle périodique de période $\pi$.

%%%%%%%%%%%%%%%%%%%%%%%%%%%%%%%%%%%%%%%%%%%%%%%%%%%%%%%%%%%
\diapo

Entraînez avec ces exercices, pour vérifier que vous avez bien compris le cours.

\end{document}
