
%%%%%%%%%%%%%%%%%% PREAMBULE %%%%%%%%%%%%%%%%%%


\documentclass[12pt]{article}

\usepackage{amsfonts,amsmath,amssymb,amsthm}
\usepackage[utf8]{inputenc}
\usepackage[T1]{fontenc}
\usepackage[francais]{babel}


% packages
\usepackage{amsfonts,amsmath,amssymb,amsthm}
\usepackage[utf8]{inputenc}
\usepackage[T1]{fontenc}
%\usepackage{lmodern}

\usepackage[francais]{babel}
\usepackage{fancybox}
\usepackage{graphicx}

\usepackage{float}

%\usepackage[usenames, x11names]{xcolor}
\usepackage{tikz}
\usepackage{datetime}

\usepackage{mathptmx}
%\usepackage{fouriernc}
%\usepackage{newcent}
\usepackage[mathcal,mathbf]{euler}

%\usepackage{palatino}
%\usepackage{newcent}


% Commande spéciale prompteur

%\usepackage{mathptmx}
%\usepackage[mathcal,mathbf]{euler}
%\usepackage{mathpple,multido}

\usepackage[a4paper]{geometry}
\geometry{top=2cm, bottom=2cm, left=1cm, right=1cm, marginparsep=1cm}

\newcommand{\change}{{\color{red}\rule{\textwidth}{1mm}\\}}

\newcounter{mydiapo}

\newcommand{\diapo}{\newpage
\hfill {\normalsize  Diapo \themydiapo \quad \texttt{[\jobname]}} \\
\stepcounter{mydiapo}}


%%%%%%% COULEURS %%%%%%%%%%

% Pour blanc sur noir :
%\pagecolor[rgb]{0.5,0.5,0.5}
% \pagecolor[rgb]{0,0,0}
% \color[rgb]{1,1,1}



%\DeclareFixedFont{\myfont}{U}{cmss}{bx}{n}{18pt}
\newcommand{\debuttexte}{
%%%%%%%%%%%%% FONTES %%%%%%%%%%%%%
\renewcommand{\baselinestretch}{1.5}
\usefont{U}{cmss}{bx}{n}
\bfseries

% Taille normale : commenter le reste !
%Taille Arnaud
%\fontsize{19}{19}\selectfont

% Taille Barbara
%\fontsize{21}{22}\selectfont

%Taille François
\fontsize{25}{30}\selectfont

%Taille Pascal
%\fontsize{25}{30}\selectfont

%Taille Laura
%\fontsize{30}{35}\selectfont


%\myfont
%\usefont{U}{cmss}{bx}{n}

%\Huge
%\addtolength{\parskip}{\baselineskip}
}


% \usepackage{hyperref}
% \hypersetup{colorlinks=true, linkcolor=blue, urlcolor=blue,
% pdftitle={Exo7 - Exercices de mathématiques}, pdfauthor={Exo7}}


%section
% \usepackage{sectsty}
% \allsectionsfont{\bf}
%\sectionfont{\color{Tomato3}\upshape\selectfont}
%\subsectionfont{\color{Tomato4}\upshape\selectfont}

%----- Ensembles : entiers, reels, complexes -----
\newcommand{\Nn}{\mathbb{N}} \newcommand{\N}{\mathbb{N}}
\newcommand{\Zz}{\mathbb{Z}} \newcommand{\Z}{\mathbb{Z}}
\newcommand{\Qq}{\mathbb{Q}} \newcommand{\Q}{\mathbb{Q}}
\newcommand{\Rr}{\mathbb{R}} \newcommand{\R}{\mathbb{R}}
\newcommand{\Cc}{\mathbb{C}} 
\newcommand{\Kk}{\mathbb{K}} \newcommand{\K}{\mathbb{K}}

%----- Modifications de symboles -----
\renewcommand{\epsilon}{\varepsilon}
\renewcommand{\Re}{\mathop{\text{Re}}\nolimits}
\renewcommand{\Im}{\mathop{\text{Im}}\nolimits}
%\newcommand{\llbracket}{\left[\kern-0.15em\left[}
%\newcommand{\rrbracket}{\right]\kern-0.15em\right]}

\renewcommand{\ge}{\geqslant}
\renewcommand{\geq}{\geqslant}
\renewcommand{\le}{\leqslant}
\renewcommand{\leq}{\leqslant}

%----- Fonctions usuelles -----
\newcommand{\ch}{\mathop{\mathrm{ch}}\nolimits}
\newcommand{\sh}{\mathop{\mathrm{sh}}\nolimits}
\renewcommand{\tanh}{\mathop{\mathrm{th}}\nolimits}
\newcommand{\cotan}{\mathop{\mathrm{cotan}}\nolimits}
\newcommand{\Arcsin}{\mathop{\mathrm{Arcsin}}\nolimits}
\newcommand{\Arccos}{\mathop{\mathrm{Arccos}}\nolimits}
\newcommand{\Arctan}{\mathop{\mathrm{Arctan}}\nolimits}
\newcommand{\Argsh}{\mathop{\mathrm{Argsh}}\nolimits}
\newcommand{\Argch}{\mathop{\mathrm{Argch}}\nolimits}
\newcommand{\Argth}{\mathop{\mathrm{Argth}}\nolimits}
\newcommand{\pgcd}{\mathop{\mathrm{pgcd}}\nolimits} 

\newcommand{\Card}{\mathop{\text{Card}}\nolimits}
\newcommand{\Ker}{\mathop{\text{Ker}}\nolimits}
\newcommand{\id}{\mathop{\text{id}}\nolimits}
\newcommand{\ii}{\mathrm{i}}
\newcommand{\dd}{\mathrm{d}}
\newcommand{\Vect}{\mathop{\text{Vect}}\nolimits}
\newcommand{\Mat}{\mathop{\mathrm{Mat}}\nolimits}
\newcommand{\rg}{\mathop{\text{rg}}\nolimits}
\newcommand{\tr}{\mathop{\text{tr}}\nolimits}
\newcommand{\ppcm}{\mathop{\text{ppcm}}\nolimits}

%----- Structure des exercices ------

\newtheoremstyle{styleexo}% name
{2ex}% Space above
{3ex}% Space below
{}% Body font
{}% Indent amount 1
{\bfseries} % Theorem head font
{}% Punctuation after theorem head
{\newline}% Space after theorem head 2
{}% Theorem head spec (can be left empty, meaning ‘normal’)

%\theoremstyle{styleexo}
\newtheorem{exo}{Exercice}
\newtheorem{ind}{Indications}
\newtheorem{cor}{Correction}


\newcommand{\exercice}[1]{} \newcommand{\finexercice}{}
%\newcommand{\exercice}[1]{{\tiny\texttt{#1}}\vspace{-2ex}} % pour afficher le numero absolu, l'auteur...
\newcommand{\enonce}{\begin{exo}} \newcommand{\finenonce}{\end{exo}}
\newcommand{\indication}{\begin{ind}} \newcommand{\finindication}{\end{ind}}
\newcommand{\correction}{\begin{cor}} \newcommand{\fincorrection}{\end{cor}}

\newcommand{\noindication}{\stepcounter{ind}}
\newcommand{\nocorrection}{\stepcounter{cor}}

\newcommand{\fiche}[1]{} \newcommand{\finfiche}{}
\newcommand{\titre}[1]{\centerline{\large \bf #1}}
\newcommand{\addcommand}[1]{}
\newcommand{\video}[1]{}

% Marge
\newcommand{\mymargin}[1]{\marginpar{{\small #1}}}



%----- Presentation ------
\setlength{\parindent}{0cm}

%\newcommand{\ExoSept}{\href{http://exo7.emath.fr}{\textbf{\textsf{Exo7}}}}

\definecolor{myred}{rgb}{0.93,0.26,0}
\definecolor{myorange}{rgb}{0.97,0.58,0}
\definecolor{myyellow}{rgb}{1,0.86,0}

\newcommand{\LogoExoSept}[1]{  % input : echelle
{\usefont{U}{cmss}{bx}{n}
\begin{tikzpicture}[scale=0.1*#1,transform shape]
  \fill[color=myorange] (0,0)--(4,0)--(4,-4)--(0,-4)--cycle;
  \fill[color=myred] (0,0)--(0,3)--(-3,3)--(-3,0)--cycle;
  \fill[color=myyellow] (4,0)--(7,4)--(3,7)--(0,3)--cycle;
  \node[scale=5] at (3.5,3.5) {Exo7};
\end{tikzpicture}}
}



\theoremstyle{definition}
%\newtheorem{proposition}{Proposition}
%\newtheorem{exemple}{Exemple}
%\newtheorem{theoreme}{Théorème}
\newtheorem{lemme}{Lemme}
\newtheorem{corollaire}{Corollaire}
%\newtheorem*{remarque*}{Remarque}
%\newtheorem*{miniexercice}{Mini-exercices}
%\newtheorem{definition}{Définition}




%definition d'un terme
\newcommand{\defi}[1]{{\color{myorange}\textbf{\emph{#1}}}}
\newcommand{\evidence}[1]{{\color{blue}\textbf{\emph{#1}}}}



 %----- Commandes divers ------

\newcommand{\codeinline}[1]{\texttt{#1}}

%%%%%%%%%%%%%%%%%%%%%%%%%%%%%%%%%%%%%%%%%%%%%%%%%%%%%%%%%%%%%
%%%%%%%%%%%%%%%%%%%%%%%%%%%%%%%%%%%%%%%%%%%%%%%%%%%%%%%%%%%%%



\begin{document}

\debuttexte


%%%%%%%%%%%%%%%%%%%%%%%%%%%%%%%%%%%%%%%%%%%%%%%%%%%%%%%%%%%
\diapo

\change
Nous poursuivons ce chapitre sur les limites et les fonctions continues par 
une deuxième leçon consacrée à la notion de limites de fonctions.

\change
Nous allons tout d'abord donner la définition de limite, c'est-à-dire préciser ce que signifie qu'une fonction admette une limite en un point, que ce point soit réel ou infini, et que la limite soit elle-même finie ou infinie.

\change
Nous verrons ensuite les propriétés importantes sur les limites, par exemple comment additionner ou multiplier des limites. Ceci nous fournira des méthodes de calcul.


%%%%%%%%%%%%%%%%%%%%%%%%%%%%%%%%%%%%%%%%%%%%%%%%%%%%%%%%%%
\diapo

Définissons tout d'abord la limite d'une fonction en un point. 
On considère un intervalle $I$ de $\Rr$, et un réel $x_0$ qui est soit dans 
$I$, soit une extrémité de $I$.

\change
Nous dirons qu'une fonction $f$ définie sur $I$ a pour limite le réel $\ell$ en $x_0$ si on peut rendre $f(x)$ arbitrairement proche de $\ell$ pour tous les $x$ suffisamment proches de $x_0$, c'est-à-dire, en langage mathématique,  si l'assertion suivante est vérifiée

$
\forall \epsilon>0 \quad \exists \delta>0 \quad \forall x\in I \quad \vert x-x_0\vert <\delta 
\implies \vert f(x)-\ell\vert <\epsilon
$

\change
Voici le graphe d'une fonction ayant une limite $\ell$ en un point $x_0$. 
Chaque fois que l'on nous donne un $\epsilon>0$, on voit que l'on peut trouver un $\delta>0$ tel que tous les $x$ situés à une distance au plus $\delta$ de $x_0$ ont une image $f(x)$ située à une distance au plus $\epsilon$ de $\ell$. C'est-à-dire que pour tout intervalle centré en $\ell$, ici en noir, on peut trouver un intervalle centré en $x_0$, ici en vert, qui est envoyé par $f$ dans l'intervalle noir. On voit en particulier que la valeur de $\delta$ dépend de $\epsilon$ : plus $\epsilon$ est petit, plus $\delta$ doit être petit.


%%%%%%%%%%%%%%%%%%%%%%%%%%%%%%%%%%%%%%%%%%%%%%%%%%%%%%%%%%
\diapo

Revenons sur la définition d'une limite finie en un point.

\change
Tout d'abord, lorsqu'elle existe, on note la limite comme ceci ou cela.

\change
Il est important de comprendre que l'inégalité $\vert x-x_0\vert <\delta$ signifie
$x \in ]x_0 - \delta, x_0+\delta[$.

\change
De même l'inégalité $\vert f(x)-\ell\vert <\epsilon$ équivaut à $f(x) \in ]\ell - \epsilon, \ell+\epsilon[$.

\change
Dans la définition, on peut remplacer certaines inégalités strictes par des inégalités larges. Mais pas toutes !

\change
N'oubliez pas que l'ordre des quantificateurs est important, on ne peut pas échanger le $\forall \epsilon$ avec le $\exists \delta$ : comme on l'a vu, $\delta$ dépend en général de $\epsilon$. 

\change
Pour marquer cette dépendance on peut écrire :
$\forall \epsilon>0 \quad \exists \delta(\epsilon) >0 \ldots$

%%%%%%%%%%%%%%%%%%%%%%%%%%%%%%%%%%%%%%%%%%%%%%%%%%%%%%%%%%
\diapo


Voyons quelques exemples. La fonction $\sqrt x$ admet une limite en tout point $x_0$ positif, qui vaut bien sûr $\sqrt{x_0}$,

\change
comme on le voit sur son graphe.

\change
La fonction partie entière $E$ elle n'a pas de limite aux points $x_0$ qui sont des entiers.

\change
Ceci se traduit sur son graphe par un saut au point $x_0$.

%%%%%%%%%%%%%%%%%%%%%%%%%%%%%%%%%%%%%%%%%%%%%%%%%%%%%%%%%%
\diapo

Donnons à présent la définition d'une limite infinie en un point. On dit qu'une fonction a pour limite $+\infty$ au point $x_0$ si on peut rendre $f(x)$ arbitrairement grand pour tous les $x$ suffisamment proches de $x_0$, ce qui s'écrit en langage mathématique : $\forall A>0 \quad \exists \delta>0 \quad \forall x\in I \quad \vert x-x_0\vert <\delta \implies f(x)>A.$

\change
Voici un exemple. On voit que pour tout réel positif $A$ on peut trouver un intervalle centré en $x_0$ (en vert) tel que pour tout $x$ dans cet intervalle $f(x)$ est plus grand que $A$.

\change
On définit de la même manière une limite qui vaut $-\infty$ en un point $x_0$ en remplaçant $ f(x)>A$ par $ f(x)<-A$.


%%%%%%%%%%%%%%%%%%%%%%%%%%%%%%%%%%%%%%%%%%%%%%%%%%%%%%%%%%
\diapo

On va à présent définir une limite, non plus en un point réel $x_0$, mais en $+\infty$. On considère pour cela une fonction définie sur un intervalle du type $]a,+\infty[$.

\change
On dit que $f$ a pour limite le réel $\ell$ en $+\infty$ si 

$
\forall \epsilon>0 \quad \exists B>0 \quad \forall x\in I \quad x>B \implies \vert f(x)-\ell\vert <\epsilon
$

On voit ici un exemple de fonction ayant une limite finie en $+\infty$. Remarquons qu'on procèderait de la même manière pour définir une limite en $-\infty$.

%%%%%%%%%%%%%%%%%%%%%%%%%%%%%%%%%%%%%%%%%%%%%%%%%%%%%%%%%%
\diapo

On dit de même que $f$ a pour limite $+\infty$ en $+\infty$ si on peut rendre $f(x)$ arbitrairement grand pour tous les $x$ suffisamment grands.

\change
Voici à présent quelques exemples très classiques de limite en l'infini. $x^n$ tend vers $+\infty$ en $+\infty$. 

\change
Par contre, sa limite en $-\infty$ dépend de la parité de $n$ :
$+\infty$ si $n$ est pair, $-\infty$ sinon.

\change
Son inverse $\frac{1}{x^n}$ tend vers $0$ en $+\infty$ et en $-\infty$.


%%%%%%%%%%%%%%%%%%%%%%%%%%%%%%%%%%%%%%%%%%%%%%%%%%%%%%%%%%
\diapo

Un autre exemple. Considérons deux polynômes $P$ et $Q$ de degrés respectifs $n$ et $m$, et dont les coefficients des termes de plus haut degré, c'est-à-dire $a_n$ et $b_m$, sont strictement positifs.

\change
Alors la limite en $+\infty$ de la fraction rationnelle $P/Q$ est donnée par les termes de plus haut degré de $P$ et $Q$ :

\change
elle vaut $+\infty$ si $n>m$

\change
$a_n/b_m$ si $P$ et $Q$ sont de même degré

\change
et enfin $0$ si $n<m$.

\change
Ceci se vérifie aisément en factorisant le numérateur et le 
dénominateur par le terme qui l'emporte en $+\infty$, 
c'est-à-dire respectivement par $x^n$ et $x^m$.


%%%%%%%%%%%%%%%%%%%%%%%%%%%%%%%%%%%%%%%%%%%%%%%%%%%%%%%%%%
\diapo

Voici la première des propriétés que nous allons voir sur les limites. Si une fonction admet une limite, alors cette limite est unique.

On ne donne pas ici la démonstration de cette proposition, qui est très similaire à celle de 
l'unicité de la limite pour les suites. 
Elle est obtenue par un raisonnement par l'absurde, en supposant qu'une fonction a deux limites distinctes en un même point.


%%%%%%%%%%%%%%%%%%%%%%%%%%%%%%%%%%%%%%%%%%%%%%%%%%%%%%%%%%
\diapo
Nous allons voir à présent comment les limites se comportent vis à vis 
des opérations usuelles d'addition, de multiplication, 
de passage à l'inverse, etc...  
Soient deux fonctions $f$ et $g$. 
On suppose que $x_0$ est ou bien un réel, ou bien $+\infty$ ou $-\infty$.

\change
On suppose que $f$ et $g$ ont des limites finies en $x_0$, que l'on note respectivement $\ell$ et $\ell'$.

\change
Alors pour tout réel $\lambda$, la fonction $\lambda\cdot f$ aura pour limite en $x_0$ le réel $\lambda\cdot \ell$

\change
de même la somme $f+g$ aura pour limite la somme des limites $\ell+\ell'$

\change
et la limite du produit est le produit des limites.

\change
Si de plus $\ell\neq 0$, alors l'inverse $1/f$ est bien défini au voisinage de $x_0$ et tend vers $1/\ell$.

\change
Enfin, si $f$ tend vers $\pm\infty$, alors $1/f$ tend vers $0$.


Ce sont des propriétés que l'on utilise sans s'en apercevoir !

Elles se montrent de manière similaire à celles sur les limites de suites. 
Nous donnerons à la fin de cette leçon un exemple de démonstration.


%%%%%%%%%%%%%%%%%%%%%%%%%%%%%%%%%%%%%%%%%%%%%%%%%%%%%%%%%%
\diapo

Voici à présent une proposition concernant la limite d'une composée de fonctions. 
Si $f$ tend vers $\ell$ en $x_0$ et si $g$ tend vers $\ell'$ en **$\ell$**, 
alors la composée $g\circ f$ est bien définie au voisinage de $x_0$ et 
tend vers $\ell'$.


Voyons un exemple de calcul de limite. 

Considérons une fonction $u$ qui tend vers $2$ en $x_0$.

\change
Posons $f(x) = \sqrt{1+\frac{1}{u(x)^2}+\ln u(x)}$ 

\change
et cherchons la limite éventuelle de $f$ en $x_0$.

\change
Tout d'abord comme $u(x) \to 2$ alors, par produit de limite, $u(x)^2 \to 4$ 
  
\change
donc $\frac{1}{u(x)^2} \to \frac14$.
  
\change
Par ailleurs comme $u(x) \to 2$ qui est strictement positif, alors $u(x)$ reste strictement positive dans un voisinage de $x_0$,
  
\change
et donc $\ln u(x)$ est bien définie dans ce voisinage et de plus, par composition des limites, $\ln u(x) \to \ln 2$.
  
\change
Cela entraîne que  $1+\frac{1}{u(x)^2}+\ln u(x)$ tend lorsque $x \to x_0$ vers $ 1+\frac 14 + \ln 2$. 
  
\change
Cette limite étant strictement positive, on en déduit que $f(x)$ est bien définie dans un voisinage de $x_0$.
  
\change
Et par composition avec la racine carrée alors $f(x)$ a bien une limite en 
$x_0$ qui vaut $\sqrt{1+\frac14 + \ln 2}$.



%%%%%%%%%%%%%%%%%%%%%%%%%%%%%%%%%%%%%%%%%%%%%%%%%%%%%%%%%%
\diapo

Il y a des situations où l'on ne peut rien dire sur les limites.
Par exemple si $\lim_{x_0} f = +\infty$ et $\lim_{x_0} g = -\infty$ alors on ne peut 
a priori rien dire sur la limite de $f+g$ ; 
cela dépend vraiment de $f$ et de $g$.

\change

On exprime cela en disant que $+\infty-\infty$ est une forme indéterminée.

Pour trouver la limite -si elle existe!- il faut étudier la situation au cas par cas.

\change


Les formes indéterminées sont du même type que pour les suites en voici
un petite liste : $0\times \infty$ ; 
$\dfrac\infty\infty$ ; $\dfrac00$ ; $1^\infty$ ; $\infty^0$.


%%%%%%%%%%%%%%%%%%%%%%%%%%%%%%%%%%%%%%%%%%%%%%%%%%%%%%%%%%
\diapo


Enfin voici une proposition très importante qui lie le comportement 
d'une limite avec les inégalités.

Si $f\leq g$ et si $\displaystyle\lim_{x_0} f=\ell\in\Rr$ et 
$\displaystyle\lim_{x_0} g=\ell'\in\Rr$, alors $\ell\leq \ell'$.

Je vous rappelle que $f\le g$ signifie que pour tout $x$,
$f(x) \le g(x)$. 

\change

On a un résultat similaire si la limite est infinie :

Si $f\leq g$ et si $\displaystyle\lim_{x_0} f=+\infty$, 
alors  $\displaystyle\lim_{x_0} g=+\infty$.

\change

Enfin voici le célèbre "Théorème des gendarmes"

Si $f\leq g\leq h$ et si 
$\displaystyle\lim_{x_0} f=\displaystyle\lim_{x_0} h=\ell\in\Rr$, 
alors $g$ a une limite en $x_0$ et $\displaystyle\lim_{x_0} g=\ell$.


\change

Ce dernier point est vraiment très utile.

Sur ce dessin, cela se lit ainsi : on suppose que $f$ tend vers une limite $\ell$
en $x_0$, on  suppose que $h$ tant aussi vers cette même limite $\ell$ en $x_0$.

Et on suppose aussi que $f(x) \le g(x) \le h(x)$ pour tout $x$.

La conclusion est double :

(1) $g$ admet une limite en $x_0$ ;

et

(2) cette limite est bien sûr la même que pour $f$ et $h$.


%%%%%%%%%%%%%%%%%%%%%%%%%%%%%%%%%%%%%%%%%%%%%%%%%%%%%%%%%%
\diapo

Il y a beaucoup de résultats dans cette leçon et je vous encourage à lire et travailler
les démonstrations dans le livre. Nous allons voir un exemple démonstration
en prouvant que 
si $f$ tend vers une limite $\ell$ non nulle en $x_0$, 
alors $\frac 1 f$ est bien définie dans un voisinage de $x_0$ et tend vers $\frac 1\ell$.

On va même supposer que $\ell$ est positif, le cas négatif se montrerait de la même manière.

\change


Montrons tout d'abord que $\frac 1 f$ est bien définie et est bornée dans un voisinage de $x_0$.

\change

Par hypothèse $f(x)$ tend vers $\ell$, c'est-à-dire 
\[
\forall \epsilon'>0 \quad \exists \delta>0 \quad  x\in ] x_0-\delta,x_0+\delta[ 
\implies \ell-\epsilon' < f(x) <\ell+\epsilon'.
\]


\change

Pour n'importe quel  $\epsilon'$ tel que $0<\epsilon'<\ell/2$,


\change


on obtient un $\delta$ correspondant à ce $\epsilon'$ 

Maintenant sur l'intervalle $]x_0-\delta,x_0+\delta [$ 

\change

alors  cette inégalité [montrer celle de gauche] implique d'une part
que $f(x)> \frac{\ell}{2}>0$ donc on peut bien parler de $1/f(x)$

\change

et d'autre part en passant à l'inverse nous obtenons pour $1/f$ l'inégalité
$\frac{1}{f(x)} < M$ où $M=2/\ell$.

On vient de montrer que $1/f$ est bien définie et est bornée.

\change

Montrons que la limite $1/f$ est $1/\ell$.


\change

Pour $x$ dans l'intervalle $]x_0-\delta,x_0+\delta [$ on a

$$
\left\vert \frac{1}{f(x)} - \frac1\ell  \right\vert = \frac{\left\vert \ell - f(x) \right\vert }{f(x)\ell} $$

\change

que l'on majore par $$\frac{M}{\ell}\left\vert \ell - f(x) \right\vert$$ 

où $M$ est la borne obtenue au-dessus.

\change

Etant donné $\epsilon>0$. On choisit $\epsilon'=(\ell \epsilon)/M$, 
alors il existe $\delta>0$ tel que

sur l'intervalle $]x_0-\delta,x_0+\delta [$


$$\left\vert \frac{1}{f(x)} - \frac1\ell  \right\vert$$

\change

est inférieur à 
$\frac{M}{\ell}\left\vert \ell - f(x) \right\vert $

\change

Mais $\left\vert \ell - f(x) \right\vert < \epsilon'$

\change

qui vaut exactement $\epsilon$.

Pour $\epsilon >0$ on a trouvé $\delta$ tel que

$\left\vert \frac{1}{f(x)} - \frac1\ell  \right\vert <\epsilon$.

Ce qui est exactement dire que $1/f$ tend vers $1/\ell$.


%%%%%%%%%%%%%%%%%%%%%%%%%%%%%%%%%%%%%%%%%%%%%%%%%%%%%%%%%%%
\diapo

Il y a beaucoup de choses dans cette leçon, vérifier maintenant 
que vous avez bien compris le cours.

\end{document}
